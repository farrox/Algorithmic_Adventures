% Filename: detect_cycle_in_linked_list.tex

\problemsection{Detect Cycle in a Linked List}
\label{problem:Detect_Cycle_in_Linked_List}

The **Detect Cycle in a Linked List** problem involves determining whether a given singly linked list contains a cycle. A cycle occurs when a node's next pointer points to a previous node in the list, creating an infinite loop. This problem is fundamental in understanding linked list manipulations and pointer-based algorithms.

---

\section*{Problem Statement}
Given the head of a singly linked list, determine if the linked list has a cycle in it. If there is a cycle, return \texttt{true}; otherwise, return \texttt{false}.

A cycle in a linked list occurs when a node's next pointer points to a previous node in the list, forming a loop.

---

\textbf{Input:}
- \texttt{head}: The head node of a singly linked list.

\textbf{Output:}
- A boolean value: \texttt{true} if there is a cycle in the linked list, \texttt{false} otherwise.

---

\textbf{Example 1:}
\begin{verbatim}
Input: head = [3,2,0,-4], pos = 1
Output: true
Explanation: There is a cycle in the linked list, where the tail connects to the second node.
\end{verbatim}

\textbf{Example 2:}
\begin{verbatim}
Input: head = [1,2], pos = 0
Output: true
Explanation: There is a cycle in the linked list, where the tail connects to the first node.
\end{verbatim}

\textbf{Example 3:}
\begin{verbatim}
Input: head = [1], pos = -1
Output: false
Explanation: There is no cycle in the linked list.
\end{verbatim}

---

\section*{Algorithmic Approach}

Detecting a cycle in a linked list can be efficiently achieved using the **Floyd’s Tortoise and Hare** algorithm, also known as the two-pointer technique. This approach uses two pointers that traverse the list at different speeds to determine if a cycle exists.

---

\subsection*{Floyd’s Tortoise and Hare Algorithm}
\textbf{Key Idea:}
Use two pointers, \texttt{slow} and \texttt{fast}. \texttt{slow} moves one step at a time, while \texttt{fast} moves two steps at a time. If there is no cycle, \texttt{fast} will reach the end of the list. If there is a cycle, \texttt{fast} will eventually meet \texttt{slow} within the cycle.

\textbf{Steps:}
\begin{enumerate}
    \item Initialize two pointers, \texttt{slow} and \texttt{fast}, both starting at the head of the linked list.
    \item Traverse the linked list:
    \begin{itemize}
        \item Move \texttt{slow} by one step.
        \item Move \texttt{fast} by two steps.
    \end{itemize}
    \item At each step, check if \texttt{slow} and \texttt{fast} meet:
    \begin{itemize}
        \item If they meet, a cycle exists; return \texttt{true}.
        \item If \texttt{fast} reaches the end (\texttt{NULL}), no cycle exists; return \texttt{false}.
    \end{itemize}
\end{enumerate}

---

\subsection*{Alternative Approaches}
\begin{itemize}
    \item \textbf{Hash Table (Set) Approach:} Traverse the linked list and store each visited node in a hash table. If a node is revisited, a cycle exists. This method uses \(O(n)\) time and \(O(n)\) space.
    \item \textbf{Marking Visited Nodes:} Modify the linked list by marking visited nodes (e.g., by changing node values or using flags). This approach can be risky as it alters the input data and may not be permissible.
\end{itemize}

---

\subsection*{Complexities}
\begin{itemize}
    \item \textbf{Time Complexity:} \(O(n)\), where \(n\) is the number of nodes in the linked list. Both the hash table and two-pointer approaches traverse the list at most a constant number of times.
    \item \textbf{Space Complexity:}
    \begin{itemize}
        \item \textbf{Floyd’s Tortoise and Hare:} \(O(1)\), as it uses only two pointers.
        \item \textbf{Hash Table Approach:} \(O(n)\), due to storing visited nodes.
    \end{itemize}
\end{itemize}

---

\section*{Python Implementation}

\subsection*{Floyd’s Tortoise and Hare (Two-Pointer) Approach}
\begin{fullwidth}
\begin{lstlisting}[language=Python]
class ListNode:
    def __init__(self, x):
        self.val = x
        self.next = None

def hasCycle(head):
    if not head or not head.next:
        return False
    
    slow = head
    fast = head.next
    
    while slow != fast:
        if not fast or not fast.next:
            return False
        slow = slow.next
        fast = fast.next.next
    
    return True

# Example usage:
# Creating a cycle: 1 -> 2 -> 3 -> 4 -> 2 ...
node1 = ListNode(1)
node2 = ListNode(2)
node3 = ListNode(3)
node4 = ListNode(4)
node1.next = node2
node2.next = node3
node3.next = node4
node4.next = node2  # Creates a cycle

print(hasCycle(node1))  # Output: True
\end{lstlisting}
\end{fullwidth}

\subsection*{Hash Table (Set) Approach}
\begin{fullwidth}
\begin{lstlisting}[language=Python]
def hasCycleUsingSet(head):
    visited = set()
    current = head
    while current:
        if current in visited:
            return True
        visited.add(current)
        current = current.next
    return False

# Example usage:
# Creating a cycle: 1 -> 2 -> 3 -> 4 -> 2 ...
node1 = ListNode(1)
node2 = ListNode(2)
node3 = ListNode(3)
node4 = ListNode(4)
node1.next = node2
node2.next = node3
node3.next = node4
node4.next = node2  # Creates a cycle

print(hasCycleUsingSet(node1))  # Output: True
\end{lstlisting}
\end{fullwidth}

---

\section*{Why These Approaches?}
\begin{itemize}
    \item \textbf{Floyd’s Tortoise and Hare:} This approach is optimal in both time and space for detecting cycles in linked lists. It avoids the overhead of additional memory structures, making it suitable for large datasets.
    \item \textbf{Hash Table Approach:} While not as space-efficient, this method is straightforward and easy to implement. It is useful when modifications to the linked list are not allowed or when simplicity is preferred over optimal space usage.
\end{itemize}

---

\section*{Alternative Approaches}
\begin{itemize}
    \item \textbf{Recursive Cycle Detection:} Implement cycle detection using recursion, keeping track of visited nodes through recursive calls. This method is generally not recommended due to the risk of stack overflow with large lists.
    \item \textbf{Modifying Node Values:} Temporarily alter node values to mark them as visited. This approach is risky as it changes the input data and may not be permissible in certain contexts.
\end{itemize}

---

\section*{Similar Problems}
\begin{itemize}
    \item \textbf{Find the Starting Node of the Cycle:} Determine the node where the cycle begins.
    \item \textbf{Linked List Intersection:} Identify if two linked lists intersect and find the intersection point.
    \item \textbf{Remove Nth Node from End:} Remove the \(n\)-th node from the end of the list.
\end{itemize}

---

\section*{Corner Cases to Test}
\begin{itemize}
    \item \textbf{Empty List:} \(\texttt{head} = NULL\).
    \item \textbf{Single Node Without Cycle:} \(\texttt{head} = [1]\).
    \item \textbf{Single Node With Cycle:} \(\texttt{head} = [1] \rightarrow [1]\).
    \item \textbf{Multiple Nodes Without Cycle:} \(\texttt{head} = [1, 2, 3, 4, 5]\).
    \item \textbf{Multiple Nodes With Cycle:} \(\texttt{head} = [1, 2, 3, 4, 5] \rightarrow 3\).
\end{itemize}

---

\section*{Conclusion}
Detecting cycles in a linked list is a crucial problem that enhances understanding of pointer manipulation and algorithmic efficiency. Floyd’s Tortoise and Hare algorithm provides an optimal solution with minimal space overhead, making it the preferred method for this task. Mastery of cycle detection not only aids in solving linked list problems but also contributes to a deeper comprehension of algorithmic design principles.