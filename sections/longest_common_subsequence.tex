% file: longest_common_subsequence.tex

\problemsection{Longest Common Subsequence}
\label{problem:longest_common_subsequence}
The "Longest Common Subsequence" (LCS) problem is a classic algorithmic challenge that falls under the category of dynamic programming. The problem entails finding the length of the longest subsequence present in both given sequences, where a subsequence is defined as a sequence that maintains the original order but not necessarily contiguously present in both strings.

\section*{Problem Statement}
Given two strings, `text1` and `text2`, the task is to return the length of their longest common subsequence. A subsequence is a new string generated from the original string by deleting some characters (possibly none) without altering the remaining characters' relative order. A common subsequence of two strings is a subsequence that is common to both strings. If there is no common subsequence, the function should return 0.

Example:

Input: \\
text1 = "abcde" \\
text2 = "ace" \\

Output: \\
3  \\

Explanation: \\
The longest common subsequence is "ace" and its length is 3.

LeetCode link: \href{https://leetcode.com/problems/longest-common-subsequence/}{Longest Common Subsequence}

\section*{Algorithmic Approach}
The dynamic programming approach involves creating a 2D array `dp` where `dp[i][j]` represents the length of the longest common subsequence of substrings `text1[0...i-1]` and `text2[0...j-1]`. This table is filled in a bottom-up manner based on whether characters at current indices in both strings match or not, as defined by the recurrence relations:
\begin{itemize}
	\item If `text1[i - 1]` equals `text2[j - 1]`, include this character in LCS: `dp[i][j] = dp[i - 1][j - 1] + 1`.
	\item If `text1[i - 1]` is not equal to `text2[j - 1]`, find the longer LCS by not including either the i-th or j-th character: `dp[i][j] = max(dp[i - 1][j], dp[i][j - 1])`.
\end{itemize}

The initial condition is when at least one of the strings is empty, leading to a `0` longest common subsequence. The final answer resides in `dp[length(text1)][length(text2)]`, determining the length of the LCS of `text1` and `text2`.

\section*{Complexities}
\begin{itemize}
	\item \textbf{Time Complexity:} The time complexity is \(O(m \times n)\), where \(m\) and \(n\) are the lengths of `text1` and `text2` respectively, owing to the nested loop to fill the `dp` table.
	\item \textbf{Space Complexity:} The space complexity is \(O(m \times n)\) to store the `dp` table. It is possible to reduce the space complexity by using only two rows at a time since we only reference the row above the current one.
\end{itemize}


ewpage
\section*{Python Implementation}
Below is the complete Python code with optimizations and comprehensive test cases:

\begin{fullwidth}
\begin{lstlisting}[language=Python]
from typing import List, Optional

class Solution:
    def longestCommonSubsequence(self, text1: str, text2: str) -> int:
        # Handle edge cases
        if not text1 or not text2:
            return 0
            
        # Optimize by making text1 the shorter string
        if len(text1) > len(text2):
            text1, text2 = text2, text1
            
        m, n = len(text1), len(text2)
        
        # Space-optimized version using only two rows
        prev = [0] * (n + 1)
        curr = [0] * (n + 1)
        
        for i in range(1, m + 1):
            for j in range(1, n + 1):
                if text1[i-1] == text2[j-1]:
                    curr[j] = prev[j-1] + 1
                else:
                    curr[j] = max(prev[j], curr[j-1])
            prev, curr = curr, prev
            
        return prev[n]
    
    def getLCS(self, text1: str, text2: str) -> str:
        """Returns the actual longest common subsequence."""
        m, n = len(text1), len(text2)
        dp = [[0] * (n + 1) for _ in range(m + 1)]
        
        # Fill the dp table
        for i in range(1, m + 1):
            for j in range(1, n + 1):
                if text1[i-1] == text2[j-1]:
                    dp[i][j] = dp[i-1][j-1] + 1
                else:
                    dp[i][j] = max(dp[i-1][j], dp[i][j-1])
        
        # Reconstruct the LCS
        lcs = []
        i, j = m, n
        while i > 0 and j > 0:
            if text1[i-1] == text2[j-1]:
                lcs.append(text1[i-1])
                i -= 1
                j -= 1
            elif dp[i-1][j] > dp[i][j-1]:
                i -= 1
            else:
                j -= 1
                
        return ''.join(reversed(lcs))

# Comprehensive test cases
def test_lcs():
    solution = Solution()
    
    # Basic cases
    assert solution.longestCommonSubsequence("abcde", "ace") == 3
    assert solution.getLCS("abcde", "ace") == "ace"
    
    # Edge cases
    assert solution.longestCommonSubsequence("", "abc") == 0
    assert solution.longestCommonSubsequence("abc", "") == 0
    assert solution.longestCommonSubsequence("", "") == 0
    
    # Same strings
    assert solution.longestCommonSubsequence("abc", "abc") == 3
    assert solution.getLCS("abc", "abc") == "abc"
    
    # No common subsequence
    assert solution.longestCommonSubsequence("abc", "def") == 0
    
    # Longer examples
    assert solution.longestCommonSubsequence(
        "AGGTAB", "GXTXAYB") == 4
    assert solution.getLCS("AGGTAB", "GXTXAYB") == "GTAB"
\end{lstlisting}
\end{fullwidth}

This code initializes a `dp` table and then iteratively builds up the solution to larger subproblems using the previously mentioned recurrence relations. By the end, the `dp` table's last entry reflects the answer to the problem.

\section*{Why this approach}
The dynamic programming approach is leveraged due to the LCS problem's overlapping subproblems and optimal substructure properties. Specifically, solving larger problems efficiently requires solving various smaller, similar problems, which naturally lends itself to a dynamic programming solution. This approach minimizes redundant calculations compared to a naive recursive solution and is therefore chosen for its time efficiency.

\section*{Optimizations}
\begin{itemize}
    \item \textbf{Space Optimization:} Using two rows instead of full matrix
    \item \textbf{Input Optimization:} Processing shorter string in outer loop
    \item \textbf{Early Termination:} Handling empty string cases immediately
    \item \textbf{Memory Efficiency:} Using array rotation instead of copying
\end{itemize}

\section*{Common Pitfalls}
\begin{itemize}
    \item Forgetting to handle empty string cases
    \item Off-by-one errors in dp table indices
    \item Not considering space optimization possibilities
    \item Incorrect reconstruction of the actual subsequence
\end{itemize}

\section*{Interview Tips}
\begin{itemize}
    \item Start with a small example to illustrate the approach
    \item Mention space optimization possibilities
    \item Discuss how to reconstruct the actual subsequence
    \item Consider follow-up questions:
        \begin{itemize}
            \item What if we need all possible LCS?
            \item How to handle very long strings?
            \item What if we have more than two strings?
        \end{itemize}
\end{itemize}

\section*{Applications}
\begin{itemize}
    \item \textbf{Bioinformatics:} DNA sequence alignment
    \item \textbf{Version Control:} File difference comparison
    \item \textbf{Natural Language Processing:} Text similarity
    \item \textbf{Data Compression:} Finding redundant patterns
\end{itemize}

\section*{Related Problems}
\begin{itemize}
    \item Longest Common Substring
    \item Shortest Common Supersequence
    \item Edit Distance
    \item Longest Increasing Subsequence
\end{itemize}