% Filename: binary_tree_right_side_view.tex

\problemsection{Binary Tree Right Side View}\marginpar{View a binary tree from the right side using BFS or DFS.}

\textbf{Problem Statement}

This problem involves determining the right side view of a binary tree. The right side view consists of the nodes visible when the tree is viewed from the right side. Specifically, you need to return a list of node values that are visible at each level of the tree from top to bottom.

\textbf{Algorithmic Approach}

To solve this problem, you can use either an iterative approach with Breadth-First Search (BFS) or a recursive approach with Depth-First Search (DFS). Below, both methods are described:

\begin{enumerate}
    \item \textbf{Iterative Approach (Breadth-First Search)}:
    \begin{itemize}
        \item \textbf{Use a Queue}: Utilize a queue to perform level-order traversal.
        \item \textbf{Level-by-Level Traversal}:
        \begin{enumerate}
            \item Initialize the queue with the root node.
            \item While the queue is not empty:
            \begin{enumerate}
                \item Determine the number of nodes at the current level (size of the queue).
                \item Iterate through all nodes at this level:
                \begin{enumerate}
                    \item Dequeue a node from the queue.
                    \item If it's the last node in the current level, add its value to the result list.
                    \item Enqueue its non-\texttt{NULL} left and right children.
                \end{enumerate}
            \end{enumerate}
        \end{enumerate}
    \end{itemize}
    
    \item \textbf{Recursive Approach (Depth-First Search)}:
    \begin{itemize}
        \item \textbf{Use a Helper Function}: Create a helper function that takes a node and its current depth as arguments.
        \item \textbf{Traverse the Tree}:
        \begin{enumerate}
            \item If the node is \texttt{NULL}, return.
            \item If the current depth equals the size of the result list, append the node's value to the result list.
            \item Recursively traverse the right subtree first, then the left subtree, incrementing the depth by one.
        \end{enumerate}
    \end{itemize}
\end{enumerate}

\textbf{Complexities}

\begin{itemize}
    \item \textbf{Time Complexity}: The time complexity is \(O(n)\), where \(n\) is the number of nodes in the tree. Each node is visited exactly once.
    \item \textbf{Space Complexity}: 
    \begin{itemize}
        \item \textbf{Iterative Approach (BFS)}: \(O(n)\), as in the worst case, the queue will hold all nodes at the last level.
        \item \textbf{Recursive Approach (DFS)}: \(O(h)\), where \(h\) is the height of the tree, due to the call stack.
    \end{itemize}
\end{itemize}

\textbf{Python Implementation}\marginpar{Implementing both BFS and DFS solutions for flexibility.}

\begin{lstlisting}[language=Python, xleftmargin=0.02\textwidth, xrightmargin=0.02\textwidth]
# Definition for a binary tree node.
class TreeNode:
    def __init__(self, val=0, left=None, right=None):
        self.val = val
        self.left = left
        self.right = right

# Iterative Approach (BFS)
from collections import deque

def rightSideViewBFS(root: TreeNode) -> list:
    if not root:
        return []
    result = []
    queue = deque([root])
    while queue:
        level_size = len(queue)
        for i in range(level_size):
            node = queue.popleft()
            # If it's the last node in the current level, add to result
            if i == level_size - 1:
                result.append(node.val)
            if node.left:
                queue.append(node.left)
            if node.right:
                queue.append(node.right)
    return result

# Recursive Approach (DFS)
def rightSideViewDFS(root: TreeNode) -> list:
    def helper(node: TreeNode, depth: int):
        if not node:
            return
        if depth == len(result):
            result.append(node.val)
        # Traverse right subtree first
        helper(node.right, depth + 1)
        helper(node.left, depth + 1)
    
    result = []
    helper(root, 0)
    return result

# Example usage:
# Constructing the following binary tree
#       1
#      / \
#     2   3
#      \   \
#       5   4
root = TreeNode(1)
root.left = TreeNode(2, None, TreeNode(5))
root.right = TreeNode(3, None, TreeNode(4))

print(rightSideViewBFS(root))  # Output: [1, 3, 4]
print(rightSideViewDFS(root))  # Output: [1, 3, 4]
\end{lstlisting}

\textbf{Explanation}

The function \texttt{rightSideView} returns the values of the nodes that are visible when the binary tree is viewed from the right side. 

\begin{itemize}
    \item \textbf{Iterative Approach (BFS)}: 
    \begin{itemize}
        \item Utilizes a queue to perform level-order traversal.
        \item At each level, it records the last node's value, which is visible from the right side.
        \item Enqueues the left and right children of each node to traverse subsequent levels.
    \end{itemize}
    
    \item \textbf{Recursive Approach (DFS)}: 
    \begin{itemize}
        \item Uses a helper function to traverse the tree depth-first, prioritizing the right subtree first.
        \item When visiting a node at a new depth, it adds the node's value to the result list.
        \item This ensures that the first node encountered at each depth is the rightmost node.
    \end{itemize}
\end{itemize}

\textbf{Why This Approach}

\begin{itemize}
    \item \textbf{Iterative Approach (BFS)}: 
    \begin{itemize}
        \item Efficiently handles level-by-level traversal, making it straightforward to identify the rightmost node at each level.
        \item Suitable for scenarios where breadth-wise information is essential.
    \end{itemize}
    
    \item \textbf{Recursive Approach (DFS)}: 
    \begin{itemize}
        \item Recursively explores the right subtree first, ensuring that the first node encountered at each depth is the rightmost node.
        \item Simplifies the implementation by leveraging the call stack to manage traversal depth.
    \end{itemize}
\end{itemize}

\textbf{Alternative Approaches}

An alternative method involves using Depth-First Search (DFS) iteratively with a stack, maintaining a record of depths and ensuring that the rightmost nodes are processed first. However, the BFS and recursive DFS approaches are generally more intuitive and easier to implement for this problem.

\textbf{Similar Problems to This One}

Similar tree traversal problems include finding the left side view of a binary tree (\hyperref[problem:binary_tree_left_side_view]{Binary Tree Left Side View}), performing a zigzag level order traversal (\hyperref[problem:binary_tree_zigzag_level_order_traversal]{Binary Tree Zigzag Level Order Traversal}), and finding the maximum depth of a binary tree (\hyperref[problem:maximum_depth_of_binary_tree]{Maximum Depth of Binary Tree}).

\textbf{Things to Keep in Mind and Tricks}

\begin{itemize}
    \item \textbf{Handling Empty Trees}: Always check if the root is \texttt{NULL} to handle empty trees gracefully.
    \item \textbf{Level Tracking}: In the iterative approach, keeping track of the current level's size helps in identifying the rightmost node.
    \item \textbf{Traversal Order}: Prioritizing the right subtree first in the recursive approach ensures that the first node encountered at each depth is the rightmost node.
    \item \textbf{Space Optimization}: Be mindful of the space used by the queue in BFS or the recursion stack in DFS, especially for very deep or wide trees.
    \item \textbf{Consistent Level Identification}: Ensure that nodes are correctly associated with their respective levels to maintain accurate traversal results.
\end{itemize}

\textbf{Corner and Special Cases to Test When Writing the Code}

\begin{itemize}
    \item \textbf{Empty Tree}: Should return an empty list.
    \item \textbf{Single Node}: Should return a list containing the single node's value.
    \item \textbf{Left-Skewed Tree}: Trees where each node has only a left child should correctly return the last node at each level.
    \item \textbf{Right-Skewed Tree}: Trees where each node has only a right child should correctly return all node values.
    \item \textbf{Balanced Tree}: Ensure that balanced trees return the rightmost nodes at each level.
    \item \textbf{Large Tree}: Testing with a large number of nodes ensures that the implementation handles depth and breadth efficiently without performance degradation.
\end{itemize}