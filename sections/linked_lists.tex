% Filename: linked_lists.tex

\chapter{Linked Lists}
\label{chap:Linked_Lists}

Like arrays, a linked list is a fundamental data structure used to represent sequential data. Unlike arrays, however, linked lists do not require contiguous blocks of memory for storage. Instead, each element (known as a node) contains both data and a reference to the next node in the sequence. This unique design enables linked lists to dynamically grow and shrink, making them more flexible for certain use cases.

\section*{What is a Linked List?}
A linked list is a linear collection of data elements, known as nodes, connected through links. Each node contains two components:
\begin{itemize}
    \item **Data:** The value stored in the node.
    \item **Link:** A pointer or reference to the next node in the sequence.
\end{itemize}

\textbf{Example:}
\[
\texttt{[Data: 1] -> [Data: 2] -> [Data: 3] -> NULL}
\]

In this example, each node points to the next node, and the final node points to \texttt{NULL}, indicating the end of the list.



\section*{Types of Linked Lists}
Linked lists come in various forms, each suited for different applications:
\begin{itemize}
    \item **Singly Linked List:** Each node has a link to the next node. Traversal is one-way.
    \item **Doubly Linked List:** Each node has links to both the previous and the next nodes, enabling two-way traversal.
    \item **Circular Linked List:** The last node links back to the first node, forming a circular structure.
\end{itemize}



\section*{Advantages of Linked Lists}
1. **Dynamic Size:** Unlike arrays, linked lists do not require a predefined size. They can grow or shrink dynamically as elements are added or removed.
2. **Efficient Insertion and Deletion:** Adding or removing a node (given its location) has a time complexity of \(O(1)\), as no shifting of elements is needed. In arrays, insertion or deletion typically requires shifting all subsequent elements, leading to \(O(n)\) complexity.
3. **Memory Utilization:** Linked lists allocate memory as needed, which can be more efficient than arrays that may preallocate excessive space.



\section*{Disadvantages of Linked Lists}
1. **Linear Access Time:** Accessing an element in a linked list requires traversing the list from the start, resulting in \(O(n)\) complexity. Arrays, in contrast, allow direct access via indexing (e.g., \texttt{arr[4]}).
2. **Increased Memory Overhead:** Each node requires extra memory to store the reference (or pointer) to the next node.
3. **Cache Unfriendliness:** Since linked list nodes are scattered across memory, they do not take full advantage of caching mechanisms, unlike arrays which store elements contiguously.



\section*{Basic Operations on Linked Lists}

\subsection*{1. Traversal}
To access elements of a linked list, traversal from the head (starting node) to the tail (last node) is necessary.

\begin{fullwidth}
\begin{lstlisting}[language=Python]
# Node definition
class Node:
    def __init__(self, data):
        self.data = data
        self.next = None

# Traversing a linked list
def traverse(head):
    current = head
    while current:
        print(current.data, end=" -> ")
        current = current.next
    print("NULL")

# Example usage
node1 = Node(1)
node2 = Node(2)
node3 = Node(3)
node1.next = node2
node2.next = node3

traverse(node1)  # Output: 1 -> 2 -> 3 -> NULL
\end{lstlisting}
\end{fullwidth}



\subsection*{2. Insertion}
Insertion in a linked list can occur at:
\begin{itemize}
    \item **The Head:** Add a new node at the beginning.
    \item **The Tail:** Add a new node at the end.
    \item **Middle Positions:** Insert a node at a specified position.
\end{itemize}



\subsection*{3. Deletion}
Nodes can be removed from:
\begin{itemize}
    \item **The Head:** Remove the first node.
    \item **The Tail:** Remove the last node.
    \item **Middle Positions:** Remove a node from a specific position.
\end{itemize}



\subsection*{4. Searching}
Searching for a value in a linked list involves traversing the list until the value is found or the end is reached.



\section*{Applications of Linked Lists}
1. **Dynamic Data Structures:** Used to implement stacks, queues, and other abstract data types.
2. **Memory-Efficient Data Representation:** Suitable for applications where frequent insertion and deletion of elements are required.
3. **Graph Representation:** Adjacency lists for graphs are implemented using linked lists.
4. **Undo Functionality:** Linked lists are used to store states for undo/redo operations.



\section*{Comparison with Arrays}
\begin{center}
\begin{tabular}{|c|c|c|}
\hline
**Feature**            & **Array**                 & **Linked List**         \\
\hline
Memory Allocation      & Fixed (contiguous)        & Dynamic (non-contiguous)\\
\hline
Access Time            & \(O(1)\) (direct access)  & \(O(n)\) (sequential)   \\
\hline
Insertion/Deletion     & \(O(n)\)                  & \(O(1)\) (at head/tail) \\
\hline
Cache Performance      & High                      & Low                     \\
\hline
\end{tabular}
\end{center}



\section*{Common Problems Involving Linked Lists}
1. **Reverse a Linked List:** Reverse the order of nodes in a linked list.
2. **Detect a Cycle in a Linked List:** Check if the linked list has a cycle using Floyd’s Tortoise and Hare algorithm.
3. **Merge Two Sorted Lists:** Combine two sorted linked lists into one sorted list.
4. **Remove Nth Node from End:** Remove the \(n\)-th node from the end of the list.
5. **Find Intersection:** Determine if two linked lists intersect and find the intersection point.



\section*{Conclusion}
Linked lists are a versatile and essential data structure for representing sequential data. While they may have some limitations compared to arrays, their dynamic nature and efficient insertion/deletion operations make them a critical tool in computer science and software development. Understanding linked lists is foundational for mastering more advanced data structures and algorithms.