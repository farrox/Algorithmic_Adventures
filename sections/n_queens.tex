% Filename: n_queens.tex

\problemsection{N-Queens Problem}
\label{problem:n_queens}
\marginnote{A classic backtracking problem that tests understanding of constraint satisfaction and recursive problem-solving. Frequently appears in interviews at top tech companies.}

\section*{Problem Statement}
Place N chess queens on an N×N chessboard so that no two queens threaten each other. A solution requires that no two queens share the same row, column, or diagonal.

% \begin{figure}[h]
%     \centering
%     \includegraphics[width=0.7\linewidth]{figs/n_queens_example}
%     \caption{Example solution for N=8 queens problem}
%     \label{fig:n_queens}
% \end{figure}

\section*{Key Insights}
Understanding these insights is crucial for developing an optimal solution:
\begin{itemize}
    \item Each row must contain exactly one queen
    \item Each column must contain exactly one queen
    \item Two queens can't share any diagonal
    \item We can place queens row by row to reduce the search space
    \item Diagonals can be represented by their row±column values
\end{itemize}

\section*{Solution Strategy}
The optimal approach uses \textbf{backtracking} with the following strategy:

\begin{enumerate}
    \item Place queens row by row, trying each column
    \item Use sets to track occupied columns and diagonals
    \item Backtrack when a placement violates constraints
    \item Record valid solutions when all queens are placed
\end{enumerate}

\marginnote{Using sets for constraint checking provides O(1) lookup time, significantly improving performance over array-based approaches.}

\section*{Python Implementation}
\begin{fullwidth}
\begin{lstlisting}[language=Python]
class Solution:
    def solveNQueens(self, n: int) -> List[List[str]]:
        def create_board() -> List[str]:
            return [''.join('Q' if j == col else '.' 
                    for j in range(n))
                    for col in state]
        
        def is_safe(row: int, col: int) -> bool:
            return (col not in cols and
                    row + col not in pos_diag and
                    row - col not in neg_diag)
        
        def backtrack(row: int) -> None:
            if row == n:  # Found a valid solution
                solutions.append(create_board())
                return
            
            for col in range(n):
                if is_safe(row, col):
                    # Place queen and update constraints
                    cols.add(col)
                    pos_diag.add(row + col)
                    neg_diag.add(row - col)
                    state.append(col)
                    
                    backtrack(row + 1)
                    
                    # Backtrack: remove queen and constraints
                    cols.remove(col)
                    pos_diag.remove(row + col)
                    neg_diag.remove(row - col)
                    state.pop()
        
        solutions: List[List[str]] = []
        state: List[int] = []
        cols: set[int] = set()
        pos_diag: set[int] = set()  # row + col
        neg_diag: set[int] = set()  # row - col
        
        backtrack(0)
        return solutions

# Comprehensive test cases
def test_n_queens():
    solution = Solution()
    
    # Edge cases
    assert len(solution.solveNQueens(1)) == 1  # Single queen
    assert len(solution.solveNQueens(2)) == 0  # Impossible
    assert len(solution.solveNQueens(3)) == 0  # Impossible
    
    # Standard cases
    assert len(solution.solveNQueens(4)) == 2
    assert len(solution.solveNQueens(8)) == 92
\end{lstlisting}
\end{fullwidth}

\section*{Implementation Details}
\begin{itemize}
    \item \textbf{State Management:}
        \begin{itemize}
            \item \texttt{state}: Records queen positions (column) for each row
            \item \texttt{cols}: Tracks occupied columns
            \item \texttt{pos\_diag}: Tracks occupied positive diagonals (row + col)
            \item \texttt{neg\_diag}: Tracks occupied negative diagonals (row - col)
        \end{itemize}
    \item \textbf{Constraint Checking:} O(1) using sets
    \item \textbf{Board Creation:} Efficient string manipulation
    \item \textbf{Type Hints:} Added for better code clarity
\end{itemize}

\section*{Complexity Analysis}
\begin{itemize}
    \item \textbf{Time Complexity:} O(N!) - we try N positions for first queen, N-1 for second, etc.
    \item \textbf{Space Complexity:} O(N) for recursion stack and sets
\end{itemize}

\section*{Common Mistakes}
\begin{itemize}
    \item Forgetting to backtrack (remove constraints)
    \item Incorrect diagonal calculations
    \item Using inefficient constraint checking methods
    \item Not handling edge cases (N=1, N=2, N=3)
\end{itemize}

\section*{Interview Tips}
\begin{itemize}
    \item Start by explaining the constraints and how to check them efficiently
    \item Mention that placing queens row by row reduces the search space
    \item Discuss how sets provide O(1) lookup for constraint checking
    \item Consider mentioning optimizations:
        \begin{itemize}
            \item Using bit manipulation for even faster constraint checking
            \item Leveraging board symmetry to reduce solutions search
            \item Pre-computing diagonal indices
        \end{itemize}
\end{itemize}

\section*{Related Problems}
\begin{itemize}
    \item Sudoku Solver
    \item Permutations
    \item Combination Sum
    \item Path with Constraints
\end{itemize}

\printindex