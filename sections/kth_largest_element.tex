% Filename: kth_largest_element.tex

\problemsection{Kth Largest Element in an Array}
\label{problem:Kth_Largest_Element}

The **Kth Largest Element in an Array** problem challenges you to find the \(k\)-th largest element efficiently, without necessarily sorting the entire array. This problem is a great exercise in optimization techniques, particularly for scenarios where sorting is not feasible due to performance constraints.

---

\section*{Problem Statement}
Given an integer array \texttt{nums} and an integer \texttt{k}, return the \(k\)-th largest element in the array. Note that the \(k\)-th largest element is determined in sorted order, not based on distinct values.

\textbf{Input:}
- \texttt{nums}: A list of integers.
- \texttt{k}: An integer representing the desired rank.

\textbf{Output:}
- An integer representing the \(k\)-th largest element in \texttt{nums}.

---

\textbf{Example 1:}
\begin{verbatim}
Input: nums = [3, 2, 1, 5, 6, 4], k = 2
Output: 5
Explanation: The sorted array is [1, 2, 3, 4, 5, 6]. The 2nd largest is 5.
\end{verbatim}

\textbf{Example 2:}
\begin{verbatim}
Input: nums = [3, 2, 3, 1, 2, 4, 5, 5, 6], k = 4
Output: 4
Explanation: The sorted array is [1, 2, 2, 3, 3, 4, 5, 5, 6]. The 4th largest is 4.
\end{verbatim}

\textbf{Constraints:}
- \(1 \leq \texttt{nums.length} \leq 10^5\)
- \(-10^4 \leq \texttt{nums}[i] \leq 10^4\)
- \(1 \leq \texttt{k} \leq \texttt{nums.length}\)

---

\section*{Algorithmic Approaches}

This problem can be efficiently solved using the following techniques:

\subsection*{1. Min-Heap Approach (\(O(n \log k)\))}
Use a min-heap of size \(k\) to maintain the \(k\)-th largest elements seen so far:
1. Push each element into the heap.
2. If the heap size exceeds \(k\), remove the smallest element (top of the heap).
3. At the end, the top of the heap is the \(k\)-th largest element.

\subsection*{2. Quickselect (\(O(n)\) Average Case)}
Quickselect is an optimization of the QuickSort algorithm that focuses only on the partition containing the desired \(k\)-th largest element:
1. Partition the array around a pivot element.
2. Compare the pivot’s position to \(n-k\) (index for \(k\)-th largest element in sorted order):
    - If equal, return the pivot.
    - If smaller, search the right partition.
    - If larger, search the left partition.
3. Repeat until the \(k\)-th largest element is found.

---

\subsection*{Complexities}
1. **Min-Heap Approach:**
   - Time Complexity: \(O(n \log k)\), as each insertion/removal operation in the heap is \(O(\log k)\).
   - Space Complexity: \(O(k)\), for the heap.

2. **Quickselect:**
   - Time Complexity: \(O(n)\) on average, \(O(n^2)\) in the worst case (unbalanced partitions).
   - Space Complexity: \(O(1)\), as it operates in-place.

---

\section*{Python Implementations}

\subsection*{Min-Heap Approach (\(O(n \log k)\))}
\begin{fullwidth}
\begin{lstlisting}[language=Python]
import heapq

def findKthLargest(nums, k):
    heap = []
    for num in nums:
        heapq.heappush(heap, num)
        if len(heap) > k:
            heapq.heappop(heap)
    return heapq.heappop(heap)

# Example usage:
nums = [3, 2, 1, 5, 6, 4]
k = 2
print(findKthLargest(nums, k))  # Output: 5
\end{lstlisting}
\end{fullwidth}

\subsection*{Quickselect Approach (\(O(n)\) Average Case)}
\begin{fullwidth}
\begin{lstlisting}[language=Python]
import random

def findKthLargest(nums, k):
    def partition(left, right, pivot_index):
        pivot = nums[pivot_index]
        nums[pivot_index], nums[right] = nums[right], nums[pivot_index]
        store_index = left
        for i in range(left, right):
            if nums[i] > pivot:
                nums[store_index], nums[i] = nums[i], nums[store_index]
                store_index += 1
        nums[store_index], nums[right] = nums[right], nums[store_index]
        return store_index

    left, right = 0, len(nums) - 1
    while True:
        pivot_index = random.randint(left, right)
        pivot_index = partition(left, right, pivot_index)
        if pivot_index == k - 1:
            return nums[pivot_index]
        elif pivot_index < k - 1:
            left = pivot_index + 1
        else:
            right = pivot_index - 1

# Example usage:
nums = [3, 2, 1, 5, 6, 4]
k = 2
print(findKthLargest(nums, k))  # Output: 5
\end{lstlisting}
\end{fullwidth}

---

\section*{Why These Approaches?}
- The Min-Heap approach is intuitive and robust, especially for streaming data or cases where \(k\) is small.
- Quickselect is ideal for large datasets as it is faster on average and has \(O(1)\) space complexity.

---

\section*{Similar Problems}
1. **Kth Smallest Element in a Sorted Matrix:** A similar ranking problem in a 2D matrix.
2. **Find Median of Two Sorted Arrays:** Combines binary search and partitioning concepts.
3. **Top K Frequent Elements:** Uses heaps or hash maps to find the most frequent elements.

---

\section*{Corner Cases to Test}
1. \(k = 1\): The largest element in the array.
2. \(k = \texttt{len(nums)}\): The smallest element in the array.
3. Array with duplicates: Ensure the rank logic works correctly.
4. Array with negative numbers: Test arrays containing both positive and negative integers.

---

\section*{Conclusion}
The \(k\)-th Largest Element in an Array problem highlights the importance of efficient selection algorithms like Quickselect and heap-based optimizations. These techniques ensure performance even for large datasets and demonstrate the versatility of divide-and-conquer and heap-based approaches in real-world scenarios.