% Filename: maximum_depth_of_binary_tree.tex

\problemsection{Maximum Depth of Binary Tree}\marginpar{Determine the maximum depth of a binary tree using recursion or iteration.}

\textbf{Problem Statement}

This problem involves finding the maximum depth (or height) of a binary tree. The maximum depth is the number of nodes along the longest path from the root node down to the farthest leaf node.

\textbf{Algorithmic Approach}

To solve this problem, you can use either a recursive approach or an iterative approach using level-order traversal. Below, both methods are described:

\begin{enumerate}
    \item \textbf{Recursive Approach (Depth-First Search)}:
    \begin{itemize}
        \item \textbf{Base Case}: If the current node is \texttt{NULL}, return 0.
        \item \textbf{Recursive Case}: 
        \begin{enumerate}
            \item Recursively find the maximum depth of the left subtree.
            \item Recursively find the maximum depth of the right subtree.
            \item The maximum depth at the current node is the greater of the two depths plus one (for the current node).
        \end{enumerate}
    \end{itemize}
    
    \item \textbf{Iterative Approach (Breadth-First Search)}:
    \begin{itemize}
        \item Use a queue to perform level-order traversal.
        \item Initialize the depth to 0.
        \item While the queue is not empty:
        \begin{enumerate}
            \item Increment the depth.
            \item For each node at the current level, enqueue its non-\texttt{NULL} left and right children.
        \end{enumerate}
    \end{itemize}
\end{enumerate}

\textbf{Complexities}

\begin{itemize}
    \item \textbf{Time Complexity}: The time complexity is \(O(n)\), where \(n\) is the number of nodes in the tree. Each node is visited exactly once.
    \item \textbf{Space Complexity}: 
    \begin{itemize}
        \item \textbf{Recursive Approach}: \(O(h)\), where \(h\) is the height of the tree, due to the call stack.
        \item \textbf{Iterative Approach}: \(O(w)\), where \(w\) is the maximum width of the tree, since at most \(w\) nodes are stored in the queue at any time.
    \end{itemize}
\end{itemize}

\textbf{Python Implementation}\marginpar{Implementing both recursive and iterative solutions for flexibility.}

\begin{lstlisting}[language=Python, xleftmargin=0.02\textwidth, xrightmargin=0.02\textwidth]
# Definition for a binary tree node.
class TreeNode:
    def __init__(self, val=0, left=None, right=None):
        self.val = val
        self.left = left
        self.right = right

# Recursive Approach
def maxDepthRecursive(root: TreeNode) -> int:
    if not root:
        return 0
    else:
        left_depth = maxDepthRecursive(root.left)
        right_depth = maxDepthRecursive(root.right)
        return max(left_depth, right_depth) + 1

# Iterative Approach
from collections import deque

def maxDepthIterative(root: TreeNode) -> int:
    if not root:
        return 0
    queue = deque([root])
    depth = 0
    while queue:
        depth += 1
        level_length = len(queue)
        for _ in range(level_length):
            node = queue.popleft()
            if node.left:
                queue.append(node.left)
            if node.right:
                queue.append(node.right)
    return depth

# Example usage:
# Constructing the following binary tree
#       3
#      / \
#     9  20
#        / \
#       15  7
root = TreeNode(3)
root.left = TreeNode(9)
root.right = TreeNode(20, TreeNode(15), TreeNode(7))

print(maxDepthRecursive(root))  # Output: 3
print(maxDepthIterative(root))  # Output: 3
\end{lstlisting}

\textbf{Explanation}

The function \texttt{maxDepth} (both recursive and iterative versions) calculates the maximum depth of a binary tree by exploring all nodes. 

\begin{itemize}
    \item \textbf{Recursive Approach}: 
    \begin{itemize}
        \item The function calls itself on the left and right children of the current node.
        \item It computes the depth of each subtree and returns the greater of the two, adding one to account for the current node.
    \end{itemize}
    
    \item \textbf{Iterative Approach}:
    \begin{itemize}
        \item Utilizes a queue to perform a level-order traversal.
        \item For each level, it increments the depth and enqueues the children of all nodes at that level.
    \end{itemize}
\end{itemize}

\textbf{Why This Approach}

\begin{itemize}
    \item \textbf{Recursive Approach}: Mirrors the natural recursive structure of trees, making the implementation straightforward and intuitive.
    \item \textbf{Iterative Approach}: Useful for environments with limited stack space or when avoiding recursion. It also provides a clear separation of tree levels.
\end{itemize}

\textbf{Alternative Approaches}

An alternative method involves using Depth-First Search (DFS) iteratively with a stack instead of recursion. While similar in purpose to the recursive approach, this method can be more memory-efficient in certain scenarios and avoids potential stack overflow issues in deep trees.

\textbf{Similar Problems to This One}

Similar tree-related problems include finding the minimum depth of a binary tree (\textit{Minimum Depth of Binary Tree}), determining if a tree is balanced (\textit{Balanced Binary Tree}), and calculating the diameter of a binary tree (\textit{Diameter of Binary Tree}).

\textbf{Things to Keep in Mind and Tricks}

\begin{itemize}
    \item \textbf{Base Cases Are Crucial}: Always handle cases where the tree or subtree is empty to prevent unnecessary computations and potential errors.
    \item \textbf{Choosing the Right Traversal}: Decide between depth-first and breadth-first approaches based on the specific requirements and constraints of the problem.
    \item \textbf{Space Optimization}: Be mindful of the space complexity, especially when dealing with very deep or very wide trees.
    \item \textbf{Understanding Tree Properties}: A solid grasp of tree properties and traversal methods enhances problem-solving efficiency.
\end{itemize}

\textbf{Corner and Special Cases to Test When Writing the Code}

\begin{itemize}
    \item \textbf{Empty Tree}: A tree with no nodes should return a depth of 0.
    \item \textbf{Single Node}: A tree with only one node should return a depth of 1.
    \item \textbf{Unbalanced Tree}: Trees that are skewed to the left or right should correctly return the depth corresponding to the longest path.
    \item \textbf{Full Binary Tree}: Trees where every node has two children should return a depth that reflects the number of levels.
    \item \textbf{Large Tree}: Testing with a large number of nodes ensures that the implementation handles deep recursion or large queues efficiently.
\end{itemize}