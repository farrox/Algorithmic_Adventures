% filename: min_stack.tex

\problemsection{Min Stack}\marginpar{This problem introduces the concept of augmenting stack operations with auxiliary data to achieve additional functionality.}

\textbf{Problem Description:}  
Design a stack that supports the following operations in constant time:

\begin{itemize}
    \item \texttt{push(x)}: Push element \texttt{x} onto the stack.
    \item \texttt{pop()}: Removes the element on top of the stack.
    \item \texttt{top()}: Get the top element.
    \item \texttt{getMin()}: Retrieve the minimum element in the stack.
\end{itemize}

\textbf{Notes:}
\begin{itemize}
    \item All operations must run in \(O(1)\) time complexity.
    \item You may assume that all inputs are valid integers.
\end{itemize}

\marginpar{Efficient retrieval of the minimum element enhances the utility of the stack in scenarios requiring dynamic minimum tracking.}

\textbf{Solution Overview:}  
To achieve constant-time retrieval of the minimum element, we can augment the stack to keep track of the current minimum at each level. There are several approaches:

\begin{enumerate}
    \item \textbf{Using an Auxiliary Stack:}
        \begin{itemize}
            \item Maintain an additional stack, \texttt{minStack}, that stores the minimum value at each level.
            \item When pushing a new element, compare it with the current minimum and push the lesser of the two onto \texttt{minStack}.
            \item When popping, pop from both the main stack and \texttt{minStack}.
            \item \texttt{getMin()} simply returns the top of \texttt{minStack}.
        \end{itemize}
    \item \textbf{Using a Single Stack with Value-Difference:}
        \begin{itemize}
            \item Store the difference between the current value and the minimum value.
            \item Use a variable to keep track of the minimum.
            \item This method is more space-efficient but more complex.
        \end{itemize}
\end{enumerate}

We will focus on the first approach for its simplicity and clarity.

\textbf{Implementation Details:}  
Here's an implementation using two stacks in Python:

\begin{fullwidth}
\begin{lstlisting}[language=Python]
class MinStack:
    def __init__(self):
        self.stack = []
        self.minStack = []
    
    def push(self, x):
        self.stack.append(x)
        # If minStack is empty or x is smaller than the current minimum
        if not self.minStack or x <= self.minStack[-1]:
            self.minStack.append(x)
        else:
            # Repeat the current minimum
            self.minStack.append(self.minStack[-1])
    
    def pop(self):
        if not self.stack:
            raise Exception("Stack Underflow")
        self.stack.pop()
        self.minStack.pop()
    
    def top(self):
        if not self.stack:
            raise Exception("Stack is empty")
        return self.stack[-1]
    
    def getMin(self):
        if not self.minStack:
            raise Exception("Stack is empty")
        return self.minStack[-1]

# Example usage:
min_stack = MinStack()
min_stack.push(-2)
min_stack.push(0)
min_stack.push(-3)
print(min_stack.getMin())   # Output: -3
min_stack.pop()
print(min_stack.top())      # Output: 0
print(min_stack.getMin())   # Output: -2
\end{lstlisting}
\end{fullwidth}\marginpar{By mirroring the main stack's operations in \texttt{minStack}, we maintain synchronization of minimum values.}

\textbf{Complexities:}

\begin{itemize}
    \item \textbf{Time Complexity:} All operations (\texttt{push}, \texttt{pop}, \texttt{top}, \texttt{getMin}) run in \(O(1)\) time.
    \item \textbf{Space Complexity:} \(O(n)\), where \(n\) is the number of elements in the stack, due to the additional \texttt{minStack}.
\end{itemize}\marginpar{The extra space used is proportional to the number of elements but ensures constant-time operations.}

\textbf{Alternative Approach:}  
Using a single stack with value-difference optimization:

\begin{itemize}
    \item Keep a variable \texttt{min} to track the current minimum.
    \item When pushing, if the new element is less than or equal to \texttt{min}, push a special marker or encoded value.
    \item When popping, if the popped value indicates a change in \texttt{min}, update \texttt{min} accordingly.
\end{itemize}

\textbf{Implementation Using Single Stack:}

\begin{fullwidth}
\begin{lstlisting}[language=Python]
class MinStack:
    def __init__(self):
        self.stack = []
        self.min = None
    
    def push(self, x):
        if not self.stack:
            self.stack.append(x)
            self.min = x
        elif x <= self.min:
            # Store previous min
            self.stack.append(2 * x - self.min)
            self.min = x
        else:
            self.stack.append(x)
    
    def pop(self):
        if not self.stack:
            raise Exception("Stack Underflow")
        top = self.stack.pop()
        if top < self.min:
            # Retrieve previous min
            self.min = 2 * self.min - top
    
    def top(self):
        if not self.stack:
            raise Exception("Stack is empty")
        top = self.stack[-1]
        if top < self.min:
            return self.min
        else:
            return top
    
    def getMin(self):
        if not self.stack:
            raise Exception("Stack is empty")
        return self.min

# Example usage:
min_stack = MinStack()
min_stack.push(-2)
min_stack.push(0)
min_stack.push(-3)
print(min_stack.getMin())   # Output: -3
min_stack.pop()
print(min_stack.top())      # Output: 0
print(min_stack.getMin())   # Output: -2
\end{lstlisting}
\end{fullwidth}\marginpar{This method saves space but adds complexity in handling encoded values.}

\textbf{Complexities of Alternative Approach:}

\begin{itemize}
    \item \textbf{Time Complexity:} All operations run in \(O(1)\) time.
    \item \textbf{Space Complexity:} \(O(n)\), but potentially less than the two-stack approach in practice.
\end{itemize}\marginpar{Optimal for environments where space is at a premium and the added complexity is acceptable.}

\textbf{Why This Approach?}

Using an auxiliary stack simplifies the implementation and makes the logic straightforward. It provides clear visibility into the minimum values at each stack level and is less error-prone compared to the single stack method.

\textbf{Corner Cases to Test:}

\begin{itemize}
    \item \textbf{Empty Stack Operations:} Ensure that calling \texttt{pop}, \texttt{top}, or \texttt{getMin} on an empty stack is handled properly.
    \item \textbf{Duplicate Minimums:} Test scenarios where multiple elements have the same minimum value.
    \item \textbf{Negative Values:} Verify that the stack handles negative integers correctly.
    \item \textbf{Single Element Stack:} Confirm that operations work as expected when only one element is in the stack.
\end{itemize}\marginpar{Thorough testing ensures robustness against edge cases and potential errors.}

\textbf{Similar Problems:}

\begin{itemize}
    \item \textbf{Max Stack:} Design a stack that supports retrieving the maximum element in constant time.
    \item \textbf{Stack with Increment Operation:} Implement a stack with an operation to increment the bottom \(k\) elements by a given value.
    \item \textbf{Design a Queue using Stacks:} Explore the reverse scenario of implementing a queue using stack operations.
\end{itemize}\marginpar{Exploring variations of the problem deepens understanding of stack manipulation and augmentation.}

\textbf{Conclusion:}

The Min Stack problem exemplifies how data structures can be enhanced to provide additional functionality without compromising efficiency. By intelligently managing auxiliary data, we can extend the capabilities of a standard stack to meet specific requirements, such as constant-time retrieval of the minimum element. Mastery of such techniques is essential for designing efficient algorithms and solving complex computational problems.\marginpar{Augmenting data structures is a key skill in advanced algorithm design.}