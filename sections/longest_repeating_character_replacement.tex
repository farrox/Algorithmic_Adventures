% Filename: longest_repeating_character_replacement.tex

\problemsection{Longest Repeating Character Replacement}
\label{problem:Longest_Repeating_Character_Replacement}

The "Longest Repeating Character Replacement" problem is a classic example of how a combination of string manipulation and the sliding window technique can be employed to solve complex problems efficiently. In this problem, you are given a string \texttt{s} and an integer \texttt{k}, which represents the maximum number of character replacements you are allowed to perform to create the longest possible substring containing the same letters.

\section*{Problem Statement}
The objective is to determine the length of the longest substring of \texttt{s} that can be obtained by replacing at most \texttt{k} characters within the substring. Replacements can be done with any uppercase English letter.

\textbf{Example:}

Input: \( s = "AABABBA", k = 1 \)

Output: \( 4 \)

\textbf{Explanation:}

Given the string \( s = "AABABBA" \) and \( k = 1 \), we can replace one occurrence of 'B' in "AABA" with 'A', resulting in "AAAA". This substring has a length of 4, making it the longest substring possible with at most one character replacement.

LeetCode link: \href{https://leetcode.com/problems/longest-repeating-character-replacement/}{Longest Repeating Character Replacement}

\section*{Algorithmic Approach}
To solve this problem efficiently, we employ the sliding window technique. By maintaining a window of characters, we can dynamically calculate the maximum frequency of a single character within the window and determine whether the remaining characters can be replaced within the allowed limit \( k \). If the number of characters to replace exceeds \( k \), the window is reduced by incrementing the start pointer.

\textbf{Steps:}
\begin{itemize}
    \item Use a sliding window to traverse the string, tracking the frequency of characters in the current window using a hash map or counter.
    \item Maintain the maximum frequency of a single character in the window.
    \item If the size of the window minus the maximum frequency exceeds \( k \), reduce the window size by moving the start pointer.
    \item Update the length of the longest valid substring found so far.
\end{itemize}

\section*{Complexities}
\begin{itemize}
    \item \textbf{Time Complexity:} \(O(n)\), where \(n\) is the length of the string. Each character is processed at most twice (once when extending the window and once when shrinking it).
    \item \textbf{Space Complexity:} \(O(1)\), as the counter only tracks the frequency of 26 possible uppercase English letters.
\end{itemize}

\section*{Python Implementation}
Below is the Python implementation of the sliding window approach:

\begin{fullwidth}
\begin{lstlisting}[language=Python]
from collections import Counter

def characterReplacement(s, k):
    count = Counter()
    max_count = 0
    max_length = 0
    left = 0
    
    for right in range(len(s)):
        count[s[right]] += 1
        max_count = max(max_count, count[s[right]])
        
        # If replacements needed exceed k, shrink the window
        if right - left + 1 - max_count > k:
            count[s[left]] -= 1
            left += 1
        
        # Update the maximum length of the valid window
        max_length = max(max_length, right - left + 1)
    
    return max_length

# Example usage:
s = "AABABBA"
k = 1
print(characterReplacement(s, k))  # Output: 4
\end{lstlisting}
\end{fullwidth}

\subsection*{Explanation of the Code}
The algorithm uses a sliding window defined by the pointers \texttt{left} and \texttt{right}. As the \texttt{right} pointer expands the window, the character frequencies are updated. If the number of characters requiring replacement exceeds \( k \), the \texttt{left} pointer is moved forward to shrink the window. At each step, the maximum valid window size is recorded.

\section*{Why This Approach?}
The sliding window approach is ideal for problems involving substrings and constraints, such as limited replacements. Its linear time complexity ensures efficiency, making it well-suited for large input sizes. The use of a counter to dynamically track character frequencies allows for real-time window validation without rescanning the substring.

\section*{Alternative Approaches}
\begin{itemize}
    \item **Dynamic Programming:**  
    A table-based approach to calculate the maximum length for substrings ending at each position. This method has a higher space complexity compared to the sliding window.
    \item **Brute Force:**  
    Check all substrings and calculate the replacements required for each. This method is inefficient with a time complexity of \(O(n^2 \cdot 26)\).
\end{itemize}

\section*{Similar Problems}
\begin{itemize}
    \item \textbf{Longest Substring with At Most Two Distinct Characters:} Find the length of the longest substring containing at most two distinct characters.
    \item \textbf{Longest Substring Without Repeating Characters:} Determine the length of the longest substring without repeating characters.
    \item \textbf{Minimum Window Substring:} Find the minimum window in a string that contains all characters of another string.
\end{itemize}

\section*{Things to Keep in Mind and Tricks}
\begin{itemize}
    \item Track the frequency of the most common character in the window to minimize unnecessary calculations.
    \item Adjust the window size dynamically to ensure that the number of replacements required does not exceed \( k \).
    \item Handle edge cases like empty strings, strings with only one character, or \( k = 0 \).
\end{itemize}

\section*{Corner and Special Cases to Test}
\begin{itemize}
    \item **Empty String:** Input: \( s = "" \), \( k = 1 \) (should return \(0\)).
    \item **All Identical Characters:** Input: \( s = "AAAA", k = 2 \) (should return \(4\)).
    \item **All Unique Characters:** Input: \( s = "ABCD", k = 1 \) (should return \(2\)).
    \item **No Replacements Allowed:** Input: \( s = "AABABBA", k = 0 \) (should return \(2\)).
\end{itemize}

\section*{Conclusion}
The **Longest Repeating Character Replacement** problem demonstrates the effectiveness of the sliding window technique in solving substring-related challenges. Mastering this problem equips you with the tools to tackle similar problems requiring dynamic adjustments to constraints while processing strings.