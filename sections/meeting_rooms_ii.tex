% Filename: meeting_rooms_ii.tex

\problemsection{Meeting Rooms II}\marginpar{This problem involves finding the minimum number of meeting rooms required.}

\textbf{Problem Description:}

Given an array of meeting time intervals consisting of start and end times \texttt{[[} \(s_1\)\texttt{,} \(e_1\)\texttt{], [} \(s_2\)\texttt{,} \(e_2\)\texttt{], ...]} where \(s_i < e_i\), find the minimum number of conference rooms required.

\textbf{Example 1:}

\begin{itemize}
    \item \textbf{Input:} \texttt{[[0,30],[5,10],[15,20]]}
    \item \textbf{Output:} \texttt{2}
\end{itemize}

\textbf{Example 2:}

\begin{itemize}
    \item \textbf{Input:} \texttt{[[7,10],[2,4]]}
    \item \textbf{Output:} \texttt{1}
\end{itemize}

\textbf{Solution Overview:}

Separate the start and end times, sort them, and use two pointers to iterate through them. Keep track of the number of rooms needed at each point in time.

% Remove the fullwidth environment
% \begin{lstlisting}[language=Python, xleftmargin=0.1\textwidth]
    % def minMeetingRooms(intervals):
    % starts = sorted([i[0] for i in intervals])
    % ends = sorted([i[1] for i in intervals])
    % s, e = 0, 0
    % numRooms = 0
    % while s < len(starts):
    %     if starts[s] < ends[e]:
    %         numRooms += 1
    %         s += 1
    %     else:
    %         e += 1
    %             s += 1
    % return numRooms

% # Example usage:
% print(minMeetingRooms([[0,30],[5,10],[15,20]]))  # Output: 2
% print(minMeetingRooms([[7,10],[2,4]]))           # Output: 1
% \end{lstlisting}