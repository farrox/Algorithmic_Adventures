% Filename: basic_calculator_ii.tex

\problemsection{Basic Calculator II}\marginpar{This problem extends the basic calculator to include multiplication and division.}

\textbf{Problem Description:}  
Implement a basic calculator to evaluate a simple expression string.

The expression string contains only non-negative integers, \texttt{'+'}, \texttt{'-'}, \texttt{'*'}, \texttt{'/'}, and empty spaces \texttt{' '}. The integer division should truncate toward zero.

You may assume that the given expression is always valid.

\textbf{Example 1:}

\begin{itemize}
    \item \textbf{Input:} \texttt{"3+2*2"}
    \item \textbf{Output:} \texttt{7}
\end{itemize}

\textbf{Example 2:}

\begin{itemize}
    \item \textbf{Input:} \texttt{" 3/2 "}
    \item \textbf{Output:} \texttt{1}
\end{itemize}

\textbf{Example 3:}

\begin{itemize}
    \item \textbf{Input:} \texttt{" 3+5 / 2 "}
    \item \textbf{Output:} \texttt{5}
\end{itemize}

\textbf{Solution Overview:}  
Use a stack to handle multiplication and division immediately, while addition and subtraction are deferred. Iterate through the string:

\begin{itemize}
    \item Build the current number if the character is a digit.
    \item If an operator or the end of the string is reached:
        \begin{itemize}
            \item If the previous operator is '+', push the number onto the stack.
            \item If '-', push the negative number onto the stack.
            \item If '*', pop the stack, multiply, and push the result.
            \item If '/', pop the stack, divide, and push the result (truncate toward zero).
            \item Update the previous operator.
            \item Reset the current number.
        \end{itemize}
\end{itemize}

At the end, sum all numbers in the stack for the final result.

\begin{lstlisting}[language=Python]
def calculate(s):
    if not s:
        return 0
    s = s.replace(' ', '')
    stack = []
    num = 0
    prev_op = '+'
    i = 0
    while i < len(s):
        char = s[i]
        if char.isdigit():
            num = num * 10 + int(char)
        if char in '+-*/' or i == len(s) - 1:
            if prev_op == '+':
                stack.append(num)
            elif prev_op == '-':
                stack.append(-num)
            elif prev_op == '*':
                stack.append(stack.pop() * num)
            elif prev_op == '/':
                temp = stack.pop()
                if temp < 0:
                    stack.append(-(-temp // num))
                else:
                    stack.append(temp // num)
            prev_op = char
            num = 0
        i += 1
    return sum(stack)

# Example usage:
print(calculate("3+2*2"))       # Output: 7
print(calculate(" 3/2 "))       # Output: 1
print(calculate(" 3+5 / 2 "))   # Output: 5
\end{lstlisting}