% Filename: edit_distance.tex

\problemsection{Edit Distance}
\label{problem:edit_distance}
\marginnote{A fundamental dynamic programming problem that demonstrates optimal substructure and is widely used in spell checkers, DNA sequence alignment, and natural language processing.}

\section*{Problem Statement}
Given two strings \texttt{word1} and \texttt{word2}, find the minimum number of operations needed to convert \texttt{word1} into \texttt{word2}. The allowed operations are:
\begin{itemize}
    \item Insert a character
    \item Delete a character
    \item Replace a character
\end{itemize}

\textbf{Examples:}
\begin{verbatim}
Input:  word1 = "horse", word2 = "ros"
Output: 3
Explanation: 
1. horse → rorse (replace 'h' with 'r')
2. rorse → rose  (delete 'r')
3. rose  → ros   (delete 'e')

Input:  word1 = "intention", word2 = "execution"
Output: 5
\end{verbatim}

\section*{Key Insights}
\begin{itemize}
    \item If characters match, no operation needed
    \item If characters differ, we have three choices:
        \begin{itemize}
            \item Insert: 1 + distance(i, j-1)
            \item Delete: 1 + distance(i-1, j)
            \item Replace: 1 + distance(i-1, j-1)
        \end{itemize}
    \item Base cases: empty string requires length operations
\end{itemize}

\section*{Dynamic Programming Approach}
The problem exhibits:
\begin{itemize}
    \item \textbf{Optimal Substructure:} Solution built from optimal solutions of subproblems
    \item \textbf{Overlapping Subproblems:} Same subproblems computed multiple times
    \item \textbf{State Definition:} dp[i][j] = minimum operations to convert word1[0:i] to word2[0:j]
\end{itemize}

\marginnote{The DP table represents the minimum operations needed for prefixes of both strings, building up to the final solution.}

\section*{Implementation}
\begin{fullwidth}
\begin{lstlisting}[language=Python]
class Solution:
    def minDistance(self, word1: str, word2: str) -> int:
        # Handle edge cases
        if not word1 and not word2:
            return 0
        if not word1:
            return len(word2)
        if not word2:
            return len(word1)
            
        m, n = len(word1), len(word2)
        # Initialize DP table
        dp = [[0] * (n + 1) for _ in range(m + 1)]
        
        # Base cases: converting to/from empty string
        for i in range(m + 1):
            dp[i][0] = i  # deletions
        for j in range(n + 1):
            dp[0][j] = j  # insertions
            
        # Fill DP table
        for i in range(1, m + 1):
            for j in range(1, n + 1):
                if word1[i-1] == word2[j-1]:
                    dp[i][j] = dp[i-1][j-1]  # no operation needed
                else:
                    dp[i][j] = 1 + min(
                        dp[i-1][j],    # deletion
                        dp[i][j-1],    # insertion
                        dp[i-1][j-1]   # replacement
                    )
        
        return dp[m][n]

# Test cases
def test_edit_distance():
    solution = Solution()
    
    # Basic cases
    assert solution.minDistance("horse", "ros") == 3
    assert solution.minDistance("intention", "execution") == 5
    
    # Edge cases
    assert solution.minDistance("", "") == 0
    assert solution.minDistance("a", "") == 1
    assert solution.minDistance("", "a") == 1
    
    # Same strings
    assert solution.minDistance("hello", "hello") == 0
    
    # Completely different strings
    assert solution.minDistance("abc", "def") == 3
\end{lstlisting}
\end{fullwidth}

\section*{Complexity Analysis}
\begin{itemize}
    \item \textbf{Time Complexity:} O(mn) where m, n are lengths of input strings
    \item \textbf{Space Complexity:} O(mn) for the DP table
    \item \textbf{Space Optimization:} Can be reduced to O(min(m,n)) using rolling arrays
\end{itemize}

\section*{Common Pitfalls}
\begin{itemize}
    \item Forgetting to handle empty string cases
    \item Incorrect base case initialization
    \item Not considering all three operations at each step
    \item Off-by-one errors in string indices
\end{itemize}

\section*{Interview Tips}
\begin{itemize}
    \item Start with a small example to illustrate the approach
    \item Draw the DP table to explain state transitions
    \item Mention space optimization possibilities
    \item Discuss real-world applications:
        \begin{itemize}
            \item Spell checkers
            \item DNA sequence alignment
            \item Natural language processing
            \item Fuzzy string matching
        \end{itemize}
\end{itemize}

\section*{Related Problems}
\begin{itemize}
    \item One Edit Distance
    \item Delete Distance
    \item Longest Common Subsequence
    \item Regular Expression Matching
\end{itemize}

\printindex
