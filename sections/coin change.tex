
ewpage
\chapter{Coin Change}
\label{chap:Coin_Change}
The "Coin Change" problem is a well-known dynamic programming challenge that requires determining the minimum number of coins needed to make a certain amount of money given coin denominations.

\section*{Problem Statement}
Given an array `coins` of different denominations and an integer `amount` representing a total amount of money, the aim is to compute the fewest number of coins required to make up that amount. If it is not possible to make the amount with any combination of the coins, the function should return -1.

\textbf{Example 1:}
\begin{verbatim}
Input: coins = [1, 2, 5], amount = 11
Output: 3
Explanation: 11 = 5 + 5 + 1
\end{verbatim}

\textbf{Example 2:}
\begin{verbatim}
Input: coins = [2], amount = 3
Output: -1
\end{verbatim}

\textbf{Example 3:}
\begin{verbatim}
Input: coins = [1], amount = 0
Output: 0
\end{verbatim}

LeetCode link: \href{https://leetcode.com/problems/coin-change/}{Coin Change}

\section*{Algorithmic Approach}
The solution utilizes dynamic programming to build a table that records the minimal number of coins necessary for each sub-amount up to the target amount. We use an array `dp` with `amount + 1` entries, where `dp[i]` holds the minimum number of coins required for the value `i`. Initially, `dp[0]` is set to 0 (since zero coins are needed for amount zero), and the other entries are set to some maximum value to symbolize an unreachable state.

\section*{Complexities}
\begin{itemize}
	\item \textbf{Time Complexity:} The overall time complexity is \(O(amount \times \text{number of coins})\) because we traverse through all sub-amounts for each coin.
	\item \textbf{Space Complexity:} The space complexity is \(O(amount)\) since we maintain a 1-dimensional DP array to store minimum coins for each amount up to the target amount.
\end{itemize}


ewpage % Start Python Implementation on a new page
\section*{Python Implementation}
Below is the complete Python code for the `coinChange` function, which finds the least number of coins needed given denominations to make up a certain amount:

\begin{fullwidth}
\begin{lstlisting}[language=Python]
def coinChange(coins, amount):
    # Initialize DP array with amount+1, which is an invalid value as it's
    # bigger than the amount itself (since the maximum number of coins to
    # make any amount is amount (using all 1s)).
    dp = [amount + 1] * (amount + 1)
    dp[0] = 0  # Base case: no coins needed for amount 0

    # Solve for all amounts from 1 to amount
    for a in range(1, amount + 1):
        for coin in coins:
            if coin <= a:
                # If coin is not greater than the amount
                dp[a] = min(dp[a], 1 + dp[a - coin])
    
    # If dp[amount] is still amount+1, then it was not possible to make change
    return dp[amount] if dp[amount] != amount + 1 else -1
\end{lstlisting}

\end{fullwidth}

This implementation first initializes the `dp` array with a default value that's not viable, signifying that for that amount, the minimum number of coins is still unknown. For every sub-amount up to the total amount, it tries each coin and determines if including that coin would result in a smaller number of coins to make the exact change.

\section*{Why this approach}
The dynamic programming approach was chosen for its ability to break down the problem into subproblems and build up the answer efficiently. This characteristic of DP ensures that we do not recompute the minimum coins for the amounts we have already solved.

\section*{Alternative approaches}
There are alternative approaches, such as a recursive solution or using a BFS approach, but these can be less efficient in terms of time complexity unless memoization is used with recursion. The greedy approach doesn't work for all sets of coin denominations.

\section*{Similar problems to this one}
This problem is similar to other classic DP problems like "Minimum Path Sum," "0/1 Knapsack," and "Longest Increasing Subsequence," where subproblems are solved to contribute to the overall solution.

\section*{Things to keep in mind and tricks}
One of the key things