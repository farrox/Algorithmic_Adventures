% Filename: letter_combinations_of_a_phone_number.tex

\problemsection{Letter Combinations of a Phone Number}
\label{problem:Letter_Combinations_of_a_Phone_Number}

The **Letter Combinations of a Phone Number** problem is a classic example of recursion and backtracking used to generate all possible combinations of letters mapped to digits on a phone keypad.

\subsection*{Problem Statement}
Given a string containing digits from \(2\) to \(9\) (inclusive), return all possible letter combinations that the digits could represent. Each digit maps to specific letters as shown on a traditional phone keypad.

\textbf{Input:}
- A string \texttt{digits}, consisting of digits from \(2\) to \(9\).

\textbf{Output:}
- A list of strings, where each string represents a possible combination of letters.

\textbf{Example 1:}

Input: \texttt{digits = "23"}

Output: \[
["ad", "ae", "af", "bd", "be", "bf", "cd", "ce", "cf"]
\]

\textbf{Example 2:}

Input: \texttt{digits = ""}

Output: \[
[]
\]

\textbf{Example 3:}

Input: \texttt{digits = "2"}

Output: \[
["a", "b", "c"]
\]

\subsection*{Mapping of Digits to Letters}
The mapping follows the traditional phone keypad:
\[
\begin{aligned}
2 &\rightarrow \{a, b, c\}, \quad
3 \rightarrow \{d, e, f\}, \quad
4 \rightarrow \{g, h, i\}, \\
5 &\rightarrow \{j, k, l\}, \quad
6 \rightarrow \{m, n, o\}, \quad
7 \rightarrow \{p, q, r, s\}, \\
8 &\rightarrow \{t, u, v\}, \quad
9 \rightarrow \{w, x, y, z\}.
\end{aligned}
\]

\subsection*{Algorithmic Approach}
The problem can be solved using a backtracking approach:
\begin{itemize}
    \item If the input string is empty, return an empty list immediately\sidenote{This handles edge cases where there are no digits to process.}.
    \item Use recursion to explore each possible letter for the current digit and proceed to the next digit.
    \item Maintain a list to track the current combination of letters.
    \item Once all digits are processed, add the complete combination to the result list.
\end{itemize}

\subsection*{Complexities}
\begin{itemize}
    \item \textbf{Time Complexity:} \(O(4^n)\), where \(n\) is the length of \texttt{digits}, since each digit maps to at most 4 letters.
    \item \textbf{Space Complexity:} \(O(n)\), for the recursion stack.
\end{itemize}

\subsection*{Python Implementation}
Below is the Python implementation using backtracking:

\begin{fullwidth}
\begin{lstlisting}[language=Python]
def letterCombinations(digits):
    if not digits:
        return []
    
    # Map digits to corresponding letters
    phone_map = {
        "2": "abc", "3": "def", "4": "ghi", "5": "jkl",
        "6": "mno", "7": "pqrs", "8": "tuv", "9": "wxyz"
    }
    
    def backtrack(index, current):
        # Base case: if the current combination is complete
        if index == len(digits):
            result.append("".join(current))
            return
        
        # Get the letters corresponding to the current digit
        letters = phone_map[digits[index]]
        for letter in letters:
            current.append(letter)  # Add the letter
            backtrack(index + 1, current)  # Recur for the next digit
            current.pop()  # Backtrack by removing the last letter
    
    result = []
    backtrack(0, [])
    return result

# Example usage:
print(letterCombinations("23"))  # Output: ["ad", "ae", "af", "bd", "be", "bf", "cd", "ce", "cf"]
\end{lstlisting}
\end{fullwidth}

\subsection*{Why This Approach?}
Backtracking is ideal for this problem because it systematically explores all valid combinations of letters, pruning invalid paths efficiently. The use of recursion makes the implementation clean and intuitive.

\subsection*{Alternative Approaches}
An iterative approach can also generate combinations using a queue. However, it requires more bookkeeping and is less elegant than the recursive backtracking solution.

\subsection*{Similar Problems}
\begin{itemize}
    \item \textbf{Subsets:} Generate all subsets of a given set.
    \item \textbf{Permutations:} Generate all permutations of a set of numbers.
    \item \textbf{Word Search:} Find words in a grid using backtracking.
\end{itemize}

\subsection*{Things to Keep in Mind and Tricks}
\begin{itemize}
    \item Handle edge cases like empty input or single-digit input separately.
    \item Precompute the mapping of digits to letters to avoid redundancy during recursion.
    \item Optimize by skipping invalid digits (e.g., \(1\) and \(0\) if included in input).
\end{itemize}

\subsection*{Corner and Special Cases to Test}
\begin{itemize}
    \item \textbf{Empty Input:} Input: \texttt{digits = ""}, Output: \texttt{[]}.
    \item \textbf{Single Digit:} Input: \texttt{digits = "7"}, Output: \texttt{["p", "q", "r", "s"]}.
    \item \textbf{Multiple Digits:} Input: \texttt{digits = "234"}, Verify all combinations are generated correctly.
    \item \textbf{Edge Case for Large Input:} Test with longer strings like \texttt{digits = "999"}.
\end{itemize}

\subsection*{Conclusion}
The **Letter Combinations of a Phone Number** problem demonstrates the power of backtracking for generating combinatorial solutions efficiently. By systematically exploring all possibilities and pruning invalid paths, the solution ensures correctness and optimal performance.