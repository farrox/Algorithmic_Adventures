% Filename: binary_tree_level_order_traversal.tex

\problemsection{Binary Tree Level Order Traversal}\marginpar{Traverse a binary tree level by level using BFS or DFS.}

\textbf{Problem Statement}

This problem involves performing a level order traversal on a binary tree. In level order traversal, nodes are visited level by level from left to right. The goal is to return a list of lists, where each sublist contains the values of nodes at each respective level of the tree.

\textbf{Algorithmic Approach}

To solve this problem, you can use either an iterative approach with Breadth-First Search (BFS) or a recursive approach with Depth-First Search (DFS). Below, both methods are described:

\begin{enumerate}
    \item \textbf{Iterative Approach (Breadth-First Search)}:
    \begin{itemize}
        \item \textbf{Use a Queue}: Utilize a queue to keep track of nodes at the current level.
        \item \textbf{Level-by-Level Traversal}:
        \begin{enumerate}
            \item Initialize the queue with the root node.
            \item While the queue is not empty:
            \begin{enumerate}
                \item Determine the number of nodes at the current level (size of the queue).
                \item Iterate through all nodes at this level:
                \begin{enumerate}
                    \item Dequeue a node from the queue.
                    \item Add its value to the current level's list.
                    \item Enqueue its non-\texttt{NULL} left and right children.
                \end{enumerate}
                \item After processing all nodes at the current level, add the level's list to the result.
            \end{enumerate}
        \end{enumerate}
    \end{itemize}
    
    \item \textbf{Recursive Approach (Depth-First Search)}:
    \begin{itemize}
        \item \textbf{Use a Helper Function}: Create a helper function that takes a node and its current level as arguments.
        \item \textbf{Traverse the Tree}:
        \begin{enumerate}
            \item If the node is \texttt{NULL}, return.
            \item If the current level is equal to the size of the result list, append a new sublist for this level.
            \item Add the node's value to the corresponding level's sublist.
            \item Recursively traverse the left and right children, incrementing the level by one.
        \end{enumerate}
    \end{itemize}
\end{enumerate}

\textbf{Complexities}

\begin{itemize}
    \item \textbf{Time Complexity}: The time complexity is \(O(n)\), where \(n\) is the number of nodes in the tree. Each node is visited exactly once.
    \item \textbf{Space Complexity}:
    \begin{itemize}
        \item \textbf{Iterative Approach (BFS)}: \(O(n)\), as in the worst case, the queue will hold all nodes at the last level.
        \item \textbf{Recursive Approach (DFS)}: \(O(n)\), due to the space required to store the result and the recursive call stack.
    \end{itemize}
\end{itemize}

\textbf{Python Implementation}\marginpar{Implementing both BFS and DFS solutions for flexibility.}

\begin{lstlisting}[language=Python, xleftmargin=0.02\textwidth, xrightmargin=0.02\textwidth]
# Definition for a binary tree node.
class TreeNode:
    def __init__(self, val=0, left=None, right=None):
        self.val = val
        self.left = left
        self.right = right

# Iterative Approach (BFS)
from collections import deque

def levelOrderBFS(root: TreeNode) -> list:
    if not root:
        return []
    result = []
    queue = deque([root])
    while queue:
        level_size = len(queue)
        current_level = []
        for _ in range(level_size):
            node = queue.popleft()
            current_level.append(node.val)
            if node.left:
                queue.append(node.left)
            if node.right:
                queue.append(node.right)
        result.append(current_level)
    return result

# Recursive Approach (DFS)
def levelOrderDFS(root: TreeNode) -> list:
    def helper(node: TreeNode, level: int):
        if not node:
            return
        if level == len(result):
            result.append([])
        result[level].append(node.val)
        helper(node.left, level + 1)
        helper(node.right, level + 1)
    
    result = []
    helper(root, 0)
    return result

# Example usage:
# Constructing the following binary tree
#       3
#      / \
#     9  20
#        / \
#       15  7
root = TreeNode(3)
root.left = TreeNode(9)
root.right = TreeNode(20, TreeNode(15), TreeNode(7))

print(levelOrderBFS(root))  # Output: [[3], [9, 20], [15, 7]]
print(levelOrderDFS(root))  # Output: [[3], [9, 20], [15, 7]]
\end{lstlisting}

\textbf{Explanation}

The function \texttt{levelOrder} performs a level order traversal of a binary tree, returning a list of lists where each sublist contains the values of nodes at each level. 

\begin{itemize}
    \item \textbf{Iterative Approach (BFS)}:
    \begin{itemize}
        \item Uses a queue to traverse the tree level by level.
        \item For each level, it records the number of nodes, processes each node by dequeuing it, and enqueues its children.
        \item After processing all nodes at the current level, it appends the collected values to the result list.
    \end{itemize}
    
    \item \textbf{Recursive Approach (DFS)}:
    \begin{itemize}
        \item Utilizes a helper function to traverse the tree depth-first while keeping track of the current level.
        \item If a new level is encountered (i.e., the current level is equal to the length of the result list), a new sublist is appended.
        \item The node's value is added to its corresponding level's sublist.
        \item Recursively processes the left and right children, incrementing the level.
    \end{itemize}
\end{itemize}

\textbf{Why This Approach}

\begin{itemize}
    \item \textbf{Iterative Approach (BFS)}: 
    \begin{itemize}
        \item Naturally fits the level order traversal requirement by exploring nodes level by level.
        \item Efficient in terms of time and space for balanced trees.
    \end{itemize}
    
    \item \textbf{Recursive Approach (DFS)}: 
    \begin{itemize}
        \item Simplifies the traversal by leveraging the call stack to manage levels.
        \item Can be more intuitive for those familiar with recursive tree traversals.
    \end{itemize}
\end{itemize}

\textbf{Alternative Approaches}

An alternative method involves using a stack to perform an iterative Depth-First Search (DFS) while keeping track of node levels. However, the BFS and recursive DFS approaches are generally more straightforward and easier to implement for level order traversal.

\textbf{Similar Problems to This One}

Similar tree traversal problems include finding the minimum depth of a binary tree (\hyperref[problem:minimum_depth_of_binary_tree]{Minimum Depth of Binary Tree}), checking if a tree is symmetric (\hyperref[problem:symmetric_tree]{Symmetric Tree}), and performing a zigzag level order traversal (\hyperref[problem:binary_tree_zigzag_level_order_traversal]{Binary Tree Zigzag Level Order Traversal}).

\textbf{Things to Keep in Mind and Tricks}

\begin{itemize}
    \item \textbf{Handling Empty Trees}: Always check if the root is \texttt{NULL} to handle empty trees gracefully.
    \item \textbf{Level Tracking}: In the iterative approach, keeping track of the current level's size helps in segregating nodes level by level.
    \item \textbf{Space Optimization}: Be mindful of the space used by the queue in BFS or the recursion stack in DFS, especially for very deep or wide trees.
    \item \textbf{Consistent Level Identification}: Ensure that nodes are correctly associated with their respective levels to maintain accurate traversal results.
\end{itemize}

\textbf{Corner and Special Cases to Test When Writing the Code}

\begin{itemize}
    \item \textbf{Empty Tree}: Should return an empty list.
    \item \textbf{Single Node}: Should return a list containing one sublist with the single node's value.
    \item \textbf{Unbalanced Tree}: Trees that are skewed to the left or right should correctly return their respective levels.
    \item \textbf{Complete Binary Tree}: Ensure that all levels except possibly the last are fully filled.
    \item \textbf{Large Tree}: Testing with a large number of nodes ensures that the implementation handles depth and breadth efficiently without performance degradation.
\end{itemize}