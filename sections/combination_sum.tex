% file: combination_sum.tex 

\problemsection{Combination Sum}
\label{problem:combination_sum}
\marginnote{A fundamental backtracking problem that tests understanding of recursive exploration and pruning techniques. Frequently appears in technical interviews.}

\section*{Problem Statement}
Given an array of distinct positive integers \texttt{candidates} and a target integer \texttt{target}, find all unique combinations of numbers from \texttt{candidates} that sum to \texttt{target}. Each number can be used unlimited times.

\textbf{Key Requirements:}
\begin{itemize}
    \item Numbers can be reused unlimited times
    \item Each combination must be unique
    \item All integers are positive
    \item The order within combinations doesn't matter
\end{itemize}

\textbf{Examples:}
\begin{lstlisting}[language=Python]
# Example 1
candidates = [2,3,6,7]
target = 7
output = [[2,2,3],[7]]

# Example 2
candidates = [2,3,5]
target = 8
output = [[2,2,2,2],[2,3,3],[3,5]]
\end{lstlisting}

\section*{Intuition}
The key insight is that at each step, we have two choices:
\begin{itemize}
    \item Include the current number again (since reuse is allowed)
    \item Move to the next number
\end{itemize}

\section*{Solution Strategy}
\begin{enumerate}
    \item Sort candidates for efficient pruning
    \item Use backtracking to explore combinations:
        \begin{itemize}
            \item Track current sum and combination
            \item Stop when sum equals target (solution found)
            \item Stop when sum exceeds target (pruning)
            \item Try including current number again
            \item Try moving to next number
        \end{itemize}
\end{enumerate}

\marginnote{Sorting the candidates enables early pruning when the sum exceeds the target, significantly improving performance.}

\section*{Optimal Implementation}
\begin{fullwidth}
\begin{lstlisting}[language=Python]
from typing import List

class Solution:
    def combinationSum(self, candidates: List[int], target: int) -> List[List[int]]:
        # Handle edge cases
        if not candidates or min(candidates) > target:
            return []
            
        # Sort for efficient pruning
        candidates.sort()
        
        def backtrack(start: int, remaining: int, path: List[int]) -> None:
            if remaining == 0:
                result.append(path[:])  # Found valid combination
                return
                
            for i in range(start, len(candidates)):
                num = candidates[i]
                if num > remaining:  # Pruning: stop if exceeding target
                    break
                    
                path.append(num)
                # Try same number again (i) since reuse is allowed
                backtrack(i, remaining - num, path)
                path.pop()  # Backtrack
        
        result: List[List[int]] = []
        backtrack(0, target, [])
        return result

# Comprehensive test cases
def test_combination_sum():
    solution = Solution()
    
    # Basic cases
    assert solution.combinationSum([2,3,6,7], 7) == [[2,2,3],[7]]
    assert solution.combinationSum([2,3,5], 8) == [[2,2,2,2],[2,3,3],[3,5]]
    
    # Edge cases
    assert solution.combinationSum([], 7) == []
    assert solution.combinationSum([5], 3) == []
    assert solution.combinationSum([1], 1) == [[1]]
    assert solution.combinationSum([1], 0) == []
\end{lstlisting}
\end{fullwidth}

\section*{Complexity Analysis}
\begin{itemize}
    \item \textbf{Time:} O($N^(T/M)$) where:
        \begin{itemize}
            \item N = length of candidates
            \item T = target value
            \item M = minimum value in candidates
        \end{itemize}
    \item \textbf{Space:} O(T/M) for recursion stack
\end{itemize}

\section*{Common Pitfalls}
\begin{itemize}
    \item \textbf{Missing Edge Cases:} Empty array, no solution possible
    \item \textbf{Incorrect Path Copying:} Using path instead of path[:]
    \item \textbf{Inefficient Pruning:} Not sorting candidates first
    \item \textbf{Memory Issues:} Not backtracking properly
\end{itemize}

\section*{Interview Tips}
\begin{itemize}
    \item Start by explaining the backtracking approach
    \item Mention optimization techniques (sorting, pruning)
    \item Discuss time/space complexity trade-offs
    \item Consider follow-up questions:
        \begin{itemize}
            \item What if negative numbers were allowed?
            \item What if each number could be used only once?
            \item How to handle duplicates in candidates?
        \end{itemize}
\end{itemize}

\section*{Similar Problems}
\begin{itemize}
    \item Combination Sum II (no reuse allowed)
    \item Combination Sum III (fixed combination size)
    \item Subset Sum
    \item Target Sum
\end{itemize}

% ... rest of the content ...