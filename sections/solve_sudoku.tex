% Filename: solve_sudoku.tex

\problemsection{Solve Sudoku}
\label{problem:Solve_Sudoku}

The **Solve Sudoku** problem involves completing a partially filled \(9 \times 9\) Sudoku grid by filling empty cells such that the final grid satisfies Sudoku rules. This problem is an excellent example of constraint satisfaction and backtracking.

\subsection*{Problem Statement}
Write a program to solve a given Sudoku puzzle by filling the empty cells. The Sudoku rules are:
\begin{itemize}
    \item Each row must contain the digits \(1\) to \(9\) with no repetitions.
    \item Each column must contain the digits \(1\) to \(9\) with no repetitions.
    \item Each of the nine \(3 \times 3\) sub-boxes must contain the digits \(1\) to \(9\) with no repetitions.
\end{itemize}

Empty cells are represented by \texttt{'.'}.

\textbf{Input:}
- A \(9 \times 9\) grid \texttt{board}, where each cell contains a digit (\(1\) to \(9\)) or \texttt{'.'}.

\textbf{Output:}
- The same grid, modified in place to represent the solved Sudoku.

\textbf{Example Input:}
\begin{verbatim}
board = [
    ["5","3",".",".","7",".",".",".","."],
    ["6",".",".","1","9","5",".",".","."],
    [".","9","8",".",".",".",".","6","."],
    ["8",".",".",".","6",".",".",".","3"],
    ["4",".",".","8",".","3",".",".","1"],
    ["7",".",".",".","2",".",".",".","6"],
    [".","6",".",".",".",".","2","8","."],
    [".",".",".","4","1","9",".",".","5"],
    [".",".",".",".","8",".",".","7","9"]
]
\end{verbatim}

\textbf{Output:}
The solved grid is:
\begin{verbatim}
board = [
    ["5","3","4","6","7","8","9","1","2"],
    ["6","7","2","1","9","5","3","4","8"],
    ["1","9","8","3","4","2","5","6","7"],
    ["8","5","9","7","6","1","4","2","3"],
    ["4","2","6","8","5","3","7","9","1"],
    ["7","1","3","9","2","4","8","5","6"],
    ["9","6","1","5","3","7","2","8","4"],
    ["2","8","7","4","1","9","6","3","5"],
    ["3","4","5","2","8","6","1","7","9"]
]
\end{verbatim}

\subsection*{Algorithmic Approach}
The problem is solved using backtracking:
\begin{itemize}
    \item Identify the next empty cell in the grid.
    \item Try placing digits \(1\) through \(9\) in the empty cell.
    \item For each digit, check if placing it maintains the validity of the Sudoku rules.
    \item If valid, recursively attempt to solve the remaining grid.
    \item Backtrack if placing a digit does not lead to a solution, removing it and trying the next digit.
    \item Stop when all cells are filled and the grid satisfies the Sudoku rules.
\end{itemize}

\subsection*{Complexities}
\begin{itemize}
    \item \textbf{Time Complexity:} Solving Sudoku is an NP-complete problem. In the worst case, the time complexity can approach \(O(9^{n})\), where \(n\) is the number of empty cells.
    \item \textbf{Space Complexity:} \(O(n)\) for the recursion stack, where \(n\) is the number of empty cells.
\end{itemize}

\subsection*{Python Implementation}
Below is the Python implementation using backtracking:

\begin{fullwidth}
\begin{lstlisting}[language=Python]
def solveSudoku(board):
    def is_valid(row, col, num):
        # Check the row
        for i in range(9):
            if board[row][i] == num:
                return False
        # Check the column
        for i in range(9):
            if board[i][col] == num:
                return False
        # Check the 3x3 sub-box
        box_row, box_col = 3 * (row // 3), 3 * (col // 3)
        for i in range(3):
            for j in range(3):
                if board[box_row + i][box_col + j] == num:
                    return False
        return True

    def backtrack():
        for i in range(9):
            for j in range(9):
                if board[i][j] == ".":
                    for num in "123456789":
                        if is_valid(i, j, num):
                            board[i][j] = num
                            if backtrack():
                                return True
                            board[i][j] = "."  # Backtrack
                    return False
        return True

    backtrack()

# Example usage:
board = [
    ["5","3",".",".","7",".",".",".","."],
    ["6",".",".","1","9","5",".",".","."],
    [".","9","8",".",".",".",".","6","."],
    ["8",".",".",".","6",".",".",".","3"],
    ["4",".",".","8",".","3",".",".","1"],
    ["7",".",".",".","2",".",".",".","6"],
    [".","6",".",".",".",".","2","8","."],
    [".",".",".","4","1","9",".",".","5"],
    [".",".",".",".","8",".",".","7","9"]
]

solveSudoku(board)
for row in board:
    print(row)
\end{lstlisting}
\end{fullwidth}

\subsection*{Why This Approach?}
Backtracking is the most natural approach for constraint satisfaction problems like Sudoku. It systematically explores all possibilities while pruning invalid paths, ensuring correctness and efficiency.

\subsection*{Similar Problems}
\begin{itemize}
    \item \textbf{N-Queens Problem:} Place \(n\) queens on an \(n \times n\) chessboard such that no two queens threaten each other.
    \item \textbf{Magic Square Problem:} Fill a grid such that the sums of rows, columns, and diagonals are equal.
\end{itemize}

\subsection*{Things to Keep in Mind and Tricks}
\begin{itemize}
    \item Use a helper function to check the validity of placing a digit, reducing redundancy in code.
    \item Test the solution with partially filled grids of varying difficulty to ensure correctness and performance.
    \item Optimize by filling cells with fewer possibilities first.
\end{itemize}

\subsection*{Corner and Special Cases to Test}
\begin{itemize}
    \item \textbf{Empty Grid:} Input: A grid with all cells empty. Output: A valid Sudoku grid.
    \item \textbf{Nearly Solved Grid:} Input: A grid with only one cell empty. Output: A completed grid.
    \item \textbf{Invalid Grid:} Input: A grid that violates Sudoku rules. Output: No solution.
\end{itemize}

\subsection*{Conclusion}
The **Solve Sudoku** problem showcases the power of backtracking for solving constraint satisfaction problems. Mastering this technique provides foundational skills for tackling a wide range of combinatorial challenges.