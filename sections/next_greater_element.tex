% Filename: next_greater_element.tex

\problemsection{Next Greater Element}\marginpar{This problem utilizes monotonic stacks to find the next greater element for each item in a list.}

\textbf{Problem Description:}

Given an array of integers \texttt{nums}, for each element in the array, find the next greater element. The Next Greater Element of a number \( x \) is the first greater number to its right in the array. If it does not exist, output \(-1\) for that number.

\textbf{Example 1:}

\begin{itemize}
    \item \textbf{Input:} \texttt{nums = [4,5,2,25]}
    \item \textbf{Output:} \texttt{[5,25,25,-1]}
\end{itemize}

\textbf{Example 2:}

\begin{itemize}
    \item \textbf{Input:} \texttt{nums = [13,7,6,12]}
    \item \textbf{Output:} \texttt{[-1,12,12,-1]}
\end{itemize}

\textbf{Solution Overview:}

Use a stack to keep track of indices whose next greater element has not been found yet. Iterate through the array:

\begin{itemize}
    \item For each element \( nums[i] \):
        \begin{itemize}
            \item While the stack is not empty and \( nums[i] > nums[stack[-1]] \):
                \begin{itemize}
                    \item Set the next greater element of \( nums[stack[-1]] \) to \( nums[i] \).
                    \item Pop the index from the stack.
                \end{itemize}
            \item Push the current index \( i \) onto the stack.
        \end{itemize}
\end{itemize}

At the end, set the next greater element of remaining indices in the stack to \(-1\).

\begin{lstlisting}[language=Python]
def nextGreaterElement(nums):
    result = [-1] * len(nums)
    stack = []
    for i in range(len(nums)):
        while stack and nums[i] > nums[stack[-1]]:
            idx = stack.pop()
            result[idx] = nums[i]
        stack.append(i)
    return result

# Example usage:
print(nextGreaterElement([4,5,2,25]))  # Output: [5,25,25,-1]
print(nextGreaterElement([13,7,6,12])) # Output: [-1,12,12,-1]
\end{lstlisting}