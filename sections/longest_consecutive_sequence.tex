% Filename: longest_consecutive_sequence.tex

\problemsection{Longest Consecutive Sequence}
\label{problem:longest_consecutive_sequence}
\marginnote{A classic array problem that demonstrates the power of hash-based data structures for achieving optimal time complexity.}

The \textbf{Longest Consecutive Sequence} problem requires finding the length of the longest sequence of consecutive numbers in an unsorted array. This must be achieved with a time complexity of \(O(n)\), which rules out the possibility of sorting the array as a pre-processing step.

\section*{Problem Statement}
Given an unsorted array of integers `nums`, return the length of the longest consecutive elements sequence. It is essential that the algorithm operates in \(O(n)\) time.

\textbf{Examples:}

\begin{itemize}
	\item \textbf{Example 1:}
	\begin{verbatim}
	Input: nums = [100,4,200,1,3,2]
	Output: 4
	Explanation: The longest consecutive elements sequence is [1, 2, 3, 4]. Therefore its length is 4.
	\end{verbatim}

	\item \textbf{Example 2:}
	\begin{verbatim}
	Input: nums = [0,3,7,2,5,8,4,6,0,1]
	Output: 9
	Explanation: The longest consecutive elements sequence is [0, 1, 2, 3, 4, 5, 6, 7, 8]. Therefore its length is 9.
	\end{verbatim}
\end{itemize}

LeetCode link: \href{https://leetcode.com/problems/longest-consecutive-sequence/}{Longest Consecutive Sequence}\index{LeetCode}

\marginnote{\href{https://leetcode.com/problems/longest-consecutive-sequence/}{[LeetCode Link]}\index{LeetCode}}
\marginnote{\href{https://www.geeksforgeeks.org/longest-consecutive-subsequence/}{[GeeksForGeeks Link]}\index{GeeksForGeeks}}
\marginnote{\href{https://www.hackerrank.com/challenges/longest-consecutive/problem}{[HackerRank Link]}\index{HackerRank}}
\marginnote{\href{https://app.codesignal.com/challenges/longest-consecutive-sequence}{[CodeSignal Link]}\index{CodeSignal}}
\marginnote{\href{https://www.interviewbit.com/problems/longest-consecutive-sequence/}{[InterviewBit Link]}\index{InterviewBit}}
\marginnote{\href{https://www.educative.io/courses/grokking-the-coding-interview/RM8y8Y3nLdY}{[Educative Link]}\index{Educative}}
\marginnote{\href{https://www.codewars.com/kata/longest-consecutive-sequence/train/python}{[Codewars Link]}\index{Codewars}}

\section*{Algorithmic Approach}
The optimal solution uses a \textbf{hash set} to achieve \(O(n)\) time complexity. The key insight is that we only need to check for sequence starts (numbers where num-1 doesn't exist in the set), eliminating redundant checks. For each sequence start, we can efficiently expand the sequence using the set's O(1) lookup.\marginnote{The hash set approach elegantly balances simplicity with optimal performance.}

\section*{Complexities}

\begin{itemize}
	\item \textbf{Time Complexity:} \(O(n)\), since each number in the array is processed once to insert into the set and at most once when finding a sequence.
	\item \textbf{Space Complexity:} \(O(n)\), as a set is used to store the elements of the array.
\end{itemize}

\newpage % Start Python Implementation on a new page
\section*{Python Implementation}
\marginnote{This implementation prioritizes readability while maintaining optimal performance.}

\begin{fullwidth}
\begin{lstlisting}[language=Python]
class Solution:
    def longestConsecutive(self, nums: List[int]) -> int:
        if not nums:  # Handle empty input
            return 0
            
        num_set = set(nums)
        max_length = 1
        
        for num in num_set:
            # Only check sequences from their starting point
            if num - 1 not in num_set:
                current_length = 1
                current = num
                
                # Expand sequence as far as possible
                while current + 1 in num_set:
                    current_length += 1
                    current += 1
                
                max_length = max(max_length, current_length)
        
        return max_length

# Test cases demonstrating various scenarios:
def test_longest_consecutive():
    solution = Solution()
    assert solution.longestConsecutive([]) == 0
    assert solution.longestConsecutive([1]) == 1
    assert solution.longestConsecutive([1,2,3,5,6,7]) == 3
    assert solution.longestConsecutive([-1,-2,-3,0,1]) == 5
\end{lstlisting}
\end{fullwidth}

This implementation uses a set to achieve constant-time look-up when checking for consecutive numbers. It starts a new count whenever a number is the beginning of a new sequence (i.e., the number just smaller is not in the set). The code efficiently avoids counting consecutive sequences multiple times by counting only from the beginning of such sequences.

\section*{Explanation}
The `longestConsecutive` function efficiently finds the longest consecutive sequence in an unsorted array by leveraging a set for constant-time look-ups. Here's a detailed breakdown of the implementation:

\begin{itemize}
	\item \textbf{Initialization:}
	\begin{itemize}
		\item \textbf{Set Creation:} Convert the input list `nums` into a set `num-set` to allow for O(1) look-up times.
		\item \textbf{Variable Setup:} Initialize `longest-streak` to keep track of the maximum sequence length found.
	\end{itemize}
	
	\item \textbf{Iteration:}
	\begin{itemize}
		\item \textbf{Sequence Start Check:} For each number `num` in `num-set`, check if `num - 1` is not present. If `num - 1` is absent, it means `num` could be the start of a new sequence.
		\item \textbf{Streak Counting:} Initialize `current-num` to `num` and `current-streak` to 1. Then, incrementally check if `current-num + 1` exists in `num-set`. If it does, continue to extend the streak.
		\item \textbf{Maximum Streak Update:} After exiting the while loop, update `longest-streak` if the current streak is longer than the previously recorded maximum.
	\end{itemize}
	
	\item \textbf{Result:}
	\begin{itemize}
		\item After iterating through all elements, return `longest-streak` as the length of the longest consecutive sequence.
	\end{itemize}
\end{itemize}

\section*{Common Pitfalls and Tips}
\begin{itemize}
    \item \textbf{Initialization:} Don't forget to handle empty input arrays explicitly.
    \item \textbf{Optimization:} Only start sequence checks from actual sequence beginnings.
    \item \textbf{Set vs List:} Using a list for lookups would degrade performance to O(n²).
    \item \textbf{Memory:} The set trades space for time - be aware of memory constraints with large inputs.
\end{itemize}

\section*{Alternative Approaches}
An alternative method might involve sorting the array first; however, as mentioned, this would violate the \(O(n)\) time complexity requirement. Another idea could be to use a **Union-Find** data structure to group consecutive elements, but that approach is generally more complex and doesn't offer any significant advantages over the set-based method in this context.

\begin{itemize}
	\item \textbf{Sorting Approach:}
	\begin{itemize}
		\item \textbf{Pros:} Intuitive and straightforward to implement.
		\item \textbf{Cons:} Higher time complexity of \(O(n \log n)\), which is not optimal for large datasets.
	\end{itemize}
	
	\item \textbf{Union-Find Approach:}
	\begin{itemize}
		\item \textbf{Pros:} Efficiently groups elements into sets.
		\item \textbf{Cons:} More complex to implement and doesn't improve upon the set-based approach's time complexity.
	\end{itemize}
\end{itemize}

\marginnote{Understanding multiple approaches enhances problem-solving flexibility and depth.}

\section*{Similar Problems to This One}
There are several other problems that involve finding sequences or patterns within arrays, such as:
\begin{itemize}
	\item \hyperref[problem:missing_ranges]{Missing Ranges}\index{Missing Ranges}
	\item \hyperref[problem:summary_ranges]{Summary Ranges}\index{Summary Ranges}
	\item \hyperref[problem:longest_substring_without_repeating_characters]{Longest Substring Without Repeating Characters}\index{Longest Substring Without Repeating Characters}
	\item \hyperref[problem:maximum_subarray]{Maximum Subarray}\index{Maximum Subarray}
\end{itemize}

\section*{Things to Keep in Mind and Tricks}
\begin{itemize}
	\item \textbf{Efficient Look-ups:} Utilizing a set for constant-time element presence checks is crucial for maintaining linear time complexity.
	\index{Efficient Look-ups}
	
	\item \textbf{Sequence Start Identification:} Only initiate sequence counts from numbers that are potential sequence starters (i.e., `num - 1` not in set).
	\index{Sequence Start Identification}
	
	\item \textbf{Avoid Redundant Operations:} By starting counts only at sequence starts, redundant checks and counts are minimized.
	\index{Avoid Redundant Operations}
	
	\item \textbf{Edge Case Handling:} Always consider edge cases such as empty arrays or arrays with all identical elements.
	\index{Edge Case Handling}
\end{itemize}

\section*{Corner and Special Cases to Test When Writing the Code}
When implementing the `longestConsecutive` function, it is crucial to test the following edge cases to ensure robustness:

\begin{itemize}
	\item \textbf{Empty Array:} `nums = []` should return `0`.
	\index{Corner Cases}
	
	\item \textbf{Single Element:} `nums = [1]` should return `1`.
	\index{Corner Cases}
	
	\item \textbf{All Elements the Same:} `nums = [2,2,2,2]` should return `1`.
	\index{Corner Cases}
	
	\item \textbf{Multiple Sequences of Same Length:} `nums = [1,2,3,10,11,12]` should return `3` as both sequences have the same length.
	\index{Corner Cases}
	
	\item \textbf{Negative Numbers:} `nums = [-1, -2, -3, 0, 1]` should return `5`.
	\index{Corner Cases}
	
	\item \textbf{Large Input Size:} Test with a very large array to ensure that the implementation performs efficiently without exceeding memory limits.
	\index{Corner Cases}
	
	\item \textbf{Non-Consecutive Numbers with Gaps:} `nums = [10, 5, 12, 3, 55, 30, 4, 11, 2]` should return `4` for the sequence `[2, 3, 4, 5]`.
	\index{Corner Cases}
\end{itemize}

\printindex