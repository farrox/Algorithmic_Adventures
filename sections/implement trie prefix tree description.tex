%filename: implement_trie_prefix_tree

\problemsection{Implement Trie (Prefix Tree)}
\label{problem:implement_trie_prefix_tree}
\marginnote{This problem utilizes a trie (prefix tree) data structure to efficiently store and retrieve strings, supporting operations like insertion, search, and prefix matching.}

The \textbf{Implement Trie (Prefix Tree)} problem involves designing a trie data structure that efficiently supports the insertion of words, searching for exact matches, and searching for words with specific prefixes. A trie is particularly effective for tasks that involve a large number of string operations, such as autocomplete systems and spell checkers.

\section*{Problem Statement}

Implement the \texttt{Trie} class:

\begin{itemize}
    \item \texttt{Trie()} Initializes the trie object.
    \item \texttt{void insert(String word)} Inserts the string \texttt{word} into the trie.
    \item \texttt{boolean search(String word)} Returns \texttt{true} if the string \texttt{word} is in the trie (i.e., was inserted before), and \texttt{false} otherwise.
    \item \texttt{boolean startsWith(String prefix)} Returns \texttt{true} if there is a previously inserted string \texttt{word} that has the prefix \texttt{prefix}, and \texttt{false} otherwise.
\end{itemize}

\textbf{Example:}

\textit{Example 1:}

\begin{verbatim}
Input:
["Trie","insert","search","search","startsWith","insert","search"]
[[],["apple"],["apple"],["app"],["app"],["app"],["app"]]

Output:
[null,null,true,false,true,null,true]

Explanation:
Trie trie = new Trie();
trie.insert("apple");
trie.search("apple");   // returns true
trie.search("app");     // returns false
trie.startsWith("app"); // returns true
trie.insert("app");   
trie.search("app");     // returns true
\end{verbatim}


\marginnote{\href{https://leetcode.com/problems/implement-trie-prefix-tree/}{[LeetCode Link]}\index{LeetCode}}
\marginnote{\href{https://www.geeksforgeeks.org/trie-insert-and-search/}{[GeeksForGeeks Link]}\index{GeeksForGeeks}}
\marginnote{\href{https://www.interviewbit.com/problems/implement-trie-prefix-tree/}{[InterviewBit Link]}\index{InterviewBit}}
\marginnote{\href{https://app.codesignal.com/challenges/implement-trie-prefix-tree}{[CodeSignal Link]}\index{CodeSignal}}
\marginnote{\href{https://www.codewars.com/kata/implement-trie-prefix-tree/train/python}{[Codewars Link]}\index{Codewars}}

\section*{Algorithmic Approach}

To efficiently support the operations required by the problem, a \textbf{trie (prefix tree)} data structure is an optimal choice. A trie is a tree-like structure where each node represents a character of a string, allowing for efficient storage and retrieval of words, especially when dealing with prefixes.

\begin{enumerate}
    \item \textbf{Insertion (\texttt{insert(word)}):}  
    Traverse the trie character by character, creating new nodes as necessary. After inserting the last character, mark the node as the end of a word.
    
    \item \textbf{Search (\texttt{search(word)}):}  
    Traverse the trie according to the characters of the word. If all characters are found and the last character node is marked as the end of a word, return \texttt{true}; otherwise, return \texttt{false}.
    
    \item \textbf{Prefix Search (\texttt{startsWith(prefix)}):}  
    Similar to search, but only verify that the prefix exists in the trie without needing the last node to be marked as the end of a word.
\end{enumerate}

\marginnote{Utilizing a trie allows for efficient insertion and search operations, with time complexities proportional to the length of the words, making it suitable for large datasets.}

\section*{Complexities}

\begin{itemize}
    \item \textbf{Time Complexity:}
    \begin{itemize}
        \item \texttt{insert}: \(O(m)\), where \(m\) is the length of the word being inserted, as each character is processed once.
        \item \texttt{search}: \(O(m)\), where \(m\) is the length of the word being searched.
        \item \texttt{startsWith}: \(O(m)\), where \(m\) is the length of the prefix being searched.
    \end{itemize}
    \item \textbf{Space Complexity:} \(O(n \cdot m)\), where \(n\) is the number of words inserted and \(m\) is the average length of the words. This accounts for all the nodes in the trie.
\end{itemize}

\section*{Python Implementation}

\marginnote{Implementing a trie involves creating nodes for each character and managing flags to indicate the end of words. Traversal methods enable efficient insertion and search operations.}

Below is the complete Python code that implements the \texttt{Trie} class to solve the problem:

\begin{fullwidth}
\begin{lstlisting}[language=Python]
class TrieNode:
    def __init__(self):
        self.children = {}
        self.is_end_of_word = False

class Trie:

    def __init__(self):
        self.root = TrieNode()

    def insert(self, word: str) -> None:
        node = self.root
        for char in word:
            if char not in node.children:
                node.children[char] = TrieNode()
            node = node.children[char]
        node.is_end_of_word = True

    def search(self, word: str) -> bool:
        node = self.root
        for char in word:
            if char not in node.children:
                return False
            node = node.children[char]
        return node.is_end_of_word

    def startsWith(self, prefix: str) -> bool:
        node = self.root
        for char in prefix:
            if char not in node.children:
                return False
            node = node.children[char]
        return True

# Example usage:
trie = Trie()
trie.insert("apple")
print(trie.search("apple"))   # Output: True
print(trie.search("app"))     # Output: False
print(trie.startsWith("app")) # Output: True
trie.insert("app")
print(trie.search("app"))     # Output: True
\end{lstlisting}
\end{fullwidth}

\section*{Explanation}

The \texttt{Trie} class is designed to efficiently handle the insertion and searching of words, including prefix-based searches. Here's a detailed breakdown of the implementation:

\subsection*{Data Structures}
\begin{itemize}
    \item \texttt{TrieNode}:  
    Each node in the trie contains:
    \begin{itemize}
        \item \texttt{children}: A dictionary mapping characters to their corresponding child nodes.
        \item \texttt{is\_end\_of\_word}: A boolean flag indicating whether the node represents the end of a word.
    \end{itemize}
    
    \item \texttt{Trie}:  
    Contains the root of the trie and methods to insert and search words.
\end{itemize}

\subsection*{Insertion (\texttt{insert(word)})}
\begin{enumerate}
    \item Start at the root node.
    \item Iterate through each character in the word:
    \begin{itemize}
        \item If the character does not exist in the current node's children, create a new \texttt{TrieNode} for it.
        \item Move to the child node corresponding to the character.
    \end{itemize}
    \item After processing all characters, mark the final node's \texttt{is\_end\_of\_word} as \texttt{True} to indicate the completion of a word.
\end{enumerate}

\subsection*{Search (\texttt{search(word)})}
\begin{enumerate}
    \item Start at the root node.
    \item Iterate through each character in the word:
    \begin{itemize}
        \item If the character is not in the current node's children, return \texttt{False}.
        \item Move to the child node corresponding to the character.
    \end{itemize}
    \item After processing all characters, return the value of \texttt{is\_end\_of\_word} to determine if the word exists in the trie.
\end{enumerate}

\subsection*{Prefix Search (\texttt{startsWith(prefix)})}
\begin{enumerate}
    \item Start at the root node.
    \item Iterate through each character in the prefix:
    \begin{itemize}
        \item If the character is not in the current node's children, return \texttt{False}.
        \item Move to the child node corresponding to the character.
    \end{itemize}
    \item After processing all characters, return \texttt{True} to indicate that the prefix exists in the trie.
\end{enumerate}

This structured approach ensures that both insertion and search operations are performed efficiently, with time complexities proportional to the length of the words or prefixes involved.

\section*{Why This Approach}

Using a trie data structure offers several advantages for this problem:

\begin{itemize}
    \item \textbf{Efficiency in Prefix Handling:}  
    Tries are optimized for operations involving prefixes, making them ideal for word storage and retrieval.
    
    \item \textbf{Flexible Search Capability:}  
    The trie structure allows for efficient implementation of both exact word searches and prefix-based searches without redundant storage.
    
    \item \textbf{Space Optimization:}  
    By sharing common prefixes among words, tries can be more space-efficient compared to storing each word independently, especially when many words share similar prefixes.
    
    \item \textbf{Scalability:}  
    The trie structure can handle a large number of words and operations within the problem constraints (\(10^5\) operations), maintaining performance even as the dataset grows.
\end{itemize}

This method is optimal for the problem constraints, ensuring both speed and accuracy in word insertion and search operations.

\section*{Alternative Approaches}

While a trie is the most efficient structure for this problem, alternative methods include:

\begin{itemize}
    \item \textbf{Hash Table with Regex Matching:}  
    Store all words in a hash table and perform regex-based searches to handle wildcards. However, this approach is less efficient, especially for large datasets, as regex matching can be time-consuming.
    
    \item \textbf{Backtracking with Lists:}  
    Maintain a list of all added words and perform backtracking searches for wildcard patterns. This method lacks the efficiency of a trie, leading to slower search times, particularly with many words or multiple wildcards.
    
    \item \textbf{Automaton-Based Methods:}  
    Implement finite automata to handle wildcard searches. While theoretically efficient, automaton-based implementations are more complex and harder to manage compared to tries.
\end{itemize}

These alternatives generally have higher time and space complexities or are more complex to implement, making them less suitable compared to the trie-based approach.

\section*{Similar Problems to This One}

Similar problems that involve dynamic word storage and flexible search capabilities include:

\begin{itemize}
    \item \textbf{Design Add and Search Words Data Structure:}  
    Design a data structure that supports adding new words and searching for words that match with wildcards.
    \index{Design Add and Search Words Data Structure}
    
    \item \textbf{Implement Magic Dictionary:}  
    Design a data structure that supports adding new words and searching for words that match with exactly one character difference.
    \index{Implement Magic Dictionary}
    
    \item \textbf{Word Search II:}  
    Given a grid of letters and a list of words, find all words that can be formed in the grid by sequentially adjacent cells. Implementing a trie can optimize the search process.
    \index{Word Search II}
    
    \item \textbf{Replace Words:}  
    Replace words in a sentence with the shortest root from a dictionary. Tries can efficiently find the appropriate root for each word.
    \index{Replace Words}
    
    \item \textbf{Longest Word in Dictionary:}  
    Find the longest word in a dictionary that can be built one character at a time by other words in the dictionary. A trie facilitates checking the existence of prefixes.
    \index{Longest Word in Dictionary}
\end{itemize}

These problems share the common theme of managing and searching words dynamically, often requiring efficient data structures like tries to handle large datasets and complex search patterns.

\section*{Things to Keep in Mind and Tricks}

When implementing the trie-based \texttt{Trie} class, consider the following tips and best practices:

\begin{itemize}
    \item \textbf{Efficient Node Representation:}  
    Use dictionaries or fixed-size arrays (for the 26 lowercase letters) to store child nodes, balancing between speed and memory usage.
    \index{Efficient Node Representation}
    
    \item \textbf{Recursive vs. Iterative Search:}  
    Implementing the search recursively simplifies handling complex patterns, but be cautious of recursion depth limits with very long words.
    \index{Recursive Search}
    
    \item \textbf{Handling Wildcards Efficiently:}  
    Optimize the recursive search by pruning paths early when mismatches are detected, reducing unnecessary computations.
    \index{Handling Wildcards}
    
    \item \textbf{Memory Management:}  
    To conserve memory, consider sharing common prefixes and avoiding redundant nodes.
    \index{Memory Management}
    
    \item \textbf{Edge Case Handling:}  
    Ensure the implementation gracefully handles edge cases such as empty strings, single-character words, and searches before any words are added.
    \index{Edge Case Handling}
    
    \item \textbf{Optimizing for Large Datasets:}  
    Given the constraint of up to \(10^5\) operations, ensure that both addition and search operations are optimized to prevent timeouts.
    \index{Optimizing for Large Datasets}
    
    \item \textbf{Code Readability and Maintenance:}  
    Write clean, well-documented code with meaningful variable names and comments to facilitate maintenance and future enhancements.
    \index{Code Readability}
\end{itemize}

\section*{Corner and Special Cases to Test When Writing the Code}

When implementing and testing the \texttt{Trie} class, ensure to cover the following corner and special cases:

\begin{itemize}
    \item \textbf{Empty String:}  
    Test inserting and searching for an empty string to verify proper handling.
    \index{Corner Cases}
    
    \item \textbf{Single Character Words:}  
    Add and search for single-character words to ensure basic functionality.
    \index{Corner Cases}
    
    \item \textbf{Duplicate Words:}  
    Insert the same word multiple times and ensure that searches return \texttt{True} without issues.
    \index{Corner Cases}
    
    \item \textbf{Non-Existent Words:}  
    Search for words that were never inserted to confirm that the search correctly returns \texttt{False}.
    \index{Corner Cases}
    
    \item \textbf{Prefixes That Are Words:}  
    Insert words that are prefixes of other words (e.g., "app" and "apple") and ensure that searches for both return \texttt{True}.
    \index{Corner Cases}
    
    \item \textbf{Very Long Words:}  
    Insert and search for words with maximum allowed length to test performance and handle potential recursion limits.
    \index{Corner Cases}
    
    \item \textbf{All Possible Characters:}  
    Insert and search for words containing the first and last letters of the alphabet (`'a'` and `'z'`) to ensure all characters are handled correctly.
    \index{Corner Cases}
    
    \item \textbf{No Words Added:}  
    Perform search operations before any words have been inserted to verify that searches return \texttt{False}.
    \index{Corner Cases}
    
    \item \textbf{Prefixes That Don't Exist:}  
    Search for prefixes that do not match any inserted words to ensure correct \texttt{False} responses.
    \index{Corner Cases}
    
    \item \textbf{Case Sensitivity:}  
    Insert and search for words with varying cases (if applicable) to ensure that the trie handles case sensitivity as intended.
    \index{Corner Cases}
\end{itemize}

\section*{Implementation Considerations}

When implementing the \texttt{Trie} class, keep in mind the following considerations to ensure robustness and efficiency:

\begin{itemize}
    \item \textbf{Exception Handling:}  
    Implement proper exception handling to manage unexpected inputs or states, such as null inputs or invalid characters.
    \index{Exception Handling}
    
    \item \textbf{Performance Optimization:}  
    Optimize the search function to minimize unnecessary traversals, especially when dealing with long words or large datasets.
    \index{Performance Optimization}
    
    \item \textbf{Memory Efficiency:}  
    Use memory-efficient data structures for trie nodes to handle large numbers of words without excessive memory consumption.
    \index{Memory Efficiency}
    
    \item \textbf{Thread Safety:}  
    If the data structure is to be used in a multithreaded environment, ensure that insert and search operations are thread-safe to prevent data races.
    \index{Thread Safety}
    
    \item \textbf{Scalability:}  
    Design the trie to handle up to \(10^5\) operations efficiently, considering both time and space constraints.
    \index{Scalability}
    
    \item \textbf{Testing and Validation:}  
    Rigorously test the implementation with various test cases, including all corner cases, to ensure correctness and reliability.
    \index{Testing and Validation}
    
    \item \textbf{Code Readability and Maintenance:}  
    Write clean, well-documented code with meaningful variable names and comments to facilitate maintenance and future enhancements.
    \index{Code Readability}
\end{itemize}

\section*{Conclusion}

The trie-based implementation of the \texttt{Trie} class provides an efficient and scalable solution for storing and retrieving words, including operations like insertion, exact search, and prefix-based search. By leveraging the structured nature of tries, the data structure ensures quick insertion and search capabilities, making it well-suited for handling large datasets with up to \(10^5\) operations. This approach balances time and space complexities effectively, offering a robust solution compared to alternative methods like hash tables or brute-force searches.

\printindex