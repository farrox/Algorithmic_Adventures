% Filename: invert_binary_tree.tex

\problemsection{Invert Binary Tree}\marginpar{Invert a binary tree using recursion or iteration.}

\textbf{Problem Statement}

This problem involves inverting a binary tree. Inverting a binary tree means swapping the left and right children of every node in the tree. The goal is to transform the original tree into its mirror image.

\textbf{Algorithmic Approach}

To solve this problem, you can use either a recursive approach or an iterative approach using a queue. Below, both methods are described:

\begin{enumerate}
    \item \textbf{Recursive Approach (Depth-First Search)}:
    \begin{itemize}
        \item \textbf{Base Case}: If the current node is \texttt{NULL}, return \texttt{NULL}.
        \item \textbf{Recursive Case}: 
        \begin{enumerate}
            \item Recursively invert the left subtree.
            \item Recursively invert the right subtree.
            \item Swap the left and right subtrees.
            \item Return the current node.
        \end{enumerate}
    \end{itemize}
    
    \item \textbf{Iterative Approach (Breadth-First Search)}:
    \begin{itemize}
        \item Use a queue to perform level-order traversal.
        \item Initialize the queue with the root node.
        \item While the queue is not empty:
        \begin{enumerate}
            \item Dequeue the current node.
            \item Swap its left and right children.
            \item Enqueue the non-\texttt{NULL} left and right children.
        \end{enumerate}
    \end{itemize}
\end{enumerate}

\textbf{Complexities}

\begin{itemize}
    \item \textbf{Time Complexity}: The time complexity is \(O(n)\), where \(n\) is the number of nodes in the tree. Each node is visited exactly once.
    \item \textbf{Space Complexity}: 
    \begin{itemize}
        \item \textbf{Recursive Approach}: \(O(h)\), where \(h\) is the height of the tree, due to the call stack.
        \item \textbf{Iterative Approach}: \(O(n)\), in the worst case, when the tree is completely unbalanced and all nodes are stored in the queue.
    \end{itemize}
\end{itemize}

\textbf{Python Implementation}\marginpar{Implementing both recursive and iterative solutions for flexibility.}

\begin{lstlisting}[language=Python, xleftmargin=0.02\textwidth, xrightmargin=0.02\textwidth]
# Definition for a binary tree node.
class TreeNode:
    def __init__(self, val=0, left=None, right=None):
        self.val = val
        self.left = left
        self.right = right

# Recursive Approach
def invertTreeRecursive(root: TreeNode) -> TreeNode:
    if root is None:
        return None
    # Recursively invert the left and right subtrees
    left_inverted = invertTreeRecursive(root.left)
    right_inverted = invertTreeRecursive(root.right)
    # Swap the left and right subtrees
    root.left, root.right = right_inverted, left_inverted
    return root

# Iterative Approach
from collections import deque

def invertTreeIterative(root: TreeNode) -> TreeNode:
    if root is None:
        return None
    queue = deque([root])
    while queue:
        current = queue.popleft()
        # Swap the left and right children
        current.left, current.right = current.right, current.left
        # Enqueue non-NULL children
        if current.left:
            queue.append(current.left)
        if current.right:
            queue.append(current.right)
    return root

# Example usage:
# Constructing the following binary tree
#       4
#     /   \
#    2     7
#   / \   / \
#  1   3 6   9

root = TreeNode(4)
root.left = TreeNode(2, TreeNode(1), TreeNode(3))
root.right = TreeNode(7, TreeNode(6), TreeNode(9))

inverted_recursive = invertTreeRecursive(root)
inverted_iterative = invertTreeIterative(root)

# The inverted tree should be:
#       4
#     /   \
#    7     2
#   / \   / \
#  9   6 3   1
\end{lstlisting}

\textbf{Explanation}

The function \texttt{invertTree} transforms a binary tree into its mirror image by swapping the left and right children of every node. 

\begin{itemize}
    \item \textbf{Recursive Approach}: 
    \begin{itemize}
        \item The function recursively inverts the left and right subtrees.
        \item After inverting the subtrees, it swaps the left and right children of the current node.
        \item This process continues until all nodes have been processed, resulting in an inverted tree.
    \end{itemize}
    
    \item \textbf{Iterative Approach}:
    \begin{itemize}
        \item The function uses a queue to perform a level-order traversal of the tree.
        \item For each node dequeued, it swaps its left and right children.
        \item Non-\texttt{NULL} children are enqueued for subsequent processing.
        \item This continues until all nodes have been visited and their children swapped.
    \end{itemize}
\end{itemize}

\textbf{Why This Approach}

\begin{itemize}
    \item \textbf{Recursive Approach}: Aligns naturally with the hierarchical structure of trees, making the implementation straightforward and intuitive. It leverages the call stack to manage traversal.
    \item \textbf{Iterative Approach}: Provides an alternative to recursion, which can be beneficial in environments with limited stack space or when dealing with very deep trees. It uses a queue to manage traversal explicitly.
\end{itemize}

\textbf{Alternative Approaches}

An alternative method involves using Depth-First Search (DFS) iteratively with a stack to simulate the recursive calls. This approach can help avoid potential stack overflow issues in languages with limited recursion depth. However, the recursive and breadth-first iterative approaches are generally more intuitive and easier to implement for this problem.

\textbf{Similar Problems to This One}

Similar tree manipulation problems include inverting a binary tree (\hyperref[problem:invert_binary_tree]{Invert Binary Tree}), checking if a tree is symmetric (\hyperref[problem:symmetric_tree]{Symmetric Tree}), and merging two binary trees (\hyperref[problem:merge_two_binary_trees]{Merge Two Binary Trees}).

\textbf{Things to Keep in Mind and Tricks}

\begin{itemize}
    \item \textbf{Base Cases Are Crucial}: Always handle cases where the tree or subtree is empty to prevent unnecessary computations and potential errors.
    \item \textbf{Traversal Choice}: Decide between recursive and iterative approaches based on the specific requirements and constraints of the problem.
    \item \textbf{Space Optimization}: Be mindful of the space complexity, especially when dealing with very deep or very wide trees.
    \item \textbf{Understanding Tree Properties}: A solid grasp of tree properties and traversal methods enhances problem-solving efficiency.
\end{itemize}

\textbf{Corner and Special Cases to Test When Writing the Code}

\begin{itemize}
    \item \textbf{Empty Tree}: A tree with no nodes should remain \texttt{NULL} after inversion.
    \item \textbf{Single Node}: A tree with only one node should remain unchanged after inversion.
    \item \textbf{Unbalanced Tree}: Trees that are skewed to the left or right should correctly invert their structure.
    \item \textbf{Balanced Tree}: Ensure that balanced trees are inverted correctly, maintaining their balanced property.
    \item \textbf{Large Tree}: Testing with a large number of nodes ensures that the implementation handles depth and breadth efficiently without performance degradation.
\end{itemize}