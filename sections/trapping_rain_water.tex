% Filename: trapping_rain_water.tex

\problemsection{Trapping Rain Water}\marginpar{This problem calculates the total trapped rainwater using a stack-based approach.}

\textbf{Problem Description:}

Given an array \texttt{height} representing an elevation map where the width of each bar is 1, compute how much water it can trap after raining.

\textbf{Example 1:}

\begin{itemize}
    \item \textbf{Input:} \texttt{height = [0,1,0,2,1,0,1,3,2,1,2,1]}
    \item \textbf{Output:} \texttt{6}
\end{itemize}

\textbf{Example 2:}

\begin{itemize}
    \item \textbf{Input:} \texttt{height = [4,2,0,3,2,5]}
    \item \textbf{Output:} \texttt{9}
\end{itemize}

\textbf{Solution Overview:}

Use a stack to keep track of the bars that are bounded by taller bars and hence can trap water. Iterate through the elevation map:

\begin{itemize}
    \item For each bar at index \( i \):
        \begin{itemize}
            \item While the stack is not empty and \( height[i] > height[stack[-1]] \):
                \begin{itemize}
                    \item Pop the top of the stack as \texttt{bottom}.
                    \item If the stack becomes empty, break.
                    \item Calculate the distance between the current index and the new top of the stack minus one.
                    \item Find the bounded height by taking the minimum of \( height[i] \) and \( height[stack[-1]] \) minus \( height[bottom] \).
                    \item Add the trapped water to the total.
                \end{itemize}
            \item Push \( i \) onto the stack.
        \end{itemize}
\end{itemize}

\textbf{Code Implementation:}

\begin{lstlisting}[language=Python]
def trap(height):
    stack = []
    water = 0
    i = 0
    while i < len(height):
        while stack and height[i] > height[stack[-1]]:
            bottom = stack.pop()
            if not stack:
                break
            distance = i - stack[-1] - 1
            bounded_height = min(height[i], height[stack[-1]]) - height[bottom]
            water += distance * bounded_height
        stack.append(i)
        i += 1
    return water

# Example usage:
print(trap([0,1,0,2,1,0,1,3,2,1,2,1]))  # Output: 6
print(trap([4,2,0,3,2,5]))              # Output: 9
\end{lstlisting}