% Filename: balanced_binary_tree.tex

\problemsection{Balanced Binary Tree}\marginpar{Determine if a binary tree is height-balanced using recursion or iteration.}

\textbf{Problem Statement}

This problem involves determining whether a binary tree is height-balanced. A binary tree is considered \textit{height-balanced} if for every node in the tree, the difference in height between its left and right subtrees is at most one.

\textbf{Algorithmic Approach}

To solve this problem, a recursive approach is commonly employed. The algorithm proceeds as follows:

\begin{enumerate}
    \item \textbf{Base Case}: If the current node is \texttt{NULL}, it is balanced by definition; return 0 as its height.
    \item \textbf{Recursive Case}: 
    \begin{enumerate}
        \item Recursively determine the height of the left subtree.
        \item If the left subtree is unbalanced (indicated by a negative value), propagate the failure by returning -1.
        \item Recursively determine the height of the right subtree.
        \item If the right subtree is unbalanced, propagate the failure by returning -1.
        \item Check the height difference between the left and right subtrees. If it exceeds one, the tree is unbalanced; return -1.
        \item If balanced, return the height of the current node, which is the greater height between the left and right subtrees plus one.
    \end{enumerate}
\end{enumerate}

\textbf{Complexities}

\begin{itemize}
    \item \textbf{Time Complexity}: The time complexity is \(O(n)\), where \(n\) is the number of nodes in the tree. Each node is visited exactly once.
    \item \textbf{Space Complexity}: The space complexity is \(O(h)\), where \(h\) is the height of the tree. This space is utilized by the call stack during recursive calls. In the worst case, the tree can be completely unbalanced, resulting in a space complexity of \(O(n)\).
\end{itemize}

\textbf{Python Implementation}\marginpar{Implementing both optimized recursive and iterative solutions for flexibility.}

\begin{lstlisting}[language=Python, xleftmargin=0.02\textwidth, xrightmargin=0.02\textwidth]
# Definition for a binary tree node.
class TreeNode:
    def __init__(self, val=0, left=None, right=None):
        self.val = val
        self.left = left
        self.right = right

# Optimized Recursive Approach
def isBalanced(root: TreeNode) -> bool:
    def check_height(node: TreeNode) -> int:
        if not node:
            return 0
        left_height = check_height(node.left)
        if left_height == -1:
            return -1
        right_height = check_height(node.right)
        if right_height == -1:
            return -1
        if abs(left_height - right_height) > 1:
            return -1
        return max(left_height, right_height) + 1
    
    return check_height(root) != -1

# Iterative Approach Using BFS
from collections import deque

def isBalancedIterative(root: TreeNode) -> bool:
    if not root:
        return True
    queue = deque([(root, 1)])
    while queue:
        node, depth = queue.popleft()
        left = node.left
        right = node.right
        if left:
            queue.append((left, depth + 1))
        if right:
            queue.append((right, depth + 1))
    # After traversal, compute heights and check balance
    def compute_height(node: TreeNode) -> int:
        if not node:
            return 0
        left = compute_height(node.left)
        if left == -1:
            return -1
        right = compute_height(node.right)
        if right == -1:
            return -1
        if abs(left - right) > 1:
            return -1
        return max(left, right) + 1
    
    return compute_height(root) != -1

# Example usage:
# Constructing the following binary tree
#       3
#      / \
#     9  20
#        / \
#       15  7
root = TreeNode(3)
root.left = TreeNode(9)
root.right = TreeNode(20, TreeNode(15), TreeNode(7))

print(isBalanced(root))            # Output: True
print(isBalancedIterative(root))   # Output: True
\end{lstlisting}

\textbf{Explanation}

The function \texttt{isBalanced} determines whether a binary tree is height-balanced by recursively calculating the height of each subtree. The helper function \texttt{check-height} returns the height of a node if its subtree is balanced or -1 if it is not. By propagating the failure state (-1) up the recursive calls, the algorithm efficiently identifies unbalanced subtrees without unnecessary computations.

The iterative approach uses Breadth-First Search (BFS) to traverse the tree level by level. After traversal, it computes the heights of subtrees and checks for balance, ensuring that the tree adheres to the height-balanced property.

\textbf{Why This Approach}

\begin{itemize}
    \item \textbf{Optimized Recursive Approach}: By immediately propagating the failure state (-1) when an unbalanced subtree is detected, the algorithm avoids redundant calculations, leading to improved efficiency.
    \item \textbf{Iterative Approach}: Provides an alternative to recursion, which can be beneficial in environments with limited stack space or when dealing with very deep trees.
\end{itemize}

\textbf{Alternative Approaches}

An alternative method involves using Depth-First Search (DFS) iteratively with a stack to simulate the recursive calls. This approach can help avoid potential stack overflow issues in languages with limited recursion depth. However, the recursive approach remains more intuitive and straightforward for this problem.

\textbf{Similar Problems to This One}

Similar tree-related problems include finding the minimum depth of a binary tree (\hyperref[problem:minimum_depth_of_binary_tree]{Minimum Depth of Binary Tree}), determining if a tree is symmetric (\hyperref[problem:symmetric_tree]{Symmetric Tree}), and calculating the diameter of a binary tree (\hyperref[problem:diameter_of_binary_tree]{Diameter of Binary Tree}).

\textbf{Things to Keep in Mind and Tricks}

\begin{itemize}
    \item \textbf{Base Cases Are Crucial}: Always handle cases where the tree or subtree is empty to prevent unnecessary computations and potential errors.
    \item \textbf{Optimizing Recursive Calls}: Propagating failure states can significantly reduce the number of recursive calls, enhancing performance.
    \item \textbf{Choosing the Right Traversal}: Decide between depth-first and breadth-first approaches based on the specific requirements and constraints of the problem.
    \item \textbf{Understanding Tree Properties}: A solid grasp of tree properties and traversal methods enhances problem-solving efficiency.
\end{itemize}

\textbf{Corner and Special Cases to Test When Writing the Code}

\begin{itemize}
    \item \textbf{Empty Tree}: A tree with no nodes should be considered balanced.
    \item \textbf{Single Node}: A tree with only one node is balanced.
    \item \textbf{Unbalanced Tree}: Trees that are skewed to the left or right should correctly identify as unbalanced.
    \item \textbf{Balanced Tree with Varying Subtree Heights}: Trees where the height difference between left and right subtrees is exactly one at some nodes.
    \item \textbf{Large Tree}: Testing with a large number of nodes ensures that the implementation handles depth and recursion efficiently.
\end{itemize}