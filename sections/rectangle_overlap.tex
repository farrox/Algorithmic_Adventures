% filename: rectangle_overlap.tex

\problemsection{Rectangle Overlap}
\label{chap:Rectangle_Overlap}
\marginnote{\href{https://leetcode.com/problems/rectangle-overlap/}{[LeetCode Link]}\index{LeetCode}}
\marginnote{\href{https://www.geeksforgeeks.org/check-if-two-rectangles-overlap/}{[GeeksForGeeks Link]}\index{GeeksForGeeks}}
\marginnote{\href{https://www.interviewbit.com/problems/rectangle-overlap/}{[InterviewBit Link]}\index{InterviewBit}}
\marginnote{\href{https://app.codesignal.com/challenges/rectangle-overlap}{[CodeSignal Link]}\index{CodeSignal}}
\marginnote{\href{https://www.codewars.com/kata/rectangle-overlap/train/python}{[Codewars Link]}\index{Codewars}}

The \textbf{Rectangle Overlap} problem is a fundamental challenge in Computational Geometry that involves determining whether two axis-aligned rectangles overlap. This problem tests one's ability to understand geometric properties, implement conditional logic, and optimize for efficient computation. Mastery of this problem is essential for applications in computer graphics, collision detection, and spatial data analysis.

\section*{Problem Statement}

Given two axis-aligned rectangles in a 2D plane, determine if they overlap. Each rectangle is defined by its bottom-left and top-right coordinates.

A rectangle is represented as a list of four integers \([x1, y1, x2, y2]\), where \((x1, y1)\) are the coordinates of the bottom-left corner, and \((x2, y2)\) are the coordinates of the top-right corner.

\textbf{Function signature in Python:}
\begin{lstlisting}[language=Python]
def isRectangleOverlap(rec1: List[int], rec2: List[int]) -> bool:
\end{lstlisting}

\section*{Examples}

\textbf{Example 1:}

\begin{verbatim}
Input: rec1 = [0,0,2,2], rec2 = [1,1,3,3]
Output: True
Explanation: The rectangles overlap in the area defined by [1,1,2,2].
\end{verbatim}

\textbf{Example 2:}

\begin{verbatim}
Input: rec1 = [0,0,1,1], rec2 = [1,0,2,1]
Output: False
Explanation: The rectangles touch at the edge but do not overlap.
\end{verbatim}

\textbf{Example 3:}

\begin{verbatim}
Input: rec1 = [0,0,1,1], rec2 = [2,2,3,3]
Output: False
Explanation: The rectangles are completely separate.
\end{verbatim}

\textbf{Example 4:}

\begin{verbatim}
Input: rec1 = [0,0,5,5], rec2 = [3,3,7,7]
Output: True
Explanation: The rectangles overlap in the area defined by [3,3,5,5].
\end{verbatim}

\textbf{Example 5:}

\begin{verbatim}
Input: rec1 = [0,0,0,0], rec2 = [0,0,0,0]
Output: False
Explanation: Both rectangles are degenerate points.
\end{verbatim}

\textbf{Constraints:}

\begin{itemize}
    \item All coordinates are integers in the range \([-10^9, 10^9]\).
    \item For each rectangle, \(x1 < x2\) and \(y1 < y2\).
\end{itemize}

LeetCode link: \href{https://leetcode.com/problems/rectangle-overlap/}{Rectangle Overlap}\index{LeetCode}

\section*{Algorithmic Approach}

To determine whether two axis-aligned rectangles overlap, we can use the following logical conditions:

1. **Non-Overlap Conditions:**
   - One rectangle is to the left of the other.
   - One rectangle is above the other.

2. **Overlap Condition:**
   - If neither of the non-overlap conditions is true, the rectangles must overlap.

\subsection*{Steps:}

1. **Extract Coordinates:**
   - For both rectangles, extract the bottom-left and top-right coordinates.

2. **Check Non-Overlap Conditions:**
   - If the right side of the first rectangle is less than or equal to the left side of the second rectangle, they do not overlap.
   - If the left side of the first rectangle is greater than or equal to the right side of the second rectangle, they do not overlap.
   - If the top side of the first rectangle is less than or equal to the bottom side of the second rectangle, they do not overlap.
   - If the bottom side of the first rectangle is greater than or equal to the top side of the second rectangle, they do not overlap.

3. **Determine Overlap:**
   - If none of the non-overlap conditions are met, the rectangles overlap.

\marginnote{This approach provides an efficient \(O(1)\) time complexity solution by leveraging simple geometric comparisons.}

\section*{Complexities}

\begin{itemize}
    \item \textbf{Time Complexity:} \(O(1)\). The algorithm performs a constant number of comparisons regardless of input size.
    
    \item \textbf{Space Complexity:} \(O(1)\). Only a fixed amount of extra space is used for variables.
\end{itemize}

\section*{Python Implementation}

\marginnote{Implementing the overlap check using coordinate comparisons ensures an optimal and straightforward solution.}

Below is the complete Python code implementing the \texttt{isRectangleOverlap} function:

\begin{fullwidth}
\begin{lstlisting}[language=Python]
from typing import List

class Solution:
    def isRectangleOverlap(self, rec1: List[int], rec2: List[int]) -> bool:
        # Extract coordinates
        left1, bottom1, right1, top1 = rec1
        left2, bottom2, right2, top2 = rec2
        
        # Check non-overlapping conditions
        if right1 <= left2 or right2 <= left1:
            return False
        if top1 <= bottom2 or top2 <= bottom1:
            return False
        
        # If none of the above, rectangles overlap
        return True

# Example usage:
solution = Solution()
print(solution.isRectangleOverlap([0,0,2,2], [1,1,3,3]))  # Output: True
print(solution.isRectangleOverlap([0,0,1,1], [1,0,2,1]))  # Output: False
print(solution.isRectangleOverlap([0,0,1,1], [2,2,3,3]))  # Output: False
print(solution.isRectangleOverlap([0,0,5,5], [3,3,7,7]))  # Output: True
print(solution.isRectangleOverlap([0,0,0,0], [0,0,0,0]))  # Output: False
\end{lstlisting}
\end{fullwidth}

This implementation efficiently checks for overlap by comparing the coordinates of the two rectangles. If any of the non-overlapping conditions are met, it returns \texttt{False}; otherwise, it returns \texttt{True}.

\section*{Explanation}

The \texttt{isRectangleOverlap} function determines whether two axis-aligned rectangles overlap by comparing their respective coordinates. Here's a detailed breakdown of the implementation:

\subsection*{1. Extract Coordinates}

\begin{itemize}
    \item For each rectangle, extract the left (\(x1\)), bottom (\(y1\)), right (\(x2\)), and top (\(y2\)) coordinates.
    \item This simplifies the comparison process by providing clear variables representing each side of the rectangles.
\end{itemize}

\subsection*{2. Check Non-Overlap Conditions}

\begin{itemize}
    \item **Horizontal Separation:**
    \begin{itemize}
        \item If the right side of the first rectangle (\(right1\)) is less than or equal to the left side of the second rectangle (\(left2\)), there is no horizontal overlap.
        \item Similarly, if the right side of the second rectangle (\(right2\)) is less than or equal to the left side of the first rectangle (\(left1\)), there is no horizontal overlap.
    \end{itemize}
    
    \item **Vertical Separation:**
    \begin{itemize}
        \item If the top side of the first rectangle (\(top1\)) is less than or equal to the bottom side of the second rectangle (\(bottom2\)), there is no vertical overlap.
        \item Similarly, if the top side of the second rectangle (\(top2\)) is less than or equal to the bottom side of the first rectangle (\(bottom1\)), there is no vertical overlap.
    \end{itemize}
    
    \item If any of these non-overlapping conditions are true, the rectangles do not overlap, and the function returns \texttt{False}.
\end{itemize}

\subsection*{3. Determine Overlap}

\begin{itemize}
    \item If none of the non-overlapping conditions are met, it implies that the rectangles overlap both horizontally and vertically.
    \item The function returns \texttt{True} in this case.
\end{itemize}

\subsection*{4. Example Walkthrough}

Consider the first example:
\begin{verbatim}
Input: rec1 = [0,0,2,2], rec2 = [1,1,3,3]
Output: True
\end{verbatim}

\begin{enumerate}
    \item Extract coordinates:
    \begin{itemize}
        \item rec1: left1 = 0, bottom1 = 0, right1 = 2, top1 = 2
        \item rec2: left2 = 1, bottom2 = 1, right2 = 3, top2 = 3
    \end{itemize}
    
    \item Check non-overlap conditions:
    \begin{itemize}
        \item \(right1 = 2\) is not less than or equal to \(left2 = 1\)
        \item \(right2 = 3\) is not less than or equal to \(left1 = 0\)
        \item \(top1 = 2\) is not less than or equal to \(bottom2 = 1\)
        \item \(top2 = 3\) is not less than or equal to \(bottom1 = 0\)
    \end{itemize}
    
    \item Since none of the non-overlapping conditions are met, the rectangles overlap.
\end{enumerate}

Thus, the function correctly returns \texttt{True}.

\section*{Why This Approach}

This approach is chosen for its simplicity and efficiency. By leveraging direct coordinate comparisons, the algorithm achieves constant time complexity without the need for complex data structures or iterative processes. It effectively handles all possible scenarios of rectangle positioning, ensuring accurate detection of overlaps.

\section*{Alternative Approaches}

\subsection*{1. Separating Axis Theorem (SAT)}

The Separating Axis Theorem is a more generalized method for detecting overlaps between convex shapes. While it is not necessary for axis-aligned rectangles, understanding SAT can be beneficial for more complex geometric problems.

\begin{lstlisting}[language=Python]
def isRectangleOverlap(rec1: List[int], rec2: List[int]) -> bool:
    # Using SAT for axis-aligned rectangles
    return not (rec1[2] <= rec2[0] or rec1[0] >= rec2[2] or
                rec1[3] <= rec2[1] or rec1[1] >= rec2[3])
\end{lstlisting}

\textbf{Note}: This implementation is functionally identical to the primary approach but leverages a more generalized geometric theorem.

\subsection*{2. Area-Based Approach}

Calculate the overlapping area between the two rectangles. If the overlapping area is positive, the rectangles overlap.

\begin{lstlisting}[language=Python]
def isRectangleOverlap(rec1: List[int], rec2: List[int]) -> bool:
    # Calculate overlap in x and y dimensions
    x_overlap = min(rec1[2], rec2[2]) - max(rec1[0], rec2[0])
    y_overlap = min(rec1[3], rec2[3]) - max(rec1[1], rec2[1])
    
    # Overlap exists if both overlaps are positive
    return x_overlap > 0 and y_overlap > 0
\end{lstlisting}

\textbf{Complexities:}
\begin{itemize}
    \item \textbf{Time Complexity:} \(O(1)\)
    \item \textbf{Space Complexity:} \(O(1)\)
\end{itemize}

\subsection*{3. Using Rectangles Intersection Function}

Utilize built-in or library functions that handle geometric intersections.

\begin{lstlisting}[language=Python]
from shapely.geometry import box

def isRectangleOverlap(rec1: List[int], rec2: List[int]) -> bool:
    rectangle1 = box(rec1[0], rec1[1], rec1[2], rec1[3])
    rectangle2 = box(rec2[0], rec2[1], rec2[2], rec2[3])
    return rectangle1.intersects(rectangle2) and not rectangle1.touches(rectangle2)
\end{lstlisting}

\textbf{Note}: This approach requires the \texttt{shapely} library and is more suitable for complex geometric operations.

\section*{Similar Problems to This One}

Several problems revolve around geometric overlap, intersection detection, and spatial reasoning, utilizing similar algorithmic strategies:

\begin{itemize}
    \item \textbf{Interval Overlap}: Determine if two intervals on a line overlap.
    \item \textbf{Circle Overlap}: Determine if two circles overlap based on their radii and centers.
    \item \textbf{Polygon Overlap}: Determine if two polygons overlap using algorithms like SAT.
    \item \textbf{Closest Pair of Points}: Find the closest pair of points in a set.
    \item \textbf{Convex Hull}: Compute the convex hull of a set of points.
    \item \textbf{Intersection of Lines}: Find the intersection point of two lines.
    \item \textbf{Point Inside Polygon}: Determine if a point lies inside a given polygon.
\end{itemize}

These problems reinforce the concepts of spatial reasoning, geometric property analysis, and efficient algorithm design in various contexts.

\section*{Things to Keep in Mind and Tricks}

When working with the \textbf{Rectangle Overlap} problem, consider the following tips and best practices to enhance efficiency and correctness:

\begin{itemize}
    \item \textbf{Understand Geometric Relationships}: Grasp the positional relationships between rectangles to simplify overlap detection.
    \index{Geometric Relationships}
    
    \item \textbf{Leverage Coordinate Comparisons}: Use direct comparisons of rectangle coordinates to determine spatial relationships.
    \index{Coordinate Comparisons}
    
    \item \textbf{Handle Edge Cases}: Consider cases where rectangles touch at edges or corners without overlapping.
    \index{Edge Cases}
    
    \item \textbf{Optimize for Efficiency}: Aim for a constant time \(O(1)\) solution by avoiding unnecessary computations or iterations.
    \index{Efficiency Optimization}
    
    \item \textbf{Avoid Floating-Point Precision Issues}: Since all coordinates are integers, floating-point precision is not a concern, simplifying the implementation.
    \index{Floating-Point Precision}
    
    \item \textbf{Use Helper Functions}: Create helper functions to encapsulate repetitive tasks, such as extracting coordinates or checking specific conditions.
    \index{Helper Functions}
    
    \item \textbf{Code Readability}: Maintain clear and readable code through meaningful variable names and structured logic.
    \index{Code Readability}
    
    \item \textbf{Test Extensively}: Implement a wide range of test cases, including overlapping, non-overlapping, and edge-touching rectangles, to ensure robustness.
    \index{Extensive Testing}
    
    \item \textbf{Understand Axis-Aligned Constraints}: Recognize that axis-aligned rectangles simplify overlap detection compared to rotated rectangles.
    \index{Axis-Aligned Constraints}
    
    \item \textbf{Simplify Logical Conditions}: Combine multiple conditions logically to streamline the overlap detection process.
    \index{Logical Conditions}
\end{itemize}

\section*{Corner and Special Cases to Test When Writing the Code}

When implementing the solution for the \textbf{Rectangle Overlap} problem, it is crucial to consider and rigorously test various edge cases to ensure robustness and correctness:

\begin{itemize}
    \item \textbf{No Overlap}: Rectangles are completely separate.
    \index{No Overlap}
    
    \item \textbf{Partial Overlap}: Rectangles overlap in one or more regions.
    \index{Partial Overlap}
    
    \item \textbf{Edge Touching}: Rectangles touch exactly at one edge without overlapping.
    \index{Edge Touching}
    
    \item \textbf{Corner Touching}: Rectangles touch exactly at one corner without overlapping.
    \index{Corner Touching}
    
    \item \textbf{One Rectangle Inside Another}: One rectangle is entirely within the other.
    \index{Rectangle Inside}
    
    \item \textbf{Identical Rectangles}: Both rectangles have the same coordinates.
    \index{Identical Rectangles}
    
    \item \textbf{Degenerate Rectangles}: Rectangles with zero area (e.g., \(x1 = x2\) or \(y1 = y2\)).
    \index{Degenerate Rectangles}
    
    \item \textbf{Large Coordinates}: Rectangles with very large coordinate values to test performance and integer handling.
    \index{Large Coordinates}
    
    \item \textbf{Negative Coordinates}: Rectangles positioned in negative coordinate space.
    \index{Negative Coordinates}
    
    \item \textbf{Mixed Overlapping Scenarios}: Combinations of the above cases to ensure comprehensive coverage.
    \index{Mixed Overlapping Scenarios}
    
    \item \textbf{Minimum and Maximum Bounds}: Rectangles at the minimum and maximum limits of the coordinate range.
    \index{Minimum and Maximum Bounds}
\end{itemize}

\section*{Implementation Considerations}

When implementing the \texttt{isRectangleOverlap} function, keep in mind the following considerations to ensure robustness and efficiency:

\begin{itemize}
    \item \textbf{Data Type Selection}: Use appropriate data types that can handle the range of input values without overflow or underflow.
    \index{Data Type Selection}
    
    \item \textbf{Optimizing Comparisons}: Structure logical conditions to short-circuit evaluations as soon as a non-overlapping condition is met.
    \index{Optimizing Comparisons}
    
    \item \textbf{Language-Specific Constraints}: Be aware of how the programming language handles integer division and comparisons.
    \index{Language-Specific Constraints}
    
    \item \textbf{Avoiding Redundant Calculations}: Ensure that each comparison contributes towards determining overlap without unnecessary repetitions.
    \index{Avoiding Redundant Calculations}
    
    \item \textbf{Code Readability and Documentation}: Maintain clear and readable code through meaningful variable names and comprehensive comments to facilitate understanding and maintenance.
    \index{Code Readability}
    
    \item \textbf{Edge Case Handling}: Implement checks for edge cases to prevent incorrect results or runtime errors.
    \index{Edge Case Handling}
    
    \item \textbf{Testing and Validation}: Develop a comprehensive suite of test cases that cover all possible scenarios, including edge cases, to validate the correctness and efficiency of the implementation.
    \index{Testing and Validation}
    
    \item \textbf{Scalability}: Design the algorithm to scale efficiently with increasing input sizes, maintaining performance and resource utilization.
    \index{Scalability}
    
    \item \textbf{Using Helper Functions}: Consider creating helper functions for repetitive tasks, such as extracting and comparing coordinates, to enhance modularity and reusability.
    \index{Helper Functions}
    
    \item \textbf{Consistent Naming Conventions}: Use consistent and descriptive naming conventions for variables to improve code clarity.
    \index{Naming Conventions}
    
    \item \textbf{Handling Floating-Point Coordinates}: Although the problem specifies integer coordinates, ensure that the implementation can handle floating-point numbers if needed in extended scenarios.
    \index{Floating-Point Coordinates}
    
    \item \textbf{Avoiding Floating-Point Precision Issues}: Since all coordinates are integers, floating-point precision is not a concern, simplifying the implementation.
    \index{Floating-Point Precision}
    
    \item \textbf{Implementing Unit Tests}: Develop unit tests for each logical condition to ensure that all scenarios are correctly handled.
    \index{Unit Tests}
    
    \item \textbf{Error Handling}: Incorporate error handling to manage invalid inputs gracefully.
    \index{Error Handling}
\end{itemize}

\section*{Conclusion}

The \textbf{Rectangle Overlap} problem exemplifies the application of fundamental geometric principles and conditional logic to solve spatial challenges efficiently. By leveraging simple coordinate comparisons, the algorithm achieves optimal time and space complexities, making it highly suitable for real-time applications such as collision detection in gaming, layout planning in graphics, and spatial data analysis. Understanding and implementing such techniques not only enhances problem-solving skills but also provides a foundation for tackling more complex Computational Geometry problems involving varied geometric shapes and interactions.

\printindex

% % filename: rectangle_overlap.tex

\problemsection{Rectangle Overlap}
\label{chap:Rectangle_Overlap}
\marginnote{\href{https://leetcode.com/problems/rectangle-overlap/}{[LeetCode Link]}\index{LeetCode}}
\marginnote{\href{https://www.geeksforgeeks.org/check-if-two-rectangles-overlap/}{[GeeksForGeeks Link]}\index{GeeksForGeeks}}
\marginnote{\href{https://www.interviewbit.com/problems/rectangle-overlap/}{[InterviewBit Link]}\index{InterviewBit}}
\marginnote{\href{https://app.codesignal.com/challenges/rectangle-overlap}{[CodeSignal Link]}\index{CodeSignal}}
\marginnote{\href{https://www.codewars.com/kata/rectangle-overlap/train/python}{[Codewars Link]}\index{Codewars}}

The \textbf{Rectangle Overlap} problem is a fundamental challenge in Computational Geometry that involves determining whether two axis-aligned rectangles overlap. This problem tests one's ability to understand geometric properties, implement conditional logic, and optimize for efficient computation. Mastery of this problem is essential for applications in computer graphics, collision detection, and spatial data analysis.

\section*{Problem Statement}

Given two axis-aligned rectangles in a 2D plane, determine if they overlap. Each rectangle is defined by its bottom-left and top-right coordinates.

A rectangle is represented as a list of four integers \([x1, y1, x2, y2]\), where \((x1, y1)\) are the coordinates of the bottom-left corner, and \((x2, y2)\) are the coordinates of the top-right corner.

\textbf{Function signature in Python:}
\begin{lstlisting}[language=Python]
def isRectangleOverlap(rec1: List[int], rec2: List[int]) -> bool:
\end{lstlisting}

\section*{Examples}

\textbf{Example 1:}

\begin{verbatim}
Input: rec1 = [0,0,2,2], rec2 = [1,1,3,3]
Output: True
Explanation: The rectangles overlap in the area defined by [1,1,2,2].
\end{verbatim}

\textbf{Example 2:}

\begin{verbatim}
Input: rec1 = [0,0,1,1], rec2 = [1,0,2,1]
Output: False
Explanation: The rectangles touch at the edge but do not overlap.
\end{verbatim}

\textbf{Example 3:}

\begin{verbatim}
Input: rec1 = [0,0,1,1], rec2 = [2,2,3,3]
Output: False
Explanation: The rectangles are completely separate.
\end{verbatim}

\textbf{Example 4:}

\begin{verbatim}
Input: rec1 = [0,0,5,5], rec2 = [3,3,7,7]
Output: True
Explanation: The rectangles overlap in the area defined by [3,3,5,5].
\end{verbatim}

\textbf{Example 5:}

\begin{verbatim}
Input: rec1 = [0,0,0,0], rec2 = [0,0,0,0]
Output: False
Explanation: Both rectangles are degenerate points.
\end{verbatim}

\textbf{Constraints:}

\begin{itemize}
    \item All coordinates are integers in the range \([-10^9, 10^9]\).
    \item For each rectangle, \(x1 < x2\) and \(y1 < y2\).
\end{itemize}

LeetCode link: \href{https://leetcode.com/problems/rectangle-overlap/}{Rectangle Overlap}\index{LeetCode}

\section*{Algorithmic Approach}

To determine whether two axis-aligned rectangles overlap, we can use the following logical conditions:

1. **Non-Overlap Conditions:**
   - One rectangle is to the left of the other.
   - One rectangle is above the other.

2. **Overlap Condition:**
   - If neither of the non-overlap conditions is true, the rectangles must overlap.

\subsection*{Steps:}

1. **Extract Coordinates:**
   - For both rectangles, extract the bottom-left and top-right coordinates.

2. **Check Non-Overlap Conditions:**
   - If the right side of the first rectangle is less than or equal to the left side of the second rectangle, they do not overlap.
   - If the left side of the first rectangle is greater than or equal to the right side of the second rectangle, they do not overlap.
   - If the top side of the first rectangle is less than or equal to the bottom side of the second rectangle, they do not overlap.
   - If the bottom side of the first rectangle is greater than or equal to the top side of the second rectangle, they do not overlap.

3. **Determine Overlap:**
   - If none of the non-overlap conditions are met, the rectangles overlap.

\marginnote{This approach provides an efficient \(O(1)\) time complexity solution by leveraging simple geometric comparisons.}

\section*{Complexities}

\begin{itemize}
    \item \textbf{Time Complexity:} \(O(1)\). The algorithm performs a constant number of comparisons regardless of input size.
    
    \item \textbf{Space Complexity:} \(O(1)\). Only a fixed amount of extra space is used for variables.
\end{itemize}

\section*{Python Implementation}

\marginnote{Implementing the overlap check using coordinate comparisons ensures an optimal and straightforward solution.}

Below is the complete Python code implementing the \texttt{isRectangleOverlap} function:

\begin{fullwidth}
\begin{lstlisting}[language=Python]
from typing import List

class Solution:
    def isRectangleOverlap(self, rec1: List[int], rec2: List[int]) -> bool:
        # Extract coordinates
        left1, bottom1, right1, top1 = rec1
        left2, bottom2, right2, top2 = rec2
        
        # Check non-overlapping conditions
        if right1 <= left2 or right2 <= left1:
            return False
        if top1 <= bottom2 or top2 <= bottom1:
            return False
        
        # If none of the above, rectangles overlap
        return True

# Example usage:
solution = Solution()
print(solution.isRectangleOverlap([0,0,2,2], [1,1,3,3]))  # Output: True
print(solution.isRectangleOverlap([0,0,1,1], [1,0,2,1]))  # Output: False
print(solution.isRectangleOverlap([0,0,1,1], [2,2,3,3]))  # Output: False
print(solution.isRectangleOverlap([0,0,5,5], [3,3,7,7]))  # Output: True
print(solution.isRectangleOverlap([0,0,0,0], [0,0,0,0]))  # Output: False
\end{lstlisting}
\end{fullwidth}

This implementation efficiently checks for overlap by comparing the coordinates of the two rectangles. If any of the non-overlapping conditions are met, it returns \texttt{False}; otherwise, it returns \texttt{True}.

\section*{Explanation}

The \texttt{isRectangleOverlap} function determines whether two axis-aligned rectangles overlap by comparing their respective coordinates. Here's a detailed breakdown of the implementation:

\subsection*{1. Extract Coordinates}

\begin{itemize}
    \item For each rectangle, extract the left (\(x1\)), bottom (\(y1\)), right (\(x2\)), and top (\(y2\)) coordinates.
    \item This simplifies the comparison process by providing clear variables representing each side of the rectangles.
\end{itemize}

\subsection*{2. Check Non-Overlap Conditions}

\begin{itemize}
    \item **Horizontal Separation:**
    \begin{itemize}
        \item If the right side of the first rectangle (\(right1\)) is less than or equal to the left side of the second rectangle (\(left2\)), there is no horizontal overlap.
        \item Similarly, if the right side of the second rectangle (\(right2\)) is less than or equal to the left side of the first rectangle (\(left1\)), there is no horizontal overlap.
    \end{itemize}
    
    \item **Vertical Separation:**
    \begin{itemize}
        \item If the top side of the first rectangle (\(top1\)) is less than or equal to the bottom side of the second rectangle (\(bottom2\)), there is no vertical overlap.
        \item Similarly, if the top side of the second rectangle (\(top2\)) is less than or equal to the bottom side of the first rectangle (\(bottom1\)), there is no vertical overlap.
    \end{itemize}
    
    \item If any of these non-overlapping conditions are true, the rectangles do not overlap, and the function returns \texttt{False}.
\end{itemize}

\subsection*{3. Determine Overlap}

\begin{itemize}
    \item If none of the non-overlapping conditions are met, it implies that the rectangles overlap both horizontally and vertically.
    \item The function returns \texttt{True} in this case.
\end{itemize}

\subsection*{4. Example Walkthrough}

Consider the first example:
\begin{verbatim}
Input: rec1 = [0,0,2,2], rec2 = [1,1,3,3]
Output: True
\end{verbatim}

\begin{enumerate}
    \item Extract coordinates:
    \begin{itemize}
        \item rec1: left1 = 0, bottom1 = 0, right1 = 2, top1 = 2
        \item rec2: left2 = 1, bottom2 = 1, right2 = 3, top2 = 3
    \end{itemize}
    
    \item Check non-overlap conditions:
    \begin{itemize}
        \item \(right1 = 2\) is not less than or equal to \(left2 = 1\)
        \item \(right2 = 3\) is not less than or equal to \(left1 = 0\)
        \item \(top1 = 2\) is not less than or equal to \(bottom2 = 1\)
        \item \(top2 = 3\) is not less than or equal to \(bottom1 = 0\)
    \end{itemize}
    
    \item Since none of the non-overlapping conditions are met, the rectangles overlap.
\end{enumerate}

Thus, the function correctly returns \texttt{True}.

\section*{Why This Approach}

This approach is chosen for its simplicity and efficiency. By leveraging direct coordinate comparisons, the algorithm achieves constant time complexity without the need for complex data structures or iterative processes. It effectively handles all possible scenarios of rectangle positioning, ensuring accurate detection of overlaps.

\section*{Alternative Approaches}

\subsection*{1. Separating Axis Theorem (SAT)}

The Separating Axis Theorem is a more generalized method for detecting overlaps between convex shapes. While it is not necessary for axis-aligned rectangles, understanding SAT can be beneficial for more complex geometric problems.

\begin{lstlisting}[language=Python]
def isRectangleOverlap(rec1: List[int], rec2: List[int]) -> bool:
    # Using SAT for axis-aligned rectangles
    return not (rec1[2] <= rec2[0] or rec1[0] >= rec2[2] or
                rec1[3] <= rec2[1] or rec1[1] >= rec2[3])
\end{lstlisting}

\textbf{Note}: This implementation is functionally identical to the primary approach but leverages a more generalized geometric theorem.

\subsection*{2. Area-Based Approach}

Calculate the overlapping area between the two rectangles. If the overlapping area is positive, the rectangles overlap.

\begin{lstlisting}[language=Python]
def isRectangleOverlap(rec1: List[int], rec2: List[int]) -> bool:
    # Calculate overlap in x and y dimensions
    x_overlap = min(rec1[2], rec2[2]) - max(rec1[0], rec2[0])
    y_overlap = min(rec1[3], rec2[3]) - max(rec1[1], rec2[1])
    
    # Overlap exists if both overlaps are positive
    return x_overlap > 0 and y_overlap > 0
\end{lstlisting}

\textbf{Complexities:}
\begin{itemize}
    \item \textbf{Time Complexity:} \(O(1)\)
    \item \textbf{Space Complexity:} \(O(1)\)
\end{itemize}

\subsection*{3. Using Rectangles Intersection Function}

Utilize built-in or library functions that handle geometric intersections.

\begin{lstlisting}[language=Python]
from shapely.geometry import box

def isRectangleOverlap(rec1: List[int], rec2: List[int]) -> bool:
    rectangle1 = box(rec1[0], rec1[1], rec1[2], rec1[3])
    rectangle2 = box(rec2[0], rec2[1], rec2[2], rec2[3])
    return rectangle1.intersects(rectangle2) and not rectangle1.touches(rectangle2)
\end{lstlisting}

\textbf{Note}: This approach requires the \texttt{shapely} library and is more suitable for complex geometric operations.

\section*{Similar Problems to This One}

Several problems revolve around geometric overlap, intersection detection, and spatial reasoning, utilizing similar algorithmic strategies:

\begin{itemize}
    \item \textbf{Interval Overlap}: Determine if two intervals on a line overlap.
    \item \textbf{Circle Overlap}: Determine if two circles overlap based on their radii and centers.
    \item \textbf{Polygon Overlap}: Determine if two polygons overlap using algorithms like SAT.
    \item \textbf{Closest Pair of Points}: Find the closest pair of points in a set.
    \item \textbf{Convex Hull}: Compute the convex hull of a set of points.
    \item \textbf{Intersection of Lines}: Find the intersection point of two lines.
    \item \textbf{Point Inside Polygon}: Determine if a point lies inside a given polygon.
\end{itemize}

These problems reinforce the concepts of spatial reasoning, geometric property analysis, and efficient algorithm design in various contexts.

\section*{Things to Keep in Mind and Tricks}

When working with the \textbf{Rectangle Overlap} problem, consider the following tips and best practices to enhance efficiency and correctness:

\begin{itemize}
    \item \textbf{Understand Geometric Relationships}: Grasp the positional relationships between rectangles to simplify overlap detection.
    \index{Geometric Relationships}
    
    \item \textbf{Leverage Coordinate Comparisons}: Use direct comparisons of rectangle coordinates to determine spatial relationships.
    \index{Coordinate Comparisons}
    
    \item \textbf{Handle Edge Cases}: Consider cases where rectangles touch at edges or corners without overlapping.
    \index{Edge Cases}
    
    \item \textbf{Optimize for Efficiency}: Aim for a constant time \(O(1)\) solution by avoiding unnecessary computations or iterations.
    \index{Efficiency Optimization}
    
    \item \textbf{Avoid Floating-Point Precision Issues}: Since all coordinates are integers, floating-point precision is not a concern, simplifying the implementation.
    \index{Floating-Point Precision}
    
    \item \textbf{Use Helper Functions}: Create helper functions to encapsulate repetitive tasks, such as extracting coordinates or checking specific conditions.
    \index{Helper Functions}
    
    \item \textbf{Code Readability}: Maintain clear and readable code through meaningful variable names and structured logic.
    \index{Code Readability}
    
    \item \textbf{Test Extensively}: Implement a wide range of test cases, including overlapping, non-overlapping, and edge-touching rectangles, to ensure robustness.
    \index{Extensive Testing}
    
    \item \textbf{Understand Axis-Aligned Constraints}: Recognize that axis-aligned rectangles simplify overlap detection compared to rotated rectangles.
    \index{Axis-Aligned Constraints}
    
    \item \textbf{Simplify Logical Conditions}: Combine multiple conditions logically to streamline the overlap detection process.
    \index{Logical Conditions}
\end{itemize}

\section*{Corner and Special Cases to Test When Writing the Code}

When implementing the solution for the \textbf{Rectangle Overlap} problem, it is crucial to consider and rigorously test various edge cases to ensure robustness and correctness:

\begin{itemize}
    \item \textbf{No Overlap}: Rectangles are completely separate.
    \index{No Overlap}
    
    \item \textbf{Partial Overlap}: Rectangles overlap in one or more regions.
    \index{Partial Overlap}
    
    \item \textbf{Edge Touching}: Rectangles touch exactly at one edge without overlapping.
    \index{Edge Touching}
    
    \item \textbf{Corner Touching}: Rectangles touch exactly at one corner without overlapping.
    \index{Corner Touching}
    
    \item \textbf{One Rectangle Inside Another}: One rectangle is entirely within the other.
    \index{Rectangle Inside}
    
    \item \textbf{Identical Rectangles}: Both rectangles have the same coordinates.
    \index{Identical Rectangles}
    
    \item \textbf{Degenerate Rectangles}: Rectangles with zero area (e.g., \(x1 = x2\) or \(y1 = y2\)).
    \index{Degenerate Rectangles}
    
    \item \textbf{Large Coordinates}: Rectangles with very large coordinate values to test performance and integer handling.
    \index{Large Coordinates}
    
    \item \textbf{Negative Coordinates}: Rectangles positioned in negative coordinate space.
    \index{Negative Coordinates}
    
    \item \textbf{Mixed Overlapping Scenarios}: Combinations of the above cases to ensure comprehensive coverage.
    \index{Mixed Overlapping Scenarios}
    
    \item \textbf{Minimum and Maximum Bounds}: Rectangles at the minimum and maximum limits of the coordinate range.
    \index{Minimum and Maximum Bounds}
\end{itemize}

\section*{Implementation Considerations}

When implementing the \texttt{isRectangleOverlap} function, keep in mind the following considerations to ensure robustness and efficiency:

\begin{itemize}
    \item \textbf{Data Type Selection}: Use appropriate data types that can handle the range of input values without overflow or underflow.
    \index{Data Type Selection}
    
    \item \textbf{Optimizing Comparisons}: Structure logical conditions to short-circuit evaluations as soon as a non-overlapping condition is met.
    \index{Optimizing Comparisons}
    
    \item \textbf{Language-Specific Constraints}: Be aware of how the programming language handles integer division and comparisons.
    \index{Language-Specific Constraints}
    
    \item \textbf{Avoiding Redundant Calculations}: Ensure that each comparison contributes towards determining overlap without unnecessary repetitions.
    \index{Avoiding Redundant Calculations}
    
    \item \textbf{Code Readability and Documentation}: Maintain clear and readable code through meaningful variable names and comprehensive comments to facilitate understanding and maintenance.
    \index{Code Readability}
    
    \item \textbf{Edge Case Handling}: Implement checks for edge cases to prevent incorrect results or runtime errors.
    \index{Edge Case Handling}
    
    \item \textbf{Testing and Validation}: Develop a comprehensive suite of test cases that cover all possible scenarios, including edge cases, to validate the correctness and efficiency of the implementation.
    \index{Testing and Validation}
    
    \item \textbf{Scalability}: Design the algorithm to scale efficiently with increasing input sizes, maintaining performance and resource utilization.
    \index{Scalability}
    
    \item \textbf{Using Helper Functions}: Consider creating helper functions for repetitive tasks, such as extracting and comparing coordinates, to enhance modularity and reusability.
    \index{Helper Functions}
    
    \item \textbf{Consistent Naming Conventions}: Use consistent and descriptive naming conventions for variables to improve code clarity.
    \index{Naming Conventions}
    
    \item \textbf{Handling Floating-Point Coordinates}: Although the problem specifies integer coordinates, ensure that the implementation can handle floating-point numbers if needed in extended scenarios.
    \index{Floating-Point Coordinates}
    
    \item \textbf{Avoiding Floating-Point Precision Issues}: Since all coordinates are integers, floating-point precision is not a concern, simplifying the implementation.
    \index{Floating-Point Precision}
    
    \item \textbf{Implementing Unit Tests}: Develop unit tests for each logical condition to ensure that all scenarios are correctly handled.
    \index{Unit Tests}
    
    \item \textbf{Error Handling}: Incorporate error handling to manage invalid inputs gracefully.
    \index{Error Handling}
\end{itemize}

\section*{Conclusion}

The \textbf{Rectangle Overlap} problem exemplifies the application of fundamental geometric principles and conditional logic to solve spatial challenges efficiently. By leveraging simple coordinate comparisons, the algorithm achieves optimal time and space complexities, making it highly suitable for real-time applications such as collision detection in gaming, layout planning in graphics, and spatial data analysis. Understanding and implementing such techniques not only enhances problem-solving skills but also provides a foundation for tackling more complex Computational Geometry problems involving varied geometric shapes and interactions.

\printindex

% % filename: rectangle_overlap.tex

\problemsection{Rectangle Overlap}
\label{chap:Rectangle_Overlap}
\marginnote{\href{https://leetcode.com/problems/rectangle-overlap/}{[LeetCode Link]}\index{LeetCode}}
\marginnote{\href{https://www.geeksforgeeks.org/check-if-two-rectangles-overlap/}{[GeeksForGeeks Link]}\index{GeeksForGeeks}}
\marginnote{\href{https://www.interviewbit.com/problems/rectangle-overlap/}{[InterviewBit Link]}\index{InterviewBit}}
\marginnote{\href{https://app.codesignal.com/challenges/rectangle-overlap}{[CodeSignal Link]}\index{CodeSignal}}
\marginnote{\href{https://www.codewars.com/kata/rectangle-overlap/train/python}{[Codewars Link]}\index{Codewars}}

The \textbf{Rectangle Overlap} problem is a fundamental challenge in Computational Geometry that involves determining whether two axis-aligned rectangles overlap. This problem tests one's ability to understand geometric properties, implement conditional logic, and optimize for efficient computation. Mastery of this problem is essential for applications in computer graphics, collision detection, and spatial data analysis.

\section*{Problem Statement}

Given two axis-aligned rectangles in a 2D plane, determine if they overlap. Each rectangle is defined by its bottom-left and top-right coordinates.

A rectangle is represented as a list of four integers \([x1, y1, x2, y2]\), where \((x1, y1)\) are the coordinates of the bottom-left corner, and \((x2, y2)\) are the coordinates of the top-right corner.

\textbf{Function signature in Python:}
\begin{lstlisting}[language=Python]
def isRectangleOverlap(rec1: List[int], rec2: List[int]) -> bool:
\end{lstlisting}

\section*{Examples}

\textbf{Example 1:}

\begin{verbatim}
Input: rec1 = [0,0,2,2], rec2 = [1,1,3,3]
Output: True
Explanation: The rectangles overlap in the area defined by [1,1,2,2].
\end{verbatim}

\textbf{Example 2:}

\begin{verbatim}
Input: rec1 = [0,0,1,1], rec2 = [1,0,2,1]
Output: False
Explanation: The rectangles touch at the edge but do not overlap.
\end{verbatim}

\textbf{Example 3:}

\begin{verbatim}
Input: rec1 = [0,0,1,1], rec2 = [2,2,3,3]
Output: False
Explanation: The rectangles are completely separate.
\end{verbatim}

\textbf{Example 4:}

\begin{verbatim}
Input: rec1 = [0,0,5,5], rec2 = [3,3,7,7]
Output: True
Explanation: The rectangles overlap in the area defined by [3,3,5,5].
\end{verbatim}

\textbf{Example 5:}

\begin{verbatim}
Input: rec1 = [0,0,0,0], rec2 = [0,0,0,0]
Output: False
Explanation: Both rectangles are degenerate points.
\end{verbatim}

\textbf{Constraints:}

\begin{itemize}
    \item All coordinates are integers in the range \([-10^9, 10^9]\).
    \item For each rectangle, \(x1 < x2\) and \(y1 < y2\).
\end{itemize}

LeetCode link: \href{https://leetcode.com/problems/rectangle-overlap/}{Rectangle Overlap}\index{LeetCode}

\section*{Algorithmic Approach}

To determine whether two axis-aligned rectangles overlap, we can use the following logical conditions:

1. **Non-Overlap Conditions:**
   - One rectangle is to the left of the other.
   - One rectangle is above the other.

2. **Overlap Condition:**
   - If neither of the non-overlap conditions is true, the rectangles must overlap.

\subsection*{Steps:}

1. **Extract Coordinates:**
   - For both rectangles, extract the bottom-left and top-right coordinates.

2. **Check Non-Overlap Conditions:**
   - If the right side of the first rectangle is less than or equal to the left side of the second rectangle, they do not overlap.
   - If the left side of the first rectangle is greater than or equal to the right side of the second rectangle, they do not overlap.
   - If the top side of the first rectangle is less than or equal to the bottom side of the second rectangle, they do not overlap.
   - If the bottom side of the first rectangle is greater than or equal to the top side of the second rectangle, they do not overlap.

3. **Determine Overlap:**
   - If none of the non-overlap conditions are met, the rectangles overlap.

\marginnote{This approach provides an efficient \(O(1)\) time complexity solution by leveraging simple geometric comparisons.}

\section*{Complexities}

\begin{itemize}
    \item \textbf{Time Complexity:} \(O(1)\). The algorithm performs a constant number of comparisons regardless of input size.
    
    \item \textbf{Space Complexity:} \(O(1)\). Only a fixed amount of extra space is used for variables.
\end{itemize}

\section*{Python Implementation}

\marginnote{Implementing the overlap check using coordinate comparisons ensures an optimal and straightforward solution.}

Below is the complete Python code implementing the \texttt{isRectangleOverlap} function:

\begin{fullwidth}
\begin{lstlisting}[language=Python]
from typing import List

class Solution:
    def isRectangleOverlap(self, rec1: List[int], rec2: List[int]) -> bool:
        # Extract coordinates
        left1, bottom1, right1, top1 = rec1
        left2, bottom2, right2, top2 = rec2
        
        # Check non-overlapping conditions
        if right1 <= left2 or right2 <= left1:
            return False
        if top1 <= bottom2 or top2 <= bottom1:
            return False
        
        # If none of the above, rectangles overlap
        return True

# Example usage:
solution = Solution()
print(solution.isRectangleOverlap([0,0,2,2], [1,1,3,3]))  # Output: True
print(solution.isRectangleOverlap([0,0,1,1], [1,0,2,1]))  # Output: False
print(solution.isRectangleOverlap([0,0,1,1], [2,2,3,3]))  # Output: False
print(solution.isRectangleOverlap([0,0,5,5], [3,3,7,7]))  # Output: True
print(solution.isRectangleOverlap([0,0,0,0], [0,0,0,0]))  # Output: False
\end{lstlisting}
\end{fullwidth}

This implementation efficiently checks for overlap by comparing the coordinates of the two rectangles. If any of the non-overlapping conditions are met, it returns \texttt{False}; otherwise, it returns \texttt{True}.

\section*{Explanation}

The \texttt{isRectangleOverlap} function determines whether two axis-aligned rectangles overlap by comparing their respective coordinates. Here's a detailed breakdown of the implementation:

\subsection*{1. Extract Coordinates}

\begin{itemize}
    \item For each rectangle, extract the left (\(x1\)), bottom (\(y1\)), right (\(x2\)), and top (\(y2\)) coordinates.
    \item This simplifies the comparison process by providing clear variables representing each side of the rectangles.
\end{itemize}

\subsection*{2. Check Non-Overlap Conditions}

\begin{itemize}
    \item **Horizontal Separation:**
    \begin{itemize}
        \item If the right side of the first rectangle (\(right1\)) is less than or equal to the left side of the second rectangle (\(left2\)), there is no horizontal overlap.
        \item Similarly, if the right side of the second rectangle (\(right2\)) is less than or equal to the left side of the first rectangle (\(left1\)), there is no horizontal overlap.
    \end{itemize}
    
    \item **Vertical Separation:**
    \begin{itemize}
        \item If the top side of the first rectangle (\(top1\)) is less than or equal to the bottom side of the second rectangle (\(bottom2\)), there is no vertical overlap.
        \item Similarly, if the top side of the second rectangle (\(top2\)) is less than or equal to the bottom side of the first rectangle (\(bottom1\)), there is no vertical overlap.
    \end{itemize}
    
    \item If any of these non-overlapping conditions are true, the rectangles do not overlap, and the function returns \texttt{False}.
\end{itemize}

\subsection*{3. Determine Overlap}

\begin{itemize}
    \item If none of the non-overlapping conditions are met, it implies that the rectangles overlap both horizontally and vertically.
    \item The function returns \texttt{True} in this case.
\end{itemize}

\subsection*{4. Example Walkthrough}

Consider the first example:
\begin{verbatim}
Input: rec1 = [0,0,2,2], rec2 = [1,1,3,3]
Output: True
\end{verbatim}

\begin{enumerate}
    \item Extract coordinates:
    \begin{itemize}
        \item rec1: left1 = 0, bottom1 = 0, right1 = 2, top1 = 2
        \item rec2: left2 = 1, bottom2 = 1, right2 = 3, top2 = 3
    \end{itemize}
    
    \item Check non-overlap conditions:
    \begin{itemize}
        \item \(right1 = 2\) is not less than or equal to \(left2 = 1\)
        \item \(right2 = 3\) is not less than or equal to \(left1 = 0\)
        \item \(top1 = 2\) is not less than or equal to \(bottom2 = 1\)
        \item \(top2 = 3\) is not less than or equal to \(bottom1 = 0\)
    \end{itemize}
    
    \item Since none of the non-overlapping conditions are met, the rectangles overlap.
\end{enumerate}

Thus, the function correctly returns \texttt{True}.

\section*{Why This Approach}

This approach is chosen for its simplicity and efficiency. By leveraging direct coordinate comparisons, the algorithm achieves constant time complexity without the need for complex data structures or iterative processes. It effectively handles all possible scenarios of rectangle positioning, ensuring accurate detection of overlaps.

\section*{Alternative Approaches}

\subsection*{1. Separating Axis Theorem (SAT)}

The Separating Axis Theorem is a more generalized method for detecting overlaps between convex shapes. While it is not necessary for axis-aligned rectangles, understanding SAT can be beneficial for more complex geometric problems.

\begin{lstlisting}[language=Python]
def isRectangleOverlap(rec1: List[int], rec2: List[int]) -> bool:
    # Using SAT for axis-aligned rectangles
    return not (rec1[2] <= rec2[0] or rec1[0] >= rec2[2] or
                rec1[3] <= rec2[1] or rec1[1] >= rec2[3])
\end{lstlisting}

\textbf{Note}: This implementation is functionally identical to the primary approach but leverages a more generalized geometric theorem.

\subsection*{2. Area-Based Approach}

Calculate the overlapping area between the two rectangles. If the overlapping area is positive, the rectangles overlap.

\begin{lstlisting}[language=Python]
def isRectangleOverlap(rec1: List[int], rec2: List[int]) -> bool:
    # Calculate overlap in x and y dimensions
    x_overlap = min(rec1[2], rec2[2]) - max(rec1[0], rec2[0])
    y_overlap = min(rec1[3], rec2[3]) - max(rec1[1], rec2[1])
    
    # Overlap exists if both overlaps are positive
    return x_overlap > 0 and y_overlap > 0
\end{lstlisting}

\textbf{Complexities:}
\begin{itemize}
    \item \textbf{Time Complexity:} \(O(1)\)
    \item \textbf{Space Complexity:} \(O(1)\)
\end{itemize}

\subsection*{3. Using Rectangles Intersection Function}

Utilize built-in or library functions that handle geometric intersections.

\begin{lstlisting}[language=Python]
from shapely.geometry import box

def isRectangleOverlap(rec1: List[int], rec2: List[int]) -> bool:
    rectangle1 = box(rec1[0], rec1[1], rec1[2], rec1[3])
    rectangle2 = box(rec2[0], rec2[1], rec2[2], rec2[3])
    return rectangle1.intersects(rectangle2) and not rectangle1.touches(rectangle2)
\end{lstlisting}

\textbf{Note}: This approach requires the \texttt{shapely} library and is more suitable for complex geometric operations.

\section*{Similar Problems to This One}

Several problems revolve around geometric overlap, intersection detection, and spatial reasoning, utilizing similar algorithmic strategies:

\begin{itemize}
    \item \textbf{Interval Overlap}: Determine if two intervals on a line overlap.
    \item \textbf{Circle Overlap}: Determine if two circles overlap based on their radii and centers.
    \item \textbf{Polygon Overlap}: Determine if two polygons overlap using algorithms like SAT.
    \item \textbf{Closest Pair of Points}: Find the closest pair of points in a set.
    \item \textbf{Convex Hull}: Compute the convex hull of a set of points.
    \item \textbf{Intersection of Lines}: Find the intersection point of two lines.
    \item \textbf{Point Inside Polygon}: Determine if a point lies inside a given polygon.
\end{itemize}

These problems reinforce the concepts of spatial reasoning, geometric property analysis, and efficient algorithm design in various contexts.

\section*{Things to Keep in Mind and Tricks}

When working with the \textbf{Rectangle Overlap} problem, consider the following tips and best practices to enhance efficiency and correctness:

\begin{itemize}
    \item \textbf{Understand Geometric Relationships}: Grasp the positional relationships between rectangles to simplify overlap detection.
    \index{Geometric Relationships}
    
    \item \textbf{Leverage Coordinate Comparisons}: Use direct comparisons of rectangle coordinates to determine spatial relationships.
    \index{Coordinate Comparisons}
    
    \item \textbf{Handle Edge Cases}: Consider cases where rectangles touch at edges or corners without overlapping.
    \index{Edge Cases}
    
    \item \textbf{Optimize for Efficiency}: Aim for a constant time \(O(1)\) solution by avoiding unnecessary computations or iterations.
    \index{Efficiency Optimization}
    
    \item \textbf{Avoid Floating-Point Precision Issues}: Since all coordinates are integers, floating-point precision is not a concern, simplifying the implementation.
    \index{Floating-Point Precision}
    
    \item \textbf{Use Helper Functions}: Create helper functions to encapsulate repetitive tasks, such as extracting coordinates or checking specific conditions.
    \index{Helper Functions}
    
    \item \textbf{Code Readability}: Maintain clear and readable code through meaningful variable names and structured logic.
    \index{Code Readability}
    
    \item \textbf{Test Extensively}: Implement a wide range of test cases, including overlapping, non-overlapping, and edge-touching rectangles, to ensure robustness.
    \index{Extensive Testing}
    
    \item \textbf{Understand Axis-Aligned Constraints}: Recognize that axis-aligned rectangles simplify overlap detection compared to rotated rectangles.
    \index{Axis-Aligned Constraints}
    
    \item \textbf{Simplify Logical Conditions}: Combine multiple conditions logically to streamline the overlap detection process.
    \index{Logical Conditions}
\end{itemize}

\section*{Corner and Special Cases to Test When Writing the Code}

When implementing the solution for the \textbf{Rectangle Overlap} problem, it is crucial to consider and rigorously test various edge cases to ensure robustness and correctness:

\begin{itemize}
    \item \textbf{No Overlap}: Rectangles are completely separate.
    \index{No Overlap}
    
    \item \textbf{Partial Overlap}: Rectangles overlap in one or more regions.
    \index{Partial Overlap}
    
    \item \textbf{Edge Touching}: Rectangles touch exactly at one edge without overlapping.
    \index{Edge Touching}
    
    \item \textbf{Corner Touching}: Rectangles touch exactly at one corner without overlapping.
    \index{Corner Touching}
    
    \item \textbf{One Rectangle Inside Another}: One rectangle is entirely within the other.
    \index{Rectangle Inside}
    
    \item \textbf{Identical Rectangles}: Both rectangles have the same coordinates.
    \index{Identical Rectangles}
    
    \item \textbf{Degenerate Rectangles}: Rectangles with zero area (e.g., \(x1 = x2\) or \(y1 = y2\)).
    \index{Degenerate Rectangles}
    
    \item \textbf{Large Coordinates}: Rectangles with very large coordinate values to test performance and integer handling.
    \index{Large Coordinates}
    
    \item \textbf{Negative Coordinates}: Rectangles positioned in negative coordinate space.
    \index{Negative Coordinates}
    
    \item \textbf{Mixed Overlapping Scenarios}: Combinations of the above cases to ensure comprehensive coverage.
    \index{Mixed Overlapping Scenarios}
    
    \item \textbf{Minimum and Maximum Bounds}: Rectangles at the minimum and maximum limits of the coordinate range.
    \index{Minimum and Maximum Bounds}
\end{itemize}

\section*{Implementation Considerations}

When implementing the \texttt{isRectangleOverlap} function, keep in mind the following considerations to ensure robustness and efficiency:

\begin{itemize}
    \item \textbf{Data Type Selection}: Use appropriate data types that can handle the range of input values without overflow or underflow.
    \index{Data Type Selection}
    
    \item \textbf{Optimizing Comparisons}: Structure logical conditions to short-circuit evaluations as soon as a non-overlapping condition is met.
    \index{Optimizing Comparisons}
    
    \item \textbf{Language-Specific Constraints}: Be aware of how the programming language handles integer division and comparisons.
    \index{Language-Specific Constraints}
    
    \item \textbf{Avoiding Redundant Calculations}: Ensure that each comparison contributes towards determining overlap without unnecessary repetitions.
    \index{Avoiding Redundant Calculations}
    
    \item \textbf{Code Readability and Documentation}: Maintain clear and readable code through meaningful variable names and comprehensive comments to facilitate understanding and maintenance.
    \index{Code Readability}
    
    \item \textbf{Edge Case Handling}: Implement checks for edge cases to prevent incorrect results or runtime errors.
    \index{Edge Case Handling}
    
    \item \textbf{Testing and Validation}: Develop a comprehensive suite of test cases that cover all possible scenarios, including edge cases, to validate the correctness and efficiency of the implementation.
    \index{Testing and Validation}
    
    \item \textbf{Scalability}: Design the algorithm to scale efficiently with increasing input sizes, maintaining performance and resource utilization.
    \index{Scalability}
    
    \item \textbf{Using Helper Functions}: Consider creating helper functions for repetitive tasks, such as extracting and comparing coordinates, to enhance modularity and reusability.
    \index{Helper Functions}
    
    \item \textbf{Consistent Naming Conventions}: Use consistent and descriptive naming conventions for variables to improve code clarity.
    \index{Naming Conventions}
    
    \item \textbf{Handling Floating-Point Coordinates}: Although the problem specifies integer coordinates, ensure that the implementation can handle floating-point numbers if needed in extended scenarios.
    \index{Floating-Point Coordinates}
    
    \item \textbf{Avoiding Floating-Point Precision Issues}: Since all coordinates are integers, floating-point precision is not a concern, simplifying the implementation.
    \index{Floating-Point Precision}
    
    \item \textbf{Implementing Unit Tests}: Develop unit tests for each logical condition to ensure that all scenarios are correctly handled.
    \index{Unit Tests}
    
    \item \textbf{Error Handling}: Incorporate error handling to manage invalid inputs gracefully.
    \index{Error Handling}
\end{itemize}

\section*{Conclusion}

The \textbf{Rectangle Overlap} problem exemplifies the application of fundamental geometric principles and conditional logic to solve spatial challenges efficiently. By leveraging simple coordinate comparisons, the algorithm achieves optimal time and space complexities, making it highly suitable for real-time applications such as collision detection in gaming, layout planning in graphics, and spatial data analysis. Understanding and implementing such techniques not only enhances problem-solving skills but also provides a foundation for tackling more complex Computational Geometry problems involving varied geometric shapes and interactions.

\printindex

% % filename: rectangle_overlap.tex

\problemsection{Rectangle Overlap}
\label{chap:Rectangle_Overlap}
\marginnote{\href{https://leetcode.com/problems/rectangle-overlap/}{[LeetCode Link]}\index{LeetCode}}
\marginnote{\href{https://www.geeksforgeeks.org/check-if-two-rectangles-overlap/}{[GeeksForGeeks Link]}\index{GeeksForGeeks}}
\marginnote{\href{https://www.interviewbit.com/problems/rectangle-overlap/}{[InterviewBit Link]}\index{InterviewBit}}
\marginnote{\href{https://app.codesignal.com/challenges/rectangle-overlap}{[CodeSignal Link]}\index{CodeSignal}}
\marginnote{\href{https://www.codewars.com/kata/rectangle-overlap/train/python}{[Codewars Link]}\index{Codewars}}

The \textbf{Rectangle Overlap} problem is a fundamental challenge in Computational Geometry that involves determining whether two axis-aligned rectangles overlap. This problem tests one's ability to understand geometric properties, implement conditional logic, and optimize for efficient computation. Mastery of this problem is essential for applications in computer graphics, collision detection, and spatial data analysis.

\section*{Problem Statement}

Given two axis-aligned rectangles in a 2D plane, determine if they overlap. Each rectangle is defined by its bottom-left and top-right coordinates.

A rectangle is represented as a list of four integers \([x1, y1, x2, y2]\), where \((x1, y1)\) are the coordinates of the bottom-left corner, and \((x2, y2)\) are the coordinates of the top-right corner.

\textbf{Function signature in Python:}
\begin{lstlisting}[language=Python]
def isRectangleOverlap(rec1: List[int], rec2: List[int]) -> bool:
\end{lstlisting}

\section*{Examples}

\textbf{Example 1:}

\begin{verbatim}
Input: rec1 = [0,0,2,2], rec2 = [1,1,3,3]
Output: True
Explanation: The rectangles overlap in the area defined by [1,1,2,2].
\end{verbatim}

\textbf{Example 2:}

\begin{verbatim}
Input: rec1 = [0,0,1,1], rec2 = [1,0,2,1]
Output: False
Explanation: The rectangles touch at the edge but do not overlap.
\end{verbatim}

\textbf{Example 3:}

\begin{verbatim}
Input: rec1 = [0,0,1,1], rec2 = [2,2,3,3]
Output: False
Explanation: The rectangles are completely separate.
\end{verbatim}

\textbf{Example 4:}

\begin{verbatim}
Input: rec1 = [0,0,5,5], rec2 = [3,3,7,7]
Output: True
Explanation: The rectangles overlap in the area defined by [3,3,5,5].
\end{verbatim}

\textbf{Example 5:}

\begin{verbatim}
Input: rec1 = [0,0,0,0], rec2 = [0,0,0,0]
Output: False
Explanation: Both rectangles are degenerate points.
\end{verbatim}

\textbf{Constraints:}

\begin{itemize}
    \item All coordinates are integers in the range \([-10^9, 10^9]\).
    \item For each rectangle, \(x1 < x2\) and \(y1 < y2\).
\end{itemize}

LeetCode link: \href{https://leetcode.com/problems/rectangle-overlap/}{Rectangle Overlap}\index{LeetCode}

\section*{Algorithmic Approach}

To determine whether two axis-aligned rectangles overlap, we can use the following logical conditions:

1. **Non-Overlap Conditions:**
   - One rectangle is to the left of the other.
   - One rectangle is above the other.

2. **Overlap Condition:**
   - If neither of the non-overlap conditions is true, the rectangles must overlap.

\subsection*{Steps:}

1. **Extract Coordinates:**
   - For both rectangles, extract the bottom-left and top-right coordinates.

2. **Check Non-Overlap Conditions:**
   - If the right side of the first rectangle is less than or equal to the left side of the second rectangle, they do not overlap.
   - If the left side of the first rectangle is greater than or equal to the right side of the second rectangle, they do not overlap.
   - If the top side of the first rectangle is less than or equal to the bottom side of the second rectangle, they do not overlap.
   - If the bottom side of the first rectangle is greater than or equal to the top side of the second rectangle, they do not overlap.

3. **Determine Overlap:**
   - If none of the non-overlap conditions are met, the rectangles overlap.

\marginnote{This approach provides an efficient \(O(1)\) time complexity solution by leveraging simple geometric comparisons.}

\section*{Complexities}

\begin{itemize}
    \item \textbf{Time Complexity:} \(O(1)\). The algorithm performs a constant number of comparisons regardless of input size.
    
    \item \textbf{Space Complexity:} \(O(1)\). Only a fixed amount of extra space is used for variables.
\end{itemize}

\section*{Python Implementation}

\marginnote{Implementing the overlap check using coordinate comparisons ensures an optimal and straightforward solution.}

Below is the complete Python code implementing the \texttt{isRectangleOverlap} function:

\begin{fullwidth}
\begin{lstlisting}[language=Python]
from typing import List

class Solution:
    def isRectangleOverlap(self, rec1: List[int], rec2: List[int]) -> bool:
        # Extract coordinates
        left1, bottom1, right1, top1 = rec1
        left2, bottom2, right2, top2 = rec2
        
        # Check non-overlapping conditions
        if right1 <= left2 or right2 <= left1:
            return False
        if top1 <= bottom2 or top2 <= bottom1:
            return False
        
        # If none of the above, rectangles overlap
        return True

# Example usage:
solution = Solution()
print(solution.isRectangleOverlap([0,0,2,2], [1,1,3,3]))  # Output: True
print(solution.isRectangleOverlap([0,0,1,1], [1,0,2,1]))  # Output: False
print(solution.isRectangleOverlap([0,0,1,1], [2,2,3,3]))  # Output: False
print(solution.isRectangleOverlap([0,0,5,5], [3,3,7,7]))  # Output: True
print(solution.isRectangleOverlap([0,0,0,0], [0,0,0,0]))  # Output: False
\end{lstlisting}
\end{fullwidth}

This implementation efficiently checks for overlap by comparing the coordinates of the two rectangles. If any of the non-overlapping conditions are met, it returns \texttt{False}; otherwise, it returns \texttt{True}.

\section*{Explanation}

The \texttt{isRectangleOverlap} function determines whether two axis-aligned rectangles overlap by comparing their respective coordinates. Here's a detailed breakdown of the implementation:

\subsection*{1. Extract Coordinates}

\begin{itemize}
    \item For each rectangle, extract the left (\(x1\)), bottom (\(y1\)), right (\(x2\)), and top (\(y2\)) coordinates.
    \item This simplifies the comparison process by providing clear variables representing each side of the rectangles.
\end{itemize}

\subsection*{2. Check Non-Overlap Conditions}

\begin{itemize}
    \item **Horizontal Separation:**
    \begin{itemize}
        \item If the right side of the first rectangle (\(right1\)) is less than or equal to the left side of the second rectangle (\(left2\)), there is no horizontal overlap.
        \item Similarly, if the right side of the second rectangle (\(right2\)) is less than or equal to the left side of the first rectangle (\(left1\)), there is no horizontal overlap.
    \end{itemize}
    
    \item **Vertical Separation:**
    \begin{itemize}
        \item If the top side of the first rectangle (\(top1\)) is less than or equal to the bottom side of the second rectangle (\(bottom2\)), there is no vertical overlap.
        \item Similarly, if the top side of the second rectangle (\(top2\)) is less than or equal to the bottom side of the first rectangle (\(bottom1\)), there is no vertical overlap.
    \end{itemize}
    
    \item If any of these non-overlapping conditions are true, the rectangles do not overlap, and the function returns \texttt{False}.
\end{itemize}

\subsection*{3. Determine Overlap}

\begin{itemize}
    \item If none of the non-overlapping conditions are met, it implies that the rectangles overlap both horizontally and vertically.
    \item The function returns \texttt{True} in this case.
\end{itemize}

\subsection*{4. Example Walkthrough}

Consider the first example:
\begin{verbatim}
Input: rec1 = [0,0,2,2], rec2 = [1,1,3,3]
Output: True
\end{verbatim}

\begin{enumerate}
    \item Extract coordinates:
    \begin{itemize}
        \item rec1: left1 = 0, bottom1 = 0, right1 = 2, top1 = 2
        \item rec2: left2 = 1, bottom2 = 1, right2 = 3, top2 = 3
    \end{itemize}
    
    \item Check non-overlap conditions:
    \begin{itemize}
        \item \(right1 = 2\) is not less than or equal to \(left2 = 1\)
        \item \(right2 = 3\) is not less than or equal to \(left1 = 0\)
        \item \(top1 = 2\) is not less than or equal to \(bottom2 = 1\)
        \item \(top2 = 3\) is not less than or equal to \(bottom1 = 0\)
    \end{itemize}
    
    \item Since none of the non-overlapping conditions are met, the rectangles overlap.
\end{enumerate}

Thus, the function correctly returns \texttt{True}.

\section*{Why This Approach}

This approach is chosen for its simplicity and efficiency. By leveraging direct coordinate comparisons, the algorithm achieves constant time complexity without the need for complex data structures or iterative processes. It effectively handles all possible scenarios of rectangle positioning, ensuring accurate detection of overlaps.

\section*{Alternative Approaches}

\subsection*{1. Separating Axis Theorem (SAT)}

The Separating Axis Theorem is a more generalized method for detecting overlaps between convex shapes. While it is not necessary for axis-aligned rectangles, understanding SAT can be beneficial for more complex geometric problems.

\begin{lstlisting}[language=Python]
def isRectangleOverlap(rec1: List[int], rec2: List[int]) -> bool:
    # Using SAT for axis-aligned rectangles
    return not (rec1[2] <= rec2[0] or rec1[0] >= rec2[2] or
                rec1[3] <= rec2[1] or rec1[1] >= rec2[3])
\end{lstlisting}

\textbf{Note}: This implementation is functionally identical to the primary approach but leverages a more generalized geometric theorem.

\subsection*{2. Area-Based Approach}

Calculate the overlapping area between the two rectangles. If the overlapping area is positive, the rectangles overlap.

\begin{lstlisting}[language=Python]
def isRectangleOverlap(rec1: List[int], rec2: List[int]) -> bool:
    # Calculate overlap in x and y dimensions
    x_overlap = min(rec1[2], rec2[2]) - max(rec1[0], rec2[0])
    y_overlap = min(rec1[3], rec2[3]) - max(rec1[1], rec2[1])
    
    # Overlap exists if both overlaps are positive
    return x_overlap > 0 and y_overlap > 0
\end{lstlisting}

\textbf{Complexities:}
\begin{itemize}
    \item \textbf{Time Complexity:} \(O(1)\)
    \item \textbf{Space Complexity:} \(O(1)\)
\end{itemize}

\subsection*{3. Using Rectangles Intersection Function}

Utilize built-in or library functions that handle geometric intersections.

\begin{lstlisting}[language=Python]
from shapely.geometry import box

def isRectangleOverlap(rec1: List[int], rec2: List[int]) -> bool:
    rectangle1 = box(rec1[0], rec1[1], rec1[2], rec1[3])
    rectangle2 = box(rec2[0], rec2[1], rec2[2], rec2[3])
    return rectangle1.intersects(rectangle2) and not rectangle1.touches(rectangle2)
\end{lstlisting}

\textbf{Note}: This approach requires the \texttt{shapely} library and is more suitable for complex geometric operations.

\section*{Similar Problems to This One}

Several problems revolve around geometric overlap, intersection detection, and spatial reasoning, utilizing similar algorithmic strategies:

\begin{itemize}
    \item \textbf{Interval Overlap}: Determine if two intervals on a line overlap.
    \item \textbf{Circle Overlap}: Determine if two circles overlap based on their radii and centers.
    \item \textbf{Polygon Overlap}: Determine if two polygons overlap using algorithms like SAT.
    \item \textbf{Closest Pair of Points}: Find the closest pair of points in a set.
    \item \textbf{Convex Hull}: Compute the convex hull of a set of points.
    \item \textbf{Intersection of Lines}: Find the intersection point of two lines.
    \item \textbf{Point Inside Polygon}: Determine if a point lies inside a given polygon.
\end{itemize}

These problems reinforce the concepts of spatial reasoning, geometric property analysis, and efficient algorithm design in various contexts.

\section*{Things to Keep in Mind and Tricks}

When working with the \textbf{Rectangle Overlap} problem, consider the following tips and best practices to enhance efficiency and correctness:

\begin{itemize}
    \item \textbf{Understand Geometric Relationships}: Grasp the positional relationships between rectangles to simplify overlap detection.
    \index{Geometric Relationships}
    
    \item \textbf{Leverage Coordinate Comparisons}: Use direct comparisons of rectangle coordinates to determine spatial relationships.
    \index{Coordinate Comparisons}
    
    \item \textbf{Handle Edge Cases}: Consider cases where rectangles touch at edges or corners without overlapping.
    \index{Edge Cases}
    
    \item \textbf{Optimize for Efficiency}: Aim for a constant time \(O(1)\) solution by avoiding unnecessary computations or iterations.
    \index{Efficiency Optimization}
    
    \item \textbf{Avoid Floating-Point Precision Issues}: Since all coordinates are integers, floating-point precision is not a concern, simplifying the implementation.
    \index{Floating-Point Precision}
    
    \item \textbf{Use Helper Functions}: Create helper functions to encapsulate repetitive tasks, such as extracting coordinates or checking specific conditions.
    \index{Helper Functions}
    
    \item \textbf{Code Readability}: Maintain clear and readable code through meaningful variable names and structured logic.
    \index{Code Readability}
    
    \item \textbf{Test Extensively}: Implement a wide range of test cases, including overlapping, non-overlapping, and edge-touching rectangles, to ensure robustness.
    \index{Extensive Testing}
    
    \item \textbf{Understand Axis-Aligned Constraints}: Recognize that axis-aligned rectangles simplify overlap detection compared to rotated rectangles.
    \index{Axis-Aligned Constraints}
    
    \item \textbf{Simplify Logical Conditions}: Combine multiple conditions logically to streamline the overlap detection process.
    \index{Logical Conditions}
\end{itemize}

\section*{Corner and Special Cases to Test When Writing the Code}

When implementing the solution for the \textbf{Rectangle Overlap} problem, it is crucial to consider and rigorously test various edge cases to ensure robustness and correctness:

\begin{itemize}
    \item \textbf{No Overlap}: Rectangles are completely separate.
    \index{No Overlap}
    
    \item \textbf{Partial Overlap}: Rectangles overlap in one or more regions.
    \index{Partial Overlap}
    
    \item \textbf{Edge Touching}: Rectangles touch exactly at one edge without overlapping.
    \index{Edge Touching}
    
    \item \textbf{Corner Touching}: Rectangles touch exactly at one corner without overlapping.
    \index{Corner Touching}
    
    \item \textbf{One Rectangle Inside Another}: One rectangle is entirely within the other.
    \index{Rectangle Inside}
    
    \item \textbf{Identical Rectangles}: Both rectangles have the same coordinates.
    \index{Identical Rectangles}
    
    \item \textbf{Degenerate Rectangles}: Rectangles with zero area (e.g., \(x1 = x2\) or \(y1 = y2\)).
    \index{Degenerate Rectangles}
    
    \item \textbf{Large Coordinates}: Rectangles with very large coordinate values to test performance and integer handling.
    \index{Large Coordinates}
    
    \item \textbf{Negative Coordinates}: Rectangles positioned in negative coordinate space.
    \index{Negative Coordinates}
    
    \item \textbf{Mixed Overlapping Scenarios}: Combinations of the above cases to ensure comprehensive coverage.
    \index{Mixed Overlapping Scenarios}
    
    \item \textbf{Minimum and Maximum Bounds}: Rectangles at the minimum and maximum limits of the coordinate range.
    \index{Minimum and Maximum Bounds}
\end{itemize}

\section*{Implementation Considerations}

When implementing the \texttt{isRectangleOverlap} function, keep in mind the following considerations to ensure robustness and efficiency:

\begin{itemize}
    \item \textbf{Data Type Selection}: Use appropriate data types that can handle the range of input values without overflow or underflow.
    \index{Data Type Selection}
    
    \item \textbf{Optimizing Comparisons}: Structure logical conditions to short-circuit evaluations as soon as a non-overlapping condition is met.
    \index{Optimizing Comparisons}
    
    \item \textbf{Language-Specific Constraints}: Be aware of how the programming language handles integer division and comparisons.
    \index{Language-Specific Constraints}
    
    \item \textbf{Avoiding Redundant Calculations}: Ensure that each comparison contributes towards determining overlap without unnecessary repetitions.
    \index{Avoiding Redundant Calculations}
    
    \item \textbf{Code Readability and Documentation}: Maintain clear and readable code through meaningful variable names and comprehensive comments to facilitate understanding and maintenance.
    \index{Code Readability}
    
    \item \textbf{Edge Case Handling}: Implement checks for edge cases to prevent incorrect results or runtime errors.
    \index{Edge Case Handling}
    
    \item \textbf{Testing and Validation}: Develop a comprehensive suite of test cases that cover all possible scenarios, including edge cases, to validate the correctness and efficiency of the implementation.
    \index{Testing and Validation}
    
    \item \textbf{Scalability}: Design the algorithm to scale efficiently with increasing input sizes, maintaining performance and resource utilization.
    \index{Scalability}
    
    \item \textbf{Using Helper Functions}: Consider creating helper functions for repetitive tasks, such as extracting and comparing coordinates, to enhance modularity and reusability.
    \index{Helper Functions}
    
    \item \textbf{Consistent Naming Conventions}: Use consistent and descriptive naming conventions for variables to improve code clarity.
    \index{Naming Conventions}
    
    \item \textbf{Handling Floating-Point Coordinates}: Although the problem specifies integer coordinates, ensure that the implementation can handle floating-point numbers if needed in extended scenarios.
    \index{Floating-Point Coordinates}
    
    \item \textbf{Avoiding Floating-Point Precision Issues}: Since all coordinates are integers, floating-point precision is not a concern, simplifying the implementation.
    \index{Floating-Point Precision}
    
    \item \textbf{Implementing Unit Tests}: Develop unit tests for each logical condition to ensure that all scenarios are correctly handled.
    \index{Unit Tests}
    
    \item \textbf{Error Handling}: Incorporate error handling to manage invalid inputs gracefully.
    \index{Error Handling}
\end{itemize}

\section*{Conclusion}

The \textbf{Rectangle Overlap} problem exemplifies the application of fundamental geometric principles and conditional logic to solve spatial challenges efficiently. By leveraging simple coordinate comparisons, the algorithm achieves optimal time and space complexities, making it highly suitable for real-time applications such as collision detection in gaming, layout planning in graphics, and spatial data analysis. Understanding and implementing such techniques not only enhances problem-solving skills but also provides a foundation for tackling more complex Computational Geometry problems involving varied geometric shapes and interactions.

\printindex

% \input{sections/rectangle_overlap}
% \input{sections/rectangle_area}
% \input{sections/k_closest_points_to_origin}
% \input{sections/the_skyline_problem}
% % filename: rectangle_area.tex

\problemsection{Rectangle Area}
\label{chap:Rectangle_Area}
\marginnote{\href{https://leetcode.com/problems/rectangle-area/}{[LeetCode Link]}\index{LeetCode}}
\marginnote{\href{https://www.geeksforgeeks.org/find-area-two-overlapping-rectangles/}{[GeeksForGeeks Link]}\index{GeeksForGeeks}}
\marginnote{\href{https://www.interviewbit.com/problems/rectangle-area/}{[InterviewBit Link]}\index{InterviewBit}}
\marginnote{\href{https://app.codesignal.com/challenges/rectangle-area}{[CodeSignal Link]}\index{CodeSignal}}
\marginnote{\href{https://www.codewars.com/kata/rectangle-area/train/python}{[Codewars Link]}\index{Codewars}}

The \textbf{Rectangle Area} problem is a classic Computational Geometry challenge that involves calculating the total area covered by two axis-aligned rectangles in a 2D plane. This problem tests one's ability to perform geometric calculations, handle overlapping scenarios, and implement efficient algorithms. Mastery of this problem is essential for applications in computer graphics, spatial analysis, and computational modeling.

\section*{Problem Statement}

Given two axis-aligned rectangles in a 2D plane, compute the total area covered by the two rectangles. The area covered by the overlapping region should be counted only once.

Each rectangle is represented as a list of four integers \([x1, y1, x2, y2]\), where \((x1, y1)\) are the coordinates of the bottom-left corner, and \((x2, y2)\) are the coordinates of the top-right corner.

\textbf{Function signature in Python:}
\begin{lstlisting}[language=Python]
def computeArea(A: List[int], B: List[int]) -> int:
\end{lstlisting}

\section*{Examples}

\textbf{Example 1:}

\begin{verbatim}
Input: A = [-3,0,3,4], B = [0,-1,9,2]
Output: 45
Explanation:
Area of A = (3 - (-3)) * (4 - 0) = 6 * 4 = 24
Area of B = (9 - 0) * (2 - (-1)) = 9 * 3 = 27
Overlapping Area = (3 - 0) * (2 - 0) = 3 * 2 = 6
Total Area = 24 + 27 - 6 = 45
\end{verbatim}

\textbf{Example 2:}

\begin{verbatim}
Input: A = [0,0,0,0], B = [0,0,0,0]
Output: 0
Explanation:
Both rectangles are degenerate points with zero area.
\end{verbatim}

\textbf{Example 3:}

\begin{verbatim}
Input: A = [0,0,2,2], B = [1,1,3,3]
Output: 7
Explanation:
Area of A = 4
Area of B = 4
Overlapping Area = 1
Total Area = 4 + 4 - 1 = 7
\end{verbatim}

\textbf{Example 4:}

\begin{verbatim}
Input: A = [0,0,1,1], B = [1,0,2,1]
Output: 2
Explanation:
Rectangles touch at the edge but do not overlap.
Area of A = 1
Area of B = 1
Overlapping Area = 0
Total Area = 1 + 1 = 2
\end{verbatim}

\textbf{Constraints:}

\begin{itemize}
    \item All coordinates are integers in the range \([-10^9, 10^9]\).
    \item For each rectangle, \(x1 < x2\) and \(y1 < y2\).
\end{itemize}

LeetCode link: \href{https://leetcode.com/problems/rectangle-area/}{Rectangle Area}\index{LeetCode}

\section*{Algorithmic Approach}

To compute the total area covered by two axis-aligned rectangles, we can follow these steps:

1. **Calculate Individual Areas:**
   - Compute the area of the first rectangle.
   - Compute the area of the second rectangle.

2. **Determine Overlapping Area:**
   - Calculate the coordinates of the overlapping rectangle, if any.
   - If the rectangles overlap, compute the area of the overlapping region.

3. **Compute Total Area:**
   - Sum the individual areas and subtract the overlapping area to avoid double-counting.

\marginnote{This approach ensures accurate area calculation by handling overlapping regions appropriately.}

\section*{Complexities}

\begin{itemize}
    \item \textbf{Time Complexity:} \(O(1)\). The algorithm performs a constant number of calculations.
    
    \item \textbf{Space Complexity:} \(O(1)\). Only a fixed amount of extra space is used for variables.
\end{itemize}

\section*{Python Implementation}

\marginnote{Implementing the area calculation with overlap consideration ensures an accurate and efficient solution.}

Below is the complete Python code implementing the \texttt{computeArea} function:

\begin{fullwidth}
\begin{lstlisting}[language=Python]
from typing import List

class Solution:
    def computeArea(self, A: List[int], B: List[int]) -> int:
        # Calculate area of rectangle A
        areaA = (A[2] - A[0]) * (A[3] - A[1])
        
        # Calculate area of rectangle B
        areaB = (B[2] - B[0]) * (B[3] - B[1])
        
        # Determine overlap coordinates
        overlap_x1 = max(A[0], B[0])
        overlap_y1 = max(A[1], B[1])
        overlap_x2 = min(A[2], B[2])
        overlap_y2 = min(A[3], B[3])
        
        # Calculate overlapping area
        overlap_width = overlap_x2 - overlap_x1
        overlap_height = overlap_y2 - overlap_y1
        overlap_area = 0
        if overlap_width > 0 and overlap_height > 0:
            overlap_area = overlap_width * overlap_height
        
        # Total area is sum of individual areas minus overlapping area
        total_area = areaA + areaB - overlap_area
        return total_area

# Example usage:
solution = Solution()
print(solution.computeArea([-3,0,3,4], [0,-1,9,2]))  # Output: 45
print(solution.computeArea([0,0,0,0], [0,0,0,0]))    # Output: 0
print(solution.computeArea([0,0,2,2], [1,1,3,3]))    # Output: 7
print(solution.computeArea([0,0,1,1], [1,0,2,1]))    # Output: 2
\end{lstlisting}
\end{fullwidth}

This implementation accurately computes the total area covered by two rectangles by accounting for any overlapping regions. It ensures that the overlapping area is not double-counted.

\section*{Explanation}

The \texttt{computeArea} function calculates the combined area of two axis-aligned rectangles by following these steps:

\subsection*{1. Calculate Individual Areas}

\begin{itemize}
    \item **Rectangle A:**
    \begin{itemize}
        \item Width: \(A[2] - A[0]\)
        \item Height: \(A[3] - A[1]\)
        \item Area: Width \(\times\) Height
    \end{itemize}
    
    \item **Rectangle B:**
    \begin{itemize}
        \item Width: \(B[2] - B[0]\)
        \item Height: \(B[3] - B[1]\)
        \item Area: Width \(\times\) Height
    \end{itemize}
\end{itemize}

\subsection*{2. Determine Overlapping Area}

\begin{itemize}
    \item **Overlap Coordinates:**
    \begin{itemize}
        \item Left (x-coordinate): \(\text{max}(A[0], B[0])\)
        \item Bottom (y-coordinate): \(\text{max}(A[1], B[1])\)
        \item Right (x-coordinate): \(\text{min}(A[2], B[2])\)
        \item Top (y-coordinate): \(\text{min}(A[3], B[3])\)
    \end{itemize}
    
    \item **Overlap Dimensions:**
    \begin{itemize}
        \item Width: \(\text{overlap\_x2} - \text{overlap\_x1}\)
        \item Height: \(\text{overlap\_y2} - \text{overlap\_y1}\)
    \end{itemize}
    
    \item **Overlap Area:**
    \begin{itemize}
        \item If both width and height are positive, the rectangles overlap, and the overlapping area is their product.
        \item Otherwise, there is no overlap, and the overlapping area is zero.
    \end{itemize}
\end{itemize}

\subsection*{3. Compute Total Area}

\begin{itemize}
    \item Total Area = Area of Rectangle A + Area of Rectangle B - Overlapping Area
\end{itemize}

\subsection*{4. Example Walkthrough}

Consider the first example:
\begin{verbatim}
Input: A = [-3,0,3,4], B = [0,-1,9,2]
Output: 45
\end{verbatim}

\begin{enumerate}
    \item **Calculate Areas:**
    \begin{itemize}
        \item Area of A = (3 - (-3)) * (4 - 0) = 6 * 4 = 24
        \item Area of B = (9 - 0) * (2 - (-1)) = 9 * 3 = 27
    \end{itemize}
    
    \item **Determine Overlap:**
    \begin{itemize}
        \item overlap\_x1 = max(-3, 0) = 0
        \item overlap\_y1 = max(0, -1) = 0
        \item overlap\_x2 = min(3, 9) = 3
        \item overlap\_y2 = min(4, 2) = 2
        \item overlap\_width = 3 - 0 = 3
        \item overlap\_height = 2 - 0 = 2
        \item overlap\_area = 3 * 2 = 6
    \end{itemize}
    
    \item **Compute Total Area:**
    \begin{itemize}
        \item Total Area = 24 + 27 - 6 = 45
    \end{itemize}
\end{enumerate}

Thus, the function correctly returns \texttt{45}.

\section*{Why This Approach}

This approach is chosen for its straightforwardness and optimal efficiency. By directly calculating the individual areas and intelligently handling the overlapping region, the algorithm ensures accurate results without unnecessary computations. Its constant time complexity makes it highly efficient, even for large coordinate values.

\section*{Alternative Approaches}

\subsection*{1. Using Intersection Dimensions}

Instead of separately calculating areas, directly compute the dimensions of the overlapping region and subtract it from the sum of individual areas.

\begin{lstlisting}[language=Python]
def computeArea(A: List[int], B: List[int]) -> int:
    # Sum of individual areas
    area = (A[2] - A[0]) * (A[3] - A[1]) + (B[2] - B[0]) * (B[3] - B[1])
    
    # Overlapping area
    overlap_width = min(A[2], B[2]) - max(A[0], B[0])
    overlap_height = min(A[3], B[3]) - max(A[1], B[1])
    
    if overlap_width > 0 and overlap_height > 0:
        area -= overlap_width * overlap_height
    
    return area
\end{lstlisting}

\subsection*{2. Using Geometry Libraries}

Leverage computational geometry libraries to handle area calculations and overlapping detections.

\begin{lstlisting}[language=Python]
from shapely.geometry import box

def computeArea(A: List[int], B: List[int]) -> int:
    rect1 = box(A[0], A[1], A[2], A[3])
    rect2 = box(B[0], B[1], B[2], B[3])
    intersection = rect1.intersection(rect2)
    return int(rect1.area + rect2.area - intersection.area)
\end{lstlisting}

\textbf{Note}: This approach requires the \texttt{shapely} library and is more suitable for complex geometric operations.

\section*{Similar Problems to This One}

Several problems involve calculating areas, handling geometric overlaps, and spatial reasoning, utilizing similar algorithmic strategies:

\begin{itemize}
    \item \textbf{Rectangle Overlap}: Determine if two rectangles overlap.
    \item \textbf{Circle Area Overlap}: Calculate the overlapping area between two circles.
    \item \textbf{Polygon Area}: Compute the area of a given polygon.
    \item \textbf{Union of Rectangles}: Calculate the total area covered by multiple rectangles, accounting for overlaps.
    \item \textbf{Intersection of Lines}: Find the intersection point of two lines.
    \item \textbf{Closest Pair of Points}: Find the closest pair of points in a set.
    \item \textbf{Convex Hull}: Compute the convex hull of a set of points.
    \item \textbf{Point Inside Polygon}: Determine if a point lies inside a given polygon.
\end{itemize}

These problems reinforce concepts of geometric calculations, area computations, and efficient algorithm design in various contexts.

\section*{Things to Keep in Mind and Tricks}

When tackling the \textbf{Rectangle Area} problem, consider the following tips and best practices to enhance efficiency and correctness:

\begin{itemize}
    \item \textbf{Understand Geometric Relationships}: Grasp the positional relationships between rectangles to simplify area calculations.
    \index{Geometric Relationships}
    
    \item \textbf{Leverage Coordinate Comparisons}: Use direct comparisons of rectangle coordinates to determine overlapping regions.
    \index{Coordinate Comparisons}
    
    \item \textbf{Handle Overlapping Scenarios}: Accurately calculate the overlapping area to avoid double-counting.
    \index{Overlapping Scenarios}
    
    \item \textbf{Optimize for Efficiency}: Aim for a constant time \(O(1)\) solution by avoiding unnecessary computations or iterations.
    \index{Efficiency Optimization}
    
    \item \textbf{Avoid Floating-Point Precision Issues}: Since all coordinates are integers, floating-point precision is not a concern, simplifying the implementation.
    \index{Floating-Point Precision}
    
    \item \textbf{Use Helper Functions}: Create helper functions to encapsulate repetitive tasks, such as calculating overlap dimensions or areas.
    \index{Helper Functions}
    
    \item \textbf{Code Readability}: Maintain clear and readable code through meaningful variable names and structured logic.
    \index{Code Readability}
    
    \item \textbf{Test Extensively}: Implement a wide range of test cases, including overlapping, non-overlapping, and edge-touching rectangles, to ensure robustness.
    \index{Extensive Testing}
    
    \item \textbf{Understand Axis-Aligned Constraints}: Recognize that axis-aligned rectangles simplify area calculations compared to rotated rectangles.
    \index{Axis-Aligned Constraints}
    
    \item \textbf{Simplify Logical Conditions}: Combine multiple conditions logically to streamline the area calculation process.
    \index{Logical Conditions}
    
    \item \textbf{Use Absolute Values}: When calculating differences, ensure that the dimensions are positive by using absolute values or proper ordering.
    \index{Absolute Values}
    
    \item \textbf{Consider Edge Cases}: Handle cases where rectangles have zero area or touch at edges/corners without overlapping.
    \index{Edge Cases}
\end{itemize}

\section*{Corner and Special Cases to Test When Writing the Code}

When implementing the solution for the \textbf{Rectangle Area} problem, it is crucial to consider and rigorously test various edge cases to ensure robustness and correctness:

\begin{itemize}
    \item \textbf{No Overlap}: Rectangles are completely separate.
    \index{No Overlap}
    
    \item \textbf{Partial Overlap}: Rectangles overlap in one or more regions.
    \index{Partial Overlap}
    
    \item \textbf{Edge Touching}: Rectangles touch exactly at one edge without overlapping.
    \index{Edge Touching}
    
    \item \textbf{Corner Touching}: Rectangles touch exactly at one corner without overlapping.
    \index{Corner Touching}
    
    \item \textbf{One Rectangle Inside Another}: One rectangle is entirely within the other.
    \index{Rectangle Inside}
    
    \item \textbf{Identical Rectangles}: Both rectangles have the same coordinates.
    \index{Identical Rectangles}
    
    \item \textbf{Degenerate Rectangles}: Rectangles with zero area (e.g., \(x1 = x2\) or \(y1 = y2\)).
    \index{Degenerate Rectangles}
    
    \item \textbf{Large Coordinates}: Rectangles with very large coordinate values to test performance and integer handling.
    \index{Large Coordinates}
    
    \item \textbf{Negative Coordinates}: Rectangles positioned in negative coordinate space.
    \index{Negative Coordinates}
    
    \item \textbf{Mixed Overlapping Scenarios}: Combinations of the above cases to ensure comprehensive coverage.
    \index{Mixed Overlapping Scenarios}
    
    \item \textbf{Minimum and Maximum Bounds}: Rectangles at the minimum and maximum limits of the coordinate range.
    \index{Minimum and Maximum Bounds}
    
    \item \textbf{Sequential Rectangles}: Multiple rectangles placed sequentially without overlapping.
    \index{Sequential Rectangles}
    
    \item \textbf{Multiple Overlaps}: Scenarios where more than two rectangles overlap in different regions.
    \index{Multiple Overlaps}
\end{itemize}

\section*{Implementation Considerations}

When implementing the \texttt{computeArea} function, keep in mind the following considerations to ensure robustness and efficiency:

\begin{itemize}
    \item \textbf{Data Type Selection}: Use appropriate data types that can handle large input values without overflow or underflow.
    \index{Data Type Selection}
    
    \item \textbf{Optimizing Comparisons}: Structure logical conditions to efficiently determine overlap dimensions.
    \index{Optimizing Comparisons}
    
    \item \textbf{Handling Large Inputs}: Design the algorithm to efficiently handle large input sizes without significant performance degradation.
    \index{Handling Large Inputs}
    
    \item \textbf{Language-Specific Constraints}: Be aware of how the programming language handles large integers and arithmetic operations.
    \index{Language-Specific Constraints}
    
    \item \textbf{Avoiding Redundant Calculations}: Ensure that each calculation contributes towards determining the final area without unnecessary repetitions.
    \index{Avoiding Redundant Calculations}
    
    \item \textbf{Code Readability and Documentation}: Maintain clear and readable code through meaningful variable names and comprehensive comments to facilitate understanding and maintenance.
    \index{Code Readability}
    
    \item \textbf{Edge Case Handling}: Implement checks for edge cases to prevent incorrect results or runtime errors.
    \index{Edge Case Handling}
    
    \item \textbf{Testing and Validation}: Develop a comprehensive suite of test cases that cover all possible scenarios, including edge cases, to validate the correctness and efficiency of the implementation.
    \index{Testing and Validation}
    
    \item \textbf{Scalability}: Design the algorithm to scale efficiently with increasing input sizes, maintaining performance and resource utilization.
    \index{Scalability}
    
    \item \textbf{Using Helper Functions}: Consider creating helper functions for repetitive tasks, such as calculating overlap dimensions, to enhance modularity and reusability.
    \index{Helper Functions}
    
    \item \textbf{Consistent Naming Conventions}: Use consistent and descriptive naming conventions for variables to improve code clarity.
    \index{Naming Conventions}
    
    \item \textbf{Implementing Unit Tests}: Develop unit tests for each logical condition to ensure that all scenarios are correctly handled.
    \index{Unit Tests}
    
    \item \textbf{Error Handling}: Incorporate error handling to manage invalid inputs gracefully.
    \index{Error Handling}
\end{itemize}

\section*{Conclusion}

The \textbf{Rectangle Area} problem showcases the application of fundamental geometric principles and efficient algorithm design to compute spatial properties accurately. By systematically calculating individual areas and intelligently handling overlapping regions, the algorithm ensures precise results without redundant computations. Understanding and implementing such techniques not only enhances problem-solving skills but also provides a foundation for tackling more complex Computational Geometry challenges involving multiple geometric entities and intricate spatial relationships.

\printindex

% \input{sections/rectangle_overlap}
% \input{sections/rectangle_area}
% \input{sections/k_closest_points_to_origin}
% \input{sections/the_skyline_problem}
% % filename: k_closest_points_to_origin.tex

\problemsection{K Closest Points to Origin}
\label{chap:K_Closest_Points_to_Origin}
\marginnote{\href{https://leetcode.com/problems/k-closest-points-to-origin/}{[LeetCode Link]}\index{LeetCode}}
\marginnote{\href{https://www.geeksforgeeks.org/find-k-closest-points-origin/}{[GeeksForGeeks Link]}\index{GeeksForGeeks}}
\marginnote{\href{https://www.interviewbit.com/problems/k-closest-points/}{[InterviewBit Link]}\index{InterviewBit}}
\marginnote{\href{https://app.codesignal.com/challenges/k-closest-points-to-origin}{[CodeSignal Link]}\index{CodeSignal}}
\marginnote{\href{https://www.codewars.com/kata/k-closest-points-to-origin/train/python}{[Codewars Link]}\index{Codewars}}

The \textbf{K Closest Points to Origin} problem is a popular algorithmic challenge in Computational Geometry that involves identifying the \(k\) points closest to the origin in a 2D plane. This problem tests one's ability to apply efficient sorting and selection algorithms, understand distance computations, and optimize for performance. Mastery of this problem is essential for applications in spatial data analysis, nearest neighbor searches, and clustering algorithms.

\section*{Problem Statement}

Given an array of points where each point is represented as \([x, y]\) in the 2D plane, and an integer \(k\), return the \(k\) closest points to the origin \((0, 0)\).

The distance between two points \((x_1, y_1)\) and \((x_2, y_2)\) is the Euclidean distance \(\sqrt{(x_1 - x_2)^2 + (y_1 - y_2)^2}\). The origin is \((0, 0)\).

\textbf{Function signature in Python:}
\begin{lstlisting}[language=Python]
def kClosest(points: List[List[int]], K: int) -> List[List[int]]:
\end{lstlisting}

\section*{Examples}

\textbf{Example 1:}

\begin{verbatim}
Input: points = [[1,3],[-2,2]], K = 1
Output: [[-2,2]]
Explanation: 
The distance between (1, 3) and the origin is sqrt(10).
The distance between (-2, 2) and the origin is sqrt(8).
Since sqrt(8) < sqrt(10), (-2, 2) is closer to the origin.
\end{verbatim}

\textbf{Example 2:}

\begin{verbatim}
Input: points = [[3,3],[5,-1],[-2,4]], K = 2
Output: [[3,3],[-2,4]]
Explanation: 
The distances are sqrt(18), sqrt(26), and sqrt(20) respectively.
The two closest points are [3,3] and [-2,4].
\end{verbatim}

\textbf{Example 3:}

\begin{verbatim}
Input: points = [[0,1],[1,0]], K = 2
Output: [[0,1],[1,0]]
Explanation: 
Both points are equally close to the origin.
\end{verbatim}

\textbf{Example 4:}

\begin{verbatim}
Input: points = [[1,0],[0,1]], K = 1
Output: [[1,0]]
Explanation: 
Both points are equally close; returning any one is acceptable.
\end{verbatim}

\textbf{Constraints:}

\begin{itemize}
    \item \(1 \leq K \leq \text{points.length} \leq 10^4\)
    \item \(-10^4 < x_i, y_i < 10^4\)
\end{itemize}

LeetCode link: \href{https://leetcode.com/problems/k-closest-points-to-origin/}{K Closest Points to Origin}\index{LeetCode}

\section*{Algorithmic Approach}

To identify the \(k\) closest points to the origin, several algorithmic strategies can be employed. The most efficient methods aim to reduce the time complexity by avoiding the need to sort the entire list of points.

\subsection*{1. Sorting Based on Distance}

Calculate the Euclidean distance of each point from the origin and sort the points based on these distances. Select the first \(k\) points from the sorted list.

\begin{enumerate}
    \item Compute the distance for each point using the formula \(distance = x^2 + y^2\).
    \item Sort the points based on the computed distances.
    \item Return the first \(k\) points from the sorted list.
\end{enumerate}

\subsection*{2. Max Heap (Priority Queue)}

Use a max heap to maintain the \(k\) closest points. Iterate through each point, add it to the heap, and if the heap size exceeds \(k\), remove the farthest point.

\begin{enumerate}
    \item Initialize a max heap.
    \item For each point, compute its distance and add it to the heap.
    \item If the heap size exceeds \(k\), remove the point with the largest distance.
    \item After processing all points, the heap contains the \(k\) closest points.
\end{enumerate}

\subsection*{3. QuickSelect (Quick Sort Partitioning)}

Utilize the QuickSelect algorithm to find the \(k\) closest points without fully sorting the list.

\begin{enumerate}
    \item Choose a pivot point and partition the list based on distances relative to the pivot.
    \item Recursively apply QuickSelect to the partition containing the \(k\) closest points.
    \item Once the \(k\) closest points are identified, return them.
\end{enumerate}

\marginnote{QuickSelect offers an average time complexity of \(O(n)\), making it highly efficient for large datasets.}

\section*{Complexities}

\begin{itemize}
    \item \textbf{Sorting Based on Distance:}
    \begin{itemize}
        \item \textbf{Time Complexity:} \(O(n \log n)\)
        \item \textbf{Space Complexity:} \(O(n)\)
    \end{itemize}
    
    \item \textbf{Max Heap (Priority Queue):}
    \begin{itemize}
        \item \textbf{Time Complexity:} \(O(n \log k)\)
        \item \textbf{Space Complexity:} \(O(k)\)
    \end{itemize}
    
    \item \textbf{QuickSelect (Quick Sort Partitioning):}
    \begin{itemize}
        \item \textbf{Time Complexity:} Average case \(O(n)\), worst case \(O(n^2)\)
        \item \textbf{Space Complexity:} \(O(1)\) (in-place)
    \end{itemize}
\end{itemize}

\section*{Python Implementation}

\marginnote{Implementing QuickSelect provides an optimal average-case solution with linear time complexity.}

Below is the complete Python code implementing the \texttt{kClosest} function using the QuickSelect approach:

\begin{fullwidth}
\begin{lstlisting}[language=Python]
from typing import List
import random

class Solution:
    def kClosest(self, points: List[List[int]], K: int) -> List[List[int]]:
        def quickselect(left, right, K_smallest):
            if left == right:
                return
            
            # Select a random pivot_index
            pivot_index = random.randint(left, right)
            
            # Partition the array
            pivot_index = partition(left, right, pivot_index)
            
            # The pivot is in its final sorted position
            if K_smallest == pivot_index:
                return
            elif K_smallest < pivot_index:
                quickselect(left, pivot_index - 1, K_smallest)
            else:
                quickselect(pivot_index + 1, right, K_smallest)
        
        def partition(left, right, pivot_index):
            pivot_distance = distance(points[pivot_index])
            # Move pivot to end
            points[pivot_index], points[right] = points[right], points[pivot_index]
            store_index = left
            for i in range(left, right):
                if distance(points[i]) < pivot_distance:
                    points[store_index], points[i] = points[i], points[store_index]
                    store_index += 1
            # Move pivot to its final place
            points[right], points[store_index] = points[store_index], points[right]
            return store_index
        
        def distance(point):
            return point[0] ** 2 + point[1] ** 2
        
        n = len(points)
        quickselect(0, n - 1, K)
        return points[:K]

# Example usage:
solution = Solution()
print(solution.kClosest([[1,3],[-2,2]], 1))            # Output: [[-2,2]]
print(solution.kClosest([[3,3],[5,-1],[-2,4]], 2))     # Output: [[3,3],[-2,4]]
print(solution.kClosest([[0,1],[1,0]], 2))             # Output: [[0,1],[1,0]]
print(solution.kClosest([[1,0],[0,1]], 1))             # Output: [[1,0]] or [[0,1]]
\end{lstlisting}
\end{fullwidth}

This implementation uses the QuickSelect algorithm to efficiently find the \(k\) closest points to the origin without fully sorting the entire list. It ensures optimal performance even with large datasets.

\section*{Explanation}

The \texttt{kClosest} function identifies the \(k\) closest points to the origin using the QuickSelect algorithm. Here's a detailed breakdown of the implementation:

\subsection*{1. Distance Calculation}

\begin{itemize}
    \item The Euclidean distance is calculated as \(distance = x^2 + y^2\). Since we only need relative distances for comparison, the square root is omitted for efficiency.
\end{itemize}

\subsection*{2. QuickSelect Algorithm}

\begin{itemize}
    \item **Pivot Selection:**
    \begin{itemize}
        \item A random pivot is chosen to enhance the average-case performance.
    \end{itemize}
    
    \item **Partitioning:**
    \begin{itemize}
        \item The array is partitioned such that points with distances less than the pivot are moved to the left, and others to the right.
        \item The pivot is placed in its correct sorted position.
    \end{itemize}
    
    \item **Recursive Selection:**
    \begin{itemize}
        \item If the pivot's position matches \(K\), the selection is complete.
        \item Otherwise, recursively apply QuickSelect to the relevant partition.
    \end{itemize}
\end{itemize}

\subsection*{3. Final Selection}

\begin{itemize}
    \item After partitioning, the first \(K\) points in the list are the \(k\) closest points to the origin.
\end{itemize}

\subsection*{4. Example Walkthrough}

Consider the first example:
\begin{verbatim}
Input: points = [[1,3],[-2,2]], K = 1
Output: [[-2,2]]
\end{verbatim}

\begin{enumerate}
    \item **Calculate Distances:**
    \begin{itemize}
        \item [1,3] : \(1^2 + 3^2 = 10\)
        \item [-2,2] : \((-2)^2 + 2^2 = 8\)
    \end{itemize}
    
    \item **QuickSelect Process:**
    \begin{itemize}
        \item Choose a pivot, say [1,3] with distance 10.
        \item Compare and rearrange:
        \begin{itemize}
            \item [-2,2] has a smaller distance (8) and is moved to the left.
        \end{itemize}
        \item After partitioning, the list becomes [[-2,2], [1,3]].
        \item Since \(K = 1\), return the first point: [[-2,2]].
    \end{itemize}
\end{enumerate}

Thus, the function correctly identifies \([-2,2]\) as the closest point to the origin.

\section*{Why This Approach}

The QuickSelect algorithm is chosen for its average-case linear time complexity \(O(n)\), making it highly efficient for large datasets compared to sorting-based methods with \(O(n \log n)\) time complexity. By avoiding the need to sort the entire list, QuickSelect provides an optimal solution for finding the \(k\) closest points.

\section*{Alternative Approaches}

\subsection*{1. Sorting Based on Distance}

Sort all points based on their distances from the origin and select the first \(k\) points.

\begin{lstlisting}[language=Python]
class Solution:
    def kClosest(self, points: List[List[int]], K: int) -> List[List[int]]:
        points.sort(key=lambda P: P[0]**2 + P[1]**2)
        return points[:K]
\end{lstlisting}

\textbf{Complexities:}
\begin{itemize}
    \item \textbf{Time Complexity:} \(O(n \log n)\)
    \item \textbf{Space Complexity:} \(O(1)\)
\end{itemize}

\subsection*{2. Max Heap (Priority Queue)}

Use a max heap to maintain the \(k\) closest points.

\begin{lstlisting}[language=Python]
import heapq

class Solution:
    def kClosest(self, points: List[List[int]], K: int) -> List[List[int]]:
        heap = []
        for (x, y) in points:
            dist = -(x**2 + y**2)  # Max heap using negative distances
            heapq.heappush(heap, (dist, [x, y]))
            if len(heap) > K:
                heapq.heappop(heap)
        return [item[1] for item in heap]
\end{lstlisting}

\textbf{Complexities:}
\begin{itemize}
    \item \textbf{Time Complexity:} \(O(n \log k)\)
    \item \textbf{Space Complexity:} \(O(k)\)
\end{itemize}

\subsection*{3. Using Built-In Functions}

Leverage built-in functions for distance calculation and selection.

\begin{lstlisting}[language=Python]
import math

class Solution:
    def kClosest(self, points: List[List[int]], K: int) -> List[List[int]]:
        points.sort(key=lambda P: math.sqrt(P[0]**2 + P[1]**2))
        return points[:K]
\end{lstlisting}

\textbf{Note}: This method is similar to the sorting approach but uses the actual Euclidean distance.

\section*{Similar Problems to This One}

Several problems involve nearest neighbor searches, spatial data analysis, and efficient selection algorithms, utilizing similar algorithmic strategies:

\begin{itemize}
    \item \textbf{Closest Pair of Points}: Find the closest pair of points in a set.
    \item \textbf{Top K Frequent Elements}: Identify the most frequent elements in a dataset.
    \item \textbf{Kth Largest Element in an Array}: Find the \(k\)-th largest element in an unsorted array.
    \item \textbf{Sliding Window Maximum}: Find the maximum in each sliding window of size \(k\) over an array.
    \item \textbf{Merge K Sorted Lists}: Merge multiple sorted lists into a single sorted list.
    \item \textbf{Find Median from Data Stream}: Continuously find the median of a stream of numbers.
    \item \textbf{Top K Closest Stars}: Find the \(k\) closest stars to Earth based on their distances.
\end{itemize}

These problems reinforce concepts of efficient selection, heap usage, and distance computations in various contexts.

\section*{Things to Keep in Mind and Tricks}

When solving the \textbf{K Closest Points to Origin} problem, consider the following tips and best practices to enhance efficiency and correctness:

\begin{itemize}
    \item \textbf{Understand Distance Calculations}: Grasp the Euclidean distance formula and recognize that the square root can be omitted for comparison purposes.
    \index{Distance Calculations}
    
    \item \textbf{Leverage Efficient Algorithms}: Use QuickSelect or heap-based methods to optimize time complexity, especially for large datasets.
    \index{Efficient Algorithms}
    
    \item \textbf{Handle Ties Appropriately}: Decide how to handle points with identical distances when \(k\) is less than the number of such points.
    \index{Handling Ties}
    
    \item \textbf{Optimize Space Usage}: Choose algorithms that minimize additional space, such as in-place QuickSelect.
    \index{Space Optimization}
    
    \item \textbf{Use Appropriate Data Structures}: Utilize heaps, lists, and helper functions effectively to manage and process data.
    \index{Data Structures}
    
    \item \textbf{Implement Helper Functions}: Create helper functions for distance calculation and partitioning to enhance code modularity.
    \index{Helper Functions}
    
    \item \textbf{Code Readability}: Maintain clear and readable code through meaningful variable names and structured logic.
    \index{Code Readability}
    
    \item \textbf{Test Extensively}: Implement a wide range of test cases, including edge cases like multiple points with the same distance, to ensure robustness.
    \index{Extensive Testing}
    
    \item \textbf{Understand Algorithm Trade-offs}: Recognize the trade-offs between different approaches in terms of time and space complexities.
    \index{Algorithm Trade-offs}
    
    \item \textbf{Use Built-In Sorting Functions}: When using sorting-based approaches, leverage built-in functions for efficiency and simplicity.
    \index{Built-In Sorting}
    
    \item \textbf{Avoid Redundant Calculations}: Ensure that distance calculations are performed only when necessary to optimize performance.
    \index{Avoiding Redundant Calculations}
    
    \item \textbf{Language-Specific Features}: Utilize language-specific features or libraries that can simplify implementation, such as heapq in Python.
    \index{Language-Specific Features}
\end{itemize}

\section*{Corner and Special Cases to Test When Writing the Code}

When implementing the solution for the \textbf{K Closest Points to Origin} problem, it is crucial to consider and rigorously test various edge cases to ensure robustness and correctness:

\begin{itemize}
    \item \textbf{Multiple Points with Same Distance}: Ensure that the algorithm handles multiple points having the same distance from the origin.
    \index{Same Distance Points}
    
    \item \textbf{Points at Origin}: Include points that are exactly at the origin \((0,0)\).
    \index{Points at Origin}
    
    \item \textbf{Negative Coordinates}: Ensure that the algorithm correctly computes distances for points with negative \(x\) or \(y\) coordinates.
    \index{Negative Coordinates}
    
    \item \textbf{Large Coordinates}: Test with points having very large or very small coordinate values to verify integer handling.
    \index{Large Coordinates}
    
    \item \textbf{K Equals Number of Points}: When \(K\) is equal to the number of points, the algorithm should return all points.
    \index{K Equals Number of Points}
    
    \item \textbf{K is One}: Test with \(K = 1\) to ensure the closest point is correctly identified.
    \index{K is One}
    
    \item \textbf{All Points Same}: All points have the same coordinates.
    \index{All Points Same}
    
    \item \textbf{K is Zero}: Although \(K\) is defined to be at least 1, ensure that the algorithm gracefully handles \(K = 0\) if allowed.
    \index{K is Zero}
    
    \item \textbf{Single Point}: Only one point is provided, and \(K = 1\).
    \index{Single Point}
    
    \item \textbf{Mixed Coordinates}: Points with a mix of positive and negative coordinates.
    \index{Mixed Coordinates}
    
    \item \textbf{Points with Zero Distance}: Multiple points at the origin.
    \index{Zero Distance Points}
    
    \item \textbf{Sparse and Dense Points}: Densely packed points and sparsely distributed points.
    \index{Sparse and Dense Points}
    
    \item \textbf{Duplicate Points}: Multiple identical points in the input list.
    \index{Duplicate Points}
    
    \item \textbf{K Greater Than Number of Unique Points}: Ensure that the algorithm handles cases where \(K\) exceeds the number of unique points if applicable.
    \index{K Greater Than Unique Points}
\end{itemize}

\section*{Implementation Considerations}

When implementing the \texttt{kClosest} function, keep in mind the following considerations to ensure robustness and efficiency:

\begin{itemize}
    \item \textbf{Data Type Selection}: Use appropriate data types that can handle large input values without overflow or precision loss.
    \index{Data Type Selection}
    
    \item \textbf{Optimizing Distance Calculations}: Avoid calculating the square root since it is unnecessary for comparison purposes.
    \index{Optimizing Distance Calculations}
    
    \item \textbf{Choosing the Right Algorithm}: Select an algorithm based on the size of the input and the value of \(K\) to optimize time and space complexities.
    \index{Choosing the Right Algorithm}
    
    \item \textbf{Language-Specific Libraries}: Utilize language-specific libraries and functions (e.g., \texttt{heapq} in Python) to simplify implementation and enhance performance.
    \index{Language-Specific Libraries}
    
    \item \textbf{Avoiding Redundant Calculations}: Ensure that each point's distance is calculated only once to optimize performance.
    \index{Avoiding Redundant Calculations}
    
    \item \textbf{Implementing Helper Functions}: Create helper functions for tasks like distance calculation and partitioning to enhance modularity and readability.
    \index{Helper Functions}
    
    \item \textbf{Edge Case Handling}: Implement checks for edge cases to prevent incorrect results or runtime errors.
    \index{Edge Case Handling}
    
    \item \textbf{Testing and Validation}: Develop a comprehensive suite of test cases that cover all possible scenarios, including edge cases, to validate the correctness and efficiency of the implementation.
    \index{Testing and Validation}
    
    \item \textbf{Scalability}: Design the algorithm to scale efficiently with increasing input sizes, maintaining performance and resource utilization.
    \index{Scalability}
    
    \item \textbf{Consistent Naming Conventions}: Use consistent and descriptive naming conventions for variables and functions to improve code clarity.
    \index{Naming Conventions}
    
    \item \textbf{Memory Management}: Ensure that the algorithm manages memory efficiently, especially when dealing with large datasets.
    \index{Memory Management}
    
    \item \textbf{Avoiding Stack Overflow}: If implementing recursive approaches, be mindful of recursion limits and potential stack overflow issues.
    \index{Avoiding Stack Overflow}
    
    \item \textbf{Implementing Iterative Solutions}: Prefer iterative solutions when recursion may lead to increased space complexity or stack overflow.
    \index{Implementing Iterative Solutions}
\end{itemize}

\section*{Conclusion}

The \textbf{K Closest Points to Origin} problem exemplifies the application of efficient selection algorithms and geometric computations to solve spatial challenges effectively. By leveraging QuickSelect or heap-based methods, the algorithm achieves optimal time and space complexities, making it highly suitable for large datasets. Understanding and implementing such techniques not only enhances problem-solving skills but also provides a foundation for tackling more advanced Computational Geometry problems involving nearest neighbor searches, clustering, and spatial data analysis.

\printindex

% \input{sections/rectangle_overlap}
% \input{sections/rectangle_area}
% \input{sections/k_closest_points_to_origin}
% \input{sections/the_skyline_problem}
% % filename: the_skyline_problem.tex

\problemsection{The Skyline Problem}
\label{chap:The_Skyline_Problem}
\marginnote{\href{https://leetcode.com/problems/the-skyline-problem/}{[LeetCode Link]}\index{LeetCode}}
\marginnote{\href{https://www.geeksforgeeks.org/the-skyline-problem/}{[GeeksForGeeks Link]}\index{GeeksForGeeks}}
\marginnote{\href{https://www.interviewbit.com/problems/the-skyline-problem/}{[InterviewBit Link]}\index{InterviewBit}}
\marginnote{\href{https://app.codesignal.com/challenges/the-skyline-problem}{[CodeSignal Link]}\index{CodeSignal}}
\marginnote{\href{https://www.codewars.com/kata/the-skyline-problem/train/python}{[Codewars Link]}\index{Codewars}}

The \textbf{Skyline Problem} is a complex Computational Geometry challenge that involves computing the skyline formed by a collection of buildings in a 2D cityscape. Each building is represented by its left and right x-coordinates and its height. The skyline is defined by a list of "key points" where the height changes. This problem tests one's ability to handle large datasets, implement efficient sweep line algorithms, and manage event-driven processing. Mastery of this problem is essential for applications in computer graphics, urban planning simulations, and geographic information systems (GIS).

\section*{Problem Statement}

You are given a list of buildings in a cityscape. Each building is represented as a triplet \([Li, Ri, Hi]\), where \(Li\) and \(Ri\) are the x-coordinates of the left and right edges of the building, respectively, and \(Hi\) is the height of the building.

The skyline should be represented as a list of key points \([x, y]\) in sorted order by \(x\)-coordinate, where \(y\) is the height of the skyline at that point. The skyline should only include critical points where the height changes.

\textbf{Function signature in Python:}
\begin{lstlisting}[language=Python]
def getSkyline(buildings: List[List[int]]) -> List[List[int]]:
\end{lstlisting}

\section*{Examples}

\textbf{Example 1:}

\begin{verbatim}
Input: buildings = [[2,9,10], [3,7,15], [5,12,12], [15,20,10], [19,24,8]]
Output: [[2,10], [3,15], [7,12], [12,0], [15,10], [20,8], [24,0]]
Explanation:
- At x=2, the first building starts, height=10.
- At x=3, the second building starts, height=15.
- At x=7, the second building ends, the third building is still ongoing, height=12.
- At x=12, the third building ends, height drops to 0.
- At x=15, the fourth building starts, height=10.
- At x=20, the fourth building ends, the fifth building is still ongoing, height=8.
- At x=24, the fifth building ends, height drops to 0.
\end{verbatim}

\textbf{Example 2:}

\begin{verbatim}
Input: buildings = [[0,2,3], [2,5,3]]
Output: [[0,3], [5,0]]
Explanation:
- The two buildings are contiguous and have the same height, so the skyline drops to 0 at x=5.
\end{verbatim}

\textbf{Example 3:}

\begin{verbatim}
Input: buildings = [[1,3,3], [2,4,4], [5,6,1]]
Output: [[1,3], [2,4], [4,0], [5,1], [6,0]]
Explanation:
- At x=1, first building starts, height=3.
- At x=2, second building starts, height=4.
- At x=4, second building ends, height drops to 0.
- At x=5, third building starts, height=1.
- At x=6, third building ends, height drops to 0.
\end{verbatim}

\textbf{Example 4:}

\begin{verbatim}
Input: buildings = [[0,5,0]]
Output: []
Explanation:
- A building with height 0 does not contribute to the skyline.
\end{verbatim}

\textbf{Constraints:}

\begin{itemize}
    \item \(1 \leq \text{buildings.length} \leq 10^4\)
    \item \(0 \leq Li < Ri \leq 10^9\)
    \item \(0 \leq Hi \leq 10^4\)
\end{itemize}

\section*{Algorithmic Approach}

The \textbf{Sweep Line Algorithm} is an efficient method for solving the Skyline Problem. It involves processing events (building start and end points) in sorted order while maintaining a data structure (typically a max heap) to keep track of active buildings. Here's a step-by-step approach:

\subsection*{1. Event Representation}

Transform each building into two events:
\begin{itemize}
    \item **Start Event:** \((Li, -Hi)\) – Negative height indicates a building starts.
    \item **End Event:** \((Ri, Hi)\) – Positive height indicates a building ends.
\end{itemize}

Sorting the events ensures that start events are processed before end events at the same x-coordinate, and taller buildings are processed before shorter ones.

\subsection*{2. Sorting the Events}

Sort all events based on:
\begin{enumerate}
    \item **x-coordinate:** Ascending order.
    \item **Height:**
    \begin{itemize}
        \item For start events, taller buildings come first.
        \item For end events, shorter buildings come first.
    \end{itemize}
\end{enumerate}

\subsection*{3. Processing the Events}

Use a max heap to keep track of active building heights. Iterate through the sorted events:
\begin{enumerate}
    \item **Start Event:**
    \begin{itemize}
        \item Add the building's height to the heap.
    \end{itemize}
    
    \item **End Event:**
    \begin{itemize}
        \item Remove the building's height from the heap.
    \end{itemize}
    
    \item **Determine Current Max Height:**
    \begin{itemize}
        \item The current max height is the top of the heap.
    \end{itemize}
    
    \item **Update Skyline:**
    \begin{itemize}
        \item If the current max height differs from the previous max height, add a new key point \([x, current\_max\_height]\).
    \end{itemize}
\end{enumerate}

\subsection*{4. Finalizing the Skyline}

After processing all events, the accumulated key points represent the skyline.

\marginnote{The Sweep Line Algorithm efficiently handles dynamic changes in active buildings, ensuring accurate skyline construction.}

\section*{Complexities}

\begin{itemize}
    \item \textbf{Time Complexity:} \(O(n \log n)\), where \(n\) is the number of buildings. Sorting the events takes \(O(n \log n)\), and each heap operation takes \(O(\log n)\).
    
    \item \textbf{Space Complexity:} \(O(n)\), due to the storage of events and the heap.
\end{itemize}

\section*{Python Implementation}

\marginnote{Implementing the Sweep Line Algorithm with a max heap ensures an efficient and accurate solution.}

Below is the complete Python code implementing the \texttt{getSkyline} function:

\begin{fullwidth}
\begin{lstlisting}[language=Python]
from typing import List
import heapq

class Solution:
    def getSkyline(self, buildings: List[List[int]]) -> List[List[int]]:
        # Create a list of all events
        # For start events, use negative height to ensure they are processed before end events
        events = []
        for L, R, H in buildings:
            events.append((L, -H))
            events.append((R, H))
        
        # Sort the events
        # First by x-coordinate, then by height
        events.sort()
        
        # Max heap to keep track of active buildings
        heap = [0]  # Initialize with ground level
        heapq.heapify(heap)
        active_heights = {0: 1}  # Dictionary to count heights
        
        result = []
        prev_max = 0
        
        for x, h in events:
            if h < 0:
                # Start of a building, add height to heap and dictionary
                heapq.heappush(heap, h)
                active_heights[h] = active_heights.get(h, 0) + 1
            else:
                # End of a building, remove height from dictionary
                active_heights[h] -= 1
                if active_heights[h] == 0:
                    del active_heights[h]
            
            # Current max height
            while heap and active_heights.get(heap[0], 0) == 0:
                heapq.heappop(heap)
            current_max = -heap[0] if heap else 0
            
            # If the max height has changed, add to result
            if current_max != prev_max:
                result.append([x, current_max])
                prev_max = current_max
        
        return result

# Example usage:
solution = Solution()
print(solution.getSkyline([[2,9,10], [3,7,15], [5,12,12], [15,20,10], [19,24,8]]))
# Output: [[2,10], [3,15], [7,12], [12,0], [15,10], [20,8], [24,0]]

print(solution.getSkyline([[0,2,3], [2,5,3]]))
# Output: [[0,3], [5,0]]

print(solution.getSkyline([[1,3,3], [2,4,4], [5,6,1]]))
# Output: [[1,3], [2,4], [4,0], [5,1], [6,0]]

print(solution.getSkyline([[0,5,0]]))
# Output: []
\end{lstlisting}
\end{fullwidth}

This implementation efficiently constructs the skyline by processing all building events in sorted order and maintaining active building heights using a max heap. It ensures that only critical points where the skyline changes are recorded.

\section*{Explanation}

The \texttt{getSkyline} function constructs the skyline formed by a set of buildings by leveraging the Sweep Line Algorithm and a max heap to track active buildings. Here's a detailed breakdown of the implementation:

\subsection*{1. Event Representation}

\begin{itemize}
    \item Each building is transformed into two events:
    \begin{itemize}
        \item **Start Event:** \((Li, -Hi)\) – Negative height indicates the start of a building.
        \item **End Event:** \((Ri, Hi)\) – Positive height indicates the end of a building.
    \end{itemize}
\end{itemize}

\subsection*{2. Sorting the Events}

\begin{itemize}
    \item Events are sorted primarily by their x-coordinate in ascending order.
    \item For events with the same x-coordinate:
    \begin{itemize}
        \item Start events (with negative heights) are processed before end events.
        \item Taller buildings are processed before shorter ones.
    \end{itemize}
\end{itemize}

\subsection*{3. Processing the Events}

\begin{itemize}
    \item **Heap Initialization:**
    \begin{itemize}
        \item A max heap is initialized with a ground level height of 0.
        \item A dictionary \texttt{active\_heights} tracks the count of active building heights.
    \end{itemize}
    
    \item **Iterating Through Events:**
    \begin{enumerate}
        \item **Start Event:**
        \begin{itemize}
            \item Add the building's height to the heap.
            \item Increment the count of the height in \texttt{active\_heights}.
        \end{itemize}
        
        \item **End Event:**
        \begin{itemize}
            \item Decrement the count of the building's height in \texttt{active\_heights}.
            \item If the count reaches zero, remove the height from the dictionary.
        \end{itemize}
        
        \item **Determine Current Max Height:**
        \begin{itemize}
            \item Remove heights from the heap that are no longer active.
            \item The current max height is the top of the heap.
        \end{itemize}
        
        \item **Update Skyline:**
        \begin{itemize}
            \item If the current max height differs from the previous max height, add a new key point \([x, current\_max\_height]\).
        \end{itemize}
    \end{enumerate}
\end{itemize}

\subsection*{4. Finalizing the Skyline}

\begin{itemize}
    \item After processing all events, the \texttt{result} list contains the key points defining the skyline.
\end{itemize}

\subsection*{5. Example Walkthrough}

Consider the first example:
\begin{verbatim}
Input: buildings = [[2,9,10], [3,7,15], [5,12,12], [15,20,10], [19,24,8]]
Output: [[2,10], [3,15], [7,12], [12,0], [15,10], [20,8], [24,0]]
\end{verbatim}

\begin{enumerate}
    \item **Event Transformation:**
    \begin{itemize}
        \item \((2, -10)\), \((9, 10)\)
        \item \((3, -15)\), \((7, 15)\)
        \item \((5, -12)\), \((12, 12)\)
        \item \((15, -10)\), \((20, 10)\)
        \item \((19, -8)\), \((24, 8)\)
    \end{itemize}
    
    \item **Sorting Events:**
    \begin{itemize}
        \item Sorted order: \((2, -10)\), \((3, -15)\), \((5, -12)\), \((7, 15)\), \((9, 10)\), \((12, 12)\), \((15, -10)\), \((19, -8)\), \((20, 10)\), \((24, 8)\)
    \end{itemize}
    
    \item **Processing Events:**
    \begin{itemize}
        \item At each event, update the heap and determine if the skyline height changes.
    \end{itemize}
    
    \item **Result Construction:**
    \begin{itemize}
        \item The resulting skyline key points are accumulated as \([[2,10], [3,15], [7,12], [12,0], [15,10], [20,8], [24,0]]\).
    \end{itemize}
\end{enumerate}

Thus, the function correctly constructs the skyline based on the buildings' positions and heights.

\section*{Why This Approach}

The Sweep Line Algorithm combined with a max heap offers an optimal solution with \(O(n \log n)\) time complexity and efficient handling of overlapping buildings. By processing events in sorted order and maintaining active building heights, the algorithm ensures that all critical points in the skyline are accurately identified without redundant computations.

\section*{Alternative Approaches}

\subsection*{1. Divide and Conquer}

Divide the set of buildings into smaller subsets, compute the skyline for each subset, and then merge the skylines.

\begin{lstlisting}[language=Python]
class Solution:
    def getSkyline(self, buildings: List[List[int]]) -> List[List[int]]:
        def merge(left, right):
            h1, h2 = 0, 0
            i, j = 0, 0
            merged = []
            while i < len(left) and j < len(right):
                if left[i][0] < right[j][0]:
                    x, h1 = left[i]
                    i += 1
                elif left[i][0] > right[j][0]:
                    x, h2 = right[j]
                    j += 1
                else:
                    x, h1 = left[i]
                    _, h2 = right[j]
                    i += 1
                    j += 1
                max_h = max(h1, h2)
                if not merged or merged[-1][1] != max_h:
                    merged.append([x, max_h])
            merged.extend(left[i:])
            merged.extend(right[j:])
            return merged
        
        def divide(buildings):
            if not buildings:
                return []
            if len(buildings) == 1:
                L, R, H = buildings[0]
                return [[L, H], [R, 0]]
            mid = len(buildings) // 2
            left = divide(buildings[:mid])
            right = divide(buildings[mid:])
            return merge(left, right)
        
        return divide(buildings)
\end{lstlisting}

\textbf{Complexities:}
\begin{itemize}
    \item \textbf{Time Complexity:} \(O(n \log n)\)
    \item \textbf{Space Complexity:} \(O(n)\)
\end{itemize}

\subsection*{2. Using Segment Trees}

Implement a segment tree to manage and query overlapping building heights dynamically.

\textbf{Note}: This approach is more complex and is generally used for advanced scenarios with multiple dynamic queries.

\section*{Similar Problems to This One}

Several problems involve skyline-like constructions, spatial data analysis, and efficient event processing, utilizing similar algorithmic strategies:

\begin{itemize}
    \item \textbf{Merge Intervals}: Merge overlapping intervals in a list.
    \item \textbf{Largest Rectangle in Histogram}: Find the largest rectangular area in a histogram.
    \item \textbf{Interval Partitioning}: Assign intervals to resources without overlap.
    \item \textbf{Line Segment Intersection}: Detect intersections among line segments.
    \item \textbf{Closest Pair of Points}: Find the closest pair of points in a set.
    \item \textbf{Convex Hull}: Compute the convex hull of a set of points.
    \item \textbf{Point Inside Polygon}: Determine if a point lies inside a given polygon.
    \item \textbf{Range Searching}: Efficiently query geometric data within a specified range.
\end{itemize}

These problems reinforce concepts of event-driven processing, spatial reasoning, and efficient algorithm design in various contexts.

\section*{Things to Keep in Mind and Tricks}

When tackling the \textbf{Skyline Problem}, consider the following tips and best practices to enhance efficiency and correctness:

\begin{itemize}
    \item \textbf{Understand Sweep Line Technique}: Grasp how the sweep line algorithm processes events in sorted order to handle dynamic changes efficiently.
    \index{Sweep Line Technique}
    
    \item \textbf{Leverage Priority Queues (Heaps)}: Use max heaps to keep track of active buildings' heights, enabling quick access to the current maximum height.
    \index{Priority Queues}
    
    \item \textbf{Handle Start and End Events Differently}: Differentiate between building start and end events to accurately manage active heights.
    \index{Start and End Events}
    
    \item \textbf{Optimize Event Sorting}: Sort events primarily by x-coordinate and secondarily by height to ensure correct processing order.
    \index{Event Sorting}
    
    \item \textbf{Manage Active Heights Efficiently}: Use data structures that allow efficient insertion, deletion, and retrieval of maximum elements.
    \index{Active Heights Management}
    
    \item \textbf{Avoid Redundant Key Points}: Only record key points when the skyline height changes to minimize the output list.
    \index{Avoiding Redundant Key Points}
    
    \item \textbf{Implement Helper Functions}: Create helper functions for tasks like distance calculation, event handling, and heap management to enhance modularity.
    \index{Helper Functions}
    
    \item \textbf{Code Readability}: Maintain clear and readable code through meaningful variable names and structured logic.
    \index{Code Readability}
    
    \item \textbf{Test Extensively}: Implement a wide range of test cases, including overlapping, non-overlapping, and edge-touching buildings, to ensure robustness.
    \index{Extensive Testing}
    
    \item \textbf{Handle Degenerate Cases}: Manage cases where buildings have zero height or identical coordinates gracefully.
    \index{Degenerate Cases}
    
    \item \textbf{Understand Geometric Relationships}: Grasp how buildings overlap and influence the skyline to simplify the algorithm.
    \index{Geometric Relationships}
    
    \item \textbf{Use Appropriate Data Structures}: Utilize appropriate data structures like heaps, lists, and dictionaries to manage and process data efficiently.
    \index{Appropriate Data Structures}
    
    \item \textbf{Optimize for Large Inputs}: Design the algorithm to handle large numbers of buildings without significant performance degradation.
    \index{Optimizing for Large Inputs}
    
    \item \textbf{Implement Iterative Solutions Carefully}: Ensure that loop conditions are correctly defined to prevent infinite loops or incorrect terminations.
    \index{Iterative Solutions}
    
    \item \textbf{Consistent Naming Conventions}: Use consistent and descriptive naming conventions for variables and functions to improve code clarity.
    \index{Naming Conventions}
\end{itemize}

\section*{Corner and Special Cases to Test When Writing the Code}

When implementing the solution for the \textbf{Skyline Problem}, it is crucial to consider and rigorously test various edge cases to ensure robustness and correctness:

\begin{itemize}
    \item \textbf{No Overlapping Buildings}: All buildings are separate and do not overlap.
    \index{No Overlapping Buildings}
    
    \item \textbf{Fully Overlapping Buildings}: Multiple buildings completely overlap each other.
    \index{Fully Overlapping Buildings}
    
    \item \textbf{Buildings Touching at Edges}: Buildings share common edges without overlapping.
    \index{Buildings Touching at Edges}
    
    \item \textbf{Buildings Touching at Corners}: Buildings share common corners without overlapping.
    \index{Buildings Touching at Corners}
    
    \item \textbf{Single Building}: Only one building is present.
    \index{Single Building}
    
    \item \textbf{Multiple Buildings with Same Start or End}: Multiple buildings start or end at the same x-coordinate.
    \index{Same Start or End}
    
    \item \textbf{Buildings with Zero Height}: Buildings that have zero height should not affect the skyline.
    \index{Buildings with Zero Height}
    
    \item \textbf{Large Number of Buildings}: Test with a large number of buildings to ensure performance and scalability.
    \index{Large Number of Buildings}
    
    \item \textbf{Buildings with Negative Coordinates}: Buildings positioned in negative coordinate space.
    \index{Negative Coordinates}
    
    \item \textbf{Boundary Values}: Buildings at the minimum and maximum limits of the coordinate range.
    \index{Boundary Values}
    
    \item \textbf{Buildings with Identical Coordinates}: Multiple buildings with the same coordinates.
    \index{Identical Coordinates}
    
    \item \textbf{Sequential Buildings}: Buildings placed sequentially without gaps.
    \index{Sequential Buildings}
    
    \item \textbf{Overlapping and Non-Overlapping Mixed}: A mix of overlapping and non-overlapping buildings.
    \index{Overlapping and Non-Overlapping Mixed}
    
    \item \textbf{Buildings with Very Large Heights}: Buildings with heights at the upper limit of the constraints.
    \index{Very Large Heights}
    
    \item \textbf{Empty Input}: No buildings are provided.
    \index{Empty Input}
\end{itemize}

\section*{Implementation Considerations}

When implementing the \texttt{getSkyline} function, keep in mind the following considerations to ensure robustness and efficiency:

\begin{itemize}
    \item \textbf{Data Type Selection}: Use appropriate data types that can handle large input values and avoid overflow or precision issues.
    \index{Data Type Selection}
    
    \item \textbf{Optimizing Event Sorting}: Efficiently sort events based on x-coordinates and heights to ensure correct processing order.
    \index{Optimizing Event Sorting}
    
    \item \textbf{Handling Large Inputs}: Design the algorithm to handle up to \(10^4\) buildings efficiently without significant performance degradation.
    \index{Handling Large Inputs}
    
    \item \textbf{Using Efficient Data Structures}: Utilize heaps, lists, and dictionaries effectively to manage and process events and active heights.
    \index{Efficient Data Structures}
    
    \item \textbf{Avoiding Redundant Calculations}: Ensure that distance and overlap calculations are performed only when necessary to optimize performance.
    \index{Avoiding Redundant Calculations}
    
    \item \textbf{Code Readability and Documentation}: Maintain clear and readable code through meaningful variable names and comprehensive comments to facilitate understanding and maintenance.
    \index{Code Readability}
    
    \item \textbf{Edge Case Handling}: Implement checks for edge cases to prevent incorrect results or runtime errors.
    \index{Edge Case Handling}
    
    \item \textbf{Implementing Helper Functions}: Create helper functions for tasks like distance calculation, event handling, and heap management to enhance modularity.
    \index{Helper Functions}
    
    \item \textbf{Consistent Naming Conventions}: Use consistent and descriptive naming conventions for variables and functions to improve code clarity.
    \index{Naming Conventions}
    
    \item \textbf{Memory Management}: Ensure that the algorithm manages memory efficiently, especially when dealing with large datasets.
    \index{Memory Management}
    
    \item \textbf{Implementing Iterative Solutions Carefully}: Ensure that loop conditions are correctly defined to prevent infinite loops or incorrect terminations.
    \index{Iterative Solutions}
    
    \item \textbf{Avoiding Floating-Point Precision Issues}: Since the problem deals with integers, floating-point precision is not a concern, simplifying the implementation.
    \index{Floating-Point Precision}
    
    \item \textbf{Testing and Validation}: Develop a comprehensive suite of test cases that cover all possible scenarios, including edge cases, to validate the correctness and efficiency of the implementation.
    \index{Testing and Validation}
    
    \item \textbf{Performance Considerations}: Optimize the loop conditions and operations to ensure that the function runs efficiently, especially for large input numbers.
    \index{Performance Considerations}
\end{itemize}

\section*{Conclusion}

The \textbf{Skyline Problem} is a quintessential example of applying advanced algorithmic techniques and geometric reasoning to solve complex spatial challenges. By leveraging the Sweep Line Algorithm and maintaining active building heights using a max heap, the solution efficiently constructs the skyline with optimal time and space complexities. Understanding and implementing such sophisticated algorithms not only enhances problem-solving skills but also provides a foundation for tackling a wide array of Computational Geometry problems in various domains, including computer graphics, urban planning simulations, and geographic information systems.

\printindex

% \input{sections/rectangle_overlap}
% \input{sections/rectangle_area}
% \input{sections/k_closest_points_to_origin}
% \input{sections/the_skyline_problem}
% % filename: rectangle_area.tex

\problemsection{Rectangle Area}
\label{chap:Rectangle_Area}
\marginnote{\href{https://leetcode.com/problems/rectangle-area/}{[LeetCode Link]}\index{LeetCode}}
\marginnote{\href{https://www.geeksforgeeks.org/find-area-two-overlapping-rectangles/}{[GeeksForGeeks Link]}\index{GeeksForGeeks}}
\marginnote{\href{https://www.interviewbit.com/problems/rectangle-area/}{[InterviewBit Link]}\index{InterviewBit}}
\marginnote{\href{https://app.codesignal.com/challenges/rectangle-area}{[CodeSignal Link]}\index{CodeSignal}}
\marginnote{\href{https://www.codewars.com/kata/rectangle-area/train/python}{[Codewars Link]}\index{Codewars}}

The \textbf{Rectangle Area} problem is a classic Computational Geometry challenge that involves calculating the total area covered by two axis-aligned rectangles in a 2D plane. This problem tests one's ability to perform geometric calculations, handle overlapping scenarios, and implement efficient algorithms. Mastery of this problem is essential for applications in computer graphics, spatial analysis, and computational modeling.

\section*{Problem Statement}

Given two axis-aligned rectangles in a 2D plane, compute the total area covered by the two rectangles. The area covered by the overlapping region should be counted only once.

Each rectangle is represented as a list of four integers \([x1, y1, x2, y2]\), where \((x1, y1)\) are the coordinates of the bottom-left corner, and \((x2, y2)\) are the coordinates of the top-right corner.

\textbf{Function signature in Python:}
\begin{lstlisting}[language=Python]
def computeArea(A: List[int], B: List[int]) -> int:
\end{lstlisting}

\section*{Examples}

\textbf{Example 1:}

\begin{verbatim}
Input: A = [-3,0,3,4], B = [0,-1,9,2]
Output: 45
Explanation:
Area of A = (3 - (-3)) * (4 - 0) = 6 * 4 = 24
Area of B = (9 - 0) * (2 - (-1)) = 9 * 3 = 27
Overlapping Area = (3 - 0) * (2 - 0) = 3 * 2 = 6
Total Area = 24 + 27 - 6 = 45
\end{verbatim}

\textbf{Example 2:}

\begin{verbatim}
Input: A = [0,0,0,0], B = [0,0,0,0]
Output: 0
Explanation:
Both rectangles are degenerate points with zero area.
\end{verbatim}

\textbf{Example 3:}

\begin{verbatim}
Input: A = [0,0,2,2], B = [1,1,3,3]
Output: 7
Explanation:
Area of A = 4
Area of B = 4
Overlapping Area = 1
Total Area = 4 + 4 - 1 = 7
\end{verbatim}

\textbf{Example 4:}

\begin{verbatim}
Input: A = [0,0,1,1], B = [1,0,2,1]
Output: 2
Explanation:
Rectangles touch at the edge but do not overlap.
Area of A = 1
Area of B = 1
Overlapping Area = 0
Total Area = 1 + 1 = 2
\end{verbatim}

\textbf{Constraints:}

\begin{itemize}
    \item All coordinates are integers in the range \([-10^9, 10^9]\).
    \item For each rectangle, \(x1 < x2\) and \(y1 < y2\).
\end{itemize}

LeetCode link: \href{https://leetcode.com/problems/rectangle-area/}{Rectangle Area}\index{LeetCode}

\section*{Algorithmic Approach}

To compute the total area covered by two axis-aligned rectangles, we can follow these steps:

1. **Calculate Individual Areas:**
   - Compute the area of the first rectangle.
   - Compute the area of the second rectangle.

2. **Determine Overlapping Area:**
   - Calculate the coordinates of the overlapping rectangle, if any.
   - If the rectangles overlap, compute the area of the overlapping region.

3. **Compute Total Area:**
   - Sum the individual areas and subtract the overlapping area to avoid double-counting.

\marginnote{This approach ensures accurate area calculation by handling overlapping regions appropriately.}

\section*{Complexities}

\begin{itemize}
    \item \textbf{Time Complexity:} \(O(1)\). The algorithm performs a constant number of calculations.
    
    \item \textbf{Space Complexity:} \(O(1)\). Only a fixed amount of extra space is used for variables.
\end{itemize}

\section*{Python Implementation}

\marginnote{Implementing the area calculation with overlap consideration ensures an accurate and efficient solution.}

Below is the complete Python code implementing the \texttt{computeArea} function:

\begin{fullwidth}
\begin{lstlisting}[language=Python]
from typing import List

class Solution:
    def computeArea(self, A: List[int], B: List[int]) -> int:
        # Calculate area of rectangle A
        areaA = (A[2] - A[0]) * (A[3] - A[1])
        
        # Calculate area of rectangle B
        areaB = (B[2] - B[0]) * (B[3] - B[1])
        
        # Determine overlap coordinates
        overlap_x1 = max(A[0], B[0])
        overlap_y1 = max(A[1], B[1])
        overlap_x2 = min(A[2], B[2])
        overlap_y2 = min(A[3], B[3])
        
        # Calculate overlapping area
        overlap_width = overlap_x2 - overlap_x1
        overlap_height = overlap_y2 - overlap_y1
        overlap_area = 0
        if overlap_width > 0 and overlap_height > 0:
            overlap_area = overlap_width * overlap_height
        
        # Total area is sum of individual areas minus overlapping area
        total_area = areaA + areaB - overlap_area
        return total_area

# Example usage:
solution = Solution()
print(solution.computeArea([-3,0,3,4], [0,-1,9,2]))  # Output: 45
print(solution.computeArea([0,0,0,0], [0,0,0,0]))    # Output: 0
print(solution.computeArea([0,0,2,2], [1,1,3,3]))    # Output: 7
print(solution.computeArea([0,0,1,1], [1,0,2,1]))    # Output: 2
\end{lstlisting}
\end{fullwidth}

This implementation accurately computes the total area covered by two rectangles by accounting for any overlapping regions. It ensures that the overlapping area is not double-counted.

\section*{Explanation}

The \texttt{computeArea} function calculates the combined area of two axis-aligned rectangles by following these steps:

\subsection*{1. Calculate Individual Areas}

\begin{itemize}
    \item **Rectangle A:**
    \begin{itemize}
        \item Width: \(A[2] - A[0]\)
        \item Height: \(A[3] - A[1]\)
        \item Area: Width \(\times\) Height
    \end{itemize}
    
    \item **Rectangle B:**
    \begin{itemize}
        \item Width: \(B[2] - B[0]\)
        \item Height: \(B[3] - B[1]\)
        \item Area: Width \(\times\) Height
    \end{itemize}
\end{itemize}

\subsection*{2. Determine Overlapping Area}

\begin{itemize}
    \item **Overlap Coordinates:**
    \begin{itemize}
        \item Left (x-coordinate): \(\text{max}(A[0], B[0])\)
        \item Bottom (y-coordinate): \(\text{max}(A[1], B[1])\)
        \item Right (x-coordinate): \(\text{min}(A[2], B[2])\)
        \item Top (y-coordinate): \(\text{min}(A[3], B[3])\)
    \end{itemize}
    
    \item **Overlap Dimensions:**
    \begin{itemize}
        \item Width: \(\text{overlap\_x2} - \text{overlap\_x1}\)
        \item Height: \(\text{overlap\_y2} - \text{overlap\_y1}\)
    \end{itemize}
    
    \item **Overlap Area:**
    \begin{itemize}
        \item If both width and height are positive, the rectangles overlap, and the overlapping area is their product.
        \item Otherwise, there is no overlap, and the overlapping area is zero.
    \end{itemize}
\end{itemize}

\subsection*{3. Compute Total Area}

\begin{itemize}
    \item Total Area = Area of Rectangle A + Area of Rectangle B - Overlapping Area
\end{itemize}

\subsection*{4. Example Walkthrough}

Consider the first example:
\begin{verbatim}
Input: A = [-3,0,3,4], B = [0,-1,9,2]
Output: 45
\end{verbatim}

\begin{enumerate}
    \item **Calculate Areas:**
    \begin{itemize}
        \item Area of A = (3 - (-3)) * (4 - 0) = 6 * 4 = 24
        \item Area of B = (9 - 0) * (2 - (-1)) = 9 * 3 = 27
    \end{itemize}
    
    \item **Determine Overlap:**
    \begin{itemize}
        \item overlap\_x1 = max(-3, 0) = 0
        \item overlap\_y1 = max(0, -1) = 0
        \item overlap\_x2 = min(3, 9) = 3
        \item overlap\_y2 = min(4, 2) = 2
        \item overlap\_width = 3 - 0 = 3
        \item overlap\_height = 2 - 0 = 2
        \item overlap\_area = 3 * 2 = 6
    \end{itemize}
    
    \item **Compute Total Area:**
    \begin{itemize}
        \item Total Area = 24 + 27 - 6 = 45
    \end{itemize}
\end{enumerate}

Thus, the function correctly returns \texttt{45}.

\section*{Why This Approach}

This approach is chosen for its straightforwardness and optimal efficiency. By directly calculating the individual areas and intelligently handling the overlapping region, the algorithm ensures accurate results without unnecessary computations. Its constant time complexity makes it highly efficient, even for large coordinate values.

\section*{Alternative Approaches}

\subsection*{1. Using Intersection Dimensions}

Instead of separately calculating areas, directly compute the dimensions of the overlapping region and subtract it from the sum of individual areas.

\begin{lstlisting}[language=Python]
def computeArea(A: List[int], B: List[int]) -> int:
    # Sum of individual areas
    area = (A[2] - A[0]) * (A[3] - A[1]) + (B[2] - B[0]) * (B[3] - B[1])
    
    # Overlapping area
    overlap_width = min(A[2], B[2]) - max(A[0], B[0])
    overlap_height = min(A[3], B[3]) - max(A[1], B[1])
    
    if overlap_width > 0 and overlap_height > 0:
        area -= overlap_width * overlap_height
    
    return area
\end{lstlisting}

\subsection*{2. Using Geometry Libraries}

Leverage computational geometry libraries to handle area calculations and overlapping detections.

\begin{lstlisting}[language=Python]
from shapely.geometry import box

def computeArea(A: List[int], B: List[int]) -> int:
    rect1 = box(A[0], A[1], A[2], A[3])
    rect2 = box(B[0], B[1], B[2], B[3])
    intersection = rect1.intersection(rect2)
    return int(rect1.area + rect2.area - intersection.area)
\end{lstlisting}

\textbf{Note}: This approach requires the \texttt{shapely} library and is more suitable for complex geometric operations.

\section*{Similar Problems to This One}

Several problems involve calculating areas, handling geometric overlaps, and spatial reasoning, utilizing similar algorithmic strategies:

\begin{itemize}
    \item \textbf{Rectangle Overlap}: Determine if two rectangles overlap.
    \item \textbf{Circle Area Overlap}: Calculate the overlapping area between two circles.
    \item \textbf{Polygon Area}: Compute the area of a given polygon.
    \item \textbf{Union of Rectangles}: Calculate the total area covered by multiple rectangles, accounting for overlaps.
    \item \textbf{Intersection of Lines}: Find the intersection point of two lines.
    \item \textbf{Closest Pair of Points}: Find the closest pair of points in a set.
    \item \textbf{Convex Hull}: Compute the convex hull of a set of points.
    \item \textbf{Point Inside Polygon}: Determine if a point lies inside a given polygon.
\end{itemize}

These problems reinforce concepts of geometric calculations, area computations, and efficient algorithm design in various contexts.

\section*{Things to Keep in Mind and Tricks}

When tackling the \textbf{Rectangle Area} problem, consider the following tips and best practices to enhance efficiency and correctness:

\begin{itemize}
    \item \textbf{Understand Geometric Relationships}: Grasp the positional relationships between rectangles to simplify area calculations.
    \index{Geometric Relationships}
    
    \item \textbf{Leverage Coordinate Comparisons}: Use direct comparisons of rectangle coordinates to determine overlapping regions.
    \index{Coordinate Comparisons}
    
    \item \textbf{Handle Overlapping Scenarios}: Accurately calculate the overlapping area to avoid double-counting.
    \index{Overlapping Scenarios}
    
    \item \textbf{Optimize for Efficiency}: Aim for a constant time \(O(1)\) solution by avoiding unnecessary computations or iterations.
    \index{Efficiency Optimization}
    
    \item \textbf{Avoid Floating-Point Precision Issues}: Since all coordinates are integers, floating-point precision is not a concern, simplifying the implementation.
    \index{Floating-Point Precision}
    
    \item \textbf{Use Helper Functions}: Create helper functions to encapsulate repetitive tasks, such as calculating overlap dimensions or areas.
    \index{Helper Functions}
    
    \item \textbf{Code Readability}: Maintain clear and readable code through meaningful variable names and structured logic.
    \index{Code Readability}
    
    \item \textbf{Test Extensively}: Implement a wide range of test cases, including overlapping, non-overlapping, and edge-touching rectangles, to ensure robustness.
    \index{Extensive Testing}
    
    \item \textbf{Understand Axis-Aligned Constraints}: Recognize that axis-aligned rectangles simplify area calculations compared to rotated rectangles.
    \index{Axis-Aligned Constraints}
    
    \item \textbf{Simplify Logical Conditions}: Combine multiple conditions logically to streamline the area calculation process.
    \index{Logical Conditions}
    
    \item \textbf{Use Absolute Values}: When calculating differences, ensure that the dimensions are positive by using absolute values or proper ordering.
    \index{Absolute Values}
    
    \item \textbf{Consider Edge Cases}: Handle cases where rectangles have zero area or touch at edges/corners without overlapping.
    \index{Edge Cases}
\end{itemize}

\section*{Corner and Special Cases to Test When Writing the Code}

When implementing the solution for the \textbf{Rectangle Area} problem, it is crucial to consider and rigorously test various edge cases to ensure robustness and correctness:

\begin{itemize}
    \item \textbf{No Overlap}: Rectangles are completely separate.
    \index{No Overlap}
    
    \item \textbf{Partial Overlap}: Rectangles overlap in one or more regions.
    \index{Partial Overlap}
    
    \item \textbf{Edge Touching}: Rectangles touch exactly at one edge without overlapping.
    \index{Edge Touching}
    
    \item \textbf{Corner Touching}: Rectangles touch exactly at one corner without overlapping.
    \index{Corner Touching}
    
    \item \textbf{One Rectangle Inside Another}: One rectangle is entirely within the other.
    \index{Rectangle Inside}
    
    \item \textbf{Identical Rectangles}: Both rectangles have the same coordinates.
    \index{Identical Rectangles}
    
    \item \textbf{Degenerate Rectangles}: Rectangles with zero area (e.g., \(x1 = x2\) or \(y1 = y2\)).
    \index{Degenerate Rectangles}
    
    \item \textbf{Large Coordinates}: Rectangles with very large coordinate values to test performance and integer handling.
    \index{Large Coordinates}
    
    \item \textbf{Negative Coordinates}: Rectangles positioned in negative coordinate space.
    \index{Negative Coordinates}
    
    \item \textbf{Mixed Overlapping Scenarios}: Combinations of the above cases to ensure comprehensive coverage.
    \index{Mixed Overlapping Scenarios}
    
    \item \textbf{Minimum and Maximum Bounds}: Rectangles at the minimum and maximum limits of the coordinate range.
    \index{Minimum and Maximum Bounds}
    
    \item \textbf{Sequential Rectangles}: Multiple rectangles placed sequentially without overlapping.
    \index{Sequential Rectangles}
    
    \item \textbf{Multiple Overlaps}: Scenarios where more than two rectangles overlap in different regions.
    \index{Multiple Overlaps}
\end{itemize}

\section*{Implementation Considerations}

When implementing the \texttt{computeArea} function, keep in mind the following considerations to ensure robustness and efficiency:

\begin{itemize}
    \item \textbf{Data Type Selection}: Use appropriate data types that can handle large input values without overflow or underflow.
    \index{Data Type Selection}
    
    \item \textbf{Optimizing Comparisons}: Structure logical conditions to efficiently determine overlap dimensions.
    \index{Optimizing Comparisons}
    
    \item \textbf{Handling Large Inputs}: Design the algorithm to efficiently handle large input sizes without significant performance degradation.
    \index{Handling Large Inputs}
    
    \item \textbf{Language-Specific Constraints}: Be aware of how the programming language handles large integers and arithmetic operations.
    \index{Language-Specific Constraints}
    
    \item \textbf{Avoiding Redundant Calculations}: Ensure that each calculation contributes towards determining the final area without unnecessary repetitions.
    \index{Avoiding Redundant Calculations}
    
    \item \textbf{Code Readability and Documentation}: Maintain clear and readable code through meaningful variable names and comprehensive comments to facilitate understanding and maintenance.
    \index{Code Readability}
    
    \item \textbf{Edge Case Handling}: Implement checks for edge cases to prevent incorrect results or runtime errors.
    \index{Edge Case Handling}
    
    \item \textbf{Testing and Validation}: Develop a comprehensive suite of test cases that cover all possible scenarios, including edge cases, to validate the correctness and efficiency of the implementation.
    \index{Testing and Validation}
    
    \item \textbf{Scalability}: Design the algorithm to scale efficiently with increasing input sizes, maintaining performance and resource utilization.
    \index{Scalability}
    
    \item \textbf{Using Helper Functions}: Consider creating helper functions for repetitive tasks, such as calculating overlap dimensions, to enhance modularity and reusability.
    \index{Helper Functions}
    
    \item \textbf{Consistent Naming Conventions}: Use consistent and descriptive naming conventions for variables to improve code clarity.
    \index{Naming Conventions}
    
    \item \textbf{Implementing Unit Tests}: Develop unit tests for each logical condition to ensure that all scenarios are correctly handled.
    \index{Unit Tests}
    
    \item \textbf{Error Handling}: Incorporate error handling to manage invalid inputs gracefully.
    \index{Error Handling}
\end{itemize}

\section*{Conclusion}

The \textbf{Rectangle Area} problem showcases the application of fundamental geometric principles and efficient algorithm design to compute spatial properties accurately. By systematically calculating individual areas and intelligently handling overlapping regions, the algorithm ensures precise results without redundant computations. Understanding and implementing such techniques not only enhances problem-solving skills but also provides a foundation for tackling more complex Computational Geometry challenges involving multiple geometric entities and intricate spatial relationships.

\printindex

% % filename: rectangle_overlap.tex

\problemsection{Rectangle Overlap}
\label{chap:Rectangle_Overlap}
\marginnote{\href{https://leetcode.com/problems/rectangle-overlap/}{[LeetCode Link]}\index{LeetCode}}
\marginnote{\href{https://www.geeksforgeeks.org/check-if-two-rectangles-overlap/}{[GeeksForGeeks Link]}\index{GeeksForGeeks}}
\marginnote{\href{https://www.interviewbit.com/problems/rectangle-overlap/}{[InterviewBit Link]}\index{InterviewBit}}
\marginnote{\href{https://app.codesignal.com/challenges/rectangle-overlap}{[CodeSignal Link]}\index{CodeSignal}}
\marginnote{\href{https://www.codewars.com/kata/rectangle-overlap/train/python}{[Codewars Link]}\index{Codewars}}

The \textbf{Rectangle Overlap} problem is a fundamental challenge in Computational Geometry that involves determining whether two axis-aligned rectangles overlap. This problem tests one's ability to understand geometric properties, implement conditional logic, and optimize for efficient computation. Mastery of this problem is essential for applications in computer graphics, collision detection, and spatial data analysis.

\section*{Problem Statement}

Given two axis-aligned rectangles in a 2D plane, determine if they overlap. Each rectangle is defined by its bottom-left and top-right coordinates.

A rectangle is represented as a list of four integers \([x1, y1, x2, y2]\), where \((x1, y1)\) are the coordinates of the bottom-left corner, and \((x2, y2)\) are the coordinates of the top-right corner.

\textbf{Function signature in Python:}
\begin{lstlisting}[language=Python]
def isRectangleOverlap(rec1: List[int], rec2: List[int]) -> bool:
\end{lstlisting}

\section*{Examples}

\textbf{Example 1:}

\begin{verbatim}
Input: rec1 = [0,0,2,2], rec2 = [1,1,3,3]
Output: True
Explanation: The rectangles overlap in the area defined by [1,1,2,2].
\end{verbatim}

\textbf{Example 2:}

\begin{verbatim}
Input: rec1 = [0,0,1,1], rec2 = [1,0,2,1]
Output: False
Explanation: The rectangles touch at the edge but do not overlap.
\end{verbatim}

\textbf{Example 3:}

\begin{verbatim}
Input: rec1 = [0,0,1,1], rec2 = [2,2,3,3]
Output: False
Explanation: The rectangles are completely separate.
\end{verbatim}

\textbf{Example 4:}

\begin{verbatim}
Input: rec1 = [0,0,5,5], rec2 = [3,3,7,7]
Output: True
Explanation: The rectangles overlap in the area defined by [3,3,5,5].
\end{verbatim}

\textbf{Example 5:}

\begin{verbatim}
Input: rec1 = [0,0,0,0], rec2 = [0,0,0,0]
Output: False
Explanation: Both rectangles are degenerate points.
\end{verbatim}

\textbf{Constraints:}

\begin{itemize}
    \item All coordinates are integers in the range \([-10^9, 10^9]\).
    \item For each rectangle, \(x1 < x2\) and \(y1 < y2\).
\end{itemize}

LeetCode link: \href{https://leetcode.com/problems/rectangle-overlap/}{Rectangle Overlap}\index{LeetCode}

\section*{Algorithmic Approach}

To determine whether two axis-aligned rectangles overlap, we can use the following logical conditions:

1. **Non-Overlap Conditions:**
   - One rectangle is to the left of the other.
   - One rectangle is above the other.

2. **Overlap Condition:**
   - If neither of the non-overlap conditions is true, the rectangles must overlap.

\subsection*{Steps:}

1. **Extract Coordinates:**
   - For both rectangles, extract the bottom-left and top-right coordinates.

2. **Check Non-Overlap Conditions:**
   - If the right side of the first rectangle is less than or equal to the left side of the second rectangle, they do not overlap.
   - If the left side of the first rectangle is greater than or equal to the right side of the second rectangle, they do not overlap.
   - If the top side of the first rectangle is less than or equal to the bottom side of the second rectangle, they do not overlap.
   - If the bottom side of the first rectangle is greater than or equal to the top side of the second rectangle, they do not overlap.

3. **Determine Overlap:**
   - If none of the non-overlap conditions are met, the rectangles overlap.

\marginnote{This approach provides an efficient \(O(1)\) time complexity solution by leveraging simple geometric comparisons.}

\section*{Complexities}

\begin{itemize}
    \item \textbf{Time Complexity:} \(O(1)\). The algorithm performs a constant number of comparisons regardless of input size.
    
    \item \textbf{Space Complexity:} \(O(1)\). Only a fixed amount of extra space is used for variables.
\end{itemize}

\section*{Python Implementation}

\marginnote{Implementing the overlap check using coordinate comparisons ensures an optimal and straightforward solution.}

Below is the complete Python code implementing the \texttt{isRectangleOverlap} function:

\begin{fullwidth}
\begin{lstlisting}[language=Python]
from typing import List

class Solution:
    def isRectangleOverlap(self, rec1: List[int], rec2: List[int]) -> bool:
        # Extract coordinates
        left1, bottom1, right1, top1 = rec1
        left2, bottom2, right2, top2 = rec2
        
        # Check non-overlapping conditions
        if right1 <= left2 or right2 <= left1:
            return False
        if top1 <= bottom2 or top2 <= bottom1:
            return False
        
        # If none of the above, rectangles overlap
        return True

# Example usage:
solution = Solution()
print(solution.isRectangleOverlap([0,0,2,2], [1,1,3,3]))  # Output: True
print(solution.isRectangleOverlap([0,0,1,1], [1,0,2,1]))  # Output: False
print(solution.isRectangleOverlap([0,0,1,1], [2,2,3,3]))  # Output: False
print(solution.isRectangleOverlap([0,0,5,5], [3,3,7,7]))  # Output: True
print(solution.isRectangleOverlap([0,0,0,0], [0,0,0,0]))  # Output: False
\end{lstlisting}
\end{fullwidth}

This implementation efficiently checks for overlap by comparing the coordinates of the two rectangles. If any of the non-overlapping conditions are met, it returns \texttt{False}; otherwise, it returns \texttt{True}.

\section*{Explanation}

The \texttt{isRectangleOverlap} function determines whether two axis-aligned rectangles overlap by comparing their respective coordinates. Here's a detailed breakdown of the implementation:

\subsection*{1. Extract Coordinates}

\begin{itemize}
    \item For each rectangle, extract the left (\(x1\)), bottom (\(y1\)), right (\(x2\)), and top (\(y2\)) coordinates.
    \item This simplifies the comparison process by providing clear variables representing each side of the rectangles.
\end{itemize}

\subsection*{2. Check Non-Overlap Conditions}

\begin{itemize}
    \item **Horizontal Separation:**
    \begin{itemize}
        \item If the right side of the first rectangle (\(right1\)) is less than or equal to the left side of the second rectangle (\(left2\)), there is no horizontal overlap.
        \item Similarly, if the right side of the second rectangle (\(right2\)) is less than or equal to the left side of the first rectangle (\(left1\)), there is no horizontal overlap.
    \end{itemize}
    
    \item **Vertical Separation:**
    \begin{itemize}
        \item If the top side of the first rectangle (\(top1\)) is less than or equal to the bottom side of the second rectangle (\(bottom2\)), there is no vertical overlap.
        \item Similarly, if the top side of the second rectangle (\(top2\)) is less than or equal to the bottom side of the first rectangle (\(bottom1\)), there is no vertical overlap.
    \end{itemize}
    
    \item If any of these non-overlapping conditions are true, the rectangles do not overlap, and the function returns \texttt{False}.
\end{itemize}

\subsection*{3. Determine Overlap}

\begin{itemize}
    \item If none of the non-overlapping conditions are met, it implies that the rectangles overlap both horizontally and vertically.
    \item The function returns \texttt{True} in this case.
\end{itemize}

\subsection*{4. Example Walkthrough}

Consider the first example:
\begin{verbatim}
Input: rec1 = [0,0,2,2], rec2 = [1,1,3,3]
Output: True
\end{verbatim}

\begin{enumerate}
    \item Extract coordinates:
    \begin{itemize}
        \item rec1: left1 = 0, bottom1 = 0, right1 = 2, top1 = 2
        \item rec2: left2 = 1, bottom2 = 1, right2 = 3, top2 = 3
    \end{itemize}
    
    \item Check non-overlap conditions:
    \begin{itemize}
        \item \(right1 = 2\) is not less than or equal to \(left2 = 1\)
        \item \(right2 = 3\) is not less than or equal to \(left1 = 0\)
        \item \(top1 = 2\) is not less than or equal to \(bottom2 = 1\)
        \item \(top2 = 3\) is not less than or equal to \(bottom1 = 0\)
    \end{itemize}
    
    \item Since none of the non-overlapping conditions are met, the rectangles overlap.
\end{enumerate}

Thus, the function correctly returns \texttt{True}.

\section*{Why This Approach}

This approach is chosen for its simplicity and efficiency. By leveraging direct coordinate comparisons, the algorithm achieves constant time complexity without the need for complex data structures or iterative processes. It effectively handles all possible scenarios of rectangle positioning, ensuring accurate detection of overlaps.

\section*{Alternative Approaches}

\subsection*{1. Separating Axis Theorem (SAT)}

The Separating Axis Theorem is a more generalized method for detecting overlaps between convex shapes. While it is not necessary for axis-aligned rectangles, understanding SAT can be beneficial for more complex geometric problems.

\begin{lstlisting}[language=Python]
def isRectangleOverlap(rec1: List[int], rec2: List[int]) -> bool:
    # Using SAT for axis-aligned rectangles
    return not (rec1[2] <= rec2[0] or rec1[0] >= rec2[2] or
                rec1[3] <= rec2[1] or rec1[1] >= rec2[3])
\end{lstlisting}

\textbf{Note}: This implementation is functionally identical to the primary approach but leverages a more generalized geometric theorem.

\subsection*{2. Area-Based Approach}

Calculate the overlapping area between the two rectangles. If the overlapping area is positive, the rectangles overlap.

\begin{lstlisting}[language=Python]
def isRectangleOverlap(rec1: List[int], rec2: List[int]) -> bool:
    # Calculate overlap in x and y dimensions
    x_overlap = min(rec1[2], rec2[2]) - max(rec1[0], rec2[0])
    y_overlap = min(rec1[3], rec2[3]) - max(rec1[1], rec2[1])
    
    # Overlap exists if both overlaps are positive
    return x_overlap > 0 and y_overlap > 0
\end{lstlisting}

\textbf{Complexities:}
\begin{itemize}
    \item \textbf{Time Complexity:} \(O(1)\)
    \item \textbf{Space Complexity:} \(O(1)\)
\end{itemize}

\subsection*{3. Using Rectangles Intersection Function}

Utilize built-in or library functions that handle geometric intersections.

\begin{lstlisting}[language=Python]
from shapely.geometry import box

def isRectangleOverlap(rec1: List[int], rec2: List[int]) -> bool:
    rectangle1 = box(rec1[0], rec1[1], rec1[2], rec1[3])
    rectangle2 = box(rec2[0], rec2[1], rec2[2], rec2[3])
    return rectangle1.intersects(rectangle2) and not rectangle1.touches(rectangle2)
\end{lstlisting}

\textbf{Note}: This approach requires the \texttt{shapely} library and is more suitable for complex geometric operations.

\section*{Similar Problems to This One}

Several problems revolve around geometric overlap, intersection detection, and spatial reasoning, utilizing similar algorithmic strategies:

\begin{itemize}
    \item \textbf{Interval Overlap}: Determine if two intervals on a line overlap.
    \item \textbf{Circle Overlap}: Determine if two circles overlap based on their radii and centers.
    \item \textbf{Polygon Overlap}: Determine if two polygons overlap using algorithms like SAT.
    \item \textbf{Closest Pair of Points}: Find the closest pair of points in a set.
    \item \textbf{Convex Hull}: Compute the convex hull of a set of points.
    \item \textbf{Intersection of Lines}: Find the intersection point of two lines.
    \item \textbf{Point Inside Polygon}: Determine if a point lies inside a given polygon.
\end{itemize}

These problems reinforce the concepts of spatial reasoning, geometric property analysis, and efficient algorithm design in various contexts.

\section*{Things to Keep in Mind and Tricks}

When working with the \textbf{Rectangle Overlap} problem, consider the following tips and best practices to enhance efficiency and correctness:

\begin{itemize}
    \item \textbf{Understand Geometric Relationships}: Grasp the positional relationships between rectangles to simplify overlap detection.
    \index{Geometric Relationships}
    
    \item \textbf{Leverage Coordinate Comparisons}: Use direct comparisons of rectangle coordinates to determine spatial relationships.
    \index{Coordinate Comparisons}
    
    \item \textbf{Handle Edge Cases}: Consider cases where rectangles touch at edges or corners without overlapping.
    \index{Edge Cases}
    
    \item \textbf{Optimize for Efficiency}: Aim for a constant time \(O(1)\) solution by avoiding unnecessary computations or iterations.
    \index{Efficiency Optimization}
    
    \item \textbf{Avoid Floating-Point Precision Issues}: Since all coordinates are integers, floating-point precision is not a concern, simplifying the implementation.
    \index{Floating-Point Precision}
    
    \item \textbf{Use Helper Functions}: Create helper functions to encapsulate repetitive tasks, such as extracting coordinates or checking specific conditions.
    \index{Helper Functions}
    
    \item \textbf{Code Readability}: Maintain clear and readable code through meaningful variable names and structured logic.
    \index{Code Readability}
    
    \item \textbf{Test Extensively}: Implement a wide range of test cases, including overlapping, non-overlapping, and edge-touching rectangles, to ensure robustness.
    \index{Extensive Testing}
    
    \item \textbf{Understand Axis-Aligned Constraints}: Recognize that axis-aligned rectangles simplify overlap detection compared to rotated rectangles.
    \index{Axis-Aligned Constraints}
    
    \item \textbf{Simplify Logical Conditions}: Combine multiple conditions logically to streamline the overlap detection process.
    \index{Logical Conditions}
\end{itemize}

\section*{Corner and Special Cases to Test When Writing the Code}

When implementing the solution for the \textbf{Rectangle Overlap} problem, it is crucial to consider and rigorously test various edge cases to ensure robustness and correctness:

\begin{itemize}
    \item \textbf{No Overlap}: Rectangles are completely separate.
    \index{No Overlap}
    
    \item \textbf{Partial Overlap}: Rectangles overlap in one or more regions.
    \index{Partial Overlap}
    
    \item \textbf{Edge Touching}: Rectangles touch exactly at one edge without overlapping.
    \index{Edge Touching}
    
    \item \textbf{Corner Touching}: Rectangles touch exactly at one corner without overlapping.
    \index{Corner Touching}
    
    \item \textbf{One Rectangle Inside Another}: One rectangle is entirely within the other.
    \index{Rectangle Inside}
    
    \item \textbf{Identical Rectangles}: Both rectangles have the same coordinates.
    \index{Identical Rectangles}
    
    \item \textbf{Degenerate Rectangles}: Rectangles with zero area (e.g., \(x1 = x2\) or \(y1 = y2\)).
    \index{Degenerate Rectangles}
    
    \item \textbf{Large Coordinates}: Rectangles with very large coordinate values to test performance and integer handling.
    \index{Large Coordinates}
    
    \item \textbf{Negative Coordinates}: Rectangles positioned in negative coordinate space.
    \index{Negative Coordinates}
    
    \item \textbf{Mixed Overlapping Scenarios}: Combinations of the above cases to ensure comprehensive coverage.
    \index{Mixed Overlapping Scenarios}
    
    \item \textbf{Minimum and Maximum Bounds}: Rectangles at the minimum and maximum limits of the coordinate range.
    \index{Minimum and Maximum Bounds}
\end{itemize}

\section*{Implementation Considerations}

When implementing the \texttt{isRectangleOverlap} function, keep in mind the following considerations to ensure robustness and efficiency:

\begin{itemize}
    \item \textbf{Data Type Selection}: Use appropriate data types that can handle the range of input values without overflow or underflow.
    \index{Data Type Selection}
    
    \item \textbf{Optimizing Comparisons}: Structure logical conditions to short-circuit evaluations as soon as a non-overlapping condition is met.
    \index{Optimizing Comparisons}
    
    \item \textbf{Language-Specific Constraints}: Be aware of how the programming language handles integer division and comparisons.
    \index{Language-Specific Constraints}
    
    \item \textbf{Avoiding Redundant Calculations}: Ensure that each comparison contributes towards determining overlap without unnecessary repetitions.
    \index{Avoiding Redundant Calculations}
    
    \item \textbf{Code Readability and Documentation}: Maintain clear and readable code through meaningful variable names and comprehensive comments to facilitate understanding and maintenance.
    \index{Code Readability}
    
    \item \textbf{Edge Case Handling}: Implement checks for edge cases to prevent incorrect results or runtime errors.
    \index{Edge Case Handling}
    
    \item \textbf{Testing and Validation}: Develop a comprehensive suite of test cases that cover all possible scenarios, including edge cases, to validate the correctness and efficiency of the implementation.
    \index{Testing and Validation}
    
    \item \textbf{Scalability}: Design the algorithm to scale efficiently with increasing input sizes, maintaining performance and resource utilization.
    \index{Scalability}
    
    \item \textbf{Using Helper Functions}: Consider creating helper functions for repetitive tasks, such as extracting and comparing coordinates, to enhance modularity and reusability.
    \index{Helper Functions}
    
    \item \textbf{Consistent Naming Conventions}: Use consistent and descriptive naming conventions for variables to improve code clarity.
    \index{Naming Conventions}
    
    \item \textbf{Handling Floating-Point Coordinates}: Although the problem specifies integer coordinates, ensure that the implementation can handle floating-point numbers if needed in extended scenarios.
    \index{Floating-Point Coordinates}
    
    \item \textbf{Avoiding Floating-Point Precision Issues}: Since all coordinates are integers, floating-point precision is not a concern, simplifying the implementation.
    \index{Floating-Point Precision}
    
    \item \textbf{Implementing Unit Tests}: Develop unit tests for each logical condition to ensure that all scenarios are correctly handled.
    \index{Unit Tests}
    
    \item \textbf{Error Handling}: Incorporate error handling to manage invalid inputs gracefully.
    \index{Error Handling}
\end{itemize}

\section*{Conclusion}

The \textbf{Rectangle Overlap} problem exemplifies the application of fundamental geometric principles and conditional logic to solve spatial challenges efficiently. By leveraging simple coordinate comparisons, the algorithm achieves optimal time and space complexities, making it highly suitable for real-time applications such as collision detection in gaming, layout planning in graphics, and spatial data analysis. Understanding and implementing such techniques not only enhances problem-solving skills but also provides a foundation for tackling more complex Computational Geometry problems involving varied geometric shapes and interactions.

\printindex

% \input{sections/rectangle_overlap}
% \input{sections/rectangle_area}
% \input{sections/k_closest_points_to_origin}
% \input{sections/the_skyline_problem}
% % filename: rectangle_area.tex

\problemsection{Rectangle Area}
\label{chap:Rectangle_Area}
\marginnote{\href{https://leetcode.com/problems/rectangle-area/}{[LeetCode Link]}\index{LeetCode}}
\marginnote{\href{https://www.geeksforgeeks.org/find-area-two-overlapping-rectangles/}{[GeeksForGeeks Link]}\index{GeeksForGeeks}}
\marginnote{\href{https://www.interviewbit.com/problems/rectangle-area/}{[InterviewBit Link]}\index{InterviewBit}}
\marginnote{\href{https://app.codesignal.com/challenges/rectangle-area}{[CodeSignal Link]}\index{CodeSignal}}
\marginnote{\href{https://www.codewars.com/kata/rectangle-area/train/python}{[Codewars Link]}\index{Codewars}}

The \textbf{Rectangle Area} problem is a classic Computational Geometry challenge that involves calculating the total area covered by two axis-aligned rectangles in a 2D plane. This problem tests one's ability to perform geometric calculations, handle overlapping scenarios, and implement efficient algorithms. Mastery of this problem is essential for applications in computer graphics, spatial analysis, and computational modeling.

\section*{Problem Statement}

Given two axis-aligned rectangles in a 2D plane, compute the total area covered by the two rectangles. The area covered by the overlapping region should be counted only once.

Each rectangle is represented as a list of four integers \([x1, y1, x2, y2]\), where \((x1, y1)\) are the coordinates of the bottom-left corner, and \((x2, y2)\) are the coordinates of the top-right corner.

\textbf{Function signature in Python:}
\begin{lstlisting}[language=Python]
def computeArea(A: List[int], B: List[int]) -> int:
\end{lstlisting}

\section*{Examples}

\textbf{Example 1:}

\begin{verbatim}
Input: A = [-3,0,3,4], B = [0,-1,9,2]
Output: 45
Explanation:
Area of A = (3 - (-3)) * (4 - 0) = 6 * 4 = 24
Area of B = (9 - 0) * (2 - (-1)) = 9 * 3 = 27
Overlapping Area = (3 - 0) * (2 - 0) = 3 * 2 = 6
Total Area = 24 + 27 - 6 = 45
\end{verbatim}

\textbf{Example 2:}

\begin{verbatim}
Input: A = [0,0,0,0], B = [0,0,0,0]
Output: 0
Explanation:
Both rectangles are degenerate points with zero area.
\end{verbatim}

\textbf{Example 3:}

\begin{verbatim}
Input: A = [0,0,2,2], B = [1,1,3,3]
Output: 7
Explanation:
Area of A = 4
Area of B = 4
Overlapping Area = 1
Total Area = 4 + 4 - 1 = 7
\end{verbatim}

\textbf{Example 4:}

\begin{verbatim}
Input: A = [0,0,1,1], B = [1,0,2,1]
Output: 2
Explanation:
Rectangles touch at the edge but do not overlap.
Area of A = 1
Area of B = 1
Overlapping Area = 0
Total Area = 1 + 1 = 2
\end{verbatim}

\textbf{Constraints:}

\begin{itemize}
    \item All coordinates are integers in the range \([-10^9, 10^9]\).
    \item For each rectangle, \(x1 < x2\) and \(y1 < y2\).
\end{itemize}

LeetCode link: \href{https://leetcode.com/problems/rectangle-area/}{Rectangle Area}\index{LeetCode}

\section*{Algorithmic Approach}

To compute the total area covered by two axis-aligned rectangles, we can follow these steps:

1. **Calculate Individual Areas:**
   - Compute the area of the first rectangle.
   - Compute the area of the second rectangle.

2. **Determine Overlapping Area:**
   - Calculate the coordinates of the overlapping rectangle, if any.
   - If the rectangles overlap, compute the area of the overlapping region.

3. **Compute Total Area:**
   - Sum the individual areas and subtract the overlapping area to avoid double-counting.

\marginnote{This approach ensures accurate area calculation by handling overlapping regions appropriately.}

\section*{Complexities}

\begin{itemize}
    \item \textbf{Time Complexity:} \(O(1)\). The algorithm performs a constant number of calculations.
    
    \item \textbf{Space Complexity:} \(O(1)\). Only a fixed amount of extra space is used for variables.
\end{itemize}

\section*{Python Implementation}

\marginnote{Implementing the area calculation with overlap consideration ensures an accurate and efficient solution.}

Below is the complete Python code implementing the \texttt{computeArea} function:

\begin{fullwidth}
\begin{lstlisting}[language=Python]
from typing import List

class Solution:
    def computeArea(self, A: List[int], B: List[int]) -> int:
        # Calculate area of rectangle A
        areaA = (A[2] - A[0]) * (A[3] - A[1])
        
        # Calculate area of rectangle B
        areaB = (B[2] - B[0]) * (B[3] - B[1])
        
        # Determine overlap coordinates
        overlap_x1 = max(A[0], B[0])
        overlap_y1 = max(A[1], B[1])
        overlap_x2 = min(A[2], B[2])
        overlap_y2 = min(A[3], B[3])
        
        # Calculate overlapping area
        overlap_width = overlap_x2 - overlap_x1
        overlap_height = overlap_y2 - overlap_y1
        overlap_area = 0
        if overlap_width > 0 and overlap_height > 0:
            overlap_area = overlap_width * overlap_height
        
        # Total area is sum of individual areas minus overlapping area
        total_area = areaA + areaB - overlap_area
        return total_area

# Example usage:
solution = Solution()
print(solution.computeArea([-3,0,3,4], [0,-1,9,2]))  # Output: 45
print(solution.computeArea([0,0,0,0], [0,0,0,0]))    # Output: 0
print(solution.computeArea([0,0,2,2], [1,1,3,3]))    # Output: 7
print(solution.computeArea([0,0,1,1], [1,0,2,1]))    # Output: 2
\end{lstlisting}
\end{fullwidth}

This implementation accurately computes the total area covered by two rectangles by accounting for any overlapping regions. It ensures that the overlapping area is not double-counted.

\section*{Explanation}

The \texttt{computeArea} function calculates the combined area of two axis-aligned rectangles by following these steps:

\subsection*{1. Calculate Individual Areas}

\begin{itemize}
    \item **Rectangle A:**
    \begin{itemize}
        \item Width: \(A[2] - A[0]\)
        \item Height: \(A[3] - A[1]\)
        \item Area: Width \(\times\) Height
    \end{itemize}
    
    \item **Rectangle B:**
    \begin{itemize}
        \item Width: \(B[2] - B[0]\)
        \item Height: \(B[3] - B[1]\)
        \item Area: Width \(\times\) Height
    \end{itemize}
\end{itemize}

\subsection*{2. Determine Overlapping Area}

\begin{itemize}
    \item **Overlap Coordinates:**
    \begin{itemize}
        \item Left (x-coordinate): \(\text{max}(A[0], B[0])\)
        \item Bottom (y-coordinate): \(\text{max}(A[1], B[1])\)
        \item Right (x-coordinate): \(\text{min}(A[2], B[2])\)
        \item Top (y-coordinate): \(\text{min}(A[3], B[3])\)
    \end{itemize}
    
    \item **Overlap Dimensions:**
    \begin{itemize}
        \item Width: \(\text{overlap\_x2} - \text{overlap\_x1}\)
        \item Height: \(\text{overlap\_y2} - \text{overlap\_y1}\)
    \end{itemize}
    
    \item **Overlap Area:**
    \begin{itemize}
        \item If both width and height are positive, the rectangles overlap, and the overlapping area is their product.
        \item Otherwise, there is no overlap, and the overlapping area is zero.
    \end{itemize}
\end{itemize}

\subsection*{3. Compute Total Area}

\begin{itemize}
    \item Total Area = Area of Rectangle A + Area of Rectangle B - Overlapping Area
\end{itemize}

\subsection*{4. Example Walkthrough}

Consider the first example:
\begin{verbatim}
Input: A = [-3,0,3,4], B = [0,-1,9,2]
Output: 45
\end{verbatim}

\begin{enumerate}
    \item **Calculate Areas:**
    \begin{itemize}
        \item Area of A = (3 - (-3)) * (4 - 0) = 6 * 4 = 24
        \item Area of B = (9 - 0) * (2 - (-1)) = 9 * 3 = 27
    \end{itemize}
    
    \item **Determine Overlap:**
    \begin{itemize}
        \item overlap\_x1 = max(-3, 0) = 0
        \item overlap\_y1 = max(0, -1) = 0
        \item overlap\_x2 = min(3, 9) = 3
        \item overlap\_y2 = min(4, 2) = 2
        \item overlap\_width = 3 - 0 = 3
        \item overlap\_height = 2 - 0 = 2
        \item overlap\_area = 3 * 2 = 6
    \end{itemize}
    
    \item **Compute Total Area:**
    \begin{itemize}
        \item Total Area = 24 + 27 - 6 = 45
    \end{itemize}
\end{enumerate}

Thus, the function correctly returns \texttt{45}.

\section*{Why This Approach}

This approach is chosen for its straightforwardness and optimal efficiency. By directly calculating the individual areas and intelligently handling the overlapping region, the algorithm ensures accurate results without unnecessary computations. Its constant time complexity makes it highly efficient, even for large coordinate values.

\section*{Alternative Approaches}

\subsection*{1. Using Intersection Dimensions}

Instead of separately calculating areas, directly compute the dimensions of the overlapping region and subtract it from the sum of individual areas.

\begin{lstlisting}[language=Python]
def computeArea(A: List[int], B: List[int]) -> int:
    # Sum of individual areas
    area = (A[2] - A[0]) * (A[3] - A[1]) + (B[2] - B[0]) * (B[3] - B[1])
    
    # Overlapping area
    overlap_width = min(A[2], B[2]) - max(A[0], B[0])
    overlap_height = min(A[3], B[3]) - max(A[1], B[1])
    
    if overlap_width > 0 and overlap_height > 0:
        area -= overlap_width * overlap_height
    
    return area
\end{lstlisting}

\subsection*{2. Using Geometry Libraries}

Leverage computational geometry libraries to handle area calculations and overlapping detections.

\begin{lstlisting}[language=Python]
from shapely.geometry import box

def computeArea(A: List[int], B: List[int]) -> int:
    rect1 = box(A[0], A[1], A[2], A[3])
    rect2 = box(B[0], B[1], B[2], B[3])
    intersection = rect1.intersection(rect2)
    return int(rect1.area + rect2.area - intersection.area)
\end{lstlisting}

\textbf{Note}: This approach requires the \texttt{shapely} library and is more suitable for complex geometric operations.

\section*{Similar Problems to This One}

Several problems involve calculating areas, handling geometric overlaps, and spatial reasoning, utilizing similar algorithmic strategies:

\begin{itemize}
    \item \textbf{Rectangle Overlap}: Determine if two rectangles overlap.
    \item \textbf{Circle Area Overlap}: Calculate the overlapping area between two circles.
    \item \textbf{Polygon Area}: Compute the area of a given polygon.
    \item \textbf{Union of Rectangles}: Calculate the total area covered by multiple rectangles, accounting for overlaps.
    \item \textbf{Intersection of Lines}: Find the intersection point of two lines.
    \item \textbf{Closest Pair of Points}: Find the closest pair of points in a set.
    \item \textbf{Convex Hull}: Compute the convex hull of a set of points.
    \item \textbf{Point Inside Polygon}: Determine if a point lies inside a given polygon.
\end{itemize}

These problems reinforce concepts of geometric calculations, area computations, and efficient algorithm design in various contexts.

\section*{Things to Keep in Mind and Tricks}

When tackling the \textbf{Rectangle Area} problem, consider the following tips and best practices to enhance efficiency and correctness:

\begin{itemize}
    \item \textbf{Understand Geometric Relationships}: Grasp the positional relationships between rectangles to simplify area calculations.
    \index{Geometric Relationships}
    
    \item \textbf{Leverage Coordinate Comparisons}: Use direct comparisons of rectangle coordinates to determine overlapping regions.
    \index{Coordinate Comparisons}
    
    \item \textbf{Handle Overlapping Scenarios}: Accurately calculate the overlapping area to avoid double-counting.
    \index{Overlapping Scenarios}
    
    \item \textbf{Optimize for Efficiency}: Aim for a constant time \(O(1)\) solution by avoiding unnecessary computations or iterations.
    \index{Efficiency Optimization}
    
    \item \textbf{Avoid Floating-Point Precision Issues}: Since all coordinates are integers, floating-point precision is not a concern, simplifying the implementation.
    \index{Floating-Point Precision}
    
    \item \textbf{Use Helper Functions}: Create helper functions to encapsulate repetitive tasks, such as calculating overlap dimensions or areas.
    \index{Helper Functions}
    
    \item \textbf{Code Readability}: Maintain clear and readable code through meaningful variable names and structured logic.
    \index{Code Readability}
    
    \item \textbf{Test Extensively}: Implement a wide range of test cases, including overlapping, non-overlapping, and edge-touching rectangles, to ensure robustness.
    \index{Extensive Testing}
    
    \item \textbf{Understand Axis-Aligned Constraints}: Recognize that axis-aligned rectangles simplify area calculations compared to rotated rectangles.
    \index{Axis-Aligned Constraints}
    
    \item \textbf{Simplify Logical Conditions}: Combine multiple conditions logically to streamline the area calculation process.
    \index{Logical Conditions}
    
    \item \textbf{Use Absolute Values}: When calculating differences, ensure that the dimensions are positive by using absolute values or proper ordering.
    \index{Absolute Values}
    
    \item \textbf{Consider Edge Cases}: Handle cases where rectangles have zero area or touch at edges/corners without overlapping.
    \index{Edge Cases}
\end{itemize}

\section*{Corner and Special Cases to Test When Writing the Code}

When implementing the solution for the \textbf{Rectangle Area} problem, it is crucial to consider and rigorously test various edge cases to ensure robustness and correctness:

\begin{itemize}
    \item \textbf{No Overlap}: Rectangles are completely separate.
    \index{No Overlap}
    
    \item \textbf{Partial Overlap}: Rectangles overlap in one or more regions.
    \index{Partial Overlap}
    
    \item \textbf{Edge Touching}: Rectangles touch exactly at one edge without overlapping.
    \index{Edge Touching}
    
    \item \textbf{Corner Touching}: Rectangles touch exactly at one corner without overlapping.
    \index{Corner Touching}
    
    \item \textbf{One Rectangle Inside Another}: One rectangle is entirely within the other.
    \index{Rectangle Inside}
    
    \item \textbf{Identical Rectangles}: Both rectangles have the same coordinates.
    \index{Identical Rectangles}
    
    \item \textbf{Degenerate Rectangles}: Rectangles with zero area (e.g., \(x1 = x2\) or \(y1 = y2\)).
    \index{Degenerate Rectangles}
    
    \item \textbf{Large Coordinates}: Rectangles with very large coordinate values to test performance and integer handling.
    \index{Large Coordinates}
    
    \item \textbf{Negative Coordinates}: Rectangles positioned in negative coordinate space.
    \index{Negative Coordinates}
    
    \item \textbf{Mixed Overlapping Scenarios}: Combinations of the above cases to ensure comprehensive coverage.
    \index{Mixed Overlapping Scenarios}
    
    \item \textbf{Minimum and Maximum Bounds}: Rectangles at the minimum and maximum limits of the coordinate range.
    \index{Minimum and Maximum Bounds}
    
    \item \textbf{Sequential Rectangles}: Multiple rectangles placed sequentially without overlapping.
    \index{Sequential Rectangles}
    
    \item \textbf{Multiple Overlaps}: Scenarios where more than two rectangles overlap in different regions.
    \index{Multiple Overlaps}
\end{itemize}

\section*{Implementation Considerations}

When implementing the \texttt{computeArea} function, keep in mind the following considerations to ensure robustness and efficiency:

\begin{itemize}
    \item \textbf{Data Type Selection}: Use appropriate data types that can handle large input values without overflow or underflow.
    \index{Data Type Selection}
    
    \item \textbf{Optimizing Comparisons}: Structure logical conditions to efficiently determine overlap dimensions.
    \index{Optimizing Comparisons}
    
    \item \textbf{Handling Large Inputs}: Design the algorithm to efficiently handle large input sizes without significant performance degradation.
    \index{Handling Large Inputs}
    
    \item \textbf{Language-Specific Constraints}: Be aware of how the programming language handles large integers and arithmetic operations.
    \index{Language-Specific Constraints}
    
    \item \textbf{Avoiding Redundant Calculations}: Ensure that each calculation contributes towards determining the final area without unnecessary repetitions.
    \index{Avoiding Redundant Calculations}
    
    \item \textbf{Code Readability and Documentation}: Maintain clear and readable code through meaningful variable names and comprehensive comments to facilitate understanding and maintenance.
    \index{Code Readability}
    
    \item \textbf{Edge Case Handling}: Implement checks for edge cases to prevent incorrect results or runtime errors.
    \index{Edge Case Handling}
    
    \item \textbf{Testing and Validation}: Develop a comprehensive suite of test cases that cover all possible scenarios, including edge cases, to validate the correctness and efficiency of the implementation.
    \index{Testing and Validation}
    
    \item \textbf{Scalability}: Design the algorithm to scale efficiently with increasing input sizes, maintaining performance and resource utilization.
    \index{Scalability}
    
    \item \textbf{Using Helper Functions}: Consider creating helper functions for repetitive tasks, such as calculating overlap dimensions, to enhance modularity and reusability.
    \index{Helper Functions}
    
    \item \textbf{Consistent Naming Conventions}: Use consistent and descriptive naming conventions for variables to improve code clarity.
    \index{Naming Conventions}
    
    \item \textbf{Implementing Unit Tests}: Develop unit tests for each logical condition to ensure that all scenarios are correctly handled.
    \index{Unit Tests}
    
    \item \textbf{Error Handling}: Incorporate error handling to manage invalid inputs gracefully.
    \index{Error Handling}
\end{itemize}

\section*{Conclusion}

The \textbf{Rectangle Area} problem showcases the application of fundamental geometric principles and efficient algorithm design to compute spatial properties accurately. By systematically calculating individual areas and intelligently handling overlapping regions, the algorithm ensures precise results without redundant computations. Understanding and implementing such techniques not only enhances problem-solving skills but also provides a foundation for tackling more complex Computational Geometry challenges involving multiple geometric entities and intricate spatial relationships.

\printindex

% \input{sections/rectangle_overlap}
% \input{sections/rectangle_area}
% \input{sections/k_closest_points_to_origin}
% \input{sections/the_skyline_problem}
% % filename: k_closest_points_to_origin.tex

\problemsection{K Closest Points to Origin}
\label{chap:K_Closest_Points_to_Origin}
\marginnote{\href{https://leetcode.com/problems/k-closest-points-to-origin/}{[LeetCode Link]}\index{LeetCode}}
\marginnote{\href{https://www.geeksforgeeks.org/find-k-closest-points-origin/}{[GeeksForGeeks Link]}\index{GeeksForGeeks}}
\marginnote{\href{https://www.interviewbit.com/problems/k-closest-points/}{[InterviewBit Link]}\index{InterviewBit}}
\marginnote{\href{https://app.codesignal.com/challenges/k-closest-points-to-origin}{[CodeSignal Link]}\index{CodeSignal}}
\marginnote{\href{https://www.codewars.com/kata/k-closest-points-to-origin/train/python}{[Codewars Link]}\index{Codewars}}

The \textbf{K Closest Points to Origin} problem is a popular algorithmic challenge in Computational Geometry that involves identifying the \(k\) points closest to the origin in a 2D plane. This problem tests one's ability to apply efficient sorting and selection algorithms, understand distance computations, and optimize for performance. Mastery of this problem is essential for applications in spatial data analysis, nearest neighbor searches, and clustering algorithms.

\section*{Problem Statement}

Given an array of points where each point is represented as \([x, y]\) in the 2D plane, and an integer \(k\), return the \(k\) closest points to the origin \((0, 0)\).

The distance between two points \((x_1, y_1)\) and \((x_2, y_2)\) is the Euclidean distance \(\sqrt{(x_1 - x_2)^2 + (y_1 - y_2)^2}\). The origin is \((0, 0)\).

\textbf{Function signature in Python:}
\begin{lstlisting}[language=Python]
def kClosest(points: List[List[int]], K: int) -> List[List[int]]:
\end{lstlisting}

\section*{Examples}

\textbf{Example 1:}

\begin{verbatim}
Input: points = [[1,3],[-2,2]], K = 1
Output: [[-2,2]]
Explanation: 
The distance between (1, 3) and the origin is sqrt(10).
The distance between (-2, 2) and the origin is sqrt(8).
Since sqrt(8) < sqrt(10), (-2, 2) is closer to the origin.
\end{verbatim}

\textbf{Example 2:}

\begin{verbatim}
Input: points = [[3,3],[5,-1],[-2,4]], K = 2
Output: [[3,3],[-2,4]]
Explanation: 
The distances are sqrt(18), sqrt(26), and sqrt(20) respectively.
The two closest points are [3,3] and [-2,4].
\end{verbatim}

\textbf{Example 3:}

\begin{verbatim}
Input: points = [[0,1],[1,0]], K = 2
Output: [[0,1],[1,0]]
Explanation: 
Both points are equally close to the origin.
\end{verbatim}

\textbf{Example 4:}

\begin{verbatim}
Input: points = [[1,0],[0,1]], K = 1
Output: [[1,0]]
Explanation: 
Both points are equally close; returning any one is acceptable.
\end{verbatim}

\textbf{Constraints:}

\begin{itemize}
    \item \(1 \leq K \leq \text{points.length} \leq 10^4\)
    \item \(-10^4 < x_i, y_i < 10^4\)
\end{itemize}

LeetCode link: \href{https://leetcode.com/problems/k-closest-points-to-origin/}{K Closest Points to Origin}\index{LeetCode}

\section*{Algorithmic Approach}

To identify the \(k\) closest points to the origin, several algorithmic strategies can be employed. The most efficient methods aim to reduce the time complexity by avoiding the need to sort the entire list of points.

\subsection*{1. Sorting Based on Distance}

Calculate the Euclidean distance of each point from the origin and sort the points based on these distances. Select the first \(k\) points from the sorted list.

\begin{enumerate}
    \item Compute the distance for each point using the formula \(distance = x^2 + y^2\).
    \item Sort the points based on the computed distances.
    \item Return the first \(k\) points from the sorted list.
\end{enumerate}

\subsection*{2. Max Heap (Priority Queue)}

Use a max heap to maintain the \(k\) closest points. Iterate through each point, add it to the heap, and if the heap size exceeds \(k\), remove the farthest point.

\begin{enumerate}
    \item Initialize a max heap.
    \item For each point, compute its distance and add it to the heap.
    \item If the heap size exceeds \(k\), remove the point with the largest distance.
    \item After processing all points, the heap contains the \(k\) closest points.
\end{enumerate}

\subsection*{3. QuickSelect (Quick Sort Partitioning)}

Utilize the QuickSelect algorithm to find the \(k\) closest points without fully sorting the list.

\begin{enumerate}
    \item Choose a pivot point and partition the list based on distances relative to the pivot.
    \item Recursively apply QuickSelect to the partition containing the \(k\) closest points.
    \item Once the \(k\) closest points are identified, return them.
\end{enumerate}

\marginnote{QuickSelect offers an average time complexity of \(O(n)\), making it highly efficient for large datasets.}

\section*{Complexities}

\begin{itemize}
    \item \textbf{Sorting Based on Distance:}
    \begin{itemize}
        \item \textbf{Time Complexity:} \(O(n \log n)\)
        \item \textbf{Space Complexity:} \(O(n)\)
    \end{itemize}
    
    \item \textbf{Max Heap (Priority Queue):}
    \begin{itemize}
        \item \textbf{Time Complexity:} \(O(n \log k)\)
        \item \textbf{Space Complexity:} \(O(k)\)
    \end{itemize}
    
    \item \textbf{QuickSelect (Quick Sort Partitioning):}
    \begin{itemize}
        \item \textbf{Time Complexity:} Average case \(O(n)\), worst case \(O(n^2)\)
        \item \textbf{Space Complexity:} \(O(1)\) (in-place)
    \end{itemize}
\end{itemize}

\section*{Python Implementation}

\marginnote{Implementing QuickSelect provides an optimal average-case solution with linear time complexity.}

Below is the complete Python code implementing the \texttt{kClosest} function using the QuickSelect approach:

\begin{fullwidth}
\begin{lstlisting}[language=Python]
from typing import List
import random

class Solution:
    def kClosest(self, points: List[List[int]], K: int) -> List[List[int]]:
        def quickselect(left, right, K_smallest):
            if left == right:
                return
            
            # Select a random pivot_index
            pivot_index = random.randint(left, right)
            
            # Partition the array
            pivot_index = partition(left, right, pivot_index)
            
            # The pivot is in its final sorted position
            if K_smallest == pivot_index:
                return
            elif K_smallest < pivot_index:
                quickselect(left, pivot_index - 1, K_smallest)
            else:
                quickselect(pivot_index + 1, right, K_smallest)
        
        def partition(left, right, pivot_index):
            pivot_distance = distance(points[pivot_index])
            # Move pivot to end
            points[pivot_index], points[right] = points[right], points[pivot_index]
            store_index = left
            for i in range(left, right):
                if distance(points[i]) < pivot_distance:
                    points[store_index], points[i] = points[i], points[store_index]
                    store_index += 1
            # Move pivot to its final place
            points[right], points[store_index] = points[store_index], points[right]
            return store_index
        
        def distance(point):
            return point[0] ** 2 + point[1] ** 2
        
        n = len(points)
        quickselect(0, n - 1, K)
        return points[:K]

# Example usage:
solution = Solution()
print(solution.kClosest([[1,3],[-2,2]], 1))            # Output: [[-2,2]]
print(solution.kClosest([[3,3],[5,-1],[-2,4]], 2))     # Output: [[3,3],[-2,4]]
print(solution.kClosest([[0,1],[1,0]], 2))             # Output: [[0,1],[1,0]]
print(solution.kClosest([[1,0],[0,1]], 1))             # Output: [[1,0]] or [[0,1]]
\end{lstlisting}
\end{fullwidth}

This implementation uses the QuickSelect algorithm to efficiently find the \(k\) closest points to the origin without fully sorting the entire list. It ensures optimal performance even with large datasets.

\section*{Explanation}

The \texttt{kClosest} function identifies the \(k\) closest points to the origin using the QuickSelect algorithm. Here's a detailed breakdown of the implementation:

\subsection*{1. Distance Calculation}

\begin{itemize}
    \item The Euclidean distance is calculated as \(distance = x^2 + y^2\). Since we only need relative distances for comparison, the square root is omitted for efficiency.
\end{itemize}

\subsection*{2. QuickSelect Algorithm}

\begin{itemize}
    \item **Pivot Selection:**
    \begin{itemize}
        \item A random pivot is chosen to enhance the average-case performance.
    \end{itemize}
    
    \item **Partitioning:**
    \begin{itemize}
        \item The array is partitioned such that points with distances less than the pivot are moved to the left, and others to the right.
        \item The pivot is placed in its correct sorted position.
    \end{itemize}
    
    \item **Recursive Selection:**
    \begin{itemize}
        \item If the pivot's position matches \(K\), the selection is complete.
        \item Otherwise, recursively apply QuickSelect to the relevant partition.
    \end{itemize}
\end{itemize}

\subsection*{3. Final Selection}

\begin{itemize}
    \item After partitioning, the first \(K\) points in the list are the \(k\) closest points to the origin.
\end{itemize}

\subsection*{4. Example Walkthrough}

Consider the first example:
\begin{verbatim}
Input: points = [[1,3],[-2,2]], K = 1
Output: [[-2,2]]
\end{verbatim}

\begin{enumerate}
    \item **Calculate Distances:**
    \begin{itemize}
        \item [1,3] : \(1^2 + 3^2 = 10\)
        \item [-2,2] : \((-2)^2 + 2^2 = 8\)
    \end{itemize}
    
    \item **QuickSelect Process:**
    \begin{itemize}
        \item Choose a pivot, say [1,3] with distance 10.
        \item Compare and rearrange:
        \begin{itemize}
            \item [-2,2] has a smaller distance (8) and is moved to the left.
        \end{itemize}
        \item After partitioning, the list becomes [[-2,2], [1,3]].
        \item Since \(K = 1\), return the first point: [[-2,2]].
    \end{itemize}
\end{enumerate}

Thus, the function correctly identifies \([-2,2]\) as the closest point to the origin.

\section*{Why This Approach}

The QuickSelect algorithm is chosen for its average-case linear time complexity \(O(n)\), making it highly efficient for large datasets compared to sorting-based methods with \(O(n \log n)\) time complexity. By avoiding the need to sort the entire list, QuickSelect provides an optimal solution for finding the \(k\) closest points.

\section*{Alternative Approaches}

\subsection*{1. Sorting Based on Distance}

Sort all points based on their distances from the origin and select the first \(k\) points.

\begin{lstlisting}[language=Python]
class Solution:
    def kClosest(self, points: List[List[int]], K: int) -> List[List[int]]:
        points.sort(key=lambda P: P[0]**2 + P[1]**2)
        return points[:K]
\end{lstlisting}

\textbf{Complexities:}
\begin{itemize}
    \item \textbf{Time Complexity:} \(O(n \log n)\)
    \item \textbf{Space Complexity:} \(O(1)\)
\end{itemize}

\subsection*{2. Max Heap (Priority Queue)}

Use a max heap to maintain the \(k\) closest points.

\begin{lstlisting}[language=Python]
import heapq

class Solution:
    def kClosest(self, points: List[List[int]], K: int) -> List[List[int]]:
        heap = []
        for (x, y) in points:
            dist = -(x**2 + y**2)  # Max heap using negative distances
            heapq.heappush(heap, (dist, [x, y]))
            if len(heap) > K:
                heapq.heappop(heap)
        return [item[1] for item in heap]
\end{lstlisting}

\textbf{Complexities:}
\begin{itemize}
    \item \textbf{Time Complexity:} \(O(n \log k)\)
    \item \textbf{Space Complexity:} \(O(k)\)
\end{itemize}

\subsection*{3. Using Built-In Functions}

Leverage built-in functions for distance calculation and selection.

\begin{lstlisting}[language=Python]
import math

class Solution:
    def kClosest(self, points: List[List[int]], K: int) -> List[List[int]]:
        points.sort(key=lambda P: math.sqrt(P[0]**2 + P[1]**2))
        return points[:K]
\end{lstlisting}

\textbf{Note}: This method is similar to the sorting approach but uses the actual Euclidean distance.

\section*{Similar Problems to This One}

Several problems involve nearest neighbor searches, spatial data analysis, and efficient selection algorithms, utilizing similar algorithmic strategies:

\begin{itemize}
    \item \textbf{Closest Pair of Points}: Find the closest pair of points in a set.
    \item \textbf{Top K Frequent Elements}: Identify the most frequent elements in a dataset.
    \item \textbf{Kth Largest Element in an Array}: Find the \(k\)-th largest element in an unsorted array.
    \item \textbf{Sliding Window Maximum}: Find the maximum in each sliding window of size \(k\) over an array.
    \item \textbf{Merge K Sorted Lists}: Merge multiple sorted lists into a single sorted list.
    \item \textbf{Find Median from Data Stream}: Continuously find the median of a stream of numbers.
    \item \textbf{Top K Closest Stars}: Find the \(k\) closest stars to Earth based on their distances.
\end{itemize}

These problems reinforce concepts of efficient selection, heap usage, and distance computations in various contexts.

\section*{Things to Keep in Mind and Tricks}

When solving the \textbf{K Closest Points to Origin} problem, consider the following tips and best practices to enhance efficiency and correctness:

\begin{itemize}
    \item \textbf{Understand Distance Calculations}: Grasp the Euclidean distance formula and recognize that the square root can be omitted for comparison purposes.
    \index{Distance Calculations}
    
    \item \textbf{Leverage Efficient Algorithms}: Use QuickSelect or heap-based methods to optimize time complexity, especially for large datasets.
    \index{Efficient Algorithms}
    
    \item \textbf{Handle Ties Appropriately}: Decide how to handle points with identical distances when \(k\) is less than the number of such points.
    \index{Handling Ties}
    
    \item \textbf{Optimize Space Usage}: Choose algorithms that minimize additional space, such as in-place QuickSelect.
    \index{Space Optimization}
    
    \item \textbf{Use Appropriate Data Structures}: Utilize heaps, lists, and helper functions effectively to manage and process data.
    \index{Data Structures}
    
    \item \textbf{Implement Helper Functions}: Create helper functions for distance calculation and partitioning to enhance code modularity.
    \index{Helper Functions}
    
    \item \textbf{Code Readability}: Maintain clear and readable code through meaningful variable names and structured logic.
    \index{Code Readability}
    
    \item \textbf{Test Extensively}: Implement a wide range of test cases, including edge cases like multiple points with the same distance, to ensure robustness.
    \index{Extensive Testing}
    
    \item \textbf{Understand Algorithm Trade-offs}: Recognize the trade-offs between different approaches in terms of time and space complexities.
    \index{Algorithm Trade-offs}
    
    \item \textbf{Use Built-In Sorting Functions}: When using sorting-based approaches, leverage built-in functions for efficiency and simplicity.
    \index{Built-In Sorting}
    
    \item \textbf{Avoid Redundant Calculations}: Ensure that distance calculations are performed only when necessary to optimize performance.
    \index{Avoiding Redundant Calculations}
    
    \item \textbf{Language-Specific Features}: Utilize language-specific features or libraries that can simplify implementation, such as heapq in Python.
    \index{Language-Specific Features}
\end{itemize}

\section*{Corner and Special Cases to Test When Writing the Code}

When implementing the solution for the \textbf{K Closest Points to Origin} problem, it is crucial to consider and rigorously test various edge cases to ensure robustness and correctness:

\begin{itemize}
    \item \textbf{Multiple Points with Same Distance}: Ensure that the algorithm handles multiple points having the same distance from the origin.
    \index{Same Distance Points}
    
    \item \textbf{Points at Origin}: Include points that are exactly at the origin \((0,0)\).
    \index{Points at Origin}
    
    \item \textbf{Negative Coordinates}: Ensure that the algorithm correctly computes distances for points with negative \(x\) or \(y\) coordinates.
    \index{Negative Coordinates}
    
    \item \textbf{Large Coordinates}: Test with points having very large or very small coordinate values to verify integer handling.
    \index{Large Coordinates}
    
    \item \textbf{K Equals Number of Points}: When \(K\) is equal to the number of points, the algorithm should return all points.
    \index{K Equals Number of Points}
    
    \item \textbf{K is One}: Test with \(K = 1\) to ensure the closest point is correctly identified.
    \index{K is One}
    
    \item \textbf{All Points Same}: All points have the same coordinates.
    \index{All Points Same}
    
    \item \textbf{K is Zero}: Although \(K\) is defined to be at least 1, ensure that the algorithm gracefully handles \(K = 0\) if allowed.
    \index{K is Zero}
    
    \item \textbf{Single Point}: Only one point is provided, and \(K = 1\).
    \index{Single Point}
    
    \item \textbf{Mixed Coordinates}: Points with a mix of positive and negative coordinates.
    \index{Mixed Coordinates}
    
    \item \textbf{Points with Zero Distance}: Multiple points at the origin.
    \index{Zero Distance Points}
    
    \item \textbf{Sparse and Dense Points}: Densely packed points and sparsely distributed points.
    \index{Sparse and Dense Points}
    
    \item \textbf{Duplicate Points}: Multiple identical points in the input list.
    \index{Duplicate Points}
    
    \item \textbf{K Greater Than Number of Unique Points}: Ensure that the algorithm handles cases where \(K\) exceeds the number of unique points if applicable.
    \index{K Greater Than Unique Points}
\end{itemize}

\section*{Implementation Considerations}

When implementing the \texttt{kClosest} function, keep in mind the following considerations to ensure robustness and efficiency:

\begin{itemize}
    \item \textbf{Data Type Selection}: Use appropriate data types that can handle large input values without overflow or precision loss.
    \index{Data Type Selection}
    
    \item \textbf{Optimizing Distance Calculations}: Avoid calculating the square root since it is unnecessary for comparison purposes.
    \index{Optimizing Distance Calculations}
    
    \item \textbf{Choosing the Right Algorithm}: Select an algorithm based on the size of the input and the value of \(K\) to optimize time and space complexities.
    \index{Choosing the Right Algorithm}
    
    \item \textbf{Language-Specific Libraries}: Utilize language-specific libraries and functions (e.g., \texttt{heapq} in Python) to simplify implementation and enhance performance.
    \index{Language-Specific Libraries}
    
    \item \textbf{Avoiding Redundant Calculations}: Ensure that each point's distance is calculated only once to optimize performance.
    \index{Avoiding Redundant Calculations}
    
    \item \textbf{Implementing Helper Functions}: Create helper functions for tasks like distance calculation and partitioning to enhance modularity and readability.
    \index{Helper Functions}
    
    \item \textbf{Edge Case Handling}: Implement checks for edge cases to prevent incorrect results or runtime errors.
    \index{Edge Case Handling}
    
    \item \textbf{Testing and Validation}: Develop a comprehensive suite of test cases that cover all possible scenarios, including edge cases, to validate the correctness and efficiency of the implementation.
    \index{Testing and Validation}
    
    \item \textbf{Scalability}: Design the algorithm to scale efficiently with increasing input sizes, maintaining performance and resource utilization.
    \index{Scalability}
    
    \item \textbf{Consistent Naming Conventions}: Use consistent and descriptive naming conventions for variables and functions to improve code clarity.
    \index{Naming Conventions}
    
    \item \textbf{Memory Management}: Ensure that the algorithm manages memory efficiently, especially when dealing with large datasets.
    \index{Memory Management}
    
    \item \textbf{Avoiding Stack Overflow}: If implementing recursive approaches, be mindful of recursion limits and potential stack overflow issues.
    \index{Avoiding Stack Overflow}
    
    \item \textbf{Implementing Iterative Solutions}: Prefer iterative solutions when recursion may lead to increased space complexity or stack overflow.
    \index{Implementing Iterative Solutions}
\end{itemize}

\section*{Conclusion}

The \textbf{K Closest Points to Origin} problem exemplifies the application of efficient selection algorithms and geometric computations to solve spatial challenges effectively. By leveraging QuickSelect or heap-based methods, the algorithm achieves optimal time and space complexities, making it highly suitable for large datasets. Understanding and implementing such techniques not only enhances problem-solving skills but also provides a foundation for tackling more advanced Computational Geometry problems involving nearest neighbor searches, clustering, and spatial data analysis.

\printindex

% \input{sections/rectangle_overlap}
% \input{sections/rectangle_area}
% \input{sections/k_closest_points_to_origin}
% \input{sections/the_skyline_problem}
% % filename: the_skyline_problem.tex

\problemsection{The Skyline Problem}
\label{chap:The_Skyline_Problem}
\marginnote{\href{https://leetcode.com/problems/the-skyline-problem/}{[LeetCode Link]}\index{LeetCode}}
\marginnote{\href{https://www.geeksforgeeks.org/the-skyline-problem/}{[GeeksForGeeks Link]}\index{GeeksForGeeks}}
\marginnote{\href{https://www.interviewbit.com/problems/the-skyline-problem/}{[InterviewBit Link]}\index{InterviewBit}}
\marginnote{\href{https://app.codesignal.com/challenges/the-skyline-problem}{[CodeSignal Link]}\index{CodeSignal}}
\marginnote{\href{https://www.codewars.com/kata/the-skyline-problem/train/python}{[Codewars Link]}\index{Codewars}}

The \textbf{Skyline Problem} is a complex Computational Geometry challenge that involves computing the skyline formed by a collection of buildings in a 2D cityscape. Each building is represented by its left and right x-coordinates and its height. The skyline is defined by a list of "key points" where the height changes. This problem tests one's ability to handle large datasets, implement efficient sweep line algorithms, and manage event-driven processing. Mastery of this problem is essential for applications in computer graphics, urban planning simulations, and geographic information systems (GIS).

\section*{Problem Statement}

You are given a list of buildings in a cityscape. Each building is represented as a triplet \([Li, Ri, Hi]\), where \(Li\) and \(Ri\) are the x-coordinates of the left and right edges of the building, respectively, and \(Hi\) is the height of the building.

The skyline should be represented as a list of key points \([x, y]\) in sorted order by \(x\)-coordinate, where \(y\) is the height of the skyline at that point. The skyline should only include critical points where the height changes.

\textbf{Function signature in Python:}
\begin{lstlisting}[language=Python]
def getSkyline(buildings: List[List[int]]) -> List[List[int]]:
\end{lstlisting}

\section*{Examples}

\textbf{Example 1:}

\begin{verbatim}
Input: buildings = [[2,9,10], [3,7,15], [5,12,12], [15,20,10], [19,24,8]]
Output: [[2,10], [3,15], [7,12], [12,0], [15,10], [20,8], [24,0]]
Explanation:
- At x=2, the first building starts, height=10.
- At x=3, the second building starts, height=15.
- At x=7, the second building ends, the third building is still ongoing, height=12.
- At x=12, the third building ends, height drops to 0.
- At x=15, the fourth building starts, height=10.
- At x=20, the fourth building ends, the fifth building is still ongoing, height=8.
- At x=24, the fifth building ends, height drops to 0.
\end{verbatim}

\textbf{Example 2:}

\begin{verbatim}
Input: buildings = [[0,2,3], [2,5,3]]
Output: [[0,3], [5,0]]
Explanation:
- The two buildings are contiguous and have the same height, so the skyline drops to 0 at x=5.
\end{verbatim}

\textbf{Example 3:}

\begin{verbatim}
Input: buildings = [[1,3,3], [2,4,4], [5,6,1]]
Output: [[1,3], [2,4], [4,0], [5,1], [6,0]]
Explanation:
- At x=1, first building starts, height=3.
- At x=2, second building starts, height=4.
- At x=4, second building ends, height drops to 0.
- At x=5, third building starts, height=1.
- At x=6, third building ends, height drops to 0.
\end{verbatim}

\textbf{Example 4:}

\begin{verbatim}
Input: buildings = [[0,5,0]]
Output: []
Explanation:
- A building with height 0 does not contribute to the skyline.
\end{verbatim}

\textbf{Constraints:}

\begin{itemize}
    \item \(1 \leq \text{buildings.length} \leq 10^4\)
    \item \(0 \leq Li < Ri \leq 10^9\)
    \item \(0 \leq Hi \leq 10^4\)
\end{itemize}

\section*{Algorithmic Approach}

The \textbf{Sweep Line Algorithm} is an efficient method for solving the Skyline Problem. It involves processing events (building start and end points) in sorted order while maintaining a data structure (typically a max heap) to keep track of active buildings. Here's a step-by-step approach:

\subsection*{1. Event Representation}

Transform each building into two events:
\begin{itemize}
    \item **Start Event:** \((Li, -Hi)\) – Negative height indicates a building starts.
    \item **End Event:** \((Ri, Hi)\) – Positive height indicates a building ends.
\end{itemize}

Sorting the events ensures that start events are processed before end events at the same x-coordinate, and taller buildings are processed before shorter ones.

\subsection*{2. Sorting the Events}

Sort all events based on:
\begin{enumerate}
    \item **x-coordinate:** Ascending order.
    \item **Height:**
    \begin{itemize}
        \item For start events, taller buildings come first.
        \item For end events, shorter buildings come first.
    \end{itemize}
\end{enumerate}

\subsection*{3. Processing the Events}

Use a max heap to keep track of active building heights. Iterate through the sorted events:
\begin{enumerate}
    \item **Start Event:**
    \begin{itemize}
        \item Add the building's height to the heap.
    \end{itemize}
    
    \item **End Event:**
    \begin{itemize}
        \item Remove the building's height from the heap.
    \end{itemize}
    
    \item **Determine Current Max Height:**
    \begin{itemize}
        \item The current max height is the top of the heap.
    \end{itemize}
    
    \item **Update Skyline:**
    \begin{itemize}
        \item If the current max height differs from the previous max height, add a new key point \([x, current\_max\_height]\).
    \end{itemize}
\end{enumerate}

\subsection*{4. Finalizing the Skyline}

After processing all events, the accumulated key points represent the skyline.

\marginnote{The Sweep Line Algorithm efficiently handles dynamic changes in active buildings, ensuring accurate skyline construction.}

\section*{Complexities}

\begin{itemize}
    \item \textbf{Time Complexity:} \(O(n \log n)\), where \(n\) is the number of buildings. Sorting the events takes \(O(n \log n)\), and each heap operation takes \(O(\log n)\).
    
    \item \textbf{Space Complexity:} \(O(n)\), due to the storage of events and the heap.
\end{itemize}

\section*{Python Implementation}

\marginnote{Implementing the Sweep Line Algorithm with a max heap ensures an efficient and accurate solution.}

Below is the complete Python code implementing the \texttt{getSkyline} function:

\begin{fullwidth}
\begin{lstlisting}[language=Python]
from typing import List
import heapq

class Solution:
    def getSkyline(self, buildings: List[List[int]]) -> List[List[int]]:
        # Create a list of all events
        # For start events, use negative height to ensure they are processed before end events
        events = []
        for L, R, H in buildings:
            events.append((L, -H))
            events.append((R, H))
        
        # Sort the events
        # First by x-coordinate, then by height
        events.sort()
        
        # Max heap to keep track of active buildings
        heap = [0]  # Initialize with ground level
        heapq.heapify(heap)
        active_heights = {0: 1}  # Dictionary to count heights
        
        result = []
        prev_max = 0
        
        for x, h in events:
            if h < 0:
                # Start of a building, add height to heap and dictionary
                heapq.heappush(heap, h)
                active_heights[h] = active_heights.get(h, 0) + 1
            else:
                # End of a building, remove height from dictionary
                active_heights[h] -= 1
                if active_heights[h] == 0:
                    del active_heights[h]
            
            # Current max height
            while heap and active_heights.get(heap[0], 0) == 0:
                heapq.heappop(heap)
            current_max = -heap[0] if heap else 0
            
            # If the max height has changed, add to result
            if current_max != prev_max:
                result.append([x, current_max])
                prev_max = current_max
        
        return result

# Example usage:
solution = Solution()
print(solution.getSkyline([[2,9,10], [3,7,15], [5,12,12], [15,20,10], [19,24,8]]))
# Output: [[2,10], [3,15], [7,12], [12,0], [15,10], [20,8], [24,0]]

print(solution.getSkyline([[0,2,3], [2,5,3]]))
# Output: [[0,3], [5,0]]

print(solution.getSkyline([[1,3,3], [2,4,4], [5,6,1]]))
# Output: [[1,3], [2,4], [4,0], [5,1], [6,0]]

print(solution.getSkyline([[0,5,0]]))
# Output: []
\end{lstlisting}
\end{fullwidth}

This implementation efficiently constructs the skyline by processing all building events in sorted order and maintaining active building heights using a max heap. It ensures that only critical points where the skyline changes are recorded.

\section*{Explanation}

The \texttt{getSkyline} function constructs the skyline formed by a set of buildings by leveraging the Sweep Line Algorithm and a max heap to track active buildings. Here's a detailed breakdown of the implementation:

\subsection*{1. Event Representation}

\begin{itemize}
    \item Each building is transformed into two events:
    \begin{itemize}
        \item **Start Event:** \((Li, -Hi)\) – Negative height indicates the start of a building.
        \item **End Event:** \((Ri, Hi)\) – Positive height indicates the end of a building.
    \end{itemize}
\end{itemize}

\subsection*{2. Sorting the Events}

\begin{itemize}
    \item Events are sorted primarily by their x-coordinate in ascending order.
    \item For events with the same x-coordinate:
    \begin{itemize}
        \item Start events (with negative heights) are processed before end events.
        \item Taller buildings are processed before shorter ones.
    \end{itemize}
\end{itemize}

\subsection*{3. Processing the Events}

\begin{itemize}
    \item **Heap Initialization:**
    \begin{itemize}
        \item A max heap is initialized with a ground level height of 0.
        \item A dictionary \texttt{active\_heights} tracks the count of active building heights.
    \end{itemize}
    
    \item **Iterating Through Events:**
    \begin{enumerate}
        \item **Start Event:**
        \begin{itemize}
            \item Add the building's height to the heap.
            \item Increment the count of the height in \texttt{active\_heights}.
        \end{itemize}
        
        \item **End Event:**
        \begin{itemize}
            \item Decrement the count of the building's height in \texttt{active\_heights}.
            \item If the count reaches zero, remove the height from the dictionary.
        \end{itemize}
        
        \item **Determine Current Max Height:**
        \begin{itemize}
            \item Remove heights from the heap that are no longer active.
            \item The current max height is the top of the heap.
        \end{itemize}
        
        \item **Update Skyline:**
        \begin{itemize}
            \item If the current max height differs from the previous max height, add a new key point \([x, current\_max\_height]\).
        \end{itemize}
    \end{enumerate}
\end{itemize}

\subsection*{4. Finalizing the Skyline}

\begin{itemize}
    \item After processing all events, the \texttt{result} list contains the key points defining the skyline.
\end{itemize}

\subsection*{5. Example Walkthrough}

Consider the first example:
\begin{verbatim}
Input: buildings = [[2,9,10], [3,7,15], [5,12,12], [15,20,10], [19,24,8]]
Output: [[2,10], [3,15], [7,12], [12,0], [15,10], [20,8], [24,0]]
\end{verbatim}

\begin{enumerate}
    \item **Event Transformation:**
    \begin{itemize}
        \item \((2, -10)\), \((9, 10)\)
        \item \((3, -15)\), \((7, 15)\)
        \item \((5, -12)\), \((12, 12)\)
        \item \((15, -10)\), \((20, 10)\)
        \item \((19, -8)\), \((24, 8)\)
    \end{itemize}
    
    \item **Sorting Events:**
    \begin{itemize}
        \item Sorted order: \((2, -10)\), \((3, -15)\), \((5, -12)\), \((7, 15)\), \((9, 10)\), \((12, 12)\), \((15, -10)\), \((19, -8)\), \((20, 10)\), \((24, 8)\)
    \end{itemize}
    
    \item **Processing Events:**
    \begin{itemize}
        \item At each event, update the heap and determine if the skyline height changes.
    \end{itemize}
    
    \item **Result Construction:**
    \begin{itemize}
        \item The resulting skyline key points are accumulated as \([[2,10], [3,15], [7,12], [12,0], [15,10], [20,8], [24,0]]\).
    \end{itemize}
\end{enumerate}

Thus, the function correctly constructs the skyline based on the buildings' positions and heights.

\section*{Why This Approach}

The Sweep Line Algorithm combined with a max heap offers an optimal solution with \(O(n \log n)\) time complexity and efficient handling of overlapping buildings. By processing events in sorted order and maintaining active building heights, the algorithm ensures that all critical points in the skyline are accurately identified without redundant computations.

\section*{Alternative Approaches}

\subsection*{1. Divide and Conquer}

Divide the set of buildings into smaller subsets, compute the skyline for each subset, and then merge the skylines.

\begin{lstlisting}[language=Python]
class Solution:
    def getSkyline(self, buildings: List[List[int]]) -> List[List[int]]:
        def merge(left, right):
            h1, h2 = 0, 0
            i, j = 0, 0
            merged = []
            while i < len(left) and j < len(right):
                if left[i][0] < right[j][0]:
                    x, h1 = left[i]
                    i += 1
                elif left[i][0] > right[j][0]:
                    x, h2 = right[j]
                    j += 1
                else:
                    x, h1 = left[i]
                    _, h2 = right[j]
                    i += 1
                    j += 1
                max_h = max(h1, h2)
                if not merged or merged[-1][1] != max_h:
                    merged.append([x, max_h])
            merged.extend(left[i:])
            merged.extend(right[j:])
            return merged
        
        def divide(buildings):
            if not buildings:
                return []
            if len(buildings) == 1:
                L, R, H = buildings[0]
                return [[L, H], [R, 0]]
            mid = len(buildings) // 2
            left = divide(buildings[:mid])
            right = divide(buildings[mid:])
            return merge(left, right)
        
        return divide(buildings)
\end{lstlisting}

\textbf{Complexities:}
\begin{itemize}
    \item \textbf{Time Complexity:} \(O(n \log n)\)
    \item \textbf{Space Complexity:} \(O(n)\)
\end{itemize}

\subsection*{2. Using Segment Trees}

Implement a segment tree to manage and query overlapping building heights dynamically.

\textbf{Note}: This approach is more complex and is generally used for advanced scenarios with multiple dynamic queries.

\section*{Similar Problems to This One}

Several problems involve skyline-like constructions, spatial data analysis, and efficient event processing, utilizing similar algorithmic strategies:

\begin{itemize}
    \item \textbf{Merge Intervals}: Merge overlapping intervals in a list.
    \item \textbf{Largest Rectangle in Histogram}: Find the largest rectangular area in a histogram.
    \item \textbf{Interval Partitioning}: Assign intervals to resources without overlap.
    \item \textbf{Line Segment Intersection}: Detect intersections among line segments.
    \item \textbf{Closest Pair of Points}: Find the closest pair of points in a set.
    \item \textbf{Convex Hull}: Compute the convex hull of a set of points.
    \item \textbf{Point Inside Polygon}: Determine if a point lies inside a given polygon.
    \item \textbf{Range Searching}: Efficiently query geometric data within a specified range.
\end{itemize}

These problems reinforce concepts of event-driven processing, spatial reasoning, and efficient algorithm design in various contexts.

\section*{Things to Keep in Mind and Tricks}

When tackling the \textbf{Skyline Problem}, consider the following tips and best practices to enhance efficiency and correctness:

\begin{itemize}
    \item \textbf{Understand Sweep Line Technique}: Grasp how the sweep line algorithm processes events in sorted order to handle dynamic changes efficiently.
    \index{Sweep Line Technique}
    
    \item \textbf{Leverage Priority Queues (Heaps)}: Use max heaps to keep track of active buildings' heights, enabling quick access to the current maximum height.
    \index{Priority Queues}
    
    \item \textbf{Handle Start and End Events Differently}: Differentiate between building start and end events to accurately manage active heights.
    \index{Start and End Events}
    
    \item \textbf{Optimize Event Sorting}: Sort events primarily by x-coordinate and secondarily by height to ensure correct processing order.
    \index{Event Sorting}
    
    \item \textbf{Manage Active Heights Efficiently}: Use data structures that allow efficient insertion, deletion, and retrieval of maximum elements.
    \index{Active Heights Management}
    
    \item \textbf{Avoid Redundant Key Points}: Only record key points when the skyline height changes to minimize the output list.
    \index{Avoiding Redundant Key Points}
    
    \item \textbf{Implement Helper Functions}: Create helper functions for tasks like distance calculation, event handling, and heap management to enhance modularity.
    \index{Helper Functions}
    
    \item \textbf{Code Readability}: Maintain clear and readable code through meaningful variable names and structured logic.
    \index{Code Readability}
    
    \item \textbf{Test Extensively}: Implement a wide range of test cases, including overlapping, non-overlapping, and edge-touching buildings, to ensure robustness.
    \index{Extensive Testing}
    
    \item \textbf{Handle Degenerate Cases}: Manage cases where buildings have zero height or identical coordinates gracefully.
    \index{Degenerate Cases}
    
    \item \textbf{Understand Geometric Relationships}: Grasp how buildings overlap and influence the skyline to simplify the algorithm.
    \index{Geometric Relationships}
    
    \item \textbf{Use Appropriate Data Structures}: Utilize appropriate data structures like heaps, lists, and dictionaries to manage and process data efficiently.
    \index{Appropriate Data Structures}
    
    \item \textbf{Optimize for Large Inputs}: Design the algorithm to handle large numbers of buildings without significant performance degradation.
    \index{Optimizing for Large Inputs}
    
    \item \textbf{Implement Iterative Solutions Carefully}: Ensure that loop conditions are correctly defined to prevent infinite loops or incorrect terminations.
    \index{Iterative Solutions}
    
    \item \textbf{Consistent Naming Conventions}: Use consistent and descriptive naming conventions for variables and functions to improve code clarity.
    \index{Naming Conventions}
\end{itemize}

\section*{Corner and Special Cases to Test When Writing the Code}

When implementing the solution for the \textbf{Skyline Problem}, it is crucial to consider and rigorously test various edge cases to ensure robustness and correctness:

\begin{itemize}
    \item \textbf{No Overlapping Buildings}: All buildings are separate and do not overlap.
    \index{No Overlapping Buildings}
    
    \item \textbf{Fully Overlapping Buildings}: Multiple buildings completely overlap each other.
    \index{Fully Overlapping Buildings}
    
    \item \textbf{Buildings Touching at Edges}: Buildings share common edges without overlapping.
    \index{Buildings Touching at Edges}
    
    \item \textbf{Buildings Touching at Corners}: Buildings share common corners without overlapping.
    \index{Buildings Touching at Corners}
    
    \item \textbf{Single Building}: Only one building is present.
    \index{Single Building}
    
    \item \textbf{Multiple Buildings with Same Start or End}: Multiple buildings start or end at the same x-coordinate.
    \index{Same Start or End}
    
    \item \textbf{Buildings with Zero Height}: Buildings that have zero height should not affect the skyline.
    \index{Buildings with Zero Height}
    
    \item \textbf{Large Number of Buildings}: Test with a large number of buildings to ensure performance and scalability.
    \index{Large Number of Buildings}
    
    \item \textbf{Buildings with Negative Coordinates}: Buildings positioned in negative coordinate space.
    \index{Negative Coordinates}
    
    \item \textbf{Boundary Values}: Buildings at the minimum and maximum limits of the coordinate range.
    \index{Boundary Values}
    
    \item \textbf{Buildings with Identical Coordinates}: Multiple buildings with the same coordinates.
    \index{Identical Coordinates}
    
    \item \textbf{Sequential Buildings}: Buildings placed sequentially without gaps.
    \index{Sequential Buildings}
    
    \item \textbf{Overlapping and Non-Overlapping Mixed}: A mix of overlapping and non-overlapping buildings.
    \index{Overlapping and Non-Overlapping Mixed}
    
    \item \textbf{Buildings with Very Large Heights}: Buildings with heights at the upper limit of the constraints.
    \index{Very Large Heights}
    
    \item \textbf{Empty Input}: No buildings are provided.
    \index{Empty Input}
\end{itemize}

\section*{Implementation Considerations}

When implementing the \texttt{getSkyline} function, keep in mind the following considerations to ensure robustness and efficiency:

\begin{itemize}
    \item \textbf{Data Type Selection}: Use appropriate data types that can handle large input values and avoid overflow or precision issues.
    \index{Data Type Selection}
    
    \item \textbf{Optimizing Event Sorting}: Efficiently sort events based on x-coordinates and heights to ensure correct processing order.
    \index{Optimizing Event Sorting}
    
    \item \textbf{Handling Large Inputs}: Design the algorithm to handle up to \(10^4\) buildings efficiently without significant performance degradation.
    \index{Handling Large Inputs}
    
    \item \textbf{Using Efficient Data Structures}: Utilize heaps, lists, and dictionaries effectively to manage and process events and active heights.
    \index{Efficient Data Structures}
    
    \item \textbf{Avoiding Redundant Calculations}: Ensure that distance and overlap calculations are performed only when necessary to optimize performance.
    \index{Avoiding Redundant Calculations}
    
    \item \textbf{Code Readability and Documentation}: Maintain clear and readable code through meaningful variable names and comprehensive comments to facilitate understanding and maintenance.
    \index{Code Readability}
    
    \item \textbf{Edge Case Handling}: Implement checks for edge cases to prevent incorrect results or runtime errors.
    \index{Edge Case Handling}
    
    \item \textbf{Implementing Helper Functions}: Create helper functions for tasks like distance calculation, event handling, and heap management to enhance modularity.
    \index{Helper Functions}
    
    \item \textbf{Consistent Naming Conventions}: Use consistent and descriptive naming conventions for variables and functions to improve code clarity.
    \index{Naming Conventions}
    
    \item \textbf{Memory Management}: Ensure that the algorithm manages memory efficiently, especially when dealing with large datasets.
    \index{Memory Management}
    
    \item \textbf{Implementing Iterative Solutions Carefully}: Ensure that loop conditions are correctly defined to prevent infinite loops or incorrect terminations.
    \index{Iterative Solutions}
    
    \item \textbf{Avoiding Floating-Point Precision Issues}: Since the problem deals with integers, floating-point precision is not a concern, simplifying the implementation.
    \index{Floating-Point Precision}
    
    \item \textbf{Testing and Validation}: Develop a comprehensive suite of test cases that cover all possible scenarios, including edge cases, to validate the correctness and efficiency of the implementation.
    \index{Testing and Validation}
    
    \item \textbf{Performance Considerations}: Optimize the loop conditions and operations to ensure that the function runs efficiently, especially for large input numbers.
    \index{Performance Considerations}
\end{itemize}

\section*{Conclusion}

The \textbf{Skyline Problem} is a quintessential example of applying advanced algorithmic techniques and geometric reasoning to solve complex spatial challenges. By leveraging the Sweep Line Algorithm and maintaining active building heights using a max heap, the solution efficiently constructs the skyline with optimal time and space complexities. Understanding and implementing such sophisticated algorithms not only enhances problem-solving skills but also provides a foundation for tackling a wide array of Computational Geometry problems in various domains, including computer graphics, urban planning simulations, and geographic information systems.

\printindex

% \input{sections/rectangle_overlap}
% \input{sections/rectangle_area}
% \input{sections/k_closest_points_to_origin}
% \input{sections/the_skyline_problem}
% % filename: k_closest_points_to_origin.tex

\problemsection{K Closest Points to Origin}
\label{chap:K_Closest_Points_to_Origin}
\marginnote{\href{https://leetcode.com/problems/k-closest-points-to-origin/}{[LeetCode Link]}\index{LeetCode}}
\marginnote{\href{https://www.geeksforgeeks.org/find-k-closest-points-origin/}{[GeeksForGeeks Link]}\index{GeeksForGeeks}}
\marginnote{\href{https://www.interviewbit.com/problems/k-closest-points/}{[InterviewBit Link]}\index{InterviewBit}}
\marginnote{\href{https://app.codesignal.com/challenges/k-closest-points-to-origin}{[CodeSignal Link]}\index{CodeSignal}}
\marginnote{\href{https://www.codewars.com/kata/k-closest-points-to-origin/train/python}{[Codewars Link]}\index{Codewars}}

The \textbf{K Closest Points to Origin} problem is a popular algorithmic challenge in Computational Geometry that involves identifying the \(k\) points closest to the origin in a 2D plane. This problem tests one's ability to apply efficient sorting and selection algorithms, understand distance computations, and optimize for performance. Mastery of this problem is essential for applications in spatial data analysis, nearest neighbor searches, and clustering algorithms.

\section*{Problem Statement}

Given an array of points where each point is represented as \([x, y]\) in the 2D plane, and an integer \(k\), return the \(k\) closest points to the origin \((0, 0)\).

The distance between two points \((x_1, y_1)\) and \((x_2, y_2)\) is the Euclidean distance \(\sqrt{(x_1 - x_2)^2 + (y_1 - y_2)^2}\). The origin is \((0, 0)\).

\textbf{Function signature in Python:}
\begin{lstlisting}[language=Python]
def kClosest(points: List[List[int]], K: int) -> List[List[int]]:
\end{lstlisting}

\section*{Examples}

\textbf{Example 1:}

\begin{verbatim}
Input: points = [[1,3],[-2,2]], K = 1
Output: [[-2,2]]
Explanation: 
The distance between (1, 3) and the origin is sqrt(10).
The distance between (-2, 2) and the origin is sqrt(8).
Since sqrt(8) < sqrt(10), (-2, 2) is closer to the origin.
\end{verbatim}

\textbf{Example 2:}

\begin{verbatim}
Input: points = [[3,3],[5,-1],[-2,4]], K = 2
Output: [[3,3],[-2,4]]
Explanation: 
The distances are sqrt(18), sqrt(26), and sqrt(20) respectively.
The two closest points are [3,3] and [-2,4].
\end{verbatim}

\textbf{Example 3:}

\begin{verbatim}
Input: points = [[0,1],[1,0]], K = 2
Output: [[0,1],[1,0]]
Explanation: 
Both points are equally close to the origin.
\end{verbatim}

\textbf{Example 4:}

\begin{verbatim}
Input: points = [[1,0],[0,1]], K = 1
Output: [[1,0]]
Explanation: 
Both points are equally close; returning any one is acceptable.
\end{verbatim}

\textbf{Constraints:}

\begin{itemize}
    \item \(1 \leq K \leq \text{points.length} \leq 10^4\)
    \item \(-10^4 < x_i, y_i < 10^4\)
\end{itemize}

LeetCode link: \href{https://leetcode.com/problems/k-closest-points-to-origin/}{K Closest Points to Origin}\index{LeetCode}

\section*{Algorithmic Approach}

To identify the \(k\) closest points to the origin, several algorithmic strategies can be employed. The most efficient methods aim to reduce the time complexity by avoiding the need to sort the entire list of points.

\subsection*{1. Sorting Based on Distance}

Calculate the Euclidean distance of each point from the origin and sort the points based on these distances. Select the first \(k\) points from the sorted list.

\begin{enumerate}
    \item Compute the distance for each point using the formula \(distance = x^2 + y^2\).
    \item Sort the points based on the computed distances.
    \item Return the first \(k\) points from the sorted list.
\end{enumerate}

\subsection*{2. Max Heap (Priority Queue)}

Use a max heap to maintain the \(k\) closest points. Iterate through each point, add it to the heap, and if the heap size exceeds \(k\), remove the farthest point.

\begin{enumerate}
    \item Initialize a max heap.
    \item For each point, compute its distance and add it to the heap.
    \item If the heap size exceeds \(k\), remove the point with the largest distance.
    \item After processing all points, the heap contains the \(k\) closest points.
\end{enumerate}

\subsection*{3. QuickSelect (Quick Sort Partitioning)}

Utilize the QuickSelect algorithm to find the \(k\) closest points without fully sorting the list.

\begin{enumerate}
    \item Choose a pivot point and partition the list based on distances relative to the pivot.
    \item Recursively apply QuickSelect to the partition containing the \(k\) closest points.
    \item Once the \(k\) closest points are identified, return them.
\end{enumerate}

\marginnote{QuickSelect offers an average time complexity of \(O(n)\), making it highly efficient for large datasets.}

\section*{Complexities}

\begin{itemize}
    \item \textbf{Sorting Based on Distance:}
    \begin{itemize}
        \item \textbf{Time Complexity:} \(O(n \log n)\)
        \item \textbf{Space Complexity:} \(O(n)\)
    \end{itemize}
    
    \item \textbf{Max Heap (Priority Queue):}
    \begin{itemize}
        \item \textbf{Time Complexity:} \(O(n \log k)\)
        \item \textbf{Space Complexity:} \(O(k)\)
    \end{itemize}
    
    \item \textbf{QuickSelect (Quick Sort Partitioning):}
    \begin{itemize}
        \item \textbf{Time Complexity:} Average case \(O(n)\), worst case \(O(n^2)\)
        \item \textbf{Space Complexity:} \(O(1)\) (in-place)
    \end{itemize}
\end{itemize}

\section*{Python Implementation}

\marginnote{Implementing QuickSelect provides an optimal average-case solution with linear time complexity.}

Below is the complete Python code implementing the \texttt{kClosest} function using the QuickSelect approach:

\begin{fullwidth}
\begin{lstlisting}[language=Python]
from typing import List
import random

class Solution:
    def kClosest(self, points: List[List[int]], K: int) -> List[List[int]]:
        def quickselect(left, right, K_smallest):
            if left == right:
                return
            
            # Select a random pivot_index
            pivot_index = random.randint(left, right)
            
            # Partition the array
            pivot_index = partition(left, right, pivot_index)
            
            # The pivot is in its final sorted position
            if K_smallest == pivot_index:
                return
            elif K_smallest < pivot_index:
                quickselect(left, pivot_index - 1, K_smallest)
            else:
                quickselect(pivot_index + 1, right, K_smallest)
        
        def partition(left, right, pivot_index):
            pivot_distance = distance(points[pivot_index])
            # Move pivot to end
            points[pivot_index], points[right] = points[right], points[pivot_index]
            store_index = left
            for i in range(left, right):
                if distance(points[i]) < pivot_distance:
                    points[store_index], points[i] = points[i], points[store_index]
                    store_index += 1
            # Move pivot to its final place
            points[right], points[store_index] = points[store_index], points[right]
            return store_index
        
        def distance(point):
            return point[0] ** 2 + point[1] ** 2
        
        n = len(points)
        quickselect(0, n - 1, K)
        return points[:K]

# Example usage:
solution = Solution()
print(solution.kClosest([[1,3],[-2,2]], 1))            # Output: [[-2,2]]
print(solution.kClosest([[3,3],[5,-1],[-2,4]], 2))     # Output: [[3,3],[-2,4]]
print(solution.kClosest([[0,1],[1,0]], 2))             # Output: [[0,1],[1,0]]
print(solution.kClosest([[1,0],[0,1]], 1))             # Output: [[1,0]] or [[0,1]]
\end{lstlisting}
\end{fullwidth}

This implementation uses the QuickSelect algorithm to efficiently find the \(k\) closest points to the origin without fully sorting the entire list. It ensures optimal performance even with large datasets.

\section*{Explanation}

The \texttt{kClosest} function identifies the \(k\) closest points to the origin using the QuickSelect algorithm. Here's a detailed breakdown of the implementation:

\subsection*{1. Distance Calculation}

\begin{itemize}
    \item The Euclidean distance is calculated as \(distance = x^2 + y^2\). Since we only need relative distances for comparison, the square root is omitted for efficiency.
\end{itemize}

\subsection*{2. QuickSelect Algorithm}

\begin{itemize}
    \item **Pivot Selection:**
    \begin{itemize}
        \item A random pivot is chosen to enhance the average-case performance.
    \end{itemize}
    
    \item **Partitioning:**
    \begin{itemize}
        \item The array is partitioned such that points with distances less than the pivot are moved to the left, and others to the right.
        \item The pivot is placed in its correct sorted position.
    \end{itemize}
    
    \item **Recursive Selection:**
    \begin{itemize}
        \item If the pivot's position matches \(K\), the selection is complete.
        \item Otherwise, recursively apply QuickSelect to the relevant partition.
    \end{itemize}
\end{itemize}

\subsection*{3. Final Selection}

\begin{itemize}
    \item After partitioning, the first \(K\) points in the list are the \(k\) closest points to the origin.
\end{itemize}

\subsection*{4. Example Walkthrough}

Consider the first example:
\begin{verbatim}
Input: points = [[1,3],[-2,2]], K = 1
Output: [[-2,2]]
\end{verbatim}

\begin{enumerate}
    \item **Calculate Distances:**
    \begin{itemize}
        \item [1,3] : \(1^2 + 3^2 = 10\)
        \item [-2,2] : \((-2)^2 + 2^2 = 8\)
    \end{itemize}
    
    \item **QuickSelect Process:**
    \begin{itemize}
        \item Choose a pivot, say [1,3] with distance 10.
        \item Compare and rearrange:
        \begin{itemize}
            \item [-2,2] has a smaller distance (8) and is moved to the left.
        \end{itemize}
        \item After partitioning, the list becomes [[-2,2], [1,3]].
        \item Since \(K = 1\), return the first point: [[-2,2]].
    \end{itemize}
\end{enumerate}

Thus, the function correctly identifies \([-2,2]\) as the closest point to the origin.

\section*{Why This Approach}

The QuickSelect algorithm is chosen for its average-case linear time complexity \(O(n)\), making it highly efficient for large datasets compared to sorting-based methods with \(O(n \log n)\) time complexity. By avoiding the need to sort the entire list, QuickSelect provides an optimal solution for finding the \(k\) closest points.

\section*{Alternative Approaches}

\subsection*{1. Sorting Based on Distance}

Sort all points based on their distances from the origin and select the first \(k\) points.

\begin{lstlisting}[language=Python]
class Solution:
    def kClosest(self, points: List[List[int]], K: int) -> List[List[int]]:
        points.sort(key=lambda P: P[0]**2 + P[1]**2)
        return points[:K]
\end{lstlisting}

\textbf{Complexities:}
\begin{itemize}
    \item \textbf{Time Complexity:} \(O(n \log n)\)
    \item \textbf{Space Complexity:} \(O(1)\)
\end{itemize}

\subsection*{2. Max Heap (Priority Queue)}

Use a max heap to maintain the \(k\) closest points.

\begin{lstlisting}[language=Python]
import heapq

class Solution:
    def kClosest(self, points: List[List[int]], K: int) -> List[List[int]]:
        heap = []
        for (x, y) in points:
            dist = -(x**2 + y**2)  # Max heap using negative distances
            heapq.heappush(heap, (dist, [x, y]))
            if len(heap) > K:
                heapq.heappop(heap)
        return [item[1] for item in heap]
\end{lstlisting}

\textbf{Complexities:}
\begin{itemize}
    \item \textbf{Time Complexity:} \(O(n \log k)\)
    \item \textbf{Space Complexity:} \(O(k)\)
\end{itemize}

\subsection*{3. Using Built-In Functions}

Leverage built-in functions for distance calculation and selection.

\begin{lstlisting}[language=Python]
import math

class Solution:
    def kClosest(self, points: List[List[int]], K: int) -> List[List[int]]:
        points.sort(key=lambda P: math.sqrt(P[0]**2 + P[1]**2))
        return points[:K]
\end{lstlisting}

\textbf{Note}: This method is similar to the sorting approach but uses the actual Euclidean distance.

\section*{Similar Problems to This One}

Several problems involve nearest neighbor searches, spatial data analysis, and efficient selection algorithms, utilizing similar algorithmic strategies:

\begin{itemize}
    \item \textbf{Closest Pair of Points}: Find the closest pair of points in a set.
    \item \textbf{Top K Frequent Elements}: Identify the most frequent elements in a dataset.
    \item \textbf{Kth Largest Element in an Array}: Find the \(k\)-th largest element in an unsorted array.
    \item \textbf{Sliding Window Maximum}: Find the maximum in each sliding window of size \(k\) over an array.
    \item \textbf{Merge K Sorted Lists}: Merge multiple sorted lists into a single sorted list.
    \item \textbf{Find Median from Data Stream}: Continuously find the median of a stream of numbers.
    \item \textbf{Top K Closest Stars}: Find the \(k\) closest stars to Earth based on their distances.
\end{itemize}

These problems reinforce concepts of efficient selection, heap usage, and distance computations in various contexts.

\section*{Things to Keep in Mind and Tricks}

When solving the \textbf{K Closest Points to Origin} problem, consider the following tips and best practices to enhance efficiency and correctness:

\begin{itemize}
    \item \textbf{Understand Distance Calculations}: Grasp the Euclidean distance formula and recognize that the square root can be omitted for comparison purposes.
    \index{Distance Calculations}
    
    \item \textbf{Leverage Efficient Algorithms}: Use QuickSelect or heap-based methods to optimize time complexity, especially for large datasets.
    \index{Efficient Algorithms}
    
    \item \textbf{Handle Ties Appropriately}: Decide how to handle points with identical distances when \(k\) is less than the number of such points.
    \index{Handling Ties}
    
    \item \textbf{Optimize Space Usage}: Choose algorithms that minimize additional space, such as in-place QuickSelect.
    \index{Space Optimization}
    
    \item \textbf{Use Appropriate Data Structures}: Utilize heaps, lists, and helper functions effectively to manage and process data.
    \index{Data Structures}
    
    \item \textbf{Implement Helper Functions}: Create helper functions for distance calculation and partitioning to enhance code modularity.
    \index{Helper Functions}
    
    \item \textbf{Code Readability}: Maintain clear and readable code through meaningful variable names and structured logic.
    \index{Code Readability}
    
    \item \textbf{Test Extensively}: Implement a wide range of test cases, including edge cases like multiple points with the same distance, to ensure robustness.
    \index{Extensive Testing}
    
    \item \textbf{Understand Algorithm Trade-offs}: Recognize the trade-offs between different approaches in terms of time and space complexities.
    \index{Algorithm Trade-offs}
    
    \item \textbf{Use Built-In Sorting Functions}: When using sorting-based approaches, leverage built-in functions for efficiency and simplicity.
    \index{Built-In Sorting}
    
    \item \textbf{Avoid Redundant Calculations}: Ensure that distance calculations are performed only when necessary to optimize performance.
    \index{Avoiding Redundant Calculations}
    
    \item \textbf{Language-Specific Features}: Utilize language-specific features or libraries that can simplify implementation, such as heapq in Python.
    \index{Language-Specific Features}
\end{itemize}

\section*{Corner and Special Cases to Test When Writing the Code}

When implementing the solution for the \textbf{K Closest Points to Origin} problem, it is crucial to consider and rigorously test various edge cases to ensure robustness and correctness:

\begin{itemize}
    \item \textbf{Multiple Points with Same Distance}: Ensure that the algorithm handles multiple points having the same distance from the origin.
    \index{Same Distance Points}
    
    \item \textbf{Points at Origin}: Include points that are exactly at the origin \((0,0)\).
    \index{Points at Origin}
    
    \item \textbf{Negative Coordinates}: Ensure that the algorithm correctly computes distances for points with negative \(x\) or \(y\) coordinates.
    \index{Negative Coordinates}
    
    \item \textbf{Large Coordinates}: Test with points having very large or very small coordinate values to verify integer handling.
    \index{Large Coordinates}
    
    \item \textbf{K Equals Number of Points}: When \(K\) is equal to the number of points, the algorithm should return all points.
    \index{K Equals Number of Points}
    
    \item \textbf{K is One}: Test with \(K = 1\) to ensure the closest point is correctly identified.
    \index{K is One}
    
    \item \textbf{All Points Same}: All points have the same coordinates.
    \index{All Points Same}
    
    \item \textbf{K is Zero}: Although \(K\) is defined to be at least 1, ensure that the algorithm gracefully handles \(K = 0\) if allowed.
    \index{K is Zero}
    
    \item \textbf{Single Point}: Only one point is provided, and \(K = 1\).
    \index{Single Point}
    
    \item \textbf{Mixed Coordinates}: Points with a mix of positive and negative coordinates.
    \index{Mixed Coordinates}
    
    \item \textbf{Points with Zero Distance}: Multiple points at the origin.
    \index{Zero Distance Points}
    
    \item \textbf{Sparse and Dense Points}: Densely packed points and sparsely distributed points.
    \index{Sparse and Dense Points}
    
    \item \textbf{Duplicate Points}: Multiple identical points in the input list.
    \index{Duplicate Points}
    
    \item \textbf{K Greater Than Number of Unique Points}: Ensure that the algorithm handles cases where \(K\) exceeds the number of unique points if applicable.
    \index{K Greater Than Unique Points}
\end{itemize}

\section*{Implementation Considerations}

When implementing the \texttt{kClosest} function, keep in mind the following considerations to ensure robustness and efficiency:

\begin{itemize}
    \item \textbf{Data Type Selection}: Use appropriate data types that can handle large input values without overflow or precision loss.
    \index{Data Type Selection}
    
    \item \textbf{Optimizing Distance Calculations}: Avoid calculating the square root since it is unnecessary for comparison purposes.
    \index{Optimizing Distance Calculations}
    
    \item \textbf{Choosing the Right Algorithm}: Select an algorithm based on the size of the input and the value of \(K\) to optimize time and space complexities.
    \index{Choosing the Right Algorithm}
    
    \item \textbf{Language-Specific Libraries}: Utilize language-specific libraries and functions (e.g., \texttt{heapq} in Python) to simplify implementation and enhance performance.
    \index{Language-Specific Libraries}
    
    \item \textbf{Avoiding Redundant Calculations}: Ensure that each point's distance is calculated only once to optimize performance.
    \index{Avoiding Redundant Calculations}
    
    \item \textbf{Implementing Helper Functions}: Create helper functions for tasks like distance calculation and partitioning to enhance modularity and readability.
    \index{Helper Functions}
    
    \item \textbf{Edge Case Handling}: Implement checks for edge cases to prevent incorrect results or runtime errors.
    \index{Edge Case Handling}
    
    \item \textbf{Testing and Validation}: Develop a comprehensive suite of test cases that cover all possible scenarios, including edge cases, to validate the correctness and efficiency of the implementation.
    \index{Testing and Validation}
    
    \item \textbf{Scalability}: Design the algorithm to scale efficiently with increasing input sizes, maintaining performance and resource utilization.
    \index{Scalability}
    
    \item \textbf{Consistent Naming Conventions}: Use consistent and descriptive naming conventions for variables and functions to improve code clarity.
    \index{Naming Conventions}
    
    \item \textbf{Memory Management}: Ensure that the algorithm manages memory efficiently, especially when dealing with large datasets.
    \index{Memory Management}
    
    \item \textbf{Avoiding Stack Overflow}: If implementing recursive approaches, be mindful of recursion limits and potential stack overflow issues.
    \index{Avoiding Stack Overflow}
    
    \item \textbf{Implementing Iterative Solutions}: Prefer iterative solutions when recursion may lead to increased space complexity or stack overflow.
    \index{Implementing Iterative Solutions}
\end{itemize}

\section*{Conclusion}

The \textbf{K Closest Points to Origin} problem exemplifies the application of efficient selection algorithms and geometric computations to solve spatial challenges effectively. By leveraging QuickSelect or heap-based methods, the algorithm achieves optimal time and space complexities, making it highly suitable for large datasets. Understanding and implementing such techniques not only enhances problem-solving skills but also provides a foundation for tackling more advanced Computational Geometry problems involving nearest neighbor searches, clustering, and spatial data analysis.

\printindex

% % filename: rectangle_overlap.tex

\problemsection{Rectangle Overlap}
\label{chap:Rectangle_Overlap}
\marginnote{\href{https://leetcode.com/problems/rectangle-overlap/}{[LeetCode Link]}\index{LeetCode}}
\marginnote{\href{https://www.geeksforgeeks.org/check-if-two-rectangles-overlap/}{[GeeksForGeeks Link]}\index{GeeksForGeeks}}
\marginnote{\href{https://www.interviewbit.com/problems/rectangle-overlap/}{[InterviewBit Link]}\index{InterviewBit}}
\marginnote{\href{https://app.codesignal.com/challenges/rectangle-overlap}{[CodeSignal Link]}\index{CodeSignal}}
\marginnote{\href{https://www.codewars.com/kata/rectangle-overlap/train/python}{[Codewars Link]}\index{Codewars}}

The \textbf{Rectangle Overlap} problem is a fundamental challenge in Computational Geometry that involves determining whether two axis-aligned rectangles overlap. This problem tests one's ability to understand geometric properties, implement conditional logic, and optimize for efficient computation. Mastery of this problem is essential for applications in computer graphics, collision detection, and spatial data analysis.

\section*{Problem Statement}

Given two axis-aligned rectangles in a 2D plane, determine if they overlap. Each rectangle is defined by its bottom-left and top-right coordinates.

A rectangle is represented as a list of four integers \([x1, y1, x2, y2]\), where \((x1, y1)\) are the coordinates of the bottom-left corner, and \((x2, y2)\) are the coordinates of the top-right corner.

\textbf{Function signature in Python:}
\begin{lstlisting}[language=Python]
def isRectangleOverlap(rec1: List[int], rec2: List[int]) -> bool:
\end{lstlisting}

\section*{Examples}

\textbf{Example 1:}

\begin{verbatim}
Input: rec1 = [0,0,2,2], rec2 = [1,1,3,3]
Output: True
Explanation: The rectangles overlap in the area defined by [1,1,2,2].
\end{verbatim}

\textbf{Example 2:}

\begin{verbatim}
Input: rec1 = [0,0,1,1], rec2 = [1,0,2,1]
Output: False
Explanation: The rectangles touch at the edge but do not overlap.
\end{verbatim}

\textbf{Example 3:}

\begin{verbatim}
Input: rec1 = [0,0,1,1], rec2 = [2,2,3,3]
Output: False
Explanation: The rectangles are completely separate.
\end{verbatim}

\textbf{Example 4:}

\begin{verbatim}
Input: rec1 = [0,0,5,5], rec2 = [3,3,7,7]
Output: True
Explanation: The rectangles overlap in the area defined by [3,3,5,5].
\end{verbatim}

\textbf{Example 5:}

\begin{verbatim}
Input: rec1 = [0,0,0,0], rec2 = [0,0,0,0]
Output: False
Explanation: Both rectangles are degenerate points.
\end{verbatim}

\textbf{Constraints:}

\begin{itemize}
    \item All coordinates are integers in the range \([-10^9, 10^9]\).
    \item For each rectangle, \(x1 < x2\) and \(y1 < y2\).
\end{itemize}

LeetCode link: \href{https://leetcode.com/problems/rectangle-overlap/}{Rectangle Overlap}\index{LeetCode}

\section*{Algorithmic Approach}

To determine whether two axis-aligned rectangles overlap, we can use the following logical conditions:

1. **Non-Overlap Conditions:**
   - One rectangle is to the left of the other.
   - One rectangle is above the other.

2. **Overlap Condition:**
   - If neither of the non-overlap conditions is true, the rectangles must overlap.

\subsection*{Steps:}

1. **Extract Coordinates:**
   - For both rectangles, extract the bottom-left and top-right coordinates.

2. **Check Non-Overlap Conditions:**
   - If the right side of the first rectangle is less than or equal to the left side of the second rectangle, they do not overlap.
   - If the left side of the first rectangle is greater than or equal to the right side of the second rectangle, they do not overlap.
   - If the top side of the first rectangle is less than or equal to the bottom side of the second rectangle, they do not overlap.
   - If the bottom side of the first rectangle is greater than or equal to the top side of the second rectangle, they do not overlap.

3. **Determine Overlap:**
   - If none of the non-overlap conditions are met, the rectangles overlap.

\marginnote{This approach provides an efficient \(O(1)\) time complexity solution by leveraging simple geometric comparisons.}

\section*{Complexities}

\begin{itemize}
    \item \textbf{Time Complexity:} \(O(1)\). The algorithm performs a constant number of comparisons regardless of input size.
    
    \item \textbf{Space Complexity:} \(O(1)\). Only a fixed amount of extra space is used for variables.
\end{itemize}

\section*{Python Implementation}

\marginnote{Implementing the overlap check using coordinate comparisons ensures an optimal and straightforward solution.}

Below is the complete Python code implementing the \texttt{isRectangleOverlap} function:

\begin{fullwidth}
\begin{lstlisting}[language=Python]
from typing import List

class Solution:
    def isRectangleOverlap(self, rec1: List[int], rec2: List[int]) -> bool:
        # Extract coordinates
        left1, bottom1, right1, top1 = rec1
        left2, bottom2, right2, top2 = rec2
        
        # Check non-overlapping conditions
        if right1 <= left2 or right2 <= left1:
            return False
        if top1 <= bottom2 or top2 <= bottom1:
            return False
        
        # If none of the above, rectangles overlap
        return True

# Example usage:
solution = Solution()
print(solution.isRectangleOverlap([0,0,2,2], [1,1,3,3]))  # Output: True
print(solution.isRectangleOverlap([0,0,1,1], [1,0,2,1]))  # Output: False
print(solution.isRectangleOverlap([0,0,1,1], [2,2,3,3]))  # Output: False
print(solution.isRectangleOverlap([0,0,5,5], [3,3,7,7]))  # Output: True
print(solution.isRectangleOverlap([0,0,0,0], [0,0,0,0]))  # Output: False
\end{lstlisting}
\end{fullwidth}

This implementation efficiently checks for overlap by comparing the coordinates of the two rectangles. If any of the non-overlapping conditions are met, it returns \texttt{False}; otherwise, it returns \texttt{True}.

\section*{Explanation}

The \texttt{isRectangleOverlap} function determines whether two axis-aligned rectangles overlap by comparing their respective coordinates. Here's a detailed breakdown of the implementation:

\subsection*{1. Extract Coordinates}

\begin{itemize}
    \item For each rectangle, extract the left (\(x1\)), bottom (\(y1\)), right (\(x2\)), and top (\(y2\)) coordinates.
    \item This simplifies the comparison process by providing clear variables representing each side of the rectangles.
\end{itemize}

\subsection*{2. Check Non-Overlap Conditions}

\begin{itemize}
    \item **Horizontal Separation:**
    \begin{itemize}
        \item If the right side of the first rectangle (\(right1\)) is less than or equal to the left side of the second rectangle (\(left2\)), there is no horizontal overlap.
        \item Similarly, if the right side of the second rectangle (\(right2\)) is less than or equal to the left side of the first rectangle (\(left1\)), there is no horizontal overlap.
    \end{itemize}
    
    \item **Vertical Separation:**
    \begin{itemize}
        \item If the top side of the first rectangle (\(top1\)) is less than or equal to the bottom side of the second rectangle (\(bottom2\)), there is no vertical overlap.
        \item Similarly, if the top side of the second rectangle (\(top2\)) is less than or equal to the bottom side of the first rectangle (\(bottom1\)), there is no vertical overlap.
    \end{itemize}
    
    \item If any of these non-overlapping conditions are true, the rectangles do not overlap, and the function returns \texttt{False}.
\end{itemize}

\subsection*{3. Determine Overlap}

\begin{itemize}
    \item If none of the non-overlapping conditions are met, it implies that the rectangles overlap both horizontally and vertically.
    \item The function returns \texttt{True} in this case.
\end{itemize}

\subsection*{4. Example Walkthrough}

Consider the first example:
\begin{verbatim}
Input: rec1 = [0,0,2,2], rec2 = [1,1,3,3]
Output: True
\end{verbatim}

\begin{enumerate}
    \item Extract coordinates:
    \begin{itemize}
        \item rec1: left1 = 0, bottom1 = 0, right1 = 2, top1 = 2
        \item rec2: left2 = 1, bottom2 = 1, right2 = 3, top2 = 3
    \end{itemize}
    
    \item Check non-overlap conditions:
    \begin{itemize}
        \item \(right1 = 2\) is not less than or equal to \(left2 = 1\)
        \item \(right2 = 3\) is not less than or equal to \(left1 = 0\)
        \item \(top1 = 2\) is not less than or equal to \(bottom2 = 1\)
        \item \(top2 = 3\) is not less than or equal to \(bottom1 = 0\)
    \end{itemize}
    
    \item Since none of the non-overlapping conditions are met, the rectangles overlap.
\end{enumerate}

Thus, the function correctly returns \texttt{True}.

\section*{Why This Approach}

This approach is chosen for its simplicity and efficiency. By leveraging direct coordinate comparisons, the algorithm achieves constant time complexity without the need for complex data structures or iterative processes. It effectively handles all possible scenarios of rectangle positioning, ensuring accurate detection of overlaps.

\section*{Alternative Approaches}

\subsection*{1. Separating Axis Theorem (SAT)}

The Separating Axis Theorem is a more generalized method for detecting overlaps between convex shapes. While it is not necessary for axis-aligned rectangles, understanding SAT can be beneficial for more complex geometric problems.

\begin{lstlisting}[language=Python]
def isRectangleOverlap(rec1: List[int], rec2: List[int]) -> bool:
    # Using SAT for axis-aligned rectangles
    return not (rec1[2] <= rec2[0] or rec1[0] >= rec2[2] or
                rec1[3] <= rec2[1] or rec1[1] >= rec2[3])
\end{lstlisting}

\textbf{Note}: This implementation is functionally identical to the primary approach but leverages a more generalized geometric theorem.

\subsection*{2. Area-Based Approach}

Calculate the overlapping area between the two rectangles. If the overlapping area is positive, the rectangles overlap.

\begin{lstlisting}[language=Python]
def isRectangleOverlap(rec1: List[int], rec2: List[int]) -> bool:
    # Calculate overlap in x and y dimensions
    x_overlap = min(rec1[2], rec2[2]) - max(rec1[0], rec2[0])
    y_overlap = min(rec1[3], rec2[3]) - max(rec1[1], rec2[1])
    
    # Overlap exists if both overlaps are positive
    return x_overlap > 0 and y_overlap > 0
\end{lstlisting}

\textbf{Complexities:}
\begin{itemize}
    \item \textbf{Time Complexity:} \(O(1)\)
    \item \textbf{Space Complexity:} \(O(1)\)
\end{itemize}

\subsection*{3. Using Rectangles Intersection Function}

Utilize built-in or library functions that handle geometric intersections.

\begin{lstlisting}[language=Python]
from shapely.geometry import box

def isRectangleOverlap(rec1: List[int], rec2: List[int]) -> bool:
    rectangle1 = box(rec1[0], rec1[1], rec1[2], rec1[3])
    rectangle2 = box(rec2[0], rec2[1], rec2[2], rec2[3])
    return rectangle1.intersects(rectangle2) and not rectangle1.touches(rectangle2)
\end{lstlisting}

\textbf{Note}: This approach requires the \texttt{shapely} library and is more suitable for complex geometric operations.

\section*{Similar Problems to This One}

Several problems revolve around geometric overlap, intersection detection, and spatial reasoning, utilizing similar algorithmic strategies:

\begin{itemize}
    \item \textbf{Interval Overlap}: Determine if two intervals on a line overlap.
    \item \textbf{Circle Overlap}: Determine if two circles overlap based on their radii and centers.
    \item \textbf{Polygon Overlap}: Determine if two polygons overlap using algorithms like SAT.
    \item \textbf{Closest Pair of Points}: Find the closest pair of points in a set.
    \item \textbf{Convex Hull}: Compute the convex hull of a set of points.
    \item \textbf{Intersection of Lines}: Find the intersection point of two lines.
    \item \textbf{Point Inside Polygon}: Determine if a point lies inside a given polygon.
\end{itemize}

These problems reinforce the concepts of spatial reasoning, geometric property analysis, and efficient algorithm design in various contexts.

\section*{Things to Keep in Mind and Tricks}

When working with the \textbf{Rectangle Overlap} problem, consider the following tips and best practices to enhance efficiency and correctness:

\begin{itemize}
    \item \textbf{Understand Geometric Relationships}: Grasp the positional relationships between rectangles to simplify overlap detection.
    \index{Geometric Relationships}
    
    \item \textbf{Leverage Coordinate Comparisons}: Use direct comparisons of rectangle coordinates to determine spatial relationships.
    \index{Coordinate Comparisons}
    
    \item \textbf{Handle Edge Cases}: Consider cases where rectangles touch at edges or corners without overlapping.
    \index{Edge Cases}
    
    \item \textbf{Optimize for Efficiency}: Aim for a constant time \(O(1)\) solution by avoiding unnecessary computations or iterations.
    \index{Efficiency Optimization}
    
    \item \textbf{Avoid Floating-Point Precision Issues}: Since all coordinates are integers, floating-point precision is not a concern, simplifying the implementation.
    \index{Floating-Point Precision}
    
    \item \textbf{Use Helper Functions}: Create helper functions to encapsulate repetitive tasks, such as extracting coordinates or checking specific conditions.
    \index{Helper Functions}
    
    \item \textbf{Code Readability}: Maintain clear and readable code through meaningful variable names and structured logic.
    \index{Code Readability}
    
    \item \textbf{Test Extensively}: Implement a wide range of test cases, including overlapping, non-overlapping, and edge-touching rectangles, to ensure robustness.
    \index{Extensive Testing}
    
    \item \textbf{Understand Axis-Aligned Constraints}: Recognize that axis-aligned rectangles simplify overlap detection compared to rotated rectangles.
    \index{Axis-Aligned Constraints}
    
    \item \textbf{Simplify Logical Conditions}: Combine multiple conditions logically to streamline the overlap detection process.
    \index{Logical Conditions}
\end{itemize}

\section*{Corner and Special Cases to Test When Writing the Code}

When implementing the solution for the \textbf{Rectangle Overlap} problem, it is crucial to consider and rigorously test various edge cases to ensure robustness and correctness:

\begin{itemize}
    \item \textbf{No Overlap}: Rectangles are completely separate.
    \index{No Overlap}
    
    \item \textbf{Partial Overlap}: Rectangles overlap in one or more regions.
    \index{Partial Overlap}
    
    \item \textbf{Edge Touching}: Rectangles touch exactly at one edge without overlapping.
    \index{Edge Touching}
    
    \item \textbf{Corner Touching}: Rectangles touch exactly at one corner without overlapping.
    \index{Corner Touching}
    
    \item \textbf{One Rectangle Inside Another}: One rectangle is entirely within the other.
    \index{Rectangle Inside}
    
    \item \textbf{Identical Rectangles}: Both rectangles have the same coordinates.
    \index{Identical Rectangles}
    
    \item \textbf{Degenerate Rectangles}: Rectangles with zero area (e.g., \(x1 = x2\) or \(y1 = y2\)).
    \index{Degenerate Rectangles}
    
    \item \textbf{Large Coordinates}: Rectangles with very large coordinate values to test performance and integer handling.
    \index{Large Coordinates}
    
    \item \textbf{Negative Coordinates}: Rectangles positioned in negative coordinate space.
    \index{Negative Coordinates}
    
    \item \textbf{Mixed Overlapping Scenarios}: Combinations of the above cases to ensure comprehensive coverage.
    \index{Mixed Overlapping Scenarios}
    
    \item \textbf{Minimum and Maximum Bounds}: Rectangles at the minimum and maximum limits of the coordinate range.
    \index{Minimum and Maximum Bounds}
\end{itemize}

\section*{Implementation Considerations}

When implementing the \texttt{isRectangleOverlap} function, keep in mind the following considerations to ensure robustness and efficiency:

\begin{itemize}
    \item \textbf{Data Type Selection}: Use appropriate data types that can handle the range of input values without overflow or underflow.
    \index{Data Type Selection}
    
    \item \textbf{Optimizing Comparisons}: Structure logical conditions to short-circuit evaluations as soon as a non-overlapping condition is met.
    \index{Optimizing Comparisons}
    
    \item \textbf{Language-Specific Constraints}: Be aware of how the programming language handles integer division and comparisons.
    \index{Language-Specific Constraints}
    
    \item \textbf{Avoiding Redundant Calculations}: Ensure that each comparison contributes towards determining overlap without unnecessary repetitions.
    \index{Avoiding Redundant Calculations}
    
    \item \textbf{Code Readability and Documentation}: Maintain clear and readable code through meaningful variable names and comprehensive comments to facilitate understanding and maintenance.
    \index{Code Readability}
    
    \item \textbf{Edge Case Handling}: Implement checks for edge cases to prevent incorrect results or runtime errors.
    \index{Edge Case Handling}
    
    \item \textbf{Testing and Validation}: Develop a comprehensive suite of test cases that cover all possible scenarios, including edge cases, to validate the correctness and efficiency of the implementation.
    \index{Testing and Validation}
    
    \item \textbf{Scalability}: Design the algorithm to scale efficiently with increasing input sizes, maintaining performance and resource utilization.
    \index{Scalability}
    
    \item \textbf{Using Helper Functions}: Consider creating helper functions for repetitive tasks, such as extracting and comparing coordinates, to enhance modularity and reusability.
    \index{Helper Functions}
    
    \item \textbf{Consistent Naming Conventions}: Use consistent and descriptive naming conventions for variables to improve code clarity.
    \index{Naming Conventions}
    
    \item \textbf{Handling Floating-Point Coordinates}: Although the problem specifies integer coordinates, ensure that the implementation can handle floating-point numbers if needed in extended scenarios.
    \index{Floating-Point Coordinates}
    
    \item \textbf{Avoiding Floating-Point Precision Issues}: Since all coordinates are integers, floating-point precision is not a concern, simplifying the implementation.
    \index{Floating-Point Precision}
    
    \item \textbf{Implementing Unit Tests}: Develop unit tests for each logical condition to ensure that all scenarios are correctly handled.
    \index{Unit Tests}
    
    \item \textbf{Error Handling}: Incorporate error handling to manage invalid inputs gracefully.
    \index{Error Handling}
\end{itemize}

\section*{Conclusion}

The \textbf{Rectangle Overlap} problem exemplifies the application of fundamental geometric principles and conditional logic to solve spatial challenges efficiently. By leveraging simple coordinate comparisons, the algorithm achieves optimal time and space complexities, making it highly suitable for real-time applications such as collision detection in gaming, layout planning in graphics, and spatial data analysis. Understanding and implementing such techniques not only enhances problem-solving skills but also provides a foundation for tackling more complex Computational Geometry problems involving varied geometric shapes and interactions.

\printindex

% \input{sections/rectangle_overlap}
% \input{sections/rectangle_area}
% \input{sections/k_closest_points_to_origin}
% \input{sections/the_skyline_problem}
% % filename: rectangle_area.tex

\problemsection{Rectangle Area}
\label{chap:Rectangle_Area}
\marginnote{\href{https://leetcode.com/problems/rectangle-area/}{[LeetCode Link]}\index{LeetCode}}
\marginnote{\href{https://www.geeksforgeeks.org/find-area-two-overlapping-rectangles/}{[GeeksForGeeks Link]}\index{GeeksForGeeks}}
\marginnote{\href{https://www.interviewbit.com/problems/rectangle-area/}{[InterviewBit Link]}\index{InterviewBit}}
\marginnote{\href{https://app.codesignal.com/challenges/rectangle-area}{[CodeSignal Link]}\index{CodeSignal}}
\marginnote{\href{https://www.codewars.com/kata/rectangle-area/train/python}{[Codewars Link]}\index{Codewars}}

The \textbf{Rectangle Area} problem is a classic Computational Geometry challenge that involves calculating the total area covered by two axis-aligned rectangles in a 2D plane. This problem tests one's ability to perform geometric calculations, handle overlapping scenarios, and implement efficient algorithms. Mastery of this problem is essential for applications in computer graphics, spatial analysis, and computational modeling.

\section*{Problem Statement}

Given two axis-aligned rectangles in a 2D plane, compute the total area covered by the two rectangles. The area covered by the overlapping region should be counted only once.

Each rectangle is represented as a list of four integers \([x1, y1, x2, y2]\), where \((x1, y1)\) are the coordinates of the bottom-left corner, and \((x2, y2)\) are the coordinates of the top-right corner.

\textbf{Function signature in Python:}
\begin{lstlisting}[language=Python]
def computeArea(A: List[int], B: List[int]) -> int:
\end{lstlisting}

\section*{Examples}

\textbf{Example 1:}

\begin{verbatim}
Input: A = [-3,0,3,4], B = [0,-1,9,2]
Output: 45
Explanation:
Area of A = (3 - (-3)) * (4 - 0) = 6 * 4 = 24
Area of B = (9 - 0) * (2 - (-1)) = 9 * 3 = 27
Overlapping Area = (3 - 0) * (2 - 0) = 3 * 2 = 6
Total Area = 24 + 27 - 6 = 45
\end{verbatim}

\textbf{Example 2:}

\begin{verbatim}
Input: A = [0,0,0,0], B = [0,0,0,0]
Output: 0
Explanation:
Both rectangles are degenerate points with zero area.
\end{verbatim}

\textbf{Example 3:}

\begin{verbatim}
Input: A = [0,0,2,2], B = [1,1,3,3]
Output: 7
Explanation:
Area of A = 4
Area of B = 4
Overlapping Area = 1
Total Area = 4 + 4 - 1 = 7
\end{verbatim}

\textbf{Example 4:}

\begin{verbatim}
Input: A = [0,0,1,1], B = [1,0,2,1]
Output: 2
Explanation:
Rectangles touch at the edge but do not overlap.
Area of A = 1
Area of B = 1
Overlapping Area = 0
Total Area = 1 + 1 = 2
\end{verbatim}

\textbf{Constraints:}

\begin{itemize}
    \item All coordinates are integers in the range \([-10^9, 10^9]\).
    \item For each rectangle, \(x1 < x2\) and \(y1 < y2\).
\end{itemize}

LeetCode link: \href{https://leetcode.com/problems/rectangle-area/}{Rectangle Area}\index{LeetCode}

\section*{Algorithmic Approach}

To compute the total area covered by two axis-aligned rectangles, we can follow these steps:

1. **Calculate Individual Areas:**
   - Compute the area of the first rectangle.
   - Compute the area of the second rectangle.

2. **Determine Overlapping Area:**
   - Calculate the coordinates of the overlapping rectangle, if any.
   - If the rectangles overlap, compute the area of the overlapping region.

3. **Compute Total Area:**
   - Sum the individual areas and subtract the overlapping area to avoid double-counting.

\marginnote{This approach ensures accurate area calculation by handling overlapping regions appropriately.}

\section*{Complexities}

\begin{itemize}
    \item \textbf{Time Complexity:} \(O(1)\). The algorithm performs a constant number of calculations.
    
    \item \textbf{Space Complexity:} \(O(1)\). Only a fixed amount of extra space is used for variables.
\end{itemize}

\section*{Python Implementation}

\marginnote{Implementing the area calculation with overlap consideration ensures an accurate and efficient solution.}

Below is the complete Python code implementing the \texttt{computeArea} function:

\begin{fullwidth}
\begin{lstlisting}[language=Python]
from typing import List

class Solution:
    def computeArea(self, A: List[int], B: List[int]) -> int:
        # Calculate area of rectangle A
        areaA = (A[2] - A[0]) * (A[3] - A[1])
        
        # Calculate area of rectangle B
        areaB = (B[2] - B[0]) * (B[3] - B[1])
        
        # Determine overlap coordinates
        overlap_x1 = max(A[0], B[0])
        overlap_y1 = max(A[1], B[1])
        overlap_x2 = min(A[2], B[2])
        overlap_y2 = min(A[3], B[3])
        
        # Calculate overlapping area
        overlap_width = overlap_x2 - overlap_x1
        overlap_height = overlap_y2 - overlap_y1
        overlap_area = 0
        if overlap_width > 0 and overlap_height > 0:
            overlap_area = overlap_width * overlap_height
        
        # Total area is sum of individual areas minus overlapping area
        total_area = areaA + areaB - overlap_area
        return total_area

# Example usage:
solution = Solution()
print(solution.computeArea([-3,0,3,4], [0,-1,9,2]))  # Output: 45
print(solution.computeArea([0,0,0,0], [0,0,0,0]))    # Output: 0
print(solution.computeArea([0,0,2,2], [1,1,3,3]))    # Output: 7
print(solution.computeArea([0,0,1,1], [1,0,2,1]))    # Output: 2
\end{lstlisting}
\end{fullwidth}

This implementation accurately computes the total area covered by two rectangles by accounting for any overlapping regions. It ensures that the overlapping area is not double-counted.

\section*{Explanation}

The \texttt{computeArea} function calculates the combined area of two axis-aligned rectangles by following these steps:

\subsection*{1. Calculate Individual Areas}

\begin{itemize}
    \item **Rectangle A:**
    \begin{itemize}
        \item Width: \(A[2] - A[0]\)
        \item Height: \(A[3] - A[1]\)
        \item Area: Width \(\times\) Height
    \end{itemize}
    
    \item **Rectangle B:**
    \begin{itemize}
        \item Width: \(B[2] - B[0]\)
        \item Height: \(B[3] - B[1]\)
        \item Area: Width \(\times\) Height
    \end{itemize}
\end{itemize}

\subsection*{2. Determine Overlapping Area}

\begin{itemize}
    \item **Overlap Coordinates:**
    \begin{itemize}
        \item Left (x-coordinate): \(\text{max}(A[0], B[0])\)
        \item Bottom (y-coordinate): \(\text{max}(A[1], B[1])\)
        \item Right (x-coordinate): \(\text{min}(A[2], B[2])\)
        \item Top (y-coordinate): \(\text{min}(A[3], B[3])\)
    \end{itemize}
    
    \item **Overlap Dimensions:**
    \begin{itemize}
        \item Width: \(\text{overlap\_x2} - \text{overlap\_x1}\)
        \item Height: \(\text{overlap\_y2} - \text{overlap\_y1}\)
    \end{itemize}
    
    \item **Overlap Area:**
    \begin{itemize}
        \item If both width and height are positive, the rectangles overlap, and the overlapping area is their product.
        \item Otherwise, there is no overlap, and the overlapping area is zero.
    \end{itemize}
\end{itemize}

\subsection*{3. Compute Total Area}

\begin{itemize}
    \item Total Area = Area of Rectangle A + Area of Rectangle B - Overlapping Area
\end{itemize}

\subsection*{4. Example Walkthrough}

Consider the first example:
\begin{verbatim}
Input: A = [-3,0,3,4], B = [0,-1,9,2]
Output: 45
\end{verbatim}

\begin{enumerate}
    \item **Calculate Areas:**
    \begin{itemize}
        \item Area of A = (3 - (-3)) * (4 - 0) = 6 * 4 = 24
        \item Area of B = (9 - 0) * (2 - (-1)) = 9 * 3 = 27
    \end{itemize}
    
    \item **Determine Overlap:**
    \begin{itemize}
        \item overlap\_x1 = max(-3, 0) = 0
        \item overlap\_y1 = max(0, -1) = 0
        \item overlap\_x2 = min(3, 9) = 3
        \item overlap\_y2 = min(4, 2) = 2
        \item overlap\_width = 3 - 0 = 3
        \item overlap\_height = 2 - 0 = 2
        \item overlap\_area = 3 * 2 = 6
    \end{itemize}
    
    \item **Compute Total Area:**
    \begin{itemize}
        \item Total Area = 24 + 27 - 6 = 45
    \end{itemize}
\end{enumerate}

Thus, the function correctly returns \texttt{45}.

\section*{Why This Approach}

This approach is chosen for its straightforwardness and optimal efficiency. By directly calculating the individual areas and intelligently handling the overlapping region, the algorithm ensures accurate results without unnecessary computations. Its constant time complexity makes it highly efficient, even for large coordinate values.

\section*{Alternative Approaches}

\subsection*{1. Using Intersection Dimensions}

Instead of separately calculating areas, directly compute the dimensions of the overlapping region and subtract it from the sum of individual areas.

\begin{lstlisting}[language=Python]
def computeArea(A: List[int], B: List[int]) -> int:
    # Sum of individual areas
    area = (A[2] - A[0]) * (A[3] - A[1]) + (B[2] - B[0]) * (B[3] - B[1])
    
    # Overlapping area
    overlap_width = min(A[2], B[2]) - max(A[0], B[0])
    overlap_height = min(A[3], B[3]) - max(A[1], B[1])
    
    if overlap_width > 0 and overlap_height > 0:
        area -= overlap_width * overlap_height
    
    return area
\end{lstlisting}

\subsection*{2. Using Geometry Libraries}

Leverage computational geometry libraries to handle area calculations and overlapping detections.

\begin{lstlisting}[language=Python]
from shapely.geometry import box

def computeArea(A: List[int], B: List[int]) -> int:
    rect1 = box(A[0], A[1], A[2], A[3])
    rect2 = box(B[0], B[1], B[2], B[3])
    intersection = rect1.intersection(rect2)
    return int(rect1.area + rect2.area - intersection.area)
\end{lstlisting}

\textbf{Note}: This approach requires the \texttt{shapely} library and is more suitable for complex geometric operations.

\section*{Similar Problems to This One}

Several problems involve calculating areas, handling geometric overlaps, and spatial reasoning, utilizing similar algorithmic strategies:

\begin{itemize}
    \item \textbf{Rectangle Overlap}: Determine if two rectangles overlap.
    \item \textbf{Circle Area Overlap}: Calculate the overlapping area between two circles.
    \item \textbf{Polygon Area}: Compute the area of a given polygon.
    \item \textbf{Union of Rectangles}: Calculate the total area covered by multiple rectangles, accounting for overlaps.
    \item \textbf{Intersection of Lines}: Find the intersection point of two lines.
    \item \textbf{Closest Pair of Points}: Find the closest pair of points in a set.
    \item \textbf{Convex Hull}: Compute the convex hull of a set of points.
    \item \textbf{Point Inside Polygon}: Determine if a point lies inside a given polygon.
\end{itemize}

These problems reinforce concepts of geometric calculations, area computations, and efficient algorithm design in various contexts.

\section*{Things to Keep in Mind and Tricks}

When tackling the \textbf{Rectangle Area} problem, consider the following tips and best practices to enhance efficiency and correctness:

\begin{itemize}
    \item \textbf{Understand Geometric Relationships}: Grasp the positional relationships between rectangles to simplify area calculations.
    \index{Geometric Relationships}
    
    \item \textbf{Leverage Coordinate Comparisons}: Use direct comparisons of rectangle coordinates to determine overlapping regions.
    \index{Coordinate Comparisons}
    
    \item \textbf{Handle Overlapping Scenarios}: Accurately calculate the overlapping area to avoid double-counting.
    \index{Overlapping Scenarios}
    
    \item \textbf{Optimize for Efficiency}: Aim for a constant time \(O(1)\) solution by avoiding unnecessary computations or iterations.
    \index{Efficiency Optimization}
    
    \item \textbf{Avoid Floating-Point Precision Issues}: Since all coordinates are integers, floating-point precision is not a concern, simplifying the implementation.
    \index{Floating-Point Precision}
    
    \item \textbf{Use Helper Functions}: Create helper functions to encapsulate repetitive tasks, such as calculating overlap dimensions or areas.
    \index{Helper Functions}
    
    \item \textbf{Code Readability}: Maintain clear and readable code through meaningful variable names and structured logic.
    \index{Code Readability}
    
    \item \textbf{Test Extensively}: Implement a wide range of test cases, including overlapping, non-overlapping, and edge-touching rectangles, to ensure robustness.
    \index{Extensive Testing}
    
    \item \textbf{Understand Axis-Aligned Constraints}: Recognize that axis-aligned rectangles simplify area calculations compared to rotated rectangles.
    \index{Axis-Aligned Constraints}
    
    \item \textbf{Simplify Logical Conditions}: Combine multiple conditions logically to streamline the area calculation process.
    \index{Logical Conditions}
    
    \item \textbf{Use Absolute Values}: When calculating differences, ensure that the dimensions are positive by using absolute values or proper ordering.
    \index{Absolute Values}
    
    \item \textbf{Consider Edge Cases}: Handle cases where rectangles have zero area or touch at edges/corners without overlapping.
    \index{Edge Cases}
\end{itemize}

\section*{Corner and Special Cases to Test When Writing the Code}

When implementing the solution for the \textbf{Rectangle Area} problem, it is crucial to consider and rigorously test various edge cases to ensure robustness and correctness:

\begin{itemize}
    \item \textbf{No Overlap}: Rectangles are completely separate.
    \index{No Overlap}
    
    \item \textbf{Partial Overlap}: Rectangles overlap in one or more regions.
    \index{Partial Overlap}
    
    \item \textbf{Edge Touching}: Rectangles touch exactly at one edge without overlapping.
    \index{Edge Touching}
    
    \item \textbf{Corner Touching}: Rectangles touch exactly at one corner without overlapping.
    \index{Corner Touching}
    
    \item \textbf{One Rectangle Inside Another}: One rectangle is entirely within the other.
    \index{Rectangle Inside}
    
    \item \textbf{Identical Rectangles}: Both rectangles have the same coordinates.
    \index{Identical Rectangles}
    
    \item \textbf{Degenerate Rectangles}: Rectangles with zero area (e.g., \(x1 = x2\) or \(y1 = y2\)).
    \index{Degenerate Rectangles}
    
    \item \textbf{Large Coordinates}: Rectangles with very large coordinate values to test performance and integer handling.
    \index{Large Coordinates}
    
    \item \textbf{Negative Coordinates}: Rectangles positioned in negative coordinate space.
    \index{Negative Coordinates}
    
    \item \textbf{Mixed Overlapping Scenarios}: Combinations of the above cases to ensure comprehensive coverage.
    \index{Mixed Overlapping Scenarios}
    
    \item \textbf{Minimum and Maximum Bounds}: Rectangles at the minimum and maximum limits of the coordinate range.
    \index{Minimum and Maximum Bounds}
    
    \item \textbf{Sequential Rectangles}: Multiple rectangles placed sequentially without overlapping.
    \index{Sequential Rectangles}
    
    \item \textbf{Multiple Overlaps}: Scenarios where more than two rectangles overlap in different regions.
    \index{Multiple Overlaps}
\end{itemize}

\section*{Implementation Considerations}

When implementing the \texttt{computeArea} function, keep in mind the following considerations to ensure robustness and efficiency:

\begin{itemize}
    \item \textbf{Data Type Selection}: Use appropriate data types that can handle large input values without overflow or underflow.
    \index{Data Type Selection}
    
    \item \textbf{Optimizing Comparisons}: Structure logical conditions to efficiently determine overlap dimensions.
    \index{Optimizing Comparisons}
    
    \item \textbf{Handling Large Inputs}: Design the algorithm to efficiently handle large input sizes without significant performance degradation.
    \index{Handling Large Inputs}
    
    \item \textbf{Language-Specific Constraints}: Be aware of how the programming language handles large integers and arithmetic operations.
    \index{Language-Specific Constraints}
    
    \item \textbf{Avoiding Redundant Calculations}: Ensure that each calculation contributes towards determining the final area without unnecessary repetitions.
    \index{Avoiding Redundant Calculations}
    
    \item \textbf{Code Readability and Documentation}: Maintain clear and readable code through meaningful variable names and comprehensive comments to facilitate understanding and maintenance.
    \index{Code Readability}
    
    \item \textbf{Edge Case Handling}: Implement checks for edge cases to prevent incorrect results or runtime errors.
    \index{Edge Case Handling}
    
    \item \textbf{Testing and Validation}: Develop a comprehensive suite of test cases that cover all possible scenarios, including edge cases, to validate the correctness and efficiency of the implementation.
    \index{Testing and Validation}
    
    \item \textbf{Scalability}: Design the algorithm to scale efficiently with increasing input sizes, maintaining performance and resource utilization.
    \index{Scalability}
    
    \item \textbf{Using Helper Functions}: Consider creating helper functions for repetitive tasks, such as calculating overlap dimensions, to enhance modularity and reusability.
    \index{Helper Functions}
    
    \item \textbf{Consistent Naming Conventions}: Use consistent and descriptive naming conventions for variables to improve code clarity.
    \index{Naming Conventions}
    
    \item \textbf{Implementing Unit Tests}: Develop unit tests for each logical condition to ensure that all scenarios are correctly handled.
    \index{Unit Tests}
    
    \item \textbf{Error Handling}: Incorporate error handling to manage invalid inputs gracefully.
    \index{Error Handling}
\end{itemize}

\section*{Conclusion}

The \textbf{Rectangle Area} problem showcases the application of fundamental geometric principles and efficient algorithm design to compute spatial properties accurately. By systematically calculating individual areas and intelligently handling overlapping regions, the algorithm ensures precise results without redundant computations. Understanding and implementing such techniques not only enhances problem-solving skills but also provides a foundation for tackling more complex Computational Geometry challenges involving multiple geometric entities and intricate spatial relationships.

\printindex

% \input{sections/rectangle_overlap}
% \input{sections/rectangle_area}
% \input{sections/k_closest_points_to_origin}
% \input{sections/the_skyline_problem}
% % filename: k_closest_points_to_origin.tex

\problemsection{K Closest Points to Origin}
\label{chap:K_Closest_Points_to_Origin}
\marginnote{\href{https://leetcode.com/problems/k-closest-points-to-origin/}{[LeetCode Link]}\index{LeetCode}}
\marginnote{\href{https://www.geeksforgeeks.org/find-k-closest-points-origin/}{[GeeksForGeeks Link]}\index{GeeksForGeeks}}
\marginnote{\href{https://www.interviewbit.com/problems/k-closest-points/}{[InterviewBit Link]}\index{InterviewBit}}
\marginnote{\href{https://app.codesignal.com/challenges/k-closest-points-to-origin}{[CodeSignal Link]}\index{CodeSignal}}
\marginnote{\href{https://www.codewars.com/kata/k-closest-points-to-origin/train/python}{[Codewars Link]}\index{Codewars}}

The \textbf{K Closest Points to Origin} problem is a popular algorithmic challenge in Computational Geometry that involves identifying the \(k\) points closest to the origin in a 2D plane. This problem tests one's ability to apply efficient sorting and selection algorithms, understand distance computations, and optimize for performance. Mastery of this problem is essential for applications in spatial data analysis, nearest neighbor searches, and clustering algorithms.

\section*{Problem Statement}

Given an array of points where each point is represented as \([x, y]\) in the 2D plane, and an integer \(k\), return the \(k\) closest points to the origin \((0, 0)\).

The distance between two points \((x_1, y_1)\) and \((x_2, y_2)\) is the Euclidean distance \(\sqrt{(x_1 - x_2)^2 + (y_1 - y_2)^2}\). The origin is \((0, 0)\).

\textbf{Function signature in Python:}
\begin{lstlisting}[language=Python]
def kClosest(points: List[List[int]], K: int) -> List[List[int]]:
\end{lstlisting}

\section*{Examples}

\textbf{Example 1:}

\begin{verbatim}
Input: points = [[1,3],[-2,2]], K = 1
Output: [[-2,2]]
Explanation: 
The distance between (1, 3) and the origin is sqrt(10).
The distance between (-2, 2) and the origin is sqrt(8).
Since sqrt(8) < sqrt(10), (-2, 2) is closer to the origin.
\end{verbatim}

\textbf{Example 2:}

\begin{verbatim}
Input: points = [[3,3],[5,-1],[-2,4]], K = 2
Output: [[3,3],[-2,4]]
Explanation: 
The distances are sqrt(18), sqrt(26), and sqrt(20) respectively.
The two closest points are [3,3] and [-2,4].
\end{verbatim}

\textbf{Example 3:}

\begin{verbatim}
Input: points = [[0,1],[1,0]], K = 2
Output: [[0,1],[1,0]]
Explanation: 
Both points are equally close to the origin.
\end{verbatim}

\textbf{Example 4:}

\begin{verbatim}
Input: points = [[1,0],[0,1]], K = 1
Output: [[1,0]]
Explanation: 
Both points are equally close; returning any one is acceptable.
\end{verbatim}

\textbf{Constraints:}

\begin{itemize}
    \item \(1 \leq K \leq \text{points.length} \leq 10^4\)
    \item \(-10^4 < x_i, y_i < 10^4\)
\end{itemize}

LeetCode link: \href{https://leetcode.com/problems/k-closest-points-to-origin/}{K Closest Points to Origin}\index{LeetCode}

\section*{Algorithmic Approach}

To identify the \(k\) closest points to the origin, several algorithmic strategies can be employed. The most efficient methods aim to reduce the time complexity by avoiding the need to sort the entire list of points.

\subsection*{1. Sorting Based on Distance}

Calculate the Euclidean distance of each point from the origin and sort the points based on these distances. Select the first \(k\) points from the sorted list.

\begin{enumerate}
    \item Compute the distance for each point using the formula \(distance = x^2 + y^2\).
    \item Sort the points based on the computed distances.
    \item Return the first \(k\) points from the sorted list.
\end{enumerate}

\subsection*{2. Max Heap (Priority Queue)}

Use a max heap to maintain the \(k\) closest points. Iterate through each point, add it to the heap, and if the heap size exceeds \(k\), remove the farthest point.

\begin{enumerate}
    \item Initialize a max heap.
    \item For each point, compute its distance and add it to the heap.
    \item If the heap size exceeds \(k\), remove the point with the largest distance.
    \item After processing all points, the heap contains the \(k\) closest points.
\end{enumerate}

\subsection*{3. QuickSelect (Quick Sort Partitioning)}

Utilize the QuickSelect algorithm to find the \(k\) closest points without fully sorting the list.

\begin{enumerate}
    \item Choose a pivot point and partition the list based on distances relative to the pivot.
    \item Recursively apply QuickSelect to the partition containing the \(k\) closest points.
    \item Once the \(k\) closest points are identified, return them.
\end{enumerate}

\marginnote{QuickSelect offers an average time complexity of \(O(n)\), making it highly efficient for large datasets.}

\section*{Complexities}

\begin{itemize}
    \item \textbf{Sorting Based on Distance:}
    \begin{itemize}
        \item \textbf{Time Complexity:} \(O(n \log n)\)
        \item \textbf{Space Complexity:} \(O(n)\)
    \end{itemize}
    
    \item \textbf{Max Heap (Priority Queue):}
    \begin{itemize}
        \item \textbf{Time Complexity:} \(O(n \log k)\)
        \item \textbf{Space Complexity:} \(O(k)\)
    \end{itemize}
    
    \item \textbf{QuickSelect (Quick Sort Partitioning):}
    \begin{itemize}
        \item \textbf{Time Complexity:} Average case \(O(n)\), worst case \(O(n^2)\)
        \item \textbf{Space Complexity:} \(O(1)\) (in-place)
    \end{itemize}
\end{itemize}

\section*{Python Implementation}

\marginnote{Implementing QuickSelect provides an optimal average-case solution with linear time complexity.}

Below is the complete Python code implementing the \texttt{kClosest} function using the QuickSelect approach:

\begin{fullwidth}
\begin{lstlisting}[language=Python]
from typing import List
import random

class Solution:
    def kClosest(self, points: List[List[int]], K: int) -> List[List[int]]:
        def quickselect(left, right, K_smallest):
            if left == right:
                return
            
            # Select a random pivot_index
            pivot_index = random.randint(left, right)
            
            # Partition the array
            pivot_index = partition(left, right, pivot_index)
            
            # The pivot is in its final sorted position
            if K_smallest == pivot_index:
                return
            elif K_smallest < pivot_index:
                quickselect(left, pivot_index - 1, K_smallest)
            else:
                quickselect(pivot_index + 1, right, K_smallest)
        
        def partition(left, right, pivot_index):
            pivot_distance = distance(points[pivot_index])
            # Move pivot to end
            points[pivot_index], points[right] = points[right], points[pivot_index]
            store_index = left
            for i in range(left, right):
                if distance(points[i]) < pivot_distance:
                    points[store_index], points[i] = points[i], points[store_index]
                    store_index += 1
            # Move pivot to its final place
            points[right], points[store_index] = points[store_index], points[right]
            return store_index
        
        def distance(point):
            return point[0] ** 2 + point[1] ** 2
        
        n = len(points)
        quickselect(0, n - 1, K)
        return points[:K]

# Example usage:
solution = Solution()
print(solution.kClosest([[1,3],[-2,2]], 1))            # Output: [[-2,2]]
print(solution.kClosest([[3,3],[5,-1],[-2,4]], 2))     # Output: [[3,3],[-2,4]]
print(solution.kClosest([[0,1],[1,0]], 2))             # Output: [[0,1],[1,0]]
print(solution.kClosest([[1,0],[0,1]], 1))             # Output: [[1,0]] or [[0,1]]
\end{lstlisting}
\end{fullwidth}

This implementation uses the QuickSelect algorithm to efficiently find the \(k\) closest points to the origin without fully sorting the entire list. It ensures optimal performance even with large datasets.

\section*{Explanation}

The \texttt{kClosest} function identifies the \(k\) closest points to the origin using the QuickSelect algorithm. Here's a detailed breakdown of the implementation:

\subsection*{1. Distance Calculation}

\begin{itemize}
    \item The Euclidean distance is calculated as \(distance = x^2 + y^2\). Since we only need relative distances for comparison, the square root is omitted for efficiency.
\end{itemize}

\subsection*{2. QuickSelect Algorithm}

\begin{itemize}
    \item **Pivot Selection:**
    \begin{itemize}
        \item A random pivot is chosen to enhance the average-case performance.
    \end{itemize}
    
    \item **Partitioning:**
    \begin{itemize}
        \item The array is partitioned such that points with distances less than the pivot are moved to the left, and others to the right.
        \item The pivot is placed in its correct sorted position.
    \end{itemize}
    
    \item **Recursive Selection:**
    \begin{itemize}
        \item If the pivot's position matches \(K\), the selection is complete.
        \item Otherwise, recursively apply QuickSelect to the relevant partition.
    \end{itemize}
\end{itemize}

\subsection*{3. Final Selection}

\begin{itemize}
    \item After partitioning, the first \(K\) points in the list are the \(k\) closest points to the origin.
\end{itemize}

\subsection*{4. Example Walkthrough}

Consider the first example:
\begin{verbatim}
Input: points = [[1,3],[-2,2]], K = 1
Output: [[-2,2]]
\end{verbatim}

\begin{enumerate}
    \item **Calculate Distances:**
    \begin{itemize}
        \item [1,3] : \(1^2 + 3^2 = 10\)
        \item [-2,2] : \((-2)^2 + 2^2 = 8\)
    \end{itemize}
    
    \item **QuickSelect Process:**
    \begin{itemize}
        \item Choose a pivot, say [1,3] with distance 10.
        \item Compare and rearrange:
        \begin{itemize}
            \item [-2,2] has a smaller distance (8) and is moved to the left.
        \end{itemize}
        \item After partitioning, the list becomes [[-2,2], [1,3]].
        \item Since \(K = 1\), return the first point: [[-2,2]].
    \end{itemize}
\end{enumerate}

Thus, the function correctly identifies \([-2,2]\) as the closest point to the origin.

\section*{Why This Approach}

The QuickSelect algorithm is chosen for its average-case linear time complexity \(O(n)\), making it highly efficient for large datasets compared to sorting-based methods with \(O(n \log n)\) time complexity. By avoiding the need to sort the entire list, QuickSelect provides an optimal solution for finding the \(k\) closest points.

\section*{Alternative Approaches}

\subsection*{1. Sorting Based on Distance}

Sort all points based on their distances from the origin and select the first \(k\) points.

\begin{lstlisting}[language=Python]
class Solution:
    def kClosest(self, points: List[List[int]], K: int) -> List[List[int]]:
        points.sort(key=lambda P: P[0]**2 + P[1]**2)
        return points[:K]
\end{lstlisting}

\textbf{Complexities:}
\begin{itemize}
    \item \textbf{Time Complexity:} \(O(n \log n)\)
    \item \textbf{Space Complexity:} \(O(1)\)
\end{itemize}

\subsection*{2. Max Heap (Priority Queue)}

Use a max heap to maintain the \(k\) closest points.

\begin{lstlisting}[language=Python]
import heapq

class Solution:
    def kClosest(self, points: List[List[int]], K: int) -> List[List[int]]:
        heap = []
        for (x, y) in points:
            dist = -(x**2 + y**2)  # Max heap using negative distances
            heapq.heappush(heap, (dist, [x, y]))
            if len(heap) > K:
                heapq.heappop(heap)
        return [item[1] for item in heap]
\end{lstlisting}

\textbf{Complexities:}
\begin{itemize}
    \item \textbf{Time Complexity:} \(O(n \log k)\)
    \item \textbf{Space Complexity:} \(O(k)\)
\end{itemize}

\subsection*{3. Using Built-In Functions}

Leverage built-in functions for distance calculation and selection.

\begin{lstlisting}[language=Python]
import math

class Solution:
    def kClosest(self, points: List[List[int]], K: int) -> List[List[int]]:
        points.sort(key=lambda P: math.sqrt(P[0]**2 + P[1]**2))
        return points[:K]
\end{lstlisting}

\textbf{Note}: This method is similar to the sorting approach but uses the actual Euclidean distance.

\section*{Similar Problems to This One}

Several problems involve nearest neighbor searches, spatial data analysis, and efficient selection algorithms, utilizing similar algorithmic strategies:

\begin{itemize}
    \item \textbf{Closest Pair of Points}: Find the closest pair of points in a set.
    \item \textbf{Top K Frequent Elements}: Identify the most frequent elements in a dataset.
    \item \textbf{Kth Largest Element in an Array}: Find the \(k\)-th largest element in an unsorted array.
    \item \textbf{Sliding Window Maximum}: Find the maximum in each sliding window of size \(k\) over an array.
    \item \textbf{Merge K Sorted Lists}: Merge multiple sorted lists into a single sorted list.
    \item \textbf{Find Median from Data Stream}: Continuously find the median of a stream of numbers.
    \item \textbf{Top K Closest Stars}: Find the \(k\) closest stars to Earth based on their distances.
\end{itemize}

These problems reinforce concepts of efficient selection, heap usage, and distance computations in various contexts.

\section*{Things to Keep in Mind and Tricks}

When solving the \textbf{K Closest Points to Origin} problem, consider the following tips and best practices to enhance efficiency and correctness:

\begin{itemize}
    \item \textbf{Understand Distance Calculations}: Grasp the Euclidean distance formula and recognize that the square root can be omitted for comparison purposes.
    \index{Distance Calculations}
    
    \item \textbf{Leverage Efficient Algorithms}: Use QuickSelect or heap-based methods to optimize time complexity, especially for large datasets.
    \index{Efficient Algorithms}
    
    \item \textbf{Handle Ties Appropriately}: Decide how to handle points with identical distances when \(k\) is less than the number of such points.
    \index{Handling Ties}
    
    \item \textbf{Optimize Space Usage}: Choose algorithms that minimize additional space, such as in-place QuickSelect.
    \index{Space Optimization}
    
    \item \textbf{Use Appropriate Data Structures}: Utilize heaps, lists, and helper functions effectively to manage and process data.
    \index{Data Structures}
    
    \item \textbf{Implement Helper Functions}: Create helper functions for distance calculation and partitioning to enhance code modularity.
    \index{Helper Functions}
    
    \item \textbf{Code Readability}: Maintain clear and readable code through meaningful variable names and structured logic.
    \index{Code Readability}
    
    \item \textbf{Test Extensively}: Implement a wide range of test cases, including edge cases like multiple points with the same distance, to ensure robustness.
    \index{Extensive Testing}
    
    \item \textbf{Understand Algorithm Trade-offs}: Recognize the trade-offs between different approaches in terms of time and space complexities.
    \index{Algorithm Trade-offs}
    
    \item \textbf{Use Built-In Sorting Functions}: When using sorting-based approaches, leverage built-in functions for efficiency and simplicity.
    \index{Built-In Sorting}
    
    \item \textbf{Avoid Redundant Calculations}: Ensure that distance calculations are performed only when necessary to optimize performance.
    \index{Avoiding Redundant Calculations}
    
    \item \textbf{Language-Specific Features}: Utilize language-specific features or libraries that can simplify implementation, such as heapq in Python.
    \index{Language-Specific Features}
\end{itemize}

\section*{Corner and Special Cases to Test When Writing the Code}

When implementing the solution for the \textbf{K Closest Points to Origin} problem, it is crucial to consider and rigorously test various edge cases to ensure robustness and correctness:

\begin{itemize}
    \item \textbf{Multiple Points with Same Distance}: Ensure that the algorithm handles multiple points having the same distance from the origin.
    \index{Same Distance Points}
    
    \item \textbf{Points at Origin}: Include points that are exactly at the origin \((0,0)\).
    \index{Points at Origin}
    
    \item \textbf{Negative Coordinates}: Ensure that the algorithm correctly computes distances for points with negative \(x\) or \(y\) coordinates.
    \index{Negative Coordinates}
    
    \item \textbf{Large Coordinates}: Test with points having very large or very small coordinate values to verify integer handling.
    \index{Large Coordinates}
    
    \item \textbf{K Equals Number of Points}: When \(K\) is equal to the number of points, the algorithm should return all points.
    \index{K Equals Number of Points}
    
    \item \textbf{K is One}: Test with \(K = 1\) to ensure the closest point is correctly identified.
    \index{K is One}
    
    \item \textbf{All Points Same}: All points have the same coordinates.
    \index{All Points Same}
    
    \item \textbf{K is Zero}: Although \(K\) is defined to be at least 1, ensure that the algorithm gracefully handles \(K = 0\) if allowed.
    \index{K is Zero}
    
    \item \textbf{Single Point}: Only one point is provided, and \(K = 1\).
    \index{Single Point}
    
    \item \textbf{Mixed Coordinates}: Points with a mix of positive and negative coordinates.
    \index{Mixed Coordinates}
    
    \item \textbf{Points with Zero Distance}: Multiple points at the origin.
    \index{Zero Distance Points}
    
    \item \textbf{Sparse and Dense Points}: Densely packed points and sparsely distributed points.
    \index{Sparse and Dense Points}
    
    \item \textbf{Duplicate Points}: Multiple identical points in the input list.
    \index{Duplicate Points}
    
    \item \textbf{K Greater Than Number of Unique Points}: Ensure that the algorithm handles cases where \(K\) exceeds the number of unique points if applicable.
    \index{K Greater Than Unique Points}
\end{itemize}

\section*{Implementation Considerations}

When implementing the \texttt{kClosest} function, keep in mind the following considerations to ensure robustness and efficiency:

\begin{itemize}
    \item \textbf{Data Type Selection}: Use appropriate data types that can handle large input values without overflow or precision loss.
    \index{Data Type Selection}
    
    \item \textbf{Optimizing Distance Calculations}: Avoid calculating the square root since it is unnecessary for comparison purposes.
    \index{Optimizing Distance Calculations}
    
    \item \textbf{Choosing the Right Algorithm}: Select an algorithm based on the size of the input and the value of \(K\) to optimize time and space complexities.
    \index{Choosing the Right Algorithm}
    
    \item \textbf{Language-Specific Libraries}: Utilize language-specific libraries and functions (e.g., \texttt{heapq} in Python) to simplify implementation and enhance performance.
    \index{Language-Specific Libraries}
    
    \item \textbf{Avoiding Redundant Calculations}: Ensure that each point's distance is calculated only once to optimize performance.
    \index{Avoiding Redundant Calculations}
    
    \item \textbf{Implementing Helper Functions}: Create helper functions for tasks like distance calculation and partitioning to enhance modularity and readability.
    \index{Helper Functions}
    
    \item \textbf{Edge Case Handling}: Implement checks for edge cases to prevent incorrect results or runtime errors.
    \index{Edge Case Handling}
    
    \item \textbf{Testing and Validation}: Develop a comprehensive suite of test cases that cover all possible scenarios, including edge cases, to validate the correctness and efficiency of the implementation.
    \index{Testing and Validation}
    
    \item \textbf{Scalability}: Design the algorithm to scale efficiently with increasing input sizes, maintaining performance and resource utilization.
    \index{Scalability}
    
    \item \textbf{Consistent Naming Conventions}: Use consistent and descriptive naming conventions for variables and functions to improve code clarity.
    \index{Naming Conventions}
    
    \item \textbf{Memory Management}: Ensure that the algorithm manages memory efficiently, especially when dealing with large datasets.
    \index{Memory Management}
    
    \item \textbf{Avoiding Stack Overflow}: If implementing recursive approaches, be mindful of recursion limits and potential stack overflow issues.
    \index{Avoiding Stack Overflow}
    
    \item \textbf{Implementing Iterative Solutions}: Prefer iterative solutions when recursion may lead to increased space complexity or stack overflow.
    \index{Implementing Iterative Solutions}
\end{itemize}

\section*{Conclusion}

The \textbf{K Closest Points to Origin} problem exemplifies the application of efficient selection algorithms and geometric computations to solve spatial challenges effectively. By leveraging QuickSelect or heap-based methods, the algorithm achieves optimal time and space complexities, making it highly suitable for large datasets. Understanding and implementing such techniques not only enhances problem-solving skills but also provides a foundation for tackling more advanced Computational Geometry problems involving nearest neighbor searches, clustering, and spatial data analysis.

\printindex

% \input{sections/rectangle_overlap}
% \input{sections/rectangle_area}
% \input{sections/k_closest_points_to_origin}
% \input{sections/the_skyline_problem}
% % filename: the_skyline_problem.tex

\problemsection{The Skyline Problem}
\label{chap:The_Skyline_Problem}
\marginnote{\href{https://leetcode.com/problems/the-skyline-problem/}{[LeetCode Link]}\index{LeetCode}}
\marginnote{\href{https://www.geeksforgeeks.org/the-skyline-problem/}{[GeeksForGeeks Link]}\index{GeeksForGeeks}}
\marginnote{\href{https://www.interviewbit.com/problems/the-skyline-problem/}{[InterviewBit Link]}\index{InterviewBit}}
\marginnote{\href{https://app.codesignal.com/challenges/the-skyline-problem}{[CodeSignal Link]}\index{CodeSignal}}
\marginnote{\href{https://www.codewars.com/kata/the-skyline-problem/train/python}{[Codewars Link]}\index{Codewars}}

The \textbf{Skyline Problem} is a complex Computational Geometry challenge that involves computing the skyline formed by a collection of buildings in a 2D cityscape. Each building is represented by its left and right x-coordinates and its height. The skyline is defined by a list of "key points" where the height changes. This problem tests one's ability to handle large datasets, implement efficient sweep line algorithms, and manage event-driven processing. Mastery of this problem is essential for applications in computer graphics, urban planning simulations, and geographic information systems (GIS).

\section*{Problem Statement}

You are given a list of buildings in a cityscape. Each building is represented as a triplet \([Li, Ri, Hi]\), where \(Li\) and \(Ri\) are the x-coordinates of the left and right edges of the building, respectively, and \(Hi\) is the height of the building.

The skyline should be represented as a list of key points \([x, y]\) in sorted order by \(x\)-coordinate, where \(y\) is the height of the skyline at that point. The skyline should only include critical points where the height changes.

\textbf{Function signature in Python:}
\begin{lstlisting}[language=Python]
def getSkyline(buildings: List[List[int]]) -> List[List[int]]:
\end{lstlisting}

\section*{Examples}

\textbf{Example 1:}

\begin{verbatim}
Input: buildings = [[2,9,10], [3,7,15], [5,12,12], [15,20,10], [19,24,8]]
Output: [[2,10], [3,15], [7,12], [12,0], [15,10], [20,8], [24,0]]
Explanation:
- At x=2, the first building starts, height=10.
- At x=3, the second building starts, height=15.
- At x=7, the second building ends, the third building is still ongoing, height=12.
- At x=12, the third building ends, height drops to 0.
- At x=15, the fourth building starts, height=10.
- At x=20, the fourth building ends, the fifth building is still ongoing, height=8.
- At x=24, the fifth building ends, height drops to 0.
\end{verbatim}

\textbf{Example 2:}

\begin{verbatim}
Input: buildings = [[0,2,3], [2,5,3]]
Output: [[0,3], [5,0]]
Explanation:
- The two buildings are contiguous and have the same height, so the skyline drops to 0 at x=5.
\end{verbatim}

\textbf{Example 3:}

\begin{verbatim}
Input: buildings = [[1,3,3], [2,4,4], [5,6,1]]
Output: [[1,3], [2,4], [4,0], [5,1], [6,0]]
Explanation:
- At x=1, first building starts, height=3.
- At x=2, second building starts, height=4.
- At x=4, second building ends, height drops to 0.
- At x=5, third building starts, height=1.
- At x=6, third building ends, height drops to 0.
\end{verbatim}

\textbf{Example 4:}

\begin{verbatim}
Input: buildings = [[0,5,0]]
Output: []
Explanation:
- A building with height 0 does not contribute to the skyline.
\end{verbatim}

\textbf{Constraints:}

\begin{itemize}
    \item \(1 \leq \text{buildings.length} \leq 10^4\)
    \item \(0 \leq Li < Ri \leq 10^9\)
    \item \(0 \leq Hi \leq 10^4\)
\end{itemize}

\section*{Algorithmic Approach}

The \textbf{Sweep Line Algorithm} is an efficient method for solving the Skyline Problem. It involves processing events (building start and end points) in sorted order while maintaining a data structure (typically a max heap) to keep track of active buildings. Here's a step-by-step approach:

\subsection*{1. Event Representation}

Transform each building into two events:
\begin{itemize}
    \item **Start Event:** \((Li, -Hi)\) – Negative height indicates a building starts.
    \item **End Event:** \((Ri, Hi)\) – Positive height indicates a building ends.
\end{itemize}

Sorting the events ensures that start events are processed before end events at the same x-coordinate, and taller buildings are processed before shorter ones.

\subsection*{2. Sorting the Events}

Sort all events based on:
\begin{enumerate}
    \item **x-coordinate:** Ascending order.
    \item **Height:**
    \begin{itemize}
        \item For start events, taller buildings come first.
        \item For end events, shorter buildings come first.
    \end{itemize}
\end{enumerate}

\subsection*{3. Processing the Events}

Use a max heap to keep track of active building heights. Iterate through the sorted events:
\begin{enumerate}
    \item **Start Event:**
    \begin{itemize}
        \item Add the building's height to the heap.
    \end{itemize}
    
    \item **End Event:**
    \begin{itemize}
        \item Remove the building's height from the heap.
    \end{itemize}
    
    \item **Determine Current Max Height:**
    \begin{itemize}
        \item The current max height is the top of the heap.
    \end{itemize}
    
    \item **Update Skyline:**
    \begin{itemize}
        \item If the current max height differs from the previous max height, add a new key point \([x, current\_max\_height]\).
    \end{itemize}
\end{enumerate}

\subsection*{4. Finalizing the Skyline}

After processing all events, the accumulated key points represent the skyline.

\marginnote{The Sweep Line Algorithm efficiently handles dynamic changes in active buildings, ensuring accurate skyline construction.}

\section*{Complexities}

\begin{itemize}
    \item \textbf{Time Complexity:} \(O(n \log n)\), where \(n\) is the number of buildings. Sorting the events takes \(O(n \log n)\), and each heap operation takes \(O(\log n)\).
    
    \item \textbf{Space Complexity:} \(O(n)\), due to the storage of events and the heap.
\end{itemize}

\section*{Python Implementation}

\marginnote{Implementing the Sweep Line Algorithm with a max heap ensures an efficient and accurate solution.}

Below is the complete Python code implementing the \texttt{getSkyline} function:

\begin{fullwidth}
\begin{lstlisting}[language=Python]
from typing import List
import heapq

class Solution:
    def getSkyline(self, buildings: List[List[int]]) -> List[List[int]]:
        # Create a list of all events
        # For start events, use negative height to ensure they are processed before end events
        events = []
        for L, R, H in buildings:
            events.append((L, -H))
            events.append((R, H))
        
        # Sort the events
        # First by x-coordinate, then by height
        events.sort()
        
        # Max heap to keep track of active buildings
        heap = [0]  # Initialize with ground level
        heapq.heapify(heap)
        active_heights = {0: 1}  # Dictionary to count heights
        
        result = []
        prev_max = 0
        
        for x, h in events:
            if h < 0:
                # Start of a building, add height to heap and dictionary
                heapq.heappush(heap, h)
                active_heights[h] = active_heights.get(h, 0) + 1
            else:
                # End of a building, remove height from dictionary
                active_heights[h] -= 1
                if active_heights[h] == 0:
                    del active_heights[h]
            
            # Current max height
            while heap and active_heights.get(heap[0], 0) == 0:
                heapq.heappop(heap)
            current_max = -heap[0] if heap else 0
            
            # If the max height has changed, add to result
            if current_max != prev_max:
                result.append([x, current_max])
                prev_max = current_max
        
        return result

# Example usage:
solution = Solution()
print(solution.getSkyline([[2,9,10], [3,7,15], [5,12,12], [15,20,10], [19,24,8]]))
# Output: [[2,10], [3,15], [7,12], [12,0], [15,10], [20,8], [24,0]]

print(solution.getSkyline([[0,2,3], [2,5,3]]))
# Output: [[0,3], [5,0]]

print(solution.getSkyline([[1,3,3], [2,4,4], [5,6,1]]))
# Output: [[1,3], [2,4], [4,0], [5,1], [6,0]]

print(solution.getSkyline([[0,5,0]]))
# Output: []
\end{lstlisting}
\end{fullwidth}

This implementation efficiently constructs the skyline by processing all building events in sorted order and maintaining active building heights using a max heap. It ensures that only critical points where the skyline changes are recorded.

\section*{Explanation}

The \texttt{getSkyline} function constructs the skyline formed by a set of buildings by leveraging the Sweep Line Algorithm and a max heap to track active buildings. Here's a detailed breakdown of the implementation:

\subsection*{1. Event Representation}

\begin{itemize}
    \item Each building is transformed into two events:
    \begin{itemize}
        \item **Start Event:** \((Li, -Hi)\) – Negative height indicates the start of a building.
        \item **End Event:** \((Ri, Hi)\) – Positive height indicates the end of a building.
    \end{itemize}
\end{itemize}

\subsection*{2. Sorting the Events}

\begin{itemize}
    \item Events are sorted primarily by their x-coordinate in ascending order.
    \item For events with the same x-coordinate:
    \begin{itemize}
        \item Start events (with negative heights) are processed before end events.
        \item Taller buildings are processed before shorter ones.
    \end{itemize}
\end{itemize}

\subsection*{3. Processing the Events}

\begin{itemize}
    \item **Heap Initialization:**
    \begin{itemize}
        \item A max heap is initialized with a ground level height of 0.
        \item A dictionary \texttt{active\_heights} tracks the count of active building heights.
    \end{itemize}
    
    \item **Iterating Through Events:**
    \begin{enumerate}
        \item **Start Event:**
        \begin{itemize}
            \item Add the building's height to the heap.
            \item Increment the count of the height in \texttt{active\_heights}.
        \end{itemize}
        
        \item **End Event:**
        \begin{itemize}
            \item Decrement the count of the building's height in \texttt{active\_heights}.
            \item If the count reaches zero, remove the height from the dictionary.
        \end{itemize}
        
        \item **Determine Current Max Height:**
        \begin{itemize}
            \item Remove heights from the heap that are no longer active.
            \item The current max height is the top of the heap.
        \end{itemize}
        
        \item **Update Skyline:**
        \begin{itemize}
            \item If the current max height differs from the previous max height, add a new key point \([x, current\_max\_height]\).
        \end{itemize}
    \end{enumerate}
\end{itemize}

\subsection*{4. Finalizing the Skyline}

\begin{itemize}
    \item After processing all events, the \texttt{result} list contains the key points defining the skyline.
\end{itemize}

\subsection*{5. Example Walkthrough}

Consider the first example:
\begin{verbatim}
Input: buildings = [[2,9,10], [3,7,15], [5,12,12], [15,20,10], [19,24,8]]
Output: [[2,10], [3,15], [7,12], [12,0], [15,10], [20,8], [24,0]]
\end{verbatim}

\begin{enumerate}
    \item **Event Transformation:**
    \begin{itemize}
        \item \((2, -10)\), \((9, 10)\)
        \item \((3, -15)\), \((7, 15)\)
        \item \((5, -12)\), \((12, 12)\)
        \item \((15, -10)\), \((20, 10)\)
        \item \((19, -8)\), \((24, 8)\)
    \end{itemize}
    
    \item **Sorting Events:**
    \begin{itemize}
        \item Sorted order: \((2, -10)\), \((3, -15)\), \((5, -12)\), \((7, 15)\), \((9, 10)\), \((12, 12)\), \((15, -10)\), \((19, -8)\), \((20, 10)\), \((24, 8)\)
    \end{itemize}
    
    \item **Processing Events:**
    \begin{itemize}
        \item At each event, update the heap and determine if the skyline height changes.
    \end{itemize}
    
    \item **Result Construction:**
    \begin{itemize}
        \item The resulting skyline key points are accumulated as \([[2,10], [3,15], [7,12], [12,0], [15,10], [20,8], [24,0]]\).
    \end{itemize}
\end{enumerate}

Thus, the function correctly constructs the skyline based on the buildings' positions and heights.

\section*{Why This Approach}

The Sweep Line Algorithm combined with a max heap offers an optimal solution with \(O(n \log n)\) time complexity and efficient handling of overlapping buildings. By processing events in sorted order and maintaining active building heights, the algorithm ensures that all critical points in the skyline are accurately identified without redundant computations.

\section*{Alternative Approaches}

\subsection*{1. Divide and Conquer}

Divide the set of buildings into smaller subsets, compute the skyline for each subset, and then merge the skylines.

\begin{lstlisting}[language=Python]
class Solution:
    def getSkyline(self, buildings: List[List[int]]) -> List[List[int]]:
        def merge(left, right):
            h1, h2 = 0, 0
            i, j = 0, 0
            merged = []
            while i < len(left) and j < len(right):
                if left[i][0] < right[j][0]:
                    x, h1 = left[i]
                    i += 1
                elif left[i][0] > right[j][0]:
                    x, h2 = right[j]
                    j += 1
                else:
                    x, h1 = left[i]
                    _, h2 = right[j]
                    i += 1
                    j += 1
                max_h = max(h1, h2)
                if not merged or merged[-1][1] != max_h:
                    merged.append([x, max_h])
            merged.extend(left[i:])
            merged.extend(right[j:])
            return merged
        
        def divide(buildings):
            if not buildings:
                return []
            if len(buildings) == 1:
                L, R, H = buildings[0]
                return [[L, H], [R, 0]]
            mid = len(buildings) // 2
            left = divide(buildings[:mid])
            right = divide(buildings[mid:])
            return merge(left, right)
        
        return divide(buildings)
\end{lstlisting}

\textbf{Complexities:}
\begin{itemize}
    \item \textbf{Time Complexity:} \(O(n \log n)\)
    \item \textbf{Space Complexity:} \(O(n)\)
\end{itemize}

\subsection*{2. Using Segment Trees}

Implement a segment tree to manage and query overlapping building heights dynamically.

\textbf{Note}: This approach is more complex and is generally used for advanced scenarios with multiple dynamic queries.

\section*{Similar Problems to This One}

Several problems involve skyline-like constructions, spatial data analysis, and efficient event processing, utilizing similar algorithmic strategies:

\begin{itemize}
    \item \textbf{Merge Intervals}: Merge overlapping intervals in a list.
    \item \textbf{Largest Rectangle in Histogram}: Find the largest rectangular area in a histogram.
    \item \textbf{Interval Partitioning}: Assign intervals to resources without overlap.
    \item \textbf{Line Segment Intersection}: Detect intersections among line segments.
    \item \textbf{Closest Pair of Points}: Find the closest pair of points in a set.
    \item \textbf{Convex Hull}: Compute the convex hull of a set of points.
    \item \textbf{Point Inside Polygon}: Determine if a point lies inside a given polygon.
    \item \textbf{Range Searching}: Efficiently query geometric data within a specified range.
\end{itemize}

These problems reinforce concepts of event-driven processing, spatial reasoning, and efficient algorithm design in various contexts.

\section*{Things to Keep in Mind and Tricks}

When tackling the \textbf{Skyline Problem}, consider the following tips and best practices to enhance efficiency and correctness:

\begin{itemize}
    \item \textbf{Understand Sweep Line Technique}: Grasp how the sweep line algorithm processes events in sorted order to handle dynamic changes efficiently.
    \index{Sweep Line Technique}
    
    \item \textbf{Leverage Priority Queues (Heaps)}: Use max heaps to keep track of active buildings' heights, enabling quick access to the current maximum height.
    \index{Priority Queues}
    
    \item \textbf{Handle Start and End Events Differently}: Differentiate between building start and end events to accurately manage active heights.
    \index{Start and End Events}
    
    \item \textbf{Optimize Event Sorting}: Sort events primarily by x-coordinate and secondarily by height to ensure correct processing order.
    \index{Event Sorting}
    
    \item \textbf{Manage Active Heights Efficiently}: Use data structures that allow efficient insertion, deletion, and retrieval of maximum elements.
    \index{Active Heights Management}
    
    \item \textbf{Avoid Redundant Key Points}: Only record key points when the skyline height changes to minimize the output list.
    \index{Avoiding Redundant Key Points}
    
    \item \textbf{Implement Helper Functions}: Create helper functions for tasks like distance calculation, event handling, and heap management to enhance modularity.
    \index{Helper Functions}
    
    \item \textbf{Code Readability}: Maintain clear and readable code through meaningful variable names and structured logic.
    \index{Code Readability}
    
    \item \textbf{Test Extensively}: Implement a wide range of test cases, including overlapping, non-overlapping, and edge-touching buildings, to ensure robustness.
    \index{Extensive Testing}
    
    \item \textbf{Handle Degenerate Cases}: Manage cases where buildings have zero height or identical coordinates gracefully.
    \index{Degenerate Cases}
    
    \item \textbf{Understand Geometric Relationships}: Grasp how buildings overlap and influence the skyline to simplify the algorithm.
    \index{Geometric Relationships}
    
    \item \textbf{Use Appropriate Data Structures}: Utilize appropriate data structures like heaps, lists, and dictionaries to manage and process data efficiently.
    \index{Appropriate Data Structures}
    
    \item \textbf{Optimize for Large Inputs}: Design the algorithm to handle large numbers of buildings without significant performance degradation.
    \index{Optimizing for Large Inputs}
    
    \item \textbf{Implement Iterative Solutions Carefully}: Ensure that loop conditions are correctly defined to prevent infinite loops or incorrect terminations.
    \index{Iterative Solutions}
    
    \item \textbf{Consistent Naming Conventions}: Use consistent and descriptive naming conventions for variables and functions to improve code clarity.
    \index{Naming Conventions}
\end{itemize}

\section*{Corner and Special Cases to Test When Writing the Code}

When implementing the solution for the \textbf{Skyline Problem}, it is crucial to consider and rigorously test various edge cases to ensure robustness and correctness:

\begin{itemize}
    \item \textbf{No Overlapping Buildings}: All buildings are separate and do not overlap.
    \index{No Overlapping Buildings}
    
    \item \textbf{Fully Overlapping Buildings}: Multiple buildings completely overlap each other.
    \index{Fully Overlapping Buildings}
    
    \item \textbf{Buildings Touching at Edges}: Buildings share common edges without overlapping.
    \index{Buildings Touching at Edges}
    
    \item \textbf{Buildings Touching at Corners}: Buildings share common corners without overlapping.
    \index{Buildings Touching at Corners}
    
    \item \textbf{Single Building}: Only one building is present.
    \index{Single Building}
    
    \item \textbf{Multiple Buildings with Same Start or End}: Multiple buildings start or end at the same x-coordinate.
    \index{Same Start or End}
    
    \item \textbf{Buildings with Zero Height}: Buildings that have zero height should not affect the skyline.
    \index{Buildings with Zero Height}
    
    \item \textbf{Large Number of Buildings}: Test with a large number of buildings to ensure performance and scalability.
    \index{Large Number of Buildings}
    
    \item \textbf{Buildings with Negative Coordinates}: Buildings positioned in negative coordinate space.
    \index{Negative Coordinates}
    
    \item \textbf{Boundary Values}: Buildings at the minimum and maximum limits of the coordinate range.
    \index{Boundary Values}
    
    \item \textbf{Buildings with Identical Coordinates}: Multiple buildings with the same coordinates.
    \index{Identical Coordinates}
    
    \item \textbf{Sequential Buildings}: Buildings placed sequentially without gaps.
    \index{Sequential Buildings}
    
    \item \textbf{Overlapping and Non-Overlapping Mixed}: A mix of overlapping and non-overlapping buildings.
    \index{Overlapping and Non-Overlapping Mixed}
    
    \item \textbf{Buildings with Very Large Heights}: Buildings with heights at the upper limit of the constraints.
    \index{Very Large Heights}
    
    \item \textbf{Empty Input}: No buildings are provided.
    \index{Empty Input}
\end{itemize}

\section*{Implementation Considerations}

When implementing the \texttt{getSkyline} function, keep in mind the following considerations to ensure robustness and efficiency:

\begin{itemize}
    \item \textbf{Data Type Selection}: Use appropriate data types that can handle large input values and avoid overflow or precision issues.
    \index{Data Type Selection}
    
    \item \textbf{Optimizing Event Sorting}: Efficiently sort events based on x-coordinates and heights to ensure correct processing order.
    \index{Optimizing Event Sorting}
    
    \item \textbf{Handling Large Inputs}: Design the algorithm to handle up to \(10^4\) buildings efficiently without significant performance degradation.
    \index{Handling Large Inputs}
    
    \item \textbf{Using Efficient Data Structures}: Utilize heaps, lists, and dictionaries effectively to manage and process events and active heights.
    \index{Efficient Data Structures}
    
    \item \textbf{Avoiding Redundant Calculations}: Ensure that distance and overlap calculations are performed only when necessary to optimize performance.
    \index{Avoiding Redundant Calculations}
    
    \item \textbf{Code Readability and Documentation}: Maintain clear and readable code through meaningful variable names and comprehensive comments to facilitate understanding and maintenance.
    \index{Code Readability}
    
    \item \textbf{Edge Case Handling}: Implement checks for edge cases to prevent incorrect results or runtime errors.
    \index{Edge Case Handling}
    
    \item \textbf{Implementing Helper Functions}: Create helper functions for tasks like distance calculation, event handling, and heap management to enhance modularity.
    \index{Helper Functions}
    
    \item \textbf{Consistent Naming Conventions}: Use consistent and descriptive naming conventions for variables and functions to improve code clarity.
    \index{Naming Conventions}
    
    \item \textbf{Memory Management}: Ensure that the algorithm manages memory efficiently, especially when dealing with large datasets.
    \index{Memory Management}
    
    \item \textbf{Implementing Iterative Solutions Carefully}: Ensure that loop conditions are correctly defined to prevent infinite loops or incorrect terminations.
    \index{Iterative Solutions}
    
    \item \textbf{Avoiding Floating-Point Precision Issues}: Since the problem deals with integers, floating-point precision is not a concern, simplifying the implementation.
    \index{Floating-Point Precision}
    
    \item \textbf{Testing and Validation}: Develop a comprehensive suite of test cases that cover all possible scenarios, including edge cases, to validate the correctness and efficiency of the implementation.
    \index{Testing and Validation}
    
    \item \textbf{Performance Considerations}: Optimize the loop conditions and operations to ensure that the function runs efficiently, especially for large input numbers.
    \index{Performance Considerations}
\end{itemize}

\section*{Conclusion}

The \textbf{Skyline Problem} is a quintessential example of applying advanced algorithmic techniques and geometric reasoning to solve complex spatial challenges. By leveraging the Sweep Line Algorithm and maintaining active building heights using a max heap, the solution efficiently constructs the skyline with optimal time and space complexities. Understanding and implementing such sophisticated algorithms not only enhances problem-solving skills but also provides a foundation for tackling a wide array of Computational Geometry problems in various domains, including computer graphics, urban planning simulations, and geographic information systems.

\printindex

% \input{sections/rectangle_overlap}
% \input{sections/rectangle_area}
% \input{sections/k_closest_points_to_origin}
% \input{sections/the_skyline_problem}
% % filename: the_skyline_problem.tex

\problemsection{The Skyline Problem}
\label{chap:The_Skyline_Problem}
\marginnote{\href{https://leetcode.com/problems/the-skyline-problem/}{[LeetCode Link]}\index{LeetCode}}
\marginnote{\href{https://www.geeksforgeeks.org/the-skyline-problem/}{[GeeksForGeeks Link]}\index{GeeksForGeeks}}
\marginnote{\href{https://www.interviewbit.com/problems/the-skyline-problem/}{[InterviewBit Link]}\index{InterviewBit}}
\marginnote{\href{https://app.codesignal.com/challenges/the-skyline-problem}{[CodeSignal Link]}\index{CodeSignal}}
\marginnote{\href{https://www.codewars.com/kata/the-skyline-problem/train/python}{[Codewars Link]}\index{Codewars}}

The \textbf{Skyline Problem} is a complex Computational Geometry challenge that involves computing the skyline formed by a collection of buildings in a 2D cityscape. Each building is represented by its left and right x-coordinates and its height. The skyline is defined by a list of "key points" where the height changes. This problem tests one's ability to handle large datasets, implement efficient sweep line algorithms, and manage event-driven processing. Mastery of this problem is essential for applications in computer graphics, urban planning simulations, and geographic information systems (GIS).

\section*{Problem Statement}

You are given a list of buildings in a cityscape. Each building is represented as a triplet \([Li, Ri, Hi]\), where \(Li\) and \(Ri\) are the x-coordinates of the left and right edges of the building, respectively, and \(Hi\) is the height of the building.

The skyline should be represented as a list of key points \([x, y]\) in sorted order by \(x\)-coordinate, where \(y\) is the height of the skyline at that point. The skyline should only include critical points where the height changes.

\textbf{Function signature in Python:}
\begin{lstlisting}[language=Python]
def getSkyline(buildings: List[List[int]]) -> List[List[int]]:
\end{lstlisting}

\section*{Examples}

\textbf{Example 1:}

\begin{verbatim}
Input: buildings = [[2,9,10], [3,7,15], [5,12,12], [15,20,10], [19,24,8]]
Output: [[2,10], [3,15], [7,12], [12,0], [15,10], [20,8], [24,0]]
Explanation:
- At x=2, the first building starts, height=10.
- At x=3, the second building starts, height=15.
- At x=7, the second building ends, the third building is still ongoing, height=12.
- At x=12, the third building ends, height drops to 0.
- At x=15, the fourth building starts, height=10.
- At x=20, the fourth building ends, the fifth building is still ongoing, height=8.
- At x=24, the fifth building ends, height drops to 0.
\end{verbatim}

\textbf{Example 2:}

\begin{verbatim}
Input: buildings = [[0,2,3], [2,5,3]]
Output: [[0,3], [5,0]]
Explanation:
- The two buildings are contiguous and have the same height, so the skyline drops to 0 at x=5.
\end{verbatim}

\textbf{Example 3:}

\begin{verbatim}
Input: buildings = [[1,3,3], [2,4,4], [5,6,1]]
Output: [[1,3], [2,4], [4,0], [5,1], [6,0]]
Explanation:
- At x=1, first building starts, height=3.
- At x=2, second building starts, height=4.
- At x=4, second building ends, height drops to 0.
- At x=5, third building starts, height=1.
- At x=6, third building ends, height drops to 0.
\end{verbatim}

\textbf{Example 4:}

\begin{verbatim}
Input: buildings = [[0,5,0]]
Output: []
Explanation:
- A building with height 0 does not contribute to the skyline.
\end{verbatim}

\textbf{Constraints:}

\begin{itemize}
    \item \(1 \leq \text{buildings.length} \leq 10^4\)
    \item \(0 \leq Li < Ri \leq 10^9\)
    \item \(0 \leq Hi \leq 10^4\)
\end{itemize}

\section*{Algorithmic Approach}

The \textbf{Sweep Line Algorithm} is an efficient method for solving the Skyline Problem. It involves processing events (building start and end points) in sorted order while maintaining a data structure (typically a max heap) to keep track of active buildings. Here's a step-by-step approach:

\subsection*{1. Event Representation}

Transform each building into two events:
\begin{itemize}
    \item **Start Event:** \((Li, -Hi)\) – Negative height indicates a building starts.
    \item **End Event:** \((Ri, Hi)\) – Positive height indicates a building ends.
\end{itemize}

Sorting the events ensures that start events are processed before end events at the same x-coordinate, and taller buildings are processed before shorter ones.

\subsection*{2. Sorting the Events}

Sort all events based on:
\begin{enumerate}
    \item **x-coordinate:** Ascending order.
    \item **Height:**
    \begin{itemize}
        \item For start events, taller buildings come first.
        \item For end events, shorter buildings come first.
    \end{itemize}
\end{enumerate}

\subsection*{3. Processing the Events}

Use a max heap to keep track of active building heights. Iterate through the sorted events:
\begin{enumerate}
    \item **Start Event:**
    \begin{itemize}
        \item Add the building's height to the heap.
    \end{itemize}
    
    \item **End Event:**
    \begin{itemize}
        \item Remove the building's height from the heap.
    \end{itemize}
    
    \item **Determine Current Max Height:**
    \begin{itemize}
        \item The current max height is the top of the heap.
    \end{itemize}
    
    \item **Update Skyline:**
    \begin{itemize}
        \item If the current max height differs from the previous max height, add a new key point \([x, current\_max\_height]\).
    \end{itemize}
\end{enumerate}

\subsection*{4. Finalizing the Skyline}

After processing all events, the accumulated key points represent the skyline.

\marginnote{The Sweep Line Algorithm efficiently handles dynamic changes in active buildings, ensuring accurate skyline construction.}

\section*{Complexities}

\begin{itemize}
    \item \textbf{Time Complexity:} \(O(n \log n)\), where \(n\) is the number of buildings. Sorting the events takes \(O(n \log n)\), and each heap operation takes \(O(\log n)\).
    
    \item \textbf{Space Complexity:} \(O(n)\), due to the storage of events and the heap.
\end{itemize}

\section*{Python Implementation}

\marginnote{Implementing the Sweep Line Algorithm with a max heap ensures an efficient and accurate solution.}

Below is the complete Python code implementing the \texttt{getSkyline} function:

\begin{fullwidth}
\begin{lstlisting}[language=Python]
from typing import List
import heapq

class Solution:
    def getSkyline(self, buildings: List[List[int]]) -> List[List[int]]:
        # Create a list of all events
        # For start events, use negative height to ensure they are processed before end events
        events = []
        for L, R, H in buildings:
            events.append((L, -H))
            events.append((R, H))
        
        # Sort the events
        # First by x-coordinate, then by height
        events.sort()
        
        # Max heap to keep track of active buildings
        heap = [0]  # Initialize with ground level
        heapq.heapify(heap)
        active_heights = {0: 1}  # Dictionary to count heights
        
        result = []
        prev_max = 0
        
        for x, h in events:
            if h < 0:
                # Start of a building, add height to heap and dictionary
                heapq.heappush(heap, h)
                active_heights[h] = active_heights.get(h, 0) + 1
            else:
                # End of a building, remove height from dictionary
                active_heights[h] -= 1
                if active_heights[h] == 0:
                    del active_heights[h]
            
            # Current max height
            while heap and active_heights.get(heap[0], 0) == 0:
                heapq.heappop(heap)
            current_max = -heap[0] if heap else 0
            
            # If the max height has changed, add to result
            if current_max != prev_max:
                result.append([x, current_max])
                prev_max = current_max
        
        return result

# Example usage:
solution = Solution()
print(solution.getSkyline([[2,9,10], [3,7,15], [5,12,12], [15,20,10], [19,24,8]]))
# Output: [[2,10], [3,15], [7,12], [12,0], [15,10], [20,8], [24,0]]

print(solution.getSkyline([[0,2,3], [2,5,3]]))
# Output: [[0,3], [5,0]]

print(solution.getSkyline([[1,3,3], [2,4,4], [5,6,1]]))
# Output: [[1,3], [2,4], [4,0], [5,1], [6,0]]

print(solution.getSkyline([[0,5,0]]))
# Output: []
\end{lstlisting}
\end{fullwidth}

This implementation efficiently constructs the skyline by processing all building events in sorted order and maintaining active building heights using a max heap. It ensures that only critical points where the skyline changes are recorded.

\section*{Explanation}

The \texttt{getSkyline} function constructs the skyline formed by a set of buildings by leveraging the Sweep Line Algorithm and a max heap to track active buildings. Here's a detailed breakdown of the implementation:

\subsection*{1. Event Representation}

\begin{itemize}
    \item Each building is transformed into two events:
    \begin{itemize}
        \item **Start Event:** \((Li, -Hi)\) – Negative height indicates the start of a building.
        \item **End Event:** \((Ri, Hi)\) – Positive height indicates the end of a building.
    \end{itemize}
\end{itemize}

\subsection*{2. Sorting the Events}

\begin{itemize}
    \item Events are sorted primarily by their x-coordinate in ascending order.
    \item For events with the same x-coordinate:
    \begin{itemize}
        \item Start events (with negative heights) are processed before end events.
        \item Taller buildings are processed before shorter ones.
    \end{itemize}
\end{itemize}

\subsection*{3. Processing the Events}

\begin{itemize}
    \item **Heap Initialization:**
    \begin{itemize}
        \item A max heap is initialized with a ground level height of 0.
        \item A dictionary \texttt{active\_heights} tracks the count of active building heights.
    \end{itemize}
    
    \item **Iterating Through Events:**
    \begin{enumerate}
        \item **Start Event:**
        \begin{itemize}
            \item Add the building's height to the heap.
            \item Increment the count of the height in \texttt{active\_heights}.
        \end{itemize}
        
        \item **End Event:**
        \begin{itemize}
            \item Decrement the count of the building's height in \texttt{active\_heights}.
            \item If the count reaches zero, remove the height from the dictionary.
        \end{itemize}
        
        \item **Determine Current Max Height:**
        \begin{itemize}
            \item Remove heights from the heap that are no longer active.
            \item The current max height is the top of the heap.
        \end{itemize}
        
        \item **Update Skyline:**
        \begin{itemize}
            \item If the current max height differs from the previous max height, add a new key point \([x, current\_max\_height]\).
        \end{itemize}
    \end{enumerate}
\end{itemize}

\subsection*{4. Finalizing the Skyline}

\begin{itemize}
    \item After processing all events, the \texttt{result} list contains the key points defining the skyline.
\end{itemize}

\subsection*{5. Example Walkthrough}

Consider the first example:
\begin{verbatim}
Input: buildings = [[2,9,10], [3,7,15], [5,12,12], [15,20,10], [19,24,8]]
Output: [[2,10], [3,15], [7,12], [12,0], [15,10], [20,8], [24,0]]
\end{verbatim}

\begin{enumerate}
    \item **Event Transformation:**
    \begin{itemize}
        \item \((2, -10)\), \((9, 10)\)
        \item \((3, -15)\), \((7, 15)\)
        \item \((5, -12)\), \((12, 12)\)
        \item \((15, -10)\), \((20, 10)\)
        \item \((19, -8)\), \((24, 8)\)
    \end{itemize}
    
    \item **Sorting Events:**
    \begin{itemize}
        \item Sorted order: \((2, -10)\), \((3, -15)\), \((5, -12)\), \((7, 15)\), \((9, 10)\), \((12, 12)\), \((15, -10)\), \((19, -8)\), \((20, 10)\), \((24, 8)\)
    \end{itemize}
    
    \item **Processing Events:**
    \begin{itemize}
        \item At each event, update the heap and determine if the skyline height changes.
    \end{itemize}
    
    \item **Result Construction:**
    \begin{itemize}
        \item The resulting skyline key points are accumulated as \([[2,10], [3,15], [7,12], [12,0], [15,10], [20,8], [24,0]]\).
    \end{itemize}
\end{enumerate}

Thus, the function correctly constructs the skyline based on the buildings' positions and heights.

\section*{Why This Approach}

The Sweep Line Algorithm combined with a max heap offers an optimal solution with \(O(n \log n)\) time complexity and efficient handling of overlapping buildings. By processing events in sorted order and maintaining active building heights, the algorithm ensures that all critical points in the skyline are accurately identified without redundant computations.

\section*{Alternative Approaches}

\subsection*{1. Divide and Conquer}

Divide the set of buildings into smaller subsets, compute the skyline for each subset, and then merge the skylines.

\begin{lstlisting}[language=Python]
class Solution:
    def getSkyline(self, buildings: List[List[int]]) -> List[List[int]]:
        def merge(left, right):
            h1, h2 = 0, 0
            i, j = 0, 0
            merged = []
            while i < len(left) and j < len(right):
                if left[i][0] < right[j][0]:
                    x, h1 = left[i]
                    i += 1
                elif left[i][0] > right[j][0]:
                    x, h2 = right[j]
                    j += 1
                else:
                    x, h1 = left[i]
                    _, h2 = right[j]
                    i += 1
                    j += 1
                max_h = max(h1, h2)
                if not merged or merged[-1][1] != max_h:
                    merged.append([x, max_h])
            merged.extend(left[i:])
            merged.extend(right[j:])
            return merged
        
        def divide(buildings):
            if not buildings:
                return []
            if len(buildings) == 1:
                L, R, H = buildings[0]
                return [[L, H], [R, 0]]
            mid = len(buildings) // 2
            left = divide(buildings[:mid])
            right = divide(buildings[mid:])
            return merge(left, right)
        
        return divide(buildings)
\end{lstlisting}

\textbf{Complexities:}
\begin{itemize}
    \item \textbf{Time Complexity:} \(O(n \log n)\)
    \item \textbf{Space Complexity:} \(O(n)\)
\end{itemize}

\subsection*{2. Using Segment Trees}

Implement a segment tree to manage and query overlapping building heights dynamically.

\textbf{Note}: This approach is more complex and is generally used for advanced scenarios with multiple dynamic queries.

\section*{Similar Problems to This One}

Several problems involve skyline-like constructions, spatial data analysis, and efficient event processing, utilizing similar algorithmic strategies:

\begin{itemize}
    \item \textbf{Merge Intervals}: Merge overlapping intervals in a list.
    \item \textbf{Largest Rectangle in Histogram}: Find the largest rectangular area in a histogram.
    \item \textbf{Interval Partitioning}: Assign intervals to resources without overlap.
    \item \textbf{Line Segment Intersection}: Detect intersections among line segments.
    \item \textbf{Closest Pair of Points}: Find the closest pair of points in a set.
    \item \textbf{Convex Hull}: Compute the convex hull of a set of points.
    \item \textbf{Point Inside Polygon}: Determine if a point lies inside a given polygon.
    \item \textbf{Range Searching}: Efficiently query geometric data within a specified range.
\end{itemize}

These problems reinforce concepts of event-driven processing, spatial reasoning, and efficient algorithm design in various contexts.

\section*{Things to Keep in Mind and Tricks}

When tackling the \textbf{Skyline Problem}, consider the following tips and best practices to enhance efficiency and correctness:

\begin{itemize}
    \item \textbf{Understand Sweep Line Technique}: Grasp how the sweep line algorithm processes events in sorted order to handle dynamic changes efficiently.
    \index{Sweep Line Technique}
    
    \item \textbf{Leverage Priority Queues (Heaps)}: Use max heaps to keep track of active buildings' heights, enabling quick access to the current maximum height.
    \index{Priority Queues}
    
    \item \textbf{Handle Start and End Events Differently}: Differentiate between building start and end events to accurately manage active heights.
    \index{Start and End Events}
    
    \item \textbf{Optimize Event Sorting}: Sort events primarily by x-coordinate and secondarily by height to ensure correct processing order.
    \index{Event Sorting}
    
    \item \textbf{Manage Active Heights Efficiently}: Use data structures that allow efficient insertion, deletion, and retrieval of maximum elements.
    \index{Active Heights Management}
    
    \item \textbf{Avoid Redundant Key Points}: Only record key points when the skyline height changes to minimize the output list.
    \index{Avoiding Redundant Key Points}
    
    \item \textbf{Implement Helper Functions}: Create helper functions for tasks like distance calculation, event handling, and heap management to enhance modularity.
    \index{Helper Functions}
    
    \item \textbf{Code Readability}: Maintain clear and readable code through meaningful variable names and structured logic.
    \index{Code Readability}
    
    \item \textbf{Test Extensively}: Implement a wide range of test cases, including overlapping, non-overlapping, and edge-touching buildings, to ensure robustness.
    \index{Extensive Testing}
    
    \item \textbf{Handle Degenerate Cases}: Manage cases where buildings have zero height or identical coordinates gracefully.
    \index{Degenerate Cases}
    
    \item \textbf{Understand Geometric Relationships}: Grasp how buildings overlap and influence the skyline to simplify the algorithm.
    \index{Geometric Relationships}
    
    \item \textbf{Use Appropriate Data Structures}: Utilize appropriate data structures like heaps, lists, and dictionaries to manage and process data efficiently.
    \index{Appropriate Data Structures}
    
    \item \textbf{Optimize for Large Inputs}: Design the algorithm to handle large numbers of buildings without significant performance degradation.
    \index{Optimizing for Large Inputs}
    
    \item \textbf{Implement Iterative Solutions Carefully}: Ensure that loop conditions are correctly defined to prevent infinite loops or incorrect terminations.
    \index{Iterative Solutions}
    
    \item \textbf{Consistent Naming Conventions}: Use consistent and descriptive naming conventions for variables and functions to improve code clarity.
    \index{Naming Conventions}
\end{itemize}

\section*{Corner and Special Cases to Test When Writing the Code}

When implementing the solution for the \textbf{Skyline Problem}, it is crucial to consider and rigorously test various edge cases to ensure robustness and correctness:

\begin{itemize}
    \item \textbf{No Overlapping Buildings}: All buildings are separate and do not overlap.
    \index{No Overlapping Buildings}
    
    \item \textbf{Fully Overlapping Buildings}: Multiple buildings completely overlap each other.
    \index{Fully Overlapping Buildings}
    
    \item \textbf{Buildings Touching at Edges}: Buildings share common edges without overlapping.
    \index{Buildings Touching at Edges}
    
    \item \textbf{Buildings Touching at Corners}: Buildings share common corners without overlapping.
    \index{Buildings Touching at Corners}
    
    \item \textbf{Single Building}: Only one building is present.
    \index{Single Building}
    
    \item \textbf{Multiple Buildings with Same Start or End}: Multiple buildings start or end at the same x-coordinate.
    \index{Same Start or End}
    
    \item \textbf{Buildings with Zero Height}: Buildings that have zero height should not affect the skyline.
    \index{Buildings with Zero Height}
    
    \item \textbf{Large Number of Buildings}: Test with a large number of buildings to ensure performance and scalability.
    \index{Large Number of Buildings}
    
    \item \textbf{Buildings with Negative Coordinates}: Buildings positioned in negative coordinate space.
    \index{Negative Coordinates}
    
    \item \textbf{Boundary Values}: Buildings at the minimum and maximum limits of the coordinate range.
    \index{Boundary Values}
    
    \item \textbf{Buildings with Identical Coordinates}: Multiple buildings with the same coordinates.
    \index{Identical Coordinates}
    
    \item \textbf{Sequential Buildings}: Buildings placed sequentially without gaps.
    \index{Sequential Buildings}
    
    \item \textbf{Overlapping and Non-Overlapping Mixed}: A mix of overlapping and non-overlapping buildings.
    \index{Overlapping and Non-Overlapping Mixed}
    
    \item \textbf{Buildings with Very Large Heights}: Buildings with heights at the upper limit of the constraints.
    \index{Very Large Heights}
    
    \item \textbf{Empty Input}: No buildings are provided.
    \index{Empty Input}
\end{itemize}

\section*{Implementation Considerations}

When implementing the \texttt{getSkyline} function, keep in mind the following considerations to ensure robustness and efficiency:

\begin{itemize}
    \item \textbf{Data Type Selection}: Use appropriate data types that can handle large input values and avoid overflow or precision issues.
    \index{Data Type Selection}
    
    \item \textbf{Optimizing Event Sorting}: Efficiently sort events based on x-coordinates and heights to ensure correct processing order.
    \index{Optimizing Event Sorting}
    
    \item \textbf{Handling Large Inputs}: Design the algorithm to handle up to \(10^4\) buildings efficiently without significant performance degradation.
    \index{Handling Large Inputs}
    
    \item \textbf{Using Efficient Data Structures}: Utilize heaps, lists, and dictionaries effectively to manage and process events and active heights.
    \index{Efficient Data Structures}
    
    \item \textbf{Avoiding Redundant Calculations}: Ensure that distance and overlap calculations are performed only when necessary to optimize performance.
    \index{Avoiding Redundant Calculations}
    
    \item \textbf{Code Readability and Documentation}: Maintain clear and readable code through meaningful variable names and comprehensive comments to facilitate understanding and maintenance.
    \index{Code Readability}
    
    \item \textbf{Edge Case Handling}: Implement checks for edge cases to prevent incorrect results or runtime errors.
    \index{Edge Case Handling}
    
    \item \textbf{Implementing Helper Functions}: Create helper functions for tasks like distance calculation, event handling, and heap management to enhance modularity.
    \index{Helper Functions}
    
    \item \textbf{Consistent Naming Conventions}: Use consistent and descriptive naming conventions for variables and functions to improve code clarity.
    \index{Naming Conventions}
    
    \item \textbf{Memory Management}: Ensure that the algorithm manages memory efficiently, especially when dealing with large datasets.
    \index{Memory Management}
    
    \item \textbf{Implementing Iterative Solutions Carefully}: Ensure that loop conditions are correctly defined to prevent infinite loops or incorrect terminations.
    \index{Iterative Solutions}
    
    \item \textbf{Avoiding Floating-Point Precision Issues}: Since the problem deals with integers, floating-point precision is not a concern, simplifying the implementation.
    \index{Floating-Point Precision}
    
    \item \textbf{Testing and Validation}: Develop a comprehensive suite of test cases that cover all possible scenarios, including edge cases, to validate the correctness and efficiency of the implementation.
    \index{Testing and Validation}
    
    \item \textbf{Performance Considerations}: Optimize the loop conditions and operations to ensure that the function runs efficiently, especially for large input numbers.
    \index{Performance Considerations}
\end{itemize}

\section*{Conclusion}

The \textbf{Skyline Problem} is a quintessential example of applying advanced algorithmic techniques and geometric reasoning to solve complex spatial challenges. By leveraging the Sweep Line Algorithm and maintaining active building heights using a max heap, the solution efficiently constructs the skyline with optimal time and space complexities. Understanding and implementing such sophisticated algorithms not only enhances problem-solving skills but also provides a foundation for tackling a wide array of Computational Geometry problems in various domains, including computer graphics, urban planning simulations, and geographic information systems.

\printindex

% % filename: rectangle_overlap.tex

\problemsection{Rectangle Overlap}
\label{chap:Rectangle_Overlap}
\marginnote{\href{https://leetcode.com/problems/rectangle-overlap/}{[LeetCode Link]}\index{LeetCode}}
\marginnote{\href{https://www.geeksforgeeks.org/check-if-two-rectangles-overlap/}{[GeeksForGeeks Link]}\index{GeeksForGeeks}}
\marginnote{\href{https://www.interviewbit.com/problems/rectangle-overlap/}{[InterviewBit Link]}\index{InterviewBit}}
\marginnote{\href{https://app.codesignal.com/challenges/rectangle-overlap}{[CodeSignal Link]}\index{CodeSignal}}
\marginnote{\href{https://www.codewars.com/kata/rectangle-overlap/train/python}{[Codewars Link]}\index{Codewars}}

The \textbf{Rectangle Overlap} problem is a fundamental challenge in Computational Geometry that involves determining whether two axis-aligned rectangles overlap. This problem tests one's ability to understand geometric properties, implement conditional logic, and optimize for efficient computation. Mastery of this problem is essential for applications in computer graphics, collision detection, and spatial data analysis.

\section*{Problem Statement}

Given two axis-aligned rectangles in a 2D plane, determine if they overlap. Each rectangle is defined by its bottom-left and top-right coordinates.

A rectangle is represented as a list of four integers \([x1, y1, x2, y2]\), where \((x1, y1)\) are the coordinates of the bottom-left corner, and \((x2, y2)\) are the coordinates of the top-right corner.

\textbf{Function signature in Python:}
\begin{lstlisting}[language=Python]
def isRectangleOverlap(rec1: List[int], rec2: List[int]) -> bool:
\end{lstlisting}

\section*{Examples}

\textbf{Example 1:}

\begin{verbatim}
Input: rec1 = [0,0,2,2], rec2 = [1,1,3,3]
Output: True
Explanation: The rectangles overlap in the area defined by [1,1,2,2].
\end{verbatim}

\textbf{Example 2:}

\begin{verbatim}
Input: rec1 = [0,0,1,1], rec2 = [1,0,2,1]
Output: False
Explanation: The rectangles touch at the edge but do not overlap.
\end{verbatim}

\textbf{Example 3:}

\begin{verbatim}
Input: rec1 = [0,0,1,1], rec2 = [2,2,3,3]
Output: False
Explanation: The rectangles are completely separate.
\end{verbatim}

\textbf{Example 4:}

\begin{verbatim}
Input: rec1 = [0,0,5,5], rec2 = [3,3,7,7]
Output: True
Explanation: The rectangles overlap in the area defined by [3,3,5,5].
\end{verbatim}

\textbf{Example 5:}

\begin{verbatim}
Input: rec1 = [0,0,0,0], rec2 = [0,0,0,0]
Output: False
Explanation: Both rectangles are degenerate points.
\end{verbatim}

\textbf{Constraints:}

\begin{itemize}
    \item All coordinates are integers in the range \([-10^9, 10^9]\).
    \item For each rectangle, \(x1 < x2\) and \(y1 < y2\).
\end{itemize}

LeetCode link: \href{https://leetcode.com/problems/rectangle-overlap/}{Rectangle Overlap}\index{LeetCode}

\section*{Algorithmic Approach}

To determine whether two axis-aligned rectangles overlap, we can use the following logical conditions:

1. **Non-Overlap Conditions:**
   - One rectangle is to the left of the other.
   - One rectangle is above the other.

2. **Overlap Condition:**
   - If neither of the non-overlap conditions is true, the rectangles must overlap.

\subsection*{Steps:}

1. **Extract Coordinates:**
   - For both rectangles, extract the bottom-left and top-right coordinates.

2. **Check Non-Overlap Conditions:**
   - If the right side of the first rectangle is less than or equal to the left side of the second rectangle, they do not overlap.
   - If the left side of the first rectangle is greater than or equal to the right side of the second rectangle, they do not overlap.
   - If the top side of the first rectangle is less than or equal to the bottom side of the second rectangle, they do not overlap.
   - If the bottom side of the first rectangle is greater than or equal to the top side of the second rectangle, they do not overlap.

3. **Determine Overlap:**
   - If none of the non-overlap conditions are met, the rectangles overlap.

\marginnote{This approach provides an efficient \(O(1)\) time complexity solution by leveraging simple geometric comparisons.}

\section*{Complexities}

\begin{itemize}
    \item \textbf{Time Complexity:} \(O(1)\). The algorithm performs a constant number of comparisons regardless of input size.
    
    \item \textbf{Space Complexity:} \(O(1)\). Only a fixed amount of extra space is used for variables.
\end{itemize}

\section*{Python Implementation}

\marginnote{Implementing the overlap check using coordinate comparisons ensures an optimal and straightforward solution.}

Below is the complete Python code implementing the \texttt{isRectangleOverlap} function:

\begin{fullwidth}
\begin{lstlisting}[language=Python]
from typing import List

class Solution:
    def isRectangleOverlap(self, rec1: List[int], rec2: List[int]) -> bool:
        # Extract coordinates
        left1, bottom1, right1, top1 = rec1
        left2, bottom2, right2, top2 = rec2
        
        # Check non-overlapping conditions
        if right1 <= left2 or right2 <= left1:
            return False
        if top1 <= bottom2 or top2 <= bottom1:
            return False
        
        # If none of the above, rectangles overlap
        return True

# Example usage:
solution = Solution()
print(solution.isRectangleOverlap([0,0,2,2], [1,1,3,3]))  # Output: True
print(solution.isRectangleOverlap([0,0,1,1], [1,0,2,1]))  # Output: False
print(solution.isRectangleOverlap([0,0,1,1], [2,2,3,3]))  # Output: False
print(solution.isRectangleOverlap([0,0,5,5], [3,3,7,7]))  # Output: True
print(solution.isRectangleOverlap([0,0,0,0], [0,0,0,0]))  # Output: False
\end{lstlisting}
\end{fullwidth}

This implementation efficiently checks for overlap by comparing the coordinates of the two rectangles. If any of the non-overlapping conditions are met, it returns \texttt{False}; otherwise, it returns \texttt{True}.

\section*{Explanation}

The \texttt{isRectangleOverlap} function determines whether two axis-aligned rectangles overlap by comparing their respective coordinates. Here's a detailed breakdown of the implementation:

\subsection*{1. Extract Coordinates}

\begin{itemize}
    \item For each rectangle, extract the left (\(x1\)), bottom (\(y1\)), right (\(x2\)), and top (\(y2\)) coordinates.
    \item This simplifies the comparison process by providing clear variables representing each side of the rectangles.
\end{itemize}

\subsection*{2. Check Non-Overlap Conditions}

\begin{itemize}
    \item **Horizontal Separation:**
    \begin{itemize}
        \item If the right side of the first rectangle (\(right1\)) is less than or equal to the left side of the second rectangle (\(left2\)), there is no horizontal overlap.
        \item Similarly, if the right side of the second rectangle (\(right2\)) is less than or equal to the left side of the first rectangle (\(left1\)), there is no horizontal overlap.
    \end{itemize}
    
    \item **Vertical Separation:**
    \begin{itemize}
        \item If the top side of the first rectangle (\(top1\)) is less than or equal to the bottom side of the second rectangle (\(bottom2\)), there is no vertical overlap.
        \item Similarly, if the top side of the second rectangle (\(top2\)) is less than or equal to the bottom side of the first rectangle (\(bottom1\)), there is no vertical overlap.
    \end{itemize}
    
    \item If any of these non-overlapping conditions are true, the rectangles do not overlap, and the function returns \texttt{False}.
\end{itemize}

\subsection*{3. Determine Overlap}

\begin{itemize}
    \item If none of the non-overlapping conditions are met, it implies that the rectangles overlap both horizontally and vertically.
    \item The function returns \texttt{True} in this case.
\end{itemize}

\subsection*{4. Example Walkthrough}

Consider the first example:
\begin{verbatim}
Input: rec1 = [0,0,2,2], rec2 = [1,1,3,3]
Output: True
\end{verbatim}

\begin{enumerate}
    \item Extract coordinates:
    \begin{itemize}
        \item rec1: left1 = 0, bottom1 = 0, right1 = 2, top1 = 2
        \item rec2: left2 = 1, bottom2 = 1, right2 = 3, top2 = 3
    \end{itemize}
    
    \item Check non-overlap conditions:
    \begin{itemize}
        \item \(right1 = 2\) is not less than or equal to \(left2 = 1\)
        \item \(right2 = 3\) is not less than or equal to \(left1 = 0\)
        \item \(top1 = 2\) is not less than or equal to \(bottom2 = 1\)
        \item \(top2 = 3\) is not less than or equal to \(bottom1 = 0\)
    \end{itemize}
    
    \item Since none of the non-overlapping conditions are met, the rectangles overlap.
\end{enumerate}

Thus, the function correctly returns \texttt{True}.

\section*{Why This Approach}

This approach is chosen for its simplicity and efficiency. By leveraging direct coordinate comparisons, the algorithm achieves constant time complexity without the need for complex data structures or iterative processes. It effectively handles all possible scenarios of rectangle positioning, ensuring accurate detection of overlaps.

\section*{Alternative Approaches}

\subsection*{1. Separating Axis Theorem (SAT)}

The Separating Axis Theorem is a more generalized method for detecting overlaps between convex shapes. While it is not necessary for axis-aligned rectangles, understanding SAT can be beneficial for more complex geometric problems.

\begin{lstlisting}[language=Python]
def isRectangleOverlap(rec1: List[int], rec2: List[int]) -> bool:
    # Using SAT for axis-aligned rectangles
    return not (rec1[2] <= rec2[0] or rec1[0] >= rec2[2] or
                rec1[3] <= rec2[1] or rec1[1] >= rec2[3])
\end{lstlisting}

\textbf{Note}: This implementation is functionally identical to the primary approach but leverages a more generalized geometric theorem.

\subsection*{2. Area-Based Approach}

Calculate the overlapping area between the two rectangles. If the overlapping area is positive, the rectangles overlap.

\begin{lstlisting}[language=Python]
def isRectangleOverlap(rec1: List[int], rec2: List[int]) -> bool:
    # Calculate overlap in x and y dimensions
    x_overlap = min(rec1[2], rec2[2]) - max(rec1[0], rec2[0])
    y_overlap = min(rec1[3], rec2[3]) - max(rec1[1], rec2[1])
    
    # Overlap exists if both overlaps are positive
    return x_overlap > 0 and y_overlap > 0
\end{lstlisting}

\textbf{Complexities:}
\begin{itemize}
    \item \textbf{Time Complexity:} \(O(1)\)
    \item \textbf{Space Complexity:} \(O(1)\)
\end{itemize}

\subsection*{3. Using Rectangles Intersection Function}

Utilize built-in or library functions that handle geometric intersections.

\begin{lstlisting}[language=Python]
from shapely.geometry import box

def isRectangleOverlap(rec1: List[int], rec2: List[int]) -> bool:
    rectangle1 = box(rec1[0], rec1[1], rec1[2], rec1[3])
    rectangle2 = box(rec2[0], rec2[1], rec2[2], rec2[3])
    return rectangle1.intersects(rectangle2) and not rectangle1.touches(rectangle2)
\end{lstlisting}

\textbf{Note}: This approach requires the \texttt{shapely} library and is more suitable for complex geometric operations.

\section*{Similar Problems to This One}

Several problems revolve around geometric overlap, intersection detection, and spatial reasoning, utilizing similar algorithmic strategies:

\begin{itemize}
    \item \textbf{Interval Overlap}: Determine if two intervals on a line overlap.
    \item \textbf{Circle Overlap}: Determine if two circles overlap based on their radii and centers.
    \item \textbf{Polygon Overlap}: Determine if two polygons overlap using algorithms like SAT.
    \item \textbf{Closest Pair of Points}: Find the closest pair of points in a set.
    \item \textbf{Convex Hull}: Compute the convex hull of a set of points.
    \item \textbf{Intersection of Lines}: Find the intersection point of two lines.
    \item \textbf{Point Inside Polygon}: Determine if a point lies inside a given polygon.
\end{itemize}

These problems reinforce the concepts of spatial reasoning, geometric property analysis, and efficient algorithm design in various contexts.

\section*{Things to Keep in Mind and Tricks}

When working with the \textbf{Rectangle Overlap} problem, consider the following tips and best practices to enhance efficiency and correctness:

\begin{itemize}
    \item \textbf{Understand Geometric Relationships}: Grasp the positional relationships between rectangles to simplify overlap detection.
    \index{Geometric Relationships}
    
    \item \textbf{Leverage Coordinate Comparisons}: Use direct comparisons of rectangle coordinates to determine spatial relationships.
    \index{Coordinate Comparisons}
    
    \item \textbf{Handle Edge Cases}: Consider cases where rectangles touch at edges or corners without overlapping.
    \index{Edge Cases}
    
    \item \textbf{Optimize for Efficiency}: Aim for a constant time \(O(1)\) solution by avoiding unnecessary computations or iterations.
    \index{Efficiency Optimization}
    
    \item \textbf{Avoid Floating-Point Precision Issues}: Since all coordinates are integers, floating-point precision is not a concern, simplifying the implementation.
    \index{Floating-Point Precision}
    
    \item \textbf{Use Helper Functions}: Create helper functions to encapsulate repetitive tasks, such as extracting coordinates or checking specific conditions.
    \index{Helper Functions}
    
    \item \textbf{Code Readability}: Maintain clear and readable code through meaningful variable names and structured logic.
    \index{Code Readability}
    
    \item \textbf{Test Extensively}: Implement a wide range of test cases, including overlapping, non-overlapping, and edge-touching rectangles, to ensure robustness.
    \index{Extensive Testing}
    
    \item \textbf{Understand Axis-Aligned Constraints}: Recognize that axis-aligned rectangles simplify overlap detection compared to rotated rectangles.
    \index{Axis-Aligned Constraints}
    
    \item \textbf{Simplify Logical Conditions}: Combine multiple conditions logically to streamline the overlap detection process.
    \index{Logical Conditions}
\end{itemize}

\section*{Corner and Special Cases to Test When Writing the Code}

When implementing the solution for the \textbf{Rectangle Overlap} problem, it is crucial to consider and rigorously test various edge cases to ensure robustness and correctness:

\begin{itemize}
    \item \textbf{No Overlap}: Rectangles are completely separate.
    \index{No Overlap}
    
    \item \textbf{Partial Overlap}: Rectangles overlap in one or more regions.
    \index{Partial Overlap}
    
    \item \textbf{Edge Touching}: Rectangles touch exactly at one edge without overlapping.
    \index{Edge Touching}
    
    \item \textbf{Corner Touching}: Rectangles touch exactly at one corner without overlapping.
    \index{Corner Touching}
    
    \item \textbf{One Rectangle Inside Another}: One rectangle is entirely within the other.
    \index{Rectangle Inside}
    
    \item \textbf{Identical Rectangles}: Both rectangles have the same coordinates.
    \index{Identical Rectangles}
    
    \item \textbf{Degenerate Rectangles}: Rectangles with zero area (e.g., \(x1 = x2\) or \(y1 = y2\)).
    \index{Degenerate Rectangles}
    
    \item \textbf{Large Coordinates}: Rectangles with very large coordinate values to test performance and integer handling.
    \index{Large Coordinates}
    
    \item \textbf{Negative Coordinates}: Rectangles positioned in negative coordinate space.
    \index{Negative Coordinates}
    
    \item \textbf{Mixed Overlapping Scenarios}: Combinations of the above cases to ensure comprehensive coverage.
    \index{Mixed Overlapping Scenarios}
    
    \item \textbf{Minimum and Maximum Bounds}: Rectangles at the minimum and maximum limits of the coordinate range.
    \index{Minimum and Maximum Bounds}
\end{itemize}

\section*{Implementation Considerations}

When implementing the \texttt{isRectangleOverlap} function, keep in mind the following considerations to ensure robustness and efficiency:

\begin{itemize}
    \item \textbf{Data Type Selection}: Use appropriate data types that can handle the range of input values without overflow or underflow.
    \index{Data Type Selection}
    
    \item \textbf{Optimizing Comparisons}: Structure logical conditions to short-circuit evaluations as soon as a non-overlapping condition is met.
    \index{Optimizing Comparisons}
    
    \item \textbf{Language-Specific Constraints}: Be aware of how the programming language handles integer division and comparisons.
    \index{Language-Specific Constraints}
    
    \item \textbf{Avoiding Redundant Calculations}: Ensure that each comparison contributes towards determining overlap without unnecessary repetitions.
    \index{Avoiding Redundant Calculations}
    
    \item \textbf{Code Readability and Documentation}: Maintain clear and readable code through meaningful variable names and comprehensive comments to facilitate understanding and maintenance.
    \index{Code Readability}
    
    \item \textbf{Edge Case Handling}: Implement checks for edge cases to prevent incorrect results or runtime errors.
    \index{Edge Case Handling}
    
    \item \textbf{Testing and Validation}: Develop a comprehensive suite of test cases that cover all possible scenarios, including edge cases, to validate the correctness and efficiency of the implementation.
    \index{Testing and Validation}
    
    \item \textbf{Scalability}: Design the algorithm to scale efficiently with increasing input sizes, maintaining performance and resource utilization.
    \index{Scalability}
    
    \item \textbf{Using Helper Functions}: Consider creating helper functions for repetitive tasks, such as extracting and comparing coordinates, to enhance modularity and reusability.
    \index{Helper Functions}
    
    \item \textbf{Consistent Naming Conventions}: Use consistent and descriptive naming conventions for variables to improve code clarity.
    \index{Naming Conventions}
    
    \item \textbf{Handling Floating-Point Coordinates}: Although the problem specifies integer coordinates, ensure that the implementation can handle floating-point numbers if needed in extended scenarios.
    \index{Floating-Point Coordinates}
    
    \item \textbf{Avoiding Floating-Point Precision Issues}: Since all coordinates are integers, floating-point precision is not a concern, simplifying the implementation.
    \index{Floating-Point Precision}
    
    \item \textbf{Implementing Unit Tests}: Develop unit tests for each logical condition to ensure that all scenarios are correctly handled.
    \index{Unit Tests}
    
    \item \textbf{Error Handling}: Incorporate error handling to manage invalid inputs gracefully.
    \index{Error Handling}
\end{itemize}

\section*{Conclusion}

The \textbf{Rectangle Overlap} problem exemplifies the application of fundamental geometric principles and conditional logic to solve spatial challenges efficiently. By leveraging simple coordinate comparisons, the algorithm achieves optimal time and space complexities, making it highly suitable for real-time applications such as collision detection in gaming, layout planning in graphics, and spatial data analysis. Understanding and implementing such techniques not only enhances problem-solving skills but also provides a foundation for tackling more complex Computational Geometry problems involving varied geometric shapes and interactions.

\printindex

% \input{sections/rectangle_overlap}
% \input{sections/rectangle_area}
% \input{sections/k_closest_points_to_origin}
% \input{sections/the_skyline_problem}
% % filename: rectangle_area.tex

\problemsection{Rectangle Area}
\label{chap:Rectangle_Area}
\marginnote{\href{https://leetcode.com/problems/rectangle-area/}{[LeetCode Link]}\index{LeetCode}}
\marginnote{\href{https://www.geeksforgeeks.org/find-area-two-overlapping-rectangles/}{[GeeksForGeeks Link]}\index{GeeksForGeeks}}
\marginnote{\href{https://www.interviewbit.com/problems/rectangle-area/}{[InterviewBit Link]}\index{InterviewBit}}
\marginnote{\href{https://app.codesignal.com/challenges/rectangle-area}{[CodeSignal Link]}\index{CodeSignal}}
\marginnote{\href{https://www.codewars.com/kata/rectangle-area/train/python}{[Codewars Link]}\index{Codewars}}

The \textbf{Rectangle Area} problem is a classic Computational Geometry challenge that involves calculating the total area covered by two axis-aligned rectangles in a 2D plane. This problem tests one's ability to perform geometric calculations, handle overlapping scenarios, and implement efficient algorithms. Mastery of this problem is essential for applications in computer graphics, spatial analysis, and computational modeling.

\section*{Problem Statement}

Given two axis-aligned rectangles in a 2D plane, compute the total area covered by the two rectangles. The area covered by the overlapping region should be counted only once.

Each rectangle is represented as a list of four integers \([x1, y1, x2, y2]\), where \((x1, y1)\) are the coordinates of the bottom-left corner, and \((x2, y2)\) are the coordinates of the top-right corner.

\textbf{Function signature in Python:}
\begin{lstlisting}[language=Python]
def computeArea(A: List[int], B: List[int]) -> int:
\end{lstlisting}

\section*{Examples}

\textbf{Example 1:}

\begin{verbatim}
Input: A = [-3,0,3,4], B = [0,-1,9,2]
Output: 45
Explanation:
Area of A = (3 - (-3)) * (4 - 0) = 6 * 4 = 24
Area of B = (9 - 0) * (2 - (-1)) = 9 * 3 = 27
Overlapping Area = (3 - 0) * (2 - 0) = 3 * 2 = 6
Total Area = 24 + 27 - 6 = 45
\end{verbatim}

\textbf{Example 2:}

\begin{verbatim}
Input: A = [0,0,0,0], B = [0,0,0,0]
Output: 0
Explanation:
Both rectangles are degenerate points with zero area.
\end{verbatim}

\textbf{Example 3:}

\begin{verbatim}
Input: A = [0,0,2,2], B = [1,1,3,3]
Output: 7
Explanation:
Area of A = 4
Area of B = 4
Overlapping Area = 1
Total Area = 4 + 4 - 1 = 7
\end{verbatim}

\textbf{Example 4:}

\begin{verbatim}
Input: A = [0,0,1,1], B = [1,0,2,1]
Output: 2
Explanation:
Rectangles touch at the edge but do not overlap.
Area of A = 1
Area of B = 1
Overlapping Area = 0
Total Area = 1 + 1 = 2
\end{verbatim}

\textbf{Constraints:}

\begin{itemize}
    \item All coordinates are integers in the range \([-10^9, 10^9]\).
    \item For each rectangle, \(x1 < x2\) and \(y1 < y2\).
\end{itemize}

LeetCode link: \href{https://leetcode.com/problems/rectangle-area/}{Rectangle Area}\index{LeetCode}

\section*{Algorithmic Approach}

To compute the total area covered by two axis-aligned rectangles, we can follow these steps:

1. **Calculate Individual Areas:**
   - Compute the area of the first rectangle.
   - Compute the area of the second rectangle.

2. **Determine Overlapping Area:**
   - Calculate the coordinates of the overlapping rectangle, if any.
   - If the rectangles overlap, compute the area of the overlapping region.

3. **Compute Total Area:**
   - Sum the individual areas and subtract the overlapping area to avoid double-counting.

\marginnote{This approach ensures accurate area calculation by handling overlapping regions appropriately.}

\section*{Complexities}

\begin{itemize}
    \item \textbf{Time Complexity:} \(O(1)\). The algorithm performs a constant number of calculations.
    
    \item \textbf{Space Complexity:} \(O(1)\). Only a fixed amount of extra space is used for variables.
\end{itemize}

\section*{Python Implementation}

\marginnote{Implementing the area calculation with overlap consideration ensures an accurate and efficient solution.}

Below is the complete Python code implementing the \texttt{computeArea} function:

\begin{fullwidth}
\begin{lstlisting}[language=Python]
from typing import List

class Solution:
    def computeArea(self, A: List[int], B: List[int]) -> int:
        # Calculate area of rectangle A
        areaA = (A[2] - A[0]) * (A[3] - A[1])
        
        # Calculate area of rectangle B
        areaB = (B[2] - B[0]) * (B[3] - B[1])
        
        # Determine overlap coordinates
        overlap_x1 = max(A[0], B[0])
        overlap_y1 = max(A[1], B[1])
        overlap_x2 = min(A[2], B[2])
        overlap_y2 = min(A[3], B[3])
        
        # Calculate overlapping area
        overlap_width = overlap_x2 - overlap_x1
        overlap_height = overlap_y2 - overlap_y1
        overlap_area = 0
        if overlap_width > 0 and overlap_height > 0:
            overlap_area = overlap_width * overlap_height
        
        # Total area is sum of individual areas minus overlapping area
        total_area = areaA + areaB - overlap_area
        return total_area

# Example usage:
solution = Solution()
print(solution.computeArea([-3,0,3,4], [0,-1,9,2]))  # Output: 45
print(solution.computeArea([0,0,0,0], [0,0,0,0]))    # Output: 0
print(solution.computeArea([0,0,2,2], [1,1,3,3]))    # Output: 7
print(solution.computeArea([0,0,1,1], [1,0,2,1]))    # Output: 2
\end{lstlisting}
\end{fullwidth}

This implementation accurately computes the total area covered by two rectangles by accounting for any overlapping regions. It ensures that the overlapping area is not double-counted.

\section*{Explanation}

The \texttt{computeArea} function calculates the combined area of two axis-aligned rectangles by following these steps:

\subsection*{1. Calculate Individual Areas}

\begin{itemize}
    \item **Rectangle A:**
    \begin{itemize}
        \item Width: \(A[2] - A[0]\)
        \item Height: \(A[3] - A[1]\)
        \item Area: Width \(\times\) Height
    \end{itemize}
    
    \item **Rectangle B:**
    \begin{itemize}
        \item Width: \(B[2] - B[0]\)
        \item Height: \(B[3] - B[1]\)
        \item Area: Width \(\times\) Height
    \end{itemize}
\end{itemize}

\subsection*{2. Determine Overlapping Area}

\begin{itemize}
    \item **Overlap Coordinates:**
    \begin{itemize}
        \item Left (x-coordinate): \(\text{max}(A[0], B[0])\)
        \item Bottom (y-coordinate): \(\text{max}(A[1], B[1])\)
        \item Right (x-coordinate): \(\text{min}(A[2], B[2])\)
        \item Top (y-coordinate): \(\text{min}(A[3], B[3])\)
    \end{itemize}
    
    \item **Overlap Dimensions:**
    \begin{itemize}
        \item Width: \(\text{overlap\_x2} - \text{overlap\_x1}\)
        \item Height: \(\text{overlap\_y2} - \text{overlap\_y1}\)
    \end{itemize}
    
    \item **Overlap Area:**
    \begin{itemize}
        \item If both width and height are positive, the rectangles overlap, and the overlapping area is their product.
        \item Otherwise, there is no overlap, and the overlapping area is zero.
    \end{itemize}
\end{itemize}

\subsection*{3. Compute Total Area}

\begin{itemize}
    \item Total Area = Area of Rectangle A + Area of Rectangle B - Overlapping Area
\end{itemize}

\subsection*{4. Example Walkthrough}

Consider the first example:
\begin{verbatim}
Input: A = [-3,0,3,4], B = [0,-1,9,2]
Output: 45
\end{verbatim}

\begin{enumerate}
    \item **Calculate Areas:**
    \begin{itemize}
        \item Area of A = (3 - (-3)) * (4 - 0) = 6 * 4 = 24
        \item Area of B = (9 - 0) * (2 - (-1)) = 9 * 3 = 27
    \end{itemize}
    
    \item **Determine Overlap:**
    \begin{itemize}
        \item overlap\_x1 = max(-3, 0) = 0
        \item overlap\_y1 = max(0, -1) = 0
        \item overlap\_x2 = min(3, 9) = 3
        \item overlap\_y2 = min(4, 2) = 2
        \item overlap\_width = 3 - 0 = 3
        \item overlap\_height = 2 - 0 = 2
        \item overlap\_area = 3 * 2 = 6
    \end{itemize}
    
    \item **Compute Total Area:**
    \begin{itemize}
        \item Total Area = 24 + 27 - 6 = 45
    \end{itemize}
\end{enumerate}

Thus, the function correctly returns \texttt{45}.

\section*{Why This Approach}

This approach is chosen for its straightforwardness and optimal efficiency. By directly calculating the individual areas and intelligently handling the overlapping region, the algorithm ensures accurate results without unnecessary computations. Its constant time complexity makes it highly efficient, even for large coordinate values.

\section*{Alternative Approaches}

\subsection*{1. Using Intersection Dimensions}

Instead of separately calculating areas, directly compute the dimensions of the overlapping region and subtract it from the sum of individual areas.

\begin{lstlisting}[language=Python]
def computeArea(A: List[int], B: List[int]) -> int:
    # Sum of individual areas
    area = (A[2] - A[0]) * (A[3] - A[1]) + (B[2] - B[0]) * (B[3] - B[1])
    
    # Overlapping area
    overlap_width = min(A[2], B[2]) - max(A[0], B[0])
    overlap_height = min(A[3], B[3]) - max(A[1], B[1])
    
    if overlap_width > 0 and overlap_height > 0:
        area -= overlap_width * overlap_height
    
    return area
\end{lstlisting}

\subsection*{2. Using Geometry Libraries}

Leverage computational geometry libraries to handle area calculations and overlapping detections.

\begin{lstlisting}[language=Python]
from shapely.geometry import box

def computeArea(A: List[int], B: List[int]) -> int:
    rect1 = box(A[0], A[1], A[2], A[3])
    rect2 = box(B[0], B[1], B[2], B[3])
    intersection = rect1.intersection(rect2)
    return int(rect1.area + rect2.area - intersection.area)
\end{lstlisting}

\textbf{Note}: This approach requires the \texttt{shapely} library and is more suitable for complex geometric operations.

\section*{Similar Problems to This One}

Several problems involve calculating areas, handling geometric overlaps, and spatial reasoning, utilizing similar algorithmic strategies:

\begin{itemize}
    \item \textbf{Rectangle Overlap}: Determine if two rectangles overlap.
    \item \textbf{Circle Area Overlap}: Calculate the overlapping area between two circles.
    \item \textbf{Polygon Area}: Compute the area of a given polygon.
    \item \textbf{Union of Rectangles}: Calculate the total area covered by multiple rectangles, accounting for overlaps.
    \item \textbf{Intersection of Lines}: Find the intersection point of two lines.
    \item \textbf{Closest Pair of Points}: Find the closest pair of points in a set.
    \item \textbf{Convex Hull}: Compute the convex hull of a set of points.
    \item \textbf{Point Inside Polygon}: Determine if a point lies inside a given polygon.
\end{itemize}

These problems reinforce concepts of geometric calculations, area computations, and efficient algorithm design in various contexts.

\section*{Things to Keep in Mind and Tricks}

When tackling the \textbf{Rectangle Area} problem, consider the following tips and best practices to enhance efficiency and correctness:

\begin{itemize}
    \item \textbf{Understand Geometric Relationships}: Grasp the positional relationships between rectangles to simplify area calculations.
    \index{Geometric Relationships}
    
    \item \textbf{Leverage Coordinate Comparisons}: Use direct comparisons of rectangle coordinates to determine overlapping regions.
    \index{Coordinate Comparisons}
    
    \item \textbf{Handle Overlapping Scenarios}: Accurately calculate the overlapping area to avoid double-counting.
    \index{Overlapping Scenarios}
    
    \item \textbf{Optimize for Efficiency}: Aim for a constant time \(O(1)\) solution by avoiding unnecessary computations or iterations.
    \index{Efficiency Optimization}
    
    \item \textbf{Avoid Floating-Point Precision Issues}: Since all coordinates are integers, floating-point precision is not a concern, simplifying the implementation.
    \index{Floating-Point Precision}
    
    \item \textbf{Use Helper Functions}: Create helper functions to encapsulate repetitive tasks, such as calculating overlap dimensions or areas.
    \index{Helper Functions}
    
    \item \textbf{Code Readability}: Maintain clear and readable code through meaningful variable names and structured logic.
    \index{Code Readability}
    
    \item \textbf{Test Extensively}: Implement a wide range of test cases, including overlapping, non-overlapping, and edge-touching rectangles, to ensure robustness.
    \index{Extensive Testing}
    
    \item \textbf{Understand Axis-Aligned Constraints}: Recognize that axis-aligned rectangles simplify area calculations compared to rotated rectangles.
    \index{Axis-Aligned Constraints}
    
    \item \textbf{Simplify Logical Conditions}: Combine multiple conditions logically to streamline the area calculation process.
    \index{Logical Conditions}
    
    \item \textbf{Use Absolute Values}: When calculating differences, ensure that the dimensions are positive by using absolute values or proper ordering.
    \index{Absolute Values}
    
    \item \textbf{Consider Edge Cases}: Handle cases where rectangles have zero area or touch at edges/corners without overlapping.
    \index{Edge Cases}
\end{itemize}

\section*{Corner and Special Cases to Test When Writing the Code}

When implementing the solution for the \textbf{Rectangle Area} problem, it is crucial to consider and rigorously test various edge cases to ensure robustness and correctness:

\begin{itemize}
    \item \textbf{No Overlap}: Rectangles are completely separate.
    \index{No Overlap}
    
    \item \textbf{Partial Overlap}: Rectangles overlap in one or more regions.
    \index{Partial Overlap}
    
    \item \textbf{Edge Touching}: Rectangles touch exactly at one edge without overlapping.
    \index{Edge Touching}
    
    \item \textbf{Corner Touching}: Rectangles touch exactly at one corner without overlapping.
    \index{Corner Touching}
    
    \item \textbf{One Rectangle Inside Another}: One rectangle is entirely within the other.
    \index{Rectangle Inside}
    
    \item \textbf{Identical Rectangles}: Both rectangles have the same coordinates.
    \index{Identical Rectangles}
    
    \item \textbf{Degenerate Rectangles}: Rectangles with zero area (e.g., \(x1 = x2\) or \(y1 = y2\)).
    \index{Degenerate Rectangles}
    
    \item \textbf{Large Coordinates}: Rectangles with very large coordinate values to test performance and integer handling.
    \index{Large Coordinates}
    
    \item \textbf{Negative Coordinates}: Rectangles positioned in negative coordinate space.
    \index{Negative Coordinates}
    
    \item \textbf{Mixed Overlapping Scenarios}: Combinations of the above cases to ensure comprehensive coverage.
    \index{Mixed Overlapping Scenarios}
    
    \item \textbf{Minimum and Maximum Bounds}: Rectangles at the minimum and maximum limits of the coordinate range.
    \index{Minimum and Maximum Bounds}
    
    \item \textbf{Sequential Rectangles}: Multiple rectangles placed sequentially without overlapping.
    \index{Sequential Rectangles}
    
    \item \textbf{Multiple Overlaps}: Scenarios where more than two rectangles overlap in different regions.
    \index{Multiple Overlaps}
\end{itemize}

\section*{Implementation Considerations}

When implementing the \texttt{computeArea} function, keep in mind the following considerations to ensure robustness and efficiency:

\begin{itemize}
    \item \textbf{Data Type Selection}: Use appropriate data types that can handle large input values without overflow or underflow.
    \index{Data Type Selection}
    
    \item \textbf{Optimizing Comparisons}: Structure logical conditions to efficiently determine overlap dimensions.
    \index{Optimizing Comparisons}
    
    \item \textbf{Handling Large Inputs}: Design the algorithm to efficiently handle large input sizes without significant performance degradation.
    \index{Handling Large Inputs}
    
    \item \textbf{Language-Specific Constraints}: Be aware of how the programming language handles large integers and arithmetic operations.
    \index{Language-Specific Constraints}
    
    \item \textbf{Avoiding Redundant Calculations}: Ensure that each calculation contributes towards determining the final area without unnecessary repetitions.
    \index{Avoiding Redundant Calculations}
    
    \item \textbf{Code Readability and Documentation}: Maintain clear and readable code through meaningful variable names and comprehensive comments to facilitate understanding and maintenance.
    \index{Code Readability}
    
    \item \textbf{Edge Case Handling}: Implement checks for edge cases to prevent incorrect results or runtime errors.
    \index{Edge Case Handling}
    
    \item \textbf{Testing and Validation}: Develop a comprehensive suite of test cases that cover all possible scenarios, including edge cases, to validate the correctness and efficiency of the implementation.
    \index{Testing and Validation}
    
    \item \textbf{Scalability}: Design the algorithm to scale efficiently with increasing input sizes, maintaining performance and resource utilization.
    \index{Scalability}
    
    \item \textbf{Using Helper Functions}: Consider creating helper functions for repetitive tasks, such as calculating overlap dimensions, to enhance modularity and reusability.
    \index{Helper Functions}
    
    \item \textbf{Consistent Naming Conventions}: Use consistent and descriptive naming conventions for variables to improve code clarity.
    \index{Naming Conventions}
    
    \item \textbf{Implementing Unit Tests}: Develop unit tests for each logical condition to ensure that all scenarios are correctly handled.
    \index{Unit Tests}
    
    \item \textbf{Error Handling}: Incorporate error handling to manage invalid inputs gracefully.
    \index{Error Handling}
\end{itemize}

\section*{Conclusion}

The \textbf{Rectangle Area} problem showcases the application of fundamental geometric principles and efficient algorithm design to compute spatial properties accurately. By systematically calculating individual areas and intelligently handling overlapping regions, the algorithm ensures precise results without redundant computations. Understanding and implementing such techniques not only enhances problem-solving skills but also provides a foundation for tackling more complex Computational Geometry challenges involving multiple geometric entities and intricate spatial relationships.

\printindex

% \input{sections/rectangle_overlap}
% \input{sections/rectangle_area}
% \input{sections/k_closest_points_to_origin}
% \input{sections/the_skyline_problem}
% % filename: k_closest_points_to_origin.tex

\problemsection{K Closest Points to Origin}
\label{chap:K_Closest_Points_to_Origin}
\marginnote{\href{https://leetcode.com/problems/k-closest-points-to-origin/}{[LeetCode Link]}\index{LeetCode}}
\marginnote{\href{https://www.geeksforgeeks.org/find-k-closest-points-origin/}{[GeeksForGeeks Link]}\index{GeeksForGeeks}}
\marginnote{\href{https://www.interviewbit.com/problems/k-closest-points/}{[InterviewBit Link]}\index{InterviewBit}}
\marginnote{\href{https://app.codesignal.com/challenges/k-closest-points-to-origin}{[CodeSignal Link]}\index{CodeSignal}}
\marginnote{\href{https://www.codewars.com/kata/k-closest-points-to-origin/train/python}{[Codewars Link]}\index{Codewars}}

The \textbf{K Closest Points to Origin} problem is a popular algorithmic challenge in Computational Geometry that involves identifying the \(k\) points closest to the origin in a 2D plane. This problem tests one's ability to apply efficient sorting and selection algorithms, understand distance computations, and optimize for performance. Mastery of this problem is essential for applications in spatial data analysis, nearest neighbor searches, and clustering algorithms.

\section*{Problem Statement}

Given an array of points where each point is represented as \([x, y]\) in the 2D plane, and an integer \(k\), return the \(k\) closest points to the origin \((0, 0)\).

The distance between two points \((x_1, y_1)\) and \((x_2, y_2)\) is the Euclidean distance \(\sqrt{(x_1 - x_2)^2 + (y_1 - y_2)^2}\). The origin is \((0, 0)\).

\textbf{Function signature in Python:}
\begin{lstlisting}[language=Python]
def kClosest(points: List[List[int]], K: int) -> List[List[int]]:
\end{lstlisting}

\section*{Examples}

\textbf{Example 1:}

\begin{verbatim}
Input: points = [[1,3],[-2,2]], K = 1
Output: [[-2,2]]
Explanation: 
The distance between (1, 3) and the origin is sqrt(10).
The distance between (-2, 2) and the origin is sqrt(8).
Since sqrt(8) < sqrt(10), (-2, 2) is closer to the origin.
\end{verbatim}

\textbf{Example 2:}

\begin{verbatim}
Input: points = [[3,3],[5,-1],[-2,4]], K = 2
Output: [[3,3],[-2,4]]
Explanation: 
The distances are sqrt(18), sqrt(26), and sqrt(20) respectively.
The two closest points are [3,3] and [-2,4].
\end{verbatim}

\textbf{Example 3:}

\begin{verbatim}
Input: points = [[0,1],[1,0]], K = 2
Output: [[0,1],[1,0]]
Explanation: 
Both points are equally close to the origin.
\end{verbatim}

\textbf{Example 4:}

\begin{verbatim}
Input: points = [[1,0],[0,1]], K = 1
Output: [[1,0]]
Explanation: 
Both points are equally close; returning any one is acceptable.
\end{verbatim}

\textbf{Constraints:}

\begin{itemize}
    \item \(1 \leq K \leq \text{points.length} \leq 10^4\)
    \item \(-10^4 < x_i, y_i < 10^4\)
\end{itemize}

LeetCode link: \href{https://leetcode.com/problems/k-closest-points-to-origin/}{K Closest Points to Origin}\index{LeetCode}

\section*{Algorithmic Approach}

To identify the \(k\) closest points to the origin, several algorithmic strategies can be employed. The most efficient methods aim to reduce the time complexity by avoiding the need to sort the entire list of points.

\subsection*{1. Sorting Based on Distance}

Calculate the Euclidean distance of each point from the origin and sort the points based on these distances. Select the first \(k\) points from the sorted list.

\begin{enumerate}
    \item Compute the distance for each point using the formula \(distance = x^2 + y^2\).
    \item Sort the points based on the computed distances.
    \item Return the first \(k\) points from the sorted list.
\end{enumerate}

\subsection*{2. Max Heap (Priority Queue)}

Use a max heap to maintain the \(k\) closest points. Iterate through each point, add it to the heap, and if the heap size exceeds \(k\), remove the farthest point.

\begin{enumerate}
    \item Initialize a max heap.
    \item For each point, compute its distance and add it to the heap.
    \item If the heap size exceeds \(k\), remove the point with the largest distance.
    \item After processing all points, the heap contains the \(k\) closest points.
\end{enumerate}

\subsection*{3. QuickSelect (Quick Sort Partitioning)}

Utilize the QuickSelect algorithm to find the \(k\) closest points without fully sorting the list.

\begin{enumerate}
    \item Choose a pivot point and partition the list based on distances relative to the pivot.
    \item Recursively apply QuickSelect to the partition containing the \(k\) closest points.
    \item Once the \(k\) closest points are identified, return them.
\end{enumerate}

\marginnote{QuickSelect offers an average time complexity of \(O(n)\), making it highly efficient for large datasets.}

\section*{Complexities}

\begin{itemize}
    \item \textbf{Sorting Based on Distance:}
    \begin{itemize}
        \item \textbf{Time Complexity:} \(O(n \log n)\)
        \item \textbf{Space Complexity:} \(O(n)\)
    \end{itemize}
    
    \item \textbf{Max Heap (Priority Queue):}
    \begin{itemize}
        \item \textbf{Time Complexity:} \(O(n \log k)\)
        \item \textbf{Space Complexity:} \(O(k)\)
    \end{itemize}
    
    \item \textbf{QuickSelect (Quick Sort Partitioning):}
    \begin{itemize}
        \item \textbf{Time Complexity:} Average case \(O(n)\), worst case \(O(n^2)\)
        \item \textbf{Space Complexity:} \(O(1)\) (in-place)
    \end{itemize}
\end{itemize}

\section*{Python Implementation}

\marginnote{Implementing QuickSelect provides an optimal average-case solution with linear time complexity.}

Below is the complete Python code implementing the \texttt{kClosest} function using the QuickSelect approach:

\begin{fullwidth}
\begin{lstlisting}[language=Python]
from typing import List
import random

class Solution:
    def kClosest(self, points: List[List[int]], K: int) -> List[List[int]]:
        def quickselect(left, right, K_smallest):
            if left == right:
                return
            
            # Select a random pivot_index
            pivot_index = random.randint(left, right)
            
            # Partition the array
            pivot_index = partition(left, right, pivot_index)
            
            # The pivot is in its final sorted position
            if K_smallest == pivot_index:
                return
            elif K_smallest < pivot_index:
                quickselect(left, pivot_index - 1, K_smallest)
            else:
                quickselect(pivot_index + 1, right, K_smallest)
        
        def partition(left, right, pivot_index):
            pivot_distance = distance(points[pivot_index])
            # Move pivot to end
            points[pivot_index], points[right] = points[right], points[pivot_index]
            store_index = left
            for i in range(left, right):
                if distance(points[i]) < pivot_distance:
                    points[store_index], points[i] = points[i], points[store_index]
                    store_index += 1
            # Move pivot to its final place
            points[right], points[store_index] = points[store_index], points[right]
            return store_index
        
        def distance(point):
            return point[0] ** 2 + point[1] ** 2
        
        n = len(points)
        quickselect(0, n - 1, K)
        return points[:K]

# Example usage:
solution = Solution()
print(solution.kClosest([[1,3],[-2,2]], 1))            # Output: [[-2,2]]
print(solution.kClosest([[3,3],[5,-1],[-2,4]], 2))     # Output: [[3,3],[-2,4]]
print(solution.kClosest([[0,1],[1,0]], 2))             # Output: [[0,1],[1,0]]
print(solution.kClosest([[1,0],[0,1]], 1))             # Output: [[1,0]] or [[0,1]]
\end{lstlisting}
\end{fullwidth}

This implementation uses the QuickSelect algorithm to efficiently find the \(k\) closest points to the origin without fully sorting the entire list. It ensures optimal performance even with large datasets.

\section*{Explanation}

The \texttt{kClosest} function identifies the \(k\) closest points to the origin using the QuickSelect algorithm. Here's a detailed breakdown of the implementation:

\subsection*{1. Distance Calculation}

\begin{itemize}
    \item The Euclidean distance is calculated as \(distance = x^2 + y^2\). Since we only need relative distances for comparison, the square root is omitted for efficiency.
\end{itemize}

\subsection*{2. QuickSelect Algorithm}

\begin{itemize}
    \item **Pivot Selection:**
    \begin{itemize}
        \item A random pivot is chosen to enhance the average-case performance.
    \end{itemize}
    
    \item **Partitioning:**
    \begin{itemize}
        \item The array is partitioned such that points with distances less than the pivot are moved to the left, and others to the right.
        \item The pivot is placed in its correct sorted position.
    \end{itemize}
    
    \item **Recursive Selection:**
    \begin{itemize}
        \item If the pivot's position matches \(K\), the selection is complete.
        \item Otherwise, recursively apply QuickSelect to the relevant partition.
    \end{itemize}
\end{itemize}

\subsection*{3. Final Selection}

\begin{itemize}
    \item After partitioning, the first \(K\) points in the list are the \(k\) closest points to the origin.
\end{itemize}

\subsection*{4. Example Walkthrough}

Consider the first example:
\begin{verbatim}
Input: points = [[1,3],[-2,2]], K = 1
Output: [[-2,2]]
\end{verbatim}

\begin{enumerate}
    \item **Calculate Distances:**
    \begin{itemize}
        \item [1,3] : \(1^2 + 3^2 = 10\)
        \item [-2,2] : \((-2)^2 + 2^2 = 8\)
    \end{itemize}
    
    \item **QuickSelect Process:**
    \begin{itemize}
        \item Choose a pivot, say [1,3] with distance 10.
        \item Compare and rearrange:
        \begin{itemize}
            \item [-2,2] has a smaller distance (8) and is moved to the left.
        \end{itemize}
        \item After partitioning, the list becomes [[-2,2], [1,3]].
        \item Since \(K = 1\), return the first point: [[-2,2]].
    \end{itemize}
\end{enumerate}

Thus, the function correctly identifies \([-2,2]\) as the closest point to the origin.

\section*{Why This Approach}

The QuickSelect algorithm is chosen for its average-case linear time complexity \(O(n)\), making it highly efficient for large datasets compared to sorting-based methods with \(O(n \log n)\) time complexity. By avoiding the need to sort the entire list, QuickSelect provides an optimal solution for finding the \(k\) closest points.

\section*{Alternative Approaches}

\subsection*{1. Sorting Based on Distance}

Sort all points based on their distances from the origin and select the first \(k\) points.

\begin{lstlisting}[language=Python]
class Solution:
    def kClosest(self, points: List[List[int]], K: int) -> List[List[int]]:
        points.sort(key=lambda P: P[0]**2 + P[1]**2)
        return points[:K]
\end{lstlisting}

\textbf{Complexities:}
\begin{itemize}
    \item \textbf{Time Complexity:} \(O(n \log n)\)
    \item \textbf{Space Complexity:} \(O(1)\)
\end{itemize}

\subsection*{2. Max Heap (Priority Queue)}

Use a max heap to maintain the \(k\) closest points.

\begin{lstlisting}[language=Python]
import heapq

class Solution:
    def kClosest(self, points: List[List[int]], K: int) -> List[List[int]]:
        heap = []
        for (x, y) in points:
            dist = -(x**2 + y**2)  # Max heap using negative distances
            heapq.heappush(heap, (dist, [x, y]))
            if len(heap) > K:
                heapq.heappop(heap)
        return [item[1] for item in heap]
\end{lstlisting}

\textbf{Complexities:}
\begin{itemize}
    \item \textbf{Time Complexity:} \(O(n \log k)\)
    \item \textbf{Space Complexity:} \(O(k)\)
\end{itemize}

\subsection*{3. Using Built-In Functions}

Leverage built-in functions for distance calculation and selection.

\begin{lstlisting}[language=Python]
import math

class Solution:
    def kClosest(self, points: List[List[int]], K: int) -> List[List[int]]:
        points.sort(key=lambda P: math.sqrt(P[0]**2 + P[1]**2))
        return points[:K]
\end{lstlisting}

\textbf{Note}: This method is similar to the sorting approach but uses the actual Euclidean distance.

\section*{Similar Problems to This One}

Several problems involve nearest neighbor searches, spatial data analysis, and efficient selection algorithms, utilizing similar algorithmic strategies:

\begin{itemize}
    \item \textbf{Closest Pair of Points}: Find the closest pair of points in a set.
    \item \textbf{Top K Frequent Elements}: Identify the most frequent elements in a dataset.
    \item \textbf{Kth Largest Element in an Array}: Find the \(k\)-th largest element in an unsorted array.
    \item \textbf{Sliding Window Maximum}: Find the maximum in each sliding window of size \(k\) over an array.
    \item \textbf{Merge K Sorted Lists}: Merge multiple sorted lists into a single sorted list.
    \item \textbf{Find Median from Data Stream}: Continuously find the median of a stream of numbers.
    \item \textbf{Top K Closest Stars}: Find the \(k\) closest stars to Earth based on their distances.
\end{itemize}

These problems reinforce concepts of efficient selection, heap usage, and distance computations in various contexts.

\section*{Things to Keep in Mind and Tricks}

When solving the \textbf{K Closest Points to Origin} problem, consider the following tips and best practices to enhance efficiency and correctness:

\begin{itemize}
    \item \textbf{Understand Distance Calculations}: Grasp the Euclidean distance formula and recognize that the square root can be omitted for comparison purposes.
    \index{Distance Calculations}
    
    \item \textbf{Leverage Efficient Algorithms}: Use QuickSelect or heap-based methods to optimize time complexity, especially for large datasets.
    \index{Efficient Algorithms}
    
    \item \textbf{Handle Ties Appropriately}: Decide how to handle points with identical distances when \(k\) is less than the number of such points.
    \index{Handling Ties}
    
    \item \textbf{Optimize Space Usage}: Choose algorithms that minimize additional space, such as in-place QuickSelect.
    \index{Space Optimization}
    
    \item \textbf{Use Appropriate Data Structures}: Utilize heaps, lists, and helper functions effectively to manage and process data.
    \index{Data Structures}
    
    \item \textbf{Implement Helper Functions}: Create helper functions for distance calculation and partitioning to enhance code modularity.
    \index{Helper Functions}
    
    \item \textbf{Code Readability}: Maintain clear and readable code through meaningful variable names and structured logic.
    \index{Code Readability}
    
    \item \textbf{Test Extensively}: Implement a wide range of test cases, including edge cases like multiple points with the same distance, to ensure robustness.
    \index{Extensive Testing}
    
    \item \textbf{Understand Algorithm Trade-offs}: Recognize the trade-offs between different approaches in terms of time and space complexities.
    \index{Algorithm Trade-offs}
    
    \item \textbf{Use Built-In Sorting Functions}: When using sorting-based approaches, leverage built-in functions for efficiency and simplicity.
    \index{Built-In Sorting}
    
    \item \textbf{Avoid Redundant Calculations}: Ensure that distance calculations are performed only when necessary to optimize performance.
    \index{Avoiding Redundant Calculations}
    
    \item \textbf{Language-Specific Features}: Utilize language-specific features or libraries that can simplify implementation, such as heapq in Python.
    \index{Language-Specific Features}
\end{itemize}

\section*{Corner and Special Cases to Test When Writing the Code}

When implementing the solution for the \textbf{K Closest Points to Origin} problem, it is crucial to consider and rigorously test various edge cases to ensure robustness and correctness:

\begin{itemize}
    \item \textbf{Multiple Points with Same Distance}: Ensure that the algorithm handles multiple points having the same distance from the origin.
    \index{Same Distance Points}
    
    \item \textbf{Points at Origin}: Include points that are exactly at the origin \((0,0)\).
    \index{Points at Origin}
    
    \item \textbf{Negative Coordinates}: Ensure that the algorithm correctly computes distances for points with negative \(x\) or \(y\) coordinates.
    \index{Negative Coordinates}
    
    \item \textbf{Large Coordinates}: Test with points having very large or very small coordinate values to verify integer handling.
    \index{Large Coordinates}
    
    \item \textbf{K Equals Number of Points}: When \(K\) is equal to the number of points, the algorithm should return all points.
    \index{K Equals Number of Points}
    
    \item \textbf{K is One}: Test with \(K = 1\) to ensure the closest point is correctly identified.
    \index{K is One}
    
    \item \textbf{All Points Same}: All points have the same coordinates.
    \index{All Points Same}
    
    \item \textbf{K is Zero}: Although \(K\) is defined to be at least 1, ensure that the algorithm gracefully handles \(K = 0\) if allowed.
    \index{K is Zero}
    
    \item \textbf{Single Point}: Only one point is provided, and \(K = 1\).
    \index{Single Point}
    
    \item \textbf{Mixed Coordinates}: Points with a mix of positive and negative coordinates.
    \index{Mixed Coordinates}
    
    \item \textbf{Points with Zero Distance}: Multiple points at the origin.
    \index{Zero Distance Points}
    
    \item \textbf{Sparse and Dense Points}: Densely packed points and sparsely distributed points.
    \index{Sparse and Dense Points}
    
    \item \textbf{Duplicate Points}: Multiple identical points in the input list.
    \index{Duplicate Points}
    
    \item \textbf{K Greater Than Number of Unique Points}: Ensure that the algorithm handles cases where \(K\) exceeds the number of unique points if applicable.
    \index{K Greater Than Unique Points}
\end{itemize}

\section*{Implementation Considerations}

When implementing the \texttt{kClosest} function, keep in mind the following considerations to ensure robustness and efficiency:

\begin{itemize}
    \item \textbf{Data Type Selection}: Use appropriate data types that can handle large input values without overflow or precision loss.
    \index{Data Type Selection}
    
    \item \textbf{Optimizing Distance Calculations}: Avoid calculating the square root since it is unnecessary for comparison purposes.
    \index{Optimizing Distance Calculations}
    
    \item \textbf{Choosing the Right Algorithm}: Select an algorithm based on the size of the input and the value of \(K\) to optimize time and space complexities.
    \index{Choosing the Right Algorithm}
    
    \item \textbf{Language-Specific Libraries}: Utilize language-specific libraries and functions (e.g., \texttt{heapq} in Python) to simplify implementation and enhance performance.
    \index{Language-Specific Libraries}
    
    \item \textbf{Avoiding Redundant Calculations}: Ensure that each point's distance is calculated only once to optimize performance.
    \index{Avoiding Redundant Calculations}
    
    \item \textbf{Implementing Helper Functions}: Create helper functions for tasks like distance calculation and partitioning to enhance modularity and readability.
    \index{Helper Functions}
    
    \item \textbf{Edge Case Handling}: Implement checks for edge cases to prevent incorrect results or runtime errors.
    \index{Edge Case Handling}
    
    \item \textbf{Testing and Validation}: Develop a comprehensive suite of test cases that cover all possible scenarios, including edge cases, to validate the correctness and efficiency of the implementation.
    \index{Testing and Validation}
    
    \item \textbf{Scalability}: Design the algorithm to scale efficiently with increasing input sizes, maintaining performance and resource utilization.
    \index{Scalability}
    
    \item \textbf{Consistent Naming Conventions}: Use consistent and descriptive naming conventions for variables and functions to improve code clarity.
    \index{Naming Conventions}
    
    \item \textbf{Memory Management}: Ensure that the algorithm manages memory efficiently, especially when dealing with large datasets.
    \index{Memory Management}
    
    \item \textbf{Avoiding Stack Overflow}: If implementing recursive approaches, be mindful of recursion limits and potential stack overflow issues.
    \index{Avoiding Stack Overflow}
    
    \item \textbf{Implementing Iterative Solutions}: Prefer iterative solutions when recursion may lead to increased space complexity or stack overflow.
    \index{Implementing Iterative Solutions}
\end{itemize}

\section*{Conclusion}

The \textbf{K Closest Points to Origin} problem exemplifies the application of efficient selection algorithms and geometric computations to solve spatial challenges effectively. By leveraging QuickSelect or heap-based methods, the algorithm achieves optimal time and space complexities, making it highly suitable for large datasets. Understanding and implementing such techniques not only enhances problem-solving skills but also provides a foundation for tackling more advanced Computational Geometry problems involving nearest neighbor searches, clustering, and spatial data analysis.

\printindex

% \input{sections/rectangle_overlap}
% \input{sections/rectangle_area}
% \input{sections/k_closest_points_to_origin}
% \input{sections/the_skyline_problem}
% % filename: the_skyline_problem.tex

\problemsection{The Skyline Problem}
\label{chap:The_Skyline_Problem}
\marginnote{\href{https://leetcode.com/problems/the-skyline-problem/}{[LeetCode Link]}\index{LeetCode}}
\marginnote{\href{https://www.geeksforgeeks.org/the-skyline-problem/}{[GeeksForGeeks Link]}\index{GeeksForGeeks}}
\marginnote{\href{https://www.interviewbit.com/problems/the-skyline-problem/}{[InterviewBit Link]}\index{InterviewBit}}
\marginnote{\href{https://app.codesignal.com/challenges/the-skyline-problem}{[CodeSignal Link]}\index{CodeSignal}}
\marginnote{\href{https://www.codewars.com/kata/the-skyline-problem/train/python}{[Codewars Link]}\index{Codewars}}

The \textbf{Skyline Problem} is a complex Computational Geometry challenge that involves computing the skyline formed by a collection of buildings in a 2D cityscape. Each building is represented by its left and right x-coordinates and its height. The skyline is defined by a list of "key points" where the height changes. This problem tests one's ability to handle large datasets, implement efficient sweep line algorithms, and manage event-driven processing. Mastery of this problem is essential for applications in computer graphics, urban planning simulations, and geographic information systems (GIS).

\section*{Problem Statement}

You are given a list of buildings in a cityscape. Each building is represented as a triplet \([Li, Ri, Hi]\), where \(Li\) and \(Ri\) are the x-coordinates of the left and right edges of the building, respectively, and \(Hi\) is the height of the building.

The skyline should be represented as a list of key points \([x, y]\) in sorted order by \(x\)-coordinate, where \(y\) is the height of the skyline at that point. The skyline should only include critical points where the height changes.

\textbf{Function signature in Python:}
\begin{lstlisting}[language=Python]
def getSkyline(buildings: List[List[int]]) -> List[List[int]]:
\end{lstlisting}

\section*{Examples}

\textbf{Example 1:}

\begin{verbatim}
Input: buildings = [[2,9,10], [3,7,15], [5,12,12], [15,20,10], [19,24,8]]
Output: [[2,10], [3,15], [7,12], [12,0], [15,10], [20,8], [24,0]]
Explanation:
- At x=2, the first building starts, height=10.
- At x=3, the second building starts, height=15.
- At x=7, the second building ends, the third building is still ongoing, height=12.
- At x=12, the third building ends, height drops to 0.
- At x=15, the fourth building starts, height=10.
- At x=20, the fourth building ends, the fifth building is still ongoing, height=8.
- At x=24, the fifth building ends, height drops to 0.
\end{verbatim}

\textbf{Example 2:}

\begin{verbatim}
Input: buildings = [[0,2,3], [2,5,3]]
Output: [[0,3], [5,0]]
Explanation:
- The two buildings are contiguous and have the same height, so the skyline drops to 0 at x=5.
\end{verbatim}

\textbf{Example 3:}

\begin{verbatim}
Input: buildings = [[1,3,3], [2,4,4], [5,6,1]]
Output: [[1,3], [2,4], [4,0], [5,1], [6,0]]
Explanation:
- At x=1, first building starts, height=3.
- At x=2, second building starts, height=4.
- At x=4, second building ends, height drops to 0.
- At x=5, third building starts, height=1.
- At x=6, third building ends, height drops to 0.
\end{verbatim}

\textbf{Example 4:}

\begin{verbatim}
Input: buildings = [[0,5,0]]
Output: []
Explanation:
- A building with height 0 does not contribute to the skyline.
\end{verbatim}

\textbf{Constraints:}

\begin{itemize}
    \item \(1 \leq \text{buildings.length} \leq 10^4\)
    \item \(0 \leq Li < Ri \leq 10^9\)
    \item \(0 \leq Hi \leq 10^4\)
\end{itemize}

\section*{Algorithmic Approach}

The \textbf{Sweep Line Algorithm} is an efficient method for solving the Skyline Problem. It involves processing events (building start and end points) in sorted order while maintaining a data structure (typically a max heap) to keep track of active buildings. Here's a step-by-step approach:

\subsection*{1. Event Representation}

Transform each building into two events:
\begin{itemize}
    \item **Start Event:** \((Li, -Hi)\) – Negative height indicates a building starts.
    \item **End Event:** \((Ri, Hi)\) – Positive height indicates a building ends.
\end{itemize}

Sorting the events ensures that start events are processed before end events at the same x-coordinate, and taller buildings are processed before shorter ones.

\subsection*{2. Sorting the Events}

Sort all events based on:
\begin{enumerate}
    \item **x-coordinate:** Ascending order.
    \item **Height:**
    \begin{itemize}
        \item For start events, taller buildings come first.
        \item For end events, shorter buildings come first.
    \end{itemize}
\end{enumerate}

\subsection*{3. Processing the Events}

Use a max heap to keep track of active building heights. Iterate through the sorted events:
\begin{enumerate}
    \item **Start Event:**
    \begin{itemize}
        \item Add the building's height to the heap.
    \end{itemize}
    
    \item **End Event:**
    \begin{itemize}
        \item Remove the building's height from the heap.
    \end{itemize}
    
    \item **Determine Current Max Height:**
    \begin{itemize}
        \item The current max height is the top of the heap.
    \end{itemize}
    
    \item **Update Skyline:**
    \begin{itemize}
        \item If the current max height differs from the previous max height, add a new key point \([x, current\_max\_height]\).
    \end{itemize}
\end{enumerate}

\subsection*{4. Finalizing the Skyline}

After processing all events, the accumulated key points represent the skyline.

\marginnote{The Sweep Line Algorithm efficiently handles dynamic changes in active buildings, ensuring accurate skyline construction.}

\section*{Complexities}

\begin{itemize}
    \item \textbf{Time Complexity:} \(O(n \log n)\), where \(n\) is the number of buildings. Sorting the events takes \(O(n \log n)\), and each heap operation takes \(O(\log n)\).
    
    \item \textbf{Space Complexity:} \(O(n)\), due to the storage of events and the heap.
\end{itemize}

\section*{Python Implementation}

\marginnote{Implementing the Sweep Line Algorithm with a max heap ensures an efficient and accurate solution.}

Below is the complete Python code implementing the \texttt{getSkyline} function:

\begin{fullwidth}
\begin{lstlisting}[language=Python]
from typing import List
import heapq

class Solution:
    def getSkyline(self, buildings: List[List[int]]) -> List[List[int]]:
        # Create a list of all events
        # For start events, use negative height to ensure they are processed before end events
        events = []
        for L, R, H in buildings:
            events.append((L, -H))
            events.append((R, H))
        
        # Sort the events
        # First by x-coordinate, then by height
        events.sort()
        
        # Max heap to keep track of active buildings
        heap = [0]  # Initialize with ground level
        heapq.heapify(heap)
        active_heights = {0: 1}  # Dictionary to count heights
        
        result = []
        prev_max = 0
        
        for x, h in events:
            if h < 0:
                # Start of a building, add height to heap and dictionary
                heapq.heappush(heap, h)
                active_heights[h] = active_heights.get(h, 0) + 1
            else:
                # End of a building, remove height from dictionary
                active_heights[h] -= 1
                if active_heights[h] == 0:
                    del active_heights[h]
            
            # Current max height
            while heap and active_heights.get(heap[0], 0) == 0:
                heapq.heappop(heap)
            current_max = -heap[0] if heap else 0
            
            # If the max height has changed, add to result
            if current_max != prev_max:
                result.append([x, current_max])
                prev_max = current_max
        
        return result

# Example usage:
solution = Solution()
print(solution.getSkyline([[2,9,10], [3,7,15], [5,12,12], [15,20,10], [19,24,8]]))
# Output: [[2,10], [3,15], [7,12], [12,0], [15,10], [20,8], [24,0]]

print(solution.getSkyline([[0,2,3], [2,5,3]]))
# Output: [[0,3], [5,0]]

print(solution.getSkyline([[1,3,3], [2,4,4], [5,6,1]]))
# Output: [[1,3], [2,4], [4,0], [5,1], [6,0]]

print(solution.getSkyline([[0,5,0]]))
# Output: []
\end{lstlisting}
\end{fullwidth}

This implementation efficiently constructs the skyline by processing all building events in sorted order and maintaining active building heights using a max heap. It ensures that only critical points where the skyline changes are recorded.

\section*{Explanation}

The \texttt{getSkyline} function constructs the skyline formed by a set of buildings by leveraging the Sweep Line Algorithm and a max heap to track active buildings. Here's a detailed breakdown of the implementation:

\subsection*{1. Event Representation}

\begin{itemize}
    \item Each building is transformed into two events:
    \begin{itemize}
        \item **Start Event:** \((Li, -Hi)\) – Negative height indicates the start of a building.
        \item **End Event:** \((Ri, Hi)\) – Positive height indicates the end of a building.
    \end{itemize}
\end{itemize}

\subsection*{2. Sorting the Events}

\begin{itemize}
    \item Events are sorted primarily by their x-coordinate in ascending order.
    \item For events with the same x-coordinate:
    \begin{itemize}
        \item Start events (with negative heights) are processed before end events.
        \item Taller buildings are processed before shorter ones.
    \end{itemize}
\end{itemize}

\subsection*{3. Processing the Events}

\begin{itemize}
    \item **Heap Initialization:**
    \begin{itemize}
        \item A max heap is initialized with a ground level height of 0.
        \item A dictionary \texttt{active\_heights} tracks the count of active building heights.
    \end{itemize}
    
    \item **Iterating Through Events:**
    \begin{enumerate}
        \item **Start Event:**
        \begin{itemize}
            \item Add the building's height to the heap.
            \item Increment the count of the height in \texttt{active\_heights}.
        \end{itemize}
        
        \item **End Event:**
        \begin{itemize}
            \item Decrement the count of the building's height in \texttt{active\_heights}.
            \item If the count reaches zero, remove the height from the dictionary.
        \end{itemize}
        
        \item **Determine Current Max Height:**
        \begin{itemize}
            \item Remove heights from the heap that are no longer active.
            \item The current max height is the top of the heap.
        \end{itemize}
        
        \item **Update Skyline:**
        \begin{itemize}
            \item If the current max height differs from the previous max height, add a new key point \([x, current\_max\_height]\).
        \end{itemize}
    \end{enumerate}
\end{itemize}

\subsection*{4. Finalizing the Skyline}

\begin{itemize}
    \item After processing all events, the \texttt{result} list contains the key points defining the skyline.
\end{itemize}

\subsection*{5. Example Walkthrough}

Consider the first example:
\begin{verbatim}
Input: buildings = [[2,9,10], [3,7,15], [5,12,12], [15,20,10], [19,24,8]]
Output: [[2,10], [3,15], [7,12], [12,0], [15,10], [20,8], [24,0]]
\end{verbatim}

\begin{enumerate}
    \item **Event Transformation:**
    \begin{itemize}
        \item \((2, -10)\), \((9, 10)\)
        \item \((3, -15)\), \((7, 15)\)
        \item \((5, -12)\), \((12, 12)\)
        \item \((15, -10)\), \((20, 10)\)
        \item \((19, -8)\), \((24, 8)\)
    \end{itemize}
    
    \item **Sorting Events:**
    \begin{itemize}
        \item Sorted order: \((2, -10)\), \((3, -15)\), \((5, -12)\), \((7, 15)\), \((9, 10)\), \((12, 12)\), \((15, -10)\), \((19, -8)\), \((20, 10)\), \((24, 8)\)
    \end{itemize}
    
    \item **Processing Events:**
    \begin{itemize}
        \item At each event, update the heap and determine if the skyline height changes.
    \end{itemize}
    
    \item **Result Construction:**
    \begin{itemize}
        \item The resulting skyline key points are accumulated as \([[2,10], [3,15], [7,12], [12,0], [15,10], [20,8], [24,0]]\).
    \end{itemize}
\end{enumerate}

Thus, the function correctly constructs the skyline based on the buildings' positions and heights.

\section*{Why This Approach}

The Sweep Line Algorithm combined with a max heap offers an optimal solution with \(O(n \log n)\) time complexity and efficient handling of overlapping buildings. By processing events in sorted order and maintaining active building heights, the algorithm ensures that all critical points in the skyline are accurately identified without redundant computations.

\section*{Alternative Approaches}

\subsection*{1. Divide and Conquer}

Divide the set of buildings into smaller subsets, compute the skyline for each subset, and then merge the skylines.

\begin{lstlisting}[language=Python]
class Solution:
    def getSkyline(self, buildings: List[List[int]]) -> List[List[int]]:
        def merge(left, right):
            h1, h2 = 0, 0
            i, j = 0, 0
            merged = []
            while i < len(left) and j < len(right):
                if left[i][0] < right[j][0]:
                    x, h1 = left[i]
                    i += 1
                elif left[i][0] > right[j][0]:
                    x, h2 = right[j]
                    j += 1
                else:
                    x, h1 = left[i]
                    _, h2 = right[j]
                    i += 1
                    j += 1
                max_h = max(h1, h2)
                if not merged or merged[-1][1] != max_h:
                    merged.append([x, max_h])
            merged.extend(left[i:])
            merged.extend(right[j:])
            return merged
        
        def divide(buildings):
            if not buildings:
                return []
            if len(buildings) == 1:
                L, R, H = buildings[0]
                return [[L, H], [R, 0]]
            mid = len(buildings) // 2
            left = divide(buildings[:mid])
            right = divide(buildings[mid:])
            return merge(left, right)
        
        return divide(buildings)
\end{lstlisting}

\textbf{Complexities:}
\begin{itemize}
    \item \textbf{Time Complexity:} \(O(n \log n)\)
    \item \textbf{Space Complexity:} \(O(n)\)
\end{itemize}

\subsection*{2. Using Segment Trees}

Implement a segment tree to manage and query overlapping building heights dynamically.

\textbf{Note}: This approach is more complex and is generally used for advanced scenarios with multiple dynamic queries.

\section*{Similar Problems to This One}

Several problems involve skyline-like constructions, spatial data analysis, and efficient event processing, utilizing similar algorithmic strategies:

\begin{itemize}
    \item \textbf{Merge Intervals}: Merge overlapping intervals in a list.
    \item \textbf{Largest Rectangle in Histogram}: Find the largest rectangular area in a histogram.
    \item \textbf{Interval Partitioning}: Assign intervals to resources without overlap.
    \item \textbf{Line Segment Intersection}: Detect intersections among line segments.
    \item \textbf{Closest Pair of Points}: Find the closest pair of points in a set.
    \item \textbf{Convex Hull}: Compute the convex hull of a set of points.
    \item \textbf{Point Inside Polygon}: Determine if a point lies inside a given polygon.
    \item \textbf{Range Searching}: Efficiently query geometric data within a specified range.
\end{itemize}

These problems reinforce concepts of event-driven processing, spatial reasoning, and efficient algorithm design in various contexts.

\section*{Things to Keep in Mind and Tricks}

When tackling the \textbf{Skyline Problem}, consider the following tips and best practices to enhance efficiency and correctness:

\begin{itemize}
    \item \textbf{Understand Sweep Line Technique}: Grasp how the sweep line algorithm processes events in sorted order to handle dynamic changes efficiently.
    \index{Sweep Line Technique}
    
    \item \textbf{Leverage Priority Queues (Heaps)}: Use max heaps to keep track of active buildings' heights, enabling quick access to the current maximum height.
    \index{Priority Queues}
    
    \item \textbf{Handle Start and End Events Differently}: Differentiate between building start and end events to accurately manage active heights.
    \index{Start and End Events}
    
    \item \textbf{Optimize Event Sorting}: Sort events primarily by x-coordinate and secondarily by height to ensure correct processing order.
    \index{Event Sorting}
    
    \item \textbf{Manage Active Heights Efficiently}: Use data structures that allow efficient insertion, deletion, and retrieval of maximum elements.
    \index{Active Heights Management}
    
    \item \textbf{Avoid Redundant Key Points}: Only record key points when the skyline height changes to minimize the output list.
    \index{Avoiding Redundant Key Points}
    
    \item \textbf{Implement Helper Functions}: Create helper functions for tasks like distance calculation, event handling, and heap management to enhance modularity.
    \index{Helper Functions}
    
    \item \textbf{Code Readability}: Maintain clear and readable code through meaningful variable names and structured logic.
    \index{Code Readability}
    
    \item \textbf{Test Extensively}: Implement a wide range of test cases, including overlapping, non-overlapping, and edge-touching buildings, to ensure robustness.
    \index{Extensive Testing}
    
    \item \textbf{Handle Degenerate Cases}: Manage cases where buildings have zero height or identical coordinates gracefully.
    \index{Degenerate Cases}
    
    \item \textbf{Understand Geometric Relationships}: Grasp how buildings overlap and influence the skyline to simplify the algorithm.
    \index{Geometric Relationships}
    
    \item \textbf{Use Appropriate Data Structures}: Utilize appropriate data structures like heaps, lists, and dictionaries to manage and process data efficiently.
    \index{Appropriate Data Structures}
    
    \item \textbf{Optimize for Large Inputs}: Design the algorithm to handle large numbers of buildings without significant performance degradation.
    \index{Optimizing for Large Inputs}
    
    \item \textbf{Implement Iterative Solutions Carefully}: Ensure that loop conditions are correctly defined to prevent infinite loops or incorrect terminations.
    \index{Iterative Solutions}
    
    \item \textbf{Consistent Naming Conventions}: Use consistent and descriptive naming conventions for variables and functions to improve code clarity.
    \index{Naming Conventions}
\end{itemize}

\section*{Corner and Special Cases to Test When Writing the Code}

When implementing the solution for the \textbf{Skyline Problem}, it is crucial to consider and rigorously test various edge cases to ensure robustness and correctness:

\begin{itemize}
    \item \textbf{No Overlapping Buildings}: All buildings are separate and do not overlap.
    \index{No Overlapping Buildings}
    
    \item \textbf{Fully Overlapping Buildings}: Multiple buildings completely overlap each other.
    \index{Fully Overlapping Buildings}
    
    \item \textbf{Buildings Touching at Edges}: Buildings share common edges without overlapping.
    \index{Buildings Touching at Edges}
    
    \item \textbf{Buildings Touching at Corners}: Buildings share common corners without overlapping.
    \index{Buildings Touching at Corners}
    
    \item \textbf{Single Building}: Only one building is present.
    \index{Single Building}
    
    \item \textbf{Multiple Buildings with Same Start or End}: Multiple buildings start or end at the same x-coordinate.
    \index{Same Start or End}
    
    \item \textbf{Buildings with Zero Height}: Buildings that have zero height should not affect the skyline.
    \index{Buildings with Zero Height}
    
    \item \textbf{Large Number of Buildings}: Test with a large number of buildings to ensure performance and scalability.
    \index{Large Number of Buildings}
    
    \item \textbf{Buildings with Negative Coordinates}: Buildings positioned in negative coordinate space.
    \index{Negative Coordinates}
    
    \item \textbf{Boundary Values}: Buildings at the minimum and maximum limits of the coordinate range.
    \index{Boundary Values}
    
    \item \textbf{Buildings with Identical Coordinates}: Multiple buildings with the same coordinates.
    \index{Identical Coordinates}
    
    \item \textbf{Sequential Buildings}: Buildings placed sequentially without gaps.
    \index{Sequential Buildings}
    
    \item \textbf{Overlapping and Non-Overlapping Mixed}: A mix of overlapping and non-overlapping buildings.
    \index{Overlapping and Non-Overlapping Mixed}
    
    \item \textbf{Buildings with Very Large Heights}: Buildings with heights at the upper limit of the constraints.
    \index{Very Large Heights}
    
    \item \textbf{Empty Input}: No buildings are provided.
    \index{Empty Input}
\end{itemize}

\section*{Implementation Considerations}

When implementing the \texttt{getSkyline} function, keep in mind the following considerations to ensure robustness and efficiency:

\begin{itemize}
    \item \textbf{Data Type Selection}: Use appropriate data types that can handle large input values and avoid overflow or precision issues.
    \index{Data Type Selection}
    
    \item \textbf{Optimizing Event Sorting}: Efficiently sort events based on x-coordinates and heights to ensure correct processing order.
    \index{Optimizing Event Sorting}
    
    \item \textbf{Handling Large Inputs}: Design the algorithm to handle up to \(10^4\) buildings efficiently without significant performance degradation.
    \index{Handling Large Inputs}
    
    \item \textbf{Using Efficient Data Structures}: Utilize heaps, lists, and dictionaries effectively to manage and process events and active heights.
    \index{Efficient Data Structures}
    
    \item \textbf{Avoiding Redundant Calculations}: Ensure that distance and overlap calculations are performed only when necessary to optimize performance.
    \index{Avoiding Redundant Calculations}
    
    \item \textbf{Code Readability and Documentation}: Maintain clear and readable code through meaningful variable names and comprehensive comments to facilitate understanding and maintenance.
    \index{Code Readability}
    
    \item \textbf{Edge Case Handling}: Implement checks for edge cases to prevent incorrect results or runtime errors.
    \index{Edge Case Handling}
    
    \item \textbf{Implementing Helper Functions}: Create helper functions for tasks like distance calculation, event handling, and heap management to enhance modularity.
    \index{Helper Functions}
    
    \item \textbf{Consistent Naming Conventions}: Use consistent and descriptive naming conventions for variables and functions to improve code clarity.
    \index{Naming Conventions}
    
    \item \textbf{Memory Management}: Ensure that the algorithm manages memory efficiently, especially when dealing with large datasets.
    \index{Memory Management}
    
    \item \textbf{Implementing Iterative Solutions Carefully}: Ensure that loop conditions are correctly defined to prevent infinite loops or incorrect terminations.
    \index{Iterative Solutions}
    
    \item \textbf{Avoiding Floating-Point Precision Issues}: Since the problem deals with integers, floating-point precision is not a concern, simplifying the implementation.
    \index{Floating-Point Precision}
    
    \item \textbf{Testing and Validation}: Develop a comprehensive suite of test cases that cover all possible scenarios, including edge cases, to validate the correctness and efficiency of the implementation.
    \index{Testing and Validation}
    
    \item \textbf{Performance Considerations}: Optimize the loop conditions and operations to ensure that the function runs efficiently, especially for large input numbers.
    \index{Performance Considerations}
\end{itemize}

\section*{Conclusion}

The \textbf{Skyline Problem} is a quintessential example of applying advanced algorithmic techniques and geometric reasoning to solve complex spatial challenges. By leveraging the Sweep Line Algorithm and maintaining active building heights using a max heap, the solution efficiently constructs the skyline with optimal time and space complexities. Understanding and implementing such sophisticated algorithms not only enhances problem-solving skills but also provides a foundation for tackling a wide array of Computational Geometry problems in various domains, including computer graphics, urban planning simulations, and geographic information systems.

\printindex

% \input{sections/rectangle_overlap}
% \input{sections/rectangle_area}
% \input{sections/k_closest_points_to_origin}
% \input{sections/the_skyline_problem}
% % filename: rectangle_area.tex

\problemsection{Rectangle Area}
\label{chap:Rectangle_Area}
\marginnote{\href{https://leetcode.com/problems/rectangle-area/}{[LeetCode Link]}\index{LeetCode}}
\marginnote{\href{https://www.geeksforgeeks.org/find-area-two-overlapping-rectangles/}{[GeeksForGeeks Link]}\index{GeeksForGeeks}}
\marginnote{\href{https://www.interviewbit.com/problems/rectangle-area/}{[InterviewBit Link]}\index{InterviewBit}}
\marginnote{\href{https://app.codesignal.com/challenges/rectangle-area}{[CodeSignal Link]}\index{CodeSignal}}
\marginnote{\href{https://www.codewars.com/kata/rectangle-area/train/python}{[Codewars Link]}\index{Codewars}}

The \textbf{Rectangle Area} problem is a classic Computational Geometry challenge that involves calculating the total area covered by two axis-aligned rectangles in a 2D plane. This problem tests one's ability to perform geometric calculations, handle overlapping scenarios, and implement efficient algorithms. Mastery of this problem is essential for applications in computer graphics, spatial analysis, and computational modeling.

\section*{Problem Statement}

Given two axis-aligned rectangles in a 2D plane, compute the total area covered by the two rectangles. The area covered by the overlapping region should be counted only once.

Each rectangle is represented as a list of four integers \([x1, y1, x2, y2]\), where \((x1, y1)\) are the coordinates of the bottom-left corner, and \((x2, y2)\) are the coordinates of the top-right corner.

\textbf{Function signature in Python:}
\begin{lstlisting}[language=Python]
def computeArea(A: List[int], B: List[int]) -> int:
\end{lstlisting}

\section*{Examples}

\textbf{Example 1:}

\begin{verbatim}
Input: A = [-3,0,3,4], B = [0,-1,9,2]
Output: 45
Explanation:
Area of A = (3 - (-3)) * (4 - 0) = 6 * 4 = 24
Area of B = (9 - 0) * (2 - (-1)) = 9 * 3 = 27
Overlapping Area = (3 - 0) * (2 - 0) = 3 * 2 = 6
Total Area = 24 + 27 - 6 = 45
\end{verbatim}

\textbf{Example 2:}

\begin{verbatim}
Input: A = [0,0,0,0], B = [0,0,0,0]
Output: 0
Explanation:
Both rectangles are degenerate points with zero area.
\end{verbatim}

\textbf{Example 3:}

\begin{verbatim}
Input: A = [0,0,2,2], B = [1,1,3,3]
Output: 7
Explanation:
Area of A = 4
Area of B = 4
Overlapping Area = 1
Total Area = 4 + 4 - 1 = 7
\end{verbatim}

\textbf{Example 4:}

\begin{verbatim}
Input: A = [0,0,1,1], B = [1,0,2,1]
Output: 2
Explanation:
Rectangles touch at the edge but do not overlap.
Area of A = 1
Area of B = 1
Overlapping Area = 0
Total Area = 1 + 1 = 2
\end{verbatim}

\textbf{Constraints:}

\begin{itemize}
    \item All coordinates are integers in the range \([-10^9, 10^9]\).
    \item For each rectangle, \(x1 < x2\) and \(y1 < y2\).
\end{itemize}

LeetCode link: \href{https://leetcode.com/problems/rectangle-area/}{Rectangle Area}\index{LeetCode}

\section*{Algorithmic Approach}

To compute the total area covered by two axis-aligned rectangles, we can follow these steps:

1. **Calculate Individual Areas:**
   - Compute the area of the first rectangle.
   - Compute the area of the second rectangle.

2. **Determine Overlapping Area:**
   - Calculate the coordinates of the overlapping rectangle, if any.
   - If the rectangles overlap, compute the area of the overlapping region.

3. **Compute Total Area:**
   - Sum the individual areas and subtract the overlapping area to avoid double-counting.

\marginnote{This approach ensures accurate area calculation by handling overlapping regions appropriately.}

\section*{Complexities}

\begin{itemize}
    \item \textbf{Time Complexity:} \(O(1)\). The algorithm performs a constant number of calculations.
    
    \item \textbf{Space Complexity:} \(O(1)\). Only a fixed amount of extra space is used for variables.
\end{itemize}

\section*{Python Implementation}

\marginnote{Implementing the area calculation with overlap consideration ensures an accurate and efficient solution.}

Below is the complete Python code implementing the \texttt{computeArea} function:

\begin{fullwidth}
\begin{lstlisting}[language=Python]
from typing import List

class Solution:
    def computeArea(self, A: List[int], B: List[int]) -> int:
        # Calculate area of rectangle A
        areaA = (A[2] - A[0]) * (A[3] - A[1])
        
        # Calculate area of rectangle B
        areaB = (B[2] - B[0]) * (B[3] - B[1])
        
        # Determine overlap coordinates
        overlap_x1 = max(A[0], B[0])
        overlap_y1 = max(A[1], B[1])
        overlap_x2 = min(A[2], B[2])
        overlap_y2 = min(A[3], B[3])
        
        # Calculate overlapping area
        overlap_width = overlap_x2 - overlap_x1
        overlap_height = overlap_y2 - overlap_y1
        overlap_area = 0
        if overlap_width > 0 and overlap_height > 0:
            overlap_area = overlap_width * overlap_height
        
        # Total area is sum of individual areas minus overlapping area
        total_area = areaA + areaB - overlap_area
        return total_area

# Example usage:
solution = Solution()
print(solution.computeArea([-3,0,3,4], [0,-1,9,2]))  # Output: 45
print(solution.computeArea([0,0,0,0], [0,0,0,0]))    # Output: 0
print(solution.computeArea([0,0,2,2], [1,1,3,3]))    # Output: 7
print(solution.computeArea([0,0,1,1], [1,0,2,1]))    # Output: 2
\end{lstlisting}
\end{fullwidth}

This implementation accurately computes the total area covered by two rectangles by accounting for any overlapping regions. It ensures that the overlapping area is not double-counted.

\section*{Explanation}

The \texttt{computeArea} function calculates the combined area of two axis-aligned rectangles by following these steps:

\subsection*{1. Calculate Individual Areas}

\begin{itemize}
    \item **Rectangle A:**
    \begin{itemize}
        \item Width: \(A[2] - A[0]\)
        \item Height: \(A[3] - A[1]\)
        \item Area: Width \(\times\) Height
    \end{itemize}
    
    \item **Rectangle B:**
    \begin{itemize}
        \item Width: \(B[2] - B[0]\)
        \item Height: \(B[3] - B[1]\)
        \item Area: Width \(\times\) Height
    \end{itemize}
\end{itemize}

\subsection*{2. Determine Overlapping Area}

\begin{itemize}
    \item **Overlap Coordinates:**
    \begin{itemize}
        \item Left (x-coordinate): \(\text{max}(A[0], B[0])\)
        \item Bottom (y-coordinate): \(\text{max}(A[1], B[1])\)
        \item Right (x-coordinate): \(\text{min}(A[2], B[2])\)
        \item Top (y-coordinate): \(\text{min}(A[3], B[3])\)
    \end{itemize}
    
    \item **Overlap Dimensions:**
    \begin{itemize}
        \item Width: \(\text{overlap\_x2} - \text{overlap\_x1}\)
        \item Height: \(\text{overlap\_y2} - \text{overlap\_y1}\)
    \end{itemize}
    
    \item **Overlap Area:**
    \begin{itemize}
        \item If both width and height are positive, the rectangles overlap, and the overlapping area is their product.
        \item Otherwise, there is no overlap, and the overlapping area is zero.
    \end{itemize}
\end{itemize}

\subsection*{3. Compute Total Area}

\begin{itemize}
    \item Total Area = Area of Rectangle A + Area of Rectangle B - Overlapping Area
\end{itemize}

\subsection*{4. Example Walkthrough}

Consider the first example:
\begin{verbatim}
Input: A = [-3,0,3,4], B = [0,-1,9,2]
Output: 45
\end{verbatim}

\begin{enumerate}
    \item **Calculate Areas:**
    \begin{itemize}
        \item Area of A = (3 - (-3)) * (4 - 0) = 6 * 4 = 24
        \item Area of B = (9 - 0) * (2 - (-1)) = 9 * 3 = 27
    \end{itemize}
    
    \item **Determine Overlap:**
    \begin{itemize}
        \item overlap\_x1 = max(-3, 0) = 0
        \item overlap\_y1 = max(0, -1) = 0
        \item overlap\_x2 = min(3, 9) = 3
        \item overlap\_y2 = min(4, 2) = 2
        \item overlap\_width = 3 - 0 = 3
        \item overlap\_height = 2 - 0 = 2
        \item overlap\_area = 3 * 2 = 6
    \end{itemize}
    
    \item **Compute Total Area:**
    \begin{itemize}
        \item Total Area = 24 + 27 - 6 = 45
    \end{itemize}
\end{enumerate}

Thus, the function correctly returns \texttt{45}.

\section*{Why This Approach}

This approach is chosen for its straightforwardness and optimal efficiency. By directly calculating the individual areas and intelligently handling the overlapping region, the algorithm ensures accurate results without unnecessary computations. Its constant time complexity makes it highly efficient, even for large coordinate values.

\section*{Alternative Approaches}

\subsection*{1. Using Intersection Dimensions}

Instead of separately calculating areas, directly compute the dimensions of the overlapping region and subtract it from the sum of individual areas.

\begin{lstlisting}[language=Python]
def computeArea(A: List[int], B: List[int]) -> int:
    # Sum of individual areas
    area = (A[2] - A[0]) * (A[3] - A[1]) + (B[2] - B[0]) * (B[3] - B[1])
    
    # Overlapping area
    overlap_width = min(A[2], B[2]) - max(A[0], B[0])
    overlap_height = min(A[3], B[3]) - max(A[1], B[1])
    
    if overlap_width > 0 and overlap_height > 0:
        area -= overlap_width * overlap_height
    
    return area
\end{lstlisting}

\subsection*{2. Using Geometry Libraries}

Leverage computational geometry libraries to handle area calculations and overlapping detections.

\begin{lstlisting}[language=Python]
from shapely.geometry import box

def computeArea(A: List[int], B: List[int]) -> int:
    rect1 = box(A[0], A[1], A[2], A[3])
    rect2 = box(B[0], B[1], B[2], B[3])
    intersection = rect1.intersection(rect2)
    return int(rect1.area + rect2.area - intersection.area)
\end{lstlisting}

\textbf{Note}: This approach requires the \texttt{shapely} library and is more suitable for complex geometric operations.

\section*{Similar Problems to This One}

Several problems involve calculating areas, handling geometric overlaps, and spatial reasoning, utilizing similar algorithmic strategies:

\begin{itemize}
    \item \textbf{Rectangle Overlap}: Determine if two rectangles overlap.
    \item \textbf{Circle Area Overlap}: Calculate the overlapping area between two circles.
    \item \textbf{Polygon Area}: Compute the area of a given polygon.
    \item \textbf{Union of Rectangles}: Calculate the total area covered by multiple rectangles, accounting for overlaps.
    \item \textbf{Intersection of Lines}: Find the intersection point of two lines.
    \item \textbf{Closest Pair of Points}: Find the closest pair of points in a set.
    \item \textbf{Convex Hull}: Compute the convex hull of a set of points.
    \item \textbf{Point Inside Polygon}: Determine if a point lies inside a given polygon.
\end{itemize}

These problems reinforce concepts of geometric calculations, area computations, and efficient algorithm design in various contexts.

\section*{Things to Keep in Mind and Tricks}

When tackling the \textbf{Rectangle Area} problem, consider the following tips and best practices to enhance efficiency and correctness:

\begin{itemize}
    \item \textbf{Understand Geometric Relationships}: Grasp the positional relationships between rectangles to simplify area calculations.
    \index{Geometric Relationships}
    
    \item \textbf{Leverage Coordinate Comparisons}: Use direct comparisons of rectangle coordinates to determine overlapping regions.
    \index{Coordinate Comparisons}
    
    \item \textbf{Handle Overlapping Scenarios}: Accurately calculate the overlapping area to avoid double-counting.
    \index{Overlapping Scenarios}
    
    \item \textbf{Optimize for Efficiency}: Aim for a constant time \(O(1)\) solution by avoiding unnecessary computations or iterations.
    \index{Efficiency Optimization}
    
    \item \textbf{Avoid Floating-Point Precision Issues}: Since all coordinates are integers, floating-point precision is not a concern, simplifying the implementation.
    \index{Floating-Point Precision}
    
    \item \textbf{Use Helper Functions}: Create helper functions to encapsulate repetitive tasks, such as calculating overlap dimensions or areas.
    \index{Helper Functions}
    
    \item \textbf{Code Readability}: Maintain clear and readable code through meaningful variable names and structured logic.
    \index{Code Readability}
    
    \item \textbf{Test Extensively}: Implement a wide range of test cases, including overlapping, non-overlapping, and edge-touching rectangles, to ensure robustness.
    \index{Extensive Testing}
    
    \item \textbf{Understand Axis-Aligned Constraints}: Recognize that axis-aligned rectangles simplify area calculations compared to rotated rectangles.
    \index{Axis-Aligned Constraints}
    
    \item \textbf{Simplify Logical Conditions}: Combine multiple conditions logically to streamline the area calculation process.
    \index{Logical Conditions}
    
    \item \textbf{Use Absolute Values}: When calculating differences, ensure that the dimensions are positive by using absolute values or proper ordering.
    \index{Absolute Values}
    
    \item \textbf{Consider Edge Cases}: Handle cases where rectangles have zero area or touch at edges/corners without overlapping.
    \index{Edge Cases}
\end{itemize}

\section*{Corner and Special Cases to Test When Writing the Code}

When implementing the solution for the \textbf{Rectangle Area} problem, it is crucial to consider and rigorously test various edge cases to ensure robustness and correctness:

\begin{itemize}
    \item \textbf{No Overlap}: Rectangles are completely separate.
    \index{No Overlap}
    
    \item \textbf{Partial Overlap}: Rectangles overlap in one or more regions.
    \index{Partial Overlap}
    
    \item \textbf{Edge Touching}: Rectangles touch exactly at one edge without overlapping.
    \index{Edge Touching}
    
    \item \textbf{Corner Touching}: Rectangles touch exactly at one corner without overlapping.
    \index{Corner Touching}
    
    \item \textbf{One Rectangle Inside Another}: One rectangle is entirely within the other.
    \index{Rectangle Inside}
    
    \item \textbf{Identical Rectangles}: Both rectangles have the same coordinates.
    \index{Identical Rectangles}
    
    \item \textbf{Degenerate Rectangles}: Rectangles with zero area (e.g., \(x1 = x2\) or \(y1 = y2\)).
    \index{Degenerate Rectangles}
    
    \item \textbf{Large Coordinates}: Rectangles with very large coordinate values to test performance and integer handling.
    \index{Large Coordinates}
    
    \item \textbf{Negative Coordinates}: Rectangles positioned in negative coordinate space.
    \index{Negative Coordinates}
    
    \item \textbf{Mixed Overlapping Scenarios}: Combinations of the above cases to ensure comprehensive coverage.
    \index{Mixed Overlapping Scenarios}
    
    \item \textbf{Minimum and Maximum Bounds}: Rectangles at the minimum and maximum limits of the coordinate range.
    \index{Minimum and Maximum Bounds}
    
    \item \textbf{Sequential Rectangles}: Multiple rectangles placed sequentially without overlapping.
    \index{Sequential Rectangles}
    
    \item \textbf{Multiple Overlaps}: Scenarios where more than two rectangles overlap in different regions.
    \index{Multiple Overlaps}
\end{itemize}

\section*{Implementation Considerations}

When implementing the \texttt{computeArea} function, keep in mind the following considerations to ensure robustness and efficiency:

\begin{itemize}
    \item \textbf{Data Type Selection}: Use appropriate data types that can handle large input values without overflow or underflow.
    \index{Data Type Selection}
    
    \item \textbf{Optimizing Comparisons}: Structure logical conditions to efficiently determine overlap dimensions.
    \index{Optimizing Comparisons}
    
    \item \textbf{Handling Large Inputs}: Design the algorithm to efficiently handle large input sizes without significant performance degradation.
    \index{Handling Large Inputs}
    
    \item \textbf{Language-Specific Constraints}: Be aware of how the programming language handles large integers and arithmetic operations.
    \index{Language-Specific Constraints}
    
    \item \textbf{Avoiding Redundant Calculations}: Ensure that each calculation contributes towards determining the final area without unnecessary repetitions.
    \index{Avoiding Redundant Calculations}
    
    \item \textbf{Code Readability and Documentation}: Maintain clear and readable code through meaningful variable names and comprehensive comments to facilitate understanding and maintenance.
    \index{Code Readability}
    
    \item \textbf{Edge Case Handling}: Implement checks for edge cases to prevent incorrect results or runtime errors.
    \index{Edge Case Handling}
    
    \item \textbf{Testing and Validation}: Develop a comprehensive suite of test cases that cover all possible scenarios, including edge cases, to validate the correctness and efficiency of the implementation.
    \index{Testing and Validation}
    
    \item \textbf{Scalability}: Design the algorithm to scale efficiently with increasing input sizes, maintaining performance and resource utilization.
    \index{Scalability}
    
    \item \textbf{Using Helper Functions}: Consider creating helper functions for repetitive tasks, such as calculating overlap dimensions, to enhance modularity and reusability.
    \index{Helper Functions}
    
    \item \textbf{Consistent Naming Conventions}: Use consistent and descriptive naming conventions for variables to improve code clarity.
    \index{Naming Conventions}
    
    \item \textbf{Implementing Unit Tests}: Develop unit tests for each logical condition to ensure that all scenarios are correctly handled.
    \index{Unit Tests}
    
    \item \textbf{Error Handling}: Incorporate error handling to manage invalid inputs gracefully.
    \index{Error Handling}
\end{itemize}

\section*{Conclusion}

The \textbf{Rectangle Area} problem showcases the application of fundamental geometric principles and efficient algorithm design to compute spatial properties accurately. By systematically calculating individual areas and intelligently handling overlapping regions, the algorithm ensures precise results without redundant computations. Understanding and implementing such techniques not only enhances problem-solving skills but also provides a foundation for tackling more complex Computational Geometry challenges involving multiple geometric entities and intricate spatial relationships.

\printindex

% % filename: rectangle_overlap.tex

\problemsection{Rectangle Overlap}
\label{chap:Rectangle_Overlap}
\marginnote{\href{https://leetcode.com/problems/rectangle-overlap/}{[LeetCode Link]}\index{LeetCode}}
\marginnote{\href{https://www.geeksforgeeks.org/check-if-two-rectangles-overlap/}{[GeeksForGeeks Link]}\index{GeeksForGeeks}}
\marginnote{\href{https://www.interviewbit.com/problems/rectangle-overlap/}{[InterviewBit Link]}\index{InterviewBit}}
\marginnote{\href{https://app.codesignal.com/challenges/rectangle-overlap}{[CodeSignal Link]}\index{CodeSignal}}
\marginnote{\href{https://www.codewars.com/kata/rectangle-overlap/train/python}{[Codewars Link]}\index{Codewars}}

The \textbf{Rectangle Overlap} problem is a fundamental challenge in Computational Geometry that involves determining whether two axis-aligned rectangles overlap. This problem tests one's ability to understand geometric properties, implement conditional logic, and optimize for efficient computation. Mastery of this problem is essential for applications in computer graphics, collision detection, and spatial data analysis.

\section*{Problem Statement}

Given two axis-aligned rectangles in a 2D plane, determine if they overlap. Each rectangle is defined by its bottom-left and top-right coordinates.

A rectangle is represented as a list of four integers \([x1, y1, x2, y2]\), where \((x1, y1)\) are the coordinates of the bottom-left corner, and \((x2, y2)\) are the coordinates of the top-right corner.

\textbf{Function signature in Python:}
\begin{lstlisting}[language=Python]
def isRectangleOverlap(rec1: List[int], rec2: List[int]) -> bool:
\end{lstlisting}

\section*{Examples}

\textbf{Example 1:}

\begin{verbatim}
Input: rec1 = [0,0,2,2], rec2 = [1,1,3,3]
Output: True
Explanation: The rectangles overlap in the area defined by [1,1,2,2].
\end{verbatim}

\textbf{Example 2:}

\begin{verbatim}
Input: rec1 = [0,0,1,1], rec2 = [1,0,2,1]
Output: False
Explanation: The rectangles touch at the edge but do not overlap.
\end{verbatim}

\textbf{Example 3:}

\begin{verbatim}
Input: rec1 = [0,0,1,1], rec2 = [2,2,3,3]
Output: False
Explanation: The rectangles are completely separate.
\end{verbatim}

\textbf{Example 4:}

\begin{verbatim}
Input: rec1 = [0,0,5,5], rec2 = [3,3,7,7]
Output: True
Explanation: The rectangles overlap in the area defined by [3,3,5,5].
\end{verbatim}

\textbf{Example 5:}

\begin{verbatim}
Input: rec1 = [0,0,0,0], rec2 = [0,0,0,0]
Output: False
Explanation: Both rectangles are degenerate points.
\end{verbatim}

\textbf{Constraints:}

\begin{itemize}
    \item All coordinates are integers in the range \([-10^9, 10^9]\).
    \item For each rectangle, \(x1 < x2\) and \(y1 < y2\).
\end{itemize}

LeetCode link: \href{https://leetcode.com/problems/rectangle-overlap/}{Rectangle Overlap}\index{LeetCode}

\section*{Algorithmic Approach}

To determine whether two axis-aligned rectangles overlap, we can use the following logical conditions:

1. **Non-Overlap Conditions:**
   - One rectangle is to the left of the other.
   - One rectangle is above the other.

2. **Overlap Condition:**
   - If neither of the non-overlap conditions is true, the rectangles must overlap.

\subsection*{Steps:}

1. **Extract Coordinates:**
   - For both rectangles, extract the bottom-left and top-right coordinates.

2. **Check Non-Overlap Conditions:**
   - If the right side of the first rectangle is less than or equal to the left side of the second rectangle, they do not overlap.
   - If the left side of the first rectangle is greater than or equal to the right side of the second rectangle, they do not overlap.
   - If the top side of the first rectangle is less than or equal to the bottom side of the second rectangle, they do not overlap.
   - If the bottom side of the first rectangle is greater than or equal to the top side of the second rectangle, they do not overlap.

3. **Determine Overlap:**
   - If none of the non-overlap conditions are met, the rectangles overlap.

\marginnote{This approach provides an efficient \(O(1)\) time complexity solution by leveraging simple geometric comparisons.}

\section*{Complexities}

\begin{itemize}
    \item \textbf{Time Complexity:} \(O(1)\). The algorithm performs a constant number of comparisons regardless of input size.
    
    \item \textbf{Space Complexity:} \(O(1)\). Only a fixed amount of extra space is used for variables.
\end{itemize}

\section*{Python Implementation}

\marginnote{Implementing the overlap check using coordinate comparisons ensures an optimal and straightforward solution.}

Below is the complete Python code implementing the \texttt{isRectangleOverlap} function:

\begin{fullwidth}
\begin{lstlisting}[language=Python]
from typing import List

class Solution:
    def isRectangleOverlap(self, rec1: List[int], rec2: List[int]) -> bool:
        # Extract coordinates
        left1, bottom1, right1, top1 = rec1
        left2, bottom2, right2, top2 = rec2
        
        # Check non-overlapping conditions
        if right1 <= left2 or right2 <= left1:
            return False
        if top1 <= bottom2 or top2 <= bottom1:
            return False
        
        # If none of the above, rectangles overlap
        return True

# Example usage:
solution = Solution()
print(solution.isRectangleOverlap([0,0,2,2], [1,1,3,3]))  # Output: True
print(solution.isRectangleOverlap([0,0,1,1], [1,0,2,1]))  # Output: False
print(solution.isRectangleOverlap([0,0,1,1], [2,2,3,3]))  # Output: False
print(solution.isRectangleOverlap([0,0,5,5], [3,3,7,7]))  # Output: True
print(solution.isRectangleOverlap([0,0,0,0], [0,0,0,0]))  # Output: False
\end{lstlisting}
\end{fullwidth}

This implementation efficiently checks for overlap by comparing the coordinates of the two rectangles. If any of the non-overlapping conditions are met, it returns \texttt{False}; otherwise, it returns \texttt{True}.

\section*{Explanation}

The \texttt{isRectangleOverlap} function determines whether two axis-aligned rectangles overlap by comparing their respective coordinates. Here's a detailed breakdown of the implementation:

\subsection*{1. Extract Coordinates}

\begin{itemize}
    \item For each rectangle, extract the left (\(x1\)), bottom (\(y1\)), right (\(x2\)), and top (\(y2\)) coordinates.
    \item This simplifies the comparison process by providing clear variables representing each side of the rectangles.
\end{itemize}

\subsection*{2. Check Non-Overlap Conditions}

\begin{itemize}
    \item **Horizontal Separation:**
    \begin{itemize}
        \item If the right side of the first rectangle (\(right1\)) is less than or equal to the left side of the second rectangle (\(left2\)), there is no horizontal overlap.
        \item Similarly, if the right side of the second rectangle (\(right2\)) is less than or equal to the left side of the first rectangle (\(left1\)), there is no horizontal overlap.
    \end{itemize}
    
    \item **Vertical Separation:**
    \begin{itemize}
        \item If the top side of the first rectangle (\(top1\)) is less than or equal to the bottom side of the second rectangle (\(bottom2\)), there is no vertical overlap.
        \item Similarly, if the top side of the second rectangle (\(top2\)) is less than or equal to the bottom side of the first rectangle (\(bottom1\)), there is no vertical overlap.
    \end{itemize}
    
    \item If any of these non-overlapping conditions are true, the rectangles do not overlap, and the function returns \texttt{False}.
\end{itemize}

\subsection*{3. Determine Overlap}

\begin{itemize}
    \item If none of the non-overlapping conditions are met, it implies that the rectangles overlap both horizontally and vertically.
    \item The function returns \texttt{True} in this case.
\end{itemize}

\subsection*{4. Example Walkthrough}

Consider the first example:
\begin{verbatim}
Input: rec1 = [0,0,2,2], rec2 = [1,1,3,3]
Output: True
\end{verbatim}

\begin{enumerate}
    \item Extract coordinates:
    \begin{itemize}
        \item rec1: left1 = 0, bottom1 = 0, right1 = 2, top1 = 2
        \item rec2: left2 = 1, bottom2 = 1, right2 = 3, top2 = 3
    \end{itemize}
    
    \item Check non-overlap conditions:
    \begin{itemize}
        \item \(right1 = 2\) is not less than or equal to \(left2 = 1\)
        \item \(right2 = 3\) is not less than or equal to \(left1 = 0\)
        \item \(top1 = 2\) is not less than or equal to \(bottom2 = 1\)
        \item \(top2 = 3\) is not less than or equal to \(bottom1 = 0\)
    \end{itemize}
    
    \item Since none of the non-overlapping conditions are met, the rectangles overlap.
\end{enumerate}

Thus, the function correctly returns \texttt{True}.

\section*{Why This Approach}

This approach is chosen for its simplicity and efficiency. By leveraging direct coordinate comparisons, the algorithm achieves constant time complexity without the need for complex data structures or iterative processes. It effectively handles all possible scenarios of rectangle positioning, ensuring accurate detection of overlaps.

\section*{Alternative Approaches}

\subsection*{1. Separating Axis Theorem (SAT)}

The Separating Axis Theorem is a more generalized method for detecting overlaps between convex shapes. While it is not necessary for axis-aligned rectangles, understanding SAT can be beneficial for more complex geometric problems.

\begin{lstlisting}[language=Python]
def isRectangleOverlap(rec1: List[int], rec2: List[int]) -> bool:
    # Using SAT for axis-aligned rectangles
    return not (rec1[2] <= rec2[0] or rec1[0] >= rec2[2] or
                rec1[3] <= rec2[1] or rec1[1] >= rec2[3])
\end{lstlisting}

\textbf{Note}: This implementation is functionally identical to the primary approach but leverages a more generalized geometric theorem.

\subsection*{2. Area-Based Approach}

Calculate the overlapping area between the two rectangles. If the overlapping area is positive, the rectangles overlap.

\begin{lstlisting}[language=Python]
def isRectangleOverlap(rec1: List[int], rec2: List[int]) -> bool:
    # Calculate overlap in x and y dimensions
    x_overlap = min(rec1[2], rec2[2]) - max(rec1[0], rec2[0])
    y_overlap = min(rec1[3], rec2[3]) - max(rec1[1], rec2[1])
    
    # Overlap exists if both overlaps are positive
    return x_overlap > 0 and y_overlap > 0
\end{lstlisting}

\textbf{Complexities:}
\begin{itemize}
    \item \textbf{Time Complexity:} \(O(1)\)
    \item \textbf{Space Complexity:} \(O(1)\)
\end{itemize}

\subsection*{3. Using Rectangles Intersection Function}

Utilize built-in or library functions that handle geometric intersections.

\begin{lstlisting}[language=Python]
from shapely.geometry import box

def isRectangleOverlap(rec1: List[int], rec2: List[int]) -> bool:
    rectangle1 = box(rec1[0], rec1[1], rec1[2], rec1[3])
    rectangle2 = box(rec2[0], rec2[1], rec2[2], rec2[3])
    return rectangle1.intersects(rectangle2) and not rectangle1.touches(rectangle2)
\end{lstlisting}

\textbf{Note}: This approach requires the \texttt{shapely} library and is more suitable for complex geometric operations.

\section*{Similar Problems to This One}

Several problems revolve around geometric overlap, intersection detection, and spatial reasoning, utilizing similar algorithmic strategies:

\begin{itemize}
    \item \textbf{Interval Overlap}: Determine if two intervals on a line overlap.
    \item \textbf{Circle Overlap}: Determine if two circles overlap based on their radii and centers.
    \item \textbf{Polygon Overlap}: Determine if two polygons overlap using algorithms like SAT.
    \item \textbf{Closest Pair of Points}: Find the closest pair of points in a set.
    \item \textbf{Convex Hull}: Compute the convex hull of a set of points.
    \item \textbf{Intersection of Lines}: Find the intersection point of two lines.
    \item \textbf{Point Inside Polygon}: Determine if a point lies inside a given polygon.
\end{itemize}

These problems reinforce the concepts of spatial reasoning, geometric property analysis, and efficient algorithm design in various contexts.

\section*{Things to Keep in Mind and Tricks}

When working with the \textbf{Rectangle Overlap} problem, consider the following tips and best practices to enhance efficiency and correctness:

\begin{itemize}
    \item \textbf{Understand Geometric Relationships}: Grasp the positional relationships between rectangles to simplify overlap detection.
    \index{Geometric Relationships}
    
    \item \textbf{Leverage Coordinate Comparisons}: Use direct comparisons of rectangle coordinates to determine spatial relationships.
    \index{Coordinate Comparisons}
    
    \item \textbf{Handle Edge Cases}: Consider cases where rectangles touch at edges or corners without overlapping.
    \index{Edge Cases}
    
    \item \textbf{Optimize for Efficiency}: Aim for a constant time \(O(1)\) solution by avoiding unnecessary computations or iterations.
    \index{Efficiency Optimization}
    
    \item \textbf{Avoid Floating-Point Precision Issues}: Since all coordinates are integers, floating-point precision is not a concern, simplifying the implementation.
    \index{Floating-Point Precision}
    
    \item \textbf{Use Helper Functions}: Create helper functions to encapsulate repetitive tasks, such as extracting coordinates or checking specific conditions.
    \index{Helper Functions}
    
    \item \textbf{Code Readability}: Maintain clear and readable code through meaningful variable names and structured logic.
    \index{Code Readability}
    
    \item \textbf{Test Extensively}: Implement a wide range of test cases, including overlapping, non-overlapping, and edge-touching rectangles, to ensure robustness.
    \index{Extensive Testing}
    
    \item \textbf{Understand Axis-Aligned Constraints}: Recognize that axis-aligned rectangles simplify overlap detection compared to rotated rectangles.
    \index{Axis-Aligned Constraints}
    
    \item \textbf{Simplify Logical Conditions}: Combine multiple conditions logically to streamline the overlap detection process.
    \index{Logical Conditions}
\end{itemize}

\section*{Corner and Special Cases to Test When Writing the Code}

When implementing the solution for the \textbf{Rectangle Overlap} problem, it is crucial to consider and rigorously test various edge cases to ensure robustness and correctness:

\begin{itemize}
    \item \textbf{No Overlap}: Rectangles are completely separate.
    \index{No Overlap}
    
    \item \textbf{Partial Overlap}: Rectangles overlap in one or more regions.
    \index{Partial Overlap}
    
    \item \textbf{Edge Touching}: Rectangles touch exactly at one edge without overlapping.
    \index{Edge Touching}
    
    \item \textbf{Corner Touching}: Rectangles touch exactly at one corner without overlapping.
    \index{Corner Touching}
    
    \item \textbf{One Rectangle Inside Another}: One rectangle is entirely within the other.
    \index{Rectangle Inside}
    
    \item \textbf{Identical Rectangles}: Both rectangles have the same coordinates.
    \index{Identical Rectangles}
    
    \item \textbf{Degenerate Rectangles}: Rectangles with zero area (e.g., \(x1 = x2\) or \(y1 = y2\)).
    \index{Degenerate Rectangles}
    
    \item \textbf{Large Coordinates}: Rectangles with very large coordinate values to test performance and integer handling.
    \index{Large Coordinates}
    
    \item \textbf{Negative Coordinates}: Rectangles positioned in negative coordinate space.
    \index{Negative Coordinates}
    
    \item \textbf{Mixed Overlapping Scenarios}: Combinations of the above cases to ensure comprehensive coverage.
    \index{Mixed Overlapping Scenarios}
    
    \item \textbf{Minimum and Maximum Bounds}: Rectangles at the minimum and maximum limits of the coordinate range.
    \index{Minimum and Maximum Bounds}
\end{itemize}

\section*{Implementation Considerations}

When implementing the \texttt{isRectangleOverlap} function, keep in mind the following considerations to ensure robustness and efficiency:

\begin{itemize}
    \item \textbf{Data Type Selection}: Use appropriate data types that can handle the range of input values without overflow or underflow.
    \index{Data Type Selection}
    
    \item \textbf{Optimizing Comparisons}: Structure logical conditions to short-circuit evaluations as soon as a non-overlapping condition is met.
    \index{Optimizing Comparisons}
    
    \item \textbf{Language-Specific Constraints}: Be aware of how the programming language handles integer division and comparisons.
    \index{Language-Specific Constraints}
    
    \item \textbf{Avoiding Redundant Calculations}: Ensure that each comparison contributes towards determining overlap without unnecessary repetitions.
    \index{Avoiding Redundant Calculations}
    
    \item \textbf{Code Readability and Documentation}: Maintain clear and readable code through meaningful variable names and comprehensive comments to facilitate understanding and maintenance.
    \index{Code Readability}
    
    \item \textbf{Edge Case Handling}: Implement checks for edge cases to prevent incorrect results or runtime errors.
    \index{Edge Case Handling}
    
    \item \textbf{Testing and Validation}: Develop a comprehensive suite of test cases that cover all possible scenarios, including edge cases, to validate the correctness and efficiency of the implementation.
    \index{Testing and Validation}
    
    \item \textbf{Scalability}: Design the algorithm to scale efficiently with increasing input sizes, maintaining performance and resource utilization.
    \index{Scalability}
    
    \item \textbf{Using Helper Functions}: Consider creating helper functions for repetitive tasks, such as extracting and comparing coordinates, to enhance modularity and reusability.
    \index{Helper Functions}
    
    \item \textbf{Consistent Naming Conventions}: Use consistent and descriptive naming conventions for variables to improve code clarity.
    \index{Naming Conventions}
    
    \item \textbf{Handling Floating-Point Coordinates}: Although the problem specifies integer coordinates, ensure that the implementation can handle floating-point numbers if needed in extended scenarios.
    \index{Floating-Point Coordinates}
    
    \item \textbf{Avoiding Floating-Point Precision Issues}: Since all coordinates are integers, floating-point precision is not a concern, simplifying the implementation.
    \index{Floating-Point Precision}
    
    \item \textbf{Implementing Unit Tests}: Develop unit tests for each logical condition to ensure that all scenarios are correctly handled.
    \index{Unit Tests}
    
    \item \textbf{Error Handling}: Incorporate error handling to manage invalid inputs gracefully.
    \index{Error Handling}
\end{itemize}

\section*{Conclusion}

The \textbf{Rectangle Overlap} problem exemplifies the application of fundamental geometric principles and conditional logic to solve spatial challenges efficiently. By leveraging simple coordinate comparisons, the algorithm achieves optimal time and space complexities, making it highly suitable for real-time applications such as collision detection in gaming, layout planning in graphics, and spatial data analysis. Understanding and implementing such techniques not only enhances problem-solving skills but also provides a foundation for tackling more complex Computational Geometry problems involving varied geometric shapes and interactions.

\printindex

% % filename: rectangle_overlap.tex

\problemsection{Rectangle Overlap}
\label{chap:Rectangle_Overlap}
\marginnote{\href{https://leetcode.com/problems/rectangle-overlap/}{[LeetCode Link]}\index{LeetCode}}
\marginnote{\href{https://www.geeksforgeeks.org/check-if-two-rectangles-overlap/}{[GeeksForGeeks Link]}\index{GeeksForGeeks}}
\marginnote{\href{https://www.interviewbit.com/problems/rectangle-overlap/}{[InterviewBit Link]}\index{InterviewBit}}
\marginnote{\href{https://app.codesignal.com/challenges/rectangle-overlap}{[CodeSignal Link]}\index{CodeSignal}}
\marginnote{\href{https://www.codewars.com/kata/rectangle-overlap/train/python}{[Codewars Link]}\index{Codewars}}

The \textbf{Rectangle Overlap} problem is a fundamental challenge in Computational Geometry that involves determining whether two axis-aligned rectangles overlap. This problem tests one's ability to understand geometric properties, implement conditional logic, and optimize for efficient computation. Mastery of this problem is essential for applications in computer graphics, collision detection, and spatial data analysis.

\section*{Problem Statement}

Given two axis-aligned rectangles in a 2D plane, determine if they overlap. Each rectangle is defined by its bottom-left and top-right coordinates.

A rectangle is represented as a list of four integers \([x1, y1, x2, y2]\), where \((x1, y1)\) are the coordinates of the bottom-left corner, and \((x2, y2)\) are the coordinates of the top-right corner.

\textbf{Function signature in Python:}
\begin{lstlisting}[language=Python]
def isRectangleOverlap(rec1: List[int], rec2: List[int]) -> bool:
\end{lstlisting}

\section*{Examples}

\textbf{Example 1:}

\begin{verbatim}
Input: rec1 = [0,0,2,2], rec2 = [1,1,3,3]
Output: True
Explanation: The rectangles overlap in the area defined by [1,1,2,2].
\end{verbatim}

\textbf{Example 2:}

\begin{verbatim}
Input: rec1 = [0,0,1,1], rec2 = [1,0,2,1]
Output: False
Explanation: The rectangles touch at the edge but do not overlap.
\end{verbatim}

\textbf{Example 3:}

\begin{verbatim}
Input: rec1 = [0,0,1,1], rec2 = [2,2,3,3]
Output: False
Explanation: The rectangles are completely separate.
\end{verbatim}

\textbf{Example 4:}

\begin{verbatim}
Input: rec1 = [0,0,5,5], rec2 = [3,3,7,7]
Output: True
Explanation: The rectangles overlap in the area defined by [3,3,5,5].
\end{verbatim}

\textbf{Example 5:}

\begin{verbatim}
Input: rec1 = [0,0,0,0], rec2 = [0,0,0,0]
Output: False
Explanation: Both rectangles are degenerate points.
\end{verbatim}

\textbf{Constraints:}

\begin{itemize}
    \item All coordinates are integers in the range \([-10^9, 10^9]\).
    \item For each rectangle, \(x1 < x2\) and \(y1 < y2\).
\end{itemize}

LeetCode link: \href{https://leetcode.com/problems/rectangle-overlap/}{Rectangle Overlap}\index{LeetCode}

\section*{Algorithmic Approach}

To determine whether two axis-aligned rectangles overlap, we can use the following logical conditions:

1. **Non-Overlap Conditions:**
   - One rectangle is to the left of the other.
   - One rectangle is above the other.

2. **Overlap Condition:**
   - If neither of the non-overlap conditions is true, the rectangles must overlap.

\subsection*{Steps:}

1. **Extract Coordinates:**
   - For both rectangles, extract the bottom-left and top-right coordinates.

2. **Check Non-Overlap Conditions:**
   - If the right side of the first rectangle is less than or equal to the left side of the second rectangle, they do not overlap.
   - If the left side of the first rectangle is greater than or equal to the right side of the second rectangle, they do not overlap.
   - If the top side of the first rectangle is less than or equal to the bottom side of the second rectangle, they do not overlap.
   - If the bottom side of the first rectangle is greater than or equal to the top side of the second rectangle, they do not overlap.

3. **Determine Overlap:**
   - If none of the non-overlap conditions are met, the rectangles overlap.

\marginnote{This approach provides an efficient \(O(1)\) time complexity solution by leveraging simple geometric comparisons.}

\section*{Complexities}

\begin{itemize}
    \item \textbf{Time Complexity:} \(O(1)\). The algorithm performs a constant number of comparisons regardless of input size.
    
    \item \textbf{Space Complexity:} \(O(1)\). Only a fixed amount of extra space is used for variables.
\end{itemize}

\section*{Python Implementation}

\marginnote{Implementing the overlap check using coordinate comparisons ensures an optimal and straightforward solution.}

Below is the complete Python code implementing the \texttt{isRectangleOverlap} function:

\begin{fullwidth}
\begin{lstlisting}[language=Python]
from typing import List

class Solution:
    def isRectangleOverlap(self, rec1: List[int], rec2: List[int]) -> bool:
        # Extract coordinates
        left1, bottom1, right1, top1 = rec1
        left2, bottom2, right2, top2 = rec2
        
        # Check non-overlapping conditions
        if right1 <= left2 or right2 <= left1:
            return False
        if top1 <= bottom2 or top2 <= bottom1:
            return False
        
        # If none of the above, rectangles overlap
        return True

# Example usage:
solution = Solution()
print(solution.isRectangleOverlap([0,0,2,2], [1,1,3,3]))  # Output: True
print(solution.isRectangleOverlap([0,0,1,1], [1,0,2,1]))  # Output: False
print(solution.isRectangleOverlap([0,0,1,1], [2,2,3,3]))  # Output: False
print(solution.isRectangleOverlap([0,0,5,5], [3,3,7,7]))  # Output: True
print(solution.isRectangleOverlap([0,0,0,0], [0,0,0,0]))  # Output: False
\end{lstlisting}
\end{fullwidth}

This implementation efficiently checks for overlap by comparing the coordinates of the two rectangles. If any of the non-overlapping conditions are met, it returns \texttt{False}; otherwise, it returns \texttt{True}.

\section*{Explanation}

The \texttt{isRectangleOverlap} function determines whether two axis-aligned rectangles overlap by comparing their respective coordinates. Here's a detailed breakdown of the implementation:

\subsection*{1. Extract Coordinates}

\begin{itemize}
    \item For each rectangle, extract the left (\(x1\)), bottom (\(y1\)), right (\(x2\)), and top (\(y2\)) coordinates.
    \item This simplifies the comparison process by providing clear variables representing each side of the rectangles.
\end{itemize}

\subsection*{2. Check Non-Overlap Conditions}

\begin{itemize}
    \item **Horizontal Separation:**
    \begin{itemize}
        \item If the right side of the first rectangle (\(right1\)) is less than or equal to the left side of the second rectangle (\(left2\)), there is no horizontal overlap.
        \item Similarly, if the right side of the second rectangle (\(right2\)) is less than or equal to the left side of the first rectangle (\(left1\)), there is no horizontal overlap.
    \end{itemize}
    
    \item **Vertical Separation:**
    \begin{itemize}
        \item If the top side of the first rectangle (\(top1\)) is less than or equal to the bottom side of the second rectangle (\(bottom2\)), there is no vertical overlap.
        \item Similarly, if the top side of the second rectangle (\(top2\)) is less than or equal to the bottom side of the first rectangle (\(bottom1\)), there is no vertical overlap.
    \end{itemize}
    
    \item If any of these non-overlapping conditions are true, the rectangles do not overlap, and the function returns \texttt{False}.
\end{itemize}

\subsection*{3. Determine Overlap}

\begin{itemize}
    \item If none of the non-overlapping conditions are met, it implies that the rectangles overlap both horizontally and vertically.
    \item The function returns \texttt{True} in this case.
\end{itemize}

\subsection*{4. Example Walkthrough}

Consider the first example:
\begin{verbatim}
Input: rec1 = [0,0,2,2], rec2 = [1,1,3,3]
Output: True
\end{verbatim}

\begin{enumerate}
    \item Extract coordinates:
    \begin{itemize}
        \item rec1: left1 = 0, bottom1 = 0, right1 = 2, top1 = 2
        \item rec2: left2 = 1, bottom2 = 1, right2 = 3, top2 = 3
    \end{itemize}
    
    \item Check non-overlap conditions:
    \begin{itemize}
        \item \(right1 = 2\) is not less than or equal to \(left2 = 1\)
        \item \(right2 = 3\) is not less than or equal to \(left1 = 0\)
        \item \(top1 = 2\) is not less than or equal to \(bottom2 = 1\)
        \item \(top2 = 3\) is not less than or equal to \(bottom1 = 0\)
    \end{itemize}
    
    \item Since none of the non-overlapping conditions are met, the rectangles overlap.
\end{enumerate}

Thus, the function correctly returns \texttt{True}.

\section*{Why This Approach}

This approach is chosen for its simplicity and efficiency. By leveraging direct coordinate comparisons, the algorithm achieves constant time complexity without the need for complex data structures or iterative processes. It effectively handles all possible scenarios of rectangle positioning, ensuring accurate detection of overlaps.

\section*{Alternative Approaches}

\subsection*{1. Separating Axis Theorem (SAT)}

The Separating Axis Theorem is a more generalized method for detecting overlaps between convex shapes. While it is not necessary for axis-aligned rectangles, understanding SAT can be beneficial for more complex geometric problems.

\begin{lstlisting}[language=Python]
def isRectangleOverlap(rec1: List[int], rec2: List[int]) -> bool:
    # Using SAT for axis-aligned rectangles
    return not (rec1[2] <= rec2[0] or rec1[0] >= rec2[2] or
                rec1[3] <= rec2[1] or rec1[1] >= rec2[3])
\end{lstlisting}

\textbf{Note}: This implementation is functionally identical to the primary approach but leverages a more generalized geometric theorem.

\subsection*{2. Area-Based Approach}

Calculate the overlapping area between the two rectangles. If the overlapping area is positive, the rectangles overlap.

\begin{lstlisting}[language=Python]
def isRectangleOverlap(rec1: List[int], rec2: List[int]) -> bool:
    # Calculate overlap in x and y dimensions
    x_overlap = min(rec1[2], rec2[2]) - max(rec1[0], rec2[0])
    y_overlap = min(rec1[3], rec2[3]) - max(rec1[1], rec2[1])
    
    # Overlap exists if both overlaps are positive
    return x_overlap > 0 and y_overlap > 0
\end{lstlisting}

\textbf{Complexities:}
\begin{itemize}
    \item \textbf{Time Complexity:} \(O(1)\)
    \item \textbf{Space Complexity:} \(O(1)\)
\end{itemize}

\subsection*{3. Using Rectangles Intersection Function}

Utilize built-in or library functions that handle geometric intersections.

\begin{lstlisting}[language=Python]
from shapely.geometry import box

def isRectangleOverlap(rec1: List[int], rec2: List[int]) -> bool:
    rectangle1 = box(rec1[0], rec1[1], rec1[2], rec1[3])
    rectangle2 = box(rec2[0], rec2[1], rec2[2], rec2[3])
    return rectangle1.intersects(rectangle2) and not rectangle1.touches(rectangle2)
\end{lstlisting}

\textbf{Note}: This approach requires the \texttt{shapely} library and is more suitable for complex geometric operations.

\section*{Similar Problems to This One}

Several problems revolve around geometric overlap, intersection detection, and spatial reasoning, utilizing similar algorithmic strategies:

\begin{itemize}
    \item \textbf{Interval Overlap}: Determine if two intervals on a line overlap.
    \item \textbf{Circle Overlap}: Determine if two circles overlap based on their radii and centers.
    \item \textbf{Polygon Overlap}: Determine if two polygons overlap using algorithms like SAT.
    \item \textbf{Closest Pair of Points}: Find the closest pair of points in a set.
    \item \textbf{Convex Hull}: Compute the convex hull of a set of points.
    \item \textbf{Intersection of Lines}: Find the intersection point of two lines.
    \item \textbf{Point Inside Polygon}: Determine if a point lies inside a given polygon.
\end{itemize}

These problems reinforce the concepts of spatial reasoning, geometric property analysis, and efficient algorithm design in various contexts.

\section*{Things to Keep in Mind and Tricks}

When working with the \textbf{Rectangle Overlap} problem, consider the following tips and best practices to enhance efficiency and correctness:

\begin{itemize}
    \item \textbf{Understand Geometric Relationships}: Grasp the positional relationships between rectangles to simplify overlap detection.
    \index{Geometric Relationships}
    
    \item \textbf{Leverage Coordinate Comparisons}: Use direct comparisons of rectangle coordinates to determine spatial relationships.
    \index{Coordinate Comparisons}
    
    \item \textbf{Handle Edge Cases}: Consider cases where rectangles touch at edges or corners without overlapping.
    \index{Edge Cases}
    
    \item \textbf{Optimize for Efficiency}: Aim for a constant time \(O(1)\) solution by avoiding unnecessary computations or iterations.
    \index{Efficiency Optimization}
    
    \item \textbf{Avoid Floating-Point Precision Issues}: Since all coordinates are integers, floating-point precision is not a concern, simplifying the implementation.
    \index{Floating-Point Precision}
    
    \item \textbf{Use Helper Functions}: Create helper functions to encapsulate repetitive tasks, such as extracting coordinates or checking specific conditions.
    \index{Helper Functions}
    
    \item \textbf{Code Readability}: Maintain clear and readable code through meaningful variable names and structured logic.
    \index{Code Readability}
    
    \item \textbf{Test Extensively}: Implement a wide range of test cases, including overlapping, non-overlapping, and edge-touching rectangles, to ensure robustness.
    \index{Extensive Testing}
    
    \item \textbf{Understand Axis-Aligned Constraints}: Recognize that axis-aligned rectangles simplify overlap detection compared to rotated rectangles.
    \index{Axis-Aligned Constraints}
    
    \item \textbf{Simplify Logical Conditions}: Combine multiple conditions logically to streamline the overlap detection process.
    \index{Logical Conditions}
\end{itemize}

\section*{Corner and Special Cases to Test When Writing the Code}

When implementing the solution for the \textbf{Rectangle Overlap} problem, it is crucial to consider and rigorously test various edge cases to ensure robustness and correctness:

\begin{itemize}
    \item \textbf{No Overlap}: Rectangles are completely separate.
    \index{No Overlap}
    
    \item \textbf{Partial Overlap}: Rectangles overlap in one or more regions.
    \index{Partial Overlap}
    
    \item \textbf{Edge Touching}: Rectangles touch exactly at one edge without overlapping.
    \index{Edge Touching}
    
    \item \textbf{Corner Touching}: Rectangles touch exactly at one corner without overlapping.
    \index{Corner Touching}
    
    \item \textbf{One Rectangle Inside Another}: One rectangle is entirely within the other.
    \index{Rectangle Inside}
    
    \item \textbf{Identical Rectangles}: Both rectangles have the same coordinates.
    \index{Identical Rectangles}
    
    \item \textbf{Degenerate Rectangles}: Rectangles with zero area (e.g., \(x1 = x2\) or \(y1 = y2\)).
    \index{Degenerate Rectangles}
    
    \item \textbf{Large Coordinates}: Rectangles with very large coordinate values to test performance and integer handling.
    \index{Large Coordinates}
    
    \item \textbf{Negative Coordinates}: Rectangles positioned in negative coordinate space.
    \index{Negative Coordinates}
    
    \item \textbf{Mixed Overlapping Scenarios}: Combinations of the above cases to ensure comprehensive coverage.
    \index{Mixed Overlapping Scenarios}
    
    \item \textbf{Minimum and Maximum Bounds}: Rectangles at the minimum and maximum limits of the coordinate range.
    \index{Minimum and Maximum Bounds}
\end{itemize}

\section*{Implementation Considerations}

When implementing the \texttt{isRectangleOverlap} function, keep in mind the following considerations to ensure robustness and efficiency:

\begin{itemize}
    \item \textbf{Data Type Selection}: Use appropriate data types that can handle the range of input values without overflow or underflow.
    \index{Data Type Selection}
    
    \item \textbf{Optimizing Comparisons}: Structure logical conditions to short-circuit evaluations as soon as a non-overlapping condition is met.
    \index{Optimizing Comparisons}
    
    \item \textbf{Language-Specific Constraints}: Be aware of how the programming language handles integer division and comparisons.
    \index{Language-Specific Constraints}
    
    \item \textbf{Avoiding Redundant Calculations}: Ensure that each comparison contributes towards determining overlap without unnecessary repetitions.
    \index{Avoiding Redundant Calculations}
    
    \item \textbf{Code Readability and Documentation}: Maintain clear and readable code through meaningful variable names and comprehensive comments to facilitate understanding and maintenance.
    \index{Code Readability}
    
    \item \textbf{Edge Case Handling}: Implement checks for edge cases to prevent incorrect results or runtime errors.
    \index{Edge Case Handling}
    
    \item \textbf{Testing and Validation}: Develop a comprehensive suite of test cases that cover all possible scenarios, including edge cases, to validate the correctness and efficiency of the implementation.
    \index{Testing and Validation}
    
    \item \textbf{Scalability}: Design the algorithm to scale efficiently with increasing input sizes, maintaining performance and resource utilization.
    \index{Scalability}
    
    \item \textbf{Using Helper Functions}: Consider creating helper functions for repetitive tasks, such as extracting and comparing coordinates, to enhance modularity and reusability.
    \index{Helper Functions}
    
    \item \textbf{Consistent Naming Conventions}: Use consistent and descriptive naming conventions for variables to improve code clarity.
    \index{Naming Conventions}
    
    \item \textbf{Handling Floating-Point Coordinates}: Although the problem specifies integer coordinates, ensure that the implementation can handle floating-point numbers if needed in extended scenarios.
    \index{Floating-Point Coordinates}
    
    \item \textbf{Avoiding Floating-Point Precision Issues}: Since all coordinates are integers, floating-point precision is not a concern, simplifying the implementation.
    \index{Floating-Point Precision}
    
    \item \textbf{Implementing Unit Tests}: Develop unit tests for each logical condition to ensure that all scenarios are correctly handled.
    \index{Unit Tests}
    
    \item \textbf{Error Handling}: Incorporate error handling to manage invalid inputs gracefully.
    \index{Error Handling}
\end{itemize}

\section*{Conclusion}

The \textbf{Rectangle Overlap} problem exemplifies the application of fundamental geometric principles and conditional logic to solve spatial challenges efficiently. By leveraging simple coordinate comparisons, the algorithm achieves optimal time and space complexities, making it highly suitable for real-time applications such as collision detection in gaming, layout planning in graphics, and spatial data analysis. Understanding and implementing such techniques not only enhances problem-solving skills but also provides a foundation for tackling more complex Computational Geometry problems involving varied geometric shapes and interactions.

\printindex

% \input{sections/rectangle_overlap}
% \input{sections/rectangle_area}
% \input{sections/k_closest_points_to_origin}
% \input{sections/the_skyline_problem}
% % filename: rectangle_area.tex

\problemsection{Rectangle Area}
\label{chap:Rectangle_Area}
\marginnote{\href{https://leetcode.com/problems/rectangle-area/}{[LeetCode Link]}\index{LeetCode}}
\marginnote{\href{https://www.geeksforgeeks.org/find-area-two-overlapping-rectangles/}{[GeeksForGeeks Link]}\index{GeeksForGeeks}}
\marginnote{\href{https://www.interviewbit.com/problems/rectangle-area/}{[InterviewBit Link]}\index{InterviewBit}}
\marginnote{\href{https://app.codesignal.com/challenges/rectangle-area}{[CodeSignal Link]}\index{CodeSignal}}
\marginnote{\href{https://www.codewars.com/kata/rectangle-area/train/python}{[Codewars Link]}\index{Codewars}}

The \textbf{Rectangle Area} problem is a classic Computational Geometry challenge that involves calculating the total area covered by two axis-aligned rectangles in a 2D plane. This problem tests one's ability to perform geometric calculations, handle overlapping scenarios, and implement efficient algorithms. Mastery of this problem is essential for applications in computer graphics, spatial analysis, and computational modeling.

\section*{Problem Statement}

Given two axis-aligned rectangles in a 2D plane, compute the total area covered by the two rectangles. The area covered by the overlapping region should be counted only once.

Each rectangle is represented as a list of four integers \([x1, y1, x2, y2]\), where \((x1, y1)\) are the coordinates of the bottom-left corner, and \((x2, y2)\) are the coordinates of the top-right corner.

\textbf{Function signature in Python:}
\begin{lstlisting}[language=Python]
def computeArea(A: List[int], B: List[int]) -> int:
\end{lstlisting}

\section*{Examples}

\textbf{Example 1:}

\begin{verbatim}
Input: A = [-3,0,3,4], B = [0,-1,9,2]
Output: 45
Explanation:
Area of A = (3 - (-3)) * (4 - 0) = 6 * 4 = 24
Area of B = (9 - 0) * (2 - (-1)) = 9 * 3 = 27
Overlapping Area = (3 - 0) * (2 - 0) = 3 * 2 = 6
Total Area = 24 + 27 - 6 = 45
\end{verbatim}

\textbf{Example 2:}

\begin{verbatim}
Input: A = [0,0,0,0], B = [0,0,0,0]
Output: 0
Explanation:
Both rectangles are degenerate points with zero area.
\end{verbatim}

\textbf{Example 3:}

\begin{verbatim}
Input: A = [0,0,2,2], B = [1,1,3,3]
Output: 7
Explanation:
Area of A = 4
Area of B = 4
Overlapping Area = 1
Total Area = 4 + 4 - 1 = 7
\end{verbatim}

\textbf{Example 4:}

\begin{verbatim}
Input: A = [0,0,1,1], B = [1,0,2,1]
Output: 2
Explanation:
Rectangles touch at the edge but do not overlap.
Area of A = 1
Area of B = 1
Overlapping Area = 0
Total Area = 1 + 1 = 2
\end{verbatim}

\textbf{Constraints:}

\begin{itemize}
    \item All coordinates are integers in the range \([-10^9, 10^9]\).
    \item For each rectangle, \(x1 < x2\) and \(y1 < y2\).
\end{itemize}

LeetCode link: \href{https://leetcode.com/problems/rectangle-area/}{Rectangle Area}\index{LeetCode}

\section*{Algorithmic Approach}

To compute the total area covered by two axis-aligned rectangles, we can follow these steps:

1. **Calculate Individual Areas:**
   - Compute the area of the first rectangle.
   - Compute the area of the second rectangle.

2. **Determine Overlapping Area:**
   - Calculate the coordinates of the overlapping rectangle, if any.
   - If the rectangles overlap, compute the area of the overlapping region.

3. **Compute Total Area:**
   - Sum the individual areas and subtract the overlapping area to avoid double-counting.

\marginnote{This approach ensures accurate area calculation by handling overlapping regions appropriately.}

\section*{Complexities}

\begin{itemize}
    \item \textbf{Time Complexity:} \(O(1)\). The algorithm performs a constant number of calculations.
    
    \item \textbf{Space Complexity:} \(O(1)\). Only a fixed amount of extra space is used for variables.
\end{itemize}

\section*{Python Implementation}

\marginnote{Implementing the area calculation with overlap consideration ensures an accurate and efficient solution.}

Below is the complete Python code implementing the \texttt{computeArea} function:

\begin{fullwidth}
\begin{lstlisting}[language=Python]
from typing import List

class Solution:
    def computeArea(self, A: List[int], B: List[int]) -> int:
        # Calculate area of rectangle A
        areaA = (A[2] - A[0]) * (A[3] - A[1])
        
        # Calculate area of rectangle B
        areaB = (B[2] - B[0]) * (B[3] - B[1])
        
        # Determine overlap coordinates
        overlap_x1 = max(A[0], B[0])
        overlap_y1 = max(A[1], B[1])
        overlap_x2 = min(A[2], B[2])
        overlap_y2 = min(A[3], B[3])
        
        # Calculate overlapping area
        overlap_width = overlap_x2 - overlap_x1
        overlap_height = overlap_y2 - overlap_y1
        overlap_area = 0
        if overlap_width > 0 and overlap_height > 0:
            overlap_area = overlap_width * overlap_height
        
        # Total area is sum of individual areas minus overlapping area
        total_area = areaA + areaB - overlap_area
        return total_area

# Example usage:
solution = Solution()
print(solution.computeArea([-3,0,3,4], [0,-1,9,2]))  # Output: 45
print(solution.computeArea([0,0,0,0], [0,0,0,0]))    # Output: 0
print(solution.computeArea([0,0,2,2], [1,1,3,3]))    # Output: 7
print(solution.computeArea([0,0,1,1], [1,0,2,1]))    # Output: 2
\end{lstlisting}
\end{fullwidth}

This implementation accurately computes the total area covered by two rectangles by accounting for any overlapping regions. It ensures that the overlapping area is not double-counted.

\section*{Explanation}

The \texttt{computeArea} function calculates the combined area of two axis-aligned rectangles by following these steps:

\subsection*{1. Calculate Individual Areas}

\begin{itemize}
    \item **Rectangle A:**
    \begin{itemize}
        \item Width: \(A[2] - A[0]\)
        \item Height: \(A[3] - A[1]\)
        \item Area: Width \(\times\) Height
    \end{itemize}
    
    \item **Rectangle B:**
    \begin{itemize}
        \item Width: \(B[2] - B[0]\)
        \item Height: \(B[3] - B[1]\)
        \item Area: Width \(\times\) Height
    \end{itemize}
\end{itemize}

\subsection*{2. Determine Overlapping Area}

\begin{itemize}
    \item **Overlap Coordinates:**
    \begin{itemize}
        \item Left (x-coordinate): \(\text{max}(A[0], B[0])\)
        \item Bottom (y-coordinate): \(\text{max}(A[1], B[1])\)
        \item Right (x-coordinate): \(\text{min}(A[2], B[2])\)
        \item Top (y-coordinate): \(\text{min}(A[3], B[3])\)
    \end{itemize}
    
    \item **Overlap Dimensions:**
    \begin{itemize}
        \item Width: \(\text{overlap\_x2} - \text{overlap\_x1}\)
        \item Height: \(\text{overlap\_y2} - \text{overlap\_y1}\)
    \end{itemize}
    
    \item **Overlap Area:**
    \begin{itemize}
        \item If both width and height are positive, the rectangles overlap, and the overlapping area is their product.
        \item Otherwise, there is no overlap, and the overlapping area is zero.
    \end{itemize}
\end{itemize}

\subsection*{3. Compute Total Area}

\begin{itemize}
    \item Total Area = Area of Rectangle A + Area of Rectangle B - Overlapping Area
\end{itemize}

\subsection*{4. Example Walkthrough}

Consider the first example:
\begin{verbatim}
Input: A = [-3,0,3,4], B = [0,-1,9,2]
Output: 45
\end{verbatim}

\begin{enumerate}
    \item **Calculate Areas:**
    \begin{itemize}
        \item Area of A = (3 - (-3)) * (4 - 0) = 6 * 4 = 24
        \item Area of B = (9 - 0) * (2 - (-1)) = 9 * 3 = 27
    \end{itemize}
    
    \item **Determine Overlap:**
    \begin{itemize}
        \item overlap\_x1 = max(-3, 0) = 0
        \item overlap\_y1 = max(0, -1) = 0
        \item overlap\_x2 = min(3, 9) = 3
        \item overlap\_y2 = min(4, 2) = 2
        \item overlap\_width = 3 - 0 = 3
        \item overlap\_height = 2 - 0 = 2
        \item overlap\_area = 3 * 2 = 6
    \end{itemize}
    
    \item **Compute Total Area:**
    \begin{itemize}
        \item Total Area = 24 + 27 - 6 = 45
    \end{itemize}
\end{enumerate}

Thus, the function correctly returns \texttt{45}.

\section*{Why This Approach}

This approach is chosen for its straightforwardness and optimal efficiency. By directly calculating the individual areas and intelligently handling the overlapping region, the algorithm ensures accurate results without unnecessary computations. Its constant time complexity makes it highly efficient, even for large coordinate values.

\section*{Alternative Approaches}

\subsection*{1. Using Intersection Dimensions}

Instead of separately calculating areas, directly compute the dimensions of the overlapping region and subtract it from the sum of individual areas.

\begin{lstlisting}[language=Python]
def computeArea(A: List[int], B: List[int]) -> int:
    # Sum of individual areas
    area = (A[2] - A[0]) * (A[3] - A[1]) + (B[2] - B[0]) * (B[3] - B[1])
    
    # Overlapping area
    overlap_width = min(A[2], B[2]) - max(A[0], B[0])
    overlap_height = min(A[3], B[3]) - max(A[1], B[1])
    
    if overlap_width > 0 and overlap_height > 0:
        area -= overlap_width * overlap_height
    
    return area
\end{lstlisting}

\subsection*{2. Using Geometry Libraries}

Leverage computational geometry libraries to handle area calculations and overlapping detections.

\begin{lstlisting}[language=Python]
from shapely.geometry import box

def computeArea(A: List[int], B: List[int]) -> int:
    rect1 = box(A[0], A[1], A[2], A[3])
    rect2 = box(B[0], B[1], B[2], B[3])
    intersection = rect1.intersection(rect2)
    return int(rect1.area + rect2.area - intersection.area)
\end{lstlisting}

\textbf{Note}: This approach requires the \texttt{shapely} library and is more suitable for complex geometric operations.

\section*{Similar Problems to This One}

Several problems involve calculating areas, handling geometric overlaps, and spatial reasoning, utilizing similar algorithmic strategies:

\begin{itemize}
    \item \textbf{Rectangle Overlap}: Determine if two rectangles overlap.
    \item \textbf{Circle Area Overlap}: Calculate the overlapping area between two circles.
    \item \textbf{Polygon Area}: Compute the area of a given polygon.
    \item \textbf{Union of Rectangles}: Calculate the total area covered by multiple rectangles, accounting for overlaps.
    \item \textbf{Intersection of Lines}: Find the intersection point of two lines.
    \item \textbf{Closest Pair of Points}: Find the closest pair of points in a set.
    \item \textbf{Convex Hull}: Compute the convex hull of a set of points.
    \item \textbf{Point Inside Polygon}: Determine if a point lies inside a given polygon.
\end{itemize}

These problems reinforce concepts of geometric calculations, area computations, and efficient algorithm design in various contexts.

\section*{Things to Keep in Mind and Tricks}

When tackling the \textbf{Rectangle Area} problem, consider the following tips and best practices to enhance efficiency and correctness:

\begin{itemize}
    \item \textbf{Understand Geometric Relationships}: Grasp the positional relationships between rectangles to simplify area calculations.
    \index{Geometric Relationships}
    
    \item \textbf{Leverage Coordinate Comparisons}: Use direct comparisons of rectangle coordinates to determine overlapping regions.
    \index{Coordinate Comparisons}
    
    \item \textbf{Handle Overlapping Scenarios}: Accurately calculate the overlapping area to avoid double-counting.
    \index{Overlapping Scenarios}
    
    \item \textbf{Optimize for Efficiency}: Aim for a constant time \(O(1)\) solution by avoiding unnecessary computations or iterations.
    \index{Efficiency Optimization}
    
    \item \textbf{Avoid Floating-Point Precision Issues}: Since all coordinates are integers, floating-point precision is not a concern, simplifying the implementation.
    \index{Floating-Point Precision}
    
    \item \textbf{Use Helper Functions}: Create helper functions to encapsulate repetitive tasks, such as calculating overlap dimensions or areas.
    \index{Helper Functions}
    
    \item \textbf{Code Readability}: Maintain clear and readable code through meaningful variable names and structured logic.
    \index{Code Readability}
    
    \item \textbf{Test Extensively}: Implement a wide range of test cases, including overlapping, non-overlapping, and edge-touching rectangles, to ensure robustness.
    \index{Extensive Testing}
    
    \item \textbf{Understand Axis-Aligned Constraints}: Recognize that axis-aligned rectangles simplify area calculations compared to rotated rectangles.
    \index{Axis-Aligned Constraints}
    
    \item \textbf{Simplify Logical Conditions}: Combine multiple conditions logically to streamline the area calculation process.
    \index{Logical Conditions}
    
    \item \textbf{Use Absolute Values}: When calculating differences, ensure that the dimensions are positive by using absolute values or proper ordering.
    \index{Absolute Values}
    
    \item \textbf{Consider Edge Cases}: Handle cases where rectangles have zero area or touch at edges/corners without overlapping.
    \index{Edge Cases}
\end{itemize}

\section*{Corner and Special Cases to Test When Writing the Code}

When implementing the solution for the \textbf{Rectangle Area} problem, it is crucial to consider and rigorously test various edge cases to ensure robustness and correctness:

\begin{itemize}
    \item \textbf{No Overlap}: Rectangles are completely separate.
    \index{No Overlap}
    
    \item \textbf{Partial Overlap}: Rectangles overlap in one or more regions.
    \index{Partial Overlap}
    
    \item \textbf{Edge Touching}: Rectangles touch exactly at one edge without overlapping.
    \index{Edge Touching}
    
    \item \textbf{Corner Touching}: Rectangles touch exactly at one corner without overlapping.
    \index{Corner Touching}
    
    \item \textbf{One Rectangle Inside Another}: One rectangle is entirely within the other.
    \index{Rectangle Inside}
    
    \item \textbf{Identical Rectangles}: Both rectangles have the same coordinates.
    \index{Identical Rectangles}
    
    \item \textbf{Degenerate Rectangles}: Rectangles with zero area (e.g., \(x1 = x2\) or \(y1 = y2\)).
    \index{Degenerate Rectangles}
    
    \item \textbf{Large Coordinates}: Rectangles with very large coordinate values to test performance and integer handling.
    \index{Large Coordinates}
    
    \item \textbf{Negative Coordinates}: Rectangles positioned in negative coordinate space.
    \index{Negative Coordinates}
    
    \item \textbf{Mixed Overlapping Scenarios}: Combinations of the above cases to ensure comprehensive coverage.
    \index{Mixed Overlapping Scenarios}
    
    \item \textbf{Minimum and Maximum Bounds}: Rectangles at the minimum and maximum limits of the coordinate range.
    \index{Minimum and Maximum Bounds}
    
    \item \textbf{Sequential Rectangles}: Multiple rectangles placed sequentially without overlapping.
    \index{Sequential Rectangles}
    
    \item \textbf{Multiple Overlaps}: Scenarios where more than two rectangles overlap in different regions.
    \index{Multiple Overlaps}
\end{itemize}

\section*{Implementation Considerations}

When implementing the \texttt{computeArea} function, keep in mind the following considerations to ensure robustness and efficiency:

\begin{itemize}
    \item \textbf{Data Type Selection}: Use appropriate data types that can handle large input values without overflow or underflow.
    \index{Data Type Selection}
    
    \item \textbf{Optimizing Comparisons}: Structure logical conditions to efficiently determine overlap dimensions.
    \index{Optimizing Comparisons}
    
    \item \textbf{Handling Large Inputs}: Design the algorithm to efficiently handle large input sizes without significant performance degradation.
    \index{Handling Large Inputs}
    
    \item \textbf{Language-Specific Constraints}: Be aware of how the programming language handles large integers and arithmetic operations.
    \index{Language-Specific Constraints}
    
    \item \textbf{Avoiding Redundant Calculations}: Ensure that each calculation contributes towards determining the final area without unnecessary repetitions.
    \index{Avoiding Redundant Calculations}
    
    \item \textbf{Code Readability and Documentation}: Maintain clear and readable code through meaningful variable names and comprehensive comments to facilitate understanding and maintenance.
    \index{Code Readability}
    
    \item \textbf{Edge Case Handling}: Implement checks for edge cases to prevent incorrect results or runtime errors.
    \index{Edge Case Handling}
    
    \item \textbf{Testing and Validation}: Develop a comprehensive suite of test cases that cover all possible scenarios, including edge cases, to validate the correctness and efficiency of the implementation.
    \index{Testing and Validation}
    
    \item \textbf{Scalability}: Design the algorithm to scale efficiently with increasing input sizes, maintaining performance and resource utilization.
    \index{Scalability}
    
    \item \textbf{Using Helper Functions}: Consider creating helper functions for repetitive tasks, such as calculating overlap dimensions, to enhance modularity and reusability.
    \index{Helper Functions}
    
    \item \textbf{Consistent Naming Conventions}: Use consistent and descriptive naming conventions for variables to improve code clarity.
    \index{Naming Conventions}
    
    \item \textbf{Implementing Unit Tests}: Develop unit tests for each logical condition to ensure that all scenarios are correctly handled.
    \index{Unit Tests}
    
    \item \textbf{Error Handling}: Incorporate error handling to manage invalid inputs gracefully.
    \index{Error Handling}
\end{itemize}

\section*{Conclusion}

The \textbf{Rectangle Area} problem showcases the application of fundamental geometric principles and efficient algorithm design to compute spatial properties accurately. By systematically calculating individual areas and intelligently handling overlapping regions, the algorithm ensures precise results without redundant computations. Understanding and implementing such techniques not only enhances problem-solving skills but also provides a foundation for tackling more complex Computational Geometry challenges involving multiple geometric entities and intricate spatial relationships.

\printindex

% \input{sections/rectangle_overlap}
% \input{sections/rectangle_area}
% \input{sections/k_closest_points_to_origin}
% \input{sections/the_skyline_problem}
% % filename: k_closest_points_to_origin.tex

\problemsection{K Closest Points to Origin}
\label{chap:K_Closest_Points_to_Origin}
\marginnote{\href{https://leetcode.com/problems/k-closest-points-to-origin/}{[LeetCode Link]}\index{LeetCode}}
\marginnote{\href{https://www.geeksforgeeks.org/find-k-closest-points-origin/}{[GeeksForGeeks Link]}\index{GeeksForGeeks}}
\marginnote{\href{https://www.interviewbit.com/problems/k-closest-points/}{[InterviewBit Link]}\index{InterviewBit}}
\marginnote{\href{https://app.codesignal.com/challenges/k-closest-points-to-origin}{[CodeSignal Link]}\index{CodeSignal}}
\marginnote{\href{https://www.codewars.com/kata/k-closest-points-to-origin/train/python}{[Codewars Link]}\index{Codewars}}

The \textbf{K Closest Points to Origin} problem is a popular algorithmic challenge in Computational Geometry that involves identifying the \(k\) points closest to the origin in a 2D plane. This problem tests one's ability to apply efficient sorting and selection algorithms, understand distance computations, and optimize for performance. Mastery of this problem is essential for applications in spatial data analysis, nearest neighbor searches, and clustering algorithms.

\section*{Problem Statement}

Given an array of points where each point is represented as \([x, y]\) in the 2D plane, and an integer \(k\), return the \(k\) closest points to the origin \((0, 0)\).

The distance between two points \((x_1, y_1)\) and \((x_2, y_2)\) is the Euclidean distance \(\sqrt{(x_1 - x_2)^2 + (y_1 - y_2)^2}\). The origin is \((0, 0)\).

\textbf{Function signature in Python:}
\begin{lstlisting}[language=Python]
def kClosest(points: List[List[int]], K: int) -> List[List[int]]:
\end{lstlisting}

\section*{Examples}

\textbf{Example 1:}

\begin{verbatim}
Input: points = [[1,3],[-2,2]], K = 1
Output: [[-2,2]]
Explanation: 
The distance between (1, 3) and the origin is sqrt(10).
The distance between (-2, 2) and the origin is sqrt(8).
Since sqrt(8) < sqrt(10), (-2, 2) is closer to the origin.
\end{verbatim}

\textbf{Example 2:}

\begin{verbatim}
Input: points = [[3,3],[5,-1],[-2,4]], K = 2
Output: [[3,3],[-2,4]]
Explanation: 
The distances are sqrt(18), sqrt(26), and sqrt(20) respectively.
The two closest points are [3,3] and [-2,4].
\end{verbatim}

\textbf{Example 3:}

\begin{verbatim}
Input: points = [[0,1],[1,0]], K = 2
Output: [[0,1],[1,0]]
Explanation: 
Both points are equally close to the origin.
\end{verbatim}

\textbf{Example 4:}

\begin{verbatim}
Input: points = [[1,0],[0,1]], K = 1
Output: [[1,0]]
Explanation: 
Both points are equally close; returning any one is acceptable.
\end{verbatim}

\textbf{Constraints:}

\begin{itemize}
    \item \(1 \leq K \leq \text{points.length} \leq 10^4\)
    \item \(-10^4 < x_i, y_i < 10^4\)
\end{itemize}

LeetCode link: \href{https://leetcode.com/problems/k-closest-points-to-origin/}{K Closest Points to Origin}\index{LeetCode}

\section*{Algorithmic Approach}

To identify the \(k\) closest points to the origin, several algorithmic strategies can be employed. The most efficient methods aim to reduce the time complexity by avoiding the need to sort the entire list of points.

\subsection*{1. Sorting Based on Distance}

Calculate the Euclidean distance of each point from the origin and sort the points based on these distances. Select the first \(k\) points from the sorted list.

\begin{enumerate}
    \item Compute the distance for each point using the formula \(distance = x^2 + y^2\).
    \item Sort the points based on the computed distances.
    \item Return the first \(k\) points from the sorted list.
\end{enumerate}

\subsection*{2. Max Heap (Priority Queue)}

Use a max heap to maintain the \(k\) closest points. Iterate through each point, add it to the heap, and if the heap size exceeds \(k\), remove the farthest point.

\begin{enumerate}
    \item Initialize a max heap.
    \item For each point, compute its distance and add it to the heap.
    \item If the heap size exceeds \(k\), remove the point with the largest distance.
    \item After processing all points, the heap contains the \(k\) closest points.
\end{enumerate}

\subsection*{3. QuickSelect (Quick Sort Partitioning)}

Utilize the QuickSelect algorithm to find the \(k\) closest points without fully sorting the list.

\begin{enumerate}
    \item Choose a pivot point and partition the list based on distances relative to the pivot.
    \item Recursively apply QuickSelect to the partition containing the \(k\) closest points.
    \item Once the \(k\) closest points are identified, return them.
\end{enumerate}

\marginnote{QuickSelect offers an average time complexity of \(O(n)\), making it highly efficient for large datasets.}

\section*{Complexities}

\begin{itemize}
    \item \textbf{Sorting Based on Distance:}
    \begin{itemize}
        \item \textbf{Time Complexity:} \(O(n \log n)\)
        \item \textbf{Space Complexity:} \(O(n)\)
    \end{itemize}
    
    \item \textbf{Max Heap (Priority Queue):}
    \begin{itemize}
        \item \textbf{Time Complexity:} \(O(n \log k)\)
        \item \textbf{Space Complexity:} \(O(k)\)
    \end{itemize}
    
    \item \textbf{QuickSelect (Quick Sort Partitioning):}
    \begin{itemize}
        \item \textbf{Time Complexity:} Average case \(O(n)\), worst case \(O(n^2)\)
        \item \textbf{Space Complexity:} \(O(1)\) (in-place)
    \end{itemize}
\end{itemize}

\section*{Python Implementation}

\marginnote{Implementing QuickSelect provides an optimal average-case solution with linear time complexity.}

Below is the complete Python code implementing the \texttt{kClosest} function using the QuickSelect approach:

\begin{fullwidth}
\begin{lstlisting}[language=Python]
from typing import List
import random

class Solution:
    def kClosest(self, points: List[List[int]], K: int) -> List[List[int]]:
        def quickselect(left, right, K_smallest):
            if left == right:
                return
            
            # Select a random pivot_index
            pivot_index = random.randint(left, right)
            
            # Partition the array
            pivot_index = partition(left, right, pivot_index)
            
            # The pivot is in its final sorted position
            if K_smallest == pivot_index:
                return
            elif K_smallest < pivot_index:
                quickselect(left, pivot_index - 1, K_smallest)
            else:
                quickselect(pivot_index + 1, right, K_smallest)
        
        def partition(left, right, pivot_index):
            pivot_distance = distance(points[pivot_index])
            # Move pivot to end
            points[pivot_index], points[right] = points[right], points[pivot_index]
            store_index = left
            for i in range(left, right):
                if distance(points[i]) < pivot_distance:
                    points[store_index], points[i] = points[i], points[store_index]
                    store_index += 1
            # Move pivot to its final place
            points[right], points[store_index] = points[store_index], points[right]
            return store_index
        
        def distance(point):
            return point[0] ** 2 + point[1] ** 2
        
        n = len(points)
        quickselect(0, n - 1, K)
        return points[:K]

# Example usage:
solution = Solution()
print(solution.kClosest([[1,3],[-2,2]], 1))            # Output: [[-2,2]]
print(solution.kClosest([[3,3],[5,-1],[-2,4]], 2))     # Output: [[3,3],[-2,4]]
print(solution.kClosest([[0,1],[1,0]], 2))             # Output: [[0,1],[1,0]]
print(solution.kClosest([[1,0],[0,1]], 1))             # Output: [[1,0]] or [[0,1]]
\end{lstlisting}
\end{fullwidth}

This implementation uses the QuickSelect algorithm to efficiently find the \(k\) closest points to the origin without fully sorting the entire list. It ensures optimal performance even with large datasets.

\section*{Explanation}

The \texttt{kClosest} function identifies the \(k\) closest points to the origin using the QuickSelect algorithm. Here's a detailed breakdown of the implementation:

\subsection*{1. Distance Calculation}

\begin{itemize}
    \item The Euclidean distance is calculated as \(distance = x^2 + y^2\). Since we only need relative distances for comparison, the square root is omitted for efficiency.
\end{itemize}

\subsection*{2. QuickSelect Algorithm}

\begin{itemize}
    \item **Pivot Selection:**
    \begin{itemize}
        \item A random pivot is chosen to enhance the average-case performance.
    \end{itemize}
    
    \item **Partitioning:**
    \begin{itemize}
        \item The array is partitioned such that points with distances less than the pivot are moved to the left, and others to the right.
        \item The pivot is placed in its correct sorted position.
    \end{itemize}
    
    \item **Recursive Selection:**
    \begin{itemize}
        \item If the pivot's position matches \(K\), the selection is complete.
        \item Otherwise, recursively apply QuickSelect to the relevant partition.
    \end{itemize}
\end{itemize}

\subsection*{3. Final Selection}

\begin{itemize}
    \item After partitioning, the first \(K\) points in the list are the \(k\) closest points to the origin.
\end{itemize}

\subsection*{4. Example Walkthrough}

Consider the first example:
\begin{verbatim}
Input: points = [[1,3],[-2,2]], K = 1
Output: [[-2,2]]
\end{verbatim}

\begin{enumerate}
    \item **Calculate Distances:**
    \begin{itemize}
        \item [1,3] : \(1^2 + 3^2 = 10\)
        \item [-2,2] : \((-2)^2 + 2^2 = 8\)
    \end{itemize}
    
    \item **QuickSelect Process:**
    \begin{itemize}
        \item Choose a pivot, say [1,3] with distance 10.
        \item Compare and rearrange:
        \begin{itemize}
            \item [-2,2] has a smaller distance (8) and is moved to the left.
        \end{itemize}
        \item After partitioning, the list becomes [[-2,2], [1,3]].
        \item Since \(K = 1\), return the first point: [[-2,2]].
    \end{itemize}
\end{enumerate}

Thus, the function correctly identifies \([-2,2]\) as the closest point to the origin.

\section*{Why This Approach}

The QuickSelect algorithm is chosen for its average-case linear time complexity \(O(n)\), making it highly efficient for large datasets compared to sorting-based methods with \(O(n \log n)\) time complexity. By avoiding the need to sort the entire list, QuickSelect provides an optimal solution for finding the \(k\) closest points.

\section*{Alternative Approaches}

\subsection*{1. Sorting Based on Distance}

Sort all points based on their distances from the origin and select the first \(k\) points.

\begin{lstlisting}[language=Python]
class Solution:
    def kClosest(self, points: List[List[int]], K: int) -> List[List[int]]:
        points.sort(key=lambda P: P[0]**2 + P[1]**2)
        return points[:K]
\end{lstlisting}

\textbf{Complexities:}
\begin{itemize}
    \item \textbf{Time Complexity:} \(O(n \log n)\)
    \item \textbf{Space Complexity:} \(O(1)\)
\end{itemize}

\subsection*{2. Max Heap (Priority Queue)}

Use a max heap to maintain the \(k\) closest points.

\begin{lstlisting}[language=Python]
import heapq

class Solution:
    def kClosest(self, points: List[List[int]], K: int) -> List[List[int]]:
        heap = []
        for (x, y) in points:
            dist = -(x**2 + y**2)  # Max heap using negative distances
            heapq.heappush(heap, (dist, [x, y]))
            if len(heap) > K:
                heapq.heappop(heap)
        return [item[1] for item in heap]
\end{lstlisting}

\textbf{Complexities:}
\begin{itemize}
    \item \textbf{Time Complexity:} \(O(n \log k)\)
    \item \textbf{Space Complexity:} \(O(k)\)
\end{itemize}

\subsection*{3. Using Built-In Functions}

Leverage built-in functions for distance calculation and selection.

\begin{lstlisting}[language=Python]
import math

class Solution:
    def kClosest(self, points: List[List[int]], K: int) -> List[List[int]]:
        points.sort(key=lambda P: math.sqrt(P[0]**2 + P[1]**2))
        return points[:K]
\end{lstlisting}

\textbf{Note}: This method is similar to the sorting approach but uses the actual Euclidean distance.

\section*{Similar Problems to This One}

Several problems involve nearest neighbor searches, spatial data analysis, and efficient selection algorithms, utilizing similar algorithmic strategies:

\begin{itemize}
    \item \textbf{Closest Pair of Points}: Find the closest pair of points in a set.
    \item \textbf{Top K Frequent Elements}: Identify the most frequent elements in a dataset.
    \item \textbf{Kth Largest Element in an Array}: Find the \(k\)-th largest element in an unsorted array.
    \item \textbf{Sliding Window Maximum}: Find the maximum in each sliding window of size \(k\) over an array.
    \item \textbf{Merge K Sorted Lists}: Merge multiple sorted lists into a single sorted list.
    \item \textbf{Find Median from Data Stream}: Continuously find the median of a stream of numbers.
    \item \textbf{Top K Closest Stars}: Find the \(k\) closest stars to Earth based on their distances.
\end{itemize}

These problems reinforce concepts of efficient selection, heap usage, and distance computations in various contexts.

\section*{Things to Keep in Mind and Tricks}

When solving the \textbf{K Closest Points to Origin} problem, consider the following tips and best practices to enhance efficiency and correctness:

\begin{itemize}
    \item \textbf{Understand Distance Calculations}: Grasp the Euclidean distance formula and recognize that the square root can be omitted for comparison purposes.
    \index{Distance Calculations}
    
    \item \textbf{Leverage Efficient Algorithms}: Use QuickSelect or heap-based methods to optimize time complexity, especially for large datasets.
    \index{Efficient Algorithms}
    
    \item \textbf{Handle Ties Appropriately}: Decide how to handle points with identical distances when \(k\) is less than the number of such points.
    \index{Handling Ties}
    
    \item \textbf{Optimize Space Usage}: Choose algorithms that minimize additional space, such as in-place QuickSelect.
    \index{Space Optimization}
    
    \item \textbf{Use Appropriate Data Structures}: Utilize heaps, lists, and helper functions effectively to manage and process data.
    \index{Data Structures}
    
    \item \textbf{Implement Helper Functions}: Create helper functions for distance calculation and partitioning to enhance code modularity.
    \index{Helper Functions}
    
    \item \textbf{Code Readability}: Maintain clear and readable code through meaningful variable names and structured logic.
    \index{Code Readability}
    
    \item \textbf{Test Extensively}: Implement a wide range of test cases, including edge cases like multiple points with the same distance, to ensure robustness.
    \index{Extensive Testing}
    
    \item \textbf{Understand Algorithm Trade-offs}: Recognize the trade-offs between different approaches in terms of time and space complexities.
    \index{Algorithm Trade-offs}
    
    \item \textbf{Use Built-In Sorting Functions}: When using sorting-based approaches, leverage built-in functions for efficiency and simplicity.
    \index{Built-In Sorting}
    
    \item \textbf{Avoid Redundant Calculations}: Ensure that distance calculations are performed only when necessary to optimize performance.
    \index{Avoiding Redundant Calculations}
    
    \item \textbf{Language-Specific Features}: Utilize language-specific features or libraries that can simplify implementation, such as heapq in Python.
    \index{Language-Specific Features}
\end{itemize}

\section*{Corner and Special Cases to Test When Writing the Code}

When implementing the solution for the \textbf{K Closest Points to Origin} problem, it is crucial to consider and rigorously test various edge cases to ensure robustness and correctness:

\begin{itemize}
    \item \textbf{Multiple Points with Same Distance}: Ensure that the algorithm handles multiple points having the same distance from the origin.
    \index{Same Distance Points}
    
    \item \textbf{Points at Origin}: Include points that are exactly at the origin \((0,0)\).
    \index{Points at Origin}
    
    \item \textbf{Negative Coordinates}: Ensure that the algorithm correctly computes distances for points with negative \(x\) or \(y\) coordinates.
    \index{Negative Coordinates}
    
    \item \textbf{Large Coordinates}: Test with points having very large or very small coordinate values to verify integer handling.
    \index{Large Coordinates}
    
    \item \textbf{K Equals Number of Points}: When \(K\) is equal to the number of points, the algorithm should return all points.
    \index{K Equals Number of Points}
    
    \item \textbf{K is One}: Test with \(K = 1\) to ensure the closest point is correctly identified.
    \index{K is One}
    
    \item \textbf{All Points Same}: All points have the same coordinates.
    \index{All Points Same}
    
    \item \textbf{K is Zero}: Although \(K\) is defined to be at least 1, ensure that the algorithm gracefully handles \(K = 0\) if allowed.
    \index{K is Zero}
    
    \item \textbf{Single Point}: Only one point is provided, and \(K = 1\).
    \index{Single Point}
    
    \item \textbf{Mixed Coordinates}: Points with a mix of positive and negative coordinates.
    \index{Mixed Coordinates}
    
    \item \textbf{Points with Zero Distance}: Multiple points at the origin.
    \index{Zero Distance Points}
    
    \item \textbf{Sparse and Dense Points}: Densely packed points and sparsely distributed points.
    \index{Sparse and Dense Points}
    
    \item \textbf{Duplicate Points}: Multiple identical points in the input list.
    \index{Duplicate Points}
    
    \item \textbf{K Greater Than Number of Unique Points}: Ensure that the algorithm handles cases where \(K\) exceeds the number of unique points if applicable.
    \index{K Greater Than Unique Points}
\end{itemize}

\section*{Implementation Considerations}

When implementing the \texttt{kClosest} function, keep in mind the following considerations to ensure robustness and efficiency:

\begin{itemize}
    \item \textbf{Data Type Selection}: Use appropriate data types that can handle large input values without overflow or precision loss.
    \index{Data Type Selection}
    
    \item \textbf{Optimizing Distance Calculations}: Avoid calculating the square root since it is unnecessary for comparison purposes.
    \index{Optimizing Distance Calculations}
    
    \item \textbf{Choosing the Right Algorithm}: Select an algorithm based on the size of the input and the value of \(K\) to optimize time and space complexities.
    \index{Choosing the Right Algorithm}
    
    \item \textbf{Language-Specific Libraries}: Utilize language-specific libraries and functions (e.g., \texttt{heapq} in Python) to simplify implementation and enhance performance.
    \index{Language-Specific Libraries}
    
    \item \textbf{Avoiding Redundant Calculations}: Ensure that each point's distance is calculated only once to optimize performance.
    \index{Avoiding Redundant Calculations}
    
    \item \textbf{Implementing Helper Functions}: Create helper functions for tasks like distance calculation and partitioning to enhance modularity and readability.
    \index{Helper Functions}
    
    \item \textbf{Edge Case Handling}: Implement checks for edge cases to prevent incorrect results or runtime errors.
    \index{Edge Case Handling}
    
    \item \textbf{Testing and Validation}: Develop a comprehensive suite of test cases that cover all possible scenarios, including edge cases, to validate the correctness and efficiency of the implementation.
    \index{Testing and Validation}
    
    \item \textbf{Scalability}: Design the algorithm to scale efficiently with increasing input sizes, maintaining performance and resource utilization.
    \index{Scalability}
    
    \item \textbf{Consistent Naming Conventions}: Use consistent and descriptive naming conventions for variables and functions to improve code clarity.
    \index{Naming Conventions}
    
    \item \textbf{Memory Management}: Ensure that the algorithm manages memory efficiently, especially when dealing with large datasets.
    \index{Memory Management}
    
    \item \textbf{Avoiding Stack Overflow}: If implementing recursive approaches, be mindful of recursion limits and potential stack overflow issues.
    \index{Avoiding Stack Overflow}
    
    \item \textbf{Implementing Iterative Solutions}: Prefer iterative solutions when recursion may lead to increased space complexity or stack overflow.
    \index{Implementing Iterative Solutions}
\end{itemize}

\section*{Conclusion}

The \textbf{K Closest Points to Origin} problem exemplifies the application of efficient selection algorithms and geometric computations to solve spatial challenges effectively. By leveraging QuickSelect or heap-based methods, the algorithm achieves optimal time and space complexities, making it highly suitable for large datasets. Understanding and implementing such techniques not only enhances problem-solving skills but also provides a foundation for tackling more advanced Computational Geometry problems involving nearest neighbor searches, clustering, and spatial data analysis.

\printindex

% \input{sections/rectangle_overlap}
% \input{sections/rectangle_area}
% \input{sections/k_closest_points_to_origin}
% \input{sections/the_skyline_problem}
% % filename: the_skyline_problem.tex

\problemsection{The Skyline Problem}
\label{chap:The_Skyline_Problem}
\marginnote{\href{https://leetcode.com/problems/the-skyline-problem/}{[LeetCode Link]}\index{LeetCode}}
\marginnote{\href{https://www.geeksforgeeks.org/the-skyline-problem/}{[GeeksForGeeks Link]}\index{GeeksForGeeks}}
\marginnote{\href{https://www.interviewbit.com/problems/the-skyline-problem/}{[InterviewBit Link]}\index{InterviewBit}}
\marginnote{\href{https://app.codesignal.com/challenges/the-skyline-problem}{[CodeSignal Link]}\index{CodeSignal}}
\marginnote{\href{https://www.codewars.com/kata/the-skyline-problem/train/python}{[Codewars Link]}\index{Codewars}}

The \textbf{Skyline Problem} is a complex Computational Geometry challenge that involves computing the skyline formed by a collection of buildings in a 2D cityscape. Each building is represented by its left and right x-coordinates and its height. The skyline is defined by a list of "key points" where the height changes. This problem tests one's ability to handle large datasets, implement efficient sweep line algorithms, and manage event-driven processing. Mastery of this problem is essential for applications in computer graphics, urban planning simulations, and geographic information systems (GIS).

\section*{Problem Statement}

You are given a list of buildings in a cityscape. Each building is represented as a triplet \([Li, Ri, Hi]\), where \(Li\) and \(Ri\) are the x-coordinates of the left and right edges of the building, respectively, and \(Hi\) is the height of the building.

The skyline should be represented as a list of key points \([x, y]\) in sorted order by \(x\)-coordinate, where \(y\) is the height of the skyline at that point. The skyline should only include critical points where the height changes.

\textbf{Function signature in Python:}
\begin{lstlisting}[language=Python]
def getSkyline(buildings: List[List[int]]) -> List[List[int]]:
\end{lstlisting}

\section*{Examples}

\textbf{Example 1:}

\begin{verbatim}
Input: buildings = [[2,9,10], [3,7,15], [5,12,12], [15,20,10], [19,24,8]]
Output: [[2,10], [3,15], [7,12], [12,0], [15,10], [20,8], [24,0]]
Explanation:
- At x=2, the first building starts, height=10.
- At x=3, the second building starts, height=15.
- At x=7, the second building ends, the third building is still ongoing, height=12.
- At x=12, the third building ends, height drops to 0.
- At x=15, the fourth building starts, height=10.
- At x=20, the fourth building ends, the fifth building is still ongoing, height=8.
- At x=24, the fifth building ends, height drops to 0.
\end{verbatim}

\textbf{Example 2:}

\begin{verbatim}
Input: buildings = [[0,2,3], [2,5,3]]
Output: [[0,3], [5,0]]
Explanation:
- The two buildings are contiguous and have the same height, so the skyline drops to 0 at x=5.
\end{verbatim}

\textbf{Example 3:}

\begin{verbatim}
Input: buildings = [[1,3,3], [2,4,4], [5,6,1]]
Output: [[1,3], [2,4], [4,0], [5,1], [6,0]]
Explanation:
- At x=1, first building starts, height=3.
- At x=2, second building starts, height=4.
- At x=4, second building ends, height drops to 0.
- At x=5, third building starts, height=1.
- At x=6, third building ends, height drops to 0.
\end{verbatim}

\textbf{Example 4:}

\begin{verbatim}
Input: buildings = [[0,5,0]]
Output: []
Explanation:
- A building with height 0 does not contribute to the skyline.
\end{verbatim}

\textbf{Constraints:}

\begin{itemize}
    \item \(1 \leq \text{buildings.length} \leq 10^4\)
    \item \(0 \leq Li < Ri \leq 10^9\)
    \item \(0 \leq Hi \leq 10^4\)
\end{itemize}

\section*{Algorithmic Approach}

The \textbf{Sweep Line Algorithm} is an efficient method for solving the Skyline Problem. It involves processing events (building start and end points) in sorted order while maintaining a data structure (typically a max heap) to keep track of active buildings. Here's a step-by-step approach:

\subsection*{1. Event Representation}

Transform each building into two events:
\begin{itemize}
    \item **Start Event:** \((Li, -Hi)\) – Negative height indicates a building starts.
    \item **End Event:** \((Ri, Hi)\) – Positive height indicates a building ends.
\end{itemize}

Sorting the events ensures that start events are processed before end events at the same x-coordinate, and taller buildings are processed before shorter ones.

\subsection*{2. Sorting the Events}

Sort all events based on:
\begin{enumerate}
    \item **x-coordinate:** Ascending order.
    \item **Height:**
    \begin{itemize}
        \item For start events, taller buildings come first.
        \item For end events, shorter buildings come first.
    \end{itemize}
\end{enumerate}

\subsection*{3. Processing the Events}

Use a max heap to keep track of active building heights. Iterate through the sorted events:
\begin{enumerate}
    \item **Start Event:**
    \begin{itemize}
        \item Add the building's height to the heap.
    \end{itemize}
    
    \item **End Event:**
    \begin{itemize}
        \item Remove the building's height from the heap.
    \end{itemize}
    
    \item **Determine Current Max Height:**
    \begin{itemize}
        \item The current max height is the top of the heap.
    \end{itemize}
    
    \item **Update Skyline:**
    \begin{itemize}
        \item If the current max height differs from the previous max height, add a new key point \([x, current\_max\_height]\).
    \end{itemize}
\end{enumerate}

\subsection*{4. Finalizing the Skyline}

After processing all events, the accumulated key points represent the skyline.

\marginnote{The Sweep Line Algorithm efficiently handles dynamic changes in active buildings, ensuring accurate skyline construction.}

\section*{Complexities}

\begin{itemize}
    \item \textbf{Time Complexity:} \(O(n \log n)\), where \(n\) is the number of buildings. Sorting the events takes \(O(n \log n)\), and each heap operation takes \(O(\log n)\).
    
    \item \textbf{Space Complexity:} \(O(n)\), due to the storage of events and the heap.
\end{itemize}

\section*{Python Implementation}

\marginnote{Implementing the Sweep Line Algorithm with a max heap ensures an efficient and accurate solution.}

Below is the complete Python code implementing the \texttt{getSkyline} function:

\begin{fullwidth}
\begin{lstlisting}[language=Python]
from typing import List
import heapq

class Solution:
    def getSkyline(self, buildings: List[List[int]]) -> List[List[int]]:
        # Create a list of all events
        # For start events, use negative height to ensure they are processed before end events
        events = []
        for L, R, H in buildings:
            events.append((L, -H))
            events.append((R, H))
        
        # Sort the events
        # First by x-coordinate, then by height
        events.sort()
        
        # Max heap to keep track of active buildings
        heap = [0]  # Initialize with ground level
        heapq.heapify(heap)
        active_heights = {0: 1}  # Dictionary to count heights
        
        result = []
        prev_max = 0
        
        for x, h in events:
            if h < 0:
                # Start of a building, add height to heap and dictionary
                heapq.heappush(heap, h)
                active_heights[h] = active_heights.get(h, 0) + 1
            else:
                # End of a building, remove height from dictionary
                active_heights[h] -= 1
                if active_heights[h] == 0:
                    del active_heights[h]
            
            # Current max height
            while heap and active_heights.get(heap[0], 0) == 0:
                heapq.heappop(heap)
            current_max = -heap[0] if heap else 0
            
            # If the max height has changed, add to result
            if current_max != prev_max:
                result.append([x, current_max])
                prev_max = current_max
        
        return result

# Example usage:
solution = Solution()
print(solution.getSkyline([[2,9,10], [3,7,15], [5,12,12], [15,20,10], [19,24,8]]))
# Output: [[2,10], [3,15], [7,12], [12,0], [15,10], [20,8], [24,0]]

print(solution.getSkyline([[0,2,3], [2,5,3]]))
# Output: [[0,3], [5,0]]

print(solution.getSkyline([[1,3,3], [2,4,4], [5,6,1]]))
# Output: [[1,3], [2,4], [4,0], [5,1], [6,0]]

print(solution.getSkyline([[0,5,0]]))
# Output: []
\end{lstlisting}
\end{fullwidth}

This implementation efficiently constructs the skyline by processing all building events in sorted order and maintaining active building heights using a max heap. It ensures that only critical points where the skyline changes are recorded.

\section*{Explanation}

The \texttt{getSkyline} function constructs the skyline formed by a set of buildings by leveraging the Sweep Line Algorithm and a max heap to track active buildings. Here's a detailed breakdown of the implementation:

\subsection*{1. Event Representation}

\begin{itemize}
    \item Each building is transformed into two events:
    \begin{itemize}
        \item **Start Event:** \((Li, -Hi)\) – Negative height indicates the start of a building.
        \item **End Event:** \((Ri, Hi)\) – Positive height indicates the end of a building.
    \end{itemize}
\end{itemize}

\subsection*{2. Sorting the Events}

\begin{itemize}
    \item Events are sorted primarily by their x-coordinate in ascending order.
    \item For events with the same x-coordinate:
    \begin{itemize}
        \item Start events (with negative heights) are processed before end events.
        \item Taller buildings are processed before shorter ones.
    \end{itemize}
\end{itemize}

\subsection*{3. Processing the Events}

\begin{itemize}
    \item **Heap Initialization:**
    \begin{itemize}
        \item A max heap is initialized with a ground level height of 0.
        \item A dictionary \texttt{active\_heights} tracks the count of active building heights.
    \end{itemize}
    
    \item **Iterating Through Events:**
    \begin{enumerate}
        \item **Start Event:**
        \begin{itemize}
            \item Add the building's height to the heap.
            \item Increment the count of the height in \texttt{active\_heights}.
        \end{itemize}
        
        \item **End Event:**
        \begin{itemize}
            \item Decrement the count of the building's height in \texttt{active\_heights}.
            \item If the count reaches zero, remove the height from the dictionary.
        \end{itemize}
        
        \item **Determine Current Max Height:**
        \begin{itemize}
            \item Remove heights from the heap that are no longer active.
            \item The current max height is the top of the heap.
        \end{itemize}
        
        \item **Update Skyline:**
        \begin{itemize}
            \item If the current max height differs from the previous max height, add a new key point \([x, current\_max\_height]\).
        \end{itemize}
    \end{enumerate}
\end{itemize}

\subsection*{4. Finalizing the Skyline}

\begin{itemize}
    \item After processing all events, the \texttt{result} list contains the key points defining the skyline.
\end{itemize}

\subsection*{5. Example Walkthrough}

Consider the first example:
\begin{verbatim}
Input: buildings = [[2,9,10], [3,7,15], [5,12,12], [15,20,10], [19,24,8]]
Output: [[2,10], [3,15], [7,12], [12,0], [15,10], [20,8], [24,0]]
\end{verbatim}

\begin{enumerate}
    \item **Event Transformation:**
    \begin{itemize}
        \item \((2, -10)\), \((9, 10)\)
        \item \((3, -15)\), \((7, 15)\)
        \item \((5, -12)\), \((12, 12)\)
        \item \((15, -10)\), \((20, 10)\)
        \item \((19, -8)\), \((24, 8)\)
    \end{itemize}
    
    \item **Sorting Events:**
    \begin{itemize}
        \item Sorted order: \((2, -10)\), \((3, -15)\), \((5, -12)\), \((7, 15)\), \((9, 10)\), \((12, 12)\), \((15, -10)\), \((19, -8)\), \((20, 10)\), \((24, 8)\)
    \end{itemize}
    
    \item **Processing Events:**
    \begin{itemize}
        \item At each event, update the heap and determine if the skyline height changes.
    \end{itemize}
    
    \item **Result Construction:**
    \begin{itemize}
        \item The resulting skyline key points are accumulated as \([[2,10], [3,15], [7,12], [12,0], [15,10], [20,8], [24,0]]\).
    \end{itemize}
\end{enumerate}

Thus, the function correctly constructs the skyline based on the buildings' positions and heights.

\section*{Why This Approach}

The Sweep Line Algorithm combined with a max heap offers an optimal solution with \(O(n \log n)\) time complexity and efficient handling of overlapping buildings. By processing events in sorted order and maintaining active building heights, the algorithm ensures that all critical points in the skyline are accurately identified without redundant computations.

\section*{Alternative Approaches}

\subsection*{1. Divide and Conquer}

Divide the set of buildings into smaller subsets, compute the skyline for each subset, and then merge the skylines.

\begin{lstlisting}[language=Python]
class Solution:
    def getSkyline(self, buildings: List[List[int]]) -> List[List[int]]:
        def merge(left, right):
            h1, h2 = 0, 0
            i, j = 0, 0
            merged = []
            while i < len(left) and j < len(right):
                if left[i][0] < right[j][0]:
                    x, h1 = left[i]
                    i += 1
                elif left[i][0] > right[j][0]:
                    x, h2 = right[j]
                    j += 1
                else:
                    x, h1 = left[i]
                    _, h2 = right[j]
                    i += 1
                    j += 1
                max_h = max(h1, h2)
                if not merged or merged[-1][1] != max_h:
                    merged.append([x, max_h])
            merged.extend(left[i:])
            merged.extend(right[j:])
            return merged
        
        def divide(buildings):
            if not buildings:
                return []
            if len(buildings) == 1:
                L, R, H = buildings[0]
                return [[L, H], [R, 0]]
            mid = len(buildings) // 2
            left = divide(buildings[:mid])
            right = divide(buildings[mid:])
            return merge(left, right)
        
        return divide(buildings)
\end{lstlisting}

\textbf{Complexities:}
\begin{itemize}
    \item \textbf{Time Complexity:} \(O(n \log n)\)
    \item \textbf{Space Complexity:} \(O(n)\)
\end{itemize}

\subsection*{2. Using Segment Trees}

Implement a segment tree to manage and query overlapping building heights dynamically.

\textbf{Note}: This approach is more complex and is generally used for advanced scenarios with multiple dynamic queries.

\section*{Similar Problems to This One}

Several problems involve skyline-like constructions, spatial data analysis, and efficient event processing, utilizing similar algorithmic strategies:

\begin{itemize}
    \item \textbf{Merge Intervals}: Merge overlapping intervals in a list.
    \item \textbf{Largest Rectangle in Histogram}: Find the largest rectangular area in a histogram.
    \item \textbf{Interval Partitioning}: Assign intervals to resources without overlap.
    \item \textbf{Line Segment Intersection}: Detect intersections among line segments.
    \item \textbf{Closest Pair of Points}: Find the closest pair of points in a set.
    \item \textbf{Convex Hull}: Compute the convex hull of a set of points.
    \item \textbf{Point Inside Polygon}: Determine if a point lies inside a given polygon.
    \item \textbf{Range Searching}: Efficiently query geometric data within a specified range.
\end{itemize}

These problems reinforce concepts of event-driven processing, spatial reasoning, and efficient algorithm design in various contexts.

\section*{Things to Keep in Mind and Tricks}

When tackling the \textbf{Skyline Problem}, consider the following tips and best practices to enhance efficiency and correctness:

\begin{itemize}
    \item \textbf{Understand Sweep Line Technique}: Grasp how the sweep line algorithm processes events in sorted order to handle dynamic changes efficiently.
    \index{Sweep Line Technique}
    
    \item \textbf{Leverage Priority Queues (Heaps)}: Use max heaps to keep track of active buildings' heights, enabling quick access to the current maximum height.
    \index{Priority Queues}
    
    \item \textbf{Handle Start and End Events Differently}: Differentiate between building start and end events to accurately manage active heights.
    \index{Start and End Events}
    
    \item \textbf{Optimize Event Sorting}: Sort events primarily by x-coordinate and secondarily by height to ensure correct processing order.
    \index{Event Sorting}
    
    \item \textbf{Manage Active Heights Efficiently}: Use data structures that allow efficient insertion, deletion, and retrieval of maximum elements.
    \index{Active Heights Management}
    
    \item \textbf{Avoid Redundant Key Points}: Only record key points when the skyline height changes to minimize the output list.
    \index{Avoiding Redundant Key Points}
    
    \item \textbf{Implement Helper Functions}: Create helper functions for tasks like distance calculation, event handling, and heap management to enhance modularity.
    \index{Helper Functions}
    
    \item \textbf{Code Readability}: Maintain clear and readable code through meaningful variable names and structured logic.
    \index{Code Readability}
    
    \item \textbf{Test Extensively}: Implement a wide range of test cases, including overlapping, non-overlapping, and edge-touching buildings, to ensure robustness.
    \index{Extensive Testing}
    
    \item \textbf{Handle Degenerate Cases}: Manage cases where buildings have zero height or identical coordinates gracefully.
    \index{Degenerate Cases}
    
    \item \textbf{Understand Geometric Relationships}: Grasp how buildings overlap and influence the skyline to simplify the algorithm.
    \index{Geometric Relationships}
    
    \item \textbf{Use Appropriate Data Structures}: Utilize appropriate data structures like heaps, lists, and dictionaries to manage and process data efficiently.
    \index{Appropriate Data Structures}
    
    \item \textbf{Optimize for Large Inputs}: Design the algorithm to handle large numbers of buildings without significant performance degradation.
    \index{Optimizing for Large Inputs}
    
    \item \textbf{Implement Iterative Solutions Carefully}: Ensure that loop conditions are correctly defined to prevent infinite loops or incorrect terminations.
    \index{Iterative Solutions}
    
    \item \textbf{Consistent Naming Conventions}: Use consistent and descriptive naming conventions for variables and functions to improve code clarity.
    \index{Naming Conventions}
\end{itemize}

\section*{Corner and Special Cases to Test When Writing the Code}

When implementing the solution for the \textbf{Skyline Problem}, it is crucial to consider and rigorously test various edge cases to ensure robustness and correctness:

\begin{itemize}
    \item \textbf{No Overlapping Buildings}: All buildings are separate and do not overlap.
    \index{No Overlapping Buildings}
    
    \item \textbf{Fully Overlapping Buildings}: Multiple buildings completely overlap each other.
    \index{Fully Overlapping Buildings}
    
    \item \textbf{Buildings Touching at Edges}: Buildings share common edges without overlapping.
    \index{Buildings Touching at Edges}
    
    \item \textbf{Buildings Touching at Corners}: Buildings share common corners without overlapping.
    \index{Buildings Touching at Corners}
    
    \item \textbf{Single Building}: Only one building is present.
    \index{Single Building}
    
    \item \textbf{Multiple Buildings with Same Start or End}: Multiple buildings start or end at the same x-coordinate.
    \index{Same Start or End}
    
    \item \textbf{Buildings with Zero Height}: Buildings that have zero height should not affect the skyline.
    \index{Buildings with Zero Height}
    
    \item \textbf{Large Number of Buildings}: Test with a large number of buildings to ensure performance and scalability.
    \index{Large Number of Buildings}
    
    \item \textbf{Buildings with Negative Coordinates}: Buildings positioned in negative coordinate space.
    \index{Negative Coordinates}
    
    \item \textbf{Boundary Values}: Buildings at the minimum and maximum limits of the coordinate range.
    \index{Boundary Values}
    
    \item \textbf{Buildings with Identical Coordinates}: Multiple buildings with the same coordinates.
    \index{Identical Coordinates}
    
    \item \textbf{Sequential Buildings}: Buildings placed sequentially without gaps.
    \index{Sequential Buildings}
    
    \item \textbf{Overlapping and Non-Overlapping Mixed}: A mix of overlapping and non-overlapping buildings.
    \index{Overlapping and Non-Overlapping Mixed}
    
    \item \textbf{Buildings with Very Large Heights}: Buildings with heights at the upper limit of the constraints.
    \index{Very Large Heights}
    
    \item \textbf{Empty Input}: No buildings are provided.
    \index{Empty Input}
\end{itemize}

\section*{Implementation Considerations}

When implementing the \texttt{getSkyline} function, keep in mind the following considerations to ensure robustness and efficiency:

\begin{itemize}
    \item \textbf{Data Type Selection}: Use appropriate data types that can handle large input values and avoid overflow or precision issues.
    \index{Data Type Selection}
    
    \item \textbf{Optimizing Event Sorting}: Efficiently sort events based on x-coordinates and heights to ensure correct processing order.
    \index{Optimizing Event Sorting}
    
    \item \textbf{Handling Large Inputs}: Design the algorithm to handle up to \(10^4\) buildings efficiently without significant performance degradation.
    \index{Handling Large Inputs}
    
    \item \textbf{Using Efficient Data Structures}: Utilize heaps, lists, and dictionaries effectively to manage and process events and active heights.
    \index{Efficient Data Structures}
    
    \item \textbf{Avoiding Redundant Calculations}: Ensure that distance and overlap calculations are performed only when necessary to optimize performance.
    \index{Avoiding Redundant Calculations}
    
    \item \textbf{Code Readability and Documentation}: Maintain clear and readable code through meaningful variable names and comprehensive comments to facilitate understanding and maintenance.
    \index{Code Readability}
    
    \item \textbf{Edge Case Handling}: Implement checks for edge cases to prevent incorrect results or runtime errors.
    \index{Edge Case Handling}
    
    \item \textbf{Implementing Helper Functions}: Create helper functions for tasks like distance calculation, event handling, and heap management to enhance modularity.
    \index{Helper Functions}
    
    \item \textbf{Consistent Naming Conventions}: Use consistent and descriptive naming conventions for variables and functions to improve code clarity.
    \index{Naming Conventions}
    
    \item \textbf{Memory Management}: Ensure that the algorithm manages memory efficiently, especially when dealing with large datasets.
    \index{Memory Management}
    
    \item \textbf{Implementing Iterative Solutions Carefully}: Ensure that loop conditions are correctly defined to prevent infinite loops or incorrect terminations.
    \index{Iterative Solutions}
    
    \item \textbf{Avoiding Floating-Point Precision Issues}: Since the problem deals with integers, floating-point precision is not a concern, simplifying the implementation.
    \index{Floating-Point Precision}
    
    \item \textbf{Testing and Validation}: Develop a comprehensive suite of test cases that cover all possible scenarios, including edge cases, to validate the correctness and efficiency of the implementation.
    \index{Testing and Validation}
    
    \item \textbf{Performance Considerations}: Optimize the loop conditions and operations to ensure that the function runs efficiently, especially for large input numbers.
    \index{Performance Considerations}
\end{itemize}

\section*{Conclusion}

The \textbf{Skyline Problem} is a quintessential example of applying advanced algorithmic techniques and geometric reasoning to solve complex spatial challenges. By leveraging the Sweep Line Algorithm and maintaining active building heights using a max heap, the solution efficiently constructs the skyline with optimal time and space complexities. Understanding and implementing such sophisticated algorithms not only enhances problem-solving skills but also provides a foundation for tackling a wide array of Computational Geometry problems in various domains, including computer graphics, urban planning simulations, and geographic information systems.

\printindex

% \input{sections/rectangle_overlap}
% \input{sections/rectangle_area}
% \input{sections/k_closest_points_to_origin}
% \input{sections/the_skyline_problem}
% % filename: rectangle_area.tex

\problemsection{Rectangle Area}
\label{chap:Rectangle_Area}
\marginnote{\href{https://leetcode.com/problems/rectangle-area/}{[LeetCode Link]}\index{LeetCode}}
\marginnote{\href{https://www.geeksforgeeks.org/find-area-two-overlapping-rectangles/}{[GeeksForGeeks Link]}\index{GeeksForGeeks}}
\marginnote{\href{https://www.interviewbit.com/problems/rectangle-area/}{[InterviewBit Link]}\index{InterviewBit}}
\marginnote{\href{https://app.codesignal.com/challenges/rectangle-area}{[CodeSignal Link]}\index{CodeSignal}}
\marginnote{\href{https://www.codewars.com/kata/rectangle-area/train/python}{[Codewars Link]}\index{Codewars}}

The \textbf{Rectangle Area} problem is a classic Computational Geometry challenge that involves calculating the total area covered by two axis-aligned rectangles in a 2D plane. This problem tests one's ability to perform geometric calculations, handle overlapping scenarios, and implement efficient algorithms. Mastery of this problem is essential for applications in computer graphics, spatial analysis, and computational modeling.

\section*{Problem Statement}

Given two axis-aligned rectangles in a 2D plane, compute the total area covered by the two rectangles. The area covered by the overlapping region should be counted only once.

Each rectangle is represented as a list of four integers \([x1, y1, x2, y2]\), where \((x1, y1)\) are the coordinates of the bottom-left corner, and \((x2, y2)\) are the coordinates of the top-right corner.

\textbf{Function signature in Python:}
\begin{lstlisting}[language=Python]
def computeArea(A: List[int], B: List[int]) -> int:
\end{lstlisting}

\section*{Examples}

\textbf{Example 1:}

\begin{verbatim}
Input: A = [-3,0,3,4], B = [0,-1,9,2]
Output: 45
Explanation:
Area of A = (3 - (-3)) * (4 - 0) = 6 * 4 = 24
Area of B = (9 - 0) * (2 - (-1)) = 9 * 3 = 27
Overlapping Area = (3 - 0) * (2 - 0) = 3 * 2 = 6
Total Area = 24 + 27 - 6 = 45
\end{verbatim}

\textbf{Example 2:}

\begin{verbatim}
Input: A = [0,0,0,0], B = [0,0,0,0]
Output: 0
Explanation:
Both rectangles are degenerate points with zero area.
\end{verbatim}

\textbf{Example 3:}

\begin{verbatim}
Input: A = [0,0,2,2], B = [1,1,3,3]
Output: 7
Explanation:
Area of A = 4
Area of B = 4
Overlapping Area = 1
Total Area = 4 + 4 - 1 = 7
\end{verbatim}

\textbf{Example 4:}

\begin{verbatim}
Input: A = [0,0,1,1], B = [1,0,2,1]
Output: 2
Explanation:
Rectangles touch at the edge but do not overlap.
Area of A = 1
Area of B = 1
Overlapping Area = 0
Total Area = 1 + 1 = 2
\end{verbatim}

\textbf{Constraints:}

\begin{itemize}
    \item All coordinates are integers in the range \([-10^9, 10^9]\).
    \item For each rectangle, \(x1 < x2\) and \(y1 < y2\).
\end{itemize}

LeetCode link: \href{https://leetcode.com/problems/rectangle-area/}{Rectangle Area}\index{LeetCode}

\section*{Algorithmic Approach}

To compute the total area covered by two axis-aligned rectangles, we can follow these steps:

1. **Calculate Individual Areas:**
   - Compute the area of the first rectangle.
   - Compute the area of the second rectangle.

2. **Determine Overlapping Area:**
   - Calculate the coordinates of the overlapping rectangle, if any.
   - If the rectangles overlap, compute the area of the overlapping region.

3. **Compute Total Area:**
   - Sum the individual areas and subtract the overlapping area to avoid double-counting.

\marginnote{This approach ensures accurate area calculation by handling overlapping regions appropriately.}

\section*{Complexities}

\begin{itemize}
    \item \textbf{Time Complexity:} \(O(1)\). The algorithm performs a constant number of calculations.
    
    \item \textbf{Space Complexity:} \(O(1)\). Only a fixed amount of extra space is used for variables.
\end{itemize}

\section*{Python Implementation}

\marginnote{Implementing the area calculation with overlap consideration ensures an accurate and efficient solution.}

Below is the complete Python code implementing the \texttt{computeArea} function:

\begin{fullwidth}
\begin{lstlisting}[language=Python]
from typing import List

class Solution:
    def computeArea(self, A: List[int], B: List[int]) -> int:
        # Calculate area of rectangle A
        areaA = (A[2] - A[0]) * (A[3] - A[1])
        
        # Calculate area of rectangle B
        areaB = (B[2] - B[0]) * (B[3] - B[1])
        
        # Determine overlap coordinates
        overlap_x1 = max(A[0], B[0])
        overlap_y1 = max(A[1], B[1])
        overlap_x2 = min(A[2], B[2])
        overlap_y2 = min(A[3], B[3])
        
        # Calculate overlapping area
        overlap_width = overlap_x2 - overlap_x1
        overlap_height = overlap_y2 - overlap_y1
        overlap_area = 0
        if overlap_width > 0 and overlap_height > 0:
            overlap_area = overlap_width * overlap_height
        
        # Total area is sum of individual areas minus overlapping area
        total_area = areaA + areaB - overlap_area
        return total_area

# Example usage:
solution = Solution()
print(solution.computeArea([-3,0,3,4], [0,-1,9,2]))  # Output: 45
print(solution.computeArea([0,0,0,0], [0,0,0,0]))    # Output: 0
print(solution.computeArea([0,0,2,2], [1,1,3,3]))    # Output: 7
print(solution.computeArea([0,0,1,1], [1,0,2,1]))    # Output: 2
\end{lstlisting}
\end{fullwidth}

This implementation accurately computes the total area covered by two rectangles by accounting for any overlapping regions. It ensures that the overlapping area is not double-counted.

\section*{Explanation}

The \texttt{computeArea} function calculates the combined area of two axis-aligned rectangles by following these steps:

\subsection*{1. Calculate Individual Areas}

\begin{itemize}
    \item **Rectangle A:**
    \begin{itemize}
        \item Width: \(A[2] - A[0]\)
        \item Height: \(A[3] - A[1]\)
        \item Area: Width \(\times\) Height
    \end{itemize}
    
    \item **Rectangle B:**
    \begin{itemize}
        \item Width: \(B[2] - B[0]\)
        \item Height: \(B[3] - B[1]\)
        \item Area: Width \(\times\) Height
    \end{itemize}
\end{itemize}

\subsection*{2. Determine Overlapping Area}

\begin{itemize}
    \item **Overlap Coordinates:**
    \begin{itemize}
        \item Left (x-coordinate): \(\text{max}(A[0], B[0])\)
        \item Bottom (y-coordinate): \(\text{max}(A[1], B[1])\)
        \item Right (x-coordinate): \(\text{min}(A[2], B[2])\)
        \item Top (y-coordinate): \(\text{min}(A[3], B[3])\)
    \end{itemize}
    
    \item **Overlap Dimensions:**
    \begin{itemize}
        \item Width: \(\text{overlap\_x2} - \text{overlap\_x1}\)
        \item Height: \(\text{overlap\_y2} - \text{overlap\_y1}\)
    \end{itemize}
    
    \item **Overlap Area:**
    \begin{itemize}
        \item If both width and height are positive, the rectangles overlap, and the overlapping area is their product.
        \item Otherwise, there is no overlap, and the overlapping area is zero.
    \end{itemize}
\end{itemize}

\subsection*{3. Compute Total Area}

\begin{itemize}
    \item Total Area = Area of Rectangle A + Area of Rectangle B - Overlapping Area
\end{itemize}

\subsection*{4. Example Walkthrough}

Consider the first example:
\begin{verbatim}
Input: A = [-3,0,3,4], B = [0,-1,9,2]
Output: 45
\end{verbatim}

\begin{enumerate}
    \item **Calculate Areas:**
    \begin{itemize}
        \item Area of A = (3 - (-3)) * (4 - 0) = 6 * 4 = 24
        \item Area of B = (9 - 0) * (2 - (-1)) = 9 * 3 = 27
    \end{itemize}
    
    \item **Determine Overlap:**
    \begin{itemize}
        \item overlap\_x1 = max(-3, 0) = 0
        \item overlap\_y1 = max(0, -1) = 0
        \item overlap\_x2 = min(3, 9) = 3
        \item overlap\_y2 = min(4, 2) = 2
        \item overlap\_width = 3 - 0 = 3
        \item overlap\_height = 2 - 0 = 2
        \item overlap\_area = 3 * 2 = 6
    \end{itemize}
    
    \item **Compute Total Area:**
    \begin{itemize}
        \item Total Area = 24 + 27 - 6 = 45
    \end{itemize}
\end{enumerate}

Thus, the function correctly returns \texttt{45}.

\section*{Why This Approach}

This approach is chosen for its straightforwardness and optimal efficiency. By directly calculating the individual areas and intelligently handling the overlapping region, the algorithm ensures accurate results without unnecessary computations. Its constant time complexity makes it highly efficient, even for large coordinate values.

\section*{Alternative Approaches}

\subsection*{1. Using Intersection Dimensions}

Instead of separately calculating areas, directly compute the dimensions of the overlapping region and subtract it from the sum of individual areas.

\begin{lstlisting}[language=Python]
def computeArea(A: List[int], B: List[int]) -> int:
    # Sum of individual areas
    area = (A[2] - A[0]) * (A[3] - A[1]) + (B[2] - B[0]) * (B[3] - B[1])
    
    # Overlapping area
    overlap_width = min(A[2], B[2]) - max(A[0], B[0])
    overlap_height = min(A[3], B[3]) - max(A[1], B[1])
    
    if overlap_width > 0 and overlap_height > 0:
        area -= overlap_width * overlap_height
    
    return area
\end{lstlisting}

\subsection*{2. Using Geometry Libraries}

Leverage computational geometry libraries to handle area calculations and overlapping detections.

\begin{lstlisting}[language=Python]
from shapely.geometry import box

def computeArea(A: List[int], B: List[int]) -> int:
    rect1 = box(A[0], A[1], A[2], A[3])
    rect2 = box(B[0], B[1], B[2], B[3])
    intersection = rect1.intersection(rect2)
    return int(rect1.area + rect2.area - intersection.area)
\end{lstlisting}

\textbf{Note}: This approach requires the \texttt{shapely} library and is more suitable for complex geometric operations.

\section*{Similar Problems to This One}

Several problems involve calculating areas, handling geometric overlaps, and spatial reasoning, utilizing similar algorithmic strategies:

\begin{itemize}
    \item \textbf{Rectangle Overlap}: Determine if two rectangles overlap.
    \item \textbf{Circle Area Overlap}: Calculate the overlapping area between two circles.
    \item \textbf{Polygon Area}: Compute the area of a given polygon.
    \item \textbf{Union of Rectangles}: Calculate the total area covered by multiple rectangles, accounting for overlaps.
    \item \textbf{Intersection of Lines}: Find the intersection point of two lines.
    \item \textbf{Closest Pair of Points}: Find the closest pair of points in a set.
    \item \textbf{Convex Hull}: Compute the convex hull of a set of points.
    \item \textbf{Point Inside Polygon}: Determine if a point lies inside a given polygon.
\end{itemize}

These problems reinforce concepts of geometric calculations, area computations, and efficient algorithm design in various contexts.

\section*{Things to Keep in Mind and Tricks}

When tackling the \textbf{Rectangle Area} problem, consider the following tips and best practices to enhance efficiency and correctness:

\begin{itemize}
    \item \textbf{Understand Geometric Relationships}: Grasp the positional relationships between rectangles to simplify area calculations.
    \index{Geometric Relationships}
    
    \item \textbf{Leverage Coordinate Comparisons}: Use direct comparisons of rectangle coordinates to determine overlapping regions.
    \index{Coordinate Comparisons}
    
    \item \textbf{Handle Overlapping Scenarios}: Accurately calculate the overlapping area to avoid double-counting.
    \index{Overlapping Scenarios}
    
    \item \textbf{Optimize for Efficiency}: Aim for a constant time \(O(1)\) solution by avoiding unnecessary computations or iterations.
    \index{Efficiency Optimization}
    
    \item \textbf{Avoid Floating-Point Precision Issues}: Since all coordinates are integers, floating-point precision is not a concern, simplifying the implementation.
    \index{Floating-Point Precision}
    
    \item \textbf{Use Helper Functions}: Create helper functions to encapsulate repetitive tasks, such as calculating overlap dimensions or areas.
    \index{Helper Functions}
    
    \item \textbf{Code Readability}: Maintain clear and readable code through meaningful variable names and structured logic.
    \index{Code Readability}
    
    \item \textbf{Test Extensively}: Implement a wide range of test cases, including overlapping, non-overlapping, and edge-touching rectangles, to ensure robustness.
    \index{Extensive Testing}
    
    \item \textbf{Understand Axis-Aligned Constraints}: Recognize that axis-aligned rectangles simplify area calculations compared to rotated rectangles.
    \index{Axis-Aligned Constraints}
    
    \item \textbf{Simplify Logical Conditions}: Combine multiple conditions logically to streamline the area calculation process.
    \index{Logical Conditions}
    
    \item \textbf{Use Absolute Values}: When calculating differences, ensure that the dimensions are positive by using absolute values or proper ordering.
    \index{Absolute Values}
    
    \item \textbf{Consider Edge Cases}: Handle cases where rectangles have zero area or touch at edges/corners without overlapping.
    \index{Edge Cases}
\end{itemize}

\section*{Corner and Special Cases to Test When Writing the Code}

When implementing the solution for the \textbf{Rectangle Area} problem, it is crucial to consider and rigorously test various edge cases to ensure robustness and correctness:

\begin{itemize}
    \item \textbf{No Overlap}: Rectangles are completely separate.
    \index{No Overlap}
    
    \item \textbf{Partial Overlap}: Rectangles overlap in one or more regions.
    \index{Partial Overlap}
    
    \item \textbf{Edge Touching}: Rectangles touch exactly at one edge without overlapping.
    \index{Edge Touching}
    
    \item \textbf{Corner Touching}: Rectangles touch exactly at one corner without overlapping.
    \index{Corner Touching}
    
    \item \textbf{One Rectangle Inside Another}: One rectangle is entirely within the other.
    \index{Rectangle Inside}
    
    \item \textbf{Identical Rectangles}: Both rectangles have the same coordinates.
    \index{Identical Rectangles}
    
    \item \textbf{Degenerate Rectangles}: Rectangles with zero area (e.g., \(x1 = x2\) or \(y1 = y2\)).
    \index{Degenerate Rectangles}
    
    \item \textbf{Large Coordinates}: Rectangles with very large coordinate values to test performance and integer handling.
    \index{Large Coordinates}
    
    \item \textbf{Negative Coordinates}: Rectangles positioned in negative coordinate space.
    \index{Negative Coordinates}
    
    \item \textbf{Mixed Overlapping Scenarios}: Combinations of the above cases to ensure comprehensive coverage.
    \index{Mixed Overlapping Scenarios}
    
    \item \textbf{Minimum and Maximum Bounds}: Rectangles at the minimum and maximum limits of the coordinate range.
    \index{Minimum and Maximum Bounds}
    
    \item \textbf{Sequential Rectangles}: Multiple rectangles placed sequentially without overlapping.
    \index{Sequential Rectangles}
    
    \item \textbf{Multiple Overlaps}: Scenarios where more than two rectangles overlap in different regions.
    \index{Multiple Overlaps}
\end{itemize}

\section*{Implementation Considerations}

When implementing the \texttt{computeArea} function, keep in mind the following considerations to ensure robustness and efficiency:

\begin{itemize}
    \item \textbf{Data Type Selection}: Use appropriate data types that can handle large input values without overflow or underflow.
    \index{Data Type Selection}
    
    \item \textbf{Optimizing Comparisons}: Structure logical conditions to efficiently determine overlap dimensions.
    \index{Optimizing Comparisons}
    
    \item \textbf{Handling Large Inputs}: Design the algorithm to efficiently handle large input sizes without significant performance degradation.
    \index{Handling Large Inputs}
    
    \item \textbf{Language-Specific Constraints}: Be aware of how the programming language handles large integers and arithmetic operations.
    \index{Language-Specific Constraints}
    
    \item \textbf{Avoiding Redundant Calculations}: Ensure that each calculation contributes towards determining the final area without unnecessary repetitions.
    \index{Avoiding Redundant Calculations}
    
    \item \textbf{Code Readability and Documentation}: Maintain clear and readable code through meaningful variable names and comprehensive comments to facilitate understanding and maintenance.
    \index{Code Readability}
    
    \item \textbf{Edge Case Handling}: Implement checks for edge cases to prevent incorrect results or runtime errors.
    \index{Edge Case Handling}
    
    \item \textbf{Testing and Validation}: Develop a comprehensive suite of test cases that cover all possible scenarios, including edge cases, to validate the correctness and efficiency of the implementation.
    \index{Testing and Validation}
    
    \item \textbf{Scalability}: Design the algorithm to scale efficiently with increasing input sizes, maintaining performance and resource utilization.
    \index{Scalability}
    
    \item \textbf{Using Helper Functions}: Consider creating helper functions for repetitive tasks, such as calculating overlap dimensions, to enhance modularity and reusability.
    \index{Helper Functions}
    
    \item \textbf{Consistent Naming Conventions}: Use consistent and descriptive naming conventions for variables to improve code clarity.
    \index{Naming Conventions}
    
    \item \textbf{Implementing Unit Tests}: Develop unit tests for each logical condition to ensure that all scenarios are correctly handled.
    \index{Unit Tests}
    
    \item \textbf{Error Handling}: Incorporate error handling to manage invalid inputs gracefully.
    \index{Error Handling}
\end{itemize}

\section*{Conclusion}

The \textbf{Rectangle Area} problem showcases the application of fundamental geometric principles and efficient algorithm design to compute spatial properties accurately. By systematically calculating individual areas and intelligently handling overlapping regions, the algorithm ensures precise results without redundant computations. Understanding and implementing such techniques not only enhances problem-solving skills but also provides a foundation for tackling more complex Computational Geometry challenges involving multiple geometric entities and intricate spatial relationships.

\printindex

% % filename: rectangle_overlap.tex

\problemsection{Rectangle Overlap}
\label{chap:Rectangle_Overlap}
\marginnote{\href{https://leetcode.com/problems/rectangle-overlap/}{[LeetCode Link]}\index{LeetCode}}
\marginnote{\href{https://www.geeksforgeeks.org/check-if-two-rectangles-overlap/}{[GeeksForGeeks Link]}\index{GeeksForGeeks}}
\marginnote{\href{https://www.interviewbit.com/problems/rectangle-overlap/}{[InterviewBit Link]}\index{InterviewBit}}
\marginnote{\href{https://app.codesignal.com/challenges/rectangle-overlap}{[CodeSignal Link]}\index{CodeSignal}}
\marginnote{\href{https://www.codewars.com/kata/rectangle-overlap/train/python}{[Codewars Link]}\index{Codewars}}

The \textbf{Rectangle Overlap} problem is a fundamental challenge in Computational Geometry that involves determining whether two axis-aligned rectangles overlap. This problem tests one's ability to understand geometric properties, implement conditional logic, and optimize for efficient computation. Mastery of this problem is essential for applications in computer graphics, collision detection, and spatial data analysis.

\section*{Problem Statement}

Given two axis-aligned rectangles in a 2D plane, determine if they overlap. Each rectangle is defined by its bottom-left and top-right coordinates.

A rectangle is represented as a list of four integers \([x1, y1, x2, y2]\), where \((x1, y1)\) are the coordinates of the bottom-left corner, and \((x2, y2)\) are the coordinates of the top-right corner.

\textbf{Function signature in Python:}
\begin{lstlisting}[language=Python]
def isRectangleOverlap(rec1: List[int], rec2: List[int]) -> bool:
\end{lstlisting}

\section*{Examples}

\textbf{Example 1:}

\begin{verbatim}
Input: rec1 = [0,0,2,2], rec2 = [1,1,3,3]
Output: True
Explanation: The rectangles overlap in the area defined by [1,1,2,2].
\end{verbatim}

\textbf{Example 2:}

\begin{verbatim}
Input: rec1 = [0,0,1,1], rec2 = [1,0,2,1]
Output: False
Explanation: The rectangles touch at the edge but do not overlap.
\end{verbatim}

\textbf{Example 3:}

\begin{verbatim}
Input: rec1 = [0,0,1,1], rec2 = [2,2,3,3]
Output: False
Explanation: The rectangles are completely separate.
\end{verbatim}

\textbf{Example 4:}

\begin{verbatim}
Input: rec1 = [0,0,5,5], rec2 = [3,3,7,7]
Output: True
Explanation: The rectangles overlap in the area defined by [3,3,5,5].
\end{verbatim}

\textbf{Example 5:}

\begin{verbatim}
Input: rec1 = [0,0,0,0], rec2 = [0,0,0,0]
Output: False
Explanation: Both rectangles are degenerate points.
\end{verbatim}

\textbf{Constraints:}

\begin{itemize}
    \item All coordinates are integers in the range \([-10^9, 10^9]\).
    \item For each rectangle, \(x1 < x2\) and \(y1 < y2\).
\end{itemize}

LeetCode link: \href{https://leetcode.com/problems/rectangle-overlap/}{Rectangle Overlap}\index{LeetCode}

\section*{Algorithmic Approach}

To determine whether two axis-aligned rectangles overlap, we can use the following logical conditions:

1. **Non-Overlap Conditions:**
   - One rectangle is to the left of the other.
   - One rectangle is above the other.

2. **Overlap Condition:**
   - If neither of the non-overlap conditions is true, the rectangles must overlap.

\subsection*{Steps:}

1. **Extract Coordinates:**
   - For both rectangles, extract the bottom-left and top-right coordinates.

2. **Check Non-Overlap Conditions:**
   - If the right side of the first rectangle is less than or equal to the left side of the second rectangle, they do not overlap.
   - If the left side of the first rectangle is greater than or equal to the right side of the second rectangle, they do not overlap.
   - If the top side of the first rectangle is less than or equal to the bottom side of the second rectangle, they do not overlap.
   - If the bottom side of the first rectangle is greater than or equal to the top side of the second rectangle, they do not overlap.

3. **Determine Overlap:**
   - If none of the non-overlap conditions are met, the rectangles overlap.

\marginnote{This approach provides an efficient \(O(1)\) time complexity solution by leveraging simple geometric comparisons.}

\section*{Complexities}

\begin{itemize}
    \item \textbf{Time Complexity:} \(O(1)\). The algorithm performs a constant number of comparisons regardless of input size.
    
    \item \textbf{Space Complexity:} \(O(1)\). Only a fixed amount of extra space is used for variables.
\end{itemize}

\section*{Python Implementation}

\marginnote{Implementing the overlap check using coordinate comparisons ensures an optimal and straightforward solution.}

Below is the complete Python code implementing the \texttt{isRectangleOverlap} function:

\begin{fullwidth}
\begin{lstlisting}[language=Python]
from typing import List

class Solution:
    def isRectangleOverlap(self, rec1: List[int], rec2: List[int]) -> bool:
        # Extract coordinates
        left1, bottom1, right1, top1 = rec1
        left2, bottom2, right2, top2 = rec2
        
        # Check non-overlapping conditions
        if right1 <= left2 or right2 <= left1:
            return False
        if top1 <= bottom2 or top2 <= bottom1:
            return False
        
        # If none of the above, rectangles overlap
        return True

# Example usage:
solution = Solution()
print(solution.isRectangleOverlap([0,0,2,2], [1,1,3,3]))  # Output: True
print(solution.isRectangleOverlap([0,0,1,1], [1,0,2,1]))  # Output: False
print(solution.isRectangleOverlap([0,0,1,1], [2,2,3,3]))  # Output: False
print(solution.isRectangleOverlap([0,0,5,5], [3,3,7,7]))  # Output: True
print(solution.isRectangleOverlap([0,0,0,0], [0,0,0,0]))  # Output: False
\end{lstlisting}
\end{fullwidth}

This implementation efficiently checks for overlap by comparing the coordinates of the two rectangles. If any of the non-overlapping conditions are met, it returns \texttt{False}; otherwise, it returns \texttt{True}.

\section*{Explanation}

The \texttt{isRectangleOverlap} function determines whether two axis-aligned rectangles overlap by comparing their respective coordinates. Here's a detailed breakdown of the implementation:

\subsection*{1. Extract Coordinates}

\begin{itemize}
    \item For each rectangle, extract the left (\(x1\)), bottom (\(y1\)), right (\(x2\)), and top (\(y2\)) coordinates.
    \item This simplifies the comparison process by providing clear variables representing each side of the rectangles.
\end{itemize}

\subsection*{2. Check Non-Overlap Conditions}

\begin{itemize}
    \item **Horizontal Separation:**
    \begin{itemize}
        \item If the right side of the first rectangle (\(right1\)) is less than or equal to the left side of the second rectangle (\(left2\)), there is no horizontal overlap.
        \item Similarly, if the right side of the second rectangle (\(right2\)) is less than or equal to the left side of the first rectangle (\(left1\)), there is no horizontal overlap.
    \end{itemize}
    
    \item **Vertical Separation:**
    \begin{itemize}
        \item If the top side of the first rectangle (\(top1\)) is less than or equal to the bottom side of the second rectangle (\(bottom2\)), there is no vertical overlap.
        \item Similarly, if the top side of the second rectangle (\(top2\)) is less than or equal to the bottom side of the first rectangle (\(bottom1\)), there is no vertical overlap.
    \end{itemize}
    
    \item If any of these non-overlapping conditions are true, the rectangles do not overlap, and the function returns \texttt{False}.
\end{itemize}

\subsection*{3. Determine Overlap}

\begin{itemize}
    \item If none of the non-overlapping conditions are met, it implies that the rectangles overlap both horizontally and vertically.
    \item The function returns \texttt{True} in this case.
\end{itemize}

\subsection*{4. Example Walkthrough}

Consider the first example:
\begin{verbatim}
Input: rec1 = [0,0,2,2], rec2 = [1,1,3,3]
Output: True
\end{verbatim}

\begin{enumerate}
    \item Extract coordinates:
    \begin{itemize}
        \item rec1: left1 = 0, bottom1 = 0, right1 = 2, top1 = 2
        \item rec2: left2 = 1, bottom2 = 1, right2 = 3, top2 = 3
    \end{itemize}
    
    \item Check non-overlap conditions:
    \begin{itemize}
        \item \(right1 = 2\) is not less than or equal to \(left2 = 1\)
        \item \(right2 = 3\) is not less than or equal to \(left1 = 0\)
        \item \(top1 = 2\) is not less than or equal to \(bottom2 = 1\)
        \item \(top2 = 3\) is not less than or equal to \(bottom1 = 0\)
    \end{itemize}
    
    \item Since none of the non-overlapping conditions are met, the rectangles overlap.
\end{enumerate}

Thus, the function correctly returns \texttt{True}.

\section*{Why This Approach}

This approach is chosen for its simplicity and efficiency. By leveraging direct coordinate comparisons, the algorithm achieves constant time complexity without the need for complex data structures or iterative processes. It effectively handles all possible scenarios of rectangle positioning, ensuring accurate detection of overlaps.

\section*{Alternative Approaches}

\subsection*{1. Separating Axis Theorem (SAT)}

The Separating Axis Theorem is a more generalized method for detecting overlaps between convex shapes. While it is not necessary for axis-aligned rectangles, understanding SAT can be beneficial for more complex geometric problems.

\begin{lstlisting}[language=Python]
def isRectangleOverlap(rec1: List[int], rec2: List[int]) -> bool:
    # Using SAT for axis-aligned rectangles
    return not (rec1[2] <= rec2[0] or rec1[0] >= rec2[2] or
                rec1[3] <= rec2[1] or rec1[1] >= rec2[3])
\end{lstlisting}

\textbf{Note}: This implementation is functionally identical to the primary approach but leverages a more generalized geometric theorem.

\subsection*{2. Area-Based Approach}

Calculate the overlapping area between the two rectangles. If the overlapping area is positive, the rectangles overlap.

\begin{lstlisting}[language=Python]
def isRectangleOverlap(rec1: List[int], rec2: List[int]) -> bool:
    # Calculate overlap in x and y dimensions
    x_overlap = min(rec1[2], rec2[2]) - max(rec1[0], rec2[0])
    y_overlap = min(rec1[3], rec2[3]) - max(rec1[1], rec2[1])
    
    # Overlap exists if both overlaps are positive
    return x_overlap > 0 and y_overlap > 0
\end{lstlisting}

\textbf{Complexities:}
\begin{itemize}
    \item \textbf{Time Complexity:} \(O(1)\)
    \item \textbf{Space Complexity:} \(O(1)\)
\end{itemize}

\subsection*{3. Using Rectangles Intersection Function}

Utilize built-in or library functions that handle geometric intersections.

\begin{lstlisting}[language=Python]
from shapely.geometry import box

def isRectangleOverlap(rec1: List[int], rec2: List[int]) -> bool:
    rectangle1 = box(rec1[0], rec1[1], rec1[2], rec1[3])
    rectangle2 = box(rec2[0], rec2[1], rec2[2], rec2[3])
    return rectangle1.intersects(rectangle2) and not rectangle1.touches(rectangle2)
\end{lstlisting}

\textbf{Note}: This approach requires the \texttt{shapely} library and is more suitable for complex geometric operations.

\section*{Similar Problems to This One}

Several problems revolve around geometric overlap, intersection detection, and spatial reasoning, utilizing similar algorithmic strategies:

\begin{itemize}
    \item \textbf{Interval Overlap}: Determine if two intervals on a line overlap.
    \item \textbf{Circle Overlap}: Determine if two circles overlap based on their radii and centers.
    \item \textbf{Polygon Overlap}: Determine if two polygons overlap using algorithms like SAT.
    \item \textbf{Closest Pair of Points}: Find the closest pair of points in a set.
    \item \textbf{Convex Hull}: Compute the convex hull of a set of points.
    \item \textbf{Intersection of Lines}: Find the intersection point of two lines.
    \item \textbf{Point Inside Polygon}: Determine if a point lies inside a given polygon.
\end{itemize}

These problems reinforce the concepts of spatial reasoning, geometric property analysis, and efficient algorithm design in various contexts.

\section*{Things to Keep in Mind and Tricks}

When working with the \textbf{Rectangle Overlap} problem, consider the following tips and best practices to enhance efficiency and correctness:

\begin{itemize}
    \item \textbf{Understand Geometric Relationships}: Grasp the positional relationships between rectangles to simplify overlap detection.
    \index{Geometric Relationships}
    
    \item \textbf{Leverage Coordinate Comparisons}: Use direct comparisons of rectangle coordinates to determine spatial relationships.
    \index{Coordinate Comparisons}
    
    \item \textbf{Handle Edge Cases}: Consider cases where rectangles touch at edges or corners without overlapping.
    \index{Edge Cases}
    
    \item \textbf{Optimize for Efficiency}: Aim for a constant time \(O(1)\) solution by avoiding unnecessary computations or iterations.
    \index{Efficiency Optimization}
    
    \item \textbf{Avoid Floating-Point Precision Issues}: Since all coordinates are integers, floating-point precision is not a concern, simplifying the implementation.
    \index{Floating-Point Precision}
    
    \item \textbf{Use Helper Functions}: Create helper functions to encapsulate repetitive tasks, such as extracting coordinates or checking specific conditions.
    \index{Helper Functions}
    
    \item \textbf{Code Readability}: Maintain clear and readable code through meaningful variable names and structured logic.
    \index{Code Readability}
    
    \item \textbf{Test Extensively}: Implement a wide range of test cases, including overlapping, non-overlapping, and edge-touching rectangles, to ensure robustness.
    \index{Extensive Testing}
    
    \item \textbf{Understand Axis-Aligned Constraints}: Recognize that axis-aligned rectangles simplify overlap detection compared to rotated rectangles.
    \index{Axis-Aligned Constraints}
    
    \item \textbf{Simplify Logical Conditions}: Combine multiple conditions logically to streamline the overlap detection process.
    \index{Logical Conditions}
\end{itemize}

\section*{Corner and Special Cases to Test When Writing the Code}

When implementing the solution for the \textbf{Rectangle Overlap} problem, it is crucial to consider and rigorously test various edge cases to ensure robustness and correctness:

\begin{itemize}
    \item \textbf{No Overlap}: Rectangles are completely separate.
    \index{No Overlap}
    
    \item \textbf{Partial Overlap}: Rectangles overlap in one or more regions.
    \index{Partial Overlap}
    
    \item \textbf{Edge Touching}: Rectangles touch exactly at one edge without overlapping.
    \index{Edge Touching}
    
    \item \textbf{Corner Touching}: Rectangles touch exactly at one corner without overlapping.
    \index{Corner Touching}
    
    \item \textbf{One Rectangle Inside Another}: One rectangle is entirely within the other.
    \index{Rectangle Inside}
    
    \item \textbf{Identical Rectangles}: Both rectangles have the same coordinates.
    \index{Identical Rectangles}
    
    \item \textbf{Degenerate Rectangles}: Rectangles with zero area (e.g., \(x1 = x2\) or \(y1 = y2\)).
    \index{Degenerate Rectangles}
    
    \item \textbf{Large Coordinates}: Rectangles with very large coordinate values to test performance and integer handling.
    \index{Large Coordinates}
    
    \item \textbf{Negative Coordinates}: Rectangles positioned in negative coordinate space.
    \index{Negative Coordinates}
    
    \item \textbf{Mixed Overlapping Scenarios}: Combinations of the above cases to ensure comprehensive coverage.
    \index{Mixed Overlapping Scenarios}
    
    \item \textbf{Minimum and Maximum Bounds}: Rectangles at the minimum and maximum limits of the coordinate range.
    \index{Minimum and Maximum Bounds}
\end{itemize}

\section*{Implementation Considerations}

When implementing the \texttt{isRectangleOverlap} function, keep in mind the following considerations to ensure robustness and efficiency:

\begin{itemize}
    \item \textbf{Data Type Selection}: Use appropriate data types that can handle the range of input values without overflow or underflow.
    \index{Data Type Selection}
    
    \item \textbf{Optimizing Comparisons}: Structure logical conditions to short-circuit evaluations as soon as a non-overlapping condition is met.
    \index{Optimizing Comparisons}
    
    \item \textbf{Language-Specific Constraints}: Be aware of how the programming language handles integer division and comparisons.
    \index{Language-Specific Constraints}
    
    \item \textbf{Avoiding Redundant Calculations}: Ensure that each comparison contributes towards determining overlap without unnecessary repetitions.
    \index{Avoiding Redundant Calculations}
    
    \item \textbf{Code Readability and Documentation}: Maintain clear and readable code through meaningful variable names and comprehensive comments to facilitate understanding and maintenance.
    \index{Code Readability}
    
    \item \textbf{Edge Case Handling}: Implement checks for edge cases to prevent incorrect results or runtime errors.
    \index{Edge Case Handling}
    
    \item \textbf{Testing and Validation}: Develop a comprehensive suite of test cases that cover all possible scenarios, including edge cases, to validate the correctness and efficiency of the implementation.
    \index{Testing and Validation}
    
    \item \textbf{Scalability}: Design the algorithm to scale efficiently with increasing input sizes, maintaining performance and resource utilization.
    \index{Scalability}
    
    \item \textbf{Using Helper Functions}: Consider creating helper functions for repetitive tasks, such as extracting and comparing coordinates, to enhance modularity and reusability.
    \index{Helper Functions}
    
    \item \textbf{Consistent Naming Conventions}: Use consistent and descriptive naming conventions for variables to improve code clarity.
    \index{Naming Conventions}
    
    \item \textbf{Handling Floating-Point Coordinates}: Although the problem specifies integer coordinates, ensure that the implementation can handle floating-point numbers if needed in extended scenarios.
    \index{Floating-Point Coordinates}
    
    \item \textbf{Avoiding Floating-Point Precision Issues}: Since all coordinates are integers, floating-point precision is not a concern, simplifying the implementation.
    \index{Floating-Point Precision}
    
    \item \textbf{Implementing Unit Tests}: Develop unit tests for each logical condition to ensure that all scenarios are correctly handled.
    \index{Unit Tests}
    
    \item \textbf{Error Handling}: Incorporate error handling to manage invalid inputs gracefully.
    \index{Error Handling}
\end{itemize}

\section*{Conclusion}

The \textbf{Rectangle Overlap} problem exemplifies the application of fundamental geometric principles and conditional logic to solve spatial challenges efficiently. By leveraging simple coordinate comparisons, the algorithm achieves optimal time and space complexities, making it highly suitable for real-time applications such as collision detection in gaming, layout planning in graphics, and spatial data analysis. Understanding and implementing such techniques not only enhances problem-solving skills but also provides a foundation for tackling more complex Computational Geometry problems involving varied geometric shapes and interactions.

\printindex

% \input{sections/rectangle_overlap}
% \input{sections/rectangle_area}
% \input{sections/k_closest_points_to_origin}
% \input{sections/the_skyline_problem}
% % filename: rectangle_area.tex

\problemsection{Rectangle Area}
\label{chap:Rectangle_Area}
\marginnote{\href{https://leetcode.com/problems/rectangle-area/}{[LeetCode Link]}\index{LeetCode}}
\marginnote{\href{https://www.geeksforgeeks.org/find-area-two-overlapping-rectangles/}{[GeeksForGeeks Link]}\index{GeeksForGeeks}}
\marginnote{\href{https://www.interviewbit.com/problems/rectangle-area/}{[InterviewBit Link]}\index{InterviewBit}}
\marginnote{\href{https://app.codesignal.com/challenges/rectangle-area}{[CodeSignal Link]}\index{CodeSignal}}
\marginnote{\href{https://www.codewars.com/kata/rectangle-area/train/python}{[Codewars Link]}\index{Codewars}}

The \textbf{Rectangle Area} problem is a classic Computational Geometry challenge that involves calculating the total area covered by two axis-aligned rectangles in a 2D plane. This problem tests one's ability to perform geometric calculations, handle overlapping scenarios, and implement efficient algorithms. Mastery of this problem is essential for applications in computer graphics, spatial analysis, and computational modeling.

\section*{Problem Statement}

Given two axis-aligned rectangles in a 2D plane, compute the total area covered by the two rectangles. The area covered by the overlapping region should be counted only once.

Each rectangle is represented as a list of four integers \([x1, y1, x2, y2]\), where \((x1, y1)\) are the coordinates of the bottom-left corner, and \((x2, y2)\) are the coordinates of the top-right corner.

\textbf{Function signature in Python:}
\begin{lstlisting}[language=Python]
def computeArea(A: List[int], B: List[int]) -> int:
\end{lstlisting}

\section*{Examples}

\textbf{Example 1:}

\begin{verbatim}
Input: A = [-3,0,3,4], B = [0,-1,9,2]
Output: 45
Explanation:
Area of A = (3 - (-3)) * (4 - 0) = 6 * 4 = 24
Area of B = (9 - 0) * (2 - (-1)) = 9 * 3 = 27
Overlapping Area = (3 - 0) * (2 - 0) = 3 * 2 = 6
Total Area = 24 + 27 - 6 = 45
\end{verbatim}

\textbf{Example 2:}

\begin{verbatim}
Input: A = [0,0,0,0], B = [0,0,0,0]
Output: 0
Explanation:
Both rectangles are degenerate points with zero area.
\end{verbatim}

\textbf{Example 3:}

\begin{verbatim}
Input: A = [0,0,2,2], B = [1,1,3,3]
Output: 7
Explanation:
Area of A = 4
Area of B = 4
Overlapping Area = 1
Total Area = 4 + 4 - 1 = 7
\end{verbatim}

\textbf{Example 4:}

\begin{verbatim}
Input: A = [0,0,1,1], B = [1,0,2,1]
Output: 2
Explanation:
Rectangles touch at the edge but do not overlap.
Area of A = 1
Area of B = 1
Overlapping Area = 0
Total Area = 1 + 1 = 2
\end{verbatim}

\textbf{Constraints:}

\begin{itemize}
    \item All coordinates are integers in the range \([-10^9, 10^9]\).
    \item For each rectangle, \(x1 < x2\) and \(y1 < y2\).
\end{itemize}

LeetCode link: \href{https://leetcode.com/problems/rectangle-area/}{Rectangle Area}\index{LeetCode}

\section*{Algorithmic Approach}

To compute the total area covered by two axis-aligned rectangles, we can follow these steps:

1. **Calculate Individual Areas:**
   - Compute the area of the first rectangle.
   - Compute the area of the second rectangle.

2. **Determine Overlapping Area:**
   - Calculate the coordinates of the overlapping rectangle, if any.
   - If the rectangles overlap, compute the area of the overlapping region.

3. **Compute Total Area:**
   - Sum the individual areas and subtract the overlapping area to avoid double-counting.

\marginnote{This approach ensures accurate area calculation by handling overlapping regions appropriately.}

\section*{Complexities}

\begin{itemize}
    \item \textbf{Time Complexity:} \(O(1)\). The algorithm performs a constant number of calculations.
    
    \item \textbf{Space Complexity:} \(O(1)\). Only a fixed amount of extra space is used for variables.
\end{itemize}

\section*{Python Implementation}

\marginnote{Implementing the area calculation with overlap consideration ensures an accurate and efficient solution.}

Below is the complete Python code implementing the \texttt{computeArea} function:

\begin{fullwidth}
\begin{lstlisting}[language=Python]
from typing import List

class Solution:
    def computeArea(self, A: List[int], B: List[int]) -> int:
        # Calculate area of rectangle A
        areaA = (A[2] - A[0]) * (A[3] - A[1])
        
        # Calculate area of rectangle B
        areaB = (B[2] - B[0]) * (B[3] - B[1])
        
        # Determine overlap coordinates
        overlap_x1 = max(A[0], B[0])
        overlap_y1 = max(A[1], B[1])
        overlap_x2 = min(A[2], B[2])
        overlap_y2 = min(A[3], B[3])
        
        # Calculate overlapping area
        overlap_width = overlap_x2 - overlap_x1
        overlap_height = overlap_y2 - overlap_y1
        overlap_area = 0
        if overlap_width > 0 and overlap_height > 0:
            overlap_area = overlap_width * overlap_height
        
        # Total area is sum of individual areas minus overlapping area
        total_area = areaA + areaB - overlap_area
        return total_area

# Example usage:
solution = Solution()
print(solution.computeArea([-3,0,3,4], [0,-1,9,2]))  # Output: 45
print(solution.computeArea([0,0,0,0], [0,0,0,0]))    # Output: 0
print(solution.computeArea([0,0,2,2], [1,1,3,3]))    # Output: 7
print(solution.computeArea([0,0,1,1], [1,0,2,1]))    # Output: 2
\end{lstlisting}
\end{fullwidth}

This implementation accurately computes the total area covered by two rectangles by accounting for any overlapping regions. It ensures that the overlapping area is not double-counted.

\section*{Explanation}

The \texttt{computeArea} function calculates the combined area of two axis-aligned rectangles by following these steps:

\subsection*{1. Calculate Individual Areas}

\begin{itemize}
    \item **Rectangle A:**
    \begin{itemize}
        \item Width: \(A[2] - A[0]\)
        \item Height: \(A[3] - A[1]\)
        \item Area: Width \(\times\) Height
    \end{itemize}
    
    \item **Rectangle B:**
    \begin{itemize}
        \item Width: \(B[2] - B[0]\)
        \item Height: \(B[3] - B[1]\)
        \item Area: Width \(\times\) Height
    \end{itemize}
\end{itemize}

\subsection*{2. Determine Overlapping Area}

\begin{itemize}
    \item **Overlap Coordinates:**
    \begin{itemize}
        \item Left (x-coordinate): \(\text{max}(A[0], B[0])\)
        \item Bottom (y-coordinate): \(\text{max}(A[1], B[1])\)
        \item Right (x-coordinate): \(\text{min}(A[2], B[2])\)
        \item Top (y-coordinate): \(\text{min}(A[3], B[3])\)
    \end{itemize}
    
    \item **Overlap Dimensions:**
    \begin{itemize}
        \item Width: \(\text{overlap\_x2} - \text{overlap\_x1}\)
        \item Height: \(\text{overlap\_y2} - \text{overlap\_y1}\)
    \end{itemize}
    
    \item **Overlap Area:**
    \begin{itemize}
        \item If both width and height are positive, the rectangles overlap, and the overlapping area is their product.
        \item Otherwise, there is no overlap, and the overlapping area is zero.
    \end{itemize}
\end{itemize}

\subsection*{3. Compute Total Area}

\begin{itemize}
    \item Total Area = Area of Rectangle A + Area of Rectangle B - Overlapping Area
\end{itemize}

\subsection*{4. Example Walkthrough}

Consider the first example:
\begin{verbatim}
Input: A = [-3,0,3,4], B = [0,-1,9,2]
Output: 45
\end{verbatim}

\begin{enumerate}
    \item **Calculate Areas:**
    \begin{itemize}
        \item Area of A = (3 - (-3)) * (4 - 0) = 6 * 4 = 24
        \item Area of B = (9 - 0) * (2 - (-1)) = 9 * 3 = 27
    \end{itemize}
    
    \item **Determine Overlap:**
    \begin{itemize}
        \item overlap\_x1 = max(-3, 0) = 0
        \item overlap\_y1 = max(0, -1) = 0
        \item overlap\_x2 = min(3, 9) = 3
        \item overlap\_y2 = min(4, 2) = 2
        \item overlap\_width = 3 - 0 = 3
        \item overlap\_height = 2 - 0 = 2
        \item overlap\_area = 3 * 2 = 6
    \end{itemize}
    
    \item **Compute Total Area:**
    \begin{itemize}
        \item Total Area = 24 + 27 - 6 = 45
    \end{itemize}
\end{enumerate}

Thus, the function correctly returns \texttt{45}.

\section*{Why This Approach}

This approach is chosen for its straightforwardness and optimal efficiency. By directly calculating the individual areas and intelligently handling the overlapping region, the algorithm ensures accurate results without unnecessary computations. Its constant time complexity makes it highly efficient, even for large coordinate values.

\section*{Alternative Approaches}

\subsection*{1. Using Intersection Dimensions}

Instead of separately calculating areas, directly compute the dimensions of the overlapping region and subtract it from the sum of individual areas.

\begin{lstlisting}[language=Python]
def computeArea(A: List[int], B: List[int]) -> int:
    # Sum of individual areas
    area = (A[2] - A[0]) * (A[3] - A[1]) + (B[2] - B[0]) * (B[3] - B[1])
    
    # Overlapping area
    overlap_width = min(A[2], B[2]) - max(A[0], B[0])
    overlap_height = min(A[3], B[3]) - max(A[1], B[1])
    
    if overlap_width > 0 and overlap_height > 0:
        area -= overlap_width * overlap_height
    
    return area
\end{lstlisting}

\subsection*{2. Using Geometry Libraries}

Leverage computational geometry libraries to handle area calculations and overlapping detections.

\begin{lstlisting}[language=Python]
from shapely.geometry import box

def computeArea(A: List[int], B: List[int]) -> int:
    rect1 = box(A[0], A[1], A[2], A[3])
    rect2 = box(B[0], B[1], B[2], B[3])
    intersection = rect1.intersection(rect2)
    return int(rect1.area + rect2.area - intersection.area)
\end{lstlisting}

\textbf{Note}: This approach requires the \texttt{shapely} library and is more suitable for complex geometric operations.

\section*{Similar Problems to This One}

Several problems involve calculating areas, handling geometric overlaps, and spatial reasoning, utilizing similar algorithmic strategies:

\begin{itemize}
    \item \textbf{Rectangle Overlap}: Determine if two rectangles overlap.
    \item \textbf{Circle Area Overlap}: Calculate the overlapping area between two circles.
    \item \textbf{Polygon Area}: Compute the area of a given polygon.
    \item \textbf{Union of Rectangles}: Calculate the total area covered by multiple rectangles, accounting for overlaps.
    \item \textbf{Intersection of Lines}: Find the intersection point of two lines.
    \item \textbf{Closest Pair of Points}: Find the closest pair of points in a set.
    \item \textbf{Convex Hull}: Compute the convex hull of a set of points.
    \item \textbf{Point Inside Polygon}: Determine if a point lies inside a given polygon.
\end{itemize}

These problems reinforce concepts of geometric calculations, area computations, and efficient algorithm design in various contexts.

\section*{Things to Keep in Mind and Tricks}

When tackling the \textbf{Rectangle Area} problem, consider the following tips and best practices to enhance efficiency and correctness:

\begin{itemize}
    \item \textbf{Understand Geometric Relationships}: Grasp the positional relationships between rectangles to simplify area calculations.
    \index{Geometric Relationships}
    
    \item \textbf{Leverage Coordinate Comparisons}: Use direct comparisons of rectangle coordinates to determine overlapping regions.
    \index{Coordinate Comparisons}
    
    \item \textbf{Handle Overlapping Scenarios}: Accurately calculate the overlapping area to avoid double-counting.
    \index{Overlapping Scenarios}
    
    \item \textbf{Optimize for Efficiency}: Aim for a constant time \(O(1)\) solution by avoiding unnecessary computations or iterations.
    \index{Efficiency Optimization}
    
    \item \textbf{Avoid Floating-Point Precision Issues}: Since all coordinates are integers, floating-point precision is not a concern, simplifying the implementation.
    \index{Floating-Point Precision}
    
    \item \textbf{Use Helper Functions}: Create helper functions to encapsulate repetitive tasks, such as calculating overlap dimensions or areas.
    \index{Helper Functions}
    
    \item \textbf{Code Readability}: Maintain clear and readable code through meaningful variable names and structured logic.
    \index{Code Readability}
    
    \item \textbf{Test Extensively}: Implement a wide range of test cases, including overlapping, non-overlapping, and edge-touching rectangles, to ensure robustness.
    \index{Extensive Testing}
    
    \item \textbf{Understand Axis-Aligned Constraints}: Recognize that axis-aligned rectangles simplify area calculations compared to rotated rectangles.
    \index{Axis-Aligned Constraints}
    
    \item \textbf{Simplify Logical Conditions}: Combine multiple conditions logically to streamline the area calculation process.
    \index{Logical Conditions}
    
    \item \textbf{Use Absolute Values}: When calculating differences, ensure that the dimensions are positive by using absolute values or proper ordering.
    \index{Absolute Values}
    
    \item \textbf{Consider Edge Cases}: Handle cases where rectangles have zero area or touch at edges/corners without overlapping.
    \index{Edge Cases}
\end{itemize}

\section*{Corner and Special Cases to Test When Writing the Code}

When implementing the solution for the \textbf{Rectangle Area} problem, it is crucial to consider and rigorously test various edge cases to ensure robustness and correctness:

\begin{itemize}
    \item \textbf{No Overlap}: Rectangles are completely separate.
    \index{No Overlap}
    
    \item \textbf{Partial Overlap}: Rectangles overlap in one or more regions.
    \index{Partial Overlap}
    
    \item \textbf{Edge Touching}: Rectangles touch exactly at one edge without overlapping.
    \index{Edge Touching}
    
    \item \textbf{Corner Touching}: Rectangles touch exactly at one corner without overlapping.
    \index{Corner Touching}
    
    \item \textbf{One Rectangle Inside Another}: One rectangle is entirely within the other.
    \index{Rectangle Inside}
    
    \item \textbf{Identical Rectangles}: Both rectangles have the same coordinates.
    \index{Identical Rectangles}
    
    \item \textbf{Degenerate Rectangles}: Rectangles with zero area (e.g., \(x1 = x2\) or \(y1 = y2\)).
    \index{Degenerate Rectangles}
    
    \item \textbf{Large Coordinates}: Rectangles with very large coordinate values to test performance and integer handling.
    \index{Large Coordinates}
    
    \item \textbf{Negative Coordinates}: Rectangles positioned in negative coordinate space.
    \index{Negative Coordinates}
    
    \item \textbf{Mixed Overlapping Scenarios}: Combinations of the above cases to ensure comprehensive coverage.
    \index{Mixed Overlapping Scenarios}
    
    \item \textbf{Minimum and Maximum Bounds}: Rectangles at the minimum and maximum limits of the coordinate range.
    \index{Minimum and Maximum Bounds}
    
    \item \textbf{Sequential Rectangles}: Multiple rectangles placed sequentially without overlapping.
    \index{Sequential Rectangles}
    
    \item \textbf{Multiple Overlaps}: Scenarios where more than two rectangles overlap in different regions.
    \index{Multiple Overlaps}
\end{itemize}

\section*{Implementation Considerations}

When implementing the \texttt{computeArea} function, keep in mind the following considerations to ensure robustness and efficiency:

\begin{itemize}
    \item \textbf{Data Type Selection}: Use appropriate data types that can handle large input values without overflow or underflow.
    \index{Data Type Selection}
    
    \item \textbf{Optimizing Comparisons}: Structure logical conditions to efficiently determine overlap dimensions.
    \index{Optimizing Comparisons}
    
    \item \textbf{Handling Large Inputs}: Design the algorithm to efficiently handle large input sizes without significant performance degradation.
    \index{Handling Large Inputs}
    
    \item \textbf{Language-Specific Constraints}: Be aware of how the programming language handles large integers and arithmetic operations.
    \index{Language-Specific Constraints}
    
    \item \textbf{Avoiding Redundant Calculations}: Ensure that each calculation contributes towards determining the final area without unnecessary repetitions.
    \index{Avoiding Redundant Calculations}
    
    \item \textbf{Code Readability and Documentation}: Maintain clear and readable code through meaningful variable names and comprehensive comments to facilitate understanding and maintenance.
    \index{Code Readability}
    
    \item \textbf{Edge Case Handling}: Implement checks for edge cases to prevent incorrect results or runtime errors.
    \index{Edge Case Handling}
    
    \item \textbf{Testing and Validation}: Develop a comprehensive suite of test cases that cover all possible scenarios, including edge cases, to validate the correctness and efficiency of the implementation.
    \index{Testing and Validation}
    
    \item \textbf{Scalability}: Design the algorithm to scale efficiently with increasing input sizes, maintaining performance and resource utilization.
    \index{Scalability}
    
    \item \textbf{Using Helper Functions}: Consider creating helper functions for repetitive tasks, such as calculating overlap dimensions, to enhance modularity and reusability.
    \index{Helper Functions}
    
    \item \textbf{Consistent Naming Conventions}: Use consistent and descriptive naming conventions for variables to improve code clarity.
    \index{Naming Conventions}
    
    \item \textbf{Implementing Unit Tests}: Develop unit tests for each logical condition to ensure that all scenarios are correctly handled.
    \index{Unit Tests}
    
    \item \textbf{Error Handling}: Incorporate error handling to manage invalid inputs gracefully.
    \index{Error Handling}
\end{itemize}

\section*{Conclusion}

The \textbf{Rectangle Area} problem showcases the application of fundamental geometric principles and efficient algorithm design to compute spatial properties accurately. By systematically calculating individual areas and intelligently handling overlapping regions, the algorithm ensures precise results without redundant computations. Understanding and implementing such techniques not only enhances problem-solving skills but also provides a foundation for tackling more complex Computational Geometry challenges involving multiple geometric entities and intricate spatial relationships.

\printindex

% \input{sections/rectangle_overlap}
% \input{sections/rectangle_area}
% \input{sections/k_closest_points_to_origin}
% \input{sections/the_skyline_problem}
% % filename: k_closest_points_to_origin.tex

\problemsection{K Closest Points to Origin}
\label{chap:K_Closest_Points_to_Origin}
\marginnote{\href{https://leetcode.com/problems/k-closest-points-to-origin/}{[LeetCode Link]}\index{LeetCode}}
\marginnote{\href{https://www.geeksforgeeks.org/find-k-closest-points-origin/}{[GeeksForGeeks Link]}\index{GeeksForGeeks}}
\marginnote{\href{https://www.interviewbit.com/problems/k-closest-points/}{[InterviewBit Link]}\index{InterviewBit}}
\marginnote{\href{https://app.codesignal.com/challenges/k-closest-points-to-origin}{[CodeSignal Link]}\index{CodeSignal}}
\marginnote{\href{https://www.codewars.com/kata/k-closest-points-to-origin/train/python}{[Codewars Link]}\index{Codewars}}

The \textbf{K Closest Points to Origin} problem is a popular algorithmic challenge in Computational Geometry that involves identifying the \(k\) points closest to the origin in a 2D plane. This problem tests one's ability to apply efficient sorting and selection algorithms, understand distance computations, and optimize for performance. Mastery of this problem is essential for applications in spatial data analysis, nearest neighbor searches, and clustering algorithms.

\section*{Problem Statement}

Given an array of points where each point is represented as \([x, y]\) in the 2D plane, and an integer \(k\), return the \(k\) closest points to the origin \((0, 0)\).

The distance between two points \((x_1, y_1)\) and \((x_2, y_2)\) is the Euclidean distance \(\sqrt{(x_1 - x_2)^2 + (y_1 - y_2)^2}\). The origin is \((0, 0)\).

\textbf{Function signature in Python:}
\begin{lstlisting}[language=Python]
def kClosest(points: List[List[int]], K: int) -> List[List[int]]:
\end{lstlisting}

\section*{Examples}

\textbf{Example 1:}

\begin{verbatim}
Input: points = [[1,3],[-2,2]], K = 1
Output: [[-2,2]]
Explanation: 
The distance between (1, 3) and the origin is sqrt(10).
The distance between (-2, 2) and the origin is sqrt(8).
Since sqrt(8) < sqrt(10), (-2, 2) is closer to the origin.
\end{verbatim}

\textbf{Example 2:}

\begin{verbatim}
Input: points = [[3,3],[5,-1],[-2,4]], K = 2
Output: [[3,3],[-2,4]]
Explanation: 
The distances are sqrt(18), sqrt(26), and sqrt(20) respectively.
The two closest points are [3,3] and [-2,4].
\end{verbatim}

\textbf{Example 3:}

\begin{verbatim}
Input: points = [[0,1],[1,0]], K = 2
Output: [[0,1],[1,0]]
Explanation: 
Both points are equally close to the origin.
\end{verbatim}

\textbf{Example 4:}

\begin{verbatim}
Input: points = [[1,0],[0,1]], K = 1
Output: [[1,0]]
Explanation: 
Both points are equally close; returning any one is acceptable.
\end{verbatim}

\textbf{Constraints:}

\begin{itemize}
    \item \(1 \leq K \leq \text{points.length} \leq 10^4\)
    \item \(-10^4 < x_i, y_i < 10^4\)
\end{itemize}

LeetCode link: \href{https://leetcode.com/problems/k-closest-points-to-origin/}{K Closest Points to Origin}\index{LeetCode}

\section*{Algorithmic Approach}

To identify the \(k\) closest points to the origin, several algorithmic strategies can be employed. The most efficient methods aim to reduce the time complexity by avoiding the need to sort the entire list of points.

\subsection*{1. Sorting Based on Distance}

Calculate the Euclidean distance of each point from the origin and sort the points based on these distances. Select the first \(k\) points from the sorted list.

\begin{enumerate}
    \item Compute the distance for each point using the formula \(distance = x^2 + y^2\).
    \item Sort the points based on the computed distances.
    \item Return the first \(k\) points from the sorted list.
\end{enumerate}

\subsection*{2. Max Heap (Priority Queue)}

Use a max heap to maintain the \(k\) closest points. Iterate through each point, add it to the heap, and if the heap size exceeds \(k\), remove the farthest point.

\begin{enumerate}
    \item Initialize a max heap.
    \item For each point, compute its distance and add it to the heap.
    \item If the heap size exceeds \(k\), remove the point with the largest distance.
    \item After processing all points, the heap contains the \(k\) closest points.
\end{enumerate}

\subsection*{3. QuickSelect (Quick Sort Partitioning)}

Utilize the QuickSelect algorithm to find the \(k\) closest points without fully sorting the list.

\begin{enumerate}
    \item Choose a pivot point and partition the list based on distances relative to the pivot.
    \item Recursively apply QuickSelect to the partition containing the \(k\) closest points.
    \item Once the \(k\) closest points are identified, return them.
\end{enumerate}

\marginnote{QuickSelect offers an average time complexity of \(O(n)\), making it highly efficient for large datasets.}

\section*{Complexities}

\begin{itemize}
    \item \textbf{Sorting Based on Distance:}
    \begin{itemize}
        \item \textbf{Time Complexity:} \(O(n \log n)\)
        \item \textbf{Space Complexity:} \(O(n)\)
    \end{itemize}
    
    \item \textbf{Max Heap (Priority Queue):}
    \begin{itemize}
        \item \textbf{Time Complexity:} \(O(n \log k)\)
        \item \textbf{Space Complexity:} \(O(k)\)
    \end{itemize}
    
    \item \textbf{QuickSelect (Quick Sort Partitioning):}
    \begin{itemize}
        \item \textbf{Time Complexity:} Average case \(O(n)\), worst case \(O(n^2)\)
        \item \textbf{Space Complexity:} \(O(1)\) (in-place)
    \end{itemize}
\end{itemize}

\section*{Python Implementation}

\marginnote{Implementing QuickSelect provides an optimal average-case solution with linear time complexity.}

Below is the complete Python code implementing the \texttt{kClosest} function using the QuickSelect approach:

\begin{fullwidth}
\begin{lstlisting}[language=Python]
from typing import List
import random

class Solution:
    def kClosest(self, points: List[List[int]], K: int) -> List[List[int]]:
        def quickselect(left, right, K_smallest):
            if left == right:
                return
            
            # Select a random pivot_index
            pivot_index = random.randint(left, right)
            
            # Partition the array
            pivot_index = partition(left, right, pivot_index)
            
            # The pivot is in its final sorted position
            if K_smallest == pivot_index:
                return
            elif K_smallest < pivot_index:
                quickselect(left, pivot_index - 1, K_smallest)
            else:
                quickselect(pivot_index + 1, right, K_smallest)
        
        def partition(left, right, pivot_index):
            pivot_distance = distance(points[pivot_index])
            # Move pivot to end
            points[pivot_index], points[right] = points[right], points[pivot_index]
            store_index = left
            for i in range(left, right):
                if distance(points[i]) < pivot_distance:
                    points[store_index], points[i] = points[i], points[store_index]
                    store_index += 1
            # Move pivot to its final place
            points[right], points[store_index] = points[store_index], points[right]
            return store_index
        
        def distance(point):
            return point[0] ** 2 + point[1] ** 2
        
        n = len(points)
        quickselect(0, n - 1, K)
        return points[:K]

# Example usage:
solution = Solution()
print(solution.kClosest([[1,3],[-2,2]], 1))            # Output: [[-2,2]]
print(solution.kClosest([[3,3],[5,-1],[-2,4]], 2))     # Output: [[3,3],[-2,4]]
print(solution.kClosest([[0,1],[1,0]], 2))             # Output: [[0,1],[1,0]]
print(solution.kClosest([[1,0],[0,1]], 1))             # Output: [[1,0]] or [[0,1]]
\end{lstlisting}
\end{fullwidth}

This implementation uses the QuickSelect algorithm to efficiently find the \(k\) closest points to the origin without fully sorting the entire list. It ensures optimal performance even with large datasets.

\section*{Explanation}

The \texttt{kClosest} function identifies the \(k\) closest points to the origin using the QuickSelect algorithm. Here's a detailed breakdown of the implementation:

\subsection*{1. Distance Calculation}

\begin{itemize}
    \item The Euclidean distance is calculated as \(distance = x^2 + y^2\). Since we only need relative distances for comparison, the square root is omitted for efficiency.
\end{itemize}

\subsection*{2. QuickSelect Algorithm}

\begin{itemize}
    \item **Pivot Selection:**
    \begin{itemize}
        \item A random pivot is chosen to enhance the average-case performance.
    \end{itemize}
    
    \item **Partitioning:**
    \begin{itemize}
        \item The array is partitioned such that points with distances less than the pivot are moved to the left, and others to the right.
        \item The pivot is placed in its correct sorted position.
    \end{itemize}
    
    \item **Recursive Selection:**
    \begin{itemize}
        \item If the pivot's position matches \(K\), the selection is complete.
        \item Otherwise, recursively apply QuickSelect to the relevant partition.
    \end{itemize}
\end{itemize}

\subsection*{3. Final Selection}

\begin{itemize}
    \item After partitioning, the first \(K\) points in the list are the \(k\) closest points to the origin.
\end{itemize}

\subsection*{4. Example Walkthrough}

Consider the first example:
\begin{verbatim}
Input: points = [[1,3],[-2,2]], K = 1
Output: [[-2,2]]
\end{verbatim}

\begin{enumerate}
    \item **Calculate Distances:**
    \begin{itemize}
        \item [1,3] : \(1^2 + 3^2 = 10\)
        \item [-2,2] : \((-2)^2 + 2^2 = 8\)
    \end{itemize}
    
    \item **QuickSelect Process:**
    \begin{itemize}
        \item Choose a pivot, say [1,3] with distance 10.
        \item Compare and rearrange:
        \begin{itemize}
            \item [-2,2] has a smaller distance (8) and is moved to the left.
        \end{itemize}
        \item After partitioning, the list becomes [[-2,2], [1,3]].
        \item Since \(K = 1\), return the first point: [[-2,2]].
    \end{itemize}
\end{enumerate}

Thus, the function correctly identifies \([-2,2]\) as the closest point to the origin.

\section*{Why This Approach}

The QuickSelect algorithm is chosen for its average-case linear time complexity \(O(n)\), making it highly efficient for large datasets compared to sorting-based methods with \(O(n \log n)\) time complexity. By avoiding the need to sort the entire list, QuickSelect provides an optimal solution for finding the \(k\) closest points.

\section*{Alternative Approaches}

\subsection*{1. Sorting Based on Distance}

Sort all points based on their distances from the origin and select the first \(k\) points.

\begin{lstlisting}[language=Python]
class Solution:
    def kClosest(self, points: List[List[int]], K: int) -> List[List[int]]:
        points.sort(key=lambda P: P[0]**2 + P[1]**2)
        return points[:K]
\end{lstlisting}

\textbf{Complexities:}
\begin{itemize}
    \item \textbf{Time Complexity:} \(O(n \log n)\)
    \item \textbf{Space Complexity:} \(O(1)\)
\end{itemize}

\subsection*{2. Max Heap (Priority Queue)}

Use a max heap to maintain the \(k\) closest points.

\begin{lstlisting}[language=Python]
import heapq

class Solution:
    def kClosest(self, points: List[List[int]], K: int) -> List[List[int]]:
        heap = []
        for (x, y) in points:
            dist = -(x**2 + y**2)  # Max heap using negative distances
            heapq.heappush(heap, (dist, [x, y]))
            if len(heap) > K:
                heapq.heappop(heap)
        return [item[1] for item in heap]
\end{lstlisting}

\textbf{Complexities:}
\begin{itemize}
    \item \textbf{Time Complexity:} \(O(n \log k)\)
    \item \textbf{Space Complexity:} \(O(k)\)
\end{itemize}

\subsection*{3. Using Built-In Functions}

Leverage built-in functions for distance calculation and selection.

\begin{lstlisting}[language=Python]
import math

class Solution:
    def kClosest(self, points: List[List[int]], K: int) -> List[List[int]]:
        points.sort(key=lambda P: math.sqrt(P[0]**2 + P[1]**2))
        return points[:K]
\end{lstlisting}

\textbf{Note}: This method is similar to the sorting approach but uses the actual Euclidean distance.

\section*{Similar Problems to This One}

Several problems involve nearest neighbor searches, spatial data analysis, and efficient selection algorithms, utilizing similar algorithmic strategies:

\begin{itemize}
    \item \textbf{Closest Pair of Points}: Find the closest pair of points in a set.
    \item \textbf{Top K Frequent Elements}: Identify the most frequent elements in a dataset.
    \item \textbf{Kth Largest Element in an Array}: Find the \(k\)-th largest element in an unsorted array.
    \item \textbf{Sliding Window Maximum}: Find the maximum in each sliding window of size \(k\) over an array.
    \item \textbf{Merge K Sorted Lists}: Merge multiple sorted lists into a single sorted list.
    \item \textbf{Find Median from Data Stream}: Continuously find the median of a stream of numbers.
    \item \textbf{Top K Closest Stars}: Find the \(k\) closest stars to Earth based on their distances.
\end{itemize}

These problems reinforce concepts of efficient selection, heap usage, and distance computations in various contexts.

\section*{Things to Keep in Mind and Tricks}

When solving the \textbf{K Closest Points to Origin} problem, consider the following tips and best practices to enhance efficiency and correctness:

\begin{itemize}
    \item \textbf{Understand Distance Calculations}: Grasp the Euclidean distance formula and recognize that the square root can be omitted for comparison purposes.
    \index{Distance Calculations}
    
    \item \textbf{Leverage Efficient Algorithms}: Use QuickSelect or heap-based methods to optimize time complexity, especially for large datasets.
    \index{Efficient Algorithms}
    
    \item \textbf{Handle Ties Appropriately}: Decide how to handle points with identical distances when \(k\) is less than the number of such points.
    \index{Handling Ties}
    
    \item \textbf{Optimize Space Usage}: Choose algorithms that minimize additional space, such as in-place QuickSelect.
    \index{Space Optimization}
    
    \item \textbf{Use Appropriate Data Structures}: Utilize heaps, lists, and helper functions effectively to manage and process data.
    \index{Data Structures}
    
    \item \textbf{Implement Helper Functions}: Create helper functions for distance calculation and partitioning to enhance code modularity.
    \index{Helper Functions}
    
    \item \textbf{Code Readability}: Maintain clear and readable code through meaningful variable names and structured logic.
    \index{Code Readability}
    
    \item \textbf{Test Extensively}: Implement a wide range of test cases, including edge cases like multiple points with the same distance, to ensure robustness.
    \index{Extensive Testing}
    
    \item \textbf{Understand Algorithm Trade-offs}: Recognize the trade-offs between different approaches in terms of time and space complexities.
    \index{Algorithm Trade-offs}
    
    \item \textbf{Use Built-In Sorting Functions}: When using sorting-based approaches, leverage built-in functions for efficiency and simplicity.
    \index{Built-In Sorting}
    
    \item \textbf{Avoid Redundant Calculations}: Ensure that distance calculations are performed only when necessary to optimize performance.
    \index{Avoiding Redundant Calculations}
    
    \item \textbf{Language-Specific Features}: Utilize language-specific features or libraries that can simplify implementation, such as heapq in Python.
    \index{Language-Specific Features}
\end{itemize}

\section*{Corner and Special Cases to Test When Writing the Code}

When implementing the solution for the \textbf{K Closest Points to Origin} problem, it is crucial to consider and rigorously test various edge cases to ensure robustness and correctness:

\begin{itemize}
    \item \textbf{Multiple Points with Same Distance}: Ensure that the algorithm handles multiple points having the same distance from the origin.
    \index{Same Distance Points}
    
    \item \textbf{Points at Origin}: Include points that are exactly at the origin \((0,0)\).
    \index{Points at Origin}
    
    \item \textbf{Negative Coordinates}: Ensure that the algorithm correctly computes distances for points with negative \(x\) or \(y\) coordinates.
    \index{Negative Coordinates}
    
    \item \textbf{Large Coordinates}: Test with points having very large or very small coordinate values to verify integer handling.
    \index{Large Coordinates}
    
    \item \textbf{K Equals Number of Points}: When \(K\) is equal to the number of points, the algorithm should return all points.
    \index{K Equals Number of Points}
    
    \item \textbf{K is One}: Test with \(K = 1\) to ensure the closest point is correctly identified.
    \index{K is One}
    
    \item \textbf{All Points Same}: All points have the same coordinates.
    \index{All Points Same}
    
    \item \textbf{K is Zero}: Although \(K\) is defined to be at least 1, ensure that the algorithm gracefully handles \(K = 0\) if allowed.
    \index{K is Zero}
    
    \item \textbf{Single Point}: Only one point is provided, and \(K = 1\).
    \index{Single Point}
    
    \item \textbf{Mixed Coordinates}: Points with a mix of positive and negative coordinates.
    \index{Mixed Coordinates}
    
    \item \textbf{Points with Zero Distance}: Multiple points at the origin.
    \index{Zero Distance Points}
    
    \item \textbf{Sparse and Dense Points}: Densely packed points and sparsely distributed points.
    \index{Sparse and Dense Points}
    
    \item \textbf{Duplicate Points}: Multiple identical points in the input list.
    \index{Duplicate Points}
    
    \item \textbf{K Greater Than Number of Unique Points}: Ensure that the algorithm handles cases where \(K\) exceeds the number of unique points if applicable.
    \index{K Greater Than Unique Points}
\end{itemize}

\section*{Implementation Considerations}

When implementing the \texttt{kClosest} function, keep in mind the following considerations to ensure robustness and efficiency:

\begin{itemize}
    \item \textbf{Data Type Selection}: Use appropriate data types that can handle large input values without overflow or precision loss.
    \index{Data Type Selection}
    
    \item \textbf{Optimizing Distance Calculations}: Avoid calculating the square root since it is unnecessary for comparison purposes.
    \index{Optimizing Distance Calculations}
    
    \item \textbf{Choosing the Right Algorithm}: Select an algorithm based on the size of the input and the value of \(K\) to optimize time and space complexities.
    \index{Choosing the Right Algorithm}
    
    \item \textbf{Language-Specific Libraries}: Utilize language-specific libraries and functions (e.g., \texttt{heapq} in Python) to simplify implementation and enhance performance.
    \index{Language-Specific Libraries}
    
    \item \textbf{Avoiding Redundant Calculations}: Ensure that each point's distance is calculated only once to optimize performance.
    \index{Avoiding Redundant Calculations}
    
    \item \textbf{Implementing Helper Functions}: Create helper functions for tasks like distance calculation and partitioning to enhance modularity and readability.
    \index{Helper Functions}
    
    \item \textbf{Edge Case Handling}: Implement checks for edge cases to prevent incorrect results or runtime errors.
    \index{Edge Case Handling}
    
    \item \textbf{Testing and Validation}: Develop a comprehensive suite of test cases that cover all possible scenarios, including edge cases, to validate the correctness and efficiency of the implementation.
    \index{Testing and Validation}
    
    \item \textbf{Scalability}: Design the algorithm to scale efficiently with increasing input sizes, maintaining performance and resource utilization.
    \index{Scalability}
    
    \item \textbf{Consistent Naming Conventions}: Use consistent and descriptive naming conventions for variables and functions to improve code clarity.
    \index{Naming Conventions}
    
    \item \textbf{Memory Management}: Ensure that the algorithm manages memory efficiently, especially when dealing with large datasets.
    \index{Memory Management}
    
    \item \textbf{Avoiding Stack Overflow}: If implementing recursive approaches, be mindful of recursion limits and potential stack overflow issues.
    \index{Avoiding Stack Overflow}
    
    \item \textbf{Implementing Iterative Solutions}: Prefer iterative solutions when recursion may lead to increased space complexity or stack overflow.
    \index{Implementing Iterative Solutions}
\end{itemize}

\section*{Conclusion}

The \textbf{K Closest Points to Origin} problem exemplifies the application of efficient selection algorithms and geometric computations to solve spatial challenges effectively. By leveraging QuickSelect or heap-based methods, the algorithm achieves optimal time and space complexities, making it highly suitable for large datasets. Understanding and implementing such techniques not only enhances problem-solving skills but also provides a foundation for tackling more advanced Computational Geometry problems involving nearest neighbor searches, clustering, and spatial data analysis.

\printindex

% \input{sections/rectangle_overlap}
% \input{sections/rectangle_area}
% \input{sections/k_closest_points_to_origin}
% \input{sections/the_skyline_problem}
% % filename: the_skyline_problem.tex

\problemsection{The Skyline Problem}
\label{chap:The_Skyline_Problem}
\marginnote{\href{https://leetcode.com/problems/the-skyline-problem/}{[LeetCode Link]}\index{LeetCode}}
\marginnote{\href{https://www.geeksforgeeks.org/the-skyline-problem/}{[GeeksForGeeks Link]}\index{GeeksForGeeks}}
\marginnote{\href{https://www.interviewbit.com/problems/the-skyline-problem/}{[InterviewBit Link]}\index{InterviewBit}}
\marginnote{\href{https://app.codesignal.com/challenges/the-skyline-problem}{[CodeSignal Link]}\index{CodeSignal}}
\marginnote{\href{https://www.codewars.com/kata/the-skyline-problem/train/python}{[Codewars Link]}\index{Codewars}}

The \textbf{Skyline Problem} is a complex Computational Geometry challenge that involves computing the skyline formed by a collection of buildings in a 2D cityscape. Each building is represented by its left and right x-coordinates and its height. The skyline is defined by a list of "key points" where the height changes. This problem tests one's ability to handle large datasets, implement efficient sweep line algorithms, and manage event-driven processing. Mastery of this problem is essential for applications in computer graphics, urban planning simulations, and geographic information systems (GIS).

\section*{Problem Statement}

You are given a list of buildings in a cityscape. Each building is represented as a triplet \([Li, Ri, Hi]\), where \(Li\) and \(Ri\) are the x-coordinates of the left and right edges of the building, respectively, and \(Hi\) is the height of the building.

The skyline should be represented as a list of key points \([x, y]\) in sorted order by \(x\)-coordinate, where \(y\) is the height of the skyline at that point. The skyline should only include critical points where the height changes.

\textbf{Function signature in Python:}
\begin{lstlisting}[language=Python]
def getSkyline(buildings: List[List[int]]) -> List[List[int]]:
\end{lstlisting}

\section*{Examples}

\textbf{Example 1:}

\begin{verbatim}
Input: buildings = [[2,9,10], [3,7,15], [5,12,12], [15,20,10], [19,24,8]]
Output: [[2,10], [3,15], [7,12], [12,0], [15,10], [20,8], [24,0]]
Explanation:
- At x=2, the first building starts, height=10.
- At x=3, the second building starts, height=15.
- At x=7, the second building ends, the third building is still ongoing, height=12.
- At x=12, the third building ends, height drops to 0.
- At x=15, the fourth building starts, height=10.
- At x=20, the fourth building ends, the fifth building is still ongoing, height=8.
- At x=24, the fifth building ends, height drops to 0.
\end{verbatim}

\textbf{Example 2:}

\begin{verbatim}
Input: buildings = [[0,2,3], [2,5,3]]
Output: [[0,3], [5,0]]
Explanation:
- The two buildings are contiguous and have the same height, so the skyline drops to 0 at x=5.
\end{verbatim}

\textbf{Example 3:}

\begin{verbatim}
Input: buildings = [[1,3,3], [2,4,4], [5,6,1]]
Output: [[1,3], [2,4], [4,0], [5,1], [6,0]]
Explanation:
- At x=1, first building starts, height=3.
- At x=2, second building starts, height=4.
- At x=4, second building ends, height drops to 0.
- At x=5, third building starts, height=1.
- At x=6, third building ends, height drops to 0.
\end{verbatim}

\textbf{Example 4:}

\begin{verbatim}
Input: buildings = [[0,5,0]]
Output: []
Explanation:
- A building with height 0 does not contribute to the skyline.
\end{verbatim}

\textbf{Constraints:}

\begin{itemize}
    \item \(1 \leq \text{buildings.length} \leq 10^4\)
    \item \(0 \leq Li < Ri \leq 10^9\)
    \item \(0 \leq Hi \leq 10^4\)
\end{itemize}

\section*{Algorithmic Approach}

The \textbf{Sweep Line Algorithm} is an efficient method for solving the Skyline Problem. It involves processing events (building start and end points) in sorted order while maintaining a data structure (typically a max heap) to keep track of active buildings. Here's a step-by-step approach:

\subsection*{1. Event Representation}

Transform each building into two events:
\begin{itemize}
    \item **Start Event:** \((Li, -Hi)\) – Negative height indicates a building starts.
    \item **End Event:** \((Ri, Hi)\) – Positive height indicates a building ends.
\end{itemize}

Sorting the events ensures that start events are processed before end events at the same x-coordinate, and taller buildings are processed before shorter ones.

\subsection*{2. Sorting the Events}

Sort all events based on:
\begin{enumerate}
    \item **x-coordinate:** Ascending order.
    \item **Height:**
    \begin{itemize}
        \item For start events, taller buildings come first.
        \item For end events, shorter buildings come first.
    \end{itemize}
\end{enumerate}

\subsection*{3. Processing the Events}

Use a max heap to keep track of active building heights. Iterate through the sorted events:
\begin{enumerate}
    \item **Start Event:**
    \begin{itemize}
        \item Add the building's height to the heap.
    \end{itemize}
    
    \item **End Event:**
    \begin{itemize}
        \item Remove the building's height from the heap.
    \end{itemize}
    
    \item **Determine Current Max Height:**
    \begin{itemize}
        \item The current max height is the top of the heap.
    \end{itemize}
    
    \item **Update Skyline:**
    \begin{itemize}
        \item If the current max height differs from the previous max height, add a new key point \([x, current\_max\_height]\).
    \end{itemize}
\end{enumerate}

\subsection*{4. Finalizing the Skyline}

After processing all events, the accumulated key points represent the skyline.

\marginnote{The Sweep Line Algorithm efficiently handles dynamic changes in active buildings, ensuring accurate skyline construction.}

\section*{Complexities}

\begin{itemize}
    \item \textbf{Time Complexity:} \(O(n \log n)\), where \(n\) is the number of buildings. Sorting the events takes \(O(n \log n)\), and each heap operation takes \(O(\log n)\).
    
    \item \textbf{Space Complexity:} \(O(n)\), due to the storage of events and the heap.
\end{itemize}

\section*{Python Implementation}

\marginnote{Implementing the Sweep Line Algorithm with a max heap ensures an efficient and accurate solution.}

Below is the complete Python code implementing the \texttt{getSkyline} function:

\begin{fullwidth}
\begin{lstlisting}[language=Python]
from typing import List
import heapq

class Solution:
    def getSkyline(self, buildings: List[List[int]]) -> List[List[int]]:
        # Create a list of all events
        # For start events, use negative height to ensure they are processed before end events
        events = []
        for L, R, H in buildings:
            events.append((L, -H))
            events.append((R, H))
        
        # Sort the events
        # First by x-coordinate, then by height
        events.sort()
        
        # Max heap to keep track of active buildings
        heap = [0]  # Initialize with ground level
        heapq.heapify(heap)
        active_heights = {0: 1}  # Dictionary to count heights
        
        result = []
        prev_max = 0
        
        for x, h in events:
            if h < 0:
                # Start of a building, add height to heap and dictionary
                heapq.heappush(heap, h)
                active_heights[h] = active_heights.get(h, 0) + 1
            else:
                # End of a building, remove height from dictionary
                active_heights[h] -= 1
                if active_heights[h] == 0:
                    del active_heights[h]
            
            # Current max height
            while heap and active_heights.get(heap[0], 0) == 0:
                heapq.heappop(heap)
            current_max = -heap[0] if heap else 0
            
            # If the max height has changed, add to result
            if current_max != prev_max:
                result.append([x, current_max])
                prev_max = current_max
        
        return result

# Example usage:
solution = Solution()
print(solution.getSkyline([[2,9,10], [3,7,15], [5,12,12], [15,20,10], [19,24,8]]))
# Output: [[2,10], [3,15], [7,12], [12,0], [15,10], [20,8], [24,0]]

print(solution.getSkyline([[0,2,3], [2,5,3]]))
# Output: [[0,3], [5,0]]

print(solution.getSkyline([[1,3,3], [2,4,4], [5,6,1]]))
# Output: [[1,3], [2,4], [4,0], [5,1], [6,0]]

print(solution.getSkyline([[0,5,0]]))
# Output: []
\end{lstlisting}
\end{fullwidth}

This implementation efficiently constructs the skyline by processing all building events in sorted order and maintaining active building heights using a max heap. It ensures that only critical points where the skyline changes are recorded.

\section*{Explanation}

The \texttt{getSkyline} function constructs the skyline formed by a set of buildings by leveraging the Sweep Line Algorithm and a max heap to track active buildings. Here's a detailed breakdown of the implementation:

\subsection*{1. Event Representation}

\begin{itemize}
    \item Each building is transformed into two events:
    \begin{itemize}
        \item **Start Event:** \((Li, -Hi)\) – Negative height indicates the start of a building.
        \item **End Event:** \((Ri, Hi)\) – Positive height indicates the end of a building.
    \end{itemize}
\end{itemize}

\subsection*{2. Sorting the Events}

\begin{itemize}
    \item Events are sorted primarily by their x-coordinate in ascending order.
    \item For events with the same x-coordinate:
    \begin{itemize}
        \item Start events (with negative heights) are processed before end events.
        \item Taller buildings are processed before shorter ones.
    \end{itemize}
\end{itemize}

\subsection*{3. Processing the Events}

\begin{itemize}
    \item **Heap Initialization:**
    \begin{itemize}
        \item A max heap is initialized with a ground level height of 0.
        \item A dictionary \texttt{active\_heights} tracks the count of active building heights.
    \end{itemize}
    
    \item **Iterating Through Events:**
    \begin{enumerate}
        \item **Start Event:**
        \begin{itemize}
            \item Add the building's height to the heap.
            \item Increment the count of the height in \texttt{active\_heights}.
        \end{itemize}
        
        \item **End Event:**
        \begin{itemize}
            \item Decrement the count of the building's height in \texttt{active\_heights}.
            \item If the count reaches zero, remove the height from the dictionary.
        \end{itemize}
        
        \item **Determine Current Max Height:**
        \begin{itemize}
            \item Remove heights from the heap that are no longer active.
            \item The current max height is the top of the heap.
        \end{itemize}
        
        \item **Update Skyline:**
        \begin{itemize}
            \item If the current max height differs from the previous max height, add a new key point \([x, current\_max\_height]\).
        \end{itemize}
    \end{enumerate}
\end{itemize}

\subsection*{4. Finalizing the Skyline}

\begin{itemize}
    \item After processing all events, the \texttt{result} list contains the key points defining the skyline.
\end{itemize}

\subsection*{5. Example Walkthrough}

Consider the first example:
\begin{verbatim}
Input: buildings = [[2,9,10], [3,7,15], [5,12,12], [15,20,10], [19,24,8]]
Output: [[2,10], [3,15], [7,12], [12,0], [15,10], [20,8], [24,0]]
\end{verbatim}

\begin{enumerate}
    \item **Event Transformation:**
    \begin{itemize}
        \item \((2, -10)\), \((9, 10)\)
        \item \((3, -15)\), \((7, 15)\)
        \item \((5, -12)\), \((12, 12)\)
        \item \((15, -10)\), \((20, 10)\)
        \item \((19, -8)\), \((24, 8)\)
    \end{itemize}
    
    \item **Sorting Events:**
    \begin{itemize}
        \item Sorted order: \((2, -10)\), \((3, -15)\), \((5, -12)\), \((7, 15)\), \((9, 10)\), \((12, 12)\), \((15, -10)\), \((19, -8)\), \((20, 10)\), \((24, 8)\)
    \end{itemize}
    
    \item **Processing Events:**
    \begin{itemize}
        \item At each event, update the heap and determine if the skyline height changes.
    \end{itemize}
    
    \item **Result Construction:**
    \begin{itemize}
        \item The resulting skyline key points are accumulated as \([[2,10], [3,15], [7,12], [12,0], [15,10], [20,8], [24,0]]\).
    \end{itemize}
\end{enumerate}

Thus, the function correctly constructs the skyline based on the buildings' positions and heights.

\section*{Why This Approach}

The Sweep Line Algorithm combined with a max heap offers an optimal solution with \(O(n \log n)\) time complexity and efficient handling of overlapping buildings. By processing events in sorted order and maintaining active building heights, the algorithm ensures that all critical points in the skyline are accurately identified without redundant computations.

\section*{Alternative Approaches}

\subsection*{1. Divide and Conquer}

Divide the set of buildings into smaller subsets, compute the skyline for each subset, and then merge the skylines.

\begin{lstlisting}[language=Python]
class Solution:
    def getSkyline(self, buildings: List[List[int]]) -> List[List[int]]:
        def merge(left, right):
            h1, h2 = 0, 0
            i, j = 0, 0
            merged = []
            while i < len(left) and j < len(right):
                if left[i][0] < right[j][0]:
                    x, h1 = left[i]
                    i += 1
                elif left[i][0] > right[j][0]:
                    x, h2 = right[j]
                    j += 1
                else:
                    x, h1 = left[i]
                    _, h2 = right[j]
                    i += 1
                    j += 1
                max_h = max(h1, h2)
                if not merged or merged[-1][1] != max_h:
                    merged.append([x, max_h])
            merged.extend(left[i:])
            merged.extend(right[j:])
            return merged
        
        def divide(buildings):
            if not buildings:
                return []
            if len(buildings) == 1:
                L, R, H = buildings[0]
                return [[L, H], [R, 0]]
            mid = len(buildings) // 2
            left = divide(buildings[:mid])
            right = divide(buildings[mid:])
            return merge(left, right)
        
        return divide(buildings)
\end{lstlisting}

\textbf{Complexities:}
\begin{itemize}
    \item \textbf{Time Complexity:} \(O(n \log n)\)
    \item \textbf{Space Complexity:} \(O(n)\)
\end{itemize}

\subsection*{2. Using Segment Trees}

Implement a segment tree to manage and query overlapping building heights dynamically.

\textbf{Note}: This approach is more complex and is generally used for advanced scenarios with multiple dynamic queries.

\section*{Similar Problems to This One}

Several problems involve skyline-like constructions, spatial data analysis, and efficient event processing, utilizing similar algorithmic strategies:

\begin{itemize}
    \item \textbf{Merge Intervals}: Merge overlapping intervals in a list.
    \item \textbf{Largest Rectangle in Histogram}: Find the largest rectangular area in a histogram.
    \item \textbf{Interval Partitioning}: Assign intervals to resources without overlap.
    \item \textbf{Line Segment Intersection}: Detect intersections among line segments.
    \item \textbf{Closest Pair of Points}: Find the closest pair of points in a set.
    \item \textbf{Convex Hull}: Compute the convex hull of a set of points.
    \item \textbf{Point Inside Polygon}: Determine if a point lies inside a given polygon.
    \item \textbf{Range Searching}: Efficiently query geometric data within a specified range.
\end{itemize}

These problems reinforce concepts of event-driven processing, spatial reasoning, and efficient algorithm design in various contexts.

\section*{Things to Keep in Mind and Tricks}

When tackling the \textbf{Skyline Problem}, consider the following tips and best practices to enhance efficiency and correctness:

\begin{itemize}
    \item \textbf{Understand Sweep Line Technique}: Grasp how the sweep line algorithm processes events in sorted order to handle dynamic changes efficiently.
    \index{Sweep Line Technique}
    
    \item \textbf{Leverage Priority Queues (Heaps)}: Use max heaps to keep track of active buildings' heights, enabling quick access to the current maximum height.
    \index{Priority Queues}
    
    \item \textbf{Handle Start and End Events Differently}: Differentiate between building start and end events to accurately manage active heights.
    \index{Start and End Events}
    
    \item \textbf{Optimize Event Sorting}: Sort events primarily by x-coordinate and secondarily by height to ensure correct processing order.
    \index{Event Sorting}
    
    \item \textbf{Manage Active Heights Efficiently}: Use data structures that allow efficient insertion, deletion, and retrieval of maximum elements.
    \index{Active Heights Management}
    
    \item \textbf{Avoid Redundant Key Points}: Only record key points when the skyline height changes to minimize the output list.
    \index{Avoiding Redundant Key Points}
    
    \item \textbf{Implement Helper Functions}: Create helper functions for tasks like distance calculation, event handling, and heap management to enhance modularity.
    \index{Helper Functions}
    
    \item \textbf{Code Readability}: Maintain clear and readable code through meaningful variable names and structured logic.
    \index{Code Readability}
    
    \item \textbf{Test Extensively}: Implement a wide range of test cases, including overlapping, non-overlapping, and edge-touching buildings, to ensure robustness.
    \index{Extensive Testing}
    
    \item \textbf{Handle Degenerate Cases}: Manage cases where buildings have zero height or identical coordinates gracefully.
    \index{Degenerate Cases}
    
    \item \textbf{Understand Geometric Relationships}: Grasp how buildings overlap and influence the skyline to simplify the algorithm.
    \index{Geometric Relationships}
    
    \item \textbf{Use Appropriate Data Structures}: Utilize appropriate data structures like heaps, lists, and dictionaries to manage and process data efficiently.
    \index{Appropriate Data Structures}
    
    \item \textbf{Optimize for Large Inputs}: Design the algorithm to handle large numbers of buildings without significant performance degradation.
    \index{Optimizing for Large Inputs}
    
    \item \textbf{Implement Iterative Solutions Carefully}: Ensure that loop conditions are correctly defined to prevent infinite loops or incorrect terminations.
    \index{Iterative Solutions}
    
    \item \textbf{Consistent Naming Conventions}: Use consistent and descriptive naming conventions for variables and functions to improve code clarity.
    \index{Naming Conventions}
\end{itemize}

\section*{Corner and Special Cases to Test When Writing the Code}

When implementing the solution for the \textbf{Skyline Problem}, it is crucial to consider and rigorously test various edge cases to ensure robustness and correctness:

\begin{itemize}
    \item \textbf{No Overlapping Buildings}: All buildings are separate and do not overlap.
    \index{No Overlapping Buildings}
    
    \item \textbf{Fully Overlapping Buildings}: Multiple buildings completely overlap each other.
    \index{Fully Overlapping Buildings}
    
    \item \textbf{Buildings Touching at Edges}: Buildings share common edges without overlapping.
    \index{Buildings Touching at Edges}
    
    \item \textbf{Buildings Touching at Corners}: Buildings share common corners without overlapping.
    \index{Buildings Touching at Corners}
    
    \item \textbf{Single Building}: Only one building is present.
    \index{Single Building}
    
    \item \textbf{Multiple Buildings with Same Start or End}: Multiple buildings start or end at the same x-coordinate.
    \index{Same Start or End}
    
    \item \textbf{Buildings with Zero Height}: Buildings that have zero height should not affect the skyline.
    \index{Buildings with Zero Height}
    
    \item \textbf{Large Number of Buildings}: Test with a large number of buildings to ensure performance and scalability.
    \index{Large Number of Buildings}
    
    \item \textbf{Buildings with Negative Coordinates}: Buildings positioned in negative coordinate space.
    \index{Negative Coordinates}
    
    \item \textbf{Boundary Values}: Buildings at the minimum and maximum limits of the coordinate range.
    \index{Boundary Values}
    
    \item \textbf{Buildings with Identical Coordinates}: Multiple buildings with the same coordinates.
    \index{Identical Coordinates}
    
    \item \textbf{Sequential Buildings}: Buildings placed sequentially without gaps.
    \index{Sequential Buildings}
    
    \item \textbf{Overlapping and Non-Overlapping Mixed}: A mix of overlapping and non-overlapping buildings.
    \index{Overlapping and Non-Overlapping Mixed}
    
    \item \textbf{Buildings with Very Large Heights}: Buildings with heights at the upper limit of the constraints.
    \index{Very Large Heights}
    
    \item \textbf{Empty Input}: No buildings are provided.
    \index{Empty Input}
\end{itemize}

\section*{Implementation Considerations}

When implementing the \texttt{getSkyline} function, keep in mind the following considerations to ensure robustness and efficiency:

\begin{itemize}
    \item \textbf{Data Type Selection}: Use appropriate data types that can handle large input values and avoid overflow or precision issues.
    \index{Data Type Selection}
    
    \item \textbf{Optimizing Event Sorting}: Efficiently sort events based on x-coordinates and heights to ensure correct processing order.
    \index{Optimizing Event Sorting}
    
    \item \textbf{Handling Large Inputs}: Design the algorithm to handle up to \(10^4\) buildings efficiently without significant performance degradation.
    \index{Handling Large Inputs}
    
    \item \textbf{Using Efficient Data Structures}: Utilize heaps, lists, and dictionaries effectively to manage and process events and active heights.
    \index{Efficient Data Structures}
    
    \item \textbf{Avoiding Redundant Calculations}: Ensure that distance and overlap calculations are performed only when necessary to optimize performance.
    \index{Avoiding Redundant Calculations}
    
    \item \textbf{Code Readability and Documentation}: Maintain clear and readable code through meaningful variable names and comprehensive comments to facilitate understanding and maintenance.
    \index{Code Readability}
    
    \item \textbf{Edge Case Handling}: Implement checks for edge cases to prevent incorrect results or runtime errors.
    \index{Edge Case Handling}
    
    \item \textbf{Implementing Helper Functions}: Create helper functions for tasks like distance calculation, event handling, and heap management to enhance modularity.
    \index{Helper Functions}
    
    \item \textbf{Consistent Naming Conventions}: Use consistent and descriptive naming conventions for variables and functions to improve code clarity.
    \index{Naming Conventions}
    
    \item \textbf{Memory Management}: Ensure that the algorithm manages memory efficiently, especially when dealing with large datasets.
    \index{Memory Management}
    
    \item \textbf{Implementing Iterative Solutions Carefully}: Ensure that loop conditions are correctly defined to prevent infinite loops or incorrect terminations.
    \index{Iterative Solutions}
    
    \item \textbf{Avoiding Floating-Point Precision Issues}: Since the problem deals with integers, floating-point precision is not a concern, simplifying the implementation.
    \index{Floating-Point Precision}
    
    \item \textbf{Testing and Validation}: Develop a comprehensive suite of test cases that cover all possible scenarios, including edge cases, to validate the correctness and efficiency of the implementation.
    \index{Testing and Validation}
    
    \item \textbf{Performance Considerations}: Optimize the loop conditions and operations to ensure that the function runs efficiently, especially for large input numbers.
    \index{Performance Considerations}
\end{itemize}

\section*{Conclusion}

The \textbf{Skyline Problem} is a quintessential example of applying advanced algorithmic techniques and geometric reasoning to solve complex spatial challenges. By leveraging the Sweep Line Algorithm and maintaining active building heights using a max heap, the solution efficiently constructs the skyline with optimal time and space complexities. Understanding and implementing such sophisticated algorithms not only enhances problem-solving skills but also provides a foundation for tackling a wide array of Computational Geometry problems in various domains, including computer graphics, urban planning simulations, and geographic information systems.

\printindex

% \input{sections/rectangle_overlap}
% \input{sections/rectangle_area}
% \input{sections/k_closest_points_to_origin}
% \input{sections/the_skyline_problem}
% % filename: k_closest_points_to_origin.tex

\problemsection{K Closest Points to Origin}
\label{chap:K_Closest_Points_to_Origin}
\marginnote{\href{https://leetcode.com/problems/k-closest-points-to-origin/}{[LeetCode Link]}\index{LeetCode}}
\marginnote{\href{https://www.geeksforgeeks.org/find-k-closest-points-origin/}{[GeeksForGeeks Link]}\index{GeeksForGeeks}}
\marginnote{\href{https://www.interviewbit.com/problems/k-closest-points/}{[InterviewBit Link]}\index{InterviewBit}}
\marginnote{\href{https://app.codesignal.com/challenges/k-closest-points-to-origin}{[CodeSignal Link]}\index{CodeSignal}}
\marginnote{\href{https://www.codewars.com/kata/k-closest-points-to-origin/train/python}{[Codewars Link]}\index{Codewars}}

The \textbf{K Closest Points to Origin} problem is a popular algorithmic challenge in Computational Geometry that involves identifying the \(k\) points closest to the origin in a 2D plane. This problem tests one's ability to apply efficient sorting and selection algorithms, understand distance computations, and optimize for performance. Mastery of this problem is essential for applications in spatial data analysis, nearest neighbor searches, and clustering algorithms.

\section*{Problem Statement}

Given an array of points where each point is represented as \([x, y]\) in the 2D plane, and an integer \(k\), return the \(k\) closest points to the origin \((0, 0)\).

The distance between two points \((x_1, y_1)\) and \((x_2, y_2)\) is the Euclidean distance \(\sqrt{(x_1 - x_2)^2 + (y_1 - y_2)^2}\). The origin is \((0, 0)\).

\textbf{Function signature in Python:}
\begin{lstlisting}[language=Python]
def kClosest(points: List[List[int]], K: int) -> List[List[int]]:
\end{lstlisting}

\section*{Examples}

\textbf{Example 1:}

\begin{verbatim}
Input: points = [[1,3],[-2,2]], K = 1
Output: [[-2,2]]
Explanation: 
The distance between (1, 3) and the origin is sqrt(10).
The distance between (-2, 2) and the origin is sqrt(8).
Since sqrt(8) < sqrt(10), (-2, 2) is closer to the origin.
\end{verbatim}

\textbf{Example 2:}

\begin{verbatim}
Input: points = [[3,3],[5,-1],[-2,4]], K = 2
Output: [[3,3],[-2,4]]
Explanation: 
The distances are sqrt(18), sqrt(26), and sqrt(20) respectively.
The two closest points are [3,3] and [-2,4].
\end{verbatim}

\textbf{Example 3:}

\begin{verbatim}
Input: points = [[0,1],[1,0]], K = 2
Output: [[0,1],[1,0]]
Explanation: 
Both points are equally close to the origin.
\end{verbatim}

\textbf{Example 4:}

\begin{verbatim}
Input: points = [[1,0],[0,1]], K = 1
Output: [[1,0]]
Explanation: 
Both points are equally close; returning any one is acceptable.
\end{verbatim}

\textbf{Constraints:}

\begin{itemize}
    \item \(1 \leq K \leq \text{points.length} \leq 10^4\)
    \item \(-10^4 < x_i, y_i < 10^4\)
\end{itemize}

LeetCode link: \href{https://leetcode.com/problems/k-closest-points-to-origin/}{K Closest Points to Origin}\index{LeetCode}

\section*{Algorithmic Approach}

To identify the \(k\) closest points to the origin, several algorithmic strategies can be employed. The most efficient methods aim to reduce the time complexity by avoiding the need to sort the entire list of points.

\subsection*{1. Sorting Based on Distance}

Calculate the Euclidean distance of each point from the origin and sort the points based on these distances. Select the first \(k\) points from the sorted list.

\begin{enumerate}
    \item Compute the distance for each point using the formula \(distance = x^2 + y^2\).
    \item Sort the points based on the computed distances.
    \item Return the first \(k\) points from the sorted list.
\end{enumerate}

\subsection*{2. Max Heap (Priority Queue)}

Use a max heap to maintain the \(k\) closest points. Iterate through each point, add it to the heap, and if the heap size exceeds \(k\), remove the farthest point.

\begin{enumerate}
    \item Initialize a max heap.
    \item For each point, compute its distance and add it to the heap.
    \item If the heap size exceeds \(k\), remove the point with the largest distance.
    \item After processing all points, the heap contains the \(k\) closest points.
\end{enumerate}

\subsection*{3. QuickSelect (Quick Sort Partitioning)}

Utilize the QuickSelect algorithm to find the \(k\) closest points without fully sorting the list.

\begin{enumerate}
    \item Choose a pivot point and partition the list based on distances relative to the pivot.
    \item Recursively apply QuickSelect to the partition containing the \(k\) closest points.
    \item Once the \(k\) closest points are identified, return them.
\end{enumerate}

\marginnote{QuickSelect offers an average time complexity of \(O(n)\), making it highly efficient for large datasets.}

\section*{Complexities}

\begin{itemize}
    \item \textbf{Sorting Based on Distance:}
    \begin{itemize}
        \item \textbf{Time Complexity:} \(O(n \log n)\)
        \item \textbf{Space Complexity:} \(O(n)\)
    \end{itemize}
    
    \item \textbf{Max Heap (Priority Queue):}
    \begin{itemize}
        \item \textbf{Time Complexity:} \(O(n \log k)\)
        \item \textbf{Space Complexity:} \(O(k)\)
    \end{itemize}
    
    \item \textbf{QuickSelect (Quick Sort Partitioning):}
    \begin{itemize}
        \item \textbf{Time Complexity:} Average case \(O(n)\), worst case \(O(n^2)\)
        \item \textbf{Space Complexity:} \(O(1)\) (in-place)
    \end{itemize}
\end{itemize}

\section*{Python Implementation}

\marginnote{Implementing QuickSelect provides an optimal average-case solution with linear time complexity.}

Below is the complete Python code implementing the \texttt{kClosest} function using the QuickSelect approach:

\begin{fullwidth}
\begin{lstlisting}[language=Python]
from typing import List
import random

class Solution:
    def kClosest(self, points: List[List[int]], K: int) -> List[List[int]]:
        def quickselect(left, right, K_smallest):
            if left == right:
                return
            
            # Select a random pivot_index
            pivot_index = random.randint(left, right)
            
            # Partition the array
            pivot_index = partition(left, right, pivot_index)
            
            # The pivot is in its final sorted position
            if K_smallest == pivot_index:
                return
            elif K_smallest < pivot_index:
                quickselect(left, pivot_index - 1, K_smallest)
            else:
                quickselect(pivot_index + 1, right, K_smallest)
        
        def partition(left, right, pivot_index):
            pivot_distance = distance(points[pivot_index])
            # Move pivot to end
            points[pivot_index], points[right] = points[right], points[pivot_index]
            store_index = left
            for i in range(left, right):
                if distance(points[i]) < pivot_distance:
                    points[store_index], points[i] = points[i], points[store_index]
                    store_index += 1
            # Move pivot to its final place
            points[right], points[store_index] = points[store_index], points[right]
            return store_index
        
        def distance(point):
            return point[0] ** 2 + point[1] ** 2
        
        n = len(points)
        quickselect(0, n - 1, K)
        return points[:K]

# Example usage:
solution = Solution()
print(solution.kClosest([[1,3],[-2,2]], 1))            # Output: [[-2,2]]
print(solution.kClosest([[3,3],[5,-1],[-2,4]], 2))     # Output: [[3,3],[-2,4]]
print(solution.kClosest([[0,1],[1,0]], 2))             # Output: [[0,1],[1,0]]
print(solution.kClosest([[1,0],[0,1]], 1))             # Output: [[1,0]] or [[0,1]]
\end{lstlisting}
\end{fullwidth}

This implementation uses the QuickSelect algorithm to efficiently find the \(k\) closest points to the origin without fully sorting the entire list. It ensures optimal performance even with large datasets.

\section*{Explanation}

The \texttt{kClosest} function identifies the \(k\) closest points to the origin using the QuickSelect algorithm. Here's a detailed breakdown of the implementation:

\subsection*{1. Distance Calculation}

\begin{itemize}
    \item The Euclidean distance is calculated as \(distance = x^2 + y^2\). Since we only need relative distances for comparison, the square root is omitted for efficiency.
\end{itemize}

\subsection*{2. QuickSelect Algorithm}

\begin{itemize}
    \item **Pivot Selection:**
    \begin{itemize}
        \item A random pivot is chosen to enhance the average-case performance.
    \end{itemize}
    
    \item **Partitioning:**
    \begin{itemize}
        \item The array is partitioned such that points with distances less than the pivot are moved to the left, and others to the right.
        \item The pivot is placed in its correct sorted position.
    \end{itemize}
    
    \item **Recursive Selection:**
    \begin{itemize}
        \item If the pivot's position matches \(K\), the selection is complete.
        \item Otherwise, recursively apply QuickSelect to the relevant partition.
    \end{itemize}
\end{itemize}

\subsection*{3. Final Selection}

\begin{itemize}
    \item After partitioning, the first \(K\) points in the list are the \(k\) closest points to the origin.
\end{itemize}

\subsection*{4. Example Walkthrough}

Consider the first example:
\begin{verbatim}
Input: points = [[1,3],[-2,2]], K = 1
Output: [[-2,2]]
\end{verbatim}

\begin{enumerate}
    \item **Calculate Distances:**
    \begin{itemize}
        \item [1,3] : \(1^2 + 3^2 = 10\)
        \item [-2,2] : \((-2)^2 + 2^2 = 8\)
    \end{itemize}
    
    \item **QuickSelect Process:**
    \begin{itemize}
        \item Choose a pivot, say [1,3] with distance 10.
        \item Compare and rearrange:
        \begin{itemize}
            \item [-2,2] has a smaller distance (8) and is moved to the left.
        \end{itemize}
        \item After partitioning, the list becomes [[-2,2], [1,3]].
        \item Since \(K = 1\), return the first point: [[-2,2]].
    \end{itemize}
\end{enumerate}

Thus, the function correctly identifies \([-2,2]\) as the closest point to the origin.

\section*{Why This Approach}

The QuickSelect algorithm is chosen for its average-case linear time complexity \(O(n)\), making it highly efficient for large datasets compared to sorting-based methods with \(O(n \log n)\) time complexity. By avoiding the need to sort the entire list, QuickSelect provides an optimal solution for finding the \(k\) closest points.

\section*{Alternative Approaches}

\subsection*{1. Sorting Based on Distance}

Sort all points based on their distances from the origin and select the first \(k\) points.

\begin{lstlisting}[language=Python]
class Solution:
    def kClosest(self, points: List[List[int]], K: int) -> List[List[int]]:
        points.sort(key=lambda P: P[0]**2 + P[1]**2)
        return points[:K]
\end{lstlisting}

\textbf{Complexities:}
\begin{itemize}
    \item \textbf{Time Complexity:} \(O(n \log n)\)
    \item \textbf{Space Complexity:} \(O(1)\)
\end{itemize}

\subsection*{2. Max Heap (Priority Queue)}

Use a max heap to maintain the \(k\) closest points.

\begin{lstlisting}[language=Python]
import heapq

class Solution:
    def kClosest(self, points: List[List[int]], K: int) -> List[List[int]]:
        heap = []
        for (x, y) in points:
            dist = -(x**2 + y**2)  # Max heap using negative distances
            heapq.heappush(heap, (dist, [x, y]))
            if len(heap) > K:
                heapq.heappop(heap)
        return [item[1] for item in heap]
\end{lstlisting}

\textbf{Complexities:}
\begin{itemize}
    \item \textbf{Time Complexity:} \(O(n \log k)\)
    \item \textbf{Space Complexity:} \(O(k)\)
\end{itemize}

\subsection*{3. Using Built-In Functions}

Leverage built-in functions for distance calculation and selection.

\begin{lstlisting}[language=Python]
import math

class Solution:
    def kClosest(self, points: List[List[int]], K: int) -> List[List[int]]:
        points.sort(key=lambda P: math.sqrt(P[0]**2 + P[1]**2))
        return points[:K]
\end{lstlisting}

\textbf{Note}: This method is similar to the sorting approach but uses the actual Euclidean distance.

\section*{Similar Problems to This One}

Several problems involve nearest neighbor searches, spatial data analysis, and efficient selection algorithms, utilizing similar algorithmic strategies:

\begin{itemize}
    \item \textbf{Closest Pair of Points}: Find the closest pair of points in a set.
    \item \textbf{Top K Frequent Elements}: Identify the most frequent elements in a dataset.
    \item \textbf{Kth Largest Element in an Array}: Find the \(k\)-th largest element in an unsorted array.
    \item \textbf{Sliding Window Maximum}: Find the maximum in each sliding window of size \(k\) over an array.
    \item \textbf{Merge K Sorted Lists}: Merge multiple sorted lists into a single sorted list.
    \item \textbf{Find Median from Data Stream}: Continuously find the median of a stream of numbers.
    \item \textbf{Top K Closest Stars}: Find the \(k\) closest stars to Earth based on their distances.
\end{itemize}

These problems reinforce concepts of efficient selection, heap usage, and distance computations in various contexts.

\section*{Things to Keep in Mind and Tricks}

When solving the \textbf{K Closest Points to Origin} problem, consider the following tips and best practices to enhance efficiency and correctness:

\begin{itemize}
    \item \textbf{Understand Distance Calculations}: Grasp the Euclidean distance formula and recognize that the square root can be omitted for comparison purposes.
    \index{Distance Calculations}
    
    \item \textbf{Leverage Efficient Algorithms}: Use QuickSelect or heap-based methods to optimize time complexity, especially for large datasets.
    \index{Efficient Algorithms}
    
    \item \textbf{Handle Ties Appropriately}: Decide how to handle points with identical distances when \(k\) is less than the number of such points.
    \index{Handling Ties}
    
    \item \textbf{Optimize Space Usage}: Choose algorithms that minimize additional space, such as in-place QuickSelect.
    \index{Space Optimization}
    
    \item \textbf{Use Appropriate Data Structures}: Utilize heaps, lists, and helper functions effectively to manage and process data.
    \index{Data Structures}
    
    \item \textbf{Implement Helper Functions}: Create helper functions for distance calculation and partitioning to enhance code modularity.
    \index{Helper Functions}
    
    \item \textbf{Code Readability}: Maintain clear and readable code through meaningful variable names and structured logic.
    \index{Code Readability}
    
    \item \textbf{Test Extensively}: Implement a wide range of test cases, including edge cases like multiple points with the same distance, to ensure robustness.
    \index{Extensive Testing}
    
    \item \textbf{Understand Algorithm Trade-offs}: Recognize the trade-offs between different approaches in terms of time and space complexities.
    \index{Algorithm Trade-offs}
    
    \item \textbf{Use Built-In Sorting Functions}: When using sorting-based approaches, leverage built-in functions for efficiency and simplicity.
    \index{Built-In Sorting}
    
    \item \textbf{Avoid Redundant Calculations}: Ensure that distance calculations are performed only when necessary to optimize performance.
    \index{Avoiding Redundant Calculations}
    
    \item \textbf{Language-Specific Features}: Utilize language-specific features or libraries that can simplify implementation, such as heapq in Python.
    \index{Language-Specific Features}
\end{itemize}

\section*{Corner and Special Cases to Test When Writing the Code}

When implementing the solution for the \textbf{K Closest Points to Origin} problem, it is crucial to consider and rigorously test various edge cases to ensure robustness and correctness:

\begin{itemize}
    \item \textbf{Multiple Points with Same Distance}: Ensure that the algorithm handles multiple points having the same distance from the origin.
    \index{Same Distance Points}
    
    \item \textbf{Points at Origin}: Include points that are exactly at the origin \((0,0)\).
    \index{Points at Origin}
    
    \item \textbf{Negative Coordinates}: Ensure that the algorithm correctly computes distances for points with negative \(x\) or \(y\) coordinates.
    \index{Negative Coordinates}
    
    \item \textbf{Large Coordinates}: Test with points having very large or very small coordinate values to verify integer handling.
    \index{Large Coordinates}
    
    \item \textbf{K Equals Number of Points}: When \(K\) is equal to the number of points, the algorithm should return all points.
    \index{K Equals Number of Points}
    
    \item \textbf{K is One}: Test with \(K = 1\) to ensure the closest point is correctly identified.
    \index{K is One}
    
    \item \textbf{All Points Same}: All points have the same coordinates.
    \index{All Points Same}
    
    \item \textbf{K is Zero}: Although \(K\) is defined to be at least 1, ensure that the algorithm gracefully handles \(K = 0\) if allowed.
    \index{K is Zero}
    
    \item \textbf{Single Point}: Only one point is provided, and \(K = 1\).
    \index{Single Point}
    
    \item \textbf{Mixed Coordinates}: Points with a mix of positive and negative coordinates.
    \index{Mixed Coordinates}
    
    \item \textbf{Points with Zero Distance}: Multiple points at the origin.
    \index{Zero Distance Points}
    
    \item \textbf{Sparse and Dense Points}: Densely packed points and sparsely distributed points.
    \index{Sparse and Dense Points}
    
    \item \textbf{Duplicate Points}: Multiple identical points in the input list.
    \index{Duplicate Points}
    
    \item \textbf{K Greater Than Number of Unique Points}: Ensure that the algorithm handles cases where \(K\) exceeds the number of unique points if applicable.
    \index{K Greater Than Unique Points}
\end{itemize}

\section*{Implementation Considerations}

When implementing the \texttt{kClosest} function, keep in mind the following considerations to ensure robustness and efficiency:

\begin{itemize}
    \item \textbf{Data Type Selection}: Use appropriate data types that can handle large input values without overflow or precision loss.
    \index{Data Type Selection}
    
    \item \textbf{Optimizing Distance Calculations}: Avoid calculating the square root since it is unnecessary for comparison purposes.
    \index{Optimizing Distance Calculations}
    
    \item \textbf{Choosing the Right Algorithm}: Select an algorithm based on the size of the input and the value of \(K\) to optimize time and space complexities.
    \index{Choosing the Right Algorithm}
    
    \item \textbf{Language-Specific Libraries}: Utilize language-specific libraries and functions (e.g., \texttt{heapq} in Python) to simplify implementation and enhance performance.
    \index{Language-Specific Libraries}
    
    \item \textbf{Avoiding Redundant Calculations}: Ensure that each point's distance is calculated only once to optimize performance.
    \index{Avoiding Redundant Calculations}
    
    \item \textbf{Implementing Helper Functions}: Create helper functions for tasks like distance calculation and partitioning to enhance modularity and readability.
    \index{Helper Functions}
    
    \item \textbf{Edge Case Handling}: Implement checks for edge cases to prevent incorrect results or runtime errors.
    \index{Edge Case Handling}
    
    \item \textbf{Testing and Validation}: Develop a comprehensive suite of test cases that cover all possible scenarios, including edge cases, to validate the correctness and efficiency of the implementation.
    \index{Testing and Validation}
    
    \item \textbf{Scalability}: Design the algorithm to scale efficiently with increasing input sizes, maintaining performance and resource utilization.
    \index{Scalability}
    
    \item \textbf{Consistent Naming Conventions}: Use consistent and descriptive naming conventions for variables and functions to improve code clarity.
    \index{Naming Conventions}
    
    \item \textbf{Memory Management}: Ensure that the algorithm manages memory efficiently, especially when dealing with large datasets.
    \index{Memory Management}
    
    \item \textbf{Avoiding Stack Overflow}: If implementing recursive approaches, be mindful of recursion limits and potential stack overflow issues.
    \index{Avoiding Stack Overflow}
    
    \item \textbf{Implementing Iterative Solutions}: Prefer iterative solutions when recursion may lead to increased space complexity or stack overflow.
    \index{Implementing Iterative Solutions}
\end{itemize}

\section*{Conclusion}

The \textbf{K Closest Points to Origin} problem exemplifies the application of efficient selection algorithms and geometric computations to solve spatial challenges effectively. By leveraging QuickSelect or heap-based methods, the algorithm achieves optimal time and space complexities, making it highly suitable for large datasets. Understanding and implementing such techniques not only enhances problem-solving skills but also provides a foundation for tackling more advanced Computational Geometry problems involving nearest neighbor searches, clustering, and spatial data analysis.

\printindex

% % filename: rectangle_overlap.tex

\problemsection{Rectangle Overlap}
\label{chap:Rectangle_Overlap}
\marginnote{\href{https://leetcode.com/problems/rectangle-overlap/}{[LeetCode Link]}\index{LeetCode}}
\marginnote{\href{https://www.geeksforgeeks.org/check-if-two-rectangles-overlap/}{[GeeksForGeeks Link]}\index{GeeksForGeeks}}
\marginnote{\href{https://www.interviewbit.com/problems/rectangle-overlap/}{[InterviewBit Link]}\index{InterviewBit}}
\marginnote{\href{https://app.codesignal.com/challenges/rectangle-overlap}{[CodeSignal Link]}\index{CodeSignal}}
\marginnote{\href{https://www.codewars.com/kata/rectangle-overlap/train/python}{[Codewars Link]}\index{Codewars}}

The \textbf{Rectangle Overlap} problem is a fundamental challenge in Computational Geometry that involves determining whether two axis-aligned rectangles overlap. This problem tests one's ability to understand geometric properties, implement conditional logic, and optimize for efficient computation. Mastery of this problem is essential for applications in computer graphics, collision detection, and spatial data analysis.

\section*{Problem Statement}

Given two axis-aligned rectangles in a 2D plane, determine if they overlap. Each rectangle is defined by its bottom-left and top-right coordinates.

A rectangle is represented as a list of four integers \([x1, y1, x2, y2]\), where \((x1, y1)\) are the coordinates of the bottom-left corner, and \((x2, y2)\) are the coordinates of the top-right corner.

\textbf{Function signature in Python:}
\begin{lstlisting}[language=Python]
def isRectangleOverlap(rec1: List[int], rec2: List[int]) -> bool:
\end{lstlisting}

\section*{Examples}

\textbf{Example 1:}

\begin{verbatim}
Input: rec1 = [0,0,2,2], rec2 = [1,1,3,3]
Output: True
Explanation: The rectangles overlap in the area defined by [1,1,2,2].
\end{verbatim}

\textbf{Example 2:}

\begin{verbatim}
Input: rec1 = [0,0,1,1], rec2 = [1,0,2,1]
Output: False
Explanation: The rectangles touch at the edge but do not overlap.
\end{verbatim}

\textbf{Example 3:}

\begin{verbatim}
Input: rec1 = [0,0,1,1], rec2 = [2,2,3,3]
Output: False
Explanation: The rectangles are completely separate.
\end{verbatim}

\textbf{Example 4:}

\begin{verbatim}
Input: rec1 = [0,0,5,5], rec2 = [3,3,7,7]
Output: True
Explanation: The rectangles overlap in the area defined by [3,3,5,5].
\end{verbatim}

\textbf{Example 5:}

\begin{verbatim}
Input: rec1 = [0,0,0,0], rec2 = [0,0,0,0]
Output: False
Explanation: Both rectangles are degenerate points.
\end{verbatim}

\textbf{Constraints:}

\begin{itemize}
    \item All coordinates are integers in the range \([-10^9, 10^9]\).
    \item For each rectangle, \(x1 < x2\) and \(y1 < y2\).
\end{itemize}

LeetCode link: \href{https://leetcode.com/problems/rectangle-overlap/}{Rectangle Overlap}\index{LeetCode}

\section*{Algorithmic Approach}

To determine whether two axis-aligned rectangles overlap, we can use the following logical conditions:

1. **Non-Overlap Conditions:**
   - One rectangle is to the left of the other.
   - One rectangle is above the other.

2. **Overlap Condition:**
   - If neither of the non-overlap conditions is true, the rectangles must overlap.

\subsection*{Steps:}

1. **Extract Coordinates:**
   - For both rectangles, extract the bottom-left and top-right coordinates.

2. **Check Non-Overlap Conditions:**
   - If the right side of the first rectangle is less than or equal to the left side of the second rectangle, they do not overlap.
   - If the left side of the first rectangle is greater than or equal to the right side of the second rectangle, they do not overlap.
   - If the top side of the first rectangle is less than or equal to the bottom side of the second rectangle, they do not overlap.
   - If the bottom side of the first rectangle is greater than or equal to the top side of the second rectangle, they do not overlap.

3. **Determine Overlap:**
   - If none of the non-overlap conditions are met, the rectangles overlap.

\marginnote{This approach provides an efficient \(O(1)\) time complexity solution by leveraging simple geometric comparisons.}

\section*{Complexities}

\begin{itemize}
    \item \textbf{Time Complexity:} \(O(1)\). The algorithm performs a constant number of comparisons regardless of input size.
    
    \item \textbf{Space Complexity:} \(O(1)\). Only a fixed amount of extra space is used for variables.
\end{itemize}

\section*{Python Implementation}

\marginnote{Implementing the overlap check using coordinate comparisons ensures an optimal and straightforward solution.}

Below is the complete Python code implementing the \texttt{isRectangleOverlap} function:

\begin{fullwidth}
\begin{lstlisting}[language=Python]
from typing import List

class Solution:
    def isRectangleOverlap(self, rec1: List[int], rec2: List[int]) -> bool:
        # Extract coordinates
        left1, bottom1, right1, top1 = rec1
        left2, bottom2, right2, top2 = rec2
        
        # Check non-overlapping conditions
        if right1 <= left2 or right2 <= left1:
            return False
        if top1 <= bottom2 or top2 <= bottom1:
            return False
        
        # If none of the above, rectangles overlap
        return True

# Example usage:
solution = Solution()
print(solution.isRectangleOverlap([0,0,2,2], [1,1,3,3]))  # Output: True
print(solution.isRectangleOverlap([0,0,1,1], [1,0,2,1]))  # Output: False
print(solution.isRectangleOverlap([0,0,1,1], [2,2,3,3]))  # Output: False
print(solution.isRectangleOverlap([0,0,5,5], [3,3,7,7]))  # Output: True
print(solution.isRectangleOverlap([0,0,0,0], [0,0,0,0]))  # Output: False
\end{lstlisting}
\end{fullwidth}

This implementation efficiently checks for overlap by comparing the coordinates of the two rectangles. If any of the non-overlapping conditions are met, it returns \texttt{False}; otherwise, it returns \texttt{True}.

\section*{Explanation}

The \texttt{isRectangleOverlap} function determines whether two axis-aligned rectangles overlap by comparing their respective coordinates. Here's a detailed breakdown of the implementation:

\subsection*{1. Extract Coordinates}

\begin{itemize}
    \item For each rectangle, extract the left (\(x1\)), bottom (\(y1\)), right (\(x2\)), and top (\(y2\)) coordinates.
    \item This simplifies the comparison process by providing clear variables representing each side of the rectangles.
\end{itemize}

\subsection*{2. Check Non-Overlap Conditions}

\begin{itemize}
    \item **Horizontal Separation:**
    \begin{itemize}
        \item If the right side of the first rectangle (\(right1\)) is less than or equal to the left side of the second rectangle (\(left2\)), there is no horizontal overlap.
        \item Similarly, if the right side of the second rectangle (\(right2\)) is less than or equal to the left side of the first rectangle (\(left1\)), there is no horizontal overlap.
    \end{itemize}
    
    \item **Vertical Separation:**
    \begin{itemize}
        \item If the top side of the first rectangle (\(top1\)) is less than or equal to the bottom side of the second rectangle (\(bottom2\)), there is no vertical overlap.
        \item Similarly, if the top side of the second rectangle (\(top2\)) is less than or equal to the bottom side of the first rectangle (\(bottom1\)), there is no vertical overlap.
    \end{itemize}
    
    \item If any of these non-overlapping conditions are true, the rectangles do not overlap, and the function returns \texttt{False}.
\end{itemize}

\subsection*{3. Determine Overlap}

\begin{itemize}
    \item If none of the non-overlapping conditions are met, it implies that the rectangles overlap both horizontally and vertically.
    \item The function returns \texttt{True} in this case.
\end{itemize}

\subsection*{4. Example Walkthrough}

Consider the first example:
\begin{verbatim}
Input: rec1 = [0,0,2,2], rec2 = [1,1,3,3]
Output: True
\end{verbatim}

\begin{enumerate}
    \item Extract coordinates:
    \begin{itemize}
        \item rec1: left1 = 0, bottom1 = 0, right1 = 2, top1 = 2
        \item rec2: left2 = 1, bottom2 = 1, right2 = 3, top2 = 3
    \end{itemize}
    
    \item Check non-overlap conditions:
    \begin{itemize}
        \item \(right1 = 2\) is not less than or equal to \(left2 = 1\)
        \item \(right2 = 3\) is not less than or equal to \(left1 = 0\)
        \item \(top1 = 2\) is not less than or equal to \(bottom2 = 1\)
        \item \(top2 = 3\) is not less than or equal to \(bottom1 = 0\)
    \end{itemize}
    
    \item Since none of the non-overlapping conditions are met, the rectangles overlap.
\end{enumerate}

Thus, the function correctly returns \texttt{True}.

\section*{Why This Approach}

This approach is chosen for its simplicity and efficiency. By leveraging direct coordinate comparisons, the algorithm achieves constant time complexity without the need for complex data structures or iterative processes. It effectively handles all possible scenarios of rectangle positioning, ensuring accurate detection of overlaps.

\section*{Alternative Approaches}

\subsection*{1. Separating Axis Theorem (SAT)}

The Separating Axis Theorem is a more generalized method for detecting overlaps between convex shapes. While it is not necessary for axis-aligned rectangles, understanding SAT can be beneficial for more complex geometric problems.

\begin{lstlisting}[language=Python]
def isRectangleOverlap(rec1: List[int], rec2: List[int]) -> bool:
    # Using SAT for axis-aligned rectangles
    return not (rec1[2] <= rec2[0] or rec1[0] >= rec2[2] or
                rec1[3] <= rec2[1] or rec1[1] >= rec2[3])
\end{lstlisting}

\textbf{Note}: This implementation is functionally identical to the primary approach but leverages a more generalized geometric theorem.

\subsection*{2. Area-Based Approach}

Calculate the overlapping area between the two rectangles. If the overlapping area is positive, the rectangles overlap.

\begin{lstlisting}[language=Python]
def isRectangleOverlap(rec1: List[int], rec2: List[int]) -> bool:
    # Calculate overlap in x and y dimensions
    x_overlap = min(rec1[2], rec2[2]) - max(rec1[0], rec2[0])
    y_overlap = min(rec1[3], rec2[3]) - max(rec1[1], rec2[1])
    
    # Overlap exists if both overlaps are positive
    return x_overlap > 0 and y_overlap > 0
\end{lstlisting}

\textbf{Complexities:}
\begin{itemize}
    \item \textbf{Time Complexity:} \(O(1)\)
    \item \textbf{Space Complexity:} \(O(1)\)
\end{itemize}

\subsection*{3. Using Rectangles Intersection Function}

Utilize built-in or library functions that handle geometric intersections.

\begin{lstlisting}[language=Python]
from shapely.geometry import box

def isRectangleOverlap(rec1: List[int], rec2: List[int]) -> bool:
    rectangle1 = box(rec1[0], rec1[1], rec1[2], rec1[3])
    rectangle2 = box(rec2[0], rec2[1], rec2[2], rec2[3])
    return rectangle1.intersects(rectangle2) and not rectangle1.touches(rectangle2)
\end{lstlisting}

\textbf{Note}: This approach requires the \texttt{shapely} library and is more suitable for complex geometric operations.

\section*{Similar Problems to This One}

Several problems revolve around geometric overlap, intersection detection, and spatial reasoning, utilizing similar algorithmic strategies:

\begin{itemize}
    \item \textbf{Interval Overlap}: Determine if two intervals on a line overlap.
    \item \textbf{Circle Overlap}: Determine if two circles overlap based on their radii and centers.
    \item \textbf{Polygon Overlap}: Determine if two polygons overlap using algorithms like SAT.
    \item \textbf{Closest Pair of Points}: Find the closest pair of points in a set.
    \item \textbf{Convex Hull}: Compute the convex hull of a set of points.
    \item \textbf{Intersection of Lines}: Find the intersection point of two lines.
    \item \textbf{Point Inside Polygon}: Determine if a point lies inside a given polygon.
\end{itemize}

These problems reinforce the concepts of spatial reasoning, geometric property analysis, and efficient algorithm design in various contexts.

\section*{Things to Keep in Mind and Tricks}

When working with the \textbf{Rectangle Overlap} problem, consider the following tips and best practices to enhance efficiency and correctness:

\begin{itemize}
    \item \textbf{Understand Geometric Relationships}: Grasp the positional relationships between rectangles to simplify overlap detection.
    \index{Geometric Relationships}
    
    \item \textbf{Leverage Coordinate Comparisons}: Use direct comparisons of rectangle coordinates to determine spatial relationships.
    \index{Coordinate Comparisons}
    
    \item \textbf{Handle Edge Cases}: Consider cases where rectangles touch at edges or corners without overlapping.
    \index{Edge Cases}
    
    \item \textbf{Optimize for Efficiency}: Aim for a constant time \(O(1)\) solution by avoiding unnecessary computations or iterations.
    \index{Efficiency Optimization}
    
    \item \textbf{Avoid Floating-Point Precision Issues}: Since all coordinates are integers, floating-point precision is not a concern, simplifying the implementation.
    \index{Floating-Point Precision}
    
    \item \textbf{Use Helper Functions}: Create helper functions to encapsulate repetitive tasks, such as extracting coordinates or checking specific conditions.
    \index{Helper Functions}
    
    \item \textbf{Code Readability}: Maintain clear and readable code through meaningful variable names and structured logic.
    \index{Code Readability}
    
    \item \textbf{Test Extensively}: Implement a wide range of test cases, including overlapping, non-overlapping, and edge-touching rectangles, to ensure robustness.
    \index{Extensive Testing}
    
    \item \textbf{Understand Axis-Aligned Constraints}: Recognize that axis-aligned rectangles simplify overlap detection compared to rotated rectangles.
    \index{Axis-Aligned Constraints}
    
    \item \textbf{Simplify Logical Conditions}: Combine multiple conditions logically to streamline the overlap detection process.
    \index{Logical Conditions}
\end{itemize}

\section*{Corner and Special Cases to Test When Writing the Code}

When implementing the solution for the \textbf{Rectangle Overlap} problem, it is crucial to consider and rigorously test various edge cases to ensure robustness and correctness:

\begin{itemize}
    \item \textbf{No Overlap}: Rectangles are completely separate.
    \index{No Overlap}
    
    \item \textbf{Partial Overlap}: Rectangles overlap in one or more regions.
    \index{Partial Overlap}
    
    \item \textbf{Edge Touching}: Rectangles touch exactly at one edge without overlapping.
    \index{Edge Touching}
    
    \item \textbf{Corner Touching}: Rectangles touch exactly at one corner without overlapping.
    \index{Corner Touching}
    
    \item \textbf{One Rectangle Inside Another}: One rectangle is entirely within the other.
    \index{Rectangle Inside}
    
    \item \textbf{Identical Rectangles}: Both rectangles have the same coordinates.
    \index{Identical Rectangles}
    
    \item \textbf{Degenerate Rectangles}: Rectangles with zero area (e.g., \(x1 = x2\) or \(y1 = y2\)).
    \index{Degenerate Rectangles}
    
    \item \textbf{Large Coordinates}: Rectangles with very large coordinate values to test performance and integer handling.
    \index{Large Coordinates}
    
    \item \textbf{Negative Coordinates}: Rectangles positioned in negative coordinate space.
    \index{Negative Coordinates}
    
    \item \textbf{Mixed Overlapping Scenarios}: Combinations of the above cases to ensure comprehensive coverage.
    \index{Mixed Overlapping Scenarios}
    
    \item \textbf{Minimum and Maximum Bounds}: Rectangles at the minimum and maximum limits of the coordinate range.
    \index{Minimum and Maximum Bounds}
\end{itemize}

\section*{Implementation Considerations}

When implementing the \texttt{isRectangleOverlap} function, keep in mind the following considerations to ensure robustness and efficiency:

\begin{itemize}
    \item \textbf{Data Type Selection}: Use appropriate data types that can handle the range of input values without overflow or underflow.
    \index{Data Type Selection}
    
    \item \textbf{Optimizing Comparisons}: Structure logical conditions to short-circuit evaluations as soon as a non-overlapping condition is met.
    \index{Optimizing Comparisons}
    
    \item \textbf{Language-Specific Constraints}: Be aware of how the programming language handles integer division and comparisons.
    \index{Language-Specific Constraints}
    
    \item \textbf{Avoiding Redundant Calculations}: Ensure that each comparison contributes towards determining overlap without unnecessary repetitions.
    \index{Avoiding Redundant Calculations}
    
    \item \textbf{Code Readability and Documentation}: Maintain clear and readable code through meaningful variable names and comprehensive comments to facilitate understanding and maintenance.
    \index{Code Readability}
    
    \item \textbf{Edge Case Handling}: Implement checks for edge cases to prevent incorrect results or runtime errors.
    \index{Edge Case Handling}
    
    \item \textbf{Testing and Validation}: Develop a comprehensive suite of test cases that cover all possible scenarios, including edge cases, to validate the correctness and efficiency of the implementation.
    \index{Testing and Validation}
    
    \item \textbf{Scalability}: Design the algorithm to scale efficiently with increasing input sizes, maintaining performance and resource utilization.
    \index{Scalability}
    
    \item \textbf{Using Helper Functions}: Consider creating helper functions for repetitive tasks, such as extracting and comparing coordinates, to enhance modularity and reusability.
    \index{Helper Functions}
    
    \item \textbf{Consistent Naming Conventions}: Use consistent and descriptive naming conventions for variables to improve code clarity.
    \index{Naming Conventions}
    
    \item \textbf{Handling Floating-Point Coordinates}: Although the problem specifies integer coordinates, ensure that the implementation can handle floating-point numbers if needed in extended scenarios.
    \index{Floating-Point Coordinates}
    
    \item \textbf{Avoiding Floating-Point Precision Issues}: Since all coordinates are integers, floating-point precision is not a concern, simplifying the implementation.
    \index{Floating-Point Precision}
    
    \item \textbf{Implementing Unit Tests}: Develop unit tests for each logical condition to ensure that all scenarios are correctly handled.
    \index{Unit Tests}
    
    \item \textbf{Error Handling}: Incorporate error handling to manage invalid inputs gracefully.
    \index{Error Handling}
\end{itemize}

\section*{Conclusion}

The \textbf{Rectangle Overlap} problem exemplifies the application of fundamental geometric principles and conditional logic to solve spatial challenges efficiently. By leveraging simple coordinate comparisons, the algorithm achieves optimal time and space complexities, making it highly suitable for real-time applications such as collision detection in gaming, layout planning in graphics, and spatial data analysis. Understanding and implementing such techniques not only enhances problem-solving skills but also provides a foundation for tackling more complex Computational Geometry problems involving varied geometric shapes and interactions.

\printindex

% \input{sections/rectangle_overlap}
% \input{sections/rectangle_area}
% \input{sections/k_closest_points_to_origin}
% \input{sections/the_skyline_problem}
% % filename: rectangle_area.tex

\problemsection{Rectangle Area}
\label{chap:Rectangle_Area}
\marginnote{\href{https://leetcode.com/problems/rectangle-area/}{[LeetCode Link]}\index{LeetCode}}
\marginnote{\href{https://www.geeksforgeeks.org/find-area-two-overlapping-rectangles/}{[GeeksForGeeks Link]}\index{GeeksForGeeks}}
\marginnote{\href{https://www.interviewbit.com/problems/rectangle-area/}{[InterviewBit Link]}\index{InterviewBit}}
\marginnote{\href{https://app.codesignal.com/challenges/rectangle-area}{[CodeSignal Link]}\index{CodeSignal}}
\marginnote{\href{https://www.codewars.com/kata/rectangle-area/train/python}{[Codewars Link]}\index{Codewars}}

The \textbf{Rectangle Area} problem is a classic Computational Geometry challenge that involves calculating the total area covered by two axis-aligned rectangles in a 2D plane. This problem tests one's ability to perform geometric calculations, handle overlapping scenarios, and implement efficient algorithms. Mastery of this problem is essential for applications in computer graphics, spatial analysis, and computational modeling.

\section*{Problem Statement}

Given two axis-aligned rectangles in a 2D plane, compute the total area covered by the two rectangles. The area covered by the overlapping region should be counted only once.

Each rectangle is represented as a list of four integers \([x1, y1, x2, y2]\), where \((x1, y1)\) are the coordinates of the bottom-left corner, and \((x2, y2)\) are the coordinates of the top-right corner.

\textbf{Function signature in Python:}
\begin{lstlisting}[language=Python]
def computeArea(A: List[int], B: List[int]) -> int:
\end{lstlisting}

\section*{Examples}

\textbf{Example 1:}

\begin{verbatim}
Input: A = [-3,0,3,4], B = [0,-1,9,2]
Output: 45
Explanation:
Area of A = (3 - (-3)) * (4 - 0) = 6 * 4 = 24
Area of B = (9 - 0) * (2 - (-1)) = 9 * 3 = 27
Overlapping Area = (3 - 0) * (2 - 0) = 3 * 2 = 6
Total Area = 24 + 27 - 6 = 45
\end{verbatim}

\textbf{Example 2:}

\begin{verbatim}
Input: A = [0,0,0,0], B = [0,0,0,0]
Output: 0
Explanation:
Both rectangles are degenerate points with zero area.
\end{verbatim}

\textbf{Example 3:}

\begin{verbatim}
Input: A = [0,0,2,2], B = [1,1,3,3]
Output: 7
Explanation:
Area of A = 4
Area of B = 4
Overlapping Area = 1
Total Area = 4 + 4 - 1 = 7
\end{verbatim}

\textbf{Example 4:}

\begin{verbatim}
Input: A = [0,0,1,1], B = [1,0,2,1]
Output: 2
Explanation:
Rectangles touch at the edge but do not overlap.
Area of A = 1
Area of B = 1
Overlapping Area = 0
Total Area = 1 + 1 = 2
\end{verbatim}

\textbf{Constraints:}

\begin{itemize}
    \item All coordinates are integers in the range \([-10^9, 10^9]\).
    \item For each rectangle, \(x1 < x2\) and \(y1 < y2\).
\end{itemize}

LeetCode link: \href{https://leetcode.com/problems/rectangle-area/}{Rectangle Area}\index{LeetCode}

\section*{Algorithmic Approach}

To compute the total area covered by two axis-aligned rectangles, we can follow these steps:

1. **Calculate Individual Areas:**
   - Compute the area of the first rectangle.
   - Compute the area of the second rectangle.

2. **Determine Overlapping Area:**
   - Calculate the coordinates of the overlapping rectangle, if any.
   - If the rectangles overlap, compute the area of the overlapping region.

3. **Compute Total Area:**
   - Sum the individual areas and subtract the overlapping area to avoid double-counting.

\marginnote{This approach ensures accurate area calculation by handling overlapping regions appropriately.}

\section*{Complexities}

\begin{itemize}
    \item \textbf{Time Complexity:} \(O(1)\). The algorithm performs a constant number of calculations.
    
    \item \textbf{Space Complexity:} \(O(1)\). Only a fixed amount of extra space is used for variables.
\end{itemize}

\section*{Python Implementation}

\marginnote{Implementing the area calculation with overlap consideration ensures an accurate and efficient solution.}

Below is the complete Python code implementing the \texttt{computeArea} function:

\begin{fullwidth}
\begin{lstlisting}[language=Python]
from typing import List

class Solution:
    def computeArea(self, A: List[int], B: List[int]) -> int:
        # Calculate area of rectangle A
        areaA = (A[2] - A[0]) * (A[3] - A[1])
        
        # Calculate area of rectangle B
        areaB = (B[2] - B[0]) * (B[3] - B[1])
        
        # Determine overlap coordinates
        overlap_x1 = max(A[0], B[0])
        overlap_y1 = max(A[1], B[1])
        overlap_x2 = min(A[2], B[2])
        overlap_y2 = min(A[3], B[3])
        
        # Calculate overlapping area
        overlap_width = overlap_x2 - overlap_x1
        overlap_height = overlap_y2 - overlap_y1
        overlap_area = 0
        if overlap_width > 0 and overlap_height > 0:
            overlap_area = overlap_width * overlap_height
        
        # Total area is sum of individual areas minus overlapping area
        total_area = areaA + areaB - overlap_area
        return total_area

# Example usage:
solution = Solution()
print(solution.computeArea([-3,0,3,4], [0,-1,9,2]))  # Output: 45
print(solution.computeArea([0,0,0,0], [0,0,0,0]))    # Output: 0
print(solution.computeArea([0,0,2,2], [1,1,3,3]))    # Output: 7
print(solution.computeArea([0,0,1,1], [1,0,2,1]))    # Output: 2
\end{lstlisting}
\end{fullwidth}

This implementation accurately computes the total area covered by two rectangles by accounting for any overlapping regions. It ensures that the overlapping area is not double-counted.

\section*{Explanation}

The \texttt{computeArea} function calculates the combined area of two axis-aligned rectangles by following these steps:

\subsection*{1. Calculate Individual Areas}

\begin{itemize}
    \item **Rectangle A:**
    \begin{itemize}
        \item Width: \(A[2] - A[0]\)
        \item Height: \(A[3] - A[1]\)
        \item Area: Width \(\times\) Height
    \end{itemize}
    
    \item **Rectangle B:**
    \begin{itemize}
        \item Width: \(B[2] - B[0]\)
        \item Height: \(B[3] - B[1]\)
        \item Area: Width \(\times\) Height
    \end{itemize}
\end{itemize}

\subsection*{2. Determine Overlapping Area}

\begin{itemize}
    \item **Overlap Coordinates:**
    \begin{itemize}
        \item Left (x-coordinate): \(\text{max}(A[0], B[0])\)
        \item Bottom (y-coordinate): \(\text{max}(A[1], B[1])\)
        \item Right (x-coordinate): \(\text{min}(A[2], B[2])\)
        \item Top (y-coordinate): \(\text{min}(A[3], B[3])\)
    \end{itemize}
    
    \item **Overlap Dimensions:**
    \begin{itemize}
        \item Width: \(\text{overlap\_x2} - \text{overlap\_x1}\)
        \item Height: \(\text{overlap\_y2} - \text{overlap\_y1}\)
    \end{itemize}
    
    \item **Overlap Area:**
    \begin{itemize}
        \item If both width and height are positive, the rectangles overlap, and the overlapping area is their product.
        \item Otherwise, there is no overlap, and the overlapping area is zero.
    \end{itemize}
\end{itemize}

\subsection*{3. Compute Total Area}

\begin{itemize}
    \item Total Area = Area of Rectangle A + Area of Rectangle B - Overlapping Area
\end{itemize}

\subsection*{4. Example Walkthrough}

Consider the first example:
\begin{verbatim}
Input: A = [-3,0,3,4], B = [0,-1,9,2]
Output: 45
\end{verbatim}

\begin{enumerate}
    \item **Calculate Areas:**
    \begin{itemize}
        \item Area of A = (3 - (-3)) * (4 - 0) = 6 * 4 = 24
        \item Area of B = (9 - 0) * (2 - (-1)) = 9 * 3 = 27
    \end{itemize}
    
    \item **Determine Overlap:**
    \begin{itemize}
        \item overlap\_x1 = max(-3, 0) = 0
        \item overlap\_y1 = max(0, -1) = 0
        \item overlap\_x2 = min(3, 9) = 3
        \item overlap\_y2 = min(4, 2) = 2
        \item overlap\_width = 3 - 0 = 3
        \item overlap\_height = 2 - 0 = 2
        \item overlap\_area = 3 * 2 = 6
    \end{itemize}
    
    \item **Compute Total Area:**
    \begin{itemize}
        \item Total Area = 24 + 27 - 6 = 45
    \end{itemize}
\end{enumerate}

Thus, the function correctly returns \texttt{45}.

\section*{Why This Approach}

This approach is chosen for its straightforwardness and optimal efficiency. By directly calculating the individual areas and intelligently handling the overlapping region, the algorithm ensures accurate results without unnecessary computations. Its constant time complexity makes it highly efficient, even for large coordinate values.

\section*{Alternative Approaches}

\subsection*{1. Using Intersection Dimensions}

Instead of separately calculating areas, directly compute the dimensions of the overlapping region and subtract it from the sum of individual areas.

\begin{lstlisting}[language=Python]
def computeArea(A: List[int], B: List[int]) -> int:
    # Sum of individual areas
    area = (A[2] - A[0]) * (A[3] - A[1]) + (B[2] - B[0]) * (B[3] - B[1])
    
    # Overlapping area
    overlap_width = min(A[2], B[2]) - max(A[0], B[0])
    overlap_height = min(A[3], B[3]) - max(A[1], B[1])
    
    if overlap_width > 0 and overlap_height > 0:
        area -= overlap_width * overlap_height
    
    return area
\end{lstlisting}

\subsection*{2. Using Geometry Libraries}

Leverage computational geometry libraries to handle area calculations and overlapping detections.

\begin{lstlisting}[language=Python]
from shapely.geometry import box

def computeArea(A: List[int], B: List[int]) -> int:
    rect1 = box(A[0], A[1], A[2], A[3])
    rect2 = box(B[0], B[1], B[2], B[3])
    intersection = rect1.intersection(rect2)
    return int(rect1.area + rect2.area - intersection.area)
\end{lstlisting}

\textbf{Note}: This approach requires the \texttt{shapely} library and is more suitable for complex geometric operations.

\section*{Similar Problems to This One}

Several problems involve calculating areas, handling geometric overlaps, and spatial reasoning, utilizing similar algorithmic strategies:

\begin{itemize}
    \item \textbf{Rectangle Overlap}: Determine if two rectangles overlap.
    \item \textbf{Circle Area Overlap}: Calculate the overlapping area between two circles.
    \item \textbf{Polygon Area}: Compute the area of a given polygon.
    \item \textbf{Union of Rectangles}: Calculate the total area covered by multiple rectangles, accounting for overlaps.
    \item \textbf{Intersection of Lines}: Find the intersection point of two lines.
    \item \textbf{Closest Pair of Points}: Find the closest pair of points in a set.
    \item \textbf{Convex Hull}: Compute the convex hull of a set of points.
    \item \textbf{Point Inside Polygon}: Determine if a point lies inside a given polygon.
\end{itemize}

These problems reinforce concepts of geometric calculations, area computations, and efficient algorithm design in various contexts.

\section*{Things to Keep in Mind and Tricks}

When tackling the \textbf{Rectangle Area} problem, consider the following tips and best practices to enhance efficiency and correctness:

\begin{itemize}
    \item \textbf{Understand Geometric Relationships}: Grasp the positional relationships between rectangles to simplify area calculations.
    \index{Geometric Relationships}
    
    \item \textbf{Leverage Coordinate Comparisons}: Use direct comparisons of rectangle coordinates to determine overlapping regions.
    \index{Coordinate Comparisons}
    
    \item \textbf{Handle Overlapping Scenarios}: Accurately calculate the overlapping area to avoid double-counting.
    \index{Overlapping Scenarios}
    
    \item \textbf{Optimize for Efficiency}: Aim for a constant time \(O(1)\) solution by avoiding unnecessary computations or iterations.
    \index{Efficiency Optimization}
    
    \item \textbf{Avoid Floating-Point Precision Issues}: Since all coordinates are integers, floating-point precision is not a concern, simplifying the implementation.
    \index{Floating-Point Precision}
    
    \item \textbf{Use Helper Functions}: Create helper functions to encapsulate repetitive tasks, such as calculating overlap dimensions or areas.
    \index{Helper Functions}
    
    \item \textbf{Code Readability}: Maintain clear and readable code through meaningful variable names and structured logic.
    \index{Code Readability}
    
    \item \textbf{Test Extensively}: Implement a wide range of test cases, including overlapping, non-overlapping, and edge-touching rectangles, to ensure robustness.
    \index{Extensive Testing}
    
    \item \textbf{Understand Axis-Aligned Constraints}: Recognize that axis-aligned rectangles simplify area calculations compared to rotated rectangles.
    \index{Axis-Aligned Constraints}
    
    \item \textbf{Simplify Logical Conditions}: Combine multiple conditions logically to streamline the area calculation process.
    \index{Logical Conditions}
    
    \item \textbf{Use Absolute Values}: When calculating differences, ensure that the dimensions are positive by using absolute values or proper ordering.
    \index{Absolute Values}
    
    \item \textbf{Consider Edge Cases}: Handle cases where rectangles have zero area or touch at edges/corners without overlapping.
    \index{Edge Cases}
\end{itemize}

\section*{Corner and Special Cases to Test When Writing the Code}

When implementing the solution for the \textbf{Rectangle Area} problem, it is crucial to consider and rigorously test various edge cases to ensure robustness and correctness:

\begin{itemize}
    \item \textbf{No Overlap}: Rectangles are completely separate.
    \index{No Overlap}
    
    \item \textbf{Partial Overlap}: Rectangles overlap in one or more regions.
    \index{Partial Overlap}
    
    \item \textbf{Edge Touching}: Rectangles touch exactly at one edge without overlapping.
    \index{Edge Touching}
    
    \item \textbf{Corner Touching}: Rectangles touch exactly at one corner without overlapping.
    \index{Corner Touching}
    
    \item \textbf{One Rectangle Inside Another}: One rectangle is entirely within the other.
    \index{Rectangle Inside}
    
    \item \textbf{Identical Rectangles}: Both rectangles have the same coordinates.
    \index{Identical Rectangles}
    
    \item \textbf{Degenerate Rectangles}: Rectangles with zero area (e.g., \(x1 = x2\) or \(y1 = y2\)).
    \index{Degenerate Rectangles}
    
    \item \textbf{Large Coordinates}: Rectangles with very large coordinate values to test performance and integer handling.
    \index{Large Coordinates}
    
    \item \textbf{Negative Coordinates}: Rectangles positioned in negative coordinate space.
    \index{Negative Coordinates}
    
    \item \textbf{Mixed Overlapping Scenarios}: Combinations of the above cases to ensure comprehensive coverage.
    \index{Mixed Overlapping Scenarios}
    
    \item \textbf{Minimum and Maximum Bounds}: Rectangles at the minimum and maximum limits of the coordinate range.
    \index{Minimum and Maximum Bounds}
    
    \item \textbf{Sequential Rectangles}: Multiple rectangles placed sequentially without overlapping.
    \index{Sequential Rectangles}
    
    \item \textbf{Multiple Overlaps}: Scenarios where more than two rectangles overlap in different regions.
    \index{Multiple Overlaps}
\end{itemize}

\section*{Implementation Considerations}

When implementing the \texttt{computeArea} function, keep in mind the following considerations to ensure robustness and efficiency:

\begin{itemize}
    \item \textbf{Data Type Selection}: Use appropriate data types that can handle large input values without overflow or underflow.
    \index{Data Type Selection}
    
    \item \textbf{Optimizing Comparisons}: Structure logical conditions to efficiently determine overlap dimensions.
    \index{Optimizing Comparisons}
    
    \item \textbf{Handling Large Inputs}: Design the algorithm to efficiently handle large input sizes without significant performance degradation.
    \index{Handling Large Inputs}
    
    \item \textbf{Language-Specific Constraints}: Be aware of how the programming language handles large integers and arithmetic operations.
    \index{Language-Specific Constraints}
    
    \item \textbf{Avoiding Redundant Calculations}: Ensure that each calculation contributes towards determining the final area without unnecessary repetitions.
    \index{Avoiding Redundant Calculations}
    
    \item \textbf{Code Readability and Documentation}: Maintain clear and readable code through meaningful variable names and comprehensive comments to facilitate understanding and maintenance.
    \index{Code Readability}
    
    \item \textbf{Edge Case Handling}: Implement checks for edge cases to prevent incorrect results or runtime errors.
    \index{Edge Case Handling}
    
    \item \textbf{Testing and Validation}: Develop a comprehensive suite of test cases that cover all possible scenarios, including edge cases, to validate the correctness and efficiency of the implementation.
    \index{Testing and Validation}
    
    \item \textbf{Scalability}: Design the algorithm to scale efficiently with increasing input sizes, maintaining performance and resource utilization.
    \index{Scalability}
    
    \item \textbf{Using Helper Functions}: Consider creating helper functions for repetitive tasks, such as calculating overlap dimensions, to enhance modularity and reusability.
    \index{Helper Functions}
    
    \item \textbf{Consistent Naming Conventions}: Use consistent and descriptive naming conventions for variables to improve code clarity.
    \index{Naming Conventions}
    
    \item \textbf{Implementing Unit Tests}: Develop unit tests for each logical condition to ensure that all scenarios are correctly handled.
    \index{Unit Tests}
    
    \item \textbf{Error Handling}: Incorporate error handling to manage invalid inputs gracefully.
    \index{Error Handling}
\end{itemize}

\section*{Conclusion}

The \textbf{Rectangle Area} problem showcases the application of fundamental geometric principles and efficient algorithm design to compute spatial properties accurately. By systematically calculating individual areas and intelligently handling overlapping regions, the algorithm ensures precise results without redundant computations. Understanding and implementing such techniques not only enhances problem-solving skills but also provides a foundation for tackling more complex Computational Geometry challenges involving multiple geometric entities and intricate spatial relationships.

\printindex

% \input{sections/rectangle_overlap}
% \input{sections/rectangle_area}
% \input{sections/k_closest_points_to_origin}
% \input{sections/the_skyline_problem}
% % filename: k_closest_points_to_origin.tex

\problemsection{K Closest Points to Origin}
\label{chap:K_Closest_Points_to_Origin}
\marginnote{\href{https://leetcode.com/problems/k-closest-points-to-origin/}{[LeetCode Link]}\index{LeetCode}}
\marginnote{\href{https://www.geeksforgeeks.org/find-k-closest-points-origin/}{[GeeksForGeeks Link]}\index{GeeksForGeeks}}
\marginnote{\href{https://www.interviewbit.com/problems/k-closest-points/}{[InterviewBit Link]}\index{InterviewBit}}
\marginnote{\href{https://app.codesignal.com/challenges/k-closest-points-to-origin}{[CodeSignal Link]}\index{CodeSignal}}
\marginnote{\href{https://www.codewars.com/kata/k-closest-points-to-origin/train/python}{[Codewars Link]}\index{Codewars}}

The \textbf{K Closest Points to Origin} problem is a popular algorithmic challenge in Computational Geometry that involves identifying the \(k\) points closest to the origin in a 2D plane. This problem tests one's ability to apply efficient sorting and selection algorithms, understand distance computations, and optimize for performance. Mastery of this problem is essential for applications in spatial data analysis, nearest neighbor searches, and clustering algorithms.

\section*{Problem Statement}

Given an array of points where each point is represented as \([x, y]\) in the 2D plane, and an integer \(k\), return the \(k\) closest points to the origin \((0, 0)\).

The distance between two points \((x_1, y_1)\) and \((x_2, y_2)\) is the Euclidean distance \(\sqrt{(x_1 - x_2)^2 + (y_1 - y_2)^2}\). The origin is \((0, 0)\).

\textbf{Function signature in Python:}
\begin{lstlisting}[language=Python]
def kClosest(points: List[List[int]], K: int) -> List[List[int]]:
\end{lstlisting}

\section*{Examples}

\textbf{Example 1:}

\begin{verbatim}
Input: points = [[1,3],[-2,2]], K = 1
Output: [[-2,2]]
Explanation: 
The distance between (1, 3) and the origin is sqrt(10).
The distance between (-2, 2) and the origin is sqrt(8).
Since sqrt(8) < sqrt(10), (-2, 2) is closer to the origin.
\end{verbatim}

\textbf{Example 2:}

\begin{verbatim}
Input: points = [[3,3],[5,-1],[-2,4]], K = 2
Output: [[3,3],[-2,4]]
Explanation: 
The distances are sqrt(18), sqrt(26), and sqrt(20) respectively.
The two closest points are [3,3] and [-2,4].
\end{verbatim}

\textbf{Example 3:}

\begin{verbatim}
Input: points = [[0,1],[1,0]], K = 2
Output: [[0,1],[1,0]]
Explanation: 
Both points are equally close to the origin.
\end{verbatim}

\textbf{Example 4:}

\begin{verbatim}
Input: points = [[1,0],[0,1]], K = 1
Output: [[1,0]]
Explanation: 
Both points are equally close; returning any one is acceptable.
\end{verbatim}

\textbf{Constraints:}

\begin{itemize}
    \item \(1 \leq K \leq \text{points.length} \leq 10^4\)
    \item \(-10^4 < x_i, y_i < 10^4\)
\end{itemize}

LeetCode link: \href{https://leetcode.com/problems/k-closest-points-to-origin/}{K Closest Points to Origin}\index{LeetCode}

\section*{Algorithmic Approach}

To identify the \(k\) closest points to the origin, several algorithmic strategies can be employed. The most efficient methods aim to reduce the time complexity by avoiding the need to sort the entire list of points.

\subsection*{1. Sorting Based on Distance}

Calculate the Euclidean distance of each point from the origin and sort the points based on these distances. Select the first \(k\) points from the sorted list.

\begin{enumerate}
    \item Compute the distance for each point using the formula \(distance = x^2 + y^2\).
    \item Sort the points based on the computed distances.
    \item Return the first \(k\) points from the sorted list.
\end{enumerate}

\subsection*{2. Max Heap (Priority Queue)}

Use a max heap to maintain the \(k\) closest points. Iterate through each point, add it to the heap, and if the heap size exceeds \(k\), remove the farthest point.

\begin{enumerate}
    \item Initialize a max heap.
    \item For each point, compute its distance and add it to the heap.
    \item If the heap size exceeds \(k\), remove the point with the largest distance.
    \item After processing all points, the heap contains the \(k\) closest points.
\end{enumerate}

\subsection*{3. QuickSelect (Quick Sort Partitioning)}

Utilize the QuickSelect algorithm to find the \(k\) closest points without fully sorting the list.

\begin{enumerate}
    \item Choose a pivot point and partition the list based on distances relative to the pivot.
    \item Recursively apply QuickSelect to the partition containing the \(k\) closest points.
    \item Once the \(k\) closest points are identified, return them.
\end{enumerate}

\marginnote{QuickSelect offers an average time complexity of \(O(n)\), making it highly efficient for large datasets.}

\section*{Complexities}

\begin{itemize}
    \item \textbf{Sorting Based on Distance:}
    \begin{itemize}
        \item \textbf{Time Complexity:} \(O(n \log n)\)
        \item \textbf{Space Complexity:} \(O(n)\)
    \end{itemize}
    
    \item \textbf{Max Heap (Priority Queue):}
    \begin{itemize}
        \item \textbf{Time Complexity:} \(O(n \log k)\)
        \item \textbf{Space Complexity:} \(O(k)\)
    \end{itemize}
    
    \item \textbf{QuickSelect (Quick Sort Partitioning):}
    \begin{itemize}
        \item \textbf{Time Complexity:} Average case \(O(n)\), worst case \(O(n^2)\)
        \item \textbf{Space Complexity:} \(O(1)\) (in-place)
    \end{itemize}
\end{itemize}

\section*{Python Implementation}

\marginnote{Implementing QuickSelect provides an optimal average-case solution with linear time complexity.}

Below is the complete Python code implementing the \texttt{kClosest} function using the QuickSelect approach:

\begin{fullwidth}
\begin{lstlisting}[language=Python]
from typing import List
import random

class Solution:
    def kClosest(self, points: List[List[int]], K: int) -> List[List[int]]:
        def quickselect(left, right, K_smallest):
            if left == right:
                return
            
            # Select a random pivot_index
            pivot_index = random.randint(left, right)
            
            # Partition the array
            pivot_index = partition(left, right, pivot_index)
            
            # The pivot is in its final sorted position
            if K_smallest == pivot_index:
                return
            elif K_smallest < pivot_index:
                quickselect(left, pivot_index - 1, K_smallest)
            else:
                quickselect(pivot_index + 1, right, K_smallest)
        
        def partition(left, right, pivot_index):
            pivot_distance = distance(points[pivot_index])
            # Move pivot to end
            points[pivot_index], points[right] = points[right], points[pivot_index]
            store_index = left
            for i in range(left, right):
                if distance(points[i]) < pivot_distance:
                    points[store_index], points[i] = points[i], points[store_index]
                    store_index += 1
            # Move pivot to its final place
            points[right], points[store_index] = points[store_index], points[right]
            return store_index
        
        def distance(point):
            return point[0] ** 2 + point[1] ** 2
        
        n = len(points)
        quickselect(0, n - 1, K)
        return points[:K]

# Example usage:
solution = Solution()
print(solution.kClosest([[1,3],[-2,2]], 1))            # Output: [[-2,2]]
print(solution.kClosest([[3,3],[5,-1],[-2,4]], 2))     # Output: [[3,3],[-2,4]]
print(solution.kClosest([[0,1],[1,0]], 2))             # Output: [[0,1],[1,0]]
print(solution.kClosest([[1,0],[0,1]], 1))             # Output: [[1,0]] or [[0,1]]
\end{lstlisting}
\end{fullwidth}

This implementation uses the QuickSelect algorithm to efficiently find the \(k\) closest points to the origin without fully sorting the entire list. It ensures optimal performance even with large datasets.

\section*{Explanation}

The \texttt{kClosest} function identifies the \(k\) closest points to the origin using the QuickSelect algorithm. Here's a detailed breakdown of the implementation:

\subsection*{1. Distance Calculation}

\begin{itemize}
    \item The Euclidean distance is calculated as \(distance = x^2 + y^2\). Since we only need relative distances for comparison, the square root is omitted for efficiency.
\end{itemize}

\subsection*{2. QuickSelect Algorithm}

\begin{itemize}
    \item **Pivot Selection:**
    \begin{itemize}
        \item A random pivot is chosen to enhance the average-case performance.
    \end{itemize}
    
    \item **Partitioning:**
    \begin{itemize}
        \item The array is partitioned such that points with distances less than the pivot are moved to the left, and others to the right.
        \item The pivot is placed in its correct sorted position.
    \end{itemize}
    
    \item **Recursive Selection:**
    \begin{itemize}
        \item If the pivot's position matches \(K\), the selection is complete.
        \item Otherwise, recursively apply QuickSelect to the relevant partition.
    \end{itemize}
\end{itemize}

\subsection*{3. Final Selection}

\begin{itemize}
    \item After partitioning, the first \(K\) points in the list are the \(k\) closest points to the origin.
\end{itemize}

\subsection*{4. Example Walkthrough}

Consider the first example:
\begin{verbatim}
Input: points = [[1,3],[-2,2]], K = 1
Output: [[-2,2]]
\end{verbatim}

\begin{enumerate}
    \item **Calculate Distances:**
    \begin{itemize}
        \item [1,3] : \(1^2 + 3^2 = 10\)
        \item [-2,2] : \((-2)^2 + 2^2 = 8\)
    \end{itemize}
    
    \item **QuickSelect Process:**
    \begin{itemize}
        \item Choose a pivot, say [1,3] with distance 10.
        \item Compare and rearrange:
        \begin{itemize}
            \item [-2,2] has a smaller distance (8) and is moved to the left.
        \end{itemize}
        \item After partitioning, the list becomes [[-2,2], [1,3]].
        \item Since \(K = 1\), return the first point: [[-2,2]].
    \end{itemize}
\end{enumerate}

Thus, the function correctly identifies \([-2,2]\) as the closest point to the origin.

\section*{Why This Approach}

The QuickSelect algorithm is chosen for its average-case linear time complexity \(O(n)\), making it highly efficient for large datasets compared to sorting-based methods with \(O(n \log n)\) time complexity. By avoiding the need to sort the entire list, QuickSelect provides an optimal solution for finding the \(k\) closest points.

\section*{Alternative Approaches}

\subsection*{1. Sorting Based on Distance}

Sort all points based on their distances from the origin and select the first \(k\) points.

\begin{lstlisting}[language=Python]
class Solution:
    def kClosest(self, points: List[List[int]], K: int) -> List[List[int]]:
        points.sort(key=lambda P: P[0]**2 + P[1]**2)
        return points[:K]
\end{lstlisting}

\textbf{Complexities:}
\begin{itemize}
    \item \textbf{Time Complexity:} \(O(n \log n)\)
    \item \textbf{Space Complexity:} \(O(1)\)
\end{itemize}

\subsection*{2. Max Heap (Priority Queue)}

Use a max heap to maintain the \(k\) closest points.

\begin{lstlisting}[language=Python]
import heapq

class Solution:
    def kClosest(self, points: List[List[int]], K: int) -> List[List[int]]:
        heap = []
        for (x, y) in points:
            dist = -(x**2 + y**2)  # Max heap using negative distances
            heapq.heappush(heap, (dist, [x, y]))
            if len(heap) > K:
                heapq.heappop(heap)
        return [item[1] for item in heap]
\end{lstlisting}

\textbf{Complexities:}
\begin{itemize}
    \item \textbf{Time Complexity:} \(O(n \log k)\)
    \item \textbf{Space Complexity:} \(O(k)\)
\end{itemize}

\subsection*{3. Using Built-In Functions}

Leverage built-in functions for distance calculation and selection.

\begin{lstlisting}[language=Python]
import math

class Solution:
    def kClosest(self, points: List[List[int]], K: int) -> List[List[int]]:
        points.sort(key=lambda P: math.sqrt(P[0]**2 + P[1]**2))
        return points[:K]
\end{lstlisting}

\textbf{Note}: This method is similar to the sorting approach but uses the actual Euclidean distance.

\section*{Similar Problems to This One}

Several problems involve nearest neighbor searches, spatial data analysis, and efficient selection algorithms, utilizing similar algorithmic strategies:

\begin{itemize}
    \item \textbf{Closest Pair of Points}: Find the closest pair of points in a set.
    \item \textbf{Top K Frequent Elements}: Identify the most frequent elements in a dataset.
    \item \textbf{Kth Largest Element in an Array}: Find the \(k\)-th largest element in an unsorted array.
    \item \textbf{Sliding Window Maximum}: Find the maximum in each sliding window of size \(k\) over an array.
    \item \textbf{Merge K Sorted Lists}: Merge multiple sorted lists into a single sorted list.
    \item \textbf{Find Median from Data Stream}: Continuously find the median of a stream of numbers.
    \item \textbf{Top K Closest Stars}: Find the \(k\) closest stars to Earth based on their distances.
\end{itemize}

These problems reinforce concepts of efficient selection, heap usage, and distance computations in various contexts.

\section*{Things to Keep in Mind and Tricks}

When solving the \textbf{K Closest Points to Origin} problem, consider the following tips and best practices to enhance efficiency and correctness:

\begin{itemize}
    \item \textbf{Understand Distance Calculations}: Grasp the Euclidean distance formula and recognize that the square root can be omitted for comparison purposes.
    \index{Distance Calculations}
    
    \item \textbf{Leverage Efficient Algorithms}: Use QuickSelect or heap-based methods to optimize time complexity, especially for large datasets.
    \index{Efficient Algorithms}
    
    \item \textbf{Handle Ties Appropriately}: Decide how to handle points with identical distances when \(k\) is less than the number of such points.
    \index{Handling Ties}
    
    \item \textbf{Optimize Space Usage}: Choose algorithms that minimize additional space, such as in-place QuickSelect.
    \index{Space Optimization}
    
    \item \textbf{Use Appropriate Data Structures}: Utilize heaps, lists, and helper functions effectively to manage and process data.
    \index{Data Structures}
    
    \item \textbf{Implement Helper Functions}: Create helper functions for distance calculation and partitioning to enhance code modularity.
    \index{Helper Functions}
    
    \item \textbf{Code Readability}: Maintain clear and readable code through meaningful variable names and structured logic.
    \index{Code Readability}
    
    \item \textbf{Test Extensively}: Implement a wide range of test cases, including edge cases like multiple points with the same distance, to ensure robustness.
    \index{Extensive Testing}
    
    \item \textbf{Understand Algorithm Trade-offs}: Recognize the trade-offs between different approaches in terms of time and space complexities.
    \index{Algorithm Trade-offs}
    
    \item \textbf{Use Built-In Sorting Functions}: When using sorting-based approaches, leverage built-in functions for efficiency and simplicity.
    \index{Built-In Sorting}
    
    \item \textbf{Avoid Redundant Calculations}: Ensure that distance calculations are performed only when necessary to optimize performance.
    \index{Avoiding Redundant Calculations}
    
    \item \textbf{Language-Specific Features}: Utilize language-specific features or libraries that can simplify implementation, such as heapq in Python.
    \index{Language-Specific Features}
\end{itemize}

\section*{Corner and Special Cases to Test When Writing the Code}

When implementing the solution for the \textbf{K Closest Points to Origin} problem, it is crucial to consider and rigorously test various edge cases to ensure robustness and correctness:

\begin{itemize}
    \item \textbf{Multiple Points with Same Distance}: Ensure that the algorithm handles multiple points having the same distance from the origin.
    \index{Same Distance Points}
    
    \item \textbf{Points at Origin}: Include points that are exactly at the origin \((0,0)\).
    \index{Points at Origin}
    
    \item \textbf{Negative Coordinates}: Ensure that the algorithm correctly computes distances for points with negative \(x\) or \(y\) coordinates.
    \index{Negative Coordinates}
    
    \item \textbf{Large Coordinates}: Test with points having very large or very small coordinate values to verify integer handling.
    \index{Large Coordinates}
    
    \item \textbf{K Equals Number of Points}: When \(K\) is equal to the number of points, the algorithm should return all points.
    \index{K Equals Number of Points}
    
    \item \textbf{K is One}: Test with \(K = 1\) to ensure the closest point is correctly identified.
    \index{K is One}
    
    \item \textbf{All Points Same}: All points have the same coordinates.
    \index{All Points Same}
    
    \item \textbf{K is Zero}: Although \(K\) is defined to be at least 1, ensure that the algorithm gracefully handles \(K = 0\) if allowed.
    \index{K is Zero}
    
    \item \textbf{Single Point}: Only one point is provided, and \(K = 1\).
    \index{Single Point}
    
    \item \textbf{Mixed Coordinates}: Points with a mix of positive and negative coordinates.
    \index{Mixed Coordinates}
    
    \item \textbf{Points with Zero Distance}: Multiple points at the origin.
    \index{Zero Distance Points}
    
    \item \textbf{Sparse and Dense Points}: Densely packed points and sparsely distributed points.
    \index{Sparse and Dense Points}
    
    \item \textbf{Duplicate Points}: Multiple identical points in the input list.
    \index{Duplicate Points}
    
    \item \textbf{K Greater Than Number of Unique Points}: Ensure that the algorithm handles cases where \(K\) exceeds the number of unique points if applicable.
    \index{K Greater Than Unique Points}
\end{itemize}

\section*{Implementation Considerations}

When implementing the \texttt{kClosest} function, keep in mind the following considerations to ensure robustness and efficiency:

\begin{itemize}
    \item \textbf{Data Type Selection}: Use appropriate data types that can handle large input values without overflow or precision loss.
    \index{Data Type Selection}
    
    \item \textbf{Optimizing Distance Calculations}: Avoid calculating the square root since it is unnecessary for comparison purposes.
    \index{Optimizing Distance Calculations}
    
    \item \textbf{Choosing the Right Algorithm}: Select an algorithm based on the size of the input and the value of \(K\) to optimize time and space complexities.
    \index{Choosing the Right Algorithm}
    
    \item \textbf{Language-Specific Libraries}: Utilize language-specific libraries and functions (e.g., \texttt{heapq} in Python) to simplify implementation and enhance performance.
    \index{Language-Specific Libraries}
    
    \item \textbf{Avoiding Redundant Calculations}: Ensure that each point's distance is calculated only once to optimize performance.
    \index{Avoiding Redundant Calculations}
    
    \item \textbf{Implementing Helper Functions}: Create helper functions for tasks like distance calculation and partitioning to enhance modularity and readability.
    \index{Helper Functions}
    
    \item \textbf{Edge Case Handling}: Implement checks for edge cases to prevent incorrect results or runtime errors.
    \index{Edge Case Handling}
    
    \item \textbf{Testing and Validation}: Develop a comprehensive suite of test cases that cover all possible scenarios, including edge cases, to validate the correctness and efficiency of the implementation.
    \index{Testing and Validation}
    
    \item \textbf{Scalability}: Design the algorithm to scale efficiently with increasing input sizes, maintaining performance and resource utilization.
    \index{Scalability}
    
    \item \textbf{Consistent Naming Conventions}: Use consistent and descriptive naming conventions for variables and functions to improve code clarity.
    \index{Naming Conventions}
    
    \item \textbf{Memory Management}: Ensure that the algorithm manages memory efficiently, especially when dealing with large datasets.
    \index{Memory Management}
    
    \item \textbf{Avoiding Stack Overflow}: If implementing recursive approaches, be mindful of recursion limits and potential stack overflow issues.
    \index{Avoiding Stack Overflow}
    
    \item \textbf{Implementing Iterative Solutions}: Prefer iterative solutions when recursion may lead to increased space complexity or stack overflow.
    \index{Implementing Iterative Solutions}
\end{itemize}

\section*{Conclusion}

The \textbf{K Closest Points to Origin} problem exemplifies the application of efficient selection algorithms and geometric computations to solve spatial challenges effectively. By leveraging QuickSelect or heap-based methods, the algorithm achieves optimal time and space complexities, making it highly suitable for large datasets. Understanding and implementing such techniques not only enhances problem-solving skills but also provides a foundation for tackling more advanced Computational Geometry problems involving nearest neighbor searches, clustering, and spatial data analysis.

\printindex

% \input{sections/rectangle_overlap}
% \input{sections/rectangle_area}
% \input{sections/k_closest_points_to_origin}
% \input{sections/the_skyline_problem}
% % filename: the_skyline_problem.tex

\problemsection{The Skyline Problem}
\label{chap:The_Skyline_Problem}
\marginnote{\href{https://leetcode.com/problems/the-skyline-problem/}{[LeetCode Link]}\index{LeetCode}}
\marginnote{\href{https://www.geeksforgeeks.org/the-skyline-problem/}{[GeeksForGeeks Link]}\index{GeeksForGeeks}}
\marginnote{\href{https://www.interviewbit.com/problems/the-skyline-problem/}{[InterviewBit Link]}\index{InterviewBit}}
\marginnote{\href{https://app.codesignal.com/challenges/the-skyline-problem}{[CodeSignal Link]}\index{CodeSignal}}
\marginnote{\href{https://www.codewars.com/kata/the-skyline-problem/train/python}{[Codewars Link]}\index{Codewars}}

The \textbf{Skyline Problem} is a complex Computational Geometry challenge that involves computing the skyline formed by a collection of buildings in a 2D cityscape. Each building is represented by its left and right x-coordinates and its height. The skyline is defined by a list of "key points" where the height changes. This problem tests one's ability to handle large datasets, implement efficient sweep line algorithms, and manage event-driven processing. Mastery of this problem is essential for applications in computer graphics, urban planning simulations, and geographic information systems (GIS).

\section*{Problem Statement}

You are given a list of buildings in a cityscape. Each building is represented as a triplet \([Li, Ri, Hi]\), where \(Li\) and \(Ri\) are the x-coordinates of the left and right edges of the building, respectively, and \(Hi\) is the height of the building.

The skyline should be represented as a list of key points \([x, y]\) in sorted order by \(x\)-coordinate, where \(y\) is the height of the skyline at that point. The skyline should only include critical points where the height changes.

\textbf{Function signature in Python:}
\begin{lstlisting}[language=Python]
def getSkyline(buildings: List[List[int]]) -> List[List[int]]:
\end{lstlisting}

\section*{Examples}

\textbf{Example 1:}

\begin{verbatim}
Input: buildings = [[2,9,10], [3,7,15], [5,12,12], [15,20,10], [19,24,8]]
Output: [[2,10], [3,15], [7,12], [12,0], [15,10], [20,8], [24,0]]
Explanation:
- At x=2, the first building starts, height=10.
- At x=3, the second building starts, height=15.
- At x=7, the second building ends, the third building is still ongoing, height=12.
- At x=12, the third building ends, height drops to 0.
- At x=15, the fourth building starts, height=10.
- At x=20, the fourth building ends, the fifth building is still ongoing, height=8.
- At x=24, the fifth building ends, height drops to 0.
\end{verbatim}

\textbf{Example 2:}

\begin{verbatim}
Input: buildings = [[0,2,3], [2,5,3]]
Output: [[0,3], [5,0]]
Explanation:
- The two buildings are contiguous and have the same height, so the skyline drops to 0 at x=5.
\end{verbatim}

\textbf{Example 3:}

\begin{verbatim}
Input: buildings = [[1,3,3], [2,4,4], [5,6,1]]
Output: [[1,3], [2,4], [4,0], [5,1], [6,0]]
Explanation:
- At x=1, first building starts, height=3.
- At x=2, second building starts, height=4.
- At x=4, second building ends, height drops to 0.
- At x=5, third building starts, height=1.
- At x=6, third building ends, height drops to 0.
\end{verbatim}

\textbf{Example 4:}

\begin{verbatim}
Input: buildings = [[0,5,0]]
Output: []
Explanation:
- A building with height 0 does not contribute to the skyline.
\end{verbatim}

\textbf{Constraints:}

\begin{itemize}
    \item \(1 \leq \text{buildings.length} \leq 10^4\)
    \item \(0 \leq Li < Ri \leq 10^9\)
    \item \(0 \leq Hi \leq 10^4\)
\end{itemize}

\section*{Algorithmic Approach}

The \textbf{Sweep Line Algorithm} is an efficient method for solving the Skyline Problem. It involves processing events (building start and end points) in sorted order while maintaining a data structure (typically a max heap) to keep track of active buildings. Here's a step-by-step approach:

\subsection*{1. Event Representation}

Transform each building into two events:
\begin{itemize}
    \item **Start Event:** \((Li, -Hi)\) – Negative height indicates a building starts.
    \item **End Event:** \((Ri, Hi)\) – Positive height indicates a building ends.
\end{itemize}

Sorting the events ensures that start events are processed before end events at the same x-coordinate, and taller buildings are processed before shorter ones.

\subsection*{2. Sorting the Events}

Sort all events based on:
\begin{enumerate}
    \item **x-coordinate:** Ascending order.
    \item **Height:**
    \begin{itemize}
        \item For start events, taller buildings come first.
        \item For end events, shorter buildings come first.
    \end{itemize}
\end{enumerate}

\subsection*{3. Processing the Events}

Use a max heap to keep track of active building heights. Iterate through the sorted events:
\begin{enumerate}
    \item **Start Event:**
    \begin{itemize}
        \item Add the building's height to the heap.
    \end{itemize}
    
    \item **End Event:**
    \begin{itemize}
        \item Remove the building's height from the heap.
    \end{itemize}
    
    \item **Determine Current Max Height:**
    \begin{itemize}
        \item The current max height is the top of the heap.
    \end{itemize}
    
    \item **Update Skyline:**
    \begin{itemize}
        \item If the current max height differs from the previous max height, add a new key point \([x, current\_max\_height]\).
    \end{itemize}
\end{enumerate}

\subsection*{4. Finalizing the Skyline}

After processing all events, the accumulated key points represent the skyline.

\marginnote{The Sweep Line Algorithm efficiently handles dynamic changes in active buildings, ensuring accurate skyline construction.}

\section*{Complexities}

\begin{itemize}
    \item \textbf{Time Complexity:} \(O(n \log n)\), where \(n\) is the number of buildings. Sorting the events takes \(O(n \log n)\), and each heap operation takes \(O(\log n)\).
    
    \item \textbf{Space Complexity:} \(O(n)\), due to the storage of events and the heap.
\end{itemize}

\section*{Python Implementation}

\marginnote{Implementing the Sweep Line Algorithm with a max heap ensures an efficient and accurate solution.}

Below is the complete Python code implementing the \texttt{getSkyline} function:

\begin{fullwidth}
\begin{lstlisting}[language=Python]
from typing import List
import heapq

class Solution:
    def getSkyline(self, buildings: List[List[int]]) -> List[List[int]]:
        # Create a list of all events
        # For start events, use negative height to ensure they are processed before end events
        events = []
        for L, R, H in buildings:
            events.append((L, -H))
            events.append((R, H))
        
        # Sort the events
        # First by x-coordinate, then by height
        events.sort()
        
        # Max heap to keep track of active buildings
        heap = [0]  # Initialize with ground level
        heapq.heapify(heap)
        active_heights = {0: 1}  # Dictionary to count heights
        
        result = []
        prev_max = 0
        
        for x, h in events:
            if h < 0:
                # Start of a building, add height to heap and dictionary
                heapq.heappush(heap, h)
                active_heights[h] = active_heights.get(h, 0) + 1
            else:
                # End of a building, remove height from dictionary
                active_heights[h] -= 1
                if active_heights[h] == 0:
                    del active_heights[h]
            
            # Current max height
            while heap and active_heights.get(heap[0], 0) == 0:
                heapq.heappop(heap)
            current_max = -heap[0] if heap else 0
            
            # If the max height has changed, add to result
            if current_max != prev_max:
                result.append([x, current_max])
                prev_max = current_max
        
        return result

# Example usage:
solution = Solution()
print(solution.getSkyline([[2,9,10], [3,7,15], [5,12,12], [15,20,10], [19,24,8]]))
# Output: [[2,10], [3,15], [7,12], [12,0], [15,10], [20,8], [24,0]]

print(solution.getSkyline([[0,2,3], [2,5,3]]))
# Output: [[0,3], [5,0]]

print(solution.getSkyline([[1,3,3], [2,4,4], [5,6,1]]))
# Output: [[1,3], [2,4], [4,0], [5,1], [6,0]]

print(solution.getSkyline([[0,5,0]]))
# Output: []
\end{lstlisting}
\end{fullwidth}

This implementation efficiently constructs the skyline by processing all building events in sorted order and maintaining active building heights using a max heap. It ensures that only critical points where the skyline changes are recorded.

\section*{Explanation}

The \texttt{getSkyline} function constructs the skyline formed by a set of buildings by leveraging the Sweep Line Algorithm and a max heap to track active buildings. Here's a detailed breakdown of the implementation:

\subsection*{1. Event Representation}

\begin{itemize}
    \item Each building is transformed into two events:
    \begin{itemize}
        \item **Start Event:** \((Li, -Hi)\) – Negative height indicates the start of a building.
        \item **End Event:** \((Ri, Hi)\) – Positive height indicates the end of a building.
    \end{itemize}
\end{itemize}

\subsection*{2. Sorting the Events}

\begin{itemize}
    \item Events are sorted primarily by their x-coordinate in ascending order.
    \item For events with the same x-coordinate:
    \begin{itemize}
        \item Start events (with negative heights) are processed before end events.
        \item Taller buildings are processed before shorter ones.
    \end{itemize}
\end{itemize}

\subsection*{3. Processing the Events}

\begin{itemize}
    \item **Heap Initialization:**
    \begin{itemize}
        \item A max heap is initialized with a ground level height of 0.
        \item A dictionary \texttt{active\_heights} tracks the count of active building heights.
    \end{itemize}
    
    \item **Iterating Through Events:**
    \begin{enumerate}
        \item **Start Event:**
        \begin{itemize}
            \item Add the building's height to the heap.
            \item Increment the count of the height in \texttt{active\_heights}.
        \end{itemize}
        
        \item **End Event:**
        \begin{itemize}
            \item Decrement the count of the building's height in \texttt{active\_heights}.
            \item If the count reaches zero, remove the height from the dictionary.
        \end{itemize}
        
        \item **Determine Current Max Height:**
        \begin{itemize}
            \item Remove heights from the heap that are no longer active.
            \item The current max height is the top of the heap.
        \end{itemize}
        
        \item **Update Skyline:**
        \begin{itemize}
            \item If the current max height differs from the previous max height, add a new key point \([x, current\_max\_height]\).
        \end{itemize}
    \end{enumerate}
\end{itemize}

\subsection*{4. Finalizing the Skyline}

\begin{itemize}
    \item After processing all events, the \texttt{result} list contains the key points defining the skyline.
\end{itemize}

\subsection*{5. Example Walkthrough}

Consider the first example:
\begin{verbatim}
Input: buildings = [[2,9,10], [3,7,15], [5,12,12], [15,20,10], [19,24,8]]
Output: [[2,10], [3,15], [7,12], [12,0], [15,10], [20,8], [24,0]]
\end{verbatim}

\begin{enumerate}
    \item **Event Transformation:**
    \begin{itemize}
        \item \((2, -10)\), \((9, 10)\)
        \item \((3, -15)\), \((7, 15)\)
        \item \((5, -12)\), \((12, 12)\)
        \item \((15, -10)\), \((20, 10)\)
        \item \((19, -8)\), \((24, 8)\)
    \end{itemize}
    
    \item **Sorting Events:**
    \begin{itemize}
        \item Sorted order: \((2, -10)\), \((3, -15)\), \((5, -12)\), \((7, 15)\), \((9, 10)\), \((12, 12)\), \((15, -10)\), \((19, -8)\), \((20, 10)\), \((24, 8)\)
    \end{itemize}
    
    \item **Processing Events:**
    \begin{itemize}
        \item At each event, update the heap and determine if the skyline height changes.
    \end{itemize}
    
    \item **Result Construction:**
    \begin{itemize}
        \item The resulting skyline key points are accumulated as \([[2,10], [3,15], [7,12], [12,0], [15,10], [20,8], [24,0]]\).
    \end{itemize}
\end{enumerate}

Thus, the function correctly constructs the skyline based on the buildings' positions and heights.

\section*{Why This Approach}

The Sweep Line Algorithm combined with a max heap offers an optimal solution with \(O(n \log n)\) time complexity and efficient handling of overlapping buildings. By processing events in sorted order and maintaining active building heights, the algorithm ensures that all critical points in the skyline are accurately identified without redundant computations.

\section*{Alternative Approaches}

\subsection*{1. Divide and Conquer}

Divide the set of buildings into smaller subsets, compute the skyline for each subset, and then merge the skylines.

\begin{lstlisting}[language=Python]
class Solution:
    def getSkyline(self, buildings: List[List[int]]) -> List[List[int]]:
        def merge(left, right):
            h1, h2 = 0, 0
            i, j = 0, 0
            merged = []
            while i < len(left) and j < len(right):
                if left[i][0] < right[j][0]:
                    x, h1 = left[i]
                    i += 1
                elif left[i][0] > right[j][0]:
                    x, h2 = right[j]
                    j += 1
                else:
                    x, h1 = left[i]
                    _, h2 = right[j]
                    i += 1
                    j += 1
                max_h = max(h1, h2)
                if not merged or merged[-1][1] != max_h:
                    merged.append([x, max_h])
            merged.extend(left[i:])
            merged.extend(right[j:])
            return merged
        
        def divide(buildings):
            if not buildings:
                return []
            if len(buildings) == 1:
                L, R, H = buildings[0]
                return [[L, H], [R, 0]]
            mid = len(buildings) // 2
            left = divide(buildings[:mid])
            right = divide(buildings[mid:])
            return merge(left, right)
        
        return divide(buildings)
\end{lstlisting}

\textbf{Complexities:}
\begin{itemize}
    \item \textbf{Time Complexity:} \(O(n \log n)\)
    \item \textbf{Space Complexity:} \(O(n)\)
\end{itemize}

\subsection*{2. Using Segment Trees}

Implement a segment tree to manage and query overlapping building heights dynamically.

\textbf{Note}: This approach is more complex and is generally used for advanced scenarios with multiple dynamic queries.

\section*{Similar Problems to This One}

Several problems involve skyline-like constructions, spatial data analysis, and efficient event processing, utilizing similar algorithmic strategies:

\begin{itemize}
    \item \textbf{Merge Intervals}: Merge overlapping intervals in a list.
    \item \textbf{Largest Rectangle in Histogram}: Find the largest rectangular area in a histogram.
    \item \textbf{Interval Partitioning}: Assign intervals to resources without overlap.
    \item \textbf{Line Segment Intersection}: Detect intersections among line segments.
    \item \textbf{Closest Pair of Points}: Find the closest pair of points in a set.
    \item \textbf{Convex Hull}: Compute the convex hull of a set of points.
    \item \textbf{Point Inside Polygon}: Determine if a point lies inside a given polygon.
    \item \textbf{Range Searching}: Efficiently query geometric data within a specified range.
\end{itemize}

These problems reinforce concepts of event-driven processing, spatial reasoning, and efficient algorithm design in various contexts.

\section*{Things to Keep in Mind and Tricks}

When tackling the \textbf{Skyline Problem}, consider the following tips and best practices to enhance efficiency and correctness:

\begin{itemize}
    \item \textbf{Understand Sweep Line Technique}: Grasp how the sweep line algorithm processes events in sorted order to handle dynamic changes efficiently.
    \index{Sweep Line Technique}
    
    \item \textbf{Leverage Priority Queues (Heaps)}: Use max heaps to keep track of active buildings' heights, enabling quick access to the current maximum height.
    \index{Priority Queues}
    
    \item \textbf{Handle Start and End Events Differently}: Differentiate between building start and end events to accurately manage active heights.
    \index{Start and End Events}
    
    \item \textbf{Optimize Event Sorting}: Sort events primarily by x-coordinate and secondarily by height to ensure correct processing order.
    \index{Event Sorting}
    
    \item \textbf{Manage Active Heights Efficiently}: Use data structures that allow efficient insertion, deletion, and retrieval of maximum elements.
    \index{Active Heights Management}
    
    \item \textbf{Avoid Redundant Key Points}: Only record key points when the skyline height changes to minimize the output list.
    \index{Avoiding Redundant Key Points}
    
    \item \textbf{Implement Helper Functions}: Create helper functions for tasks like distance calculation, event handling, and heap management to enhance modularity.
    \index{Helper Functions}
    
    \item \textbf{Code Readability}: Maintain clear and readable code through meaningful variable names and structured logic.
    \index{Code Readability}
    
    \item \textbf{Test Extensively}: Implement a wide range of test cases, including overlapping, non-overlapping, and edge-touching buildings, to ensure robustness.
    \index{Extensive Testing}
    
    \item \textbf{Handle Degenerate Cases}: Manage cases where buildings have zero height or identical coordinates gracefully.
    \index{Degenerate Cases}
    
    \item \textbf{Understand Geometric Relationships}: Grasp how buildings overlap and influence the skyline to simplify the algorithm.
    \index{Geometric Relationships}
    
    \item \textbf{Use Appropriate Data Structures}: Utilize appropriate data structures like heaps, lists, and dictionaries to manage and process data efficiently.
    \index{Appropriate Data Structures}
    
    \item \textbf{Optimize for Large Inputs}: Design the algorithm to handle large numbers of buildings without significant performance degradation.
    \index{Optimizing for Large Inputs}
    
    \item \textbf{Implement Iterative Solutions Carefully}: Ensure that loop conditions are correctly defined to prevent infinite loops or incorrect terminations.
    \index{Iterative Solutions}
    
    \item \textbf{Consistent Naming Conventions}: Use consistent and descriptive naming conventions for variables and functions to improve code clarity.
    \index{Naming Conventions}
\end{itemize}

\section*{Corner and Special Cases to Test When Writing the Code}

When implementing the solution for the \textbf{Skyline Problem}, it is crucial to consider and rigorously test various edge cases to ensure robustness and correctness:

\begin{itemize}
    \item \textbf{No Overlapping Buildings}: All buildings are separate and do not overlap.
    \index{No Overlapping Buildings}
    
    \item \textbf{Fully Overlapping Buildings}: Multiple buildings completely overlap each other.
    \index{Fully Overlapping Buildings}
    
    \item \textbf{Buildings Touching at Edges}: Buildings share common edges without overlapping.
    \index{Buildings Touching at Edges}
    
    \item \textbf{Buildings Touching at Corners}: Buildings share common corners without overlapping.
    \index{Buildings Touching at Corners}
    
    \item \textbf{Single Building}: Only one building is present.
    \index{Single Building}
    
    \item \textbf{Multiple Buildings with Same Start or End}: Multiple buildings start or end at the same x-coordinate.
    \index{Same Start or End}
    
    \item \textbf{Buildings with Zero Height}: Buildings that have zero height should not affect the skyline.
    \index{Buildings with Zero Height}
    
    \item \textbf{Large Number of Buildings}: Test with a large number of buildings to ensure performance and scalability.
    \index{Large Number of Buildings}
    
    \item \textbf{Buildings with Negative Coordinates}: Buildings positioned in negative coordinate space.
    \index{Negative Coordinates}
    
    \item \textbf{Boundary Values}: Buildings at the minimum and maximum limits of the coordinate range.
    \index{Boundary Values}
    
    \item \textbf{Buildings with Identical Coordinates}: Multiple buildings with the same coordinates.
    \index{Identical Coordinates}
    
    \item \textbf{Sequential Buildings}: Buildings placed sequentially without gaps.
    \index{Sequential Buildings}
    
    \item \textbf{Overlapping and Non-Overlapping Mixed}: A mix of overlapping and non-overlapping buildings.
    \index{Overlapping and Non-Overlapping Mixed}
    
    \item \textbf{Buildings with Very Large Heights}: Buildings with heights at the upper limit of the constraints.
    \index{Very Large Heights}
    
    \item \textbf{Empty Input}: No buildings are provided.
    \index{Empty Input}
\end{itemize}

\section*{Implementation Considerations}

When implementing the \texttt{getSkyline} function, keep in mind the following considerations to ensure robustness and efficiency:

\begin{itemize}
    \item \textbf{Data Type Selection}: Use appropriate data types that can handle large input values and avoid overflow or precision issues.
    \index{Data Type Selection}
    
    \item \textbf{Optimizing Event Sorting}: Efficiently sort events based on x-coordinates and heights to ensure correct processing order.
    \index{Optimizing Event Sorting}
    
    \item \textbf{Handling Large Inputs}: Design the algorithm to handle up to \(10^4\) buildings efficiently without significant performance degradation.
    \index{Handling Large Inputs}
    
    \item \textbf{Using Efficient Data Structures}: Utilize heaps, lists, and dictionaries effectively to manage and process events and active heights.
    \index{Efficient Data Structures}
    
    \item \textbf{Avoiding Redundant Calculations}: Ensure that distance and overlap calculations are performed only when necessary to optimize performance.
    \index{Avoiding Redundant Calculations}
    
    \item \textbf{Code Readability and Documentation}: Maintain clear and readable code through meaningful variable names and comprehensive comments to facilitate understanding and maintenance.
    \index{Code Readability}
    
    \item \textbf{Edge Case Handling}: Implement checks for edge cases to prevent incorrect results or runtime errors.
    \index{Edge Case Handling}
    
    \item \textbf{Implementing Helper Functions}: Create helper functions for tasks like distance calculation, event handling, and heap management to enhance modularity.
    \index{Helper Functions}
    
    \item \textbf{Consistent Naming Conventions}: Use consistent and descriptive naming conventions for variables and functions to improve code clarity.
    \index{Naming Conventions}
    
    \item \textbf{Memory Management}: Ensure that the algorithm manages memory efficiently, especially when dealing with large datasets.
    \index{Memory Management}
    
    \item \textbf{Implementing Iterative Solutions Carefully}: Ensure that loop conditions are correctly defined to prevent infinite loops or incorrect terminations.
    \index{Iterative Solutions}
    
    \item \textbf{Avoiding Floating-Point Precision Issues}: Since the problem deals with integers, floating-point precision is not a concern, simplifying the implementation.
    \index{Floating-Point Precision}
    
    \item \textbf{Testing and Validation}: Develop a comprehensive suite of test cases that cover all possible scenarios, including edge cases, to validate the correctness and efficiency of the implementation.
    \index{Testing and Validation}
    
    \item \textbf{Performance Considerations}: Optimize the loop conditions and operations to ensure that the function runs efficiently, especially for large input numbers.
    \index{Performance Considerations}
\end{itemize}

\section*{Conclusion}

The \textbf{Skyline Problem} is a quintessential example of applying advanced algorithmic techniques and geometric reasoning to solve complex spatial challenges. By leveraging the Sweep Line Algorithm and maintaining active building heights using a max heap, the solution efficiently constructs the skyline with optimal time and space complexities. Understanding and implementing such sophisticated algorithms not only enhances problem-solving skills but also provides a foundation for tackling a wide array of Computational Geometry problems in various domains, including computer graphics, urban planning simulations, and geographic information systems.

\printindex

% \input{sections/rectangle_overlap}
% \input{sections/rectangle_area}
% \input{sections/k_closest_points_to_origin}
% \input{sections/the_skyline_problem}
% % filename: the_skyline_problem.tex

\problemsection{The Skyline Problem}
\label{chap:The_Skyline_Problem}
\marginnote{\href{https://leetcode.com/problems/the-skyline-problem/}{[LeetCode Link]}\index{LeetCode}}
\marginnote{\href{https://www.geeksforgeeks.org/the-skyline-problem/}{[GeeksForGeeks Link]}\index{GeeksForGeeks}}
\marginnote{\href{https://www.interviewbit.com/problems/the-skyline-problem/}{[InterviewBit Link]}\index{InterviewBit}}
\marginnote{\href{https://app.codesignal.com/challenges/the-skyline-problem}{[CodeSignal Link]}\index{CodeSignal}}
\marginnote{\href{https://www.codewars.com/kata/the-skyline-problem/train/python}{[Codewars Link]}\index{Codewars}}

The \textbf{Skyline Problem} is a complex Computational Geometry challenge that involves computing the skyline formed by a collection of buildings in a 2D cityscape. Each building is represented by its left and right x-coordinates and its height. The skyline is defined by a list of "key points" where the height changes. This problem tests one's ability to handle large datasets, implement efficient sweep line algorithms, and manage event-driven processing. Mastery of this problem is essential for applications in computer graphics, urban planning simulations, and geographic information systems (GIS).

\section*{Problem Statement}

You are given a list of buildings in a cityscape. Each building is represented as a triplet \([Li, Ri, Hi]\), where \(Li\) and \(Ri\) are the x-coordinates of the left and right edges of the building, respectively, and \(Hi\) is the height of the building.

The skyline should be represented as a list of key points \([x, y]\) in sorted order by \(x\)-coordinate, where \(y\) is the height of the skyline at that point. The skyline should only include critical points where the height changes.

\textbf{Function signature in Python:}
\begin{lstlisting}[language=Python]
def getSkyline(buildings: List[List[int]]) -> List[List[int]]:
\end{lstlisting}

\section*{Examples}

\textbf{Example 1:}

\begin{verbatim}
Input: buildings = [[2,9,10], [3,7,15], [5,12,12], [15,20,10], [19,24,8]]
Output: [[2,10], [3,15], [7,12], [12,0], [15,10], [20,8], [24,0]]
Explanation:
- At x=2, the first building starts, height=10.
- At x=3, the second building starts, height=15.
- At x=7, the second building ends, the third building is still ongoing, height=12.
- At x=12, the third building ends, height drops to 0.
- At x=15, the fourth building starts, height=10.
- At x=20, the fourth building ends, the fifth building is still ongoing, height=8.
- At x=24, the fifth building ends, height drops to 0.
\end{verbatim}

\textbf{Example 2:}

\begin{verbatim}
Input: buildings = [[0,2,3], [2,5,3]]
Output: [[0,3], [5,0]]
Explanation:
- The two buildings are contiguous and have the same height, so the skyline drops to 0 at x=5.
\end{verbatim}

\textbf{Example 3:}

\begin{verbatim}
Input: buildings = [[1,3,3], [2,4,4], [5,6,1]]
Output: [[1,3], [2,4], [4,0], [5,1], [6,0]]
Explanation:
- At x=1, first building starts, height=3.
- At x=2, second building starts, height=4.
- At x=4, second building ends, height drops to 0.
- At x=5, third building starts, height=1.
- At x=6, third building ends, height drops to 0.
\end{verbatim}

\textbf{Example 4:}

\begin{verbatim}
Input: buildings = [[0,5,0]]
Output: []
Explanation:
- A building with height 0 does not contribute to the skyline.
\end{verbatim}

\textbf{Constraints:}

\begin{itemize}
    \item \(1 \leq \text{buildings.length} \leq 10^4\)
    \item \(0 \leq Li < Ri \leq 10^9\)
    \item \(0 \leq Hi \leq 10^4\)
\end{itemize}

\section*{Algorithmic Approach}

The \textbf{Sweep Line Algorithm} is an efficient method for solving the Skyline Problem. It involves processing events (building start and end points) in sorted order while maintaining a data structure (typically a max heap) to keep track of active buildings. Here's a step-by-step approach:

\subsection*{1. Event Representation}

Transform each building into two events:
\begin{itemize}
    \item **Start Event:** \((Li, -Hi)\) – Negative height indicates a building starts.
    \item **End Event:** \((Ri, Hi)\) – Positive height indicates a building ends.
\end{itemize}

Sorting the events ensures that start events are processed before end events at the same x-coordinate, and taller buildings are processed before shorter ones.

\subsection*{2. Sorting the Events}

Sort all events based on:
\begin{enumerate}
    \item **x-coordinate:** Ascending order.
    \item **Height:**
    \begin{itemize}
        \item For start events, taller buildings come first.
        \item For end events, shorter buildings come first.
    \end{itemize}
\end{enumerate}

\subsection*{3. Processing the Events}

Use a max heap to keep track of active building heights. Iterate through the sorted events:
\begin{enumerate}
    \item **Start Event:**
    \begin{itemize}
        \item Add the building's height to the heap.
    \end{itemize}
    
    \item **End Event:**
    \begin{itemize}
        \item Remove the building's height from the heap.
    \end{itemize}
    
    \item **Determine Current Max Height:**
    \begin{itemize}
        \item The current max height is the top of the heap.
    \end{itemize}
    
    \item **Update Skyline:**
    \begin{itemize}
        \item If the current max height differs from the previous max height, add a new key point \([x, current\_max\_height]\).
    \end{itemize}
\end{enumerate}

\subsection*{4. Finalizing the Skyline}

After processing all events, the accumulated key points represent the skyline.

\marginnote{The Sweep Line Algorithm efficiently handles dynamic changes in active buildings, ensuring accurate skyline construction.}

\section*{Complexities}

\begin{itemize}
    \item \textbf{Time Complexity:} \(O(n \log n)\), where \(n\) is the number of buildings. Sorting the events takes \(O(n \log n)\), and each heap operation takes \(O(\log n)\).
    
    \item \textbf{Space Complexity:} \(O(n)\), due to the storage of events and the heap.
\end{itemize}

\section*{Python Implementation}

\marginnote{Implementing the Sweep Line Algorithm with a max heap ensures an efficient and accurate solution.}

Below is the complete Python code implementing the \texttt{getSkyline} function:

\begin{fullwidth}
\begin{lstlisting}[language=Python]
from typing import List
import heapq

class Solution:
    def getSkyline(self, buildings: List[List[int]]) -> List[List[int]]:
        # Create a list of all events
        # For start events, use negative height to ensure they are processed before end events
        events = []
        for L, R, H in buildings:
            events.append((L, -H))
            events.append((R, H))
        
        # Sort the events
        # First by x-coordinate, then by height
        events.sort()
        
        # Max heap to keep track of active buildings
        heap = [0]  # Initialize with ground level
        heapq.heapify(heap)
        active_heights = {0: 1}  # Dictionary to count heights
        
        result = []
        prev_max = 0
        
        for x, h in events:
            if h < 0:
                # Start of a building, add height to heap and dictionary
                heapq.heappush(heap, h)
                active_heights[h] = active_heights.get(h, 0) + 1
            else:
                # End of a building, remove height from dictionary
                active_heights[h] -= 1
                if active_heights[h] == 0:
                    del active_heights[h]
            
            # Current max height
            while heap and active_heights.get(heap[0], 0) == 0:
                heapq.heappop(heap)
            current_max = -heap[0] if heap else 0
            
            # If the max height has changed, add to result
            if current_max != prev_max:
                result.append([x, current_max])
                prev_max = current_max
        
        return result

# Example usage:
solution = Solution()
print(solution.getSkyline([[2,9,10], [3,7,15], [5,12,12], [15,20,10], [19,24,8]]))
# Output: [[2,10], [3,15], [7,12], [12,0], [15,10], [20,8], [24,0]]

print(solution.getSkyline([[0,2,3], [2,5,3]]))
# Output: [[0,3], [5,0]]

print(solution.getSkyline([[1,3,3], [2,4,4], [5,6,1]]))
# Output: [[1,3], [2,4], [4,0], [5,1], [6,0]]

print(solution.getSkyline([[0,5,0]]))
# Output: []
\end{lstlisting}
\end{fullwidth}

This implementation efficiently constructs the skyline by processing all building events in sorted order and maintaining active building heights using a max heap. It ensures that only critical points where the skyline changes are recorded.

\section*{Explanation}

The \texttt{getSkyline} function constructs the skyline formed by a set of buildings by leveraging the Sweep Line Algorithm and a max heap to track active buildings. Here's a detailed breakdown of the implementation:

\subsection*{1. Event Representation}

\begin{itemize}
    \item Each building is transformed into two events:
    \begin{itemize}
        \item **Start Event:** \((Li, -Hi)\) – Negative height indicates the start of a building.
        \item **End Event:** \((Ri, Hi)\) – Positive height indicates the end of a building.
    \end{itemize}
\end{itemize}

\subsection*{2. Sorting the Events}

\begin{itemize}
    \item Events are sorted primarily by their x-coordinate in ascending order.
    \item For events with the same x-coordinate:
    \begin{itemize}
        \item Start events (with negative heights) are processed before end events.
        \item Taller buildings are processed before shorter ones.
    \end{itemize}
\end{itemize}

\subsection*{3. Processing the Events}

\begin{itemize}
    \item **Heap Initialization:**
    \begin{itemize}
        \item A max heap is initialized with a ground level height of 0.
        \item A dictionary \texttt{active\_heights} tracks the count of active building heights.
    \end{itemize}
    
    \item **Iterating Through Events:**
    \begin{enumerate}
        \item **Start Event:**
        \begin{itemize}
            \item Add the building's height to the heap.
            \item Increment the count of the height in \texttt{active\_heights}.
        \end{itemize}
        
        \item **End Event:**
        \begin{itemize}
            \item Decrement the count of the building's height in \texttt{active\_heights}.
            \item If the count reaches zero, remove the height from the dictionary.
        \end{itemize}
        
        \item **Determine Current Max Height:**
        \begin{itemize}
            \item Remove heights from the heap that are no longer active.
            \item The current max height is the top of the heap.
        \end{itemize}
        
        \item **Update Skyline:**
        \begin{itemize}
            \item If the current max height differs from the previous max height, add a new key point \([x, current\_max\_height]\).
        \end{itemize}
    \end{enumerate}
\end{itemize}

\subsection*{4. Finalizing the Skyline}

\begin{itemize}
    \item After processing all events, the \texttt{result} list contains the key points defining the skyline.
\end{itemize}

\subsection*{5. Example Walkthrough}

Consider the first example:
\begin{verbatim}
Input: buildings = [[2,9,10], [3,7,15], [5,12,12], [15,20,10], [19,24,8]]
Output: [[2,10], [3,15], [7,12], [12,0], [15,10], [20,8], [24,0]]
\end{verbatim}

\begin{enumerate}
    \item **Event Transformation:**
    \begin{itemize}
        \item \((2, -10)\), \((9, 10)\)
        \item \((3, -15)\), \((7, 15)\)
        \item \((5, -12)\), \((12, 12)\)
        \item \((15, -10)\), \((20, 10)\)
        \item \((19, -8)\), \((24, 8)\)
    \end{itemize}
    
    \item **Sorting Events:**
    \begin{itemize}
        \item Sorted order: \((2, -10)\), \((3, -15)\), \((5, -12)\), \((7, 15)\), \((9, 10)\), \((12, 12)\), \((15, -10)\), \((19, -8)\), \((20, 10)\), \((24, 8)\)
    \end{itemize}
    
    \item **Processing Events:**
    \begin{itemize}
        \item At each event, update the heap and determine if the skyline height changes.
    \end{itemize}
    
    \item **Result Construction:**
    \begin{itemize}
        \item The resulting skyline key points are accumulated as \([[2,10], [3,15], [7,12], [12,0], [15,10], [20,8], [24,0]]\).
    \end{itemize}
\end{enumerate}

Thus, the function correctly constructs the skyline based on the buildings' positions and heights.

\section*{Why This Approach}

The Sweep Line Algorithm combined with a max heap offers an optimal solution with \(O(n \log n)\) time complexity and efficient handling of overlapping buildings. By processing events in sorted order and maintaining active building heights, the algorithm ensures that all critical points in the skyline are accurately identified without redundant computations.

\section*{Alternative Approaches}

\subsection*{1. Divide and Conquer}

Divide the set of buildings into smaller subsets, compute the skyline for each subset, and then merge the skylines.

\begin{lstlisting}[language=Python]
class Solution:
    def getSkyline(self, buildings: List[List[int]]) -> List[List[int]]:
        def merge(left, right):
            h1, h2 = 0, 0
            i, j = 0, 0
            merged = []
            while i < len(left) and j < len(right):
                if left[i][0] < right[j][0]:
                    x, h1 = left[i]
                    i += 1
                elif left[i][0] > right[j][0]:
                    x, h2 = right[j]
                    j += 1
                else:
                    x, h1 = left[i]
                    _, h2 = right[j]
                    i += 1
                    j += 1
                max_h = max(h1, h2)
                if not merged or merged[-1][1] != max_h:
                    merged.append([x, max_h])
            merged.extend(left[i:])
            merged.extend(right[j:])
            return merged
        
        def divide(buildings):
            if not buildings:
                return []
            if len(buildings) == 1:
                L, R, H = buildings[0]
                return [[L, H], [R, 0]]
            mid = len(buildings) // 2
            left = divide(buildings[:mid])
            right = divide(buildings[mid:])
            return merge(left, right)
        
        return divide(buildings)
\end{lstlisting}

\textbf{Complexities:}
\begin{itemize}
    \item \textbf{Time Complexity:} \(O(n \log n)\)
    \item \textbf{Space Complexity:} \(O(n)\)
\end{itemize}

\subsection*{2. Using Segment Trees}

Implement a segment tree to manage and query overlapping building heights dynamically.

\textbf{Note}: This approach is more complex and is generally used for advanced scenarios with multiple dynamic queries.

\section*{Similar Problems to This One}

Several problems involve skyline-like constructions, spatial data analysis, and efficient event processing, utilizing similar algorithmic strategies:

\begin{itemize}
    \item \textbf{Merge Intervals}: Merge overlapping intervals in a list.
    \item \textbf{Largest Rectangle in Histogram}: Find the largest rectangular area in a histogram.
    \item \textbf{Interval Partitioning}: Assign intervals to resources without overlap.
    \item \textbf{Line Segment Intersection}: Detect intersections among line segments.
    \item \textbf{Closest Pair of Points}: Find the closest pair of points in a set.
    \item \textbf{Convex Hull}: Compute the convex hull of a set of points.
    \item \textbf{Point Inside Polygon}: Determine if a point lies inside a given polygon.
    \item \textbf{Range Searching}: Efficiently query geometric data within a specified range.
\end{itemize}

These problems reinforce concepts of event-driven processing, spatial reasoning, and efficient algorithm design in various contexts.

\section*{Things to Keep in Mind and Tricks}

When tackling the \textbf{Skyline Problem}, consider the following tips and best practices to enhance efficiency and correctness:

\begin{itemize}
    \item \textbf{Understand Sweep Line Technique}: Grasp how the sweep line algorithm processes events in sorted order to handle dynamic changes efficiently.
    \index{Sweep Line Technique}
    
    \item \textbf{Leverage Priority Queues (Heaps)}: Use max heaps to keep track of active buildings' heights, enabling quick access to the current maximum height.
    \index{Priority Queues}
    
    \item \textbf{Handle Start and End Events Differently}: Differentiate between building start and end events to accurately manage active heights.
    \index{Start and End Events}
    
    \item \textbf{Optimize Event Sorting}: Sort events primarily by x-coordinate and secondarily by height to ensure correct processing order.
    \index{Event Sorting}
    
    \item \textbf{Manage Active Heights Efficiently}: Use data structures that allow efficient insertion, deletion, and retrieval of maximum elements.
    \index{Active Heights Management}
    
    \item \textbf{Avoid Redundant Key Points}: Only record key points when the skyline height changes to minimize the output list.
    \index{Avoiding Redundant Key Points}
    
    \item \textbf{Implement Helper Functions}: Create helper functions for tasks like distance calculation, event handling, and heap management to enhance modularity.
    \index{Helper Functions}
    
    \item \textbf{Code Readability}: Maintain clear and readable code through meaningful variable names and structured logic.
    \index{Code Readability}
    
    \item \textbf{Test Extensively}: Implement a wide range of test cases, including overlapping, non-overlapping, and edge-touching buildings, to ensure robustness.
    \index{Extensive Testing}
    
    \item \textbf{Handle Degenerate Cases}: Manage cases where buildings have zero height or identical coordinates gracefully.
    \index{Degenerate Cases}
    
    \item \textbf{Understand Geometric Relationships}: Grasp how buildings overlap and influence the skyline to simplify the algorithm.
    \index{Geometric Relationships}
    
    \item \textbf{Use Appropriate Data Structures}: Utilize appropriate data structures like heaps, lists, and dictionaries to manage and process data efficiently.
    \index{Appropriate Data Structures}
    
    \item \textbf{Optimize for Large Inputs}: Design the algorithm to handle large numbers of buildings without significant performance degradation.
    \index{Optimizing for Large Inputs}
    
    \item \textbf{Implement Iterative Solutions Carefully}: Ensure that loop conditions are correctly defined to prevent infinite loops or incorrect terminations.
    \index{Iterative Solutions}
    
    \item \textbf{Consistent Naming Conventions}: Use consistent and descriptive naming conventions for variables and functions to improve code clarity.
    \index{Naming Conventions}
\end{itemize}

\section*{Corner and Special Cases to Test When Writing the Code}

When implementing the solution for the \textbf{Skyline Problem}, it is crucial to consider and rigorously test various edge cases to ensure robustness and correctness:

\begin{itemize}
    \item \textbf{No Overlapping Buildings}: All buildings are separate and do not overlap.
    \index{No Overlapping Buildings}
    
    \item \textbf{Fully Overlapping Buildings}: Multiple buildings completely overlap each other.
    \index{Fully Overlapping Buildings}
    
    \item \textbf{Buildings Touching at Edges}: Buildings share common edges without overlapping.
    \index{Buildings Touching at Edges}
    
    \item \textbf{Buildings Touching at Corners}: Buildings share common corners without overlapping.
    \index{Buildings Touching at Corners}
    
    \item \textbf{Single Building}: Only one building is present.
    \index{Single Building}
    
    \item \textbf{Multiple Buildings with Same Start or End}: Multiple buildings start or end at the same x-coordinate.
    \index{Same Start or End}
    
    \item \textbf{Buildings with Zero Height}: Buildings that have zero height should not affect the skyline.
    \index{Buildings with Zero Height}
    
    \item \textbf{Large Number of Buildings}: Test with a large number of buildings to ensure performance and scalability.
    \index{Large Number of Buildings}
    
    \item \textbf{Buildings with Negative Coordinates}: Buildings positioned in negative coordinate space.
    \index{Negative Coordinates}
    
    \item \textbf{Boundary Values}: Buildings at the minimum and maximum limits of the coordinate range.
    \index{Boundary Values}
    
    \item \textbf{Buildings with Identical Coordinates}: Multiple buildings with the same coordinates.
    \index{Identical Coordinates}
    
    \item \textbf{Sequential Buildings}: Buildings placed sequentially without gaps.
    \index{Sequential Buildings}
    
    \item \textbf{Overlapping and Non-Overlapping Mixed}: A mix of overlapping and non-overlapping buildings.
    \index{Overlapping and Non-Overlapping Mixed}
    
    \item \textbf{Buildings with Very Large Heights}: Buildings with heights at the upper limit of the constraints.
    \index{Very Large Heights}
    
    \item \textbf{Empty Input}: No buildings are provided.
    \index{Empty Input}
\end{itemize}

\section*{Implementation Considerations}

When implementing the \texttt{getSkyline} function, keep in mind the following considerations to ensure robustness and efficiency:

\begin{itemize}
    \item \textbf{Data Type Selection}: Use appropriate data types that can handle large input values and avoid overflow or precision issues.
    \index{Data Type Selection}
    
    \item \textbf{Optimizing Event Sorting}: Efficiently sort events based on x-coordinates and heights to ensure correct processing order.
    \index{Optimizing Event Sorting}
    
    \item \textbf{Handling Large Inputs}: Design the algorithm to handle up to \(10^4\) buildings efficiently without significant performance degradation.
    \index{Handling Large Inputs}
    
    \item \textbf{Using Efficient Data Structures}: Utilize heaps, lists, and dictionaries effectively to manage and process events and active heights.
    \index{Efficient Data Structures}
    
    \item \textbf{Avoiding Redundant Calculations}: Ensure that distance and overlap calculations are performed only when necessary to optimize performance.
    \index{Avoiding Redundant Calculations}
    
    \item \textbf{Code Readability and Documentation}: Maintain clear and readable code through meaningful variable names and comprehensive comments to facilitate understanding and maintenance.
    \index{Code Readability}
    
    \item \textbf{Edge Case Handling}: Implement checks for edge cases to prevent incorrect results or runtime errors.
    \index{Edge Case Handling}
    
    \item \textbf{Implementing Helper Functions}: Create helper functions for tasks like distance calculation, event handling, and heap management to enhance modularity.
    \index{Helper Functions}
    
    \item \textbf{Consistent Naming Conventions}: Use consistent and descriptive naming conventions for variables and functions to improve code clarity.
    \index{Naming Conventions}
    
    \item \textbf{Memory Management}: Ensure that the algorithm manages memory efficiently, especially when dealing with large datasets.
    \index{Memory Management}
    
    \item \textbf{Implementing Iterative Solutions Carefully}: Ensure that loop conditions are correctly defined to prevent infinite loops or incorrect terminations.
    \index{Iterative Solutions}
    
    \item \textbf{Avoiding Floating-Point Precision Issues}: Since the problem deals with integers, floating-point precision is not a concern, simplifying the implementation.
    \index{Floating-Point Precision}
    
    \item \textbf{Testing and Validation}: Develop a comprehensive suite of test cases that cover all possible scenarios, including edge cases, to validate the correctness and efficiency of the implementation.
    \index{Testing and Validation}
    
    \item \textbf{Performance Considerations}: Optimize the loop conditions and operations to ensure that the function runs efficiently, especially for large input numbers.
    \index{Performance Considerations}
\end{itemize}

\section*{Conclusion}

The \textbf{Skyline Problem} is a quintessential example of applying advanced algorithmic techniques and geometric reasoning to solve complex spatial challenges. By leveraging the Sweep Line Algorithm and maintaining active building heights using a max heap, the solution efficiently constructs the skyline with optimal time and space complexities. Understanding and implementing such sophisticated algorithms not only enhances problem-solving skills but also provides a foundation for tackling a wide array of Computational Geometry problems in various domains, including computer graphics, urban planning simulations, and geographic information systems.

\printindex

% % filename: rectangle_overlap.tex

\problemsection{Rectangle Overlap}
\label{chap:Rectangle_Overlap}
\marginnote{\href{https://leetcode.com/problems/rectangle-overlap/}{[LeetCode Link]}\index{LeetCode}}
\marginnote{\href{https://www.geeksforgeeks.org/check-if-two-rectangles-overlap/}{[GeeksForGeeks Link]}\index{GeeksForGeeks}}
\marginnote{\href{https://www.interviewbit.com/problems/rectangle-overlap/}{[InterviewBit Link]}\index{InterviewBit}}
\marginnote{\href{https://app.codesignal.com/challenges/rectangle-overlap}{[CodeSignal Link]}\index{CodeSignal}}
\marginnote{\href{https://www.codewars.com/kata/rectangle-overlap/train/python}{[Codewars Link]}\index{Codewars}}

The \textbf{Rectangle Overlap} problem is a fundamental challenge in Computational Geometry that involves determining whether two axis-aligned rectangles overlap. This problem tests one's ability to understand geometric properties, implement conditional logic, and optimize for efficient computation. Mastery of this problem is essential for applications in computer graphics, collision detection, and spatial data analysis.

\section*{Problem Statement}

Given two axis-aligned rectangles in a 2D plane, determine if they overlap. Each rectangle is defined by its bottom-left and top-right coordinates.

A rectangle is represented as a list of four integers \([x1, y1, x2, y2]\), where \((x1, y1)\) are the coordinates of the bottom-left corner, and \((x2, y2)\) are the coordinates of the top-right corner.

\textbf{Function signature in Python:}
\begin{lstlisting}[language=Python]
def isRectangleOverlap(rec1: List[int], rec2: List[int]) -> bool:
\end{lstlisting}

\section*{Examples}

\textbf{Example 1:}

\begin{verbatim}
Input: rec1 = [0,0,2,2], rec2 = [1,1,3,3]
Output: True
Explanation: The rectangles overlap in the area defined by [1,1,2,2].
\end{verbatim}

\textbf{Example 2:}

\begin{verbatim}
Input: rec1 = [0,0,1,1], rec2 = [1,0,2,1]
Output: False
Explanation: The rectangles touch at the edge but do not overlap.
\end{verbatim}

\textbf{Example 3:}

\begin{verbatim}
Input: rec1 = [0,0,1,1], rec2 = [2,2,3,3]
Output: False
Explanation: The rectangles are completely separate.
\end{verbatim}

\textbf{Example 4:}

\begin{verbatim}
Input: rec1 = [0,0,5,5], rec2 = [3,3,7,7]
Output: True
Explanation: The rectangles overlap in the area defined by [3,3,5,5].
\end{verbatim}

\textbf{Example 5:}

\begin{verbatim}
Input: rec1 = [0,0,0,0], rec2 = [0,0,0,0]
Output: False
Explanation: Both rectangles are degenerate points.
\end{verbatim}

\textbf{Constraints:}

\begin{itemize}
    \item All coordinates are integers in the range \([-10^9, 10^9]\).
    \item For each rectangle, \(x1 < x2\) and \(y1 < y2\).
\end{itemize}

LeetCode link: \href{https://leetcode.com/problems/rectangle-overlap/}{Rectangle Overlap}\index{LeetCode}

\section*{Algorithmic Approach}

To determine whether two axis-aligned rectangles overlap, we can use the following logical conditions:

1. **Non-Overlap Conditions:**
   - One rectangle is to the left of the other.
   - One rectangle is above the other.

2. **Overlap Condition:**
   - If neither of the non-overlap conditions is true, the rectangles must overlap.

\subsection*{Steps:}

1. **Extract Coordinates:**
   - For both rectangles, extract the bottom-left and top-right coordinates.

2. **Check Non-Overlap Conditions:**
   - If the right side of the first rectangle is less than or equal to the left side of the second rectangle, they do not overlap.
   - If the left side of the first rectangle is greater than or equal to the right side of the second rectangle, they do not overlap.
   - If the top side of the first rectangle is less than or equal to the bottom side of the second rectangle, they do not overlap.
   - If the bottom side of the first rectangle is greater than or equal to the top side of the second rectangle, they do not overlap.

3. **Determine Overlap:**
   - If none of the non-overlap conditions are met, the rectangles overlap.

\marginnote{This approach provides an efficient \(O(1)\) time complexity solution by leveraging simple geometric comparisons.}

\section*{Complexities}

\begin{itemize}
    \item \textbf{Time Complexity:} \(O(1)\). The algorithm performs a constant number of comparisons regardless of input size.
    
    \item \textbf{Space Complexity:} \(O(1)\). Only a fixed amount of extra space is used for variables.
\end{itemize}

\section*{Python Implementation}

\marginnote{Implementing the overlap check using coordinate comparisons ensures an optimal and straightforward solution.}

Below is the complete Python code implementing the \texttt{isRectangleOverlap} function:

\begin{fullwidth}
\begin{lstlisting}[language=Python]
from typing import List

class Solution:
    def isRectangleOverlap(self, rec1: List[int], rec2: List[int]) -> bool:
        # Extract coordinates
        left1, bottom1, right1, top1 = rec1
        left2, bottom2, right2, top2 = rec2
        
        # Check non-overlapping conditions
        if right1 <= left2 or right2 <= left1:
            return False
        if top1 <= bottom2 or top2 <= bottom1:
            return False
        
        # If none of the above, rectangles overlap
        return True

# Example usage:
solution = Solution()
print(solution.isRectangleOverlap([0,0,2,2], [1,1,3,3]))  # Output: True
print(solution.isRectangleOverlap([0,0,1,1], [1,0,2,1]))  # Output: False
print(solution.isRectangleOverlap([0,0,1,1], [2,2,3,3]))  # Output: False
print(solution.isRectangleOverlap([0,0,5,5], [3,3,7,7]))  # Output: True
print(solution.isRectangleOverlap([0,0,0,0], [0,0,0,0]))  # Output: False
\end{lstlisting}
\end{fullwidth}

This implementation efficiently checks for overlap by comparing the coordinates of the two rectangles. If any of the non-overlapping conditions are met, it returns \texttt{False}; otherwise, it returns \texttt{True}.

\section*{Explanation}

The \texttt{isRectangleOverlap} function determines whether two axis-aligned rectangles overlap by comparing their respective coordinates. Here's a detailed breakdown of the implementation:

\subsection*{1. Extract Coordinates}

\begin{itemize}
    \item For each rectangle, extract the left (\(x1\)), bottom (\(y1\)), right (\(x2\)), and top (\(y2\)) coordinates.
    \item This simplifies the comparison process by providing clear variables representing each side of the rectangles.
\end{itemize}

\subsection*{2. Check Non-Overlap Conditions}

\begin{itemize}
    \item **Horizontal Separation:**
    \begin{itemize}
        \item If the right side of the first rectangle (\(right1\)) is less than or equal to the left side of the second rectangle (\(left2\)), there is no horizontal overlap.
        \item Similarly, if the right side of the second rectangle (\(right2\)) is less than or equal to the left side of the first rectangle (\(left1\)), there is no horizontal overlap.
    \end{itemize}
    
    \item **Vertical Separation:**
    \begin{itemize}
        \item If the top side of the first rectangle (\(top1\)) is less than or equal to the bottom side of the second rectangle (\(bottom2\)), there is no vertical overlap.
        \item Similarly, if the top side of the second rectangle (\(top2\)) is less than or equal to the bottom side of the first rectangle (\(bottom1\)), there is no vertical overlap.
    \end{itemize}
    
    \item If any of these non-overlapping conditions are true, the rectangles do not overlap, and the function returns \texttt{False}.
\end{itemize}

\subsection*{3. Determine Overlap}

\begin{itemize}
    \item If none of the non-overlapping conditions are met, it implies that the rectangles overlap both horizontally and vertically.
    \item The function returns \texttt{True} in this case.
\end{itemize}

\subsection*{4. Example Walkthrough}

Consider the first example:
\begin{verbatim}
Input: rec1 = [0,0,2,2], rec2 = [1,1,3,3]
Output: True
\end{verbatim}

\begin{enumerate}
    \item Extract coordinates:
    \begin{itemize}
        \item rec1: left1 = 0, bottom1 = 0, right1 = 2, top1 = 2
        \item rec2: left2 = 1, bottom2 = 1, right2 = 3, top2 = 3
    \end{itemize}
    
    \item Check non-overlap conditions:
    \begin{itemize}
        \item \(right1 = 2\) is not less than or equal to \(left2 = 1\)
        \item \(right2 = 3\) is not less than or equal to \(left1 = 0\)
        \item \(top1 = 2\) is not less than or equal to \(bottom2 = 1\)
        \item \(top2 = 3\) is not less than or equal to \(bottom1 = 0\)
    \end{itemize}
    
    \item Since none of the non-overlapping conditions are met, the rectangles overlap.
\end{enumerate}

Thus, the function correctly returns \texttt{True}.

\section*{Why This Approach}

This approach is chosen for its simplicity and efficiency. By leveraging direct coordinate comparisons, the algorithm achieves constant time complexity without the need for complex data structures or iterative processes. It effectively handles all possible scenarios of rectangle positioning, ensuring accurate detection of overlaps.

\section*{Alternative Approaches}

\subsection*{1. Separating Axis Theorem (SAT)}

The Separating Axis Theorem is a more generalized method for detecting overlaps between convex shapes. While it is not necessary for axis-aligned rectangles, understanding SAT can be beneficial for more complex geometric problems.

\begin{lstlisting}[language=Python]
def isRectangleOverlap(rec1: List[int], rec2: List[int]) -> bool:
    # Using SAT for axis-aligned rectangles
    return not (rec1[2] <= rec2[0] or rec1[0] >= rec2[2] or
                rec1[3] <= rec2[1] or rec1[1] >= rec2[3])
\end{lstlisting}

\textbf{Note}: This implementation is functionally identical to the primary approach but leverages a more generalized geometric theorem.

\subsection*{2. Area-Based Approach}

Calculate the overlapping area between the two rectangles. If the overlapping area is positive, the rectangles overlap.

\begin{lstlisting}[language=Python]
def isRectangleOverlap(rec1: List[int], rec2: List[int]) -> bool:
    # Calculate overlap in x and y dimensions
    x_overlap = min(rec1[2], rec2[2]) - max(rec1[0], rec2[0])
    y_overlap = min(rec1[3], rec2[3]) - max(rec1[1], rec2[1])
    
    # Overlap exists if both overlaps are positive
    return x_overlap > 0 and y_overlap > 0
\end{lstlisting}

\textbf{Complexities:}
\begin{itemize}
    \item \textbf{Time Complexity:} \(O(1)\)
    \item \textbf{Space Complexity:} \(O(1)\)
\end{itemize}

\subsection*{3. Using Rectangles Intersection Function}

Utilize built-in or library functions that handle geometric intersections.

\begin{lstlisting}[language=Python]
from shapely.geometry import box

def isRectangleOverlap(rec1: List[int], rec2: List[int]) -> bool:
    rectangle1 = box(rec1[0], rec1[1], rec1[2], rec1[3])
    rectangle2 = box(rec2[0], rec2[1], rec2[2], rec2[3])
    return rectangle1.intersects(rectangle2) and not rectangle1.touches(rectangle2)
\end{lstlisting}

\textbf{Note}: This approach requires the \texttt{shapely} library and is more suitable for complex geometric operations.

\section*{Similar Problems to This One}

Several problems revolve around geometric overlap, intersection detection, and spatial reasoning, utilizing similar algorithmic strategies:

\begin{itemize}
    \item \textbf{Interval Overlap}: Determine if two intervals on a line overlap.
    \item \textbf{Circle Overlap}: Determine if two circles overlap based on their radii and centers.
    \item \textbf{Polygon Overlap}: Determine if two polygons overlap using algorithms like SAT.
    \item \textbf{Closest Pair of Points}: Find the closest pair of points in a set.
    \item \textbf{Convex Hull}: Compute the convex hull of a set of points.
    \item \textbf{Intersection of Lines}: Find the intersection point of two lines.
    \item \textbf{Point Inside Polygon}: Determine if a point lies inside a given polygon.
\end{itemize}

These problems reinforce the concepts of spatial reasoning, geometric property analysis, and efficient algorithm design in various contexts.

\section*{Things to Keep in Mind and Tricks}

When working with the \textbf{Rectangle Overlap} problem, consider the following tips and best practices to enhance efficiency and correctness:

\begin{itemize}
    \item \textbf{Understand Geometric Relationships}: Grasp the positional relationships between rectangles to simplify overlap detection.
    \index{Geometric Relationships}
    
    \item \textbf{Leverage Coordinate Comparisons}: Use direct comparisons of rectangle coordinates to determine spatial relationships.
    \index{Coordinate Comparisons}
    
    \item \textbf{Handle Edge Cases}: Consider cases where rectangles touch at edges or corners without overlapping.
    \index{Edge Cases}
    
    \item \textbf{Optimize for Efficiency}: Aim for a constant time \(O(1)\) solution by avoiding unnecessary computations or iterations.
    \index{Efficiency Optimization}
    
    \item \textbf{Avoid Floating-Point Precision Issues}: Since all coordinates are integers, floating-point precision is not a concern, simplifying the implementation.
    \index{Floating-Point Precision}
    
    \item \textbf{Use Helper Functions}: Create helper functions to encapsulate repetitive tasks, such as extracting coordinates or checking specific conditions.
    \index{Helper Functions}
    
    \item \textbf{Code Readability}: Maintain clear and readable code through meaningful variable names and structured logic.
    \index{Code Readability}
    
    \item \textbf{Test Extensively}: Implement a wide range of test cases, including overlapping, non-overlapping, and edge-touching rectangles, to ensure robustness.
    \index{Extensive Testing}
    
    \item \textbf{Understand Axis-Aligned Constraints}: Recognize that axis-aligned rectangles simplify overlap detection compared to rotated rectangles.
    \index{Axis-Aligned Constraints}
    
    \item \textbf{Simplify Logical Conditions}: Combine multiple conditions logically to streamline the overlap detection process.
    \index{Logical Conditions}
\end{itemize}

\section*{Corner and Special Cases to Test When Writing the Code}

When implementing the solution for the \textbf{Rectangle Overlap} problem, it is crucial to consider and rigorously test various edge cases to ensure robustness and correctness:

\begin{itemize}
    \item \textbf{No Overlap}: Rectangles are completely separate.
    \index{No Overlap}
    
    \item \textbf{Partial Overlap}: Rectangles overlap in one or more regions.
    \index{Partial Overlap}
    
    \item \textbf{Edge Touching}: Rectangles touch exactly at one edge without overlapping.
    \index{Edge Touching}
    
    \item \textbf{Corner Touching}: Rectangles touch exactly at one corner without overlapping.
    \index{Corner Touching}
    
    \item \textbf{One Rectangle Inside Another}: One rectangle is entirely within the other.
    \index{Rectangle Inside}
    
    \item \textbf{Identical Rectangles}: Both rectangles have the same coordinates.
    \index{Identical Rectangles}
    
    \item \textbf{Degenerate Rectangles}: Rectangles with zero area (e.g., \(x1 = x2\) or \(y1 = y2\)).
    \index{Degenerate Rectangles}
    
    \item \textbf{Large Coordinates}: Rectangles with very large coordinate values to test performance and integer handling.
    \index{Large Coordinates}
    
    \item \textbf{Negative Coordinates}: Rectangles positioned in negative coordinate space.
    \index{Negative Coordinates}
    
    \item \textbf{Mixed Overlapping Scenarios}: Combinations of the above cases to ensure comprehensive coverage.
    \index{Mixed Overlapping Scenarios}
    
    \item \textbf{Minimum and Maximum Bounds}: Rectangles at the minimum and maximum limits of the coordinate range.
    \index{Minimum and Maximum Bounds}
\end{itemize}

\section*{Implementation Considerations}

When implementing the \texttt{isRectangleOverlap} function, keep in mind the following considerations to ensure robustness and efficiency:

\begin{itemize}
    \item \textbf{Data Type Selection}: Use appropriate data types that can handle the range of input values without overflow or underflow.
    \index{Data Type Selection}
    
    \item \textbf{Optimizing Comparisons}: Structure logical conditions to short-circuit evaluations as soon as a non-overlapping condition is met.
    \index{Optimizing Comparisons}
    
    \item \textbf{Language-Specific Constraints}: Be aware of how the programming language handles integer division and comparisons.
    \index{Language-Specific Constraints}
    
    \item \textbf{Avoiding Redundant Calculations}: Ensure that each comparison contributes towards determining overlap without unnecessary repetitions.
    \index{Avoiding Redundant Calculations}
    
    \item \textbf{Code Readability and Documentation}: Maintain clear and readable code through meaningful variable names and comprehensive comments to facilitate understanding and maintenance.
    \index{Code Readability}
    
    \item \textbf{Edge Case Handling}: Implement checks for edge cases to prevent incorrect results or runtime errors.
    \index{Edge Case Handling}
    
    \item \textbf{Testing and Validation}: Develop a comprehensive suite of test cases that cover all possible scenarios, including edge cases, to validate the correctness and efficiency of the implementation.
    \index{Testing and Validation}
    
    \item \textbf{Scalability}: Design the algorithm to scale efficiently with increasing input sizes, maintaining performance and resource utilization.
    \index{Scalability}
    
    \item \textbf{Using Helper Functions}: Consider creating helper functions for repetitive tasks, such as extracting and comparing coordinates, to enhance modularity and reusability.
    \index{Helper Functions}
    
    \item \textbf{Consistent Naming Conventions}: Use consistent and descriptive naming conventions for variables to improve code clarity.
    \index{Naming Conventions}
    
    \item \textbf{Handling Floating-Point Coordinates}: Although the problem specifies integer coordinates, ensure that the implementation can handle floating-point numbers if needed in extended scenarios.
    \index{Floating-Point Coordinates}
    
    \item \textbf{Avoiding Floating-Point Precision Issues}: Since all coordinates are integers, floating-point precision is not a concern, simplifying the implementation.
    \index{Floating-Point Precision}
    
    \item \textbf{Implementing Unit Tests}: Develop unit tests for each logical condition to ensure that all scenarios are correctly handled.
    \index{Unit Tests}
    
    \item \textbf{Error Handling}: Incorporate error handling to manage invalid inputs gracefully.
    \index{Error Handling}
\end{itemize}

\section*{Conclusion}

The \textbf{Rectangle Overlap} problem exemplifies the application of fundamental geometric principles and conditional logic to solve spatial challenges efficiently. By leveraging simple coordinate comparisons, the algorithm achieves optimal time and space complexities, making it highly suitable for real-time applications such as collision detection in gaming, layout planning in graphics, and spatial data analysis. Understanding and implementing such techniques not only enhances problem-solving skills but also provides a foundation for tackling more complex Computational Geometry problems involving varied geometric shapes and interactions.

\printindex

% \input{sections/rectangle_overlap}
% \input{sections/rectangle_area}
% \input{sections/k_closest_points_to_origin}
% \input{sections/the_skyline_problem}
% % filename: rectangle_area.tex

\problemsection{Rectangle Area}
\label{chap:Rectangle_Area}
\marginnote{\href{https://leetcode.com/problems/rectangle-area/}{[LeetCode Link]}\index{LeetCode}}
\marginnote{\href{https://www.geeksforgeeks.org/find-area-two-overlapping-rectangles/}{[GeeksForGeeks Link]}\index{GeeksForGeeks}}
\marginnote{\href{https://www.interviewbit.com/problems/rectangle-area/}{[InterviewBit Link]}\index{InterviewBit}}
\marginnote{\href{https://app.codesignal.com/challenges/rectangle-area}{[CodeSignal Link]}\index{CodeSignal}}
\marginnote{\href{https://www.codewars.com/kata/rectangle-area/train/python}{[Codewars Link]}\index{Codewars}}

The \textbf{Rectangle Area} problem is a classic Computational Geometry challenge that involves calculating the total area covered by two axis-aligned rectangles in a 2D plane. This problem tests one's ability to perform geometric calculations, handle overlapping scenarios, and implement efficient algorithms. Mastery of this problem is essential for applications in computer graphics, spatial analysis, and computational modeling.

\section*{Problem Statement}

Given two axis-aligned rectangles in a 2D plane, compute the total area covered by the two rectangles. The area covered by the overlapping region should be counted only once.

Each rectangle is represented as a list of four integers \([x1, y1, x2, y2]\), where \((x1, y1)\) are the coordinates of the bottom-left corner, and \((x2, y2)\) are the coordinates of the top-right corner.

\textbf{Function signature in Python:}
\begin{lstlisting}[language=Python]
def computeArea(A: List[int], B: List[int]) -> int:
\end{lstlisting}

\section*{Examples}

\textbf{Example 1:}

\begin{verbatim}
Input: A = [-3,0,3,4], B = [0,-1,9,2]
Output: 45
Explanation:
Area of A = (3 - (-3)) * (4 - 0) = 6 * 4 = 24
Area of B = (9 - 0) * (2 - (-1)) = 9 * 3 = 27
Overlapping Area = (3 - 0) * (2 - 0) = 3 * 2 = 6
Total Area = 24 + 27 - 6 = 45
\end{verbatim}

\textbf{Example 2:}

\begin{verbatim}
Input: A = [0,0,0,0], B = [0,0,0,0]
Output: 0
Explanation:
Both rectangles are degenerate points with zero area.
\end{verbatim}

\textbf{Example 3:}

\begin{verbatim}
Input: A = [0,0,2,2], B = [1,1,3,3]
Output: 7
Explanation:
Area of A = 4
Area of B = 4
Overlapping Area = 1
Total Area = 4 + 4 - 1 = 7
\end{verbatim}

\textbf{Example 4:}

\begin{verbatim}
Input: A = [0,0,1,1], B = [1,0,2,1]
Output: 2
Explanation:
Rectangles touch at the edge but do not overlap.
Area of A = 1
Area of B = 1
Overlapping Area = 0
Total Area = 1 + 1 = 2
\end{verbatim}

\textbf{Constraints:}

\begin{itemize}
    \item All coordinates are integers in the range \([-10^9, 10^9]\).
    \item For each rectangle, \(x1 < x2\) and \(y1 < y2\).
\end{itemize}

LeetCode link: \href{https://leetcode.com/problems/rectangle-area/}{Rectangle Area}\index{LeetCode}

\section*{Algorithmic Approach}

To compute the total area covered by two axis-aligned rectangles, we can follow these steps:

1. **Calculate Individual Areas:**
   - Compute the area of the first rectangle.
   - Compute the area of the second rectangle.

2. **Determine Overlapping Area:**
   - Calculate the coordinates of the overlapping rectangle, if any.
   - If the rectangles overlap, compute the area of the overlapping region.

3. **Compute Total Area:**
   - Sum the individual areas and subtract the overlapping area to avoid double-counting.

\marginnote{This approach ensures accurate area calculation by handling overlapping regions appropriately.}

\section*{Complexities}

\begin{itemize}
    \item \textbf{Time Complexity:} \(O(1)\). The algorithm performs a constant number of calculations.
    
    \item \textbf{Space Complexity:} \(O(1)\). Only a fixed amount of extra space is used for variables.
\end{itemize}

\section*{Python Implementation}

\marginnote{Implementing the area calculation with overlap consideration ensures an accurate and efficient solution.}

Below is the complete Python code implementing the \texttt{computeArea} function:

\begin{fullwidth}
\begin{lstlisting}[language=Python]
from typing import List

class Solution:
    def computeArea(self, A: List[int], B: List[int]) -> int:
        # Calculate area of rectangle A
        areaA = (A[2] - A[0]) * (A[3] - A[1])
        
        # Calculate area of rectangle B
        areaB = (B[2] - B[0]) * (B[3] - B[1])
        
        # Determine overlap coordinates
        overlap_x1 = max(A[0], B[0])
        overlap_y1 = max(A[1], B[1])
        overlap_x2 = min(A[2], B[2])
        overlap_y2 = min(A[3], B[3])
        
        # Calculate overlapping area
        overlap_width = overlap_x2 - overlap_x1
        overlap_height = overlap_y2 - overlap_y1
        overlap_area = 0
        if overlap_width > 0 and overlap_height > 0:
            overlap_area = overlap_width * overlap_height
        
        # Total area is sum of individual areas minus overlapping area
        total_area = areaA + areaB - overlap_area
        return total_area

# Example usage:
solution = Solution()
print(solution.computeArea([-3,0,3,4], [0,-1,9,2]))  # Output: 45
print(solution.computeArea([0,0,0,0], [0,0,0,0]))    # Output: 0
print(solution.computeArea([0,0,2,2], [1,1,3,3]))    # Output: 7
print(solution.computeArea([0,0,1,1], [1,0,2,1]))    # Output: 2
\end{lstlisting}
\end{fullwidth}

This implementation accurately computes the total area covered by two rectangles by accounting for any overlapping regions. It ensures that the overlapping area is not double-counted.

\section*{Explanation}

The \texttt{computeArea} function calculates the combined area of two axis-aligned rectangles by following these steps:

\subsection*{1. Calculate Individual Areas}

\begin{itemize}
    \item **Rectangle A:**
    \begin{itemize}
        \item Width: \(A[2] - A[0]\)
        \item Height: \(A[3] - A[1]\)
        \item Area: Width \(\times\) Height
    \end{itemize}
    
    \item **Rectangle B:**
    \begin{itemize}
        \item Width: \(B[2] - B[0]\)
        \item Height: \(B[3] - B[1]\)
        \item Area: Width \(\times\) Height
    \end{itemize}
\end{itemize}

\subsection*{2. Determine Overlapping Area}

\begin{itemize}
    \item **Overlap Coordinates:**
    \begin{itemize}
        \item Left (x-coordinate): \(\text{max}(A[0], B[0])\)
        \item Bottom (y-coordinate): \(\text{max}(A[1], B[1])\)
        \item Right (x-coordinate): \(\text{min}(A[2], B[2])\)
        \item Top (y-coordinate): \(\text{min}(A[3], B[3])\)
    \end{itemize}
    
    \item **Overlap Dimensions:**
    \begin{itemize}
        \item Width: \(\text{overlap\_x2} - \text{overlap\_x1}\)
        \item Height: \(\text{overlap\_y2} - \text{overlap\_y1}\)
    \end{itemize}
    
    \item **Overlap Area:**
    \begin{itemize}
        \item If both width and height are positive, the rectangles overlap, and the overlapping area is their product.
        \item Otherwise, there is no overlap, and the overlapping area is zero.
    \end{itemize}
\end{itemize}

\subsection*{3. Compute Total Area}

\begin{itemize}
    \item Total Area = Area of Rectangle A + Area of Rectangle B - Overlapping Area
\end{itemize}

\subsection*{4. Example Walkthrough}

Consider the first example:
\begin{verbatim}
Input: A = [-3,0,3,4], B = [0,-1,9,2]
Output: 45
\end{verbatim}

\begin{enumerate}
    \item **Calculate Areas:**
    \begin{itemize}
        \item Area of A = (3 - (-3)) * (4 - 0) = 6 * 4 = 24
        \item Area of B = (9 - 0) * (2 - (-1)) = 9 * 3 = 27
    \end{itemize}
    
    \item **Determine Overlap:**
    \begin{itemize}
        \item overlap\_x1 = max(-3, 0) = 0
        \item overlap\_y1 = max(0, -1) = 0
        \item overlap\_x2 = min(3, 9) = 3
        \item overlap\_y2 = min(4, 2) = 2
        \item overlap\_width = 3 - 0 = 3
        \item overlap\_height = 2 - 0 = 2
        \item overlap\_area = 3 * 2 = 6
    \end{itemize}
    
    \item **Compute Total Area:**
    \begin{itemize}
        \item Total Area = 24 + 27 - 6 = 45
    \end{itemize}
\end{enumerate}

Thus, the function correctly returns \texttt{45}.

\section*{Why This Approach}

This approach is chosen for its straightforwardness and optimal efficiency. By directly calculating the individual areas and intelligently handling the overlapping region, the algorithm ensures accurate results without unnecessary computations. Its constant time complexity makes it highly efficient, even for large coordinate values.

\section*{Alternative Approaches}

\subsection*{1. Using Intersection Dimensions}

Instead of separately calculating areas, directly compute the dimensions of the overlapping region and subtract it from the sum of individual areas.

\begin{lstlisting}[language=Python]
def computeArea(A: List[int], B: List[int]) -> int:
    # Sum of individual areas
    area = (A[2] - A[0]) * (A[3] - A[1]) + (B[2] - B[0]) * (B[3] - B[1])
    
    # Overlapping area
    overlap_width = min(A[2], B[2]) - max(A[0], B[0])
    overlap_height = min(A[3], B[3]) - max(A[1], B[1])
    
    if overlap_width > 0 and overlap_height > 0:
        area -= overlap_width * overlap_height
    
    return area
\end{lstlisting}

\subsection*{2. Using Geometry Libraries}

Leverage computational geometry libraries to handle area calculations and overlapping detections.

\begin{lstlisting}[language=Python]
from shapely.geometry import box

def computeArea(A: List[int], B: List[int]) -> int:
    rect1 = box(A[0], A[1], A[2], A[3])
    rect2 = box(B[0], B[1], B[2], B[3])
    intersection = rect1.intersection(rect2)
    return int(rect1.area + rect2.area - intersection.area)
\end{lstlisting}

\textbf{Note}: This approach requires the \texttt{shapely} library and is more suitable for complex geometric operations.

\section*{Similar Problems to This One}

Several problems involve calculating areas, handling geometric overlaps, and spatial reasoning, utilizing similar algorithmic strategies:

\begin{itemize}
    \item \textbf{Rectangle Overlap}: Determine if two rectangles overlap.
    \item \textbf{Circle Area Overlap}: Calculate the overlapping area between two circles.
    \item \textbf{Polygon Area}: Compute the area of a given polygon.
    \item \textbf{Union of Rectangles}: Calculate the total area covered by multiple rectangles, accounting for overlaps.
    \item \textbf{Intersection of Lines}: Find the intersection point of two lines.
    \item \textbf{Closest Pair of Points}: Find the closest pair of points in a set.
    \item \textbf{Convex Hull}: Compute the convex hull of a set of points.
    \item \textbf{Point Inside Polygon}: Determine if a point lies inside a given polygon.
\end{itemize}

These problems reinforce concepts of geometric calculations, area computations, and efficient algorithm design in various contexts.

\section*{Things to Keep in Mind and Tricks}

When tackling the \textbf{Rectangle Area} problem, consider the following tips and best practices to enhance efficiency and correctness:

\begin{itemize}
    \item \textbf{Understand Geometric Relationships}: Grasp the positional relationships between rectangles to simplify area calculations.
    \index{Geometric Relationships}
    
    \item \textbf{Leverage Coordinate Comparisons}: Use direct comparisons of rectangle coordinates to determine overlapping regions.
    \index{Coordinate Comparisons}
    
    \item \textbf{Handle Overlapping Scenarios}: Accurately calculate the overlapping area to avoid double-counting.
    \index{Overlapping Scenarios}
    
    \item \textbf{Optimize for Efficiency}: Aim for a constant time \(O(1)\) solution by avoiding unnecessary computations or iterations.
    \index{Efficiency Optimization}
    
    \item \textbf{Avoid Floating-Point Precision Issues}: Since all coordinates are integers, floating-point precision is not a concern, simplifying the implementation.
    \index{Floating-Point Precision}
    
    \item \textbf{Use Helper Functions}: Create helper functions to encapsulate repetitive tasks, such as calculating overlap dimensions or areas.
    \index{Helper Functions}
    
    \item \textbf{Code Readability}: Maintain clear and readable code through meaningful variable names and structured logic.
    \index{Code Readability}
    
    \item \textbf{Test Extensively}: Implement a wide range of test cases, including overlapping, non-overlapping, and edge-touching rectangles, to ensure robustness.
    \index{Extensive Testing}
    
    \item \textbf{Understand Axis-Aligned Constraints}: Recognize that axis-aligned rectangles simplify area calculations compared to rotated rectangles.
    \index{Axis-Aligned Constraints}
    
    \item \textbf{Simplify Logical Conditions}: Combine multiple conditions logically to streamline the area calculation process.
    \index{Logical Conditions}
    
    \item \textbf{Use Absolute Values}: When calculating differences, ensure that the dimensions are positive by using absolute values or proper ordering.
    \index{Absolute Values}
    
    \item \textbf{Consider Edge Cases}: Handle cases where rectangles have zero area or touch at edges/corners without overlapping.
    \index{Edge Cases}
\end{itemize}

\section*{Corner and Special Cases to Test When Writing the Code}

When implementing the solution for the \textbf{Rectangle Area} problem, it is crucial to consider and rigorously test various edge cases to ensure robustness and correctness:

\begin{itemize}
    \item \textbf{No Overlap}: Rectangles are completely separate.
    \index{No Overlap}
    
    \item \textbf{Partial Overlap}: Rectangles overlap in one or more regions.
    \index{Partial Overlap}
    
    \item \textbf{Edge Touching}: Rectangles touch exactly at one edge without overlapping.
    \index{Edge Touching}
    
    \item \textbf{Corner Touching}: Rectangles touch exactly at one corner without overlapping.
    \index{Corner Touching}
    
    \item \textbf{One Rectangle Inside Another}: One rectangle is entirely within the other.
    \index{Rectangle Inside}
    
    \item \textbf{Identical Rectangles}: Both rectangles have the same coordinates.
    \index{Identical Rectangles}
    
    \item \textbf{Degenerate Rectangles}: Rectangles with zero area (e.g., \(x1 = x2\) or \(y1 = y2\)).
    \index{Degenerate Rectangles}
    
    \item \textbf{Large Coordinates}: Rectangles with very large coordinate values to test performance and integer handling.
    \index{Large Coordinates}
    
    \item \textbf{Negative Coordinates}: Rectangles positioned in negative coordinate space.
    \index{Negative Coordinates}
    
    \item \textbf{Mixed Overlapping Scenarios}: Combinations of the above cases to ensure comprehensive coverage.
    \index{Mixed Overlapping Scenarios}
    
    \item \textbf{Minimum and Maximum Bounds}: Rectangles at the minimum and maximum limits of the coordinate range.
    \index{Minimum and Maximum Bounds}
    
    \item \textbf{Sequential Rectangles}: Multiple rectangles placed sequentially without overlapping.
    \index{Sequential Rectangles}
    
    \item \textbf{Multiple Overlaps}: Scenarios where more than two rectangles overlap in different regions.
    \index{Multiple Overlaps}
\end{itemize}

\section*{Implementation Considerations}

When implementing the \texttt{computeArea} function, keep in mind the following considerations to ensure robustness and efficiency:

\begin{itemize}
    \item \textbf{Data Type Selection}: Use appropriate data types that can handle large input values without overflow or underflow.
    \index{Data Type Selection}
    
    \item \textbf{Optimizing Comparisons}: Structure logical conditions to efficiently determine overlap dimensions.
    \index{Optimizing Comparisons}
    
    \item \textbf{Handling Large Inputs}: Design the algorithm to efficiently handle large input sizes without significant performance degradation.
    \index{Handling Large Inputs}
    
    \item \textbf{Language-Specific Constraints}: Be aware of how the programming language handles large integers and arithmetic operations.
    \index{Language-Specific Constraints}
    
    \item \textbf{Avoiding Redundant Calculations}: Ensure that each calculation contributes towards determining the final area without unnecessary repetitions.
    \index{Avoiding Redundant Calculations}
    
    \item \textbf{Code Readability and Documentation}: Maintain clear and readable code through meaningful variable names and comprehensive comments to facilitate understanding and maintenance.
    \index{Code Readability}
    
    \item \textbf{Edge Case Handling}: Implement checks for edge cases to prevent incorrect results or runtime errors.
    \index{Edge Case Handling}
    
    \item \textbf{Testing and Validation}: Develop a comprehensive suite of test cases that cover all possible scenarios, including edge cases, to validate the correctness and efficiency of the implementation.
    \index{Testing and Validation}
    
    \item \textbf{Scalability}: Design the algorithm to scale efficiently with increasing input sizes, maintaining performance and resource utilization.
    \index{Scalability}
    
    \item \textbf{Using Helper Functions}: Consider creating helper functions for repetitive tasks, such as calculating overlap dimensions, to enhance modularity and reusability.
    \index{Helper Functions}
    
    \item \textbf{Consistent Naming Conventions}: Use consistent and descriptive naming conventions for variables to improve code clarity.
    \index{Naming Conventions}
    
    \item \textbf{Implementing Unit Tests}: Develop unit tests for each logical condition to ensure that all scenarios are correctly handled.
    \index{Unit Tests}
    
    \item \textbf{Error Handling}: Incorporate error handling to manage invalid inputs gracefully.
    \index{Error Handling}
\end{itemize}

\section*{Conclusion}

The \textbf{Rectangle Area} problem showcases the application of fundamental geometric principles and efficient algorithm design to compute spatial properties accurately. By systematically calculating individual areas and intelligently handling overlapping regions, the algorithm ensures precise results without redundant computations. Understanding and implementing such techniques not only enhances problem-solving skills but also provides a foundation for tackling more complex Computational Geometry challenges involving multiple geometric entities and intricate spatial relationships.

\printindex

% \input{sections/rectangle_overlap}
% \input{sections/rectangle_area}
% \input{sections/k_closest_points_to_origin}
% \input{sections/the_skyline_problem}
% % filename: k_closest_points_to_origin.tex

\problemsection{K Closest Points to Origin}
\label{chap:K_Closest_Points_to_Origin}
\marginnote{\href{https://leetcode.com/problems/k-closest-points-to-origin/}{[LeetCode Link]}\index{LeetCode}}
\marginnote{\href{https://www.geeksforgeeks.org/find-k-closest-points-origin/}{[GeeksForGeeks Link]}\index{GeeksForGeeks}}
\marginnote{\href{https://www.interviewbit.com/problems/k-closest-points/}{[InterviewBit Link]}\index{InterviewBit}}
\marginnote{\href{https://app.codesignal.com/challenges/k-closest-points-to-origin}{[CodeSignal Link]}\index{CodeSignal}}
\marginnote{\href{https://www.codewars.com/kata/k-closest-points-to-origin/train/python}{[Codewars Link]}\index{Codewars}}

The \textbf{K Closest Points to Origin} problem is a popular algorithmic challenge in Computational Geometry that involves identifying the \(k\) points closest to the origin in a 2D plane. This problem tests one's ability to apply efficient sorting and selection algorithms, understand distance computations, and optimize for performance. Mastery of this problem is essential for applications in spatial data analysis, nearest neighbor searches, and clustering algorithms.

\section*{Problem Statement}

Given an array of points where each point is represented as \([x, y]\) in the 2D plane, and an integer \(k\), return the \(k\) closest points to the origin \((0, 0)\).

The distance between two points \((x_1, y_1)\) and \((x_2, y_2)\) is the Euclidean distance \(\sqrt{(x_1 - x_2)^2 + (y_1 - y_2)^2}\). The origin is \((0, 0)\).

\textbf{Function signature in Python:}
\begin{lstlisting}[language=Python]
def kClosest(points: List[List[int]], K: int) -> List[List[int]]:
\end{lstlisting}

\section*{Examples}

\textbf{Example 1:}

\begin{verbatim}
Input: points = [[1,3],[-2,2]], K = 1
Output: [[-2,2]]
Explanation: 
The distance between (1, 3) and the origin is sqrt(10).
The distance between (-2, 2) and the origin is sqrt(8).
Since sqrt(8) < sqrt(10), (-2, 2) is closer to the origin.
\end{verbatim}

\textbf{Example 2:}

\begin{verbatim}
Input: points = [[3,3],[5,-1],[-2,4]], K = 2
Output: [[3,3],[-2,4]]
Explanation: 
The distances are sqrt(18), sqrt(26), and sqrt(20) respectively.
The two closest points are [3,3] and [-2,4].
\end{verbatim}

\textbf{Example 3:}

\begin{verbatim}
Input: points = [[0,1],[1,0]], K = 2
Output: [[0,1],[1,0]]
Explanation: 
Both points are equally close to the origin.
\end{verbatim}

\textbf{Example 4:}

\begin{verbatim}
Input: points = [[1,0],[0,1]], K = 1
Output: [[1,0]]
Explanation: 
Both points are equally close; returning any one is acceptable.
\end{verbatim}

\textbf{Constraints:}

\begin{itemize}
    \item \(1 \leq K \leq \text{points.length} \leq 10^4\)
    \item \(-10^4 < x_i, y_i < 10^4\)
\end{itemize}

LeetCode link: \href{https://leetcode.com/problems/k-closest-points-to-origin/}{K Closest Points to Origin}\index{LeetCode}

\section*{Algorithmic Approach}

To identify the \(k\) closest points to the origin, several algorithmic strategies can be employed. The most efficient methods aim to reduce the time complexity by avoiding the need to sort the entire list of points.

\subsection*{1. Sorting Based on Distance}

Calculate the Euclidean distance of each point from the origin and sort the points based on these distances. Select the first \(k\) points from the sorted list.

\begin{enumerate}
    \item Compute the distance for each point using the formula \(distance = x^2 + y^2\).
    \item Sort the points based on the computed distances.
    \item Return the first \(k\) points from the sorted list.
\end{enumerate}

\subsection*{2. Max Heap (Priority Queue)}

Use a max heap to maintain the \(k\) closest points. Iterate through each point, add it to the heap, and if the heap size exceeds \(k\), remove the farthest point.

\begin{enumerate}
    \item Initialize a max heap.
    \item For each point, compute its distance and add it to the heap.
    \item If the heap size exceeds \(k\), remove the point with the largest distance.
    \item After processing all points, the heap contains the \(k\) closest points.
\end{enumerate}

\subsection*{3. QuickSelect (Quick Sort Partitioning)}

Utilize the QuickSelect algorithm to find the \(k\) closest points without fully sorting the list.

\begin{enumerate}
    \item Choose a pivot point and partition the list based on distances relative to the pivot.
    \item Recursively apply QuickSelect to the partition containing the \(k\) closest points.
    \item Once the \(k\) closest points are identified, return them.
\end{enumerate}

\marginnote{QuickSelect offers an average time complexity of \(O(n)\), making it highly efficient for large datasets.}

\section*{Complexities}

\begin{itemize}
    \item \textbf{Sorting Based on Distance:}
    \begin{itemize}
        \item \textbf{Time Complexity:} \(O(n \log n)\)
        \item \textbf{Space Complexity:} \(O(n)\)
    \end{itemize}
    
    \item \textbf{Max Heap (Priority Queue):}
    \begin{itemize}
        \item \textbf{Time Complexity:} \(O(n \log k)\)
        \item \textbf{Space Complexity:} \(O(k)\)
    \end{itemize}
    
    \item \textbf{QuickSelect (Quick Sort Partitioning):}
    \begin{itemize}
        \item \textbf{Time Complexity:} Average case \(O(n)\), worst case \(O(n^2)\)
        \item \textbf{Space Complexity:} \(O(1)\) (in-place)
    \end{itemize}
\end{itemize}

\section*{Python Implementation}

\marginnote{Implementing QuickSelect provides an optimal average-case solution with linear time complexity.}

Below is the complete Python code implementing the \texttt{kClosest} function using the QuickSelect approach:

\begin{fullwidth}
\begin{lstlisting}[language=Python]
from typing import List
import random

class Solution:
    def kClosest(self, points: List[List[int]], K: int) -> List[List[int]]:
        def quickselect(left, right, K_smallest):
            if left == right:
                return
            
            # Select a random pivot_index
            pivot_index = random.randint(left, right)
            
            # Partition the array
            pivot_index = partition(left, right, pivot_index)
            
            # The pivot is in its final sorted position
            if K_smallest == pivot_index:
                return
            elif K_smallest < pivot_index:
                quickselect(left, pivot_index - 1, K_smallest)
            else:
                quickselect(pivot_index + 1, right, K_smallest)
        
        def partition(left, right, pivot_index):
            pivot_distance = distance(points[pivot_index])
            # Move pivot to end
            points[pivot_index], points[right] = points[right], points[pivot_index]
            store_index = left
            for i in range(left, right):
                if distance(points[i]) < pivot_distance:
                    points[store_index], points[i] = points[i], points[store_index]
                    store_index += 1
            # Move pivot to its final place
            points[right], points[store_index] = points[store_index], points[right]
            return store_index
        
        def distance(point):
            return point[0] ** 2 + point[1] ** 2
        
        n = len(points)
        quickselect(0, n - 1, K)
        return points[:K]

# Example usage:
solution = Solution()
print(solution.kClosest([[1,3],[-2,2]], 1))            # Output: [[-2,2]]
print(solution.kClosest([[3,3],[5,-1],[-2,4]], 2))     # Output: [[3,3],[-2,4]]
print(solution.kClosest([[0,1],[1,0]], 2))             # Output: [[0,1],[1,0]]
print(solution.kClosest([[1,0],[0,1]], 1))             # Output: [[1,0]] or [[0,1]]
\end{lstlisting}
\end{fullwidth}

This implementation uses the QuickSelect algorithm to efficiently find the \(k\) closest points to the origin without fully sorting the entire list. It ensures optimal performance even with large datasets.

\section*{Explanation}

The \texttt{kClosest} function identifies the \(k\) closest points to the origin using the QuickSelect algorithm. Here's a detailed breakdown of the implementation:

\subsection*{1. Distance Calculation}

\begin{itemize}
    \item The Euclidean distance is calculated as \(distance = x^2 + y^2\). Since we only need relative distances for comparison, the square root is omitted for efficiency.
\end{itemize}

\subsection*{2. QuickSelect Algorithm}

\begin{itemize}
    \item **Pivot Selection:**
    \begin{itemize}
        \item A random pivot is chosen to enhance the average-case performance.
    \end{itemize}
    
    \item **Partitioning:**
    \begin{itemize}
        \item The array is partitioned such that points with distances less than the pivot are moved to the left, and others to the right.
        \item The pivot is placed in its correct sorted position.
    \end{itemize}
    
    \item **Recursive Selection:**
    \begin{itemize}
        \item If the pivot's position matches \(K\), the selection is complete.
        \item Otherwise, recursively apply QuickSelect to the relevant partition.
    \end{itemize}
\end{itemize}

\subsection*{3. Final Selection}

\begin{itemize}
    \item After partitioning, the first \(K\) points in the list are the \(k\) closest points to the origin.
\end{itemize}

\subsection*{4. Example Walkthrough}

Consider the first example:
\begin{verbatim}
Input: points = [[1,3],[-2,2]], K = 1
Output: [[-2,2]]
\end{verbatim}

\begin{enumerate}
    \item **Calculate Distances:**
    \begin{itemize}
        \item [1,3] : \(1^2 + 3^2 = 10\)
        \item [-2,2] : \((-2)^2 + 2^2 = 8\)
    \end{itemize}
    
    \item **QuickSelect Process:**
    \begin{itemize}
        \item Choose a pivot, say [1,3] with distance 10.
        \item Compare and rearrange:
        \begin{itemize}
            \item [-2,2] has a smaller distance (8) and is moved to the left.
        \end{itemize}
        \item After partitioning, the list becomes [[-2,2], [1,3]].
        \item Since \(K = 1\), return the first point: [[-2,2]].
    \end{itemize}
\end{enumerate}

Thus, the function correctly identifies \([-2,2]\) as the closest point to the origin.

\section*{Why This Approach}

The QuickSelect algorithm is chosen for its average-case linear time complexity \(O(n)\), making it highly efficient for large datasets compared to sorting-based methods with \(O(n \log n)\) time complexity. By avoiding the need to sort the entire list, QuickSelect provides an optimal solution for finding the \(k\) closest points.

\section*{Alternative Approaches}

\subsection*{1. Sorting Based on Distance}

Sort all points based on their distances from the origin and select the first \(k\) points.

\begin{lstlisting}[language=Python]
class Solution:
    def kClosest(self, points: List[List[int]], K: int) -> List[List[int]]:
        points.sort(key=lambda P: P[0]**2 + P[1]**2)
        return points[:K]
\end{lstlisting}

\textbf{Complexities:}
\begin{itemize}
    \item \textbf{Time Complexity:} \(O(n \log n)\)
    \item \textbf{Space Complexity:} \(O(1)\)
\end{itemize}

\subsection*{2. Max Heap (Priority Queue)}

Use a max heap to maintain the \(k\) closest points.

\begin{lstlisting}[language=Python]
import heapq

class Solution:
    def kClosest(self, points: List[List[int]], K: int) -> List[List[int]]:
        heap = []
        for (x, y) in points:
            dist = -(x**2 + y**2)  # Max heap using negative distances
            heapq.heappush(heap, (dist, [x, y]))
            if len(heap) > K:
                heapq.heappop(heap)
        return [item[1] for item in heap]
\end{lstlisting}

\textbf{Complexities:}
\begin{itemize}
    \item \textbf{Time Complexity:} \(O(n \log k)\)
    \item \textbf{Space Complexity:} \(O(k)\)
\end{itemize}

\subsection*{3. Using Built-In Functions}

Leverage built-in functions for distance calculation and selection.

\begin{lstlisting}[language=Python]
import math

class Solution:
    def kClosest(self, points: List[List[int]], K: int) -> List[List[int]]:
        points.sort(key=lambda P: math.sqrt(P[0]**2 + P[1]**2))
        return points[:K]
\end{lstlisting}

\textbf{Note}: This method is similar to the sorting approach but uses the actual Euclidean distance.

\section*{Similar Problems to This One}

Several problems involve nearest neighbor searches, spatial data analysis, and efficient selection algorithms, utilizing similar algorithmic strategies:

\begin{itemize}
    \item \textbf{Closest Pair of Points}: Find the closest pair of points in a set.
    \item \textbf{Top K Frequent Elements}: Identify the most frequent elements in a dataset.
    \item \textbf{Kth Largest Element in an Array}: Find the \(k\)-th largest element in an unsorted array.
    \item \textbf{Sliding Window Maximum}: Find the maximum in each sliding window of size \(k\) over an array.
    \item \textbf{Merge K Sorted Lists}: Merge multiple sorted lists into a single sorted list.
    \item \textbf{Find Median from Data Stream}: Continuously find the median of a stream of numbers.
    \item \textbf{Top K Closest Stars}: Find the \(k\) closest stars to Earth based on their distances.
\end{itemize}

These problems reinforce concepts of efficient selection, heap usage, and distance computations in various contexts.

\section*{Things to Keep in Mind and Tricks}

When solving the \textbf{K Closest Points to Origin} problem, consider the following tips and best practices to enhance efficiency and correctness:

\begin{itemize}
    \item \textbf{Understand Distance Calculations}: Grasp the Euclidean distance formula and recognize that the square root can be omitted for comparison purposes.
    \index{Distance Calculations}
    
    \item \textbf{Leverage Efficient Algorithms}: Use QuickSelect or heap-based methods to optimize time complexity, especially for large datasets.
    \index{Efficient Algorithms}
    
    \item \textbf{Handle Ties Appropriately}: Decide how to handle points with identical distances when \(k\) is less than the number of such points.
    \index{Handling Ties}
    
    \item \textbf{Optimize Space Usage}: Choose algorithms that minimize additional space, such as in-place QuickSelect.
    \index{Space Optimization}
    
    \item \textbf{Use Appropriate Data Structures}: Utilize heaps, lists, and helper functions effectively to manage and process data.
    \index{Data Structures}
    
    \item \textbf{Implement Helper Functions}: Create helper functions for distance calculation and partitioning to enhance code modularity.
    \index{Helper Functions}
    
    \item \textbf{Code Readability}: Maintain clear and readable code through meaningful variable names and structured logic.
    \index{Code Readability}
    
    \item \textbf{Test Extensively}: Implement a wide range of test cases, including edge cases like multiple points with the same distance, to ensure robustness.
    \index{Extensive Testing}
    
    \item \textbf{Understand Algorithm Trade-offs}: Recognize the trade-offs between different approaches in terms of time and space complexities.
    \index{Algorithm Trade-offs}
    
    \item \textbf{Use Built-In Sorting Functions}: When using sorting-based approaches, leverage built-in functions for efficiency and simplicity.
    \index{Built-In Sorting}
    
    \item \textbf{Avoid Redundant Calculations}: Ensure that distance calculations are performed only when necessary to optimize performance.
    \index{Avoiding Redundant Calculations}
    
    \item \textbf{Language-Specific Features}: Utilize language-specific features or libraries that can simplify implementation, such as heapq in Python.
    \index{Language-Specific Features}
\end{itemize}

\section*{Corner and Special Cases to Test When Writing the Code}

When implementing the solution for the \textbf{K Closest Points to Origin} problem, it is crucial to consider and rigorously test various edge cases to ensure robustness and correctness:

\begin{itemize}
    \item \textbf{Multiple Points with Same Distance}: Ensure that the algorithm handles multiple points having the same distance from the origin.
    \index{Same Distance Points}
    
    \item \textbf{Points at Origin}: Include points that are exactly at the origin \((0,0)\).
    \index{Points at Origin}
    
    \item \textbf{Negative Coordinates}: Ensure that the algorithm correctly computes distances for points with negative \(x\) or \(y\) coordinates.
    \index{Negative Coordinates}
    
    \item \textbf{Large Coordinates}: Test with points having very large or very small coordinate values to verify integer handling.
    \index{Large Coordinates}
    
    \item \textbf{K Equals Number of Points}: When \(K\) is equal to the number of points, the algorithm should return all points.
    \index{K Equals Number of Points}
    
    \item \textbf{K is One}: Test with \(K = 1\) to ensure the closest point is correctly identified.
    \index{K is One}
    
    \item \textbf{All Points Same}: All points have the same coordinates.
    \index{All Points Same}
    
    \item \textbf{K is Zero}: Although \(K\) is defined to be at least 1, ensure that the algorithm gracefully handles \(K = 0\) if allowed.
    \index{K is Zero}
    
    \item \textbf{Single Point}: Only one point is provided, and \(K = 1\).
    \index{Single Point}
    
    \item \textbf{Mixed Coordinates}: Points with a mix of positive and negative coordinates.
    \index{Mixed Coordinates}
    
    \item \textbf{Points with Zero Distance}: Multiple points at the origin.
    \index{Zero Distance Points}
    
    \item \textbf{Sparse and Dense Points}: Densely packed points and sparsely distributed points.
    \index{Sparse and Dense Points}
    
    \item \textbf{Duplicate Points}: Multiple identical points in the input list.
    \index{Duplicate Points}
    
    \item \textbf{K Greater Than Number of Unique Points}: Ensure that the algorithm handles cases where \(K\) exceeds the number of unique points if applicable.
    \index{K Greater Than Unique Points}
\end{itemize}

\section*{Implementation Considerations}

When implementing the \texttt{kClosest} function, keep in mind the following considerations to ensure robustness and efficiency:

\begin{itemize}
    \item \textbf{Data Type Selection}: Use appropriate data types that can handle large input values without overflow or precision loss.
    \index{Data Type Selection}
    
    \item \textbf{Optimizing Distance Calculations}: Avoid calculating the square root since it is unnecessary for comparison purposes.
    \index{Optimizing Distance Calculations}
    
    \item \textbf{Choosing the Right Algorithm}: Select an algorithm based on the size of the input and the value of \(K\) to optimize time and space complexities.
    \index{Choosing the Right Algorithm}
    
    \item \textbf{Language-Specific Libraries}: Utilize language-specific libraries and functions (e.g., \texttt{heapq} in Python) to simplify implementation and enhance performance.
    \index{Language-Specific Libraries}
    
    \item \textbf{Avoiding Redundant Calculations}: Ensure that each point's distance is calculated only once to optimize performance.
    \index{Avoiding Redundant Calculations}
    
    \item \textbf{Implementing Helper Functions}: Create helper functions for tasks like distance calculation and partitioning to enhance modularity and readability.
    \index{Helper Functions}
    
    \item \textbf{Edge Case Handling}: Implement checks for edge cases to prevent incorrect results or runtime errors.
    \index{Edge Case Handling}
    
    \item \textbf{Testing and Validation}: Develop a comprehensive suite of test cases that cover all possible scenarios, including edge cases, to validate the correctness and efficiency of the implementation.
    \index{Testing and Validation}
    
    \item \textbf{Scalability}: Design the algorithm to scale efficiently with increasing input sizes, maintaining performance and resource utilization.
    \index{Scalability}
    
    \item \textbf{Consistent Naming Conventions}: Use consistent and descriptive naming conventions for variables and functions to improve code clarity.
    \index{Naming Conventions}
    
    \item \textbf{Memory Management}: Ensure that the algorithm manages memory efficiently, especially when dealing with large datasets.
    \index{Memory Management}
    
    \item \textbf{Avoiding Stack Overflow}: If implementing recursive approaches, be mindful of recursion limits and potential stack overflow issues.
    \index{Avoiding Stack Overflow}
    
    \item \textbf{Implementing Iterative Solutions}: Prefer iterative solutions when recursion may lead to increased space complexity or stack overflow.
    \index{Implementing Iterative Solutions}
\end{itemize}

\section*{Conclusion}

The \textbf{K Closest Points to Origin} problem exemplifies the application of efficient selection algorithms and geometric computations to solve spatial challenges effectively. By leveraging QuickSelect or heap-based methods, the algorithm achieves optimal time and space complexities, making it highly suitable for large datasets. Understanding and implementing such techniques not only enhances problem-solving skills but also provides a foundation for tackling more advanced Computational Geometry problems involving nearest neighbor searches, clustering, and spatial data analysis.

\printindex

% \input{sections/rectangle_overlap}
% \input{sections/rectangle_area}
% \input{sections/k_closest_points_to_origin}
% \input{sections/the_skyline_problem}
% % filename: the_skyline_problem.tex

\problemsection{The Skyline Problem}
\label{chap:The_Skyline_Problem}
\marginnote{\href{https://leetcode.com/problems/the-skyline-problem/}{[LeetCode Link]}\index{LeetCode}}
\marginnote{\href{https://www.geeksforgeeks.org/the-skyline-problem/}{[GeeksForGeeks Link]}\index{GeeksForGeeks}}
\marginnote{\href{https://www.interviewbit.com/problems/the-skyline-problem/}{[InterviewBit Link]}\index{InterviewBit}}
\marginnote{\href{https://app.codesignal.com/challenges/the-skyline-problem}{[CodeSignal Link]}\index{CodeSignal}}
\marginnote{\href{https://www.codewars.com/kata/the-skyline-problem/train/python}{[Codewars Link]}\index{Codewars}}

The \textbf{Skyline Problem} is a complex Computational Geometry challenge that involves computing the skyline formed by a collection of buildings in a 2D cityscape. Each building is represented by its left and right x-coordinates and its height. The skyline is defined by a list of "key points" where the height changes. This problem tests one's ability to handle large datasets, implement efficient sweep line algorithms, and manage event-driven processing. Mastery of this problem is essential for applications in computer graphics, urban planning simulations, and geographic information systems (GIS).

\section*{Problem Statement}

You are given a list of buildings in a cityscape. Each building is represented as a triplet \([Li, Ri, Hi]\), where \(Li\) and \(Ri\) are the x-coordinates of the left and right edges of the building, respectively, and \(Hi\) is the height of the building.

The skyline should be represented as a list of key points \([x, y]\) in sorted order by \(x\)-coordinate, where \(y\) is the height of the skyline at that point. The skyline should only include critical points where the height changes.

\textbf{Function signature in Python:}
\begin{lstlisting}[language=Python]
def getSkyline(buildings: List[List[int]]) -> List[List[int]]:
\end{lstlisting}

\section*{Examples}

\textbf{Example 1:}

\begin{verbatim}
Input: buildings = [[2,9,10], [3,7,15], [5,12,12], [15,20,10], [19,24,8]]
Output: [[2,10], [3,15], [7,12], [12,0], [15,10], [20,8], [24,0]]
Explanation:
- At x=2, the first building starts, height=10.
- At x=3, the second building starts, height=15.
- At x=7, the second building ends, the third building is still ongoing, height=12.
- At x=12, the third building ends, height drops to 0.
- At x=15, the fourth building starts, height=10.
- At x=20, the fourth building ends, the fifth building is still ongoing, height=8.
- At x=24, the fifth building ends, height drops to 0.
\end{verbatim}

\textbf{Example 2:}

\begin{verbatim}
Input: buildings = [[0,2,3], [2,5,3]]
Output: [[0,3], [5,0]]
Explanation:
- The two buildings are contiguous and have the same height, so the skyline drops to 0 at x=5.
\end{verbatim}

\textbf{Example 3:}

\begin{verbatim}
Input: buildings = [[1,3,3], [2,4,4], [5,6,1]]
Output: [[1,3], [2,4], [4,0], [5,1], [6,0]]
Explanation:
- At x=1, first building starts, height=3.
- At x=2, second building starts, height=4.
- At x=4, second building ends, height drops to 0.
- At x=5, third building starts, height=1.
- At x=6, third building ends, height drops to 0.
\end{verbatim}

\textbf{Example 4:}

\begin{verbatim}
Input: buildings = [[0,5,0]]
Output: []
Explanation:
- A building with height 0 does not contribute to the skyline.
\end{verbatim}

\textbf{Constraints:}

\begin{itemize}
    \item \(1 \leq \text{buildings.length} \leq 10^4\)
    \item \(0 \leq Li < Ri \leq 10^9\)
    \item \(0 \leq Hi \leq 10^4\)
\end{itemize}

\section*{Algorithmic Approach}

The \textbf{Sweep Line Algorithm} is an efficient method for solving the Skyline Problem. It involves processing events (building start and end points) in sorted order while maintaining a data structure (typically a max heap) to keep track of active buildings. Here's a step-by-step approach:

\subsection*{1. Event Representation}

Transform each building into two events:
\begin{itemize}
    \item **Start Event:** \((Li, -Hi)\) – Negative height indicates a building starts.
    \item **End Event:** \((Ri, Hi)\) – Positive height indicates a building ends.
\end{itemize}

Sorting the events ensures that start events are processed before end events at the same x-coordinate, and taller buildings are processed before shorter ones.

\subsection*{2. Sorting the Events}

Sort all events based on:
\begin{enumerate}
    \item **x-coordinate:** Ascending order.
    \item **Height:**
    \begin{itemize}
        \item For start events, taller buildings come first.
        \item For end events, shorter buildings come first.
    \end{itemize}
\end{enumerate}

\subsection*{3. Processing the Events}

Use a max heap to keep track of active building heights. Iterate through the sorted events:
\begin{enumerate}
    \item **Start Event:**
    \begin{itemize}
        \item Add the building's height to the heap.
    \end{itemize}
    
    \item **End Event:**
    \begin{itemize}
        \item Remove the building's height from the heap.
    \end{itemize}
    
    \item **Determine Current Max Height:**
    \begin{itemize}
        \item The current max height is the top of the heap.
    \end{itemize}
    
    \item **Update Skyline:**
    \begin{itemize}
        \item If the current max height differs from the previous max height, add a new key point \([x, current\_max\_height]\).
    \end{itemize}
\end{enumerate}

\subsection*{4. Finalizing the Skyline}

After processing all events, the accumulated key points represent the skyline.

\marginnote{The Sweep Line Algorithm efficiently handles dynamic changes in active buildings, ensuring accurate skyline construction.}

\section*{Complexities}

\begin{itemize}
    \item \textbf{Time Complexity:} \(O(n \log n)\), where \(n\) is the number of buildings. Sorting the events takes \(O(n \log n)\), and each heap operation takes \(O(\log n)\).
    
    \item \textbf{Space Complexity:} \(O(n)\), due to the storage of events and the heap.
\end{itemize}

\section*{Python Implementation}

\marginnote{Implementing the Sweep Line Algorithm with a max heap ensures an efficient and accurate solution.}

Below is the complete Python code implementing the \texttt{getSkyline} function:

\begin{fullwidth}
\begin{lstlisting}[language=Python]
from typing import List
import heapq

class Solution:
    def getSkyline(self, buildings: List[List[int]]) -> List[List[int]]:
        # Create a list of all events
        # For start events, use negative height to ensure they are processed before end events
        events = []
        for L, R, H in buildings:
            events.append((L, -H))
            events.append((R, H))
        
        # Sort the events
        # First by x-coordinate, then by height
        events.sort()
        
        # Max heap to keep track of active buildings
        heap = [0]  # Initialize with ground level
        heapq.heapify(heap)
        active_heights = {0: 1}  # Dictionary to count heights
        
        result = []
        prev_max = 0
        
        for x, h in events:
            if h < 0:
                # Start of a building, add height to heap and dictionary
                heapq.heappush(heap, h)
                active_heights[h] = active_heights.get(h, 0) + 1
            else:
                # End of a building, remove height from dictionary
                active_heights[h] -= 1
                if active_heights[h] == 0:
                    del active_heights[h]
            
            # Current max height
            while heap and active_heights.get(heap[0], 0) == 0:
                heapq.heappop(heap)
            current_max = -heap[0] if heap else 0
            
            # If the max height has changed, add to result
            if current_max != prev_max:
                result.append([x, current_max])
                prev_max = current_max
        
        return result

# Example usage:
solution = Solution()
print(solution.getSkyline([[2,9,10], [3,7,15], [5,12,12], [15,20,10], [19,24,8]]))
# Output: [[2,10], [3,15], [7,12], [12,0], [15,10], [20,8], [24,0]]

print(solution.getSkyline([[0,2,3], [2,5,3]]))
# Output: [[0,3], [5,0]]

print(solution.getSkyline([[1,3,3], [2,4,4], [5,6,1]]))
# Output: [[1,3], [2,4], [4,0], [5,1], [6,0]]

print(solution.getSkyline([[0,5,0]]))
# Output: []
\end{lstlisting}
\end{fullwidth}

This implementation efficiently constructs the skyline by processing all building events in sorted order and maintaining active building heights using a max heap. It ensures that only critical points where the skyline changes are recorded.

\section*{Explanation}

The \texttt{getSkyline} function constructs the skyline formed by a set of buildings by leveraging the Sweep Line Algorithm and a max heap to track active buildings. Here's a detailed breakdown of the implementation:

\subsection*{1. Event Representation}

\begin{itemize}
    \item Each building is transformed into two events:
    \begin{itemize}
        \item **Start Event:** \((Li, -Hi)\) – Negative height indicates the start of a building.
        \item **End Event:** \((Ri, Hi)\) – Positive height indicates the end of a building.
    \end{itemize}
\end{itemize}

\subsection*{2. Sorting the Events}

\begin{itemize}
    \item Events are sorted primarily by their x-coordinate in ascending order.
    \item For events with the same x-coordinate:
    \begin{itemize}
        \item Start events (with negative heights) are processed before end events.
        \item Taller buildings are processed before shorter ones.
    \end{itemize}
\end{itemize}

\subsection*{3. Processing the Events}

\begin{itemize}
    \item **Heap Initialization:**
    \begin{itemize}
        \item A max heap is initialized with a ground level height of 0.
        \item A dictionary \texttt{active\_heights} tracks the count of active building heights.
    \end{itemize}
    
    \item **Iterating Through Events:**
    \begin{enumerate}
        \item **Start Event:**
        \begin{itemize}
            \item Add the building's height to the heap.
            \item Increment the count of the height in \texttt{active\_heights}.
        \end{itemize}
        
        \item **End Event:**
        \begin{itemize}
            \item Decrement the count of the building's height in \texttt{active\_heights}.
            \item If the count reaches zero, remove the height from the dictionary.
        \end{itemize}
        
        \item **Determine Current Max Height:**
        \begin{itemize}
            \item Remove heights from the heap that are no longer active.
            \item The current max height is the top of the heap.
        \end{itemize}
        
        \item **Update Skyline:**
        \begin{itemize}
            \item If the current max height differs from the previous max height, add a new key point \([x, current\_max\_height]\).
        \end{itemize}
    \end{enumerate}
\end{itemize}

\subsection*{4. Finalizing the Skyline}

\begin{itemize}
    \item After processing all events, the \texttt{result} list contains the key points defining the skyline.
\end{itemize}

\subsection*{5. Example Walkthrough}

Consider the first example:
\begin{verbatim}
Input: buildings = [[2,9,10], [3,7,15], [5,12,12], [15,20,10], [19,24,8]]
Output: [[2,10], [3,15], [7,12], [12,0], [15,10], [20,8], [24,0]]
\end{verbatim}

\begin{enumerate}
    \item **Event Transformation:**
    \begin{itemize}
        \item \((2, -10)\), \((9, 10)\)
        \item \((3, -15)\), \((7, 15)\)
        \item \((5, -12)\), \((12, 12)\)
        \item \((15, -10)\), \((20, 10)\)
        \item \((19, -8)\), \((24, 8)\)
    \end{itemize}
    
    \item **Sorting Events:**
    \begin{itemize}
        \item Sorted order: \((2, -10)\), \((3, -15)\), \((5, -12)\), \((7, 15)\), \((9, 10)\), \((12, 12)\), \((15, -10)\), \((19, -8)\), \((20, 10)\), \((24, 8)\)
    \end{itemize}
    
    \item **Processing Events:**
    \begin{itemize}
        \item At each event, update the heap and determine if the skyline height changes.
    \end{itemize}
    
    \item **Result Construction:**
    \begin{itemize}
        \item The resulting skyline key points are accumulated as \([[2,10], [3,15], [7,12], [12,0], [15,10], [20,8], [24,0]]\).
    \end{itemize}
\end{enumerate}

Thus, the function correctly constructs the skyline based on the buildings' positions and heights.

\section*{Why This Approach}

The Sweep Line Algorithm combined with a max heap offers an optimal solution with \(O(n \log n)\) time complexity and efficient handling of overlapping buildings. By processing events in sorted order and maintaining active building heights, the algorithm ensures that all critical points in the skyline are accurately identified without redundant computations.

\section*{Alternative Approaches}

\subsection*{1. Divide and Conquer}

Divide the set of buildings into smaller subsets, compute the skyline for each subset, and then merge the skylines.

\begin{lstlisting}[language=Python]
class Solution:
    def getSkyline(self, buildings: List[List[int]]) -> List[List[int]]:
        def merge(left, right):
            h1, h2 = 0, 0
            i, j = 0, 0
            merged = []
            while i < len(left) and j < len(right):
                if left[i][0] < right[j][0]:
                    x, h1 = left[i]
                    i += 1
                elif left[i][0] > right[j][0]:
                    x, h2 = right[j]
                    j += 1
                else:
                    x, h1 = left[i]
                    _, h2 = right[j]
                    i += 1
                    j += 1
                max_h = max(h1, h2)
                if not merged or merged[-1][1] != max_h:
                    merged.append([x, max_h])
            merged.extend(left[i:])
            merged.extend(right[j:])
            return merged
        
        def divide(buildings):
            if not buildings:
                return []
            if len(buildings) == 1:
                L, R, H = buildings[0]
                return [[L, H], [R, 0]]
            mid = len(buildings) // 2
            left = divide(buildings[:mid])
            right = divide(buildings[mid:])
            return merge(left, right)
        
        return divide(buildings)
\end{lstlisting}

\textbf{Complexities:}
\begin{itemize}
    \item \textbf{Time Complexity:} \(O(n \log n)\)
    \item \textbf{Space Complexity:} \(O(n)\)
\end{itemize}

\subsection*{2. Using Segment Trees}

Implement a segment tree to manage and query overlapping building heights dynamically.

\textbf{Note}: This approach is more complex and is generally used for advanced scenarios with multiple dynamic queries.

\section*{Similar Problems to This One}

Several problems involve skyline-like constructions, spatial data analysis, and efficient event processing, utilizing similar algorithmic strategies:

\begin{itemize}
    \item \textbf{Merge Intervals}: Merge overlapping intervals in a list.
    \item \textbf{Largest Rectangle in Histogram}: Find the largest rectangular area in a histogram.
    \item \textbf{Interval Partitioning}: Assign intervals to resources without overlap.
    \item \textbf{Line Segment Intersection}: Detect intersections among line segments.
    \item \textbf{Closest Pair of Points}: Find the closest pair of points in a set.
    \item \textbf{Convex Hull}: Compute the convex hull of a set of points.
    \item \textbf{Point Inside Polygon}: Determine if a point lies inside a given polygon.
    \item \textbf{Range Searching}: Efficiently query geometric data within a specified range.
\end{itemize}

These problems reinforce concepts of event-driven processing, spatial reasoning, and efficient algorithm design in various contexts.

\section*{Things to Keep in Mind and Tricks}

When tackling the \textbf{Skyline Problem}, consider the following tips and best practices to enhance efficiency and correctness:

\begin{itemize}
    \item \textbf{Understand Sweep Line Technique}: Grasp how the sweep line algorithm processes events in sorted order to handle dynamic changes efficiently.
    \index{Sweep Line Technique}
    
    \item \textbf{Leverage Priority Queues (Heaps)}: Use max heaps to keep track of active buildings' heights, enabling quick access to the current maximum height.
    \index{Priority Queues}
    
    \item \textbf{Handle Start and End Events Differently}: Differentiate between building start and end events to accurately manage active heights.
    \index{Start and End Events}
    
    \item \textbf{Optimize Event Sorting}: Sort events primarily by x-coordinate and secondarily by height to ensure correct processing order.
    \index{Event Sorting}
    
    \item \textbf{Manage Active Heights Efficiently}: Use data structures that allow efficient insertion, deletion, and retrieval of maximum elements.
    \index{Active Heights Management}
    
    \item \textbf{Avoid Redundant Key Points}: Only record key points when the skyline height changes to minimize the output list.
    \index{Avoiding Redundant Key Points}
    
    \item \textbf{Implement Helper Functions}: Create helper functions for tasks like distance calculation, event handling, and heap management to enhance modularity.
    \index{Helper Functions}
    
    \item \textbf{Code Readability}: Maintain clear and readable code through meaningful variable names and structured logic.
    \index{Code Readability}
    
    \item \textbf{Test Extensively}: Implement a wide range of test cases, including overlapping, non-overlapping, and edge-touching buildings, to ensure robustness.
    \index{Extensive Testing}
    
    \item \textbf{Handle Degenerate Cases}: Manage cases where buildings have zero height or identical coordinates gracefully.
    \index{Degenerate Cases}
    
    \item \textbf{Understand Geometric Relationships}: Grasp how buildings overlap and influence the skyline to simplify the algorithm.
    \index{Geometric Relationships}
    
    \item \textbf{Use Appropriate Data Structures}: Utilize appropriate data structures like heaps, lists, and dictionaries to manage and process data efficiently.
    \index{Appropriate Data Structures}
    
    \item \textbf{Optimize for Large Inputs}: Design the algorithm to handle large numbers of buildings without significant performance degradation.
    \index{Optimizing for Large Inputs}
    
    \item \textbf{Implement Iterative Solutions Carefully}: Ensure that loop conditions are correctly defined to prevent infinite loops or incorrect terminations.
    \index{Iterative Solutions}
    
    \item \textbf{Consistent Naming Conventions}: Use consistent and descriptive naming conventions for variables and functions to improve code clarity.
    \index{Naming Conventions}
\end{itemize}

\section*{Corner and Special Cases to Test When Writing the Code}

When implementing the solution for the \textbf{Skyline Problem}, it is crucial to consider and rigorously test various edge cases to ensure robustness and correctness:

\begin{itemize}
    \item \textbf{No Overlapping Buildings}: All buildings are separate and do not overlap.
    \index{No Overlapping Buildings}
    
    \item \textbf{Fully Overlapping Buildings}: Multiple buildings completely overlap each other.
    \index{Fully Overlapping Buildings}
    
    \item \textbf{Buildings Touching at Edges}: Buildings share common edges without overlapping.
    \index{Buildings Touching at Edges}
    
    \item \textbf{Buildings Touching at Corners}: Buildings share common corners without overlapping.
    \index{Buildings Touching at Corners}
    
    \item \textbf{Single Building}: Only one building is present.
    \index{Single Building}
    
    \item \textbf{Multiple Buildings with Same Start or End}: Multiple buildings start or end at the same x-coordinate.
    \index{Same Start or End}
    
    \item \textbf{Buildings with Zero Height}: Buildings that have zero height should not affect the skyline.
    \index{Buildings with Zero Height}
    
    \item \textbf{Large Number of Buildings}: Test with a large number of buildings to ensure performance and scalability.
    \index{Large Number of Buildings}
    
    \item \textbf{Buildings with Negative Coordinates}: Buildings positioned in negative coordinate space.
    \index{Negative Coordinates}
    
    \item \textbf{Boundary Values}: Buildings at the minimum and maximum limits of the coordinate range.
    \index{Boundary Values}
    
    \item \textbf{Buildings with Identical Coordinates}: Multiple buildings with the same coordinates.
    \index{Identical Coordinates}
    
    \item \textbf{Sequential Buildings}: Buildings placed sequentially without gaps.
    \index{Sequential Buildings}
    
    \item \textbf{Overlapping and Non-Overlapping Mixed}: A mix of overlapping and non-overlapping buildings.
    \index{Overlapping and Non-Overlapping Mixed}
    
    \item \textbf{Buildings with Very Large Heights}: Buildings with heights at the upper limit of the constraints.
    \index{Very Large Heights}
    
    \item \textbf{Empty Input}: No buildings are provided.
    \index{Empty Input}
\end{itemize}

\section*{Implementation Considerations}

When implementing the \texttt{getSkyline} function, keep in mind the following considerations to ensure robustness and efficiency:

\begin{itemize}
    \item \textbf{Data Type Selection}: Use appropriate data types that can handle large input values and avoid overflow or precision issues.
    \index{Data Type Selection}
    
    \item \textbf{Optimizing Event Sorting}: Efficiently sort events based on x-coordinates and heights to ensure correct processing order.
    \index{Optimizing Event Sorting}
    
    \item \textbf{Handling Large Inputs}: Design the algorithm to handle up to \(10^4\) buildings efficiently without significant performance degradation.
    \index{Handling Large Inputs}
    
    \item \textbf{Using Efficient Data Structures}: Utilize heaps, lists, and dictionaries effectively to manage and process events and active heights.
    \index{Efficient Data Structures}
    
    \item \textbf{Avoiding Redundant Calculations}: Ensure that distance and overlap calculations are performed only when necessary to optimize performance.
    \index{Avoiding Redundant Calculations}
    
    \item \textbf{Code Readability and Documentation}: Maintain clear and readable code through meaningful variable names and comprehensive comments to facilitate understanding and maintenance.
    \index{Code Readability}
    
    \item \textbf{Edge Case Handling}: Implement checks for edge cases to prevent incorrect results or runtime errors.
    \index{Edge Case Handling}
    
    \item \textbf{Implementing Helper Functions}: Create helper functions for tasks like distance calculation, event handling, and heap management to enhance modularity.
    \index{Helper Functions}
    
    \item \textbf{Consistent Naming Conventions}: Use consistent and descriptive naming conventions for variables and functions to improve code clarity.
    \index{Naming Conventions}
    
    \item \textbf{Memory Management}: Ensure that the algorithm manages memory efficiently, especially when dealing with large datasets.
    \index{Memory Management}
    
    \item \textbf{Implementing Iterative Solutions Carefully}: Ensure that loop conditions are correctly defined to prevent infinite loops or incorrect terminations.
    \index{Iterative Solutions}
    
    \item \textbf{Avoiding Floating-Point Precision Issues}: Since the problem deals with integers, floating-point precision is not a concern, simplifying the implementation.
    \index{Floating-Point Precision}
    
    \item \textbf{Testing and Validation}: Develop a comprehensive suite of test cases that cover all possible scenarios, including edge cases, to validate the correctness and efficiency of the implementation.
    \index{Testing and Validation}
    
    \item \textbf{Performance Considerations}: Optimize the loop conditions and operations to ensure that the function runs efficiently, especially for large input numbers.
    \index{Performance Considerations}
\end{itemize}

\section*{Conclusion}

The \textbf{Skyline Problem} is a quintessential example of applying advanced algorithmic techniques and geometric reasoning to solve complex spatial challenges. By leveraging the Sweep Line Algorithm and maintaining active building heights using a max heap, the solution efficiently constructs the skyline with optimal time and space complexities. Understanding and implementing such sophisticated algorithms not only enhances problem-solving skills but also provides a foundation for tackling a wide array of Computational Geometry problems in various domains, including computer graphics, urban planning simulations, and geographic information systems.

\printindex

% \input{sections/rectangle_overlap}
% \input{sections/rectangle_area}
% \input{sections/k_closest_points_to_origin}
% \input{sections/the_skyline_problem}
% % filename: k_closest_points_to_origin.tex

\problemsection{K Closest Points to Origin}
\label{chap:K_Closest_Points_to_Origin}
\marginnote{\href{https://leetcode.com/problems/k-closest-points-to-origin/}{[LeetCode Link]}\index{LeetCode}}
\marginnote{\href{https://www.geeksforgeeks.org/find-k-closest-points-origin/}{[GeeksForGeeks Link]}\index{GeeksForGeeks}}
\marginnote{\href{https://www.interviewbit.com/problems/k-closest-points/}{[InterviewBit Link]}\index{InterviewBit}}
\marginnote{\href{https://app.codesignal.com/challenges/k-closest-points-to-origin}{[CodeSignal Link]}\index{CodeSignal}}
\marginnote{\href{https://www.codewars.com/kata/k-closest-points-to-origin/train/python}{[Codewars Link]}\index{Codewars}}

The \textbf{K Closest Points to Origin} problem is a popular algorithmic challenge in Computational Geometry that involves identifying the \(k\) points closest to the origin in a 2D plane. This problem tests one's ability to apply efficient sorting and selection algorithms, understand distance computations, and optimize for performance. Mastery of this problem is essential for applications in spatial data analysis, nearest neighbor searches, and clustering algorithms.

\section*{Problem Statement}

Given an array of points where each point is represented as \([x, y]\) in the 2D plane, and an integer \(k\), return the \(k\) closest points to the origin \((0, 0)\).

The distance between two points \((x_1, y_1)\) and \((x_2, y_2)\) is the Euclidean distance \(\sqrt{(x_1 - x_2)^2 + (y_1 - y_2)^2}\). The origin is \((0, 0)\).

\textbf{Function signature in Python:}
\begin{lstlisting}[language=Python]
def kClosest(points: List[List[int]], K: int) -> List[List[int]]:
\end{lstlisting}

\section*{Examples}

\textbf{Example 1:}

\begin{verbatim}
Input: points = [[1,3],[-2,2]], K = 1
Output: [[-2,2]]
Explanation: 
The distance between (1, 3) and the origin is sqrt(10).
The distance between (-2, 2) and the origin is sqrt(8).
Since sqrt(8) < sqrt(10), (-2, 2) is closer to the origin.
\end{verbatim}

\textbf{Example 2:}

\begin{verbatim}
Input: points = [[3,3],[5,-1],[-2,4]], K = 2
Output: [[3,3],[-2,4]]
Explanation: 
The distances are sqrt(18), sqrt(26), and sqrt(20) respectively.
The two closest points are [3,3] and [-2,4].
\end{verbatim}

\textbf{Example 3:}

\begin{verbatim}
Input: points = [[0,1],[1,0]], K = 2
Output: [[0,1],[1,0]]
Explanation: 
Both points are equally close to the origin.
\end{verbatim}

\textbf{Example 4:}

\begin{verbatim}
Input: points = [[1,0],[0,1]], K = 1
Output: [[1,0]]
Explanation: 
Both points are equally close; returning any one is acceptable.
\end{verbatim}

\textbf{Constraints:}

\begin{itemize}
    \item \(1 \leq K \leq \text{points.length} \leq 10^4\)
    \item \(-10^4 < x_i, y_i < 10^4\)
\end{itemize}

LeetCode link: \href{https://leetcode.com/problems/k-closest-points-to-origin/}{K Closest Points to Origin}\index{LeetCode}

\section*{Algorithmic Approach}

To identify the \(k\) closest points to the origin, several algorithmic strategies can be employed. The most efficient methods aim to reduce the time complexity by avoiding the need to sort the entire list of points.

\subsection*{1. Sorting Based on Distance}

Calculate the Euclidean distance of each point from the origin and sort the points based on these distances. Select the first \(k\) points from the sorted list.

\begin{enumerate}
    \item Compute the distance for each point using the formula \(distance = x^2 + y^2\).
    \item Sort the points based on the computed distances.
    \item Return the first \(k\) points from the sorted list.
\end{enumerate}

\subsection*{2. Max Heap (Priority Queue)}

Use a max heap to maintain the \(k\) closest points. Iterate through each point, add it to the heap, and if the heap size exceeds \(k\), remove the farthest point.

\begin{enumerate}
    \item Initialize a max heap.
    \item For each point, compute its distance and add it to the heap.
    \item If the heap size exceeds \(k\), remove the point with the largest distance.
    \item After processing all points, the heap contains the \(k\) closest points.
\end{enumerate}

\subsection*{3. QuickSelect (Quick Sort Partitioning)}

Utilize the QuickSelect algorithm to find the \(k\) closest points without fully sorting the list.

\begin{enumerate}
    \item Choose a pivot point and partition the list based on distances relative to the pivot.
    \item Recursively apply QuickSelect to the partition containing the \(k\) closest points.
    \item Once the \(k\) closest points are identified, return them.
\end{enumerate}

\marginnote{QuickSelect offers an average time complexity of \(O(n)\), making it highly efficient for large datasets.}

\section*{Complexities}

\begin{itemize}
    \item \textbf{Sorting Based on Distance:}
    \begin{itemize}
        \item \textbf{Time Complexity:} \(O(n \log n)\)
        \item \textbf{Space Complexity:} \(O(n)\)
    \end{itemize}
    
    \item \textbf{Max Heap (Priority Queue):}
    \begin{itemize}
        \item \textbf{Time Complexity:} \(O(n \log k)\)
        \item \textbf{Space Complexity:} \(O(k)\)
    \end{itemize}
    
    \item \textbf{QuickSelect (Quick Sort Partitioning):}
    \begin{itemize}
        \item \textbf{Time Complexity:} Average case \(O(n)\), worst case \(O(n^2)\)
        \item \textbf{Space Complexity:} \(O(1)\) (in-place)
    \end{itemize}
\end{itemize}

\section*{Python Implementation}

\marginnote{Implementing QuickSelect provides an optimal average-case solution with linear time complexity.}

Below is the complete Python code implementing the \texttt{kClosest} function using the QuickSelect approach:

\begin{fullwidth}
\begin{lstlisting}[language=Python]
from typing import List
import random

class Solution:
    def kClosest(self, points: List[List[int]], K: int) -> List[List[int]]:
        def quickselect(left, right, K_smallest):
            if left == right:
                return
            
            # Select a random pivot_index
            pivot_index = random.randint(left, right)
            
            # Partition the array
            pivot_index = partition(left, right, pivot_index)
            
            # The pivot is in its final sorted position
            if K_smallest == pivot_index:
                return
            elif K_smallest < pivot_index:
                quickselect(left, pivot_index - 1, K_smallest)
            else:
                quickselect(pivot_index + 1, right, K_smallest)
        
        def partition(left, right, pivot_index):
            pivot_distance = distance(points[pivot_index])
            # Move pivot to end
            points[pivot_index], points[right] = points[right], points[pivot_index]
            store_index = left
            for i in range(left, right):
                if distance(points[i]) < pivot_distance:
                    points[store_index], points[i] = points[i], points[store_index]
                    store_index += 1
            # Move pivot to its final place
            points[right], points[store_index] = points[store_index], points[right]
            return store_index
        
        def distance(point):
            return point[0] ** 2 + point[1] ** 2
        
        n = len(points)
        quickselect(0, n - 1, K)
        return points[:K]

# Example usage:
solution = Solution()
print(solution.kClosest([[1,3],[-2,2]], 1))            # Output: [[-2,2]]
print(solution.kClosest([[3,3],[5,-1],[-2,4]], 2))     # Output: [[3,3],[-2,4]]
print(solution.kClosest([[0,1],[1,0]], 2))             # Output: [[0,1],[1,0]]
print(solution.kClosest([[1,0],[0,1]], 1))             # Output: [[1,0]] or [[0,1]]
\end{lstlisting}
\end{fullwidth}

This implementation uses the QuickSelect algorithm to efficiently find the \(k\) closest points to the origin without fully sorting the entire list. It ensures optimal performance even with large datasets.

\section*{Explanation}

The \texttt{kClosest} function identifies the \(k\) closest points to the origin using the QuickSelect algorithm. Here's a detailed breakdown of the implementation:

\subsection*{1. Distance Calculation}

\begin{itemize}
    \item The Euclidean distance is calculated as \(distance = x^2 + y^2\). Since we only need relative distances for comparison, the square root is omitted for efficiency.
\end{itemize}

\subsection*{2. QuickSelect Algorithm}

\begin{itemize}
    \item **Pivot Selection:**
    \begin{itemize}
        \item A random pivot is chosen to enhance the average-case performance.
    \end{itemize}
    
    \item **Partitioning:**
    \begin{itemize}
        \item The array is partitioned such that points with distances less than the pivot are moved to the left, and others to the right.
        \item The pivot is placed in its correct sorted position.
    \end{itemize}
    
    \item **Recursive Selection:**
    \begin{itemize}
        \item If the pivot's position matches \(K\), the selection is complete.
        \item Otherwise, recursively apply QuickSelect to the relevant partition.
    \end{itemize}
\end{itemize}

\subsection*{3. Final Selection}

\begin{itemize}
    \item After partitioning, the first \(K\) points in the list are the \(k\) closest points to the origin.
\end{itemize}

\subsection*{4. Example Walkthrough}

Consider the first example:
\begin{verbatim}
Input: points = [[1,3],[-2,2]], K = 1
Output: [[-2,2]]
\end{verbatim}

\begin{enumerate}
    \item **Calculate Distances:**
    \begin{itemize}
        \item [1,3] : \(1^2 + 3^2 = 10\)
        \item [-2,2] : \((-2)^2 + 2^2 = 8\)
    \end{itemize}
    
    \item **QuickSelect Process:**
    \begin{itemize}
        \item Choose a pivot, say [1,3] with distance 10.
        \item Compare and rearrange:
        \begin{itemize}
            \item [-2,2] has a smaller distance (8) and is moved to the left.
        \end{itemize}
        \item After partitioning, the list becomes [[-2,2], [1,3]].
        \item Since \(K = 1\), return the first point: [[-2,2]].
    \end{itemize}
\end{enumerate}

Thus, the function correctly identifies \([-2,2]\) as the closest point to the origin.

\section*{Why This Approach}

The QuickSelect algorithm is chosen for its average-case linear time complexity \(O(n)\), making it highly efficient for large datasets compared to sorting-based methods with \(O(n \log n)\) time complexity. By avoiding the need to sort the entire list, QuickSelect provides an optimal solution for finding the \(k\) closest points.

\section*{Alternative Approaches}

\subsection*{1. Sorting Based on Distance}

Sort all points based on their distances from the origin and select the first \(k\) points.

\begin{lstlisting}[language=Python]
class Solution:
    def kClosest(self, points: List[List[int]], K: int) -> List[List[int]]:
        points.sort(key=lambda P: P[0]**2 + P[1]**2)
        return points[:K]
\end{lstlisting}

\textbf{Complexities:}
\begin{itemize}
    \item \textbf{Time Complexity:} \(O(n \log n)\)
    \item \textbf{Space Complexity:} \(O(1)\)
\end{itemize}

\subsection*{2. Max Heap (Priority Queue)}

Use a max heap to maintain the \(k\) closest points.

\begin{lstlisting}[language=Python]
import heapq

class Solution:
    def kClosest(self, points: List[List[int]], K: int) -> List[List[int]]:
        heap = []
        for (x, y) in points:
            dist = -(x**2 + y**2)  # Max heap using negative distances
            heapq.heappush(heap, (dist, [x, y]))
            if len(heap) > K:
                heapq.heappop(heap)
        return [item[1] for item in heap]
\end{lstlisting}

\textbf{Complexities:}
\begin{itemize}
    \item \textbf{Time Complexity:} \(O(n \log k)\)
    \item \textbf{Space Complexity:} \(O(k)\)
\end{itemize}

\subsection*{3. Using Built-In Functions}

Leverage built-in functions for distance calculation and selection.

\begin{lstlisting}[language=Python]
import math

class Solution:
    def kClosest(self, points: List[List[int]], K: int) -> List[List[int]]:
        points.sort(key=lambda P: math.sqrt(P[0]**2 + P[1]**2))
        return points[:K]
\end{lstlisting}

\textbf{Note}: This method is similar to the sorting approach but uses the actual Euclidean distance.

\section*{Similar Problems to This One}

Several problems involve nearest neighbor searches, spatial data analysis, and efficient selection algorithms, utilizing similar algorithmic strategies:

\begin{itemize}
    \item \textbf{Closest Pair of Points}: Find the closest pair of points in a set.
    \item \textbf{Top K Frequent Elements}: Identify the most frequent elements in a dataset.
    \item \textbf{Kth Largest Element in an Array}: Find the \(k\)-th largest element in an unsorted array.
    \item \textbf{Sliding Window Maximum}: Find the maximum in each sliding window of size \(k\) over an array.
    \item \textbf{Merge K Sorted Lists}: Merge multiple sorted lists into a single sorted list.
    \item \textbf{Find Median from Data Stream}: Continuously find the median of a stream of numbers.
    \item \textbf{Top K Closest Stars}: Find the \(k\) closest stars to Earth based on their distances.
\end{itemize}

These problems reinforce concepts of efficient selection, heap usage, and distance computations in various contexts.

\section*{Things to Keep in Mind and Tricks}

When solving the \textbf{K Closest Points to Origin} problem, consider the following tips and best practices to enhance efficiency and correctness:

\begin{itemize}
    \item \textbf{Understand Distance Calculations}: Grasp the Euclidean distance formula and recognize that the square root can be omitted for comparison purposes.
    \index{Distance Calculations}
    
    \item \textbf{Leverage Efficient Algorithms}: Use QuickSelect or heap-based methods to optimize time complexity, especially for large datasets.
    \index{Efficient Algorithms}
    
    \item \textbf{Handle Ties Appropriately}: Decide how to handle points with identical distances when \(k\) is less than the number of such points.
    \index{Handling Ties}
    
    \item \textbf{Optimize Space Usage}: Choose algorithms that minimize additional space, such as in-place QuickSelect.
    \index{Space Optimization}
    
    \item \textbf{Use Appropriate Data Structures}: Utilize heaps, lists, and helper functions effectively to manage and process data.
    \index{Data Structures}
    
    \item \textbf{Implement Helper Functions}: Create helper functions for distance calculation and partitioning to enhance code modularity.
    \index{Helper Functions}
    
    \item \textbf{Code Readability}: Maintain clear and readable code through meaningful variable names and structured logic.
    \index{Code Readability}
    
    \item \textbf{Test Extensively}: Implement a wide range of test cases, including edge cases like multiple points with the same distance, to ensure robustness.
    \index{Extensive Testing}
    
    \item \textbf{Understand Algorithm Trade-offs}: Recognize the trade-offs between different approaches in terms of time and space complexities.
    \index{Algorithm Trade-offs}
    
    \item \textbf{Use Built-In Sorting Functions}: When using sorting-based approaches, leverage built-in functions for efficiency and simplicity.
    \index{Built-In Sorting}
    
    \item \textbf{Avoid Redundant Calculations}: Ensure that distance calculations are performed only when necessary to optimize performance.
    \index{Avoiding Redundant Calculations}
    
    \item \textbf{Language-Specific Features}: Utilize language-specific features or libraries that can simplify implementation, such as heapq in Python.
    \index{Language-Specific Features}
\end{itemize}

\section*{Corner and Special Cases to Test When Writing the Code}

When implementing the solution for the \textbf{K Closest Points to Origin} problem, it is crucial to consider and rigorously test various edge cases to ensure robustness and correctness:

\begin{itemize}
    \item \textbf{Multiple Points with Same Distance}: Ensure that the algorithm handles multiple points having the same distance from the origin.
    \index{Same Distance Points}
    
    \item \textbf{Points at Origin}: Include points that are exactly at the origin \((0,0)\).
    \index{Points at Origin}
    
    \item \textbf{Negative Coordinates}: Ensure that the algorithm correctly computes distances for points with negative \(x\) or \(y\) coordinates.
    \index{Negative Coordinates}
    
    \item \textbf{Large Coordinates}: Test with points having very large or very small coordinate values to verify integer handling.
    \index{Large Coordinates}
    
    \item \textbf{K Equals Number of Points}: When \(K\) is equal to the number of points, the algorithm should return all points.
    \index{K Equals Number of Points}
    
    \item \textbf{K is One}: Test with \(K = 1\) to ensure the closest point is correctly identified.
    \index{K is One}
    
    \item \textbf{All Points Same}: All points have the same coordinates.
    \index{All Points Same}
    
    \item \textbf{K is Zero}: Although \(K\) is defined to be at least 1, ensure that the algorithm gracefully handles \(K = 0\) if allowed.
    \index{K is Zero}
    
    \item \textbf{Single Point}: Only one point is provided, and \(K = 1\).
    \index{Single Point}
    
    \item \textbf{Mixed Coordinates}: Points with a mix of positive and negative coordinates.
    \index{Mixed Coordinates}
    
    \item \textbf{Points with Zero Distance}: Multiple points at the origin.
    \index{Zero Distance Points}
    
    \item \textbf{Sparse and Dense Points}: Densely packed points and sparsely distributed points.
    \index{Sparse and Dense Points}
    
    \item \textbf{Duplicate Points}: Multiple identical points in the input list.
    \index{Duplicate Points}
    
    \item \textbf{K Greater Than Number of Unique Points}: Ensure that the algorithm handles cases where \(K\) exceeds the number of unique points if applicable.
    \index{K Greater Than Unique Points}
\end{itemize}

\section*{Implementation Considerations}

When implementing the \texttt{kClosest} function, keep in mind the following considerations to ensure robustness and efficiency:

\begin{itemize}
    \item \textbf{Data Type Selection}: Use appropriate data types that can handle large input values without overflow or precision loss.
    \index{Data Type Selection}
    
    \item \textbf{Optimizing Distance Calculations}: Avoid calculating the square root since it is unnecessary for comparison purposes.
    \index{Optimizing Distance Calculations}
    
    \item \textbf{Choosing the Right Algorithm}: Select an algorithm based on the size of the input and the value of \(K\) to optimize time and space complexities.
    \index{Choosing the Right Algorithm}
    
    \item \textbf{Language-Specific Libraries}: Utilize language-specific libraries and functions (e.g., \texttt{heapq} in Python) to simplify implementation and enhance performance.
    \index{Language-Specific Libraries}
    
    \item \textbf{Avoiding Redundant Calculations}: Ensure that each point's distance is calculated only once to optimize performance.
    \index{Avoiding Redundant Calculations}
    
    \item \textbf{Implementing Helper Functions}: Create helper functions for tasks like distance calculation and partitioning to enhance modularity and readability.
    \index{Helper Functions}
    
    \item \textbf{Edge Case Handling}: Implement checks for edge cases to prevent incorrect results or runtime errors.
    \index{Edge Case Handling}
    
    \item \textbf{Testing and Validation}: Develop a comprehensive suite of test cases that cover all possible scenarios, including edge cases, to validate the correctness and efficiency of the implementation.
    \index{Testing and Validation}
    
    \item \textbf{Scalability}: Design the algorithm to scale efficiently with increasing input sizes, maintaining performance and resource utilization.
    \index{Scalability}
    
    \item \textbf{Consistent Naming Conventions}: Use consistent and descriptive naming conventions for variables and functions to improve code clarity.
    \index{Naming Conventions}
    
    \item \textbf{Memory Management}: Ensure that the algorithm manages memory efficiently, especially when dealing with large datasets.
    \index{Memory Management}
    
    \item \textbf{Avoiding Stack Overflow}: If implementing recursive approaches, be mindful of recursion limits and potential stack overflow issues.
    \index{Avoiding Stack Overflow}
    
    \item \textbf{Implementing Iterative Solutions}: Prefer iterative solutions when recursion may lead to increased space complexity or stack overflow.
    \index{Implementing Iterative Solutions}
\end{itemize}

\section*{Conclusion}

The \textbf{K Closest Points to Origin} problem exemplifies the application of efficient selection algorithms and geometric computations to solve spatial challenges effectively. By leveraging QuickSelect or heap-based methods, the algorithm achieves optimal time and space complexities, making it highly suitable for large datasets. Understanding and implementing such techniques not only enhances problem-solving skills but also provides a foundation for tackling more advanced Computational Geometry problems involving nearest neighbor searches, clustering, and spatial data analysis.

\printindex

% % filename: rectangle_overlap.tex

\problemsection{Rectangle Overlap}
\label{chap:Rectangle_Overlap}
\marginnote{\href{https://leetcode.com/problems/rectangle-overlap/}{[LeetCode Link]}\index{LeetCode}}
\marginnote{\href{https://www.geeksforgeeks.org/check-if-two-rectangles-overlap/}{[GeeksForGeeks Link]}\index{GeeksForGeeks}}
\marginnote{\href{https://www.interviewbit.com/problems/rectangle-overlap/}{[InterviewBit Link]}\index{InterviewBit}}
\marginnote{\href{https://app.codesignal.com/challenges/rectangle-overlap}{[CodeSignal Link]}\index{CodeSignal}}
\marginnote{\href{https://www.codewars.com/kata/rectangle-overlap/train/python}{[Codewars Link]}\index{Codewars}}

The \textbf{Rectangle Overlap} problem is a fundamental challenge in Computational Geometry that involves determining whether two axis-aligned rectangles overlap. This problem tests one's ability to understand geometric properties, implement conditional logic, and optimize for efficient computation. Mastery of this problem is essential for applications in computer graphics, collision detection, and spatial data analysis.

\section*{Problem Statement}

Given two axis-aligned rectangles in a 2D plane, determine if they overlap. Each rectangle is defined by its bottom-left and top-right coordinates.

A rectangle is represented as a list of four integers \([x1, y1, x2, y2]\), where \((x1, y1)\) are the coordinates of the bottom-left corner, and \((x2, y2)\) are the coordinates of the top-right corner.

\textbf{Function signature in Python:}
\begin{lstlisting}[language=Python]
def isRectangleOverlap(rec1: List[int], rec2: List[int]) -> bool:
\end{lstlisting}

\section*{Examples}

\textbf{Example 1:}

\begin{verbatim}
Input: rec1 = [0,0,2,2], rec2 = [1,1,3,3]
Output: True
Explanation: The rectangles overlap in the area defined by [1,1,2,2].
\end{verbatim}

\textbf{Example 2:}

\begin{verbatim}
Input: rec1 = [0,0,1,1], rec2 = [1,0,2,1]
Output: False
Explanation: The rectangles touch at the edge but do not overlap.
\end{verbatim}

\textbf{Example 3:}

\begin{verbatim}
Input: rec1 = [0,0,1,1], rec2 = [2,2,3,3]
Output: False
Explanation: The rectangles are completely separate.
\end{verbatim}

\textbf{Example 4:}

\begin{verbatim}
Input: rec1 = [0,0,5,5], rec2 = [3,3,7,7]
Output: True
Explanation: The rectangles overlap in the area defined by [3,3,5,5].
\end{verbatim}

\textbf{Example 5:}

\begin{verbatim}
Input: rec1 = [0,0,0,0], rec2 = [0,0,0,0]
Output: False
Explanation: Both rectangles are degenerate points.
\end{verbatim}

\textbf{Constraints:}

\begin{itemize}
    \item All coordinates are integers in the range \([-10^9, 10^9]\).
    \item For each rectangle, \(x1 < x2\) and \(y1 < y2\).
\end{itemize}

LeetCode link: \href{https://leetcode.com/problems/rectangle-overlap/}{Rectangle Overlap}\index{LeetCode}

\section*{Algorithmic Approach}

To determine whether two axis-aligned rectangles overlap, we can use the following logical conditions:

1. **Non-Overlap Conditions:**
   - One rectangle is to the left of the other.
   - One rectangle is above the other.

2. **Overlap Condition:**
   - If neither of the non-overlap conditions is true, the rectangles must overlap.

\subsection*{Steps:}

1. **Extract Coordinates:**
   - For both rectangles, extract the bottom-left and top-right coordinates.

2. **Check Non-Overlap Conditions:**
   - If the right side of the first rectangle is less than or equal to the left side of the second rectangle, they do not overlap.
   - If the left side of the first rectangle is greater than or equal to the right side of the second rectangle, they do not overlap.
   - If the top side of the first rectangle is less than or equal to the bottom side of the second rectangle, they do not overlap.
   - If the bottom side of the first rectangle is greater than or equal to the top side of the second rectangle, they do not overlap.

3. **Determine Overlap:**
   - If none of the non-overlap conditions are met, the rectangles overlap.

\marginnote{This approach provides an efficient \(O(1)\) time complexity solution by leveraging simple geometric comparisons.}

\section*{Complexities}

\begin{itemize}
    \item \textbf{Time Complexity:} \(O(1)\). The algorithm performs a constant number of comparisons regardless of input size.
    
    \item \textbf{Space Complexity:} \(O(1)\). Only a fixed amount of extra space is used for variables.
\end{itemize}

\section*{Python Implementation}

\marginnote{Implementing the overlap check using coordinate comparisons ensures an optimal and straightforward solution.}

Below is the complete Python code implementing the \texttt{isRectangleOverlap} function:

\begin{fullwidth}
\begin{lstlisting}[language=Python]
from typing import List

class Solution:
    def isRectangleOverlap(self, rec1: List[int], rec2: List[int]) -> bool:
        # Extract coordinates
        left1, bottom1, right1, top1 = rec1
        left2, bottom2, right2, top2 = rec2
        
        # Check non-overlapping conditions
        if right1 <= left2 or right2 <= left1:
            return False
        if top1 <= bottom2 or top2 <= bottom1:
            return False
        
        # If none of the above, rectangles overlap
        return True

# Example usage:
solution = Solution()
print(solution.isRectangleOverlap([0,0,2,2], [1,1,3,3]))  # Output: True
print(solution.isRectangleOverlap([0,0,1,1], [1,0,2,1]))  # Output: False
print(solution.isRectangleOverlap([0,0,1,1], [2,2,3,3]))  # Output: False
print(solution.isRectangleOverlap([0,0,5,5], [3,3,7,7]))  # Output: True
print(solution.isRectangleOverlap([0,0,0,0], [0,0,0,0]))  # Output: False
\end{lstlisting}
\end{fullwidth}

This implementation efficiently checks for overlap by comparing the coordinates of the two rectangles. If any of the non-overlapping conditions are met, it returns \texttt{False}; otherwise, it returns \texttt{True}.

\section*{Explanation}

The \texttt{isRectangleOverlap} function determines whether two axis-aligned rectangles overlap by comparing their respective coordinates. Here's a detailed breakdown of the implementation:

\subsection*{1. Extract Coordinates}

\begin{itemize}
    \item For each rectangle, extract the left (\(x1\)), bottom (\(y1\)), right (\(x2\)), and top (\(y2\)) coordinates.
    \item This simplifies the comparison process by providing clear variables representing each side of the rectangles.
\end{itemize}

\subsection*{2. Check Non-Overlap Conditions}

\begin{itemize}
    \item **Horizontal Separation:**
    \begin{itemize}
        \item If the right side of the first rectangle (\(right1\)) is less than or equal to the left side of the second rectangle (\(left2\)), there is no horizontal overlap.
        \item Similarly, if the right side of the second rectangle (\(right2\)) is less than or equal to the left side of the first rectangle (\(left1\)), there is no horizontal overlap.
    \end{itemize}
    
    \item **Vertical Separation:**
    \begin{itemize}
        \item If the top side of the first rectangle (\(top1\)) is less than or equal to the bottom side of the second rectangle (\(bottom2\)), there is no vertical overlap.
        \item Similarly, if the top side of the second rectangle (\(top2\)) is less than or equal to the bottom side of the first rectangle (\(bottom1\)), there is no vertical overlap.
    \end{itemize}
    
    \item If any of these non-overlapping conditions are true, the rectangles do not overlap, and the function returns \texttt{False}.
\end{itemize}

\subsection*{3. Determine Overlap}

\begin{itemize}
    \item If none of the non-overlapping conditions are met, it implies that the rectangles overlap both horizontally and vertically.
    \item The function returns \texttt{True} in this case.
\end{itemize}

\subsection*{4. Example Walkthrough}

Consider the first example:
\begin{verbatim}
Input: rec1 = [0,0,2,2], rec2 = [1,1,3,3]
Output: True
\end{verbatim}

\begin{enumerate}
    \item Extract coordinates:
    \begin{itemize}
        \item rec1: left1 = 0, bottom1 = 0, right1 = 2, top1 = 2
        \item rec2: left2 = 1, bottom2 = 1, right2 = 3, top2 = 3
    \end{itemize}
    
    \item Check non-overlap conditions:
    \begin{itemize}
        \item \(right1 = 2\) is not less than or equal to \(left2 = 1\)
        \item \(right2 = 3\) is not less than or equal to \(left1 = 0\)
        \item \(top1 = 2\) is not less than or equal to \(bottom2 = 1\)
        \item \(top2 = 3\) is not less than or equal to \(bottom1 = 0\)
    \end{itemize}
    
    \item Since none of the non-overlapping conditions are met, the rectangles overlap.
\end{enumerate}

Thus, the function correctly returns \texttt{True}.

\section*{Why This Approach}

This approach is chosen for its simplicity and efficiency. By leveraging direct coordinate comparisons, the algorithm achieves constant time complexity without the need for complex data structures or iterative processes. It effectively handles all possible scenarios of rectangle positioning, ensuring accurate detection of overlaps.

\section*{Alternative Approaches}

\subsection*{1. Separating Axis Theorem (SAT)}

The Separating Axis Theorem is a more generalized method for detecting overlaps between convex shapes. While it is not necessary for axis-aligned rectangles, understanding SAT can be beneficial for more complex geometric problems.

\begin{lstlisting}[language=Python]
def isRectangleOverlap(rec1: List[int], rec2: List[int]) -> bool:
    # Using SAT for axis-aligned rectangles
    return not (rec1[2] <= rec2[0] or rec1[0] >= rec2[2] or
                rec1[3] <= rec2[1] or rec1[1] >= rec2[3])
\end{lstlisting}

\textbf{Note}: This implementation is functionally identical to the primary approach but leverages a more generalized geometric theorem.

\subsection*{2. Area-Based Approach}

Calculate the overlapping area between the two rectangles. If the overlapping area is positive, the rectangles overlap.

\begin{lstlisting}[language=Python]
def isRectangleOverlap(rec1: List[int], rec2: List[int]) -> bool:
    # Calculate overlap in x and y dimensions
    x_overlap = min(rec1[2], rec2[2]) - max(rec1[0], rec2[0])
    y_overlap = min(rec1[3], rec2[3]) - max(rec1[1], rec2[1])
    
    # Overlap exists if both overlaps are positive
    return x_overlap > 0 and y_overlap > 0
\end{lstlisting}

\textbf{Complexities:}
\begin{itemize}
    \item \textbf{Time Complexity:} \(O(1)\)
    \item \textbf{Space Complexity:} \(O(1)\)
\end{itemize}

\subsection*{3. Using Rectangles Intersection Function}

Utilize built-in or library functions that handle geometric intersections.

\begin{lstlisting}[language=Python]
from shapely.geometry import box

def isRectangleOverlap(rec1: List[int], rec2: List[int]) -> bool:
    rectangle1 = box(rec1[0], rec1[1], rec1[2], rec1[3])
    rectangle2 = box(rec2[0], rec2[1], rec2[2], rec2[3])
    return rectangle1.intersects(rectangle2) and not rectangle1.touches(rectangle2)
\end{lstlisting}

\textbf{Note}: This approach requires the \texttt{shapely} library and is more suitable for complex geometric operations.

\section*{Similar Problems to This One}

Several problems revolve around geometric overlap, intersection detection, and spatial reasoning, utilizing similar algorithmic strategies:

\begin{itemize}
    \item \textbf{Interval Overlap}: Determine if two intervals on a line overlap.
    \item \textbf{Circle Overlap}: Determine if two circles overlap based on their radii and centers.
    \item \textbf{Polygon Overlap}: Determine if two polygons overlap using algorithms like SAT.
    \item \textbf{Closest Pair of Points}: Find the closest pair of points in a set.
    \item \textbf{Convex Hull}: Compute the convex hull of a set of points.
    \item \textbf{Intersection of Lines}: Find the intersection point of two lines.
    \item \textbf{Point Inside Polygon}: Determine if a point lies inside a given polygon.
\end{itemize}

These problems reinforce the concepts of spatial reasoning, geometric property analysis, and efficient algorithm design in various contexts.

\section*{Things to Keep in Mind and Tricks}

When working with the \textbf{Rectangle Overlap} problem, consider the following tips and best practices to enhance efficiency and correctness:

\begin{itemize}
    \item \textbf{Understand Geometric Relationships}: Grasp the positional relationships between rectangles to simplify overlap detection.
    \index{Geometric Relationships}
    
    \item \textbf{Leverage Coordinate Comparisons}: Use direct comparisons of rectangle coordinates to determine spatial relationships.
    \index{Coordinate Comparisons}
    
    \item \textbf{Handle Edge Cases}: Consider cases where rectangles touch at edges or corners without overlapping.
    \index{Edge Cases}
    
    \item \textbf{Optimize for Efficiency}: Aim for a constant time \(O(1)\) solution by avoiding unnecessary computations or iterations.
    \index{Efficiency Optimization}
    
    \item \textbf{Avoid Floating-Point Precision Issues}: Since all coordinates are integers, floating-point precision is not a concern, simplifying the implementation.
    \index{Floating-Point Precision}
    
    \item \textbf{Use Helper Functions}: Create helper functions to encapsulate repetitive tasks, such as extracting coordinates or checking specific conditions.
    \index{Helper Functions}
    
    \item \textbf{Code Readability}: Maintain clear and readable code through meaningful variable names and structured logic.
    \index{Code Readability}
    
    \item \textbf{Test Extensively}: Implement a wide range of test cases, including overlapping, non-overlapping, and edge-touching rectangles, to ensure robustness.
    \index{Extensive Testing}
    
    \item \textbf{Understand Axis-Aligned Constraints}: Recognize that axis-aligned rectangles simplify overlap detection compared to rotated rectangles.
    \index{Axis-Aligned Constraints}
    
    \item \textbf{Simplify Logical Conditions}: Combine multiple conditions logically to streamline the overlap detection process.
    \index{Logical Conditions}
\end{itemize}

\section*{Corner and Special Cases to Test When Writing the Code}

When implementing the solution for the \textbf{Rectangle Overlap} problem, it is crucial to consider and rigorously test various edge cases to ensure robustness and correctness:

\begin{itemize}
    \item \textbf{No Overlap}: Rectangles are completely separate.
    \index{No Overlap}
    
    \item \textbf{Partial Overlap}: Rectangles overlap in one or more regions.
    \index{Partial Overlap}
    
    \item \textbf{Edge Touching}: Rectangles touch exactly at one edge without overlapping.
    \index{Edge Touching}
    
    \item \textbf{Corner Touching}: Rectangles touch exactly at one corner without overlapping.
    \index{Corner Touching}
    
    \item \textbf{One Rectangle Inside Another}: One rectangle is entirely within the other.
    \index{Rectangle Inside}
    
    \item \textbf{Identical Rectangles}: Both rectangles have the same coordinates.
    \index{Identical Rectangles}
    
    \item \textbf{Degenerate Rectangles}: Rectangles with zero area (e.g., \(x1 = x2\) or \(y1 = y2\)).
    \index{Degenerate Rectangles}
    
    \item \textbf{Large Coordinates}: Rectangles with very large coordinate values to test performance and integer handling.
    \index{Large Coordinates}
    
    \item \textbf{Negative Coordinates}: Rectangles positioned in negative coordinate space.
    \index{Negative Coordinates}
    
    \item \textbf{Mixed Overlapping Scenarios}: Combinations of the above cases to ensure comprehensive coverage.
    \index{Mixed Overlapping Scenarios}
    
    \item \textbf{Minimum and Maximum Bounds}: Rectangles at the minimum and maximum limits of the coordinate range.
    \index{Minimum and Maximum Bounds}
\end{itemize}

\section*{Implementation Considerations}

When implementing the \texttt{isRectangleOverlap} function, keep in mind the following considerations to ensure robustness and efficiency:

\begin{itemize}
    \item \textbf{Data Type Selection}: Use appropriate data types that can handle the range of input values without overflow or underflow.
    \index{Data Type Selection}
    
    \item \textbf{Optimizing Comparisons}: Structure logical conditions to short-circuit evaluations as soon as a non-overlapping condition is met.
    \index{Optimizing Comparisons}
    
    \item \textbf{Language-Specific Constraints}: Be aware of how the programming language handles integer division and comparisons.
    \index{Language-Specific Constraints}
    
    \item \textbf{Avoiding Redundant Calculations}: Ensure that each comparison contributes towards determining overlap without unnecessary repetitions.
    \index{Avoiding Redundant Calculations}
    
    \item \textbf{Code Readability and Documentation}: Maintain clear and readable code through meaningful variable names and comprehensive comments to facilitate understanding and maintenance.
    \index{Code Readability}
    
    \item \textbf{Edge Case Handling}: Implement checks for edge cases to prevent incorrect results or runtime errors.
    \index{Edge Case Handling}
    
    \item \textbf{Testing and Validation}: Develop a comprehensive suite of test cases that cover all possible scenarios, including edge cases, to validate the correctness and efficiency of the implementation.
    \index{Testing and Validation}
    
    \item \textbf{Scalability}: Design the algorithm to scale efficiently with increasing input sizes, maintaining performance and resource utilization.
    \index{Scalability}
    
    \item \textbf{Using Helper Functions}: Consider creating helper functions for repetitive tasks, such as extracting and comparing coordinates, to enhance modularity and reusability.
    \index{Helper Functions}
    
    \item \textbf{Consistent Naming Conventions}: Use consistent and descriptive naming conventions for variables to improve code clarity.
    \index{Naming Conventions}
    
    \item \textbf{Handling Floating-Point Coordinates}: Although the problem specifies integer coordinates, ensure that the implementation can handle floating-point numbers if needed in extended scenarios.
    \index{Floating-Point Coordinates}
    
    \item \textbf{Avoiding Floating-Point Precision Issues}: Since all coordinates are integers, floating-point precision is not a concern, simplifying the implementation.
    \index{Floating-Point Precision}
    
    \item \textbf{Implementing Unit Tests}: Develop unit tests for each logical condition to ensure that all scenarios are correctly handled.
    \index{Unit Tests}
    
    \item \textbf{Error Handling}: Incorporate error handling to manage invalid inputs gracefully.
    \index{Error Handling}
\end{itemize}

\section*{Conclusion}

The \textbf{Rectangle Overlap} problem exemplifies the application of fundamental geometric principles and conditional logic to solve spatial challenges efficiently. By leveraging simple coordinate comparisons, the algorithm achieves optimal time and space complexities, making it highly suitable for real-time applications such as collision detection in gaming, layout planning in graphics, and spatial data analysis. Understanding and implementing such techniques not only enhances problem-solving skills but also provides a foundation for tackling more complex Computational Geometry problems involving varied geometric shapes and interactions.

\printindex

% % filename: rectangle_overlap.tex

\problemsection{Rectangle Overlap}
\label{chap:Rectangle_Overlap}
\marginnote{\href{https://leetcode.com/problems/rectangle-overlap/}{[LeetCode Link]}\index{LeetCode}}
\marginnote{\href{https://www.geeksforgeeks.org/check-if-two-rectangles-overlap/}{[GeeksForGeeks Link]}\index{GeeksForGeeks}}
\marginnote{\href{https://www.interviewbit.com/problems/rectangle-overlap/}{[InterviewBit Link]}\index{InterviewBit}}
\marginnote{\href{https://app.codesignal.com/challenges/rectangle-overlap}{[CodeSignal Link]}\index{CodeSignal}}
\marginnote{\href{https://www.codewars.com/kata/rectangle-overlap/train/python}{[Codewars Link]}\index{Codewars}}

The \textbf{Rectangle Overlap} problem is a fundamental challenge in Computational Geometry that involves determining whether two axis-aligned rectangles overlap. This problem tests one's ability to understand geometric properties, implement conditional logic, and optimize for efficient computation. Mastery of this problem is essential for applications in computer graphics, collision detection, and spatial data analysis.

\section*{Problem Statement}

Given two axis-aligned rectangles in a 2D plane, determine if they overlap. Each rectangle is defined by its bottom-left and top-right coordinates.

A rectangle is represented as a list of four integers \([x1, y1, x2, y2]\), where \((x1, y1)\) are the coordinates of the bottom-left corner, and \((x2, y2)\) are the coordinates of the top-right corner.

\textbf{Function signature in Python:}
\begin{lstlisting}[language=Python]
def isRectangleOverlap(rec1: List[int], rec2: List[int]) -> bool:
\end{lstlisting}

\section*{Examples}

\textbf{Example 1:}

\begin{verbatim}
Input: rec1 = [0,0,2,2], rec2 = [1,1,3,3]
Output: True
Explanation: The rectangles overlap in the area defined by [1,1,2,2].
\end{verbatim}

\textbf{Example 2:}

\begin{verbatim}
Input: rec1 = [0,0,1,1], rec2 = [1,0,2,1]
Output: False
Explanation: The rectangles touch at the edge but do not overlap.
\end{verbatim}

\textbf{Example 3:}

\begin{verbatim}
Input: rec1 = [0,0,1,1], rec2 = [2,2,3,3]
Output: False
Explanation: The rectangles are completely separate.
\end{verbatim}

\textbf{Example 4:}

\begin{verbatim}
Input: rec1 = [0,0,5,5], rec2 = [3,3,7,7]
Output: True
Explanation: The rectangles overlap in the area defined by [3,3,5,5].
\end{verbatim}

\textbf{Example 5:}

\begin{verbatim}
Input: rec1 = [0,0,0,0], rec2 = [0,0,0,0]
Output: False
Explanation: Both rectangles are degenerate points.
\end{verbatim}

\textbf{Constraints:}

\begin{itemize}
    \item All coordinates are integers in the range \([-10^9, 10^9]\).
    \item For each rectangle, \(x1 < x2\) and \(y1 < y2\).
\end{itemize}

LeetCode link: \href{https://leetcode.com/problems/rectangle-overlap/}{Rectangle Overlap}\index{LeetCode}

\section*{Algorithmic Approach}

To determine whether two axis-aligned rectangles overlap, we can use the following logical conditions:

1. **Non-Overlap Conditions:**
   - One rectangle is to the left of the other.
   - One rectangle is above the other.

2. **Overlap Condition:**
   - If neither of the non-overlap conditions is true, the rectangles must overlap.

\subsection*{Steps:}

1. **Extract Coordinates:**
   - For both rectangles, extract the bottom-left and top-right coordinates.

2. **Check Non-Overlap Conditions:**
   - If the right side of the first rectangle is less than or equal to the left side of the second rectangle, they do not overlap.
   - If the left side of the first rectangle is greater than or equal to the right side of the second rectangle, they do not overlap.
   - If the top side of the first rectangle is less than or equal to the bottom side of the second rectangle, they do not overlap.
   - If the bottom side of the first rectangle is greater than or equal to the top side of the second rectangle, they do not overlap.

3. **Determine Overlap:**
   - If none of the non-overlap conditions are met, the rectangles overlap.

\marginnote{This approach provides an efficient \(O(1)\) time complexity solution by leveraging simple geometric comparisons.}

\section*{Complexities}

\begin{itemize}
    \item \textbf{Time Complexity:} \(O(1)\). The algorithm performs a constant number of comparisons regardless of input size.
    
    \item \textbf{Space Complexity:} \(O(1)\). Only a fixed amount of extra space is used for variables.
\end{itemize}

\section*{Python Implementation}

\marginnote{Implementing the overlap check using coordinate comparisons ensures an optimal and straightforward solution.}

Below is the complete Python code implementing the \texttt{isRectangleOverlap} function:

\begin{fullwidth}
\begin{lstlisting}[language=Python]
from typing import List

class Solution:
    def isRectangleOverlap(self, rec1: List[int], rec2: List[int]) -> bool:
        # Extract coordinates
        left1, bottom1, right1, top1 = rec1
        left2, bottom2, right2, top2 = rec2
        
        # Check non-overlapping conditions
        if right1 <= left2 or right2 <= left1:
            return False
        if top1 <= bottom2 or top2 <= bottom1:
            return False
        
        # If none of the above, rectangles overlap
        return True

# Example usage:
solution = Solution()
print(solution.isRectangleOverlap([0,0,2,2], [1,1,3,3]))  # Output: True
print(solution.isRectangleOverlap([0,0,1,1], [1,0,2,1]))  # Output: False
print(solution.isRectangleOverlap([0,0,1,1], [2,2,3,3]))  # Output: False
print(solution.isRectangleOverlap([0,0,5,5], [3,3,7,7]))  # Output: True
print(solution.isRectangleOverlap([0,0,0,0], [0,0,0,0]))  # Output: False
\end{lstlisting}
\end{fullwidth}

This implementation efficiently checks for overlap by comparing the coordinates of the two rectangles. If any of the non-overlapping conditions are met, it returns \texttt{False}; otherwise, it returns \texttt{True}.

\section*{Explanation}

The \texttt{isRectangleOverlap} function determines whether two axis-aligned rectangles overlap by comparing their respective coordinates. Here's a detailed breakdown of the implementation:

\subsection*{1. Extract Coordinates}

\begin{itemize}
    \item For each rectangle, extract the left (\(x1\)), bottom (\(y1\)), right (\(x2\)), and top (\(y2\)) coordinates.
    \item This simplifies the comparison process by providing clear variables representing each side of the rectangles.
\end{itemize}

\subsection*{2. Check Non-Overlap Conditions}

\begin{itemize}
    \item **Horizontal Separation:**
    \begin{itemize}
        \item If the right side of the first rectangle (\(right1\)) is less than or equal to the left side of the second rectangle (\(left2\)), there is no horizontal overlap.
        \item Similarly, if the right side of the second rectangle (\(right2\)) is less than or equal to the left side of the first rectangle (\(left1\)), there is no horizontal overlap.
    \end{itemize}
    
    \item **Vertical Separation:**
    \begin{itemize}
        \item If the top side of the first rectangle (\(top1\)) is less than or equal to the bottom side of the second rectangle (\(bottom2\)), there is no vertical overlap.
        \item Similarly, if the top side of the second rectangle (\(top2\)) is less than or equal to the bottom side of the first rectangle (\(bottom1\)), there is no vertical overlap.
    \end{itemize}
    
    \item If any of these non-overlapping conditions are true, the rectangles do not overlap, and the function returns \texttt{False}.
\end{itemize}

\subsection*{3. Determine Overlap}

\begin{itemize}
    \item If none of the non-overlapping conditions are met, it implies that the rectangles overlap both horizontally and vertically.
    \item The function returns \texttt{True} in this case.
\end{itemize}

\subsection*{4. Example Walkthrough}

Consider the first example:
\begin{verbatim}
Input: rec1 = [0,0,2,2], rec2 = [1,1,3,3]
Output: True
\end{verbatim}

\begin{enumerate}
    \item Extract coordinates:
    \begin{itemize}
        \item rec1: left1 = 0, bottom1 = 0, right1 = 2, top1 = 2
        \item rec2: left2 = 1, bottom2 = 1, right2 = 3, top2 = 3
    \end{itemize}
    
    \item Check non-overlap conditions:
    \begin{itemize}
        \item \(right1 = 2\) is not less than or equal to \(left2 = 1\)
        \item \(right2 = 3\) is not less than or equal to \(left1 = 0\)
        \item \(top1 = 2\) is not less than or equal to \(bottom2 = 1\)
        \item \(top2 = 3\) is not less than or equal to \(bottom1 = 0\)
    \end{itemize}
    
    \item Since none of the non-overlapping conditions are met, the rectangles overlap.
\end{enumerate}

Thus, the function correctly returns \texttt{True}.

\section*{Why This Approach}

This approach is chosen for its simplicity and efficiency. By leveraging direct coordinate comparisons, the algorithm achieves constant time complexity without the need for complex data structures or iterative processes. It effectively handles all possible scenarios of rectangle positioning, ensuring accurate detection of overlaps.

\section*{Alternative Approaches}

\subsection*{1. Separating Axis Theorem (SAT)}

The Separating Axis Theorem is a more generalized method for detecting overlaps between convex shapes. While it is not necessary for axis-aligned rectangles, understanding SAT can be beneficial for more complex geometric problems.

\begin{lstlisting}[language=Python]
def isRectangleOverlap(rec1: List[int], rec2: List[int]) -> bool:
    # Using SAT for axis-aligned rectangles
    return not (rec1[2] <= rec2[0] or rec1[0] >= rec2[2] or
                rec1[3] <= rec2[1] or rec1[1] >= rec2[3])
\end{lstlisting}

\textbf{Note}: This implementation is functionally identical to the primary approach but leverages a more generalized geometric theorem.

\subsection*{2. Area-Based Approach}

Calculate the overlapping area between the two rectangles. If the overlapping area is positive, the rectangles overlap.

\begin{lstlisting}[language=Python]
def isRectangleOverlap(rec1: List[int], rec2: List[int]) -> bool:
    # Calculate overlap in x and y dimensions
    x_overlap = min(rec1[2], rec2[2]) - max(rec1[0], rec2[0])
    y_overlap = min(rec1[3], rec2[3]) - max(rec1[1], rec2[1])
    
    # Overlap exists if both overlaps are positive
    return x_overlap > 0 and y_overlap > 0
\end{lstlisting}

\textbf{Complexities:}
\begin{itemize}
    \item \textbf{Time Complexity:} \(O(1)\)
    \item \textbf{Space Complexity:} \(O(1)\)
\end{itemize}

\subsection*{3. Using Rectangles Intersection Function}

Utilize built-in or library functions that handle geometric intersections.

\begin{lstlisting}[language=Python]
from shapely.geometry import box

def isRectangleOverlap(rec1: List[int], rec2: List[int]) -> bool:
    rectangle1 = box(rec1[0], rec1[1], rec1[2], rec1[3])
    rectangle2 = box(rec2[0], rec2[1], rec2[2], rec2[3])
    return rectangle1.intersects(rectangle2) and not rectangle1.touches(rectangle2)
\end{lstlisting}

\textbf{Note}: This approach requires the \texttt{shapely} library and is more suitable for complex geometric operations.

\section*{Similar Problems to This One}

Several problems revolve around geometric overlap, intersection detection, and spatial reasoning, utilizing similar algorithmic strategies:

\begin{itemize}
    \item \textbf{Interval Overlap}: Determine if two intervals on a line overlap.
    \item \textbf{Circle Overlap}: Determine if two circles overlap based on their radii and centers.
    \item \textbf{Polygon Overlap}: Determine if two polygons overlap using algorithms like SAT.
    \item \textbf{Closest Pair of Points}: Find the closest pair of points in a set.
    \item \textbf{Convex Hull}: Compute the convex hull of a set of points.
    \item \textbf{Intersection of Lines}: Find the intersection point of two lines.
    \item \textbf{Point Inside Polygon}: Determine if a point lies inside a given polygon.
\end{itemize}

These problems reinforce the concepts of spatial reasoning, geometric property analysis, and efficient algorithm design in various contexts.

\section*{Things to Keep in Mind and Tricks}

When working with the \textbf{Rectangle Overlap} problem, consider the following tips and best practices to enhance efficiency and correctness:

\begin{itemize}
    \item \textbf{Understand Geometric Relationships}: Grasp the positional relationships between rectangles to simplify overlap detection.
    \index{Geometric Relationships}
    
    \item \textbf{Leverage Coordinate Comparisons}: Use direct comparisons of rectangle coordinates to determine spatial relationships.
    \index{Coordinate Comparisons}
    
    \item \textbf{Handle Edge Cases}: Consider cases where rectangles touch at edges or corners without overlapping.
    \index{Edge Cases}
    
    \item \textbf{Optimize for Efficiency}: Aim for a constant time \(O(1)\) solution by avoiding unnecessary computations or iterations.
    \index{Efficiency Optimization}
    
    \item \textbf{Avoid Floating-Point Precision Issues}: Since all coordinates are integers, floating-point precision is not a concern, simplifying the implementation.
    \index{Floating-Point Precision}
    
    \item \textbf{Use Helper Functions}: Create helper functions to encapsulate repetitive tasks, such as extracting coordinates or checking specific conditions.
    \index{Helper Functions}
    
    \item \textbf{Code Readability}: Maintain clear and readable code through meaningful variable names and structured logic.
    \index{Code Readability}
    
    \item \textbf{Test Extensively}: Implement a wide range of test cases, including overlapping, non-overlapping, and edge-touching rectangles, to ensure robustness.
    \index{Extensive Testing}
    
    \item \textbf{Understand Axis-Aligned Constraints}: Recognize that axis-aligned rectangles simplify overlap detection compared to rotated rectangles.
    \index{Axis-Aligned Constraints}
    
    \item \textbf{Simplify Logical Conditions}: Combine multiple conditions logically to streamline the overlap detection process.
    \index{Logical Conditions}
\end{itemize}

\section*{Corner and Special Cases to Test When Writing the Code}

When implementing the solution for the \textbf{Rectangle Overlap} problem, it is crucial to consider and rigorously test various edge cases to ensure robustness and correctness:

\begin{itemize}
    \item \textbf{No Overlap}: Rectangles are completely separate.
    \index{No Overlap}
    
    \item \textbf{Partial Overlap}: Rectangles overlap in one or more regions.
    \index{Partial Overlap}
    
    \item \textbf{Edge Touching}: Rectangles touch exactly at one edge without overlapping.
    \index{Edge Touching}
    
    \item \textbf{Corner Touching}: Rectangles touch exactly at one corner without overlapping.
    \index{Corner Touching}
    
    \item \textbf{One Rectangle Inside Another}: One rectangle is entirely within the other.
    \index{Rectangle Inside}
    
    \item \textbf{Identical Rectangles}: Both rectangles have the same coordinates.
    \index{Identical Rectangles}
    
    \item \textbf{Degenerate Rectangles}: Rectangles with zero area (e.g., \(x1 = x2\) or \(y1 = y2\)).
    \index{Degenerate Rectangles}
    
    \item \textbf{Large Coordinates}: Rectangles with very large coordinate values to test performance and integer handling.
    \index{Large Coordinates}
    
    \item \textbf{Negative Coordinates}: Rectangles positioned in negative coordinate space.
    \index{Negative Coordinates}
    
    \item \textbf{Mixed Overlapping Scenarios}: Combinations of the above cases to ensure comprehensive coverage.
    \index{Mixed Overlapping Scenarios}
    
    \item \textbf{Minimum and Maximum Bounds}: Rectangles at the minimum and maximum limits of the coordinate range.
    \index{Minimum and Maximum Bounds}
\end{itemize}

\section*{Implementation Considerations}

When implementing the \texttt{isRectangleOverlap} function, keep in mind the following considerations to ensure robustness and efficiency:

\begin{itemize}
    \item \textbf{Data Type Selection}: Use appropriate data types that can handle the range of input values without overflow or underflow.
    \index{Data Type Selection}
    
    \item \textbf{Optimizing Comparisons}: Structure logical conditions to short-circuit evaluations as soon as a non-overlapping condition is met.
    \index{Optimizing Comparisons}
    
    \item \textbf{Language-Specific Constraints}: Be aware of how the programming language handles integer division and comparisons.
    \index{Language-Specific Constraints}
    
    \item \textbf{Avoiding Redundant Calculations}: Ensure that each comparison contributes towards determining overlap without unnecessary repetitions.
    \index{Avoiding Redundant Calculations}
    
    \item \textbf{Code Readability and Documentation}: Maintain clear and readable code through meaningful variable names and comprehensive comments to facilitate understanding and maintenance.
    \index{Code Readability}
    
    \item \textbf{Edge Case Handling}: Implement checks for edge cases to prevent incorrect results or runtime errors.
    \index{Edge Case Handling}
    
    \item \textbf{Testing and Validation}: Develop a comprehensive suite of test cases that cover all possible scenarios, including edge cases, to validate the correctness and efficiency of the implementation.
    \index{Testing and Validation}
    
    \item \textbf{Scalability}: Design the algorithm to scale efficiently with increasing input sizes, maintaining performance and resource utilization.
    \index{Scalability}
    
    \item \textbf{Using Helper Functions}: Consider creating helper functions for repetitive tasks, such as extracting and comparing coordinates, to enhance modularity and reusability.
    \index{Helper Functions}
    
    \item \textbf{Consistent Naming Conventions}: Use consistent and descriptive naming conventions for variables to improve code clarity.
    \index{Naming Conventions}
    
    \item \textbf{Handling Floating-Point Coordinates}: Although the problem specifies integer coordinates, ensure that the implementation can handle floating-point numbers if needed in extended scenarios.
    \index{Floating-Point Coordinates}
    
    \item \textbf{Avoiding Floating-Point Precision Issues}: Since all coordinates are integers, floating-point precision is not a concern, simplifying the implementation.
    \index{Floating-Point Precision}
    
    \item \textbf{Implementing Unit Tests}: Develop unit tests for each logical condition to ensure that all scenarios are correctly handled.
    \index{Unit Tests}
    
    \item \textbf{Error Handling}: Incorporate error handling to manage invalid inputs gracefully.
    \index{Error Handling}
\end{itemize}

\section*{Conclusion}

The \textbf{Rectangle Overlap} problem exemplifies the application of fundamental geometric principles and conditional logic to solve spatial challenges efficiently. By leveraging simple coordinate comparisons, the algorithm achieves optimal time and space complexities, making it highly suitable for real-time applications such as collision detection in gaming, layout planning in graphics, and spatial data analysis. Understanding and implementing such techniques not only enhances problem-solving skills but also provides a foundation for tackling more complex Computational Geometry problems involving varied geometric shapes and interactions.

\printindex

% \input{sections/rectangle_overlap}
% \input{sections/rectangle_area}
% \input{sections/k_closest_points_to_origin}
% \input{sections/the_skyline_problem}
% % filename: rectangle_area.tex

\problemsection{Rectangle Area}
\label{chap:Rectangle_Area}
\marginnote{\href{https://leetcode.com/problems/rectangle-area/}{[LeetCode Link]}\index{LeetCode}}
\marginnote{\href{https://www.geeksforgeeks.org/find-area-two-overlapping-rectangles/}{[GeeksForGeeks Link]}\index{GeeksForGeeks}}
\marginnote{\href{https://www.interviewbit.com/problems/rectangle-area/}{[InterviewBit Link]}\index{InterviewBit}}
\marginnote{\href{https://app.codesignal.com/challenges/rectangle-area}{[CodeSignal Link]}\index{CodeSignal}}
\marginnote{\href{https://www.codewars.com/kata/rectangle-area/train/python}{[Codewars Link]}\index{Codewars}}

The \textbf{Rectangle Area} problem is a classic Computational Geometry challenge that involves calculating the total area covered by two axis-aligned rectangles in a 2D plane. This problem tests one's ability to perform geometric calculations, handle overlapping scenarios, and implement efficient algorithms. Mastery of this problem is essential for applications in computer graphics, spatial analysis, and computational modeling.

\section*{Problem Statement}

Given two axis-aligned rectangles in a 2D plane, compute the total area covered by the two rectangles. The area covered by the overlapping region should be counted only once.

Each rectangle is represented as a list of four integers \([x1, y1, x2, y2]\), where \((x1, y1)\) are the coordinates of the bottom-left corner, and \((x2, y2)\) are the coordinates of the top-right corner.

\textbf{Function signature in Python:}
\begin{lstlisting}[language=Python]
def computeArea(A: List[int], B: List[int]) -> int:
\end{lstlisting}

\section*{Examples}

\textbf{Example 1:}

\begin{verbatim}
Input: A = [-3,0,3,4], B = [0,-1,9,2]
Output: 45
Explanation:
Area of A = (3 - (-3)) * (4 - 0) = 6 * 4 = 24
Area of B = (9 - 0) * (2 - (-1)) = 9 * 3 = 27
Overlapping Area = (3 - 0) * (2 - 0) = 3 * 2 = 6
Total Area = 24 + 27 - 6 = 45
\end{verbatim}

\textbf{Example 2:}

\begin{verbatim}
Input: A = [0,0,0,0], B = [0,0,0,0]
Output: 0
Explanation:
Both rectangles are degenerate points with zero area.
\end{verbatim}

\textbf{Example 3:}

\begin{verbatim}
Input: A = [0,0,2,2], B = [1,1,3,3]
Output: 7
Explanation:
Area of A = 4
Area of B = 4
Overlapping Area = 1
Total Area = 4 + 4 - 1 = 7
\end{verbatim}

\textbf{Example 4:}

\begin{verbatim}
Input: A = [0,0,1,1], B = [1,0,2,1]
Output: 2
Explanation:
Rectangles touch at the edge but do not overlap.
Area of A = 1
Area of B = 1
Overlapping Area = 0
Total Area = 1 + 1 = 2
\end{verbatim}

\textbf{Constraints:}

\begin{itemize}
    \item All coordinates are integers in the range \([-10^9, 10^9]\).
    \item For each rectangle, \(x1 < x2\) and \(y1 < y2\).
\end{itemize}

LeetCode link: \href{https://leetcode.com/problems/rectangle-area/}{Rectangle Area}\index{LeetCode}

\section*{Algorithmic Approach}

To compute the total area covered by two axis-aligned rectangles, we can follow these steps:

1. **Calculate Individual Areas:**
   - Compute the area of the first rectangle.
   - Compute the area of the second rectangle.

2. **Determine Overlapping Area:**
   - Calculate the coordinates of the overlapping rectangle, if any.
   - If the rectangles overlap, compute the area of the overlapping region.

3. **Compute Total Area:**
   - Sum the individual areas and subtract the overlapping area to avoid double-counting.

\marginnote{This approach ensures accurate area calculation by handling overlapping regions appropriately.}

\section*{Complexities}

\begin{itemize}
    \item \textbf{Time Complexity:} \(O(1)\). The algorithm performs a constant number of calculations.
    
    \item \textbf{Space Complexity:} \(O(1)\). Only a fixed amount of extra space is used for variables.
\end{itemize}

\section*{Python Implementation}

\marginnote{Implementing the area calculation with overlap consideration ensures an accurate and efficient solution.}

Below is the complete Python code implementing the \texttt{computeArea} function:

\begin{fullwidth}
\begin{lstlisting}[language=Python]
from typing import List

class Solution:
    def computeArea(self, A: List[int], B: List[int]) -> int:
        # Calculate area of rectangle A
        areaA = (A[2] - A[0]) * (A[3] - A[1])
        
        # Calculate area of rectangle B
        areaB = (B[2] - B[0]) * (B[3] - B[1])
        
        # Determine overlap coordinates
        overlap_x1 = max(A[0], B[0])
        overlap_y1 = max(A[1], B[1])
        overlap_x2 = min(A[2], B[2])
        overlap_y2 = min(A[3], B[3])
        
        # Calculate overlapping area
        overlap_width = overlap_x2 - overlap_x1
        overlap_height = overlap_y2 - overlap_y1
        overlap_area = 0
        if overlap_width > 0 and overlap_height > 0:
            overlap_area = overlap_width * overlap_height
        
        # Total area is sum of individual areas minus overlapping area
        total_area = areaA + areaB - overlap_area
        return total_area

# Example usage:
solution = Solution()
print(solution.computeArea([-3,0,3,4], [0,-1,9,2]))  # Output: 45
print(solution.computeArea([0,0,0,0], [0,0,0,0]))    # Output: 0
print(solution.computeArea([0,0,2,2], [1,1,3,3]))    # Output: 7
print(solution.computeArea([0,0,1,1], [1,0,2,1]))    # Output: 2
\end{lstlisting}
\end{fullwidth}

This implementation accurately computes the total area covered by two rectangles by accounting for any overlapping regions. It ensures that the overlapping area is not double-counted.

\section*{Explanation}

The \texttt{computeArea} function calculates the combined area of two axis-aligned rectangles by following these steps:

\subsection*{1. Calculate Individual Areas}

\begin{itemize}
    \item **Rectangle A:**
    \begin{itemize}
        \item Width: \(A[2] - A[0]\)
        \item Height: \(A[3] - A[1]\)
        \item Area: Width \(\times\) Height
    \end{itemize}
    
    \item **Rectangle B:**
    \begin{itemize}
        \item Width: \(B[2] - B[0]\)
        \item Height: \(B[3] - B[1]\)
        \item Area: Width \(\times\) Height
    \end{itemize}
\end{itemize}

\subsection*{2. Determine Overlapping Area}

\begin{itemize}
    \item **Overlap Coordinates:**
    \begin{itemize}
        \item Left (x-coordinate): \(\text{max}(A[0], B[0])\)
        \item Bottom (y-coordinate): \(\text{max}(A[1], B[1])\)
        \item Right (x-coordinate): \(\text{min}(A[2], B[2])\)
        \item Top (y-coordinate): \(\text{min}(A[3], B[3])\)
    \end{itemize}
    
    \item **Overlap Dimensions:**
    \begin{itemize}
        \item Width: \(\text{overlap\_x2} - \text{overlap\_x1}\)
        \item Height: \(\text{overlap\_y2} - \text{overlap\_y1}\)
    \end{itemize}
    
    \item **Overlap Area:**
    \begin{itemize}
        \item If both width and height are positive, the rectangles overlap, and the overlapping area is their product.
        \item Otherwise, there is no overlap, and the overlapping area is zero.
    \end{itemize}
\end{itemize}

\subsection*{3. Compute Total Area}

\begin{itemize}
    \item Total Area = Area of Rectangle A + Area of Rectangle B - Overlapping Area
\end{itemize}

\subsection*{4. Example Walkthrough}

Consider the first example:
\begin{verbatim}
Input: A = [-3,0,3,4], B = [0,-1,9,2]
Output: 45
\end{verbatim}

\begin{enumerate}
    \item **Calculate Areas:**
    \begin{itemize}
        \item Area of A = (3 - (-3)) * (4 - 0) = 6 * 4 = 24
        \item Area of B = (9 - 0) * (2 - (-1)) = 9 * 3 = 27
    \end{itemize}
    
    \item **Determine Overlap:**
    \begin{itemize}
        \item overlap\_x1 = max(-3, 0) = 0
        \item overlap\_y1 = max(0, -1) = 0
        \item overlap\_x2 = min(3, 9) = 3
        \item overlap\_y2 = min(4, 2) = 2
        \item overlap\_width = 3 - 0 = 3
        \item overlap\_height = 2 - 0 = 2
        \item overlap\_area = 3 * 2 = 6
    \end{itemize}
    
    \item **Compute Total Area:**
    \begin{itemize}
        \item Total Area = 24 + 27 - 6 = 45
    \end{itemize}
\end{enumerate}

Thus, the function correctly returns \texttt{45}.

\section*{Why This Approach}

This approach is chosen for its straightforwardness and optimal efficiency. By directly calculating the individual areas and intelligently handling the overlapping region, the algorithm ensures accurate results without unnecessary computations. Its constant time complexity makes it highly efficient, even for large coordinate values.

\section*{Alternative Approaches}

\subsection*{1. Using Intersection Dimensions}

Instead of separately calculating areas, directly compute the dimensions of the overlapping region and subtract it from the sum of individual areas.

\begin{lstlisting}[language=Python]
def computeArea(A: List[int], B: List[int]) -> int:
    # Sum of individual areas
    area = (A[2] - A[0]) * (A[3] - A[1]) + (B[2] - B[0]) * (B[3] - B[1])
    
    # Overlapping area
    overlap_width = min(A[2], B[2]) - max(A[0], B[0])
    overlap_height = min(A[3], B[3]) - max(A[1], B[1])
    
    if overlap_width > 0 and overlap_height > 0:
        area -= overlap_width * overlap_height
    
    return area
\end{lstlisting}

\subsection*{2. Using Geometry Libraries}

Leverage computational geometry libraries to handle area calculations and overlapping detections.

\begin{lstlisting}[language=Python]
from shapely.geometry import box

def computeArea(A: List[int], B: List[int]) -> int:
    rect1 = box(A[0], A[1], A[2], A[3])
    rect2 = box(B[0], B[1], B[2], B[3])
    intersection = rect1.intersection(rect2)
    return int(rect1.area + rect2.area - intersection.area)
\end{lstlisting}

\textbf{Note}: This approach requires the \texttt{shapely} library and is more suitable for complex geometric operations.

\section*{Similar Problems to This One}

Several problems involve calculating areas, handling geometric overlaps, and spatial reasoning, utilizing similar algorithmic strategies:

\begin{itemize}
    \item \textbf{Rectangle Overlap}: Determine if two rectangles overlap.
    \item \textbf{Circle Area Overlap}: Calculate the overlapping area between two circles.
    \item \textbf{Polygon Area}: Compute the area of a given polygon.
    \item \textbf{Union of Rectangles}: Calculate the total area covered by multiple rectangles, accounting for overlaps.
    \item \textbf{Intersection of Lines}: Find the intersection point of two lines.
    \item \textbf{Closest Pair of Points}: Find the closest pair of points in a set.
    \item \textbf{Convex Hull}: Compute the convex hull of a set of points.
    \item \textbf{Point Inside Polygon}: Determine if a point lies inside a given polygon.
\end{itemize}

These problems reinforce concepts of geometric calculations, area computations, and efficient algorithm design in various contexts.

\section*{Things to Keep in Mind and Tricks}

When tackling the \textbf{Rectangle Area} problem, consider the following tips and best practices to enhance efficiency and correctness:

\begin{itemize}
    \item \textbf{Understand Geometric Relationships}: Grasp the positional relationships between rectangles to simplify area calculations.
    \index{Geometric Relationships}
    
    \item \textbf{Leverage Coordinate Comparisons}: Use direct comparisons of rectangle coordinates to determine overlapping regions.
    \index{Coordinate Comparisons}
    
    \item \textbf{Handle Overlapping Scenarios}: Accurately calculate the overlapping area to avoid double-counting.
    \index{Overlapping Scenarios}
    
    \item \textbf{Optimize for Efficiency}: Aim for a constant time \(O(1)\) solution by avoiding unnecessary computations or iterations.
    \index{Efficiency Optimization}
    
    \item \textbf{Avoid Floating-Point Precision Issues}: Since all coordinates are integers, floating-point precision is not a concern, simplifying the implementation.
    \index{Floating-Point Precision}
    
    \item \textbf{Use Helper Functions}: Create helper functions to encapsulate repetitive tasks, such as calculating overlap dimensions or areas.
    \index{Helper Functions}
    
    \item \textbf{Code Readability}: Maintain clear and readable code through meaningful variable names and structured logic.
    \index{Code Readability}
    
    \item \textbf{Test Extensively}: Implement a wide range of test cases, including overlapping, non-overlapping, and edge-touching rectangles, to ensure robustness.
    \index{Extensive Testing}
    
    \item \textbf{Understand Axis-Aligned Constraints}: Recognize that axis-aligned rectangles simplify area calculations compared to rotated rectangles.
    \index{Axis-Aligned Constraints}
    
    \item \textbf{Simplify Logical Conditions}: Combine multiple conditions logically to streamline the area calculation process.
    \index{Logical Conditions}
    
    \item \textbf{Use Absolute Values}: When calculating differences, ensure that the dimensions are positive by using absolute values or proper ordering.
    \index{Absolute Values}
    
    \item \textbf{Consider Edge Cases}: Handle cases where rectangles have zero area or touch at edges/corners without overlapping.
    \index{Edge Cases}
\end{itemize}

\section*{Corner and Special Cases to Test When Writing the Code}

When implementing the solution for the \textbf{Rectangle Area} problem, it is crucial to consider and rigorously test various edge cases to ensure robustness and correctness:

\begin{itemize}
    \item \textbf{No Overlap}: Rectangles are completely separate.
    \index{No Overlap}
    
    \item \textbf{Partial Overlap}: Rectangles overlap in one or more regions.
    \index{Partial Overlap}
    
    \item \textbf{Edge Touching}: Rectangles touch exactly at one edge without overlapping.
    \index{Edge Touching}
    
    \item \textbf{Corner Touching}: Rectangles touch exactly at one corner without overlapping.
    \index{Corner Touching}
    
    \item \textbf{One Rectangle Inside Another}: One rectangle is entirely within the other.
    \index{Rectangle Inside}
    
    \item \textbf{Identical Rectangles}: Both rectangles have the same coordinates.
    \index{Identical Rectangles}
    
    \item \textbf{Degenerate Rectangles}: Rectangles with zero area (e.g., \(x1 = x2\) or \(y1 = y2\)).
    \index{Degenerate Rectangles}
    
    \item \textbf{Large Coordinates}: Rectangles with very large coordinate values to test performance and integer handling.
    \index{Large Coordinates}
    
    \item \textbf{Negative Coordinates}: Rectangles positioned in negative coordinate space.
    \index{Negative Coordinates}
    
    \item \textbf{Mixed Overlapping Scenarios}: Combinations of the above cases to ensure comprehensive coverage.
    \index{Mixed Overlapping Scenarios}
    
    \item \textbf{Minimum and Maximum Bounds}: Rectangles at the minimum and maximum limits of the coordinate range.
    \index{Minimum and Maximum Bounds}
    
    \item \textbf{Sequential Rectangles}: Multiple rectangles placed sequentially without overlapping.
    \index{Sequential Rectangles}
    
    \item \textbf{Multiple Overlaps}: Scenarios where more than two rectangles overlap in different regions.
    \index{Multiple Overlaps}
\end{itemize}

\section*{Implementation Considerations}

When implementing the \texttt{computeArea} function, keep in mind the following considerations to ensure robustness and efficiency:

\begin{itemize}
    \item \textbf{Data Type Selection}: Use appropriate data types that can handle large input values without overflow or underflow.
    \index{Data Type Selection}
    
    \item \textbf{Optimizing Comparisons}: Structure logical conditions to efficiently determine overlap dimensions.
    \index{Optimizing Comparisons}
    
    \item \textbf{Handling Large Inputs}: Design the algorithm to efficiently handle large input sizes without significant performance degradation.
    \index{Handling Large Inputs}
    
    \item \textbf{Language-Specific Constraints}: Be aware of how the programming language handles large integers and arithmetic operations.
    \index{Language-Specific Constraints}
    
    \item \textbf{Avoiding Redundant Calculations}: Ensure that each calculation contributes towards determining the final area without unnecessary repetitions.
    \index{Avoiding Redundant Calculations}
    
    \item \textbf{Code Readability and Documentation}: Maintain clear and readable code through meaningful variable names and comprehensive comments to facilitate understanding and maintenance.
    \index{Code Readability}
    
    \item \textbf{Edge Case Handling}: Implement checks for edge cases to prevent incorrect results or runtime errors.
    \index{Edge Case Handling}
    
    \item \textbf{Testing and Validation}: Develop a comprehensive suite of test cases that cover all possible scenarios, including edge cases, to validate the correctness and efficiency of the implementation.
    \index{Testing and Validation}
    
    \item \textbf{Scalability}: Design the algorithm to scale efficiently with increasing input sizes, maintaining performance and resource utilization.
    \index{Scalability}
    
    \item \textbf{Using Helper Functions}: Consider creating helper functions for repetitive tasks, such as calculating overlap dimensions, to enhance modularity and reusability.
    \index{Helper Functions}
    
    \item \textbf{Consistent Naming Conventions}: Use consistent and descriptive naming conventions for variables to improve code clarity.
    \index{Naming Conventions}
    
    \item \textbf{Implementing Unit Tests}: Develop unit tests for each logical condition to ensure that all scenarios are correctly handled.
    \index{Unit Tests}
    
    \item \textbf{Error Handling}: Incorporate error handling to manage invalid inputs gracefully.
    \index{Error Handling}
\end{itemize}

\section*{Conclusion}

The \textbf{Rectangle Area} problem showcases the application of fundamental geometric principles and efficient algorithm design to compute spatial properties accurately. By systematically calculating individual areas and intelligently handling overlapping regions, the algorithm ensures precise results without redundant computations. Understanding and implementing such techniques not only enhances problem-solving skills but also provides a foundation for tackling more complex Computational Geometry challenges involving multiple geometric entities and intricate spatial relationships.

\printindex

% \input{sections/rectangle_overlap}
% \input{sections/rectangle_area}
% \input{sections/k_closest_points_to_origin}
% \input{sections/the_skyline_problem}
% % filename: k_closest_points_to_origin.tex

\problemsection{K Closest Points to Origin}
\label{chap:K_Closest_Points_to_Origin}
\marginnote{\href{https://leetcode.com/problems/k-closest-points-to-origin/}{[LeetCode Link]}\index{LeetCode}}
\marginnote{\href{https://www.geeksforgeeks.org/find-k-closest-points-origin/}{[GeeksForGeeks Link]}\index{GeeksForGeeks}}
\marginnote{\href{https://www.interviewbit.com/problems/k-closest-points/}{[InterviewBit Link]}\index{InterviewBit}}
\marginnote{\href{https://app.codesignal.com/challenges/k-closest-points-to-origin}{[CodeSignal Link]}\index{CodeSignal}}
\marginnote{\href{https://www.codewars.com/kata/k-closest-points-to-origin/train/python}{[Codewars Link]}\index{Codewars}}

The \textbf{K Closest Points to Origin} problem is a popular algorithmic challenge in Computational Geometry that involves identifying the \(k\) points closest to the origin in a 2D plane. This problem tests one's ability to apply efficient sorting and selection algorithms, understand distance computations, and optimize for performance. Mastery of this problem is essential for applications in spatial data analysis, nearest neighbor searches, and clustering algorithms.

\section*{Problem Statement}

Given an array of points where each point is represented as \([x, y]\) in the 2D plane, and an integer \(k\), return the \(k\) closest points to the origin \((0, 0)\).

The distance between two points \((x_1, y_1)\) and \((x_2, y_2)\) is the Euclidean distance \(\sqrt{(x_1 - x_2)^2 + (y_1 - y_2)^2}\). The origin is \((0, 0)\).

\textbf{Function signature in Python:}
\begin{lstlisting}[language=Python]
def kClosest(points: List[List[int]], K: int) -> List[List[int]]:
\end{lstlisting}

\section*{Examples}

\textbf{Example 1:}

\begin{verbatim}
Input: points = [[1,3],[-2,2]], K = 1
Output: [[-2,2]]
Explanation: 
The distance between (1, 3) and the origin is sqrt(10).
The distance between (-2, 2) and the origin is sqrt(8).
Since sqrt(8) < sqrt(10), (-2, 2) is closer to the origin.
\end{verbatim}

\textbf{Example 2:}

\begin{verbatim}
Input: points = [[3,3],[5,-1],[-2,4]], K = 2
Output: [[3,3],[-2,4]]
Explanation: 
The distances are sqrt(18), sqrt(26), and sqrt(20) respectively.
The two closest points are [3,3] and [-2,4].
\end{verbatim}

\textbf{Example 3:}

\begin{verbatim}
Input: points = [[0,1],[1,0]], K = 2
Output: [[0,1],[1,0]]
Explanation: 
Both points are equally close to the origin.
\end{verbatim}

\textbf{Example 4:}

\begin{verbatim}
Input: points = [[1,0],[0,1]], K = 1
Output: [[1,0]]
Explanation: 
Both points are equally close; returning any one is acceptable.
\end{verbatim}

\textbf{Constraints:}

\begin{itemize}
    \item \(1 \leq K \leq \text{points.length} \leq 10^4\)
    \item \(-10^4 < x_i, y_i < 10^4\)
\end{itemize}

LeetCode link: \href{https://leetcode.com/problems/k-closest-points-to-origin/}{K Closest Points to Origin}\index{LeetCode}

\section*{Algorithmic Approach}

To identify the \(k\) closest points to the origin, several algorithmic strategies can be employed. The most efficient methods aim to reduce the time complexity by avoiding the need to sort the entire list of points.

\subsection*{1. Sorting Based on Distance}

Calculate the Euclidean distance of each point from the origin and sort the points based on these distances. Select the first \(k\) points from the sorted list.

\begin{enumerate}
    \item Compute the distance for each point using the formula \(distance = x^2 + y^2\).
    \item Sort the points based on the computed distances.
    \item Return the first \(k\) points from the sorted list.
\end{enumerate}

\subsection*{2. Max Heap (Priority Queue)}

Use a max heap to maintain the \(k\) closest points. Iterate through each point, add it to the heap, and if the heap size exceeds \(k\), remove the farthest point.

\begin{enumerate}
    \item Initialize a max heap.
    \item For each point, compute its distance and add it to the heap.
    \item If the heap size exceeds \(k\), remove the point with the largest distance.
    \item After processing all points, the heap contains the \(k\) closest points.
\end{enumerate}

\subsection*{3. QuickSelect (Quick Sort Partitioning)}

Utilize the QuickSelect algorithm to find the \(k\) closest points without fully sorting the list.

\begin{enumerate}
    \item Choose a pivot point and partition the list based on distances relative to the pivot.
    \item Recursively apply QuickSelect to the partition containing the \(k\) closest points.
    \item Once the \(k\) closest points are identified, return them.
\end{enumerate}

\marginnote{QuickSelect offers an average time complexity of \(O(n)\), making it highly efficient for large datasets.}

\section*{Complexities}

\begin{itemize}
    \item \textbf{Sorting Based on Distance:}
    \begin{itemize}
        \item \textbf{Time Complexity:} \(O(n \log n)\)
        \item \textbf{Space Complexity:} \(O(n)\)
    \end{itemize}
    
    \item \textbf{Max Heap (Priority Queue):}
    \begin{itemize}
        \item \textbf{Time Complexity:} \(O(n \log k)\)
        \item \textbf{Space Complexity:} \(O(k)\)
    \end{itemize}
    
    \item \textbf{QuickSelect (Quick Sort Partitioning):}
    \begin{itemize}
        \item \textbf{Time Complexity:} Average case \(O(n)\), worst case \(O(n^2)\)
        \item \textbf{Space Complexity:} \(O(1)\) (in-place)
    \end{itemize}
\end{itemize}

\section*{Python Implementation}

\marginnote{Implementing QuickSelect provides an optimal average-case solution with linear time complexity.}

Below is the complete Python code implementing the \texttt{kClosest} function using the QuickSelect approach:

\begin{fullwidth}
\begin{lstlisting}[language=Python]
from typing import List
import random

class Solution:
    def kClosest(self, points: List[List[int]], K: int) -> List[List[int]]:
        def quickselect(left, right, K_smallest):
            if left == right:
                return
            
            # Select a random pivot_index
            pivot_index = random.randint(left, right)
            
            # Partition the array
            pivot_index = partition(left, right, pivot_index)
            
            # The pivot is in its final sorted position
            if K_smallest == pivot_index:
                return
            elif K_smallest < pivot_index:
                quickselect(left, pivot_index - 1, K_smallest)
            else:
                quickselect(pivot_index + 1, right, K_smallest)
        
        def partition(left, right, pivot_index):
            pivot_distance = distance(points[pivot_index])
            # Move pivot to end
            points[pivot_index], points[right] = points[right], points[pivot_index]
            store_index = left
            for i in range(left, right):
                if distance(points[i]) < pivot_distance:
                    points[store_index], points[i] = points[i], points[store_index]
                    store_index += 1
            # Move pivot to its final place
            points[right], points[store_index] = points[store_index], points[right]
            return store_index
        
        def distance(point):
            return point[0] ** 2 + point[1] ** 2
        
        n = len(points)
        quickselect(0, n - 1, K)
        return points[:K]

# Example usage:
solution = Solution()
print(solution.kClosest([[1,3],[-2,2]], 1))            # Output: [[-2,2]]
print(solution.kClosest([[3,3],[5,-1],[-2,4]], 2))     # Output: [[3,3],[-2,4]]
print(solution.kClosest([[0,1],[1,0]], 2))             # Output: [[0,1],[1,0]]
print(solution.kClosest([[1,0],[0,1]], 1))             # Output: [[1,0]] or [[0,1]]
\end{lstlisting}
\end{fullwidth}

This implementation uses the QuickSelect algorithm to efficiently find the \(k\) closest points to the origin without fully sorting the entire list. It ensures optimal performance even with large datasets.

\section*{Explanation}

The \texttt{kClosest} function identifies the \(k\) closest points to the origin using the QuickSelect algorithm. Here's a detailed breakdown of the implementation:

\subsection*{1. Distance Calculation}

\begin{itemize}
    \item The Euclidean distance is calculated as \(distance = x^2 + y^2\). Since we only need relative distances for comparison, the square root is omitted for efficiency.
\end{itemize}

\subsection*{2. QuickSelect Algorithm}

\begin{itemize}
    \item **Pivot Selection:**
    \begin{itemize}
        \item A random pivot is chosen to enhance the average-case performance.
    \end{itemize}
    
    \item **Partitioning:**
    \begin{itemize}
        \item The array is partitioned such that points with distances less than the pivot are moved to the left, and others to the right.
        \item The pivot is placed in its correct sorted position.
    \end{itemize}
    
    \item **Recursive Selection:**
    \begin{itemize}
        \item If the pivot's position matches \(K\), the selection is complete.
        \item Otherwise, recursively apply QuickSelect to the relevant partition.
    \end{itemize}
\end{itemize}

\subsection*{3. Final Selection}

\begin{itemize}
    \item After partitioning, the first \(K\) points in the list are the \(k\) closest points to the origin.
\end{itemize}

\subsection*{4. Example Walkthrough}

Consider the first example:
\begin{verbatim}
Input: points = [[1,3],[-2,2]], K = 1
Output: [[-2,2]]
\end{verbatim}

\begin{enumerate}
    \item **Calculate Distances:**
    \begin{itemize}
        \item [1,3] : \(1^2 + 3^2 = 10\)
        \item [-2,2] : \((-2)^2 + 2^2 = 8\)
    \end{itemize}
    
    \item **QuickSelect Process:**
    \begin{itemize}
        \item Choose a pivot, say [1,3] with distance 10.
        \item Compare and rearrange:
        \begin{itemize}
            \item [-2,2] has a smaller distance (8) and is moved to the left.
        \end{itemize}
        \item After partitioning, the list becomes [[-2,2], [1,3]].
        \item Since \(K = 1\), return the first point: [[-2,2]].
    \end{itemize}
\end{enumerate}

Thus, the function correctly identifies \([-2,2]\) as the closest point to the origin.

\section*{Why This Approach}

The QuickSelect algorithm is chosen for its average-case linear time complexity \(O(n)\), making it highly efficient for large datasets compared to sorting-based methods with \(O(n \log n)\) time complexity. By avoiding the need to sort the entire list, QuickSelect provides an optimal solution for finding the \(k\) closest points.

\section*{Alternative Approaches}

\subsection*{1. Sorting Based on Distance}

Sort all points based on their distances from the origin and select the first \(k\) points.

\begin{lstlisting}[language=Python]
class Solution:
    def kClosest(self, points: List[List[int]], K: int) -> List[List[int]]:
        points.sort(key=lambda P: P[0]**2 + P[1]**2)
        return points[:K]
\end{lstlisting}

\textbf{Complexities:}
\begin{itemize}
    \item \textbf{Time Complexity:} \(O(n \log n)\)
    \item \textbf{Space Complexity:} \(O(1)\)
\end{itemize}

\subsection*{2. Max Heap (Priority Queue)}

Use a max heap to maintain the \(k\) closest points.

\begin{lstlisting}[language=Python]
import heapq

class Solution:
    def kClosest(self, points: List[List[int]], K: int) -> List[List[int]]:
        heap = []
        for (x, y) in points:
            dist = -(x**2 + y**2)  # Max heap using negative distances
            heapq.heappush(heap, (dist, [x, y]))
            if len(heap) > K:
                heapq.heappop(heap)
        return [item[1] for item in heap]
\end{lstlisting}

\textbf{Complexities:}
\begin{itemize}
    \item \textbf{Time Complexity:} \(O(n \log k)\)
    \item \textbf{Space Complexity:} \(O(k)\)
\end{itemize}

\subsection*{3. Using Built-In Functions}

Leverage built-in functions for distance calculation and selection.

\begin{lstlisting}[language=Python]
import math

class Solution:
    def kClosest(self, points: List[List[int]], K: int) -> List[List[int]]:
        points.sort(key=lambda P: math.sqrt(P[0]**2 + P[1]**2))
        return points[:K]
\end{lstlisting}

\textbf{Note}: This method is similar to the sorting approach but uses the actual Euclidean distance.

\section*{Similar Problems to This One}

Several problems involve nearest neighbor searches, spatial data analysis, and efficient selection algorithms, utilizing similar algorithmic strategies:

\begin{itemize}
    \item \textbf{Closest Pair of Points}: Find the closest pair of points in a set.
    \item \textbf{Top K Frequent Elements}: Identify the most frequent elements in a dataset.
    \item \textbf{Kth Largest Element in an Array}: Find the \(k\)-th largest element in an unsorted array.
    \item \textbf{Sliding Window Maximum}: Find the maximum in each sliding window of size \(k\) over an array.
    \item \textbf{Merge K Sorted Lists}: Merge multiple sorted lists into a single sorted list.
    \item \textbf{Find Median from Data Stream}: Continuously find the median of a stream of numbers.
    \item \textbf{Top K Closest Stars}: Find the \(k\) closest stars to Earth based on their distances.
\end{itemize}

These problems reinforce concepts of efficient selection, heap usage, and distance computations in various contexts.

\section*{Things to Keep in Mind and Tricks}

When solving the \textbf{K Closest Points to Origin} problem, consider the following tips and best practices to enhance efficiency and correctness:

\begin{itemize}
    \item \textbf{Understand Distance Calculations}: Grasp the Euclidean distance formula and recognize that the square root can be omitted for comparison purposes.
    \index{Distance Calculations}
    
    \item \textbf{Leverage Efficient Algorithms}: Use QuickSelect or heap-based methods to optimize time complexity, especially for large datasets.
    \index{Efficient Algorithms}
    
    \item \textbf{Handle Ties Appropriately}: Decide how to handle points with identical distances when \(k\) is less than the number of such points.
    \index{Handling Ties}
    
    \item \textbf{Optimize Space Usage}: Choose algorithms that minimize additional space, such as in-place QuickSelect.
    \index{Space Optimization}
    
    \item \textbf{Use Appropriate Data Structures}: Utilize heaps, lists, and helper functions effectively to manage and process data.
    \index{Data Structures}
    
    \item \textbf{Implement Helper Functions}: Create helper functions for distance calculation and partitioning to enhance code modularity.
    \index{Helper Functions}
    
    \item \textbf{Code Readability}: Maintain clear and readable code through meaningful variable names and structured logic.
    \index{Code Readability}
    
    \item \textbf{Test Extensively}: Implement a wide range of test cases, including edge cases like multiple points with the same distance, to ensure robustness.
    \index{Extensive Testing}
    
    \item \textbf{Understand Algorithm Trade-offs}: Recognize the trade-offs between different approaches in terms of time and space complexities.
    \index{Algorithm Trade-offs}
    
    \item \textbf{Use Built-In Sorting Functions}: When using sorting-based approaches, leverage built-in functions for efficiency and simplicity.
    \index{Built-In Sorting}
    
    \item \textbf{Avoid Redundant Calculations}: Ensure that distance calculations are performed only when necessary to optimize performance.
    \index{Avoiding Redundant Calculations}
    
    \item \textbf{Language-Specific Features}: Utilize language-specific features or libraries that can simplify implementation, such as heapq in Python.
    \index{Language-Specific Features}
\end{itemize}

\section*{Corner and Special Cases to Test When Writing the Code}

When implementing the solution for the \textbf{K Closest Points to Origin} problem, it is crucial to consider and rigorously test various edge cases to ensure robustness and correctness:

\begin{itemize}
    \item \textbf{Multiple Points with Same Distance}: Ensure that the algorithm handles multiple points having the same distance from the origin.
    \index{Same Distance Points}
    
    \item \textbf{Points at Origin}: Include points that are exactly at the origin \((0,0)\).
    \index{Points at Origin}
    
    \item \textbf{Negative Coordinates}: Ensure that the algorithm correctly computes distances for points with negative \(x\) or \(y\) coordinates.
    \index{Negative Coordinates}
    
    \item \textbf{Large Coordinates}: Test with points having very large or very small coordinate values to verify integer handling.
    \index{Large Coordinates}
    
    \item \textbf{K Equals Number of Points}: When \(K\) is equal to the number of points, the algorithm should return all points.
    \index{K Equals Number of Points}
    
    \item \textbf{K is One}: Test with \(K = 1\) to ensure the closest point is correctly identified.
    \index{K is One}
    
    \item \textbf{All Points Same}: All points have the same coordinates.
    \index{All Points Same}
    
    \item \textbf{K is Zero}: Although \(K\) is defined to be at least 1, ensure that the algorithm gracefully handles \(K = 0\) if allowed.
    \index{K is Zero}
    
    \item \textbf{Single Point}: Only one point is provided, and \(K = 1\).
    \index{Single Point}
    
    \item \textbf{Mixed Coordinates}: Points with a mix of positive and negative coordinates.
    \index{Mixed Coordinates}
    
    \item \textbf{Points with Zero Distance}: Multiple points at the origin.
    \index{Zero Distance Points}
    
    \item \textbf{Sparse and Dense Points}: Densely packed points and sparsely distributed points.
    \index{Sparse and Dense Points}
    
    \item \textbf{Duplicate Points}: Multiple identical points in the input list.
    \index{Duplicate Points}
    
    \item \textbf{K Greater Than Number of Unique Points}: Ensure that the algorithm handles cases where \(K\) exceeds the number of unique points if applicable.
    \index{K Greater Than Unique Points}
\end{itemize}

\section*{Implementation Considerations}

When implementing the \texttt{kClosest} function, keep in mind the following considerations to ensure robustness and efficiency:

\begin{itemize}
    \item \textbf{Data Type Selection}: Use appropriate data types that can handle large input values without overflow or precision loss.
    \index{Data Type Selection}
    
    \item \textbf{Optimizing Distance Calculations}: Avoid calculating the square root since it is unnecessary for comparison purposes.
    \index{Optimizing Distance Calculations}
    
    \item \textbf{Choosing the Right Algorithm}: Select an algorithm based on the size of the input and the value of \(K\) to optimize time and space complexities.
    \index{Choosing the Right Algorithm}
    
    \item \textbf{Language-Specific Libraries}: Utilize language-specific libraries and functions (e.g., \texttt{heapq} in Python) to simplify implementation and enhance performance.
    \index{Language-Specific Libraries}
    
    \item \textbf{Avoiding Redundant Calculations}: Ensure that each point's distance is calculated only once to optimize performance.
    \index{Avoiding Redundant Calculations}
    
    \item \textbf{Implementing Helper Functions}: Create helper functions for tasks like distance calculation and partitioning to enhance modularity and readability.
    \index{Helper Functions}
    
    \item \textbf{Edge Case Handling}: Implement checks for edge cases to prevent incorrect results or runtime errors.
    \index{Edge Case Handling}
    
    \item \textbf{Testing and Validation}: Develop a comprehensive suite of test cases that cover all possible scenarios, including edge cases, to validate the correctness and efficiency of the implementation.
    \index{Testing and Validation}
    
    \item \textbf{Scalability}: Design the algorithm to scale efficiently with increasing input sizes, maintaining performance and resource utilization.
    \index{Scalability}
    
    \item \textbf{Consistent Naming Conventions}: Use consistent and descriptive naming conventions for variables and functions to improve code clarity.
    \index{Naming Conventions}
    
    \item \textbf{Memory Management}: Ensure that the algorithm manages memory efficiently, especially when dealing with large datasets.
    \index{Memory Management}
    
    \item \textbf{Avoiding Stack Overflow}: If implementing recursive approaches, be mindful of recursion limits and potential stack overflow issues.
    \index{Avoiding Stack Overflow}
    
    \item \textbf{Implementing Iterative Solutions}: Prefer iterative solutions when recursion may lead to increased space complexity or stack overflow.
    \index{Implementing Iterative Solutions}
\end{itemize}

\section*{Conclusion}

The \textbf{K Closest Points to Origin} problem exemplifies the application of efficient selection algorithms and geometric computations to solve spatial challenges effectively. By leveraging QuickSelect or heap-based methods, the algorithm achieves optimal time and space complexities, making it highly suitable for large datasets. Understanding and implementing such techniques not only enhances problem-solving skills but also provides a foundation for tackling more advanced Computational Geometry problems involving nearest neighbor searches, clustering, and spatial data analysis.

\printindex

% \input{sections/rectangle_overlap}
% \input{sections/rectangle_area}
% \input{sections/k_closest_points_to_origin}
% \input{sections/the_skyline_problem}
% % filename: the_skyline_problem.tex

\problemsection{The Skyline Problem}
\label{chap:The_Skyline_Problem}
\marginnote{\href{https://leetcode.com/problems/the-skyline-problem/}{[LeetCode Link]}\index{LeetCode}}
\marginnote{\href{https://www.geeksforgeeks.org/the-skyline-problem/}{[GeeksForGeeks Link]}\index{GeeksForGeeks}}
\marginnote{\href{https://www.interviewbit.com/problems/the-skyline-problem/}{[InterviewBit Link]}\index{InterviewBit}}
\marginnote{\href{https://app.codesignal.com/challenges/the-skyline-problem}{[CodeSignal Link]}\index{CodeSignal}}
\marginnote{\href{https://www.codewars.com/kata/the-skyline-problem/train/python}{[Codewars Link]}\index{Codewars}}

The \textbf{Skyline Problem} is a complex Computational Geometry challenge that involves computing the skyline formed by a collection of buildings in a 2D cityscape. Each building is represented by its left and right x-coordinates and its height. The skyline is defined by a list of "key points" where the height changes. This problem tests one's ability to handle large datasets, implement efficient sweep line algorithms, and manage event-driven processing. Mastery of this problem is essential for applications in computer graphics, urban planning simulations, and geographic information systems (GIS).

\section*{Problem Statement}

You are given a list of buildings in a cityscape. Each building is represented as a triplet \([Li, Ri, Hi]\), where \(Li\) and \(Ri\) are the x-coordinates of the left and right edges of the building, respectively, and \(Hi\) is the height of the building.

The skyline should be represented as a list of key points \([x, y]\) in sorted order by \(x\)-coordinate, where \(y\) is the height of the skyline at that point. The skyline should only include critical points where the height changes.

\textbf{Function signature in Python:}
\begin{lstlisting}[language=Python]
def getSkyline(buildings: List[List[int]]) -> List[List[int]]:
\end{lstlisting}

\section*{Examples}

\textbf{Example 1:}

\begin{verbatim}
Input: buildings = [[2,9,10], [3,7,15], [5,12,12], [15,20,10], [19,24,8]]
Output: [[2,10], [3,15], [7,12], [12,0], [15,10], [20,8], [24,0]]
Explanation:
- At x=2, the first building starts, height=10.
- At x=3, the second building starts, height=15.
- At x=7, the second building ends, the third building is still ongoing, height=12.
- At x=12, the third building ends, height drops to 0.
- At x=15, the fourth building starts, height=10.
- At x=20, the fourth building ends, the fifth building is still ongoing, height=8.
- At x=24, the fifth building ends, height drops to 0.
\end{verbatim}

\textbf{Example 2:}

\begin{verbatim}
Input: buildings = [[0,2,3], [2,5,3]]
Output: [[0,3], [5,0]]
Explanation:
- The two buildings are contiguous and have the same height, so the skyline drops to 0 at x=5.
\end{verbatim}

\textbf{Example 3:}

\begin{verbatim}
Input: buildings = [[1,3,3], [2,4,4], [5,6,1]]
Output: [[1,3], [2,4], [4,0], [5,1], [6,0]]
Explanation:
- At x=1, first building starts, height=3.
- At x=2, second building starts, height=4.
- At x=4, second building ends, height drops to 0.
- At x=5, third building starts, height=1.
- At x=6, third building ends, height drops to 0.
\end{verbatim}

\textbf{Example 4:}

\begin{verbatim}
Input: buildings = [[0,5,0]]
Output: []
Explanation:
- A building with height 0 does not contribute to the skyline.
\end{verbatim}

\textbf{Constraints:}

\begin{itemize}
    \item \(1 \leq \text{buildings.length} \leq 10^4\)
    \item \(0 \leq Li < Ri \leq 10^9\)
    \item \(0 \leq Hi \leq 10^4\)
\end{itemize}

\section*{Algorithmic Approach}

The \textbf{Sweep Line Algorithm} is an efficient method for solving the Skyline Problem. It involves processing events (building start and end points) in sorted order while maintaining a data structure (typically a max heap) to keep track of active buildings. Here's a step-by-step approach:

\subsection*{1. Event Representation}

Transform each building into two events:
\begin{itemize}
    \item **Start Event:** \((Li, -Hi)\) – Negative height indicates a building starts.
    \item **End Event:** \((Ri, Hi)\) – Positive height indicates a building ends.
\end{itemize}

Sorting the events ensures that start events are processed before end events at the same x-coordinate, and taller buildings are processed before shorter ones.

\subsection*{2. Sorting the Events}

Sort all events based on:
\begin{enumerate}
    \item **x-coordinate:** Ascending order.
    \item **Height:**
    \begin{itemize}
        \item For start events, taller buildings come first.
        \item For end events, shorter buildings come first.
    \end{itemize}
\end{enumerate}

\subsection*{3. Processing the Events}

Use a max heap to keep track of active building heights. Iterate through the sorted events:
\begin{enumerate}
    \item **Start Event:**
    \begin{itemize}
        \item Add the building's height to the heap.
    \end{itemize}
    
    \item **End Event:**
    \begin{itemize}
        \item Remove the building's height from the heap.
    \end{itemize}
    
    \item **Determine Current Max Height:**
    \begin{itemize}
        \item The current max height is the top of the heap.
    \end{itemize}
    
    \item **Update Skyline:**
    \begin{itemize}
        \item If the current max height differs from the previous max height, add a new key point \([x, current\_max\_height]\).
    \end{itemize}
\end{enumerate}

\subsection*{4. Finalizing the Skyline}

After processing all events, the accumulated key points represent the skyline.

\marginnote{The Sweep Line Algorithm efficiently handles dynamic changes in active buildings, ensuring accurate skyline construction.}

\section*{Complexities}

\begin{itemize}
    \item \textbf{Time Complexity:} \(O(n \log n)\), where \(n\) is the number of buildings. Sorting the events takes \(O(n \log n)\), and each heap operation takes \(O(\log n)\).
    
    \item \textbf{Space Complexity:} \(O(n)\), due to the storage of events and the heap.
\end{itemize}

\section*{Python Implementation}

\marginnote{Implementing the Sweep Line Algorithm with a max heap ensures an efficient and accurate solution.}

Below is the complete Python code implementing the \texttt{getSkyline} function:

\begin{fullwidth}
\begin{lstlisting}[language=Python]
from typing import List
import heapq

class Solution:
    def getSkyline(self, buildings: List[List[int]]) -> List[List[int]]:
        # Create a list of all events
        # For start events, use negative height to ensure they are processed before end events
        events = []
        for L, R, H in buildings:
            events.append((L, -H))
            events.append((R, H))
        
        # Sort the events
        # First by x-coordinate, then by height
        events.sort()
        
        # Max heap to keep track of active buildings
        heap = [0]  # Initialize with ground level
        heapq.heapify(heap)
        active_heights = {0: 1}  # Dictionary to count heights
        
        result = []
        prev_max = 0
        
        for x, h in events:
            if h < 0:
                # Start of a building, add height to heap and dictionary
                heapq.heappush(heap, h)
                active_heights[h] = active_heights.get(h, 0) + 1
            else:
                # End of a building, remove height from dictionary
                active_heights[h] -= 1
                if active_heights[h] == 0:
                    del active_heights[h]
            
            # Current max height
            while heap and active_heights.get(heap[0], 0) == 0:
                heapq.heappop(heap)
            current_max = -heap[0] if heap else 0
            
            # If the max height has changed, add to result
            if current_max != prev_max:
                result.append([x, current_max])
                prev_max = current_max
        
        return result

# Example usage:
solution = Solution()
print(solution.getSkyline([[2,9,10], [3,7,15], [5,12,12], [15,20,10], [19,24,8]]))
# Output: [[2,10], [3,15], [7,12], [12,0], [15,10], [20,8], [24,0]]

print(solution.getSkyline([[0,2,3], [2,5,3]]))
# Output: [[0,3], [5,0]]

print(solution.getSkyline([[1,3,3], [2,4,4], [5,6,1]]))
# Output: [[1,3], [2,4], [4,0], [5,1], [6,0]]

print(solution.getSkyline([[0,5,0]]))
# Output: []
\end{lstlisting}
\end{fullwidth}

This implementation efficiently constructs the skyline by processing all building events in sorted order and maintaining active building heights using a max heap. It ensures that only critical points where the skyline changes are recorded.

\section*{Explanation}

The \texttt{getSkyline} function constructs the skyline formed by a set of buildings by leveraging the Sweep Line Algorithm and a max heap to track active buildings. Here's a detailed breakdown of the implementation:

\subsection*{1. Event Representation}

\begin{itemize}
    \item Each building is transformed into two events:
    \begin{itemize}
        \item **Start Event:** \((Li, -Hi)\) – Negative height indicates the start of a building.
        \item **End Event:** \((Ri, Hi)\) – Positive height indicates the end of a building.
    \end{itemize}
\end{itemize}

\subsection*{2. Sorting the Events}

\begin{itemize}
    \item Events are sorted primarily by their x-coordinate in ascending order.
    \item For events with the same x-coordinate:
    \begin{itemize}
        \item Start events (with negative heights) are processed before end events.
        \item Taller buildings are processed before shorter ones.
    \end{itemize}
\end{itemize}

\subsection*{3. Processing the Events}

\begin{itemize}
    \item **Heap Initialization:**
    \begin{itemize}
        \item A max heap is initialized with a ground level height of 0.
        \item A dictionary \texttt{active\_heights} tracks the count of active building heights.
    \end{itemize}
    
    \item **Iterating Through Events:**
    \begin{enumerate}
        \item **Start Event:**
        \begin{itemize}
            \item Add the building's height to the heap.
            \item Increment the count of the height in \texttt{active\_heights}.
        \end{itemize}
        
        \item **End Event:**
        \begin{itemize}
            \item Decrement the count of the building's height in \texttt{active\_heights}.
            \item If the count reaches zero, remove the height from the dictionary.
        \end{itemize}
        
        \item **Determine Current Max Height:**
        \begin{itemize}
            \item Remove heights from the heap that are no longer active.
            \item The current max height is the top of the heap.
        \end{itemize}
        
        \item **Update Skyline:**
        \begin{itemize}
            \item If the current max height differs from the previous max height, add a new key point \([x, current\_max\_height]\).
        \end{itemize}
    \end{enumerate}
\end{itemize}

\subsection*{4. Finalizing the Skyline}

\begin{itemize}
    \item After processing all events, the \texttt{result} list contains the key points defining the skyline.
\end{itemize}

\subsection*{5. Example Walkthrough}

Consider the first example:
\begin{verbatim}
Input: buildings = [[2,9,10], [3,7,15], [5,12,12], [15,20,10], [19,24,8]]
Output: [[2,10], [3,15], [7,12], [12,0], [15,10], [20,8], [24,0]]
\end{verbatim}

\begin{enumerate}
    \item **Event Transformation:**
    \begin{itemize}
        \item \((2, -10)\), \((9, 10)\)
        \item \((3, -15)\), \((7, 15)\)
        \item \((5, -12)\), \((12, 12)\)
        \item \((15, -10)\), \((20, 10)\)
        \item \((19, -8)\), \((24, 8)\)
    \end{itemize}
    
    \item **Sorting Events:**
    \begin{itemize}
        \item Sorted order: \((2, -10)\), \((3, -15)\), \((5, -12)\), \((7, 15)\), \((9, 10)\), \((12, 12)\), \((15, -10)\), \((19, -8)\), \((20, 10)\), \((24, 8)\)
    \end{itemize}
    
    \item **Processing Events:**
    \begin{itemize}
        \item At each event, update the heap and determine if the skyline height changes.
    \end{itemize}
    
    \item **Result Construction:**
    \begin{itemize}
        \item The resulting skyline key points are accumulated as \([[2,10], [3,15], [7,12], [12,0], [15,10], [20,8], [24,0]]\).
    \end{itemize}
\end{enumerate}

Thus, the function correctly constructs the skyline based on the buildings' positions and heights.

\section*{Why This Approach}

The Sweep Line Algorithm combined with a max heap offers an optimal solution with \(O(n \log n)\) time complexity and efficient handling of overlapping buildings. By processing events in sorted order and maintaining active building heights, the algorithm ensures that all critical points in the skyline are accurately identified without redundant computations.

\section*{Alternative Approaches}

\subsection*{1. Divide and Conquer}

Divide the set of buildings into smaller subsets, compute the skyline for each subset, and then merge the skylines.

\begin{lstlisting}[language=Python]
class Solution:
    def getSkyline(self, buildings: List[List[int]]) -> List[List[int]]:
        def merge(left, right):
            h1, h2 = 0, 0
            i, j = 0, 0
            merged = []
            while i < len(left) and j < len(right):
                if left[i][0] < right[j][0]:
                    x, h1 = left[i]
                    i += 1
                elif left[i][0] > right[j][0]:
                    x, h2 = right[j]
                    j += 1
                else:
                    x, h1 = left[i]
                    _, h2 = right[j]
                    i += 1
                    j += 1
                max_h = max(h1, h2)
                if not merged or merged[-1][1] != max_h:
                    merged.append([x, max_h])
            merged.extend(left[i:])
            merged.extend(right[j:])
            return merged
        
        def divide(buildings):
            if not buildings:
                return []
            if len(buildings) == 1:
                L, R, H = buildings[0]
                return [[L, H], [R, 0]]
            mid = len(buildings) // 2
            left = divide(buildings[:mid])
            right = divide(buildings[mid:])
            return merge(left, right)
        
        return divide(buildings)
\end{lstlisting}

\textbf{Complexities:}
\begin{itemize}
    \item \textbf{Time Complexity:} \(O(n \log n)\)
    \item \textbf{Space Complexity:} \(O(n)\)
\end{itemize}

\subsection*{2. Using Segment Trees}

Implement a segment tree to manage and query overlapping building heights dynamically.

\textbf{Note}: This approach is more complex and is generally used for advanced scenarios with multiple dynamic queries.

\section*{Similar Problems to This One}

Several problems involve skyline-like constructions, spatial data analysis, and efficient event processing, utilizing similar algorithmic strategies:

\begin{itemize}
    \item \textbf{Merge Intervals}: Merge overlapping intervals in a list.
    \item \textbf{Largest Rectangle in Histogram}: Find the largest rectangular area in a histogram.
    \item \textbf{Interval Partitioning}: Assign intervals to resources without overlap.
    \item \textbf{Line Segment Intersection}: Detect intersections among line segments.
    \item \textbf{Closest Pair of Points}: Find the closest pair of points in a set.
    \item \textbf{Convex Hull}: Compute the convex hull of a set of points.
    \item \textbf{Point Inside Polygon}: Determine if a point lies inside a given polygon.
    \item \textbf{Range Searching}: Efficiently query geometric data within a specified range.
\end{itemize}

These problems reinforce concepts of event-driven processing, spatial reasoning, and efficient algorithm design in various contexts.

\section*{Things to Keep in Mind and Tricks}

When tackling the \textbf{Skyline Problem}, consider the following tips and best practices to enhance efficiency and correctness:

\begin{itemize}
    \item \textbf{Understand Sweep Line Technique}: Grasp how the sweep line algorithm processes events in sorted order to handle dynamic changes efficiently.
    \index{Sweep Line Technique}
    
    \item \textbf{Leverage Priority Queues (Heaps)}: Use max heaps to keep track of active buildings' heights, enabling quick access to the current maximum height.
    \index{Priority Queues}
    
    \item \textbf{Handle Start and End Events Differently}: Differentiate between building start and end events to accurately manage active heights.
    \index{Start and End Events}
    
    \item \textbf{Optimize Event Sorting}: Sort events primarily by x-coordinate and secondarily by height to ensure correct processing order.
    \index{Event Sorting}
    
    \item \textbf{Manage Active Heights Efficiently}: Use data structures that allow efficient insertion, deletion, and retrieval of maximum elements.
    \index{Active Heights Management}
    
    \item \textbf{Avoid Redundant Key Points}: Only record key points when the skyline height changes to minimize the output list.
    \index{Avoiding Redundant Key Points}
    
    \item \textbf{Implement Helper Functions}: Create helper functions for tasks like distance calculation, event handling, and heap management to enhance modularity.
    \index{Helper Functions}
    
    \item \textbf{Code Readability}: Maintain clear and readable code through meaningful variable names and structured logic.
    \index{Code Readability}
    
    \item \textbf{Test Extensively}: Implement a wide range of test cases, including overlapping, non-overlapping, and edge-touching buildings, to ensure robustness.
    \index{Extensive Testing}
    
    \item \textbf{Handle Degenerate Cases}: Manage cases where buildings have zero height or identical coordinates gracefully.
    \index{Degenerate Cases}
    
    \item \textbf{Understand Geometric Relationships}: Grasp how buildings overlap and influence the skyline to simplify the algorithm.
    \index{Geometric Relationships}
    
    \item \textbf{Use Appropriate Data Structures}: Utilize appropriate data structures like heaps, lists, and dictionaries to manage and process data efficiently.
    \index{Appropriate Data Structures}
    
    \item \textbf{Optimize for Large Inputs}: Design the algorithm to handle large numbers of buildings without significant performance degradation.
    \index{Optimizing for Large Inputs}
    
    \item \textbf{Implement Iterative Solutions Carefully}: Ensure that loop conditions are correctly defined to prevent infinite loops or incorrect terminations.
    \index{Iterative Solutions}
    
    \item \textbf{Consistent Naming Conventions}: Use consistent and descriptive naming conventions for variables and functions to improve code clarity.
    \index{Naming Conventions}
\end{itemize}

\section*{Corner and Special Cases to Test When Writing the Code}

When implementing the solution for the \textbf{Skyline Problem}, it is crucial to consider and rigorously test various edge cases to ensure robustness and correctness:

\begin{itemize}
    \item \textbf{No Overlapping Buildings}: All buildings are separate and do not overlap.
    \index{No Overlapping Buildings}
    
    \item \textbf{Fully Overlapping Buildings}: Multiple buildings completely overlap each other.
    \index{Fully Overlapping Buildings}
    
    \item \textbf{Buildings Touching at Edges}: Buildings share common edges without overlapping.
    \index{Buildings Touching at Edges}
    
    \item \textbf{Buildings Touching at Corners}: Buildings share common corners without overlapping.
    \index{Buildings Touching at Corners}
    
    \item \textbf{Single Building}: Only one building is present.
    \index{Single Building}
    
    \item \textbf{Multiple Buildings with Same Start or End}: Multiple buildings start or end at the same x-coordinate.
    \index{Same Start or End}
    
    \item \textbf{Buildings with Zero Height}: Buildings that have zero height should not affect the skyline.
    \index{Buildings with Zero Height}
    
    \item \textbf{Large Number of Buildings}: Test with a large number of buildings to ensure performance and scalability.
    \index{Large Number of Buildings}
    
    \item \textbf{Buildings with Negative Coordinates}: Buildings positioned in negative coordinate space.
    \index{Negative Coordinates}
    
    \item \textbf{Boundary Values}: Buildings at the minimum and maximum limits of the coordinate range.
    \index{Boundary Values}
    
    \item \textbf{Buildings with Identical Coordinates}: Multiple buildings with the same coordinates.
    \index{Identical Coordinates}
    
    \item \textbf{Sequential Buildings}: Buildings placed sequentially without gaps.
    \index{Sequential Buildings}
    
    \item \textbf{Overlapping and Non-Overlapping Mixed}: A mix of overlapping and non-overlapping buildings.
    \index{Overlapping and Non-Overlapping Mixed}
    
    \item \textbf{Buildings with Very Large Heights}: Buildings with heights at the upper limit of the constraints.
    \index{Very Large Heights}
    
    \item \textbf{Empty Input}: No buildings are provided.
    \index{Empty Input}
\end{itemize}

\section*{Implementation Considerations}

When implementing the \texttt{getSkyline} function, keep in mind the following considerations to ensure robustness and efficiency:

\begin{itemize}
    \item \textbf{Data Type Selection}: Use appropriate data types that can handle large input values and avoid overflow or precision issues.
    \index{Data Type Selection}
    
    \item \textbf{Optimizing Event Sorting}: Efficiently sort events based on x-coordinates and heights to ensure correct processing order.
    \index{Optimizing Event Sorting}
    
    \item \textbf{Handling Large Inputs}: Design the algorithm to handle up to \(10^4\) buildings efficiently without significant performance degradation.
    \index{Handling Large Inputs}
    
    \item \textbf{Using Efficient Data Structures}: Utilize heaps, lists, and dictionaries effectively to manage and process events and active heights.
    \index{Efficient Data Structures}
    
    \item \textbf{Avoiding Redundant Calculations}: Ensure that distance and overlap calculations are performed only when necessary to optimize performance.
    \index{Avoiding Redundant Calculations}
    
    \item \textbf{Code Readability and Documentation}: Maintain clear and readable code through meaningful variable names and comprehensive comments to facilitate understanding and maintenance.
    \index{Code Readability}
    
    \item \textbf{Edge Case Handling}: Implement checks for edge cases to prevent incorrect results or runtime errors.
    \index{Edge Case Handling}
    
    \item \textbf{Implementing Helper Functions}: Create helper functions for tasks like distance calculation, event handling, and heap management to enhance modularity.
    \index{Helper Functions}
    
    \item \textbf{Consistent Naming Conventions}: Use consistent and descriptive naming conventions for variables and functions to improve code clarity.
    \index{Naming Conventions}
    
    \item \textbf{Memory Management}: Ensure that the algorithm manages memory efficiently, especially when dealing with large datasets.
    \index{Memory Management}
    
    \item \textbf{Implementing Iterative Solutions Carefully}: Ensure that loop conditions are correctly defined to prevent infinite loops or incorrect terminations.
    \index{Iterative Solutions}
    
    \item \textbf{Avoiding Floating-Point Precision Issues}: Since the problem deals with integers, floating-point precision is not a concern, simplifying the implementation.
    \index{Floating-Point Precision}
    
    \item \textbf{Testing and Validation}: Develop a comprehensive suite of test cases that cover all possible scenarios, including edge cases, to validate the correctness and efficiency of the implementation.
    \index{Testing and Validation}
    
    \item \textbf{Performance Considerations}: Optimize the loop conditions and operations to ensure that the function runs efficiently, especially for large input numbers.
    \index{Performance Considerations}
\end{itemize}

\section*{Conclusion}

The \textbf{Skyline Problem} is a quintessential example of applying advanced algorithmic techniques and geometric reasoning to solve complex spatial challenges. By leveraging the Sweep Line Algorithm and maintaining active building heights using a max heap, the solution efficiently constructs the skyline with optimal time and space complexities. Understanding and implementing such sophisticated algorithms not only enhances problem-solving skills but also provides a foundation for tackling a wide array of Computational Geometry problems in various domains, including computer graphics, urban planning simulations, and geographic information systems.

\printindex

% \input{sections/rectangle_overlap}
% \input{sections/rectangle_area}
% \input{sections/k_closest_points_to_origin}
% \input{sections/the_skyline_problem}
% % filename: rectangle_area.tex

\problemsection{Rectangle Area}
\label{chap:Rectangle_Area}
\marginnote{\href{https://leetcode.com/problems/rectangle-area/}{[LeetCode Link]}\index{LeetCode}}
\marginnote{\href{https://www.geeksforgeeks.org/find-area-two-overlapping-rectangles/}{[GeeksForGeeks Link]}\index{GeeksForGeeks}}
\marginnote{\href{https://www.interviewbit.com/problems/rectangle-area/}{[InterviewBit Link]}\index{InterviewBit}}
\marginnote{\href{https://app.codesignal.com/challenges/rectangle-area}{[CodeSignal Link]}\index{CodeSignal}}
\marginnote{\href{https://www.codewars.com/kata/rectangle-area/train/python}{[Codewars Link]}\index{Codewars}}

The \textbf{Rectangle Area} problem is a classic Computational Geometry challenge that involves calculating the total area covered by two axis-aligned rectangles in a 2D plane. This problem tests one's ability to perform geometric calculations, handle overlapping scenarios, and implement efficient algorithms. Mastery of this problem is essential for applications in computer graphics, spatial analysis, and computational modeling.

\section*{Problem Statement}

Given two axis-aligned rectangles in a 2D plane, compute the total area covered by the two rectangles. The area covered by the overlapping region should be counted only once.

Each rectangle is represented as a list of four integers \([x1, y1, x2, y2]\), where \((x1, y1)\) are the coordinates of the bottom-left corner, and \((x2, y2)\) are the coordinates of the top-right corner.

\textbf{Function signature in Python:}
\begin{lstlisting}[language=Python]
def computeArea(A: List[int], B: List[int]) -> int:
\end{lstlisting}

\section*{Examples}

\textbf{Example 1:}

\begin{verbatim}
Input: A = [-3,0,3,4], B = [0,-1,9,2]
Output: 45
Explanation:
Area of A = (3 - (-3)) * (4 - 0) = 6 * 4 = 24
Area of B = (9 - 0) * (2 - (-1)) = 9 * 3 = 27
Overlapping Area = (3 - 0) * (2 - 0) = 3 * 2 = 6
Total Area = 24 + 27 - 6 = 45
\end{verbatim}

\textbf{Example 2:}

\begin{verbatim}
Input: A = [0,0,0,0], B = [0,0,0,0]
Output: 0
Explanation:
Both rectangles are degenerate points with zero area.
\end{verbatim}

\textbf{Example 3:}

\begin{verbatim}
Input: A = [0,0,2,2], B = [1,1,3,3]
Output: 7
Explanation:
Area of A = 4
Area of B = 4
Overlapping Area = 1
Total Area = 4 + 4 - 1 = 7
\end{verbatim}

\textbf{Example 4:}

\begin{verbatim}
Input: A = [0,0,1,1], B = [1,0,2,1]
Output: 2
Explanation:
Rectangles touch at the edge but do not overlap.
Area of A = 1
Area of B = 1
Overlapping Area = 0
Total Area = 1 + 1 = 2
\end{verbatim}

\textbf{Constraints:}

\begin{itemize}
    \item All coordinates are integers in the range \([-10^9, 10^9]\).
    \item For each rectangle, \(x1 < x2\) and \(y1 < y2\).
\end{itemize}

LeetCode link: \href{https://leetcode.com/problems/rectangle-area/}{Rectangle Area}\index{LeetCode}

\section*{Algorithmic Approach}

To compute the total area covered by two axis-aligned rectangles, we can follow these steps:

1. **Calculate Individual Areas:**
   - Compute the area of the first rectangle.
   - Compute the area of the second rectangle.

2. **Determine Overlapping Area:**
   - Calculate the coordinates of the overlapping rectangle, if any.
   - If the rectangles overlap, compute the area of the overlapping region.

3. **Compute Total Area:**
   - Sum the individual areas and subtract the overlapping area to avoid double-counting.

\marginnote{This approach ensures accurate area calculation by handling overlapping regions appropriately.}

\section*{Complexities}

\begin{itemize}
    \item \textbf{Time Complexity:} \(O(1)\). The algorithm performs a constant number of calculations.
    
    \item \textbf{Space Complexity:} \(O(1)\). Only a fixed amount of extra space is used for variables.
\end{itemize}

\section*{Python Implementation}

\marginnote{Implementing the area calculation with overlap consideration ensures an accurate and efficient solution.}

Below is the complete Python code implementing the \texttt{computeArea} function:

\begin{fullwidth}
\begin{lstlisting}[language=Python]
from typing import List

class Solution:
    def computeArea(self, A: List[int], B: List[int]) -> int:
        # Calculate area of rectangle A
        areaA = (A[2] - A[0]) * (A[3] - A[1])
        
        # Calculate area of rectangle B
        areaB = (B[2] - B[0]) * (B[3] - B[1])
        
        # Determine overlap coordinates
        overlap_x1 = max(A[0], B[0])
        overlap_y1 = max(A[1], B[1])
        overlap_x2 = min(A[2], B[2])
        overlap_y2 = min(A[3], B[3])
        
        # Calculate overlapping area
        overlap_width = overlap_x2 - overlap_x1
        overlap_height = overlap_y2 - overlap_y1
        overlap_area = 0
        if overlap_width > 0 and overlap_height > 0:
            overlap_area = overlap_width * overlap_height
        
        # Total area is sum of individual areas minus overlapping area
        total_area = areaA + areaB - overlap_area
        return total_area

# Example usage:
solution = Solution()
print(solution.computeArea([-3,0,3,4], [0,-1,9,2]))  # Output: 45
print(solution.computeArea([0,0,0,0], [0,0,0,0]))    # Output: 0
print(solution.computeArea([0,0,2,2], [1,1,3,3]))    # Output: 7
print(solution.computeArea([0,0,1,1], [1,0,2,1]))    # Output: 2
\end{lstlisting}
\end{fullwidth}

This implementation accurately computes the total area covered by two rectangles by accounting for any overlapping regions. It ensures that the overlapping area is not double-counted.

\section*{Explanation}

The \texttt{computeArea} function calculates the combined area of two axis-aligned rectangles by following these steps:

\subsection*{1. Calculate Individual Areas}

\begin{itemize}
    \item **Rectangle A:**
    \begin{itemize}
        \item Width: \(A[2] - A[0]\)
        \item Height: \(A[3] - A[1]\)
        \item Area: Width \(\times\) Height
    \end{itemize}
    
    \item **Rectangle B:**
    \begin{itemize}
        \item Width: \(B[2] - B[0]\)
        \item Height: \(B[3] - B[1]\)
        \item Area: Width \(\times\) Height
    \end{itemize}
\end{itemize}

\subsection*{2. Determine Overlapping Area}

\begin{itemize}
    \item **Overlap Coordinates:**
    \begin{itemize}
        \item Left (x-coordinate): \(\text{max}(A[0], B[0])\)
        \item Bottom (y-coordinate): \(\text{max}(A[1], B[1])\)
        \item Right (x-coordinate): \(\text{min}(A[2], B[2])\)
        \item Top (y-coordinate): \(\text{min}(A[3], B[3])\)
    \end{itemize}
    
    \item **Overlap Dimensions:**
    \begin{itemize}
        \item Width: \(\text{overlap\_x2} - \text{overlap\_x1}\)
        \item Height: \(\text{overlap\_y2} - \text{overlap\_y1}\)
    \end{itemize}
    
    \item **Overlap Area:**
    \begin{itemize}
        \item If both width and height are positive, the rectangles overlap, and the overlapping area is their product.
        \item Otherwise, there is no overlap, and the overlapping area is zero.
    \end{itemize}
\end{itemize}

\subsection*{3. Compute Total Area}

\begin{itemize}
    \item Total Area = Area of Rectangle A + Area of Rectangle B - Overlapping Area
\end{itemize}

\subsection*{4. Example Walkthrough}

Consider the first example:
\begin{verbatim}
Input: A = [-3,0,3,4], B = [0,-1,9,2]
Output: 45
\end{verbatim}

\begin{enumerate}
    \item **Calculate Areas:**
    \begin{itemize}
        \item Area of A = (3 - (-3)) * (4 - 0) = 6 * 4 = 24
        \item Area of B = (9 - 0) * (2 - (-1)) = 9 * 3 = 27
    \end{itemize}
    
    \item **Determine Overlap:**
    \begin{itemize}
        \item overlap\_x1 = max(-3, 0) = 0
        \item overlap\_y1 = max(0, -1) = 0
        \item overlap\_x2 = min(3, 9) = 3
        \item overlap\_y2 = min(4, 2) = 2
        \item overlap\_width = 3 - 0 = 3
        \item overlap\_height = 2 - 0 = 2
        \item overlap\_area = 3 * 2 = 6
    \end{itemize}
    
    \item **Compute Total Area:**
    \begin{itemize}
        \item Total Area = 24 + 27 - 6 = 45
    \end{itemize}
\end{enumerate}

Thus, the function correctly returns \texttt{45}.

\section*{Why This Approach}

This approach is chosen for its straightforwardness and optimal efficiency. By directly calculating the individual areas and intelligently handling the overlapping region, the algorithm ensures accurate results without unnecessary computations. Its constant time complexity makes it highly efficient, even for large coordinate values.

\section*{Alternative Approaches}

\subsection*{1. Using Intersection Dimensions}

Instead of separately calculating areas, directly compute the dimensions of the overlapping region and subtract it from the sum of individual areas.

\begin{lstlisting}[language=Python]
def computeArea(A: List[int], B: List[int]) -> int:
    # Sum of individual areas
    area = (A[2] - A[0]) * (A[3] - A[1]) + (B[2] - B[0]) * (B[3] - B[1])
    
    # Overlapping area
    overlap_width = min(A[2], B[2]) - max(A[0], B[0])
    overlap_height = min(A[3], B[3]) - max(A[1], B[1])
    
    if overlap_width > 0 and overlap_height > 0:
        area -= overlap_width * overlap_height
    
    return area
\end{lstlisting}

\subsection*{2. Using Geometry Libraries}

Leverage computational geometry libraries to handle area calculations and overlapping detections.

\begin{lstlisting}[language=Python]
from shapely.geometry import box

def computeArea(A: List[int], B: List[int]) -> int:
    rect1 = box(A[0], A[1], A[2], A[3])
    rect2 = box(B[0], B[1], B[2], B[3])
    intersection = rect1.intersection(rect2)
    return int(rect1.area + rect2.area - intersection.area)
\end{lstlisting}

\textbf{Note}: This approach requires the \texttt{shapely} library and is more suitable for complex geometric operations.

\section*{Similar Problems to This One}

Several problems involve calculating areas, handling geometric overlaps, and spatial reasoning, utilizing similar algorithmic strategies:

\begin{itemize}
    \item \textbf{Rectangle Overlap}: Determine if two rectangles overlap.
    \item \textbf{Circle Area Overlap}: Calculate the overlapping area between two circles.
    \item \textbf{Polygon Area}: Compute the area of a given polygon.
    \item \textbf{Union of Rectangles}: Calculate the total area covered by multiple rectangles, accounting for overlaps.
    \item \textbf{Intersection of Lines}: Find the intersection point of two lines.
    \item \textbf{Closest Pair of Points}: Find the closest pair of points in a set.
    \item \textbf{Convex Hull}: Compute the convex hull of a set of points.
    \item \textbf{Point Inside Polygon}: Determine if a point lies inside a given polygon.
\end{itemize}

These problems reinforce concepts of geometric calculations, area computations, and efficient algorithm design in various contexts.

\section*{Things to Keep in Mind and Tricks}

When tackling the \textbf{Rectangle Area} problem, consider the following tips and best practices to enhance efficiency and correctness:

\begin{itemize}
    \item \textbf{Understand Geometric Relationships}: Grasp the positional relationships between rectangles to simplify area calculations.
    \index{Geometric Relationships}
    
    \item \textbf{Leverage Coordinate Comparisons}: Use direct comparisons of rectangle coordinates to determine overlapping regions.
    \index{Coordinate Comparisons}
    
    \item \textbf{Handle Overlapping Scenarios}: Accurately calculate the overlapping area to avoid double-counting.
    \index{Overlapping Scenarios}
    
    \item \textbf{Optimize for Efficiency}: Aim for a constant time \(O(1)\) solution by avoiding unnecessary computations or iterations.
    \index{Efficiency Optimization}
    
    \item \textbf{Avoid Floating-Point Precision Issues}: Since all coordinates are integers, floating-point precision is not a concern, simplifying the implementation.
    \index{Floating-Point Precision}
    
    \item \textbf{Use Helper Functions}: Create helper functions to encapsulate repetitive tasks, such as calculating overlap dimensions or areas.
    \index{Helper Functions}
    
    \item \textbf{Code Readability}: Maintain clear and readable code through meaningful variable names and structured logic.
    \index{Code Readability}
    
    \item \textbf{Test Extensively}: Implement a wide range of test cases, including overlapping, non-overlapping, and edge-touching rectangles, to ensure robustness.
    \index{Extensive Testing}
    
    \item \textbf{Understand Axis-Aligned Constraints}: Recognize that axis-aligned rectangles simplify area calculations compared to rotated rectangles.
    \index{Axis-Aligned Constraints}
    
    \item \textbf{Simplify Logical Conditions}: Combine multiple conditions logically to streamline the area calculation process.
    \index{Logical Conditions}
    
    \item \textbf{Use Absolute Values}: When calculating differences, ensure that the dimensions are positive by using absolute values or proper ordering.
    \index{Absolute Values}
    
    \item \textbf{Consider Edge Cases}: Handle cases where rectangles have zero area or touch at edges/corners without overlapping.
    \index{Edge Cases}
\end{itemize}

\section*{Corner and Special Cases to Test When Writing the Code}

When implementing the solution for the \textbf{Rectangle Area} problem, it is crucial to consider and rigorously test various edge cases to ensure robustness and correctness:

\begin{itemize}
    \item \textbf{No Overlap}: Rectangles are completely separate.
    \index{No Overlap}
    
    \item \textbf{Partial Overlap}: Rectangles overlap in one or more regions.
    \index{Partial Overlap}
    
    \item \textbf{Edge Touching}: Rectangles touch exactly at one edge without overlapping.
    \index{Edge Touching}
    
    \item \textbf{Corner Touching}: Rectangles touch exactly at one corner without overlapping.
    \index{Corner Touching}
    
    \item \textbf{One Rectangle Inside Another}: One rectangle is entirely within the other.
    \index{Rectangle Inside}
    
    \item \textbf{Identical Rectangles}: Both rectangles have the same coordinates.
    \index{Identical Rectangles}
    
    \item \textbf{Degenerate Rectangles}: Rectangles with zero area (e.g., \(x1 = x2\) or \(y1 = y2\)).
    \index{Degenerate Rectangles}
    
    \item \textbf{Large Coordinates}: Rectangles with very large coordinate values to test performance and integer handling.
    \index{Large Coordinates}
    
    \item \textbf{Negative Coordinates}: Rectangles positioned in negative coordinate space.
    \index{Negative Coordinates}
    
    \item \textbf{Mixed Overlapping Scenarios}: Combinations of the above cases to ensure comprehensive coverage.
    \index{Mixed Overlapping Scenarios}
    
    \item \textbf{Minimum and Maximum Bounds}: Rectangles at the minimum and maximum limits of the coordinate range.
    \index{Minimum and Maximum Bounds}
    
    \item \textbf{Sequential Rectangles}: Multiple rectangles placed sequentially without overlapping.
    \index{Sequential Rectangles}
    
    \item \textbf{Multiple Overlaps}: Scenarios where more than two rectangles overlap in different regions.
    \index{Multiple Overlaps}
\end{itemize}

\section*{Implementation Considerations}

When implementing the \texttt{computeArea} function, keep in mind the following considerations to ensure robustness and efficiency:

\begin{itemize}
    \item \textbf{Data Type Selection}: Use appropriate data types that can handle large input values without overflow or underflow.
    \index{Data Type Selection}
    
    \item \textbf{Optimizing Comparisons}: Structure logical conditions to efficiently determine overlap dimensions.
    \index{Optimizing Comparisons}
    
    \item \textbf{Handling Large Inputs}: Design the algorithm to efficiently handle large input sizes without significant performance degradation.
    \index{Handling Large Inputs}
    
    \item \textbf{Language-Specific Constraints}: Be aware of how the programming language handles large integers and arithmetic operations.
    \index{Language-Specific Constraints}
    
    \item \textbf{Avoiding Redundant Calculations}: Ensure that each calculation contributes towards determining the final area without unnecessary repetitions.
    \index{Avoiding Redundant Calculations}
    
    \item \textbf{Code Readability and Documentation}: Maintain clear and readable code through meaningful variable names and comprehensive comments to facilitate understanding and maintenance.
    \index{Code Readability}
    
    \item \textbf{Edge Case Handling}: Implement checks for edge cases to prevent incorrect results or runtime errors.
    \index{Edge Case Handling}
    
    \item \textbf{Testing and Validation}: Develop a comprehensive suite of test cases that cover all possible scenarios, including edge cases, to validate the correctness and efficiency of the implementation.
    \index{Testing and Validation}
    
    \item \textbf{Scalability}: Design the algorithm to scale efficiently with increasing input sizes, maintaining performance and resource utilization.
    \index{Scalability}
    
    \item \textbf{Using Helper Functions}: Consider creating helper functions for repetitive tasks, such as calculating overlap dimensions, to enhance modularity and reusability.
    \index{Helper Functions}
    
    \item \textbf{Consistent Naming Conventions}: Use consistent and descriptive naming conventions for variables to improve code clarity.
    \index{Naming Conventions}
    
    \item \textbf{Implementing Unit Tests}: Develop unit tests for each logical condition to ensure that all scenarios are correctly handled.
    \index{Unit Tests}
    
    \item \textbf{Error Handling}: Incorporate error handling to manage invalid inputs gracefully.
    \index{Error Handling}
\end{itemize}

\section*{Conclusion}

The \textbf{Rectangle Area} problem showcases the application of fundamental geometric principles and efficient algorithm design to compute spatial properties accurately. By systematically calculating individual areas and intelligently handling overlapping regions, the algorithm ensures precise results without redundant computations. Understanding and implementing such techniques not only enhances problem-solving skills but also provides a foundation for tackling more complex Computational Geometry challenges involving multiple geometric entities and intricate spatial relationships.

\printindex

% % filename: rectangle_overlap.tex

\problemsection{Rectangle Overlap}
\label{chap:Rectangle_Overlap}
\marginnote{\href{https://leetcode.com/problems/rectangle-overlap/}{[LeetCode Link]}\index{LeetCode}}
\marginnote{\href{https://www.geeksforgeeks.org/check-if-two-rectangles-overlap/}{[GeeksForGeeks Link]}\index{GeeksForGeeks}}
\marginnote{\href{https://www.interviewbit.com/problems/rectangle-overlap/}{[InterviewBit Link]}\index{InterviewBit}}
\marginnote{\href{https://app.codesignal.com/challenges/rectangle-overlap}{[CodeSignal Link]}\index{CodeSignal}}
\marginnote{\href{https://www.codewars.com/kata/rectangle-overlap/train/python}{[Codewars Link]}\index{Codewars}}

The \textbf{Rectangle Overlap} problem is a fundamental challenge in Computational Geometry that involves determining whether two axis-aligned rectangles overlap. This problem tests one's ability to understand geometric properties, implement conditional logic, and optimize for efficient computation. Mastery of this problem is essential for applications in computer graphics, collision detection, and spatial data analysis.

\section*{Problem Statement}

Given two axis-aligned rectangles in a 2D plane, determine if they overlap. Each rectangle is defined by its bottom-left and top-right coordinates.

A rectangle is represented as a list of four integers \([x1, y1, x2, y2]\), where \((x1, y1)\) are the coordinates of the bottom-left corner, and \((x2, y2)\) are the coordinates of the top-right corner.

\textbf{Function signature in Python:}
\begin{lstlisting}[language=Python]
def isRectangleOverlap(rec1: List[int], rec2: List[int]) -> bool:
\end{lstlisting}

\section*{Examples}

\textbf{Example 1:}

\begin{verbatim}
Input: rec1 = [0,0,2,2], rec2 = [1,1,3,3]
Output: True
Explanation: The rectangles overlap in the area defined by [1,1,2,2].
\end{verbatim}

\textbf{Example 2:}

\begin{verbatim}
Input: rec1 = [0,0,1,1], rec2 = [1,0,2,1]
Output: False
Explanation: The rectangles touch at the edge but do not overlap.
\end{verbatim}

\textbf{Example 3:}

\begin{verbatim}
Input: rec1 = [0,0,1,1], rec2 = [2,2,3,3]
Output: False
Explanation: The rectangles are completely separate.
\end{verbatim}

\textbf{Example 4:}

\begin{verbatim}
Input: rec1 = [0,0,5,5], rec2 = [3,3,7,7]
Output: True
Explanation: The rectangles overlap in the area defined by [3,3,5,5].
\end{verbatim}

\textbf{Example 5:}

\begin{verbatim}
Input: rec1 = [0,0,0,0], rec2 = [0,0,0,0]
Output: False
Explanation: Both rectangles are degenerate points.
\end{verbatim}

\textbf{Constraints:}

\begin{itemize}
    \item All coordinates are integers in the range \([-10^9, 10^9]\).
    \item For each rectangle, \(x1 < x2\) and \(y1 < y2\).
\end{itemize}

LeetCode link: \href{https://leetcode.com/problems/rectangle-overlap/}{Rectangle Overlap}\index{LeetCode}

\section*{Algorithmic Approach}

To determine whether two axis-aligned rectangles overlap, we can use the following logical conditions:

1. **Non-Overlap Conditions:**
   - One rectangle is to the left of the other.
   - One rectangle is above the other.

2. **Overlap Condition:**
   - If neither of the non-overlap conditions is true, the rectangles must overlap.

\subsection*{Steps:}

1. **Extract Coordinates:**
   - For both rectangles, extract the bottom-left and top-right coordinates.

2. **Check Non-Overlap Conditions:**
   - If the right side of the first rectangle is less than or equal to the left side of the second rectangle, they do not overlap.
   - If the left side of the first rectangle is greater than or equal to the right side of the second rectangle, they do not overlap.
   - If the top side of the first rectangle is less than or equal to the bottom side of the second rectangle, they do not overlap.
   - If the bottom side of the first rectangle is greater than or equal to the top side of the second rectangle, they do not overlap.

3. **Determine Overlap:**
   - If none of the non-overlap conditions are met, the rectangles overlap.

\marginnote{This approach provides an efficient \(O(1)\) time complexity solution by leveraging simple geometric comparisons.}

\section*{Complexities}

\begin{itemize}
    \item \textbf{Time Complexity:} \(O(1)\). The algorithm performs a constant number of comparisons regardless of input size.
    
    \item \textbf{Space Complexity:} \(O(1)\). Only a fixed amount of extra space is used for variables.
\end{itemize}

\section*{Python Implementation}

\marginnote{Implementing the overlap check using coordinate comparisons ensures an optimal and straightforward solution.}

Below is the complete Python code implementing the \texttt{isRectangleOverlap} function:

\begin{fullwidth}
\begin{lstlisting}[language=Python]
from typing import List

class Solution:
    def isRectangleOverlap(self, rec1: List[int], rec2: List[int]) -> bool:
        # Extract coordinates
        left1, bottom1, right1, top1 = rec1
        left2, bottom2, right2, top2 = rec2
        
        # Check non-overlapping conditions
        if right1 <= left2 or right2 <= left1:
            return False
        if top1 <= bottom2 or top2 <= bottom1:
            return False
        
        # If none of the above, rectangles overlap
        return True

# Example usage:
solution = Solution()
print(solution.isRectangleOverlap([0,0,2,2], [1,1,3,3]))  # Output: True
print(solution.isRectangleOverlap([0,0,1,1], [1,0,2,1]))  # Output: False
print(solution.isRectangleOverlap([0,0,1,1], [2,2,3,3]))  # Output: False
print(solution.isRectangleOverlap([0,0,5,5], [3,3,7,7]))  # Output: True
print(solution.isRectangleOverlap([0,0,0,0], [0,0,0,0]))  # Output: False
\end{lstlisting}
\end{fullwidth}

This implementation efficiently checks for overlap by comparing the coordinates of the two rectangles. If any of the non-overlapping conditions are met, it returns \texttt{False}; otherwise, it returns \texttt{True}.

\section*{Explanation}

The \texttt{isRectangleOverlap} function determines whether two axis-aligned rectangles overlap by comparing their respective coordinates. Here's a detailed breakdown of the implementation:

\subsection*{1. Extract Coordinates}

\begin{itemize}
    \item For each rectangle, extract the left (\(x1\)), bottom (\(y1\)), right (\(x2\)), and top (\(y2\)) coordinates.
    \item This simplifies the comparison process by providing clear variables representing each side of the rectangles.
\end{itemize}

\subsection*{2. Check Non-Overlap Conditions}

\begin{itemize}
    \item **Horizontal Separation:**
    \begin{itemize}
        \item If the right side of the first rectangle (\(right1\)) is less than or equal to the left side of the second rectangle (\(left2\)), there is no horizontal overlap.
        \item Similarly, if the right side of the second rectangle (\(right2\)) is less than or equal to the left side of the first rectangle (\(left1\)), there is no horizontal overlap.
    \end{itemize}
    
    \item **Vertical Separation:**
    \begin{itemize}
        \item If the top side of the first rectangle (\(top1\)) is less than or equal to the bottom side of the second rectangle (\(bottom2\)), there is no vertical overlap.
        \item Similarly, if the top side of the second rectangle (\(top2\)) is less than or equal to the bottom side of the first rectangle (\(bottom1\)), there is no vertical overlap.
    \end{itemize}
    
    \item If any of these non-overlapping conditions are true, the rectangles do not overlap, and the function returns \texttt{False}.
\end{itemize}

\subsection*{3. Determine Overlap}

\begin{itemize}
    \item If none of the non-overlapping conditions are met, it implies that the rectangles overlap both horizontally and vertically.
    \item The function returns \texttt{True} in this case.
\end{itemize}

\subsection*{4. Example Walkthrough}

Consider the first example:
\begin{verbatim}
Input: rec1 = [0,0,2,2], rec2 = [1,1,3,3]
Output: True
\end{verbatim}

\begin{enumerate}
    \item Extract coordinates:
    \begin{itemize}
        \item rec1: left1 = 0, bottom1 = 0, right1 = 2, top1 = 2
        \item rec2: left2 = 1, bottom2 = 1, right2 = 3, top2 = 3
    \end{itemize}
    
    \item Check non-overlap conditions:
    \begin{itemize}
        \item \(right1 = 2\) is not less than or equal to \(left2 = 1\)
        \item \(right2 = 3\) is not less than or equal to \(left1 = 0\)
        \item \(top1 = 2\) is not less than or equal to \(bottom2 = 1\)
        \item \(top2 = 3\) is not less than or equal to \(bottom1 = 0\)
    \end{itemize}
    
    \item Since none of the non-overlapping conditions are met, the rectangles overlap.
\end{enumerate}

Thus, the function correctly returns \texttt{True}.

\section*{Why This Approach}

This approach is chosen for its simplicity and efficiency. By leveraging direct coordinate comparisons, the algorithm achieves constant time complexity without the need for complex data structures or iterative processes. It effectively handles all possible scenarios of rectangle positioning, ensuring accurate detection of overlaps.

\section*{Alternative Approaches}

\subsection*{1. Separating Axis Theorem (SAT)}

The Separating Axis Theorem is a more generalized method for detecting overlaps between convex shapes. While it is not necessary for axis-aligned rectangles, understanding SAT can be beneficial for more complex geometric problems.

\begin{lstlisting}[language=Python]
def isRectangleOverlap(rec1: List[int], rec2: List[int]) -> bool:
    # Using SAT for axis-aligned rectangles
    return not (rec1[2] <= rec2[0] or rec1[0] >= rec2[2] or
                rec1[3] <= rec2[1] or rec1[1] >= rec2[3])
\end{lstlisting}

\textbf{Note}: This implementation is functionally identical to the primary approach but leverages a more generalized geometric theorem.

\subsection*{2. Area-Based Approach}

Calculate the overlapping area between the two rectangles. If the overlapping area is positive, the rectangles overlap.

\begin{lstlisting}[language=Python]
def isRectangleOverlap(rec1: List[int], rec2: List[int]) -> bool:
    # Calculate overlap in x and y dimensions
    x_overlap = min(rec1[2], rec2[2]) - max(rec1[0], rec2[0])
    y_overlap = min(rec1[3], rec2[3]) - max(rec1[1], rec2[1])
    
    # Overlap exists if both overlaps are positive
    return x_overlap > 0 and y_overlap > 0
\end{lstlisting}

\textbf{Complexities:}
\begin{itemize}
    \item \textbf{Time Complexity:} \(O(1)\)
    \item \textbf{Space Complexity:} \(O(1)\)
\end{itemize}

\subsection*{3. Using Rectangles Intersection Function}

Utilize built-in or library functions that handle geometric intersections.

\begin{lstlisting}[language=Python]
from shapely.geometry import box

def isRectangleOverlap(rec1: List[int], rec2: List[int]) -> bool:
    rectangle1 = box(rec1[0], rec1[1], rec1[2], rec1[3])
    rectangle2 = box(rec2[0], rec2[1], rec2[2], rec2[3])
    return rectangle1.intersects(rectangle2) and not rectangle1.touches(rectangle2)
\end{lstlisting}

\textbf{Note}: This approach requires the \texttt{shapely} library and is more suitable for complex geometric operations.

\section*{Similar Problems to This One}

Several problems revolve around geometric overlap, intersection detection, and spatial reasoning, utilizing similar algorithmic strategies:

\begin{itemize}
    \item \textbf{Interval Overlap}: Determine if two intervals on a line overlap.
    \item \textbf{Circle Overlap}: Determine if two circles overlap based on their radii and centers.
    \item \textbf{Polygon Overlap}: Determine if two polygons overlap using algorithms like SAT.
    \item \textbf{Closest Pair of Points}: Find the closest pair of points in a set.
    \item \textbf{Convex Hull}: Compute the convex hull of a set of points.
    \item \textbf{Intersection of Lines}: Find the intersection point of two lines.
    \item \textbf{Point Inside Polygon}: Determine if a point lies inside a given polygon.
\end{itemize}

These problems reinforce the concepts of spatial reasoning, geometric property analysis, and efficient algorithm design in various contexts.

\section*{Things to Keep in Mind and Tricks}

When working with the \textbf{Rectangle Overlap} problem, consider the following tips and best practices to enhance efficiency and correctness:

\begin{itemize}
    \item \textbf{Understand Geometric Relationships}: Grasp the positional relationships between rectangles to simplify overlap detection.
    \index{Geometric Relationships}
    
    \item \textbf{Leverage Coordinate Comparisons}: Use direct comparisons of rectangle coordinates to determine spatial relationships.
    \index{Coordinate Comparisons}
    
    \item \textbf{Handle Edge Cases}: Consider cases where rectangles touch at edges or corners without overlapping.
    \index{Edge Cases}
    
    \item \textbf{Optimize for Efficiency}: Aim for a constant time \(O(1)\) solution by avoiding unnecessary computations or iterations.
    \index{Efficiency Optimization}
    
    \item \textbf{Avoid Floating-Point Precision Issues}: Since all coordinates are integers, floating-point precision is not a concern, simplifying the implementation.
    \index{Floating-Point Precision}
    
    \item \textbf{Use Helper Functions}: Create helper functions to encapsulate repetitive tasks, such as extracting coordinates or checking specific conditions.
    \index{Helper Functions}
    
    \item \textbf{Code Readability}: Maintain clear and readable code through meaningful variable names and structured logic.
    \index{Code Readability}
    
    \item \textbf{Test Extensively}: Implement a wide range of test cases, including overlapping, non-overlapping, and edge-touching rectangles, to ensure robustness.
    \index{Extensive Testing}
    
    \item \textbf{Understand Axis-Aligned Constraints}: Recognize that axis-aligned rectangles simplify overlap detection compared to rotated rectangles.
    \index{Axis-Aligned Constraints}
    
    \item \textbf{Simplify Logical Conditions}: Combine multiple conditions logically to streamline the overlap detection process.
    \index{Logical Conditions}
\end{itemize}

\section*{Corner and Special Cases to Test When Writing the Code}

When implementing the solution for the \textbf{Rectangle Overlap} problem, it is crucial to consider and rigorously test various edge cases to ensure robustness and correctness:

\begin{itemize}
    \item \textbf{No Overlap}: Rectangles are completely separate.
    \index{No Overlap}
    
    \item \textbf{Partial Overlap}: Rectangles overlap in one or more regions.
    \index{Partial Overlap}
    
    \item \textbf{Edge Touching}: Rectangles touch exactly at one edge without overlapping.
    \index{Edge Touching}
    
    \item \textbf{Corner Touching}: Rectangles touch exactly at one corner without overlapping.
    \index{Corner Touching}
    
    \item \textbf{One Rectangle Inside Another}: One rectangle is entirely within the other.
    \index{Rectangle Inside}
    
    \item \textbf{Identical Rectangles}: Both rectangles have the same coordinates.
    \index{Identical Rectangles}
    
    \item \textbf{Degenerate Rectangles}: Rectangles with zero area (e.g., \(x1 = x2\) or \(y1 = y2\)).
    \index{Degenerate Rectangles}
    
    \item \textbf{Large Coordinates}: Rectangles with very large coordinate values to test performance and integer handling.
    \index{Large Coordinates}
    
    \item \textbf{Negative Coordinates}: Rectangles positioned in negative coordinate space.
    \index{Negative Coordinates}
    
    \item \textbf{Mixed Overlapping Scenarios}: Combinations of the above cases to ensure comprehensive coverage.
    \index{Mixed Overlapping Scenarios}
    
    \item \textbf{Minimum and Maximum Bounds}: Rectangles at the minimum and maximum limits of the coordinate range.
    \index{Minimum and Maximum Bounds}
\end{itemize}

\section*{Implementation Considerations}

When implementing the \texttt{isRectangleOverlap} function, keep in mind the following considerations to ensure robustness and efficiency:

\begin{itemize}
    \item \textbf{Data Type Selection}: Use appropriate data types that can handle the range of input values without overflow or underflow.
    \index{Data Type Selection}
    
    \item \textbf{Optimizing Comparisons}: Structure logical conditions to short-circuit evaluations as soon as a non-overlapping condition is met.
    \index{Optimizing Comparisons}
    
    \item \textbf{Language-Specific Constraints}: Be aware of how the programming language handles integer division and comparisons.
    \index{Language-Specific Constraints}
    
    \item \textbf{Avoiding Redundant Calculations}: Ensure that each comparison contributes towards determining overlap without unnecessary repetitions.
    \index{Avoiding Redundant Calculations}
    
    \item \textbf{Code Readability and Documentation}: Maintain clear and readable code through meaningful variable names and comprehensive comments to facilitate understanding and maintenance.
    \index{Code Readability}
    
    \item \textbf{Edge Case Handling}: Implement checks for edge cases to prevent incorrect results or runtime errors.
    \index{Edge Case Handling}
    
    \item \textbf{Testing and Validation}: Develop a comprehensive suite of test cases that cover all possible scenarios, including edge cases, to validate the correctness and efficiency of the implementation.
    \index{Testing and Validation}
    
    \item \textbf{Scalability}: Design the algorithm to scale efficiently with increasing input sizes, maintaining performance and resource utilization.
    \index{Scalability}
    
    \item \textbf{Using Helper Functions}: Consider creating helper functions for repetitive tasks, such as extracting and comparing coordinates, to enhance modularity and reusability.
    \index{Helper Functions}
    
    \item \textbf{Consistent Naming Conventions}: Use consistent and descriptive naming conventions for variables to improve code clarity.
    \index{Naming Conventions}
    
    \item \textbf{Handling Floating-Point Coordinates}: Although the problem specifies integer coordinates, ensure that the implementation can handle floating-point numbers if needed in extended scenarios.
    \index{Floating-Point Coordinates}
    
    \item \textbf{Avoiding Floating-Point Precision Issues}: Since all coordinates are integers, floating-point precision is not a concern, simplifying the implementation.
    \index{Floating-Point Precision}
    
    \item \textbf{Implementing Unit Tests}: Develop unit tests for each logical condition to ensure that all scenarios are correctly handled.
    \index{Unit Tests}
    
    \item \textbf{Error Handling}: Incorporate error handling to manage invalid inputs gracefully.
    \index{Error Handling}
\end{itemize}

\section*{Conclusion}

The \textbf{Rectangle Overlap} problem exemplifies the application of fundamental geometric principles and conditional logic to solve spatial challenges efficiently. By leveraging simple coordinate comparisons, the algorithm achieves optimal time and space complexities, making it highly suitable for real-time applications such as collision detection in gaming, layout planning in graphics, and spatial data analysis. Understanding and implementing such techniques not only enhances problem-solving skills but also provides a foundation for tackling more complex Computational Geometry problems involving varied geometric shapes and interactions.

\printindex

% \input{sections/rectangle_overlap}
% \input{sections/rectangle_area}
% \input{sections/k_closest_points_to_origin}
% \input{sections/the_skyline_problem}
% % filename: rectangle_area.tex

\problemsection{Rectangle Area}
\label{chap:Rectangle_Area}
\marginnote{\href{https://leetcode.com/problems/rectangle-area/}{[LeetCode Link]}\index{LeetCode}}
\marginnote{\href{https://www.geeksforgeeks.org/find-area-two-overlapping-rectangles/}{[GeeksForGeeks Link]}\index{GeeksForGeeks}}
\marginnote{\href{https://www.interviewbit.com/problems/rectangle-area/}{[InterviewBit Link]}\index{InterviewBit}}
\marginnote{\href{https://app.codesignal.com/challenges/rectangle-area}{[CodeSignal Link]}\index{CodeSignal}}
\marginnote{\href{https://www.codewars.com/kata/rectangle-area/train/python}{[Codewars Link]}\index{Codewars}}

The \textbf{Rectangle Area} problem is a classic Computational Geometry challenge that involves calculating the total area covered by two axis-aligned rectangles in a 2D plane. This problem tests one's ability to perform geometric calculations, handle overlapping scenarios, and implement efficient algorithms. Mastery of this problem is essential for applications in computer graphics, spatial analysis, and computational modeling.

\section*{Problem Statement}

Given two axis-aligned rectangles in a 2D plane, compute the total area covered by the two rectangles. The area covered by the overlapping region should be counted only once.

Each rectangle is represented as a list of four integers \([x1, y1, x2, y2]\), where \((x1, y1)\) are the coordinates of the bottom-left corner, and \((x2, y2)\) are the coordinates of the top-right corner.

\textbf{Function signature in Python:}
\begin{lstlisting}[language=Python]
def computeArea(A: List[int], B: List[int]) -> int:
\end{lstlisting}

\section*{Examples}

\textbf{Example 1:}

\begin{verbatim}
Input: A = [-3,0,3,4], B = [0,-1,9,2]
Output: 45
Explanation:
Area of A = (3 - (-3)) * (4 - 0) = 6 * 4 = 24
Area of B = (9 - 0) * (2 - (-1)) = 9 * 3 = 27
Overlapping Area = (3 - 0) * (2 - 0) = 3 * 2 = 6
Total Area = 24 + 27 - 6 = 45
\end{verbatim}

\textbf{Example 2:}

\begin{verbatim}
Input: A = [0,0,0,0], B = [0,0,0,0]
Output: 0
Explanation:
Both rectangles are degenerate points with zero area.
\end{verbatim}

\textbf{Example 3:}

\begin{verbatim}
Input: A = [0,0,2,2], B = [1,1,3,3]
Output: 7
Explanation:
Area of A = 4
Area of B = 4
Overlapping Area = 1
Total Area = 4 + 4 - 1 = 7
\end{verbatim}

\textbf{Example 4:}

\begin{verbatim}
Input: A = [0,0,1,1], B = [1,0,2,1]
Output: 2
Explanation:
Rectangles touch at the edge but do not overlap.
Area of A = 1
Area of B = 1
Overlapping Area = 0
Total Area = 1 + 1 = 2
\end{verbatim}

\textbf{Constraints:}

\begin{itemize}
    \item All coordinates are integers in the range \([-10^9, 10^9]\).
    \item For each rectangle, \(x1 < x2\) and \(y1 < y2\).
\end{itemize}

LeetCode link: \href{https://leetcode.com/problems/rectangle-area/}{Rectangle Area}\index{LeetCode}

\section*{Algorithmic Approach}

To compute the total area covered by two axis-aligned rectangles, we can follow these steps:

1. **Calculate Individual Areas:**
   - Compute the area of the first rectangle.
   - Compute the area of the second rectangle.

2. **Determine Overlapping Area:**
   - Calculate the coordinates of the overlapping rectangle, if any.
   - If the rectangles overlap, compute the area of the overlapping region.

3. **Compute Total Area:**
   - Sum the individual areas and subtract the overlapping area to avoid double-counting.

\marginnote{This approach ensures accurate area calculation by handling overlapping regions appropriately.}

\section*{Complexities}

\begin{itemize}
    \item \textbf{Time Complexity:} \(O(1)\). The algorithm performs a constant number of calculations.
    
    \item \textbf{Space Complexity:} \(O(1)\). Only a fixed amount of extra space is used for variables.
\end{itemize}

\section*{Python Implementation}

\marginnote{Implementing the area calculation with overlap consideration ensures an accurate and efficient solution.}

Below is the complete Python code implementing the \texttt{computeArea} function:

\begin{fullwidth}
\begin{lstlisting}[language=Python]
from typing import List

class Solution:
    def computeArea(self, A: List[int], B: List[int]) -> int:
        # Calculate area of rectangle A
        areaA = (A[2] - A[0]) * (A[3] - A[1])
        
        # Calculate area of rectangle B
        areaB = (B[2] - B[0]) * (B[3] - B[1])
        
        # Determine overlap coordinates
        overlap_x1 = max(A[0], B[0])
        overlap_y1 = max(A[1], B[1])
        overlap_x2 = min(A[2], B[2])
        overlap_y2 = min(A[3], B[3])
        
        # Calculate overlapping area
        overlap_width = overlap_x2 - overlap_x1
        overlap_height = overlap_y2 - overlap_y1
        overlap_area = 0
        if overlap_width > 0 and overlap_height > 0:
            overlap_area = overlap_width * overlap_height
        
        # Total area is sum of individual areas minus overlapping area
        total_area = areaA + areaB - overlap_area
        return total_area

# Example usage:
solution = Solution()
print(solution.computeArea([-3,0,3,4], [0,-1,9,2]))  # Output: 45
print(solution.computeArea([0,0,0,0], [0,0,0,0]))    # Output: 0
print(solution.computeArea([0,0,2,2], [1,1,3,3]))    # Output: 7
print(solution.computeArea([0,0,1,1], [1,0,2,1]))    # Output: 2
\end{lstlisting}
\end{fullwidth}

This implementation accurately computes the total area covered by two rectangles by accounting for any overlapping regions. It ensures that the overlapping area is not double-counted.

\section*{Explanation}

The \texttt{computeArea} function calculates the combined area of two axis-aligned rectangles by following these steps:

\subsection*{1. Calculate Individual Areas}

\begin{itemize}
    \item **Rectangle A:**
    \begin{itemize}
        \item Width: \(A[2] - A[0]\)
        \item Height: \(A[3] - A[1]\)
        \item Area: Width \(\times\) Height
    \end{itemize}
    
    \item **Rectangle B:**
    \begin{itemize}
        \item Width: \(B[2] - B[0]\)
        \item Height: \(B[3] - B[1]\)
        \item Area: Width \(\times\) Height
    \end{itemize}
\end{itemize}

\subsection*{2. Determine Overlapping Area}

\begin{itemize}
    \item **Overlap Coordinates:**
    \begin{itemize}
        \item Left (x-coordinate): \(\text{max}(A[0], B[0])\)
        \item Bottom (y-coordinate): \(\text{max}(A[1], B[1])\)
        \item Right (x-coordinate): \(\text{min}(A[2], B[2])\)
        \item Top (y-coordinate): \(\text{min}(A[3], B[3])\)
    \end{itemize}
    
    \item **Overlap Dimensions:**
    \begin{itemize}
        \item Width: \(\text{overlap\_x2} - \text{overlap\_x1}\)
        \item Height: \(\text{overlap\_y2} - \text{overlap\_y1}\)
    \end{itemize}
    
    \item **Overlap Area:**
    \begin{itemize}
        \item If both width and height are positive, the rectangles overlap, and the overlapping area is their product.
        \item Otherwise, there is no overlap, and the overlapping area is zero.
    \end{itemize}
\end{itemize}

\subsection*{3. Compute Total Area}

\begin{itemize}
    \item Total Area = Area of Rectangle A + Area of Rectangle B - Overlapping Area
\end{itemize}

\subsection*{4. Example Walkthrough}

Consider the first example:
\begin{verbatim}
Input: A = [-3,0,3,4], B = [0,-1,9,2]
Output: 45
\end{verbatim}

\begin{enumerate}
    \item **Calculate Areas:**
    \begin{itemize}
        \item Area of A = (3 - (-3)) * (4 - 0) = 6 * 4 = 24
        \item Area of B = (9 - 0) * (2 - (-1)) = 9 * 3 = 27
    \end{itemize}
    
    \item **Determine Overlap:**
    \begin{itemize}
        \item overlap\_x1 = max(-3, 0) = 0
        \item overlap\_y1 = max(0, -1) = 0
        \item overlap\_x2 = min(3, 9) = 3
        \item overlap\_y2 = min(4, 2) = 2
        \item overlap\_width = 3 - 0 = 3
        \item overlap\_height = 2 - 0 = 2
        \item overlap\_area = 3 * 2 = 6
    \end{itemize}
    
    \item **Compute Total Area:**
    \begin{itemize}
        \item Total Area = 24 + 27 - 6 = 45
    \end{itemize}
\end{enumerate}

Thus, the function correctly returns \texttt{45}.

\section*{Why This Approach}

This approach is chosen for its straightforwardness and optimal efficiency. By directly calculating the individual areas and intelligently handling the overlapping region, the algorithm ensures accurate results without unnecessary computations. Its constant time complexity makes it highly efficient, even for large coordinate values.

\section*{Alternative Approaches}

\subsection*{1. Using Intersection Dimensions}

Instead of separately calculating areas, directly compute the dimensions of the overlapping region and subtract it from the sum of individual areas.

\begin{lstlisting}[language=Python]
def computeArea(A: List[int], B: List[int]) -> int:
    # Sum of individual areas
    area = (A[2] - A[0]) * (A[3] - A[1]) + (B[2] - B[0]) * (B[3] - B[1])
    
    # Overlapping area
    overlap_width = min(A[2], B[2]) - max(A[0], B[0])
    overlap_height = min(A[3], B[3]) - max(A[1], B[1])
    
    if overlap_width > 0 and overlap_height > 0:
        area -= overlap_width * overlap_height
    
    return area
\end{lstlisting}

\subsection*{2. Using Geometry Libraries}

Leverage computational geometry libraries to handle area calculations and overlapping detections.

\begin{lstlisting}[language=Python]
from shapely.geometry import box

def computeArea(A: List[int], B: List[int]) -> int:
    rect1 = box(A[0], A[1], A[2], A[3])
    rect2 = box(B[0], B[1], B[2], B[3])
    intersection = rect1.intersection(rect2)
    return int(rect1.area + rect2.area - intersection.area)
\end{lstlisting}

\textbf{Note}: This approach requires the \texttt{shapely} library and is more suitable for complex geometric operations.

\section*{Similar Problems to This One}

Several problems involve calculating areas, handling geometric overlaps, and spatial reasoning, utilizing similar algorithmic strategies:

\begin{itemize}
    \item \textbf{Rectangle Overlap}: Determine if two rectangles overlap.
    \item \textbf{Circle Area Overlap}: Calculate the overlapping area between two circles.
    \item \textbf{Polygon Area}: Compute the area of a given polygon.
    \item \textbf{Union of Rectangles}: Calculate the total area covered by multiple rectangles, accounting for overlaps.
    \item \textbf{Intersection of Lines}: Find the intersection point of two lines.
    \item \textbf{Closest Pair of Points}: Find the closest pair of points in a set.
    \item \textbf{Convex Hull}: Compute the convex hull of a set of points.
    \item \textbf{Point Inside Polygon}: Determine if a point lies inside a given polygon.
\end{itemize}

These problems reinforce concepts of geometric calculations, area computations, and efficient algorithm design in various contexts.

\section*{Things to Keep in Mind and Tricks}

When tackling the \textbf{Rectangle Area} problem, consider the following tips and best practices to enhance efficiency and correctness:

\begin{itemize}
    \item \textbf{Understand Geometric Relationships}: Grasp the positional relationships between rectangles to simplify area calculations.
    \index{Geometric Relationships}
    
    \item \textbf{Leverage Coordinate Comparisons}: Use direct comparisons of rectangle coordinates to determine overlapping regions.
    \index{Coordinate Comparisons}
    
    \item \textbf{Handle Overlapping Scenarios}: Accurately calculate the overlapping area to avoid double-counting.
    \index{Overlapping Scenarios}
    
    \item \textbf{Optimize for Efficiency}: Aim for a constant time \(O(1)\) solution by avoiding unnecessary computations or iterations.
    \index{Efficiency Optimization}
    
    \item \textbf{Avoid Floating-Point Precision Issues}: Since all coordinates are integers, floating-point precision is not a concern, simplifying the implementation.
    \index{Floating-Point Precision}
    
    \item \textbf{Use Helper Functions}: Create helper functions to encapsulate repetitive tasks, such as calculating overlap dimensions or areas.
    \index{Helper Functions}
    
    \item \textbf{Code Readability}: Maintain clear and readable code through meaningful variable names and structured logic.
    \index{Code Readability}
    
    \item \textbf{Test Extensively}: Implement a wide range of test cases, including overlapping, non-overlapping, and edge-touching rectangles, to ensure robustness.
    \index{Extensive Testing}
    
    \item \textbf{Understand Axis-Aligned Constraints}: Recognize that axis-aligned rectangles simplify area calculations compared to rotated rectangles.
    \index{Axis-Aligned Constraints}
    
    \item \textbf{Simplify Logical Conditions}: Combine multiple conditions logically to streamline the area calculation process.
    \index{Logical Conditions}
    
    \item \textbf{Use Absolute Values}: When calculating differences, ensure that the dimensions are positive by using absolute values or proper ordering.
    \index{Absolute Values}
    
    \item \textbf{Consider Edge Cases}: Handle cases where rectangles have zero area or touch at edges/corners without overlapping.
    \index{Edge Cases}
\end{itemize}

\section*{Corner and Special Cases to Test When Writing the Code}

When implementing the solution for the \textbf{Rectangle Area} problem, it is crucial to consider and rigorously test various edge cases to ensure robustness and correctness:

\begin{itemize}
    \item \textbf{No Overlap}: Rectangles are completely separate.
    \index{No Overlap}
    
    \item \textbf{Partial Overlap}: Rectangles overlap in one or more regions.
    \index{Partial Overlap}
    
    \item \textbf{Edge Touching}: Rectangles touch exactly at one edge without overlapping.
    \index{Edge Touching}
    
    \item \textbf{Corner Touching}: Rectangles touch exactly at one corner without overlapping.
    \index{Corner Touching}
    
    \item \textbf{One Rectangle Inside Another}: One rectangle is entirely within the other.
    \index{Rectangle Inside}
    
    \item \textbf{Identical Rectangles}: Both rectangles have the same coordinates.
    \index{Identical Rectangles}
    
    \item \textbf{Degenerate Rectangles}: Rectangles with zero area (e.g., \(x1 = x2\) or \(y1 = y2\)).
    \index{Degenerate Rectangles}
    
    \item \textbf{Large Coordinates}: Rectangles with very large coordinate values to test performance and integer handling.
    \index{Large Coordinates}
    
    \item \textbf{Negative Coordinates}: Rectangles positioned in negative coordinate space.
    \index{Negative Coordinates}
    
    \item \textbf{Mixed Overlapping Scenarios}: Combinations of the above cases to ensure comprehensive coverage.
    \index{Mixed Overlapping Scenarios}
    
    \item \textbf{Minimum and Maximum Bounds}: Rectangles at the minimum and maximum limits of the coordinate range.
    \index{Minimum and Maximum Bounds}
    
    \item \textbf{Sequential Rectangles}: Multiple rectangles placed sequentially without overlapping.
    \index{Sequential Rectangles}
    
    \item \textbf{Multiple Overlaps}: Scenarios where more than two rectangles overlap in different regions.
    \index{Multiple Overlaps}
\end{itemize}

\section*{Implementation Considerations}

When implementing the \texttt{computeArea} function, keep in mind the following considerations to ensure robustness and efficiency:

\begin{itemize}
    \item \textbf{Data Type Selection}: Use appropriate data types that can handle large input values without overflow or underflow.
    \index{Data Type Selection}
    
    \item \textbf{Optimizing Comparisons}: Structure logical conditions to efficiently determine overlap dimensions.
    \index{Optimizing Comparisons}
    
    \item \textbf{Handling Large Inputs}: Design the algorithm to efficiently handle large input sizes without significant performance degradation.
    \index{Handling Large Inputs}
    
    \item \textbf{Language-Specific Constraints}: Be aware of how the programming language handles large integers and arithmetic operations.
    \index{Language-Specific Constraints}
    
    \item \textbf{Avoiding Redundant Calculations}: Ensure that each calculation contributes towards determining the final area without unnecessary repetitions.
    \index{Avoiding Redundant Calculations}
    
    \item \textbf{Code Readability and Documentation}: Maintain clear and readable code through meaningful variable names and comprehensive comments to facilitate understanding and maintenance.
    \index{Code Readability}
    
    \item \textbf{Edge Case Handling}: Implement checks for edge cases to prevent incorrect results or runtime errors.
    \index{Edge Case Handling}
    
    \item \textbf{Testing and Validation}: Develop a comprehensive suite of test cases that cover all possible scenarios, including edge cases, to validate the correctness and efficiency of the implementation.
    \index{Testing and Validation}
    
    \item \textbf{Scalability}: Design the algorithm to scale efficiently with increasing input sizes, maintaining performance and resource utilization.
    \index{Scalability}
    
    \item \textbf{Using Helper Functions}: Consider creating helper functions for repetitive tasks, such as calculating overlap dimensions, to enhance modularity and reusability.
    \index{Helper Functions}
    
    \item \textbf{Consistent Naming Conventions}: Use consistent and descriptive naming conventions for variables to improve code clarity.
    \index{Naming Conventions}
    
    \item \textbf{Implementing Unit Tests}: Develop unit tests for each logical condition to ensure that all scenarios are correctly handled.
    \index{Unit Tests}
    
    \item \textbf{Error Handling}: Incorporate error handling to manage invalid inputs gracefully.
    \index{Error Handling}
\end{itemize}

\section*{Conclusion}

The \textbf{Rectangle Area} problem showcases the application of fundamental geometric principles and efficient algorithm design to compute spatial properties accurately. By systematically calculating individual areas and intelligently handling overlapping regions, the algorithm ensures precise results without redundant computations. Understanding and implementing such techniques not only enhances problem-solving skills but also provides a foundation for tackling more complex Computational Geometry challenges involving multiple geometric entities and intricate spatial relationships.

\printindex

% \input{sections/rectangle_overlap}
% \input{sections/rectangle_area}
% \input{sections/k_closest_points_to_origin}
% \input{sections/the_skyline_problem}
% % filename: k_closest_points_to_origin.tex

\problemsection{K Closest Points to Origin}
\label{chap:K_Closest_Points_to_Origin}
\marginnote{\href{https://leetcode.com/problems/k-closest-points-to-origin/}{[LeetCode Link]}\index{LeetCode}}
\marginnote{\href{https://www.geeksforgeeks.org/find-k-closest-points-origin/}{[GeeksForGeeks Link]}\index{GeeksForGeeks}}
\marginnote{\href{https://www.interviewbit.com/problems/k-closest-points/}{[InterviewBit Link]}\index{InterviewBit}}
\marginnote{\href{https://app.codesignal.com/challenges/k-closest-points-to-origin}{[CodeSignal Link]}\index{CodeSignal}}
\marginnote{\href{https://www.codewars.com/kata/k-closest-points-to-origin/train/python}{[Codewars Link]}\index{Codewars}}

The \textbf{K Closest Points to Origin} problem is a popular algorithmic challenge in Computational Geometry that involves identifying the \(k\) points closest to the origin in a 2D plane. This problem tests one's ability to apply efficient sorting and selection algorithms, understand distance computations, and optimize for performance. Mastery of this problem is essential for applications in spatial data analysis, nearest neighbor searches, and clustering algorithms.

\section*{Problem Statement}

Given an array of points where each point is represented as \([x, y]\) in the 2D plane, and an integer \(k\), return the \(k\) closest points to the origin \((0, 0)\).

The distance between two points \((x_1, y_1)\) and \((x_2, y_2)\) is the Euclidean distance \(\sqrt{(x_1 - x_2)^2 + (y_1 - y_2)^2}\). The origin is \((0, 0)\).

\textbf{Function signature in Python:}
\begin{lstlisting}[language=Python]
def kClosest(points: List[List[int]], K: int) -> List[List[int]]:
\end{lstlisting}

\section*{Examples}

\textbf{Example 1:}

\begin{verbatim}
Input: points = [[1,3],[-2,2]], K = 1
Output: [[-2,2]]
Explanation: 
The distance between (1, 3) and the origin is sqrt(10).
The distance between (-2, 2) and the origin is sqrt(8).
Since sqrt(8) < sqrt(10), (-2, 2) is closer to the origin.
\end{verbatim}

\textbf{Example 2:}

\begin{verbatim}
Input: points = [[3,3],[5,-1],[-2,4]], K = 2
Output: [[3,3],[-2,4]]
Explanation: 
The distances are sqrt(18), sqrt(26), and sqrt(20) respectively.
The two closest points are [3,3] and [-2,4].
\end{verbatim}

\textbf{Example 3:}

\begin{verbatim}
Input: points = [[0,1],[1,0]], K = 2
Output: [[0,1],[1,0]]
Explanation: 
Both points are equally close to the origin.
\end{verbatim}

\textbf{Example 4:}

\begin{verbatim}
Input: points = [[1,0],[0,1]], K = 1
Output: [[1,0]]
Explanation: 
Both points are equally close; returning any one is acceptable.
\end{verbatim}

\textbf{Constraints:}

\begin{itemize}
    \item \(1 \leq K \leq \text{points.length} \leq 10^4\)
    \item \(-10^4 < x_i, y_i < 10^4\)
\end{itemize}

LeetCode link: \href{https://leetcode.com/problems/k-closest-points-to-origin/}{K Closest Points to Origin}\index{LeetCode}

\section*{Algorithmic Approach}

To identify the \(k\) closest points to the origin, several algorithmic strategies can be employed. The most efficient methods aim to reduce the time complexity by avoiding the need to sort the entire list of points.

\subsection*{1. Sorting Based on Distance}

Calculate the Euclidean distance of each point from the origin and sort the points based on these distances. Select the first \(k\) points from the sorted list.

\begin{enumerate}
    \item Compute the distance for each point using the formula \(distance = x^2 + y^2\).
    \item Sort the points based on the computed distances.
    \item Return the first \(k\) points from the sorted list.
\end{enumerate}

\subsection*{2. Max Heap (Priority Queue)}

Use a max heap to maintain the \(k\) closest points. Iterate through each point, add it to the heap, and if the heap size exceeds \(k\), remove the farthest point.

\begin{enumerate}
    \item Initialize a max heap.
    \item For each point, compute its distance and add it to the heap.
    \item If the heap size exceeds \(k\), remove the point with the largest distance.
    \item After processing all points, the heap contains the \(k\) closest points.
\end{enumerate}

\subsection*{3. QuickSelect (Quick Sort Partitioning)}

Utilize the QuickSelect algorithm to find the \(k\) closest points without fully sorting the list.

\begin{enumerate}
    \item Choose a pivot point and partition the list based on distances relative to the pivot.
    \item Recursively apply QuickSelect to the partition containing the \(k\) closest points.
    \item Once the \(k\) closest points are identified, return them.
\end{enumerate}

\marginnote{QuickSelect offers an average time complexity of \(O(n)\), making it highly efficient for large datasets.}

\section*{Complexities}

\begin{itemize}
    \item \textbf{Sorting Based on Distance:}
    \begin{itemize}
        \item \textbf{Time Complexity:} \(O(n \log n)\)
        \item \textbf{Space Complexity:} \(O(n)\)
    \end{itemize}
    
    \item \textbf{Max Heap (Priority Queue):}
    \begin{itemize}
        \item \textbf{Time Complexity:} \(O(n \log k)\)
        \item \textbf{Space Complexity:} \(O(k)\)
    \end{itemize}
    
    \item \textbf{QuickSelect (Quick Sort Partitioning):}
    \begin{itemize}
        \item \textbf{Time Complexity:} Average case \(O(n)\), worst case \(O(n^2)\)
        \item \textbf{Space Complexity:} \(O(1)\) (in-place)
    \end{itemize}
\end{itemize}

\section*{Python Implementation}

\marginnote{Implementing QuickSelect provides an optimal average-case solution with linear time complexity.}

Below is the complete Python code implementing the \texttt{kClosest} function using the QuickSelect approach:

\begin{fullwidth}
\begin{lstlisting}[language=Python]
from typing import List
import random

class Solution:
    def kClosest(self, points: List[List[int]], K: int) -> List[List[int]]:
        def quickselect(left, right, K_smallest):
            if left == right:
                return
            
            # Select a random pivot_index
            pivot_index = random.randint(left, right)
            
            # Partition the array
            pivot_index = partition(left, right, pivot_index)
            
            # The pivot is in its final sorted position
            if K_smallest == pivot_index:
                return
            elif K_smallest < pivot_index:
                quickselect(left, pivot_index - 1, K_smallest)
            else:
                quickselect(pivot_index + 1, right, K_smallest)
        
        def partition(left, right, pivot_index):
            pivot_distance = distance(points[pivot_index])
            # Move pivot to end
            points[pivot_index], points[right] = points[right], points[pivot_index]
            store_index = left
            for i in range(left, right):
                if distance(points[i]) < pivot_distance:
                    points[store_index], points[i] = points[i], points[store_index]
                    store_index += 1
            # Move pivot to its final place
            points[right], points[store_index] = points[store_index], points[right]
            return store_index
        
        def distance(point):
            return point[0] ** 2 + point[1] ** 2
        
        n = len(points)
        quickselect(0, n - 1, K)
        return points[:K]

# Example usage:
solution = Solution()
print(solution.kClosest([[1,3],[-2,2]], 1))            # Output: [[-2,2]]
print(solution.kClosest([[3,3],[5,-1],[-2,4]], 2))     # Output: [[3,3],[-2,4]]
print(solution.kClosest([[0,1],[1,0]], 2))             # Output: [[0,1],[1,0]]
print(solution.kClosest([[1,0],[0,1]], 1))             # Output: [[1,0]] or [[0,1]]
\end{lstlisting}
\end{fullwidth}

This implementation uses the QuickSelect algorithm to efficiently find the \(k\) closest points to the origin without fully sorting the entire list. It ensures optimal performance even with large datasets.

\section*{Explanation}

The \texttt{kClosest} function identifies the \(k\) closest points to the origin using the QuickSelect algorithm. Here's a detailed breakdown of the implementation:

\subsection*{1. Distance Calculation}

\begin{itemize}
    \item The Euclidean distance is calculated as \(distance = x^2 + y^2\). Since we only need relative distances for comparison, the square root is omitted for efficiency.
\end{itemize}

\subsection*{2. QuickSelect Algorithm}

\begin{itemize}
    \item **Pivot Selection:**
    \begin{itemize}
        \item A random pivot is chosen to enhance the average-case performance.
    \end{itemize}
    
    \item **Partitioning:**
    \begin{itemize}
        \item The array is partitioned such that points with distances less than the pivot are moved to the left, and others to the right.
        \item The pivot is placed in its correct sorted position.
    \end{itemize}
    
    \item **Recursive Selection:**
    \begin{itemize}
        \item If the pivot's position matches \(K\), the selection is complete.
        \item Otherwise, recursively apply QuickSelect to the relevant partition.
    \end{itemize}
\end{itemize}

\subsection*{3. Final Selection}

\begin{itemize}
    \item After partitioning, the first \(K\) points in the list are the \(k\) closest points to the origin.
\end{itemize}

\subsection*{4. Example Walkthrough}

Consider the first example:
\begin{verbatim}
Input: points = [[1,3],[-2,2]], K = 1
Output: [[-2,2]]
\end{verbatim}

\begin{enumerate}
    \item **Calculate Distances:**
    \begin{itemize}
        \item [1,3] : \(1^2 + 3^2 = 10\)
        \item [-2,2] : \((-2)^2 + 2^2 = 8\)
    \end{itemize}
    
    \item **QuickSelect Process:**
    \begin{itemize}
        \item Choose a pivot, say [1,3] with distance 10.
        \item Compare and rearrange:
        \begin{itemize}
            \item [-2,2] has a smaller distance (8) and is moved to the left.
        \end{itemize}
        \item After partitioning, the list becomes [[-2,2], [1,3]].
        \item Since \(K = 1\), return the first point: [[-2,2]].
    \end{itemize}
\end{enumerate}

Thus, the function correctly identifies \([-2,2]\) as the closest point to the origin.

\section*{Why This Approach}

The QuickSelect algorithm is chosen for its average-case linear time complexity \(O(n)\), making it highly efficient for large datasets compared to sorting-based methods with \(O(n \log n)\) time complexity. By avoiding the need to sort the entire list, QuickSelect provides an optimal solution for finding the \(k\) closest points.

\section*{Alternative Approaches}

\subsection*{1. Sorting Based on Distance}

Sort all points based on their distances from the origin and select the first \(k\) points.

\begin{lstlisting}[language=Python]
class Solution:
    def kClosest(self, points: List[List[int]], K: int) -> List[List[int]]:
        points.sort(key=lambda P: P[0]**2 + P[1]**2)
        return points[:K]
\end{lstlisting}

\textbf{Complexities:}
\begin{itemize}
    \item \textbf{Time Complexity:} \(O(n \log n)\)
    \item \textbf{Space Complexity:} \(O(1)\)
\end{itemize}

\subsection*{2. Max Heap (Priority Queue)}

Use a max heap to maintain the \(k\) closest points.

\begin{lstlisting}[language=Python]
import heapq

class Solution:
    def kClosest(self, points: List[List[int]], K: int) -> List[List[int]]:
        heap = []
        for (x, y) in points:
            dist = -(x**2 + y**2)  # Max heap using negative distances
            heapq.heappush(heap, (dist, [x, y]))
            if len(heap) > K:
                heapq.heappop(heap)
        return [item[1] for item in heap]
\end{lstlisting}

\textbf{Complexities:}
\begin{itemize}
    \item \textbf{Time Complexity:} \(O(n \log k)\)
    \item \textbf{Space Complexity:} \(O(k)\)
\end{itemize}

\subsection*{3. Using Built-In Functions}

Leverage built-in functions for distance calculation and selection.

\begin{lstlisting}[language=Python]
import math

class Solution:
    def kClosest(self, points: List[List[int]], K: int) -> List[List[int]]:
        points.sort(key=lambda P: math.sqrt(P[0]**2 + P[1]**2))
        return points[:K]
\end{lstlisting}

\textbf{Note}: This method is similar to the sorting approach but uses the actual Euclidean distance.

\section*{Similar Problems to This One}

Several problems involve nearest neighbor searches, spatial data analysis, and efficient selection algorithms, utilizing similar algorithmic strategies:

\begin{itemize}
    \item \textbf{Closest Pair of Points}: Find the closest pair of points in a set.
    \item \textbf{Top K Frequent Elements}: Identify the most frequent elements in a dataset.
    \item \textbf{Kth Largest Element in an Array}: Find the \(k\)-th largest element in an unsorted array.
    \item \textbf{Sliding Window Maximum}: Find the maximum in each sliding window of size \(k\) over an array.
    \item \textbf{Merge K Sorted Lists}: Merge multiple sorted lists into a single sorted list.
    \item \textbf{Find Median from Data Stream}: Continuously find the median of a stream of numbers.
    \item \textbf{Top K Closest Stars}: Find the \(k\) closest stars to Earth based on their distances.
\end{itemize}

These problems reinforce concepts of efficient selection, heap usage, and distance computations in various contexts.

\section*{Things to Keep in Mind and Tricks}

When solving the \textbf{K Closest Points to Origin} problem, consider the following tips and best practices to enhance efficiency and correctness:

\begin{itemize}
    \item \textbf{Understand Distance Calculations}: Grasp the Euclidean distance formula and recognize that the square root can be omitted for comparison purposes.
    \index{Distance Calculations}
    
    \item \textbf{Leverage Efficient Algorithms}: Use QuickSelect or heap-based methods to optimize time complexity, especially for large datasets.
    \index{Efficient Algorithms}
    
    \item \textbf{Handle Ties Appropriately}: Decide how to handle points with identical distances when \(k\) is less than the number of such points.
    \index{Handling Ties}
    
    \item \textbf{Optimize Space Usage}: Choose algorithms that minimize additional space, such as in-place QuickSelect.
    \index{Space Optimization}
    
    \item \textbf{Use Appropriate Data Structures}: Utilize heaps, lists, and helper functions effectively to manage and process data.
    \index{Data Structures}
    
    \item \textbf{Implement Helper Functions}: Create helper functions for distance calculation and partitioning to enhance code modularity.
    \index{Helper Functions}
    
    \item \textbf{Code Readability}: Maintain clear and readable code through meaningful variable names and structured logic.
    \index{Code Readability}
    
    \item \textbf{Test Extensively}: Implement a wide range of test cases, including edge cases like multiple points with the same distance, to ensure robustness.
    \index{Extensive Testing}
    
    \item \textbf{Understand Algorithm Trade-offs}: Recognize the trade-offs between different approaches in terms of time and space complexities.
    \index{Algorithm Trade-offs}
    
    \item \textbf{Use Built-In Sorting Functions}: When using sorting-based approaches, leverage built-in functions for efficiency and simplicity.
    \index{Built-In Sorting}
    
    \item \textbf{Avoid Redundant Calculations}: Ensure that distance calculations are performed only when necessary to optimize performance.
    \index{Avoiding Redundant Calculations}
    
    \item \textbf{Language-Specific Features}: Utilize language-specific features or libraries that can simplify implementation, such as heapq in Python.
    \index{Language-Specific Features}
\end{itemize}

\section*{Corner and Special Cases to Test When Writing the Code}

When implementing the solution for the \textbf{K Closest Points to Origin} problem, it is crucial to consider and rigorously test various edge cases to ensure robustness and correctness:

\begin{itemize}
    \item \textbf{Multiple Points with Same Distance}: Ensure that the algorithm handles multiple points having the same distance from the origin.
    \index{Same Distance Points}
    
    \item \textbf{Points at Origin}: Include points that are exactly at the origin \((0,0)\).
    \index{Points at Origin}
    
    \item \textbf{Negative Coordinates}: Ensure that the algorithm correctly computes distances for points with negative \(x\) or \(y\) coordinates.
    \index{Negative Coordinates}
    
    \item \textbf{Large Coordinates}: Test with points having very large or very small coordinate values to verify integer handling.
    \index{Large Coordinates}
    
    \item \textbf{K Equals Number of Points}: When \(K\) is equal to the number of points, the algorithm should return all points.
    \index{K Equals Number of Points}
    
    \item \textbf{K is One}: Test with \(K = 1\) to ensure the closest point is correctly identified.
    \index{K is One}
    
    \item \textbf{All Points Same}: All points have the same coordinates.
    \index{All Points Same}
    
    \item \textbf{K is Zero}: Although \(K\) is defined to be at least 1, ensure that the algorithm gracefully handles \(K = 0\) if allowed.
    \index{K is Zero}
    
    \item \textbf{Single Point}: Only one point is provided, and \(K = 1\).
    \index{Single Point}
    
    \item \textbf{Mixed Coordinates}: Points with a mix of positive and negative coordinates.
    \index{Mixed Coordinates}
    
    \item \textbf{Points with Zero Distance}: Multiple points at the origin.
    \index{Zero Distance Points}
    
    \item \textbf{Sparse and Dense Points}: Densely packed points and sparsely distributed points.
    \index{Sparse and Dense Points}
    
    \item \textbf{Duplicate Points}: Multiple identical points in the input list.
    \index{Duplicate Points}
    
    \item \textbf{K Greater Than Number of Unique Points}: Ensure that the algorithm handles cases where \(K\) exceeds the number of unique points if applicable.
    \index{K Greater Than Unique Points}
\end{itemize}

\section*{Implementation Considerations}

When implementing the \texttt{kClosest} function, keep in mind the following considerations to ensure robustness and efficiency:

\begin{itemize}
    \item \textbf{Data Type Selection}: Use appropriate data types that can handle large input values without overflow or precision loss.
    \index{Data Type Selection}
    
    \item \textbf{Optimizing Distance Calculations}: Avoid calculating the square root since it is unnecessary for comparison purposes.
    \index{Optimizing Distance Calculations}
    
    \item \textbf{Choosing the Right Algorithm}: Select an algorithm based on the size of the input and the value of \(K\) to optimize time and space complexities.
    \index{Choosing the Right Algorithm}
    
    \item \textbf{Language-Specific Libraries}: Utilize language-specific libraries and functions (e.g., \texttt{heapq} in Python) to simplify implementation and enhance performance.
    \index{Language-Specific Libraries}
    
    \item \textbf{Avoiding Redundant Calculations}: Ensure that each point's distance is calculated only once to optimize performance.
    \index{Avoiding Redundant Calculations}
    
    \item \textbf{Implementing Helper Functions}: Create helper functions for tasks like distance calculation and partitioning to enhance modularity and readability.
    \index{Helper Functions}
    
    \item \textbf{Edge Case Handling}: Implement checks for edge cases to prevent incorrect results or runtime errors.
    \index{Edge Case Handling}
    
    \item \textbf{Testing and Validation}: Develop a comprehensive suite of test cases that cover all possible scenarios, including edge cases, to validate the correctness and efficiency of the implementation.
    \index{Testing and Validation}
    
    \item \textbf{Scalability}: Design the algorithm to scale efficiently with increasing input sizes, maintaining performance and resource utilization.
    \index{Scalability}
    
    \item \textbf{Consistent Naming Conventions}: Use consistent and descriptive naming conventions for variables and functions to improve code clarity.
    \index{Naming Conventions}
    
    \item \textbf{Memory Management}: Ensure that the algorithm manages memory efficiently, especially when dealing with large datasets.
    \index{Memory Management}
    
    \item \textbf{Avoiding Stack Overflow}: If implementing recursive approaches, be mindful of recursion limits and potential stack overflow issues.
    \index{Avoiding Stack Overflow}
    
    \item \textbf{Implementing Iterative Solutions}: Prefer iterative solutions when recursion may lead to increased space complexity or stack overflow.
    \index{Implementing Iterative Solutions}
\end{itemize}

\section*{Conclusion}

The \textbf{K Closest Points to Origin} problem exemplifies the application of efficient selection algorithms and geometric computations to solve spatial challenges effectively. By leveraging QuickSelect or heap-based methods, the algorithm achieves optimal time and space complexities, making it highly suitable for large datasets. Understanding and implementing such techniques not only enhances problem-solving skills but also provides a foundation for tackling more advanced Computational Geometry problems involving nearest neighbor searches, clustering, and spatial data analysis.

\printindex

% \input{sections/rectangle_overlap}
% \input{sections/rectangle_area}
% \input{sections/k_closest_points_to_origin}
% \input{sections/the_skyline_problem}
% % filename: the_skyline_problem.tex

\problemsection{The Skyline Problem}
\label{chap:The_Skyline_Problem}
\marginnote{\href{https://leetcode.com/problems/the-skyline-problem/}{[LeetCode Link]}\index{LeetCode}}
\marginnote{\href{https://www.geeksforgeeks.org/the-skyline-problem/}{[GeeksForGeeks Link]}\index{GeeksForGeeks}}
\marginnote{\href{https://www.interviewbit.com/problems/the-skyline-problem/}{[InterviewBit Link]}\index{InterviewBit}}
\marginnote{\href{https://app.codesignal.com/challenges/the-skyline-problem}{[CodeSignal Link]}\index{CodeSignal}}
\marginnote{\href{https://www.codewars.com/kata/the-skyline-problem/train/python}{[Codewars Link]}\index{Codewars}}

The \textbf{Skyline Problem} is a complex Computational Geometry challenge that involves computing the skyline formed by a collection of buildings in a 2D cityscape. Each building is represented by its left and right x-coordinates and its height. The skyline is defined by a list of "key points" where the height changes. This problem tests one's ability to handle large datasets, implement efficient sweep line algorithms, and manage event-driven processing. Mastery of this problem is essential for applications in computer graphics, urban planning simulations, and geographic information systems (GIS).

\section*{Problem Statement}

You are given a list of buildings in a cityscape. Each building is represented as a triplet \([Li, Ri, Hi]\), where \(Li\) and \(Ri\) are the x-coordinates of the left and right edges of the building, respectively, and \(Hi\) is the height of the building.

The skyline should be represented as a list of key points \([x, y]\) in sorted order by \(x\)-coordinate, where \(y\) is the height of the skyline at that point. The skyline should only include critical points where the height changes.

\textbf{Function signature in Python:}
\begin{lstlisting}[language=Python]
def getSkyline(buildings: List[List[int]]) -> List[List[int]]:
\end{lstlisting}

\section*{Examples}

\textbf{Example 1:}

\begin{verbatim}
Input: buildings = [[2,9,10], [3,7,15], [5,12,12], [15,20,10], [19,24,8]]
Output: [[2,10], [3,15], [7,12], [12,0], [15,10], [20,8], [24,0]]
Explanation:
- At x=2, the first building starts, height=10.
- At x=3, the second building starts, height=15.
- At x=7, the second building ends, the third building is still ongoing, height=12.
- At x=12, the third building ends, height drops to 0.
- At x=15, the fourth building starts, height=10.
- At x=20, the fourth building ends, the fifth building is still ongoing, height=8.
- At x=24, the fifth building ends, height drops to 0.
\end{verbatim}

\textbf{Example 2:}

\begin{verbatim}
Input: buildings = [[0,2,3], [2,5,3]]
Output: [[0,3], [5,0]]
Explanation:
- The two buildings are contiguous and have the same height, so the skyline drops to 0 at x=5.
\end{verbatim}

\textbf{Example 3:}

\begin{verbatim}
Input: buildings = [[1,3,3], [2,4,4], [5,6,1]]
Output: [[1,3], [2,4], [4,0], [5,1], [6,0]]
Explanation:
- At x=1, first building starts, height=3.
- At x=2, second building starts, height=4.
- At x=4, second building ends, height drops to 0.
- At x=5, third building starts, height=1.
- At x=6, third building ends, height drops to 0.
\end{verbatim}

\textbf{Example 4:}

\begin{verbatim}
Input: buildings = [[0,5,0]]
Output: []
Explanation:
- A building with height 0 does not contribute to the skyline.
\end{verbatim}

\textbf{Constraints:}

\begin{itemize}
    \item \(1 \leq \text{buildings.length} \leq 10^4\)
    \item \(0 \leq Li < Ri \leq 10^9\)
    \item \(0 \leq Hi \leq 10^4\)
\end{itemize}

\section*{Algorithmic Approach}

The \textbf{Sweep Line Algorithm} is an efficient method for solving the Skyline Problem. It involves processing events (building start and end points) in sorted order while maintaining a data structure (typically a max heap) to keep track of active buildings. Here's a step-by-step approach:

\subsection*{1. Event Representation}

Transform each building into two events:
\begin{itemize}
    \item **Start Event:** \((Li, -Hi)\) – Negative height indicates a building starts.
    \item **End Event:** \((Ri, Hi)\) – Positive height indicates a building ends.
\end{itemize}

Sorting the events ensures that start events are processed before end events at the same x-coordinate, and taller buildings are processed before shorter ones.

\subsection*{2. Sorting the Events}

Sort all events based on:
\begin{enumerate}
    \item **x-coordinate:** Ascending order.
    \item **Height:**
    \begin{itemize}
        \item For start events, taller buildings come first.
        \item For end events, shorter buildings come first.
    \end{itemize}
\end{enumerate}

\subsection*{3. Processing the Events}

Use a max heap to keep track of active building heights. Iterate through the sorted events:
\begin{enumerate}
    \item **Start Event:**
    \begin{itemize}
        \item Add the building's height to the heap.
    \end{itemize}
    
    \item **End Event:**
    \begin{itemize}
        \item Remove the building's height from the heap.
    \end{itemize}
    
    \item **Determine Current Max Height:**
    \begin{itemize}
        \item The current max height is the top of the heap.
    \end{itemize}
    
    \item **Update Skyline:**
    \begin{itemize}
        \item If the current max height differs from the previous max height, add a new key point \([x, current\_max\_height]\).
    \end{itemize}
\end{enumerate}

\subsection*{4. Finalizing the Skyline}

After processing all events, the accumulated key points represent the skyline.

\marginnote{The Sweep Line Algorithm efficiently handles dynamic changes in active buildings, ensuring accurate skyline construction.}

\section*{Complexities}

\begin{itemize}
    \item \textbf{Time Complexity:} \(O(n \log n)\), where \(n\) is the number of buildings. Sorting the events takes \(O(n \log n)\), and each heap operation takes \(O(\log n)\).
    
    \item \textbf{Space Complexity:} \(O(n)\), due to the storage of events and the heap.
\end{itemize}

\section*{Python Implementation}

\marginnote{Implementing the Sweep Line Algorithm with a max heap ensures an efficient and accurate solution.}

Below is the complete Python code implementing the \texttt{getSkyline} function:

\begin{fullwidth}
\begin{lstlisting}[language=Python]
from typing import List
import heapq

class Solution:
    def getSkyline(self, buildings: List[List[int]]) -> List[List[int]]:
        # Create a list of all events
        # For start events, use negative height to ensure they are processed before end events
        events = []
        for L, R, H in buildings:
            events.append((L, -H))
            events.append((R, H))
        
        # Sort the events
        # First by x-coordinate, then by height
        events.sort()
        
        # Max heap to keep track of active buildings
        heap = [0]  # Initialize with ground level
        heapq.heapify(heap)
        active_heights = {0: 1}  # Dictionary to count heights
        
        result = []
        prev_max = 0
        
        for x, h in events:
            if h < 0:
                # Start of a building, add height to heap and dictionary
                heapq.heappush(heap, h)
                active_heights[h] = active_heights.get(h, 0) + 1
            else:
                # End of a building, remove height from dictionary
                active_heights[h] -= 1
                if active_heights[h] == 0:
                    del active_heights[h]
            
            # Current max height
            while heap and active_heights.get(heap[0], 0) == 0:
                heapq.heappop(heap)
            current_max = -heap[0] if heap else 0
            
            # If the max height has changed, add to result
            if current_max != prev_max:
                result.append([x, current_max])
                prev_max = current_max
        
        return result

# Example usage:
solution = Solution()
print(solution.getSkyline([[2,9,10], [3,7,15], [5,12,12], [15,20,10], [19,24,8]]))
# Output: [[2,10], [3,15], [7,12], [12,0], [15,10], [20,8], [24,0]]

print(solution.getSkyline([[0,2,3], [2,5,3]]))
# Output: [[0,3], [5,0]]

print(solution.getSkyline([[1,3,3], [2,4,4], [5,6,1]]))
# Output: [[1,3], [2,4], [4,0], [5,1], [6,0]]

print(solution.getSkyline([[0,5,0]]))
# Output: []
\end{lstlisting}
\end{fullwidth}

This implementation efficiently constructs the skyline by processing all building events in sorted order and maintaining active building heights using a max heap. It ensures that only critical points where the skyline changes are recorded.

\section*{Explanation}

The \texttt{getSkyline} function constructs the skyline formed by a set of buildings by leveraging the Sweep Line Algorithm and a max heap to track active buildings. Here's a detailed breakdown of the implementation:

\subsection*{1. Event Representation}

\begin{itemize}
    \item Each building is transformed into two events:
    \begin{itemize}
        \item **Start Event:** \((Li, -Hi)\) – Negative height indicates the start of a building.
        \item **End Event:** \((Ri, Hi)\) – Positive height indicates the end of a building.
    \end{itemize}
\end{itemize}

\subsection*{2. Sorting the Events}

\begin{itemize}
    \item Events are sorted primarily by their x-coordinate in ascending order.
    \item For events with the same x-coordinate:
    \begin{itemize}
        \item Start events (with negative heights) are processed before end events.
        \item Taller buildings are processed before shorter ones.
    \end{itemize}
\end{itemize}

\subsection*{3. Processing the Events}

\begin{itemize}
    \item **Heap Initialization:**
    \begin{itemize}
        \item A max heap is initialized with a ground level height of 0.
        \item A dictionary \texttt{active\_heights} tracks the count of active building heights.
    \end{itemize}
    
    \item **Iterating Through Events:**
    \begin{enumerate}
        \item **Start Event:**
        \begin{itemize}
            \item Add the building's height to the heap.
            \item Increment the count of the height in \texttt{active\_heights}.
        \end{itemize}
        
        \item **End Event:**
        \begin{itemize}
            \item Decrement the count of the building's height in \texttt{active\_heights}.
            \item If the count reaches zero, remove the height from the dictionary.
        \end{itemize}
        
        \item **Determine Current Max Height:**
        \begin{itemize}
            \item Remove heights from the heap that are no longer active.
            \item The current max height is the top of the heap.
        \end{itemize}
        
        \item **Update Skyline:**
        \begin{itemize}
            \item If the current max height differs from the previous max height, add a new key point \([x, current\_max\_height]\).
        \end{itemize}
    \end{enumerate}
\end{itemize}

\subsection*{4. Finalizing the Skyline}

\begin{itemize}
    \item After processing all events, the \texttt{result} list contains the key points defining the skyline.
\end{itemize}

\subsection*{5. Example Walkthrough}

Consider the first example:
\begin{verbatim}
Input: buildings = [[2,9,10], [3,7,15], [5,12,12], [15,20,10], [19,24,8]]
Output: [[2,10], [3,15], [7,12], [12,0], [15,10], [20,8], [24,0]]
\end{verbatim}

\begin{enumerate}
    \item **Event Transformation:**
    \begin{itemize}
        \item \((2, -10)\), \((9, 10)\)
        \item \((3, -15)\), \((7, 15)\)
        \item \((5, -12)\), \((12, 12)\)
        \item \((15, -10)\), \((20, 10)\)
        \item \((19, -8)\), \((24, 8)\)
    \end{itemize}
    
    \item **Sorting Events:**
    \begin{itemize}
        \item Sorted order: \((2, -10)\), \((3, -15)\), \((5, -12)\), \((7, 15)\), \((9, 10)\), \((12, 12)\), \((15, -10)\), \((19, -8)\), \((20, 10)\), \((24, 8)\)
    \end{itemize}
    
    \item **Processing Events:**
    \begin{itemize}
        \item At each event, update the heap and determine if the skyline height changes.
    \end{itemize}
    
    \item **Result Construction:**
    \begin{itemize}
        \item The resulting skyline key points are accumulated as \([[2,10], [3,15], [7,12], [12,0], [15,10], [20,8], [24,0]]\).
    \end{itemize}
\end{enumerate}

Thus, the function correctly constructs the skyline based on the buildings' positions and heights.

\section*{Why This Approach}

The Sweep Line Algorithm combined with a max heap offers an optimal solution with \(O(n \log n)\) time complexity and efficient handling of overlapping buildings. By processing events in sorted order and maintaining active building heights, the algorithm ensures that all critical points in the skyline are accurately identified without redundant computations.

\section*{Alternative Approaches}

\subsection*{1. Divide and Conquer}

Divide the set of buildings into smaller subsets, compute the skyline for each subset, and then merge the skylines.

\begin{lstlisting}[language=Python]
class Solution:
    def getSkyline(self, buildings: List[List[int]]) -> List[List[int]]:
        def merge(left, right):
            h1, h2 = 0, 0
            i, j = 0, 0
            merged = []
            while i < len(left) and j < len(right):
                if left[i][0] < right[j][0]:
                    x, h1 = left[i]
                    i += 1
                elif left[i][0] > right[j][0]:
                    x, h2 = right[j]
                    j += 1
                else:
                    x, h1 = left[i]
                    _, h2 = right[j]
                    i += 1
                    j += 1
                max_h = max(h1, h2)
                if not merged or merged[-1][1] != max_h:
                    merged.append([x, max_h])
            merged.extend(left[i:])
            merged.extend(right[j:])
            return merged
        
        def divide(buildings):
            if not buildings:
                return []
            if len(buildings) == 1:
                L, R, H = buildings[0]
                return [[L, H], [R, 0]]
            mid = len(buildings) // 2
            left = divide(buildings[:mid])
            right = divide(buildings[mid:])
            return merge(left, right)
        
        return divide(buildings)
\end{lstlisting}

\textbf{Complexities:}
\begin{itemize}
    \item \textbf{Time Complexity:} \(O(n \log n)\)
    \item \textbf{Space Complexity:} \(O(n)\)
\end{itemize}

\subsection*{2. Using Segment Trees}

Implement a segment tree to manage and query overlapping building heights dynamically.

\textbf{Note}: This approach is more complex and is generally used for advanced scenarios with multiple dynamic queries.

\section*{Similar Problems to This One}

Several problems involve skyline-like constructions, spatial data analysis, and efficient event processing, utilizing similar algorithmic strategies:

\begin{itemize}
    \item \textbf{Merge Intervals}: Merge overlapping intervals in a list.
    \item \textbf{Largest Rectangle in Histogram}: Find the largest rectangular area in a histogram.
    \item \textbf{Interval Partitioning}: Assign intervals to resources without overlap.
    \item \textbf{Line Segment Intersection}: Detect intersections among line segments.
    \item \textbf{Closest Pair of Points}: Find the closest pair of points in a set.
    \item \textbf{Convex Hull}: Compute the convex hull of a set of points.
    \item \textbf{Point Inside Polygon}: Determine if a point lies inside a given polygon.
    \item \textbf{Range Searching}: Efficiently query geometric data within a specified range.
\end{itemize}

These problems reinforce concepts of event-driven processing, spatial reasoning, and efficient algorithm design in various contexts.

\section*{Things to Keep in Mind and Tricks}

When tackling the \textbf{Skyline Problem}, consider the following tips and best practices to enhance efficiency and correctness:

\begin{itemize}
    \item \textbf{Understand Sweep Line Technique}: Grasp how the sweep line algorithm processes events in sorted order to handle dynamic changes efficiently.
    \index{Sweep Line Technique}
    
    \item \textbf{Leverage Priority Queues (Heaps)}: Use max heaps to keep track of active buildings' heights, enabling quick access to the current maximum height.
    \index{Priority Queues}
    
    \item \textbf{Handle Start and End Events Differently}: Differentiate between building start and end events to accurately manage active heights.
    \index{Start and End Events}
    
    \item \textbf{Optimize Event Sorting}: Sort events primarily by x-coordinate and secondarily by height to ensure correct processing order.
    \index{Event Sorting}
    
    \item \textbf{Manage Active Heights Efficiently}: Use data structures that allow efficient insertion, deletion, and retrieval of maximum elements.
    \index{Active Heights Management}
    
    \item \textbf{Avoid Redundant Key Points}: Only record key points when the skyline height changes to minimize the output list.
    \index{Avoiding Redundant Key Points}
    
    \item \textbf{Implement Helper Functions}: Create helper functions for tasks like distance calculation, event handling, and heap management to enhance modularity.
    \index{Helper Functions}
    
    \item \textbf{Code Readability}: Maintain clear and readable code through meaningful variable names and structured logic.
    \index{Code Readability}
    
    \item \textbf{Test Extensively}: Implement a wide range of test cases, including overlapping, non-overlapping, and edge-touching buildings, to ensure robustness.
    \index{Extensive Testing}
    
    \item \textbf{Handle Degenerate Cases}: Manage cases where buildings have zero height or identical coordinates gracefully.
    \index{Degenerate Cases}
    
    \item \textbf{Understand Geometric Relationships}: Grasp how buildings overlap and influence the skyline to simplify the algorithm.
    \index{Geometric Relationships}
    
    \item \textbf{Use Appropriate Data Structures}: Utilize appropriate data structures like heaps, lists, and dictionaries to manage and process data efficiently.
    \index{Appropriate Data Structures}
    
    \item \textbf{Optimize for Large Inputs}: Design the algorithm to handle large numbers of buildings without significant performance degradation.
    \index{Optimizing for Large Inputs}
    
    \item \textbf{Implement Iterative Solutions Carefully}: Ensure that loop conditions are correctly defined to prevent infinite loops or incorrect terminations.
    \index{Iterative Solutions}
    
    \item \textbf{Consistent Naming Conventions}: Use consistent and descriptive naming conventions for variables and functions to improve code clarity.
    \index{Naming Conventions}
\end{itemize}

\section*{Corner and Special Cases to Test When Writing the Code}

When implementing the solution for the \textbf{Skyline Problem}, it is crucial to consider and rigorously test various edge cases to ensure robustness and correctness:

\begin{itemize}
    \item \textbf{No Overlapping Buildings}: All buildings are separate and do not overlap.
    \index{No Overlapping Buildings}
    
    \item \textbf{Fully Overlapping Buildings}: Multiple buildings completely overlap each other.
    \index{Fully Overlapping Buildings}
    
    \item \textbf{Buildings Touching at Edges}: Buildings share common edges without overlapping.
    \index{Buildings Touching at Edges}
    
    \item \textbf{Buildings Touching at Corners}: Buildings share common corners without overlapping.
    \index{Buildings Touching at Corners}
    
    \item \textbf{Single Building}: Only one building is present.
    \index{Single Building}
    
    \item \textbf{Multiple Buildings with Same Start or End}: Multiple buildings start or end at the same x-coordinate.
    \index{Same Start or End}
    
    \item \textbf{Buildings with Zero Height}: Buildings that have zero height should not affect the skyline.
    \index{Buildings with Zero Height}
    
    \item \textbf{Large Number of Buildings}: Test with a large number of buildings to ensure performance and scalability.
    \index{Large Number of Buildings}
    
    \item \textbf{Buildings with Negative Coordinates}: Buildings positioned in negative coordinate space.
    \index{Negative Coordinates}
    
    \item \textbf{Boundary Values}: Buildings at the minimum and maximum limits of the coordinate range.
    \index{Boundary Values}
    
    \item \textbf{Buildings with Identical Coordinates}: Multiple buildings with the same coordinates.
    \index{Identical Coordinates}
    
    \item \textbf{Sequential Buildings}: Buildings placed sequentially without gaps.
    \index{Sequential Buildings}
    
    \item \textbf{Overlapping and Non-Overlapping Mixed}: A mix of overlapping and non-overlapping buildings.
    \index{Overlapping and Non-Overlapping Mixed}
    
    \item \textbf{Buildings with Very Large Heights}: Buildings with heights at the upper limit of the constraints.
    \index{Very Large Heights}
    
    \item \textbf{Empty Input}: No buildings are provided.
    \index{Empty Input}
\end{itemize}

\section*{Implementation Considerations}

When implementing the \texttt{getSkyline} function, keep in mind the following considerations to ensure robustness and efficiency:

\begin{itemize}
    \item \textbf{Data Type Selection}: Use appropriate data types that can handle large input values and avoid overflow or precision issues.
    \index{Data Type Selection}
    
    \item \textbf{Optimizing Event Sorting}: Efficiently sort events based on x-coordinates and heights to ensure correct processing order.
    \index{Optimizing Event Sorting}
    
    \item \textbf{Handling Large Inputs}: Design the algorithm to handle up to \(10^4\) buildings efficiently without significant performance degradation.
    \index{Handling Large Inputs}
    
    \item \textbf{Using Efficient Data Structures}: Utilize heaps, lists, and dictionaries effectively to manage and process events and active heights.
    \index{Efficient Data Structures}
    
    \item \textbf{Avoiding Redundant Calculations}: Ensure that distance and overlap calculations are performed only when necessary to optimize performance.
    \index{Avoiding Redundant Calculations}
    
    \item \textbf{Code Readability and Documentation}: Maintain clear and readable code through meaningful variable names and comprehensive comments to facilitate understanding and maintenance.
    \index{Code Readability}
    
    \item \textbf{Edge Case Handling}: Implement checks for edge cases to prevent incorrect results or runtime errors.
    \index{Edge Case Handling}
    
    \item \textbf{Implementing Helper Functions}: Create helper functions for tasks like distance calculation, event handling, and heap management to enhance modularity.
    \index{Helper Functions}
    
    \item \textbf{Consistent Naming Conventions}: Use consistent and descriptive naming conventions for variables and functions to improve code clarity.
    \index{Naming Conventions}
    
    \item \textbf{Memory Management}: Ensure that the algorithm manages memory efficiently, especially when dealing with large datasets.
    \index{Memory Management}
    
    \item \textbf{Implementing Iterative Solutions Carefully}: Ensure that loop conditions are correctly defined to prevent infinite loops or incorrect terminations.
    \index{Iterative Solutions}
    
    \item \textbf{Avoiding Floating-Point Precision Issues}: Since the problem deals with integers, floating-point precision is not a concern, simplifying the implementation.
    \index{Floating-Point Precision}
    
    \item \textbf{Testing and Validation}: Develop a comprehensive suite of test cases that cover all possible scenarios, including edge cases, to validate the correctness and efficiency of the implementation.
    \index{Testing and Validation}
    
    \item \textbf{Performance Considerations}: Optimize the loop conditions and operations to ensure that the function runs efficiently, especially for large input numbers.
    \index{Performance Considerations}
\end{itemize}

\section*{Conclusion}

The \textbf{Skyline Problem} is a quintessential example of applying advanced algorithmic techniques and geometric reasoning to solve complex spatial challenges. By leveraging the Sweep Line Algorithm and maintaining active building heights using a max heap, the solution efficiently constructs the skyline with optimal time and space complexities. Understanding and implementing such sophisticated algorithms not only enhances problem-solving skills but also provides a foundation for tackling a wide array of Computational Geometry problems in various domains, including computer graphics, urban planning simulations, and geographic information systems.

\printindex

% \input{sections/rectangle_overlap}
% \input{sections/rectangle_area}
% \input{sections/k_closest_points_to_origin}
% \input{sections/the_skyline_problem}
% % filename: k_closest_points_to_origin.tex

\problemsection{K Closest Points to Origin}
\label{chap:K_Closest_Points_to_Origin}
\marginnote{\href{https://leetcode.com/problems/k-closest-points-to-origin/}{[LeetCode Link]}\index{LeetCode}}
\marginnote{\href{https://www.geeksforgeeks.org/find-k-closest-points-origin/}{[GeeksForGeeks Link]}\index{GeeksForGeeks}}
\marginnote{\href{https://www.interviewbit.com/problems/k-closest-points/}{[InterviewBit Link]}\index{InterviewBit}}
\marginnote{\href{https://app.codesignal.com/challenges/k-closest-points-to-origin}{[CodeSignal Link]}\index{CodeSignal}}
\marginnote{\href{https://www.codewars.com/kata/k-closest-points-to-origin/train/python}{[Codewars Link]}\index{Codewars}}

The \textbf{K Closest Points to Origin} problem is a popular algorithmic challenge in Computational Geometry that involves identifying the \(k\) points closest to the origin in a 2D plane. This problem tests one's ability to apply efficient sorting and selection algorithms, understand distance computations, and optimize for performance. Mastery of this problem is essential for applications in spatial data analysis, nearest neighbor searches, and clustering algorithms.

\section*{Problem Statement}

Given an array of points where each point is represented as \([x, y]\) in the 2D plane, and an integer \(k\), return the \(k\) closest points to the origin \((0, 0)\).

The distance between two points \((x_1, y_1)\) and \((x_2, y_2)\) is the Euclidean distance \(\sqrt{(x_1 - x_2)^2 + (y_1 - y_2)^2}\). The origin is \((0, 0)\).

\textbf{Function signature in Python:}
\begin{lstlisting}[language=Python]
def kClosest(points: List[List[int]], K: int) -> List[List[int]]:
\end{lstlisting}

\section*{Examples}

\textbf{Example 1:}

\begin{verbatim}
Input: points = [[1,3],[-2,2]], K = 1
Output: [[-2,2]]
Explanation: 
The distance between (1, 3) and the origin is sqrt(10).
The distance between (-2, 2) and the origin is sqrt(8).
Since sqrt(8) < sqrt(10), (-2, 2) is closer to the origin.
\end{verbatim}

\textbf{Example 2:}

\begin{verbatim}
Input: points = [[3,3],[5,-1],[-2,4]], K = 2
Output: [[3,3],[-2,4]]
Explanation: 
The distances are sqrt(18), sqrt(26), and sqrt(20) respectively.
The two closest points are [3,3] and [-2,4].
\end{verbatim}

\textbf{Example 3:}

\begin{verbatim}
Input: points = [[0,1],[1,0]], K = 2
Output: [[0,1],[1,0]]
Explanation: 
Both points are equally close to the origin.
\end{verbatim}

\textbf{Example 4:}

\begin{verbatim}
Input: points = [[1,0],[0,1]], K = 1
Output: [[1,0]]
Explanation: 
Both points are equally close; returning any one is acceptable.
\end{verbatim}

\textbf{Constraints:}

\begin{itemize}
    \item \(1 \leq K \leq \text{points.length} \leq 10^4\)
    \item \(-10^4 < x_i, y_i < 10^4\)
\end{itemize}

LeetCode link: \href{https://leetcode.com/problems/k-closest-points-to-origin/}{K Closest Points to Origin}\index{LeetCode}

\section*{Algorithmic Approach}

To identify the \(k\) closest points to the origin, several algorithmic strategies can be employed. The most efficient methods aim to reduce the time complexity by avoiding the need to sort the entire list of points.

\subsection*{1. Sorting Based on Distance}

Calculate the Euclidean distance of each point from the origin and sort the points based on these distances. Select the first \(k\) points from the sorted list.

\begin{enumerate}
    \item Compute the distance for each point using the formula \(distance = x^2 + y^2\).
    \item Sort the points based on the computed distances.
    \item Return the first \(k\) points from the sorted list.
\end{enumerate}

\subsection*{2. Max Heap (Priority Queue)}

Use a max heap to maintain the \(k\) closest points. Iterate through each point, add it to the heap, and if the heap size exceeds \(k\), remove the farthest point.

\begin{enumerate}
    \item Initialize a max heap.
    \item For each point, compute its distance and add it to the heap.
    \item If the heap size exceeds \(k\), remove the point with the largest distance.
    \item After processing all points, the heap contains the \(k\) closest points.
\end{enumerate}

\subsection*{3. QuickSelect (Quick Sort Partitioning)}

Utilize the QuickSelect algorithm to find the \(k\) closest points without fully sorting the list.

\begin{enumerate}
    \item Choose a pivot point and partition the list based on distances relative to the pivot.
    \item Recursively apply QuickSelect to the partition containing the \(k\) closest points.
    \item Once the \(k\) closest points are identified, return them.
\end{enumerate}

\marginnote{QuickSelect offers an average time complexity of \(O(n)\), making it highly efficient for large datasets.}

\section*{Complexities}

\begin{itemize}
    \item \textbf{Sorting Based on Distance:}
    \begin{itemize}
        \item \textbf{Time Complexity:} \(O(n \log n)\)
        \item \textbf{Space Complexity:} \(O(n)\)
    \end{itemize}
    
    \item \textbf{Max Heap (Priority Queue):}
    \begin{itemize}
        \item \textbf{Time Complexity:} \(O(n \log k)\)
        \item \textbf{Space Complexity:} \(O(k)\)
    \end{itemize}
    
    \item \textbf{QuickSelect (Quick Sort Partitioning):}
    \begin{itemize}
        \item \textbf{Time Complexity:} Average case \(O(n)\), worst case \(O(n^2)\)
        \item \textbf{Space Complexity:} \(O(1)\) (in-place)
    \end{itemize}
\end{itemize}

\section*{Python Implementation}

\marginnote{Implementing QuickSelect provides an optimal average-case solution with linear time complexity.}

Below is the complete Python code implementing the \texttt{kClosest} function using the QuickSelect approach:

\begin{fullwidth}
\begin{lstlisting}[language=Python]
from typing import List
import random

class Solution:
    def kClosest(self, points: List[List[int]], K: int) -> List[List[int]]:
        def quickselect(left, right, K_smallest):
            if left == right:
                return
            
            # Select a random pivot_index
            pivot_index = random.randint(left, right)
            
            # Partition the array
            pivot_index = partition(left, right, pivot_index)
            
            # The pivot is in its final sorted position
            if K_smallest == pivot_index:
                return
            elif K_smallest < pivot_index:
                quickselect(left, pivot_index - 1, K_smallest)
            else:
                quickselect(pivot_index + 1, right, K_smallest)
        
        def partition(left, right, pivot_index):
            pivot_distance = distance(points[pivot_index])
            # Move pivot to end
            points[pivot_index], points[right] = points[right], points[pivot_index]
            store_index = left
            for i in range(left, right):
                if distance(points[i]) < pivot_distance:
                    points[store_index], points[i] = points[i], points[store_index]
                    store_index += 1
            # Move pivot to its final place
            points[right], points[store_index] = points[store_index], points[right]
            return store_index
        
        def distance(point):
            return point[0] ** 2 + point[1] ** 2
        
        n = len(points)
        quickselect(0, n - 1, K)
        return points[:K]

# Example usage:
solution = Solution()
print(solution.kClosest([[1,3],[-2,2]], 1))            # Output: [[-2,2]]
print(solution.kClosest([[3,3],[5,-1],[-2,4]], 2))     # Output: [[3,3],[-2,4]]
print(solution.kClosest([[0,1],[1,0]], 2))             # Output: [[0,1],[1,0]]
print(solution.kClosest([[1,0],[0,1]], 1))             # Output: [[1,0]] or [[0,1]]
\end{lstlisting}
\end{fullwidth}

This implementation uses the QuickSelect algorithm to efficiently find the \(k\) closest points to the origin without fully sorting the entire list. It ensures optimal performance even with large datasets.

\section*{Explanation}

The \texttt{kClosest} function identifies the \(k\) closest points to the origin using the QuickSelect algorithm. Here's a detailed breakdown of the implementation:

\subsection*{1. Distance Calculation}

\begin{itemize}
    \item The Euclidean distance is calculated as \(distance = x^2 + y^2\). Since we only need relative distances for comparison, the square root is omitted for efficiency.
\end{itemize}

\subsection*{2. QuickSelect Algorithm}

\begin{itemize}
    \item **Pivot Selection:**
    \begin{itemize}
        \item A random pivot is chosen to enhance the average-case performance.
    \end{itemize}
    
    \item **Partitioning:**
    \begin{itemize}
        \item The array is partitioned such that points with distances less than the pivot are moved to the left, and others to the right.
        \item The pivot is placed in its correct sorted position.
    \end{itemize}
    
    \item **Recursive Selection:**
    \begin{itemize}
        \item If the pivot's position matches \(K\), the selection is complete.
        \item Otherwise, recursively apply QuickSelect to the relevant partition.
    \end{itemize}
\end{itemize}

\subsection*{3. Final Selection}

\begin{itemize}
    \item After partitioning, the first \(K\) points in the list are the \(k\) closest points to the origin.
\end{itemize}

\subsection*{4. Example Walkthrough}

Consider the first example:
\begin{verbatim}
Input: points = [[1,3],[-2,2]], K = 1
Output: [[-2,2]]
\end{verbatim}

\begin{enumerate}
    \item **Calculate Distances:**
    \begin{itemize}
        \item [1,3] : \(1^2 + 3^2 = 10\)
        \item [-2,2] : \((-2)^2 + 2^2 = 8\)
    \end{itemize}
    
    \item **QuickSelect Process:**
    \begin{itemize}
        \item Choose a pivot, say [1,3] with distance 10.
        \item Compare and rearrange:
        \begin{itemize}
            \item [-2,2] has a smaller distance (8) and is moved to the left.
        \end{itemize}
        \item After partitioning, the list becomes [[-2,2], [1,3]].
        \item Since \(K = 1\), return the first point: [[-2,2]].
    \end{itemize}
\end{enumerate}

Thus, the function correctly identifies \([-2,2]\) as the closest point to the origin.

\section*{Why This Approach}

The QuickSelect algorithm is chosen for its average-case linear time complexity \(O(n)\), making it highly efficient for large datasets compared to sorting-based methods with \(O(n \log n)\) time complexity. By avoiding the need to sort the entire list, QuickSelect provides an optimal solution for finding the \(k\) closest points.

\section*{Alternative Approaches}

\subsection*{1. Sorting Based on Distance}

Sort all points based on their distances from the origin and select the first \(k\) points.

\begin{lstlisting}[language=Python]
class Solution:
    def kClosest(self, points: List[List[int]], K: int) -> List[List[int]]:
        points.sort(key=lambda P: P[0]**2 + P[1]**2)
        return points[:K]
\end{lstlisting}

\textbf{Complexities:}
\begin{itemize}
    \item \textbf{Time Complexity:} \(O(n \log n)\)
    \item \textbf{Space Complexity:} \(O(1)\)
\end{itemize}

\subsection*{2. Max Heap (Priority Queue)}

Use a max heap to maintain the \(k\) closest points.

\begin{lstlisting}[language=Python]
import heapq

class Solution:
    def kClosest(self, points: List[List[int]], K: int) -> List[List[int]]:
        heap = []
        for (x, y) in points:
            dist = -(x**2 + y**2)  # Max heap using negative distances
            heapq.heappush(heap, (dist, [x, y]))
            if len(heap) > K:
                heapq.heappop(heap)
        return [item[1] for item in heap]
\end{lstlisting}

\textbf{Complexities:}
\begin{itemize}
    \item \textbf{Time Complexity:} \(O(n \log k)\)
    \item \textbf{Space Complexity:} \(O(k)\)
\end{itemize}

\subsection*{3. Using Built-In Functions}

Leverage built-in functions for distance calculation and selection.

\begin{lstlisting}[language=Python]
import math

class Solution:
    def kClosest(self, points: List[List[int]], K: int) -> List[List[int]]:
        points.sort(key=lambda P: math.sqrt(P[0]**2 + P[1]**2))
        return points[:K]
\end{lstlisting}

\textbf{Note}: This method is similar to the sorting approach but uses the actual Euclidean distance.

\section*{Similar Problems to This One}

Several problems involve nearest neighbor searches, spatial data analysis, and efficient selection algorithms, utilizing similar algorithmic strategies:

\begin{itemize}
    \item \textbf{Closest Pair of Points}: Find the closest pair of points in a set.
    \item \textbf{Top K Frequent Elements}: Identify the most frequent elements in a dataset.
    \item \textbf{Kth Largest Element in an Array}: Find the \(k\)-th largest element in an unsorted array.
    \item \textbf{Sliding Window Maximum}: Find the maximum in each sliding window of size \(k\) over an array.
    \item \textbf{Merge K Sorted Lists}: Merge multiple sorted lists into a single sorted list.
    \item \textbf{Find Median from Data Stream}: Continuously find the median of a stream of numbers.
    \item \textbf{Top K Closest Stars}: Find the \(k\) closest stars to Earth based on their distances.
\end{itemize}

These problems reinforce concepts of efficient selection, heap usage, and distance computations in various contexts.

\section*{Things to Keep in Mind and Tricks}

When solving the \textbf{K Closest Points to Origin} problem, consider the following tips and best practices to enhance efficiency and correctness:

\begin{itemize}
    \item \textbf{Understand Distance Calculations}: Grasp the Euclidean distance formula and recognize that the square root can be omitted for comparison purposes.
    \index{Distance Calculations}
    
    \item \textbf{Leverage Efficient Algorithms}: Use QuickSelect or heap-based methods to optimize time complexity, especially for large datasets.
    \index{Efficient Algorithms}
    
    \item \textbf{Handle Ties Appropriately}: Decide how to handle points with identical distances when \(k\) is less than the number of such points.
    \index{Handling Ties}
    
    \item \textbf{Optimize Space Usage}: Choose algorithms that minimize additional space, such as in-place QuickSelect.
    \index{Space Optimization}
    
    \item \textbf{Use Appropriate Data Structures}: Utilize heaps, lists, and helper functions effectively to manage and process data.
    \index{Data Structures}
    
    \item \textbf{Implement Helper Functions}: Create helper functions for distance calculation and partitioning to enhance code modularity.
    \index{Helper Functions}
    
    \item \textbf{Code Readability}: Maintain clear and readable code through meaningful variable names and structured logic.
    \index{Code Readability}
    
    \item \textbf{Test Extensively}: Implement a wide range of test cases, including edge cases like multiple points with the same distance, to ensure robustness.
    \index{Extensive Testing}
    
    \item \textbf{Understand Algorithm Trade-offs}: Recognize the trade-offs between different approaches in terms of time and space complexities.
    \index{Algorithm Trade-offs}
    
    \item \textbf{Use Built-In Sorting Functions}: When using sorting-based approaches, leverage built-in functions for efficiency and simplicity.
    \index{Built-In Sorting}
    
    \item \textbf{Avoid Redundant Calculations}: Ensure that distance calculations are performed only when necessary to optimize performance.
    \index{Avoiding Redundant Calculations}
    
    \item \textbf{Language-Specific Features}: Utilize language-specific features or libraries that can simplify implementation, such as heapq in Python.
    \index{Language-Specific Features}
\end{itemize}

\section*{Corner and Special Cases to Test When Writing the Code}

When implementing the solution for the \textbf{K Closest Points to Origin} problem, it is crucial to consider and rigorously test various edge cases to ensure robustness and correctness:

\begin{itemize}
    \item \textbf{Multiple Points with Same Distance}: Ensure that the algorithm handles multiple points having the same distance from the origin.
    \index{Same Distance Points}
    
    \item \textbf{Points at Origin}: Include points that are exactly at the origin \((0,0)\).
    \index{Points at Origin}
    
    \item \textbf{Negative Coordinates}: Ensure that the algorithm correctly computes distances for points with negative \(x\) or \(y\) coordinates.
    \index{Negative Coordinates}
    
    \item \textbf{Large Coordinates}: Test with points having very large or very small coordinate values to verify integer handling.
    \index{Large Coordinates}
    
    \item \textbf{K Equals Number of Points}: When \(K\) is equal to the number of points, the algorithm should return all points.
    \index{K Equals Number of Points}
    
    \item \textbf{K is One}: Test with \(K = 1\) to ensure the closest point is correctly identified.
    \index{K is One}
    
    \item \textbf{All Points Same}: All points have the same coordinates.
    \index{All Points Same}
    
    \item \textbf{K is Zero}: Although \(K\) is defined to be at least 1, ensure that the algorithm gracefully handles \(K = 0\) if allowed.
    \index{K is Zero}
    
    \item \textbf{Single Point}: Only one point is provided, and \(K = 1\).
    \index{Single Point}
    
    \item \textbf{Mixed Coordinates}: Points with a mix of positive and negative coordinates.
    \index{Mixed Coordinates}
    
    \item \textbf{Points with Zero Distance}: Multiple points at the origin.
    \index{Zero Distance Points}
    
    \item \textbf{Sparse and Dense Points}: Densely packed points and sparsely distributed points.
    \index{Sparse and Dense Points}
    
    \item \textbf{Duplicate Points}: Multiple identical points in the input list.
    \index{Duplicate Points}
    
    \item \textbf{K Greater Than Number of Unique Points}: Ensure that the algorithm handles cases where \(K\) exceeds the number of unique points if applicable.
    \index{K Greater Than Unique Points}
\end{itemize}

\section*{Implementation Considerations}

When implementing the \texttt{kClosest} function, keep in mind the following considerations to ensure robustness and efficiency:

\begin{itemize}
    \item \textbf{Data Type Selection}: Use appropriate data types that can handle large input values without overflow or precision loss.
    \index{Data Type Selection}
    
    \item \textbf{Optimizing Distance Calculations}: Avoid calculating the square root since it is unnecessary for comparison purposes.
    \index{Optimizing Distance Calculations}
    
    \item \textbf{Choosing the Right Algorithm}: Select an algorithm based on the size of the input and the value of \(K\) to optimize time and space complexities.
    \index{Choosing the Right Algorithm}
    
    \item \textbf{Language-Specific Libraries}: Utilize language-specific libraries and functions (e.g., \texttt{heapq} in Python) to simplify implementation and enhance performance.
    \index{Language-Specific Libraries}
    
    \item \textbf{Avoiding Redundant Calculations}: Ensure that each point's distance is calculated only once to optimize performance.
    \index{Avoiding Redundant Calculations}
    
    \item \textbf{Implementing Helper Functions}: Create helper functions for tasks like distance calculation and partitioning to enhance modularity and readability.
    \index{Helper Functions}
    
    \item \textbf{Edge Case Handling}: Implement checks for edge cases to prevent incorrect results or runtime errors.
    \index{Edge Case Handling}
    
    \item \textbf{Testing and Validation}: Develop a comprehensive suite of test cases that cover all possible scenarios, including edge cases, to validate the correctness and efficiency of the implementation.
    \index{Testing and Validation}
    
    \item \textbf{Scalability}: Design the algorithm to scale efficiently with increasing input sizes, maintaining performance and resource utilization.
    \index{Scalability}
    
    \item \textbf{Consistent Naming Conventions}: Use consistent and descriptive naming conventions for variables and functions to improve code clarity.
    \index{Naming Conventions}
    
    \item \textbf{Memory Management}: Ensure that the algorithm manages memory efficiently, especially when dealing with large datasets.
    \index{Memory Management}
    
    \item \textbf{Avoiding Stack Overflow}: If implementing recursive approaches, be mindful of recursion limits and potential stack overflow issues.
    \index{Avoiding Stack Overflow}
    
    \item \textbf{Implementing Iterative Solutions}: Prefer iterative solutions when recursion may lead to increased space complexity or stack overflow.
    \index{Implementing Iterative Solutions}
\end{itemize}

\section*{Conclusion}

The \textbf{K Closest Points to Origin} problem exemplifies the application of efficient selection algorithms and geometric computations to solve spatial challenges effectively. By leveraging QuickSelect or heap-based methods, the algorithm achieves optimal time and space complexities, making it highly suitable for large datasets. Understanding and implementing such techniques not only enhances problem-solving skills but also provides a foundation for tackling more advanced Computational Geometry problems involving nearest neighbor searches, clustering, and spatial data analysis.

\printindex

% % filename: rectangle_overlap.tex

\problemsection{Rectangle Overlap}
\label{chap:Rectangle_Overlap}
\marginnote{\href{https://leetcode.com/problems/rectangle-overlap/}{[LeetCode Link]}\index{LeetCode}}
\marginnote{\href{https://www.geeksforgeeks.org/check-if-two-rectangles-overlap/}{[GeeksForGeeks Link]}\index{GeeksForGeeks}}
\marginnote{\href{https://www.interviewbit.com/problems/rectangle-overlap/}{[InterviewBit Link]}\index{InterviewBit}}
\marginnote{\href{https://app.codesignal.com/challenges/rectangle-overlap}{[CodeSignal Link]}\index{CodeSignal}}
\marginnote{\href{https://www.codewars.com/kata/rectangle-overlap/train/python}{[Codewars Link]}\index{Codewars}}

The \textbf{Rectangle Overlap} problem is a fundamental challenge in Computational Geometry that involves determining whether two axis-aligned rectangles overlap. This problem tests one's ability to understand geometric properties, implement conditional logic, and optimize for efficient computation. Mastery of this problem is essential for applications in computer graphics, collision detection, and spatial data analysis.

\section*{Problem Statement}

Given two axis-aligned rectangles in a 2D plane, determine if they overlap. Each rectangle is defined by its bottom-left and top-right coordinates.

A rectangle is represented as a list of four integers \([x1, y1, x2, y2]\), where \((x1, y1)\) are the coordinates of the bottom-left corner, and \((x2, y2)\) are the coordinates of the top-right corner.

\textbf{Function signature in Python:}
\begin{lstlisting}[language=Python]
def isRectangleOverlap(rec1: List[int], rec2: List[int]) -> bool:
\end{lstlisting}

\section*{Examples}

\textbf{Example 1:}

\begin{verbatim}
Input: rec1 = [0,0,2,2], rec2 = [1,1,3,3]
Output: True
Explanation: The rectangles overlap in the area defined by [1,1,2,2].
\end{verbatim}

\textbf{Example 2:}

\begin{verbatim}
Input: rec1 = [0,0,1,1], rec2 = [1,0,2,1]
Output: False
Explanation: The rectangles touch at the edge but do not overlap.
\end{verbatim}

\textbf{Example 3:}

\begin{verbatim}
Input: rec1 = [0,0,1,1], rec2 = [2,2,3,3]
Output: False
Explanation: The rectangles are completely separate.
\end{verbatim}

\textbf{Example 4:}

\begin{verbatim}
Input: rec1 = [0,0,5,5], rec2 = [3,3,7,7]
Output: True
Explanation: The rectangles overlap in the area defined by [3,3,5,5].
\end{verbatim}

\textbf{Example 5:}

\begin{verbatim}
Input: rec1 = [0,0,0,0], rec2 = [0,0,0,0]
Output: False
Explanation: Both rectangles are degenerate points.
\end{verbatim}

\textbf{Constraints:}

\begin{itemize}
    \item All coordinates are integers in the range \([-10^9, 10^9]\).
    \item For each rectangle, \(x1 < x2\) and \(y1 < y2\).
\end{itemize}

LeetCode link: \href{https://leetcode.com/problems/rectangle-overlap/}{Rectangle Overlap}\index{LeetCode}

\section*{Algorithmic Approach}

To determine whether two axis-aligned rectangles overlap, we can use the following logical conditions:

1. **Non-Overlap Conditions:**
   - One rectangle is to the left of the other.
   - One rectangle is above the other.

2. **Overlap Condition:**
   - If neither of the non-overlap conditions is true, the rectangles must overlap.

\subsection*{Steps:}

1. **Extract Coordinates:**
   - For both rectangles, extract the bottom-left and top-right coordinates.

2. **Check Non-Overlap Conditions:**
   - If the right side of the first rectangle is less than or equal to the left side of the second rectangle, they do not overlap.
   - If the left side of the first rectangle is greater than or equal to the right side of the second rectangle, they do not overlap.
   - If the top side of the first rectangle is less than or equal to the bottom side of the second rectangle, they do not overlap.
   - If the bottom side of the first rectangle is greater than or equal to the top side of the second rectangle, they do not overlap.

3. **Determine Overlap:**
   - If none of the non-overlap conditions are met, the rectangles overlap.

\marginnote{This approach provides an efficient \(O(1)\) time complexity solution by leveraging simple geometric comparisons.}

\section*{Complexities}

\begin{itemize}
    \item \textbf{Time Complexity:} \(O(1)\). The algorithm performs a constant number of comparisons regardless of input size.
    
    \item \textbf{Space Complexity:} \(O(1)\). Only a fixed amount of extra space is used for variables.
\end{itemize}

\section*{Python Implementation}

\marginnote{Implementing the overlap check using coordinate comparisons ensures an optimal and straightforward solution.}

Below is the complete Python code implementing the \texttt{isRectangleOverlap} function:

\begin{fullwidth}
\begin{lstlisting}[language=Python]
from typing import List

class Solution:
    def isRectangleOverlap(self, rec1: List[int], rec2: List[int]) -> bool:
        # Extract coordinates
        left1, bottom1, right1, top1 = rec1
        left2, bottom2, right2, top2 = rec2
        
        # Check non-overlapping conditions
        if right1 <= left2 or right2 <= left1:
            return False
        if top1 <= bottom2 or top2 <= bottom1:
            return False
        
        # If none of the above, rectangles overlap
        return True

# Example usage:
solution = Solution()
print(solution.isRectangleOverlap([0,0,2,2], [1,1,3,3]))  # Output: True
print(solution.isRectangleOverlap([0,0,1,1], [1,0,2,1]))  # Output: False
print(solution.isRectangleOverlap([0,0,1,1], [2,2,3,3]))  # Output: False
print(solution.isRectangleOverlap([0,0,5,5], [3,3,7,7]))  # Output: True
print(solution.isRectangleOverlap([0,0,0,0], [0,0,0,0]))  # Output: False
\end{lstlisting}
\end{fullwidth}

This implementation efficiently checks for overlap by comparing the coordinates of the two rectangles. If any of the non-overlapping conditions are met, it returns \texttt{False}; otherwise, it returns \texttt{True}.

\section*{Explanation}

The \texttt{isRectangleOverlap} function determines whether two axis-aligned rectangles overlap by comparing their respective coordinates. Here's a detailed breakdown of the implementation:

\subsection*{1. Extract Coordinates}

\begin{itemize}
    \item For each rectangle, extract the left (\(x1\)), bottom (\(y1\)), right (\(x2\)), and top (\(y2\)) coordinates.
    \item This simplifies the comparison process by providing clear variables representing each side of the rectangles.
\end{itemize}

\subsection*{2. Check Non-Overlap Conditions}

\begin{itemize}
    \item **Horizontal Separation:**
    \begin{itemize}
        \item If the right side of the first rectangle (\(right1\)) is less than or equal to the left side of the second rectangle (\(left2\)), there is no horizontal overlap.
        \item Similarly, if the right side of the second rectangle (\(right2\)) is less than or equal to the left side of the first rectangle (\(left1\)), there is no horizontal overlap.
    \end{itemize}
    
    \item **Vertical Separation:**
    \begin{itemize}
        \item If the top side of the first rectangle (\(top1\)) is less than or equal to the bottom side of the second rectangle (\(bottom2\)), there is no vertical overlap.
        \item Similarly, if the top side of the second rectangle (\(top2\)) is less than or equal to the bottom side of the first rectangle (\(bottom1\)), there is no vertical overlap.
    \end{itemize}
    
    \item If any of these non-overlapping conditions are true, the rectangles do not overlap, and the function returns \texttt{False}.
\end{itemize}

\subsection*{3. Determine Overlap}

\begin{itemize}
    \item If none of the non-overlapping conditions are met, it implies that the rectangles overlap both horizontally and vertically.
    \item The function returns \texttt{True} in this case.
\end{itemize}

\subsection*{4. Example Walkthrough}

Consider the first example:
\begin{verbatim}
Input: rec1 = [0,0,2,2], rec2 = [1,1,3,3]
Output: True
\end{verbatim}

\begin{enumerate}
    \item Extract coordinates:
    \begin{itemize}
        \item rec1: left1 = 0, bottom1 = 0, right1 = 2, top1 = 2
        \item rec2: left2 = 1, bottom2 = 1, right2 = 3, top2 = 3
    \end{itemize}
    
    \item Check non-overlap conditions:
    \begin{itemize}
        \item \(right1 = 2\) is not less than or equal to \(left2 = 1\)
        \item \(right2 = 3\) is not less than or equal to \(left1 = 0\)
        \item \(top1 = 2\) is not less than or equal to \(bottom2 = 1\)
        \item \(top2 = 3\) is not less than or equal to \(bottom1 = 0\)
    \end{itemize}
    
    \item Since none of the non-overlapping conditions are met, the rectangles overlap.
\end{enumerate}

Thus, the function correctly returns \texttt{True}.

\section*{Why This Approach}

This approach is chosen for its simplicity and efficiency. By leveraging direct coordinate comparisons, the algorithm achieves constant time complexity without the need for complex data structures or iterative processes. It effectively handles all possible scenarios of rectangle positioning, ensuring accurate detection of overlaps.

\section*{Alternative Approaches}

\subsection*{1. Separating Axis Theorem (SAT)}

The Separating Axis Theorem is a more generalized method for detecting overlaps between convex shapes. While it is not necessary for axis-aligned rectangles, understanding SAT can be beneficial for more complex geometric problems.

\begin{lstlisting}[language=Python]
def isRectangleOverlap(rec1: List[int], rec2: List[int]) -> bool:
    # Using SAT for axis-aligned rectangles
    return not (rec1[2] <= rec2[0] or rec1[0] >= rec2[2] or
                rec1[3] <= rec2[1] or rec1[1] >= rec2[3])
\end{lstlisting}

\textbf{Note}: This implementation is functionally identical to the primary approach but leverages a more generalized geometric theorem.

\subsection*{2. Area-Based Approach}

Calculate the overlapping area between the two rectangles. If the overlapping area is positive, the rectangles overlap.

\begin{lstlisting}[language=Python]
def isRectangleOverlap(rec1: List[int], rec2: List[int]) -> bool:
    # Calculate overlap in x and y dimensions
    x_overlap = min(rec1[2], rec2[2]) - max(rec1[0], rec2[0])
    y_overlap = min(rec1[3], rec2[3]) - max(rec1[1], rec2[1])
    
    # Overlap exists if both overlaps are positive
    return x_overlap > 0 and y_overlap > 0
\end{lstlisting}

\textbf{Complexities:}
\begin{itemize}
    \item \textbf{Time Complexity:} \(O(1)\)
    \item \textbf{Space Complexity:} \(O(1)\)
\end{itemize}

\subsection*{3. Using Rectangles Intersection Function}

Utilize built-in or library functions that handle geometric intersections.

\begin{lstlisting}[language=Python]
from shapely.geometry import box

def isRectangleOverlap(rec1: List[int], rec2: List[int]) -> bool:
    rectangle1 = box(rec1[0], rec1[1], rec1[2], rec1[3])
    rectangle2 = box(rec2[0], rec2[1], rec2[2], rec2[3])
    return rectangle1.intersects(rectangle2) and not rectangle1.touches(rectangle2)
\end{lstlisting}

\textbf{Note}: This approach requires the \texttt{shapely} library and is more suitable for complex geometric operations.

\section*{Similar Problems to This One}

Several problems revolve around geometric overlap, intersection detection, and spatial reasoning, utilizing similar algorithmic strategies:

\begin{itemize}
    \item \textbf{Interval Overlap}: Determine if two intervals on a line overlap.
    \item \textbf{Circle Overlap}: Determine if two circles overlap based on their radii and centers.
    \item \textbf{Polygon Overlap}: Determine if two polygons overlap using algorithms like SAT.
    \item \textbf{Closest Pair of Points}: Find the closest pair of points in a set.
    \item \textbf{Convex Hull}: Compute the convex hull of a set of points.
    \item \textbf{Intersection of Lines}: Find the intersection point of two lines.
    \item \textbf{Point Inside Polygon}: Determine if a point lies inside a given polygon.
\end{itemize}

These problems reinforce the concepts of spatial reasoning, geometric property analysis, and efficient algorithm design in various contexts.

\section*{Things to Keep in Mind and Tricks}

When working with the \textbf{Rectangle Overlap} problem, consider the following tips and best practices to enhance efficiency and correctness:

\begin{itemize}
    \item \textbf{Understand Geometric Relationships}: Grasp the positional relationships between rectangles to simplify overlap detection.
    \index{Geometric Relationships}
    
    \item \textbf{Leverage Coordinate Comparisons}: Use direct comparisons of rectangle coordinates to determine spatial relationships.
    \index{Coordinate Comparisons}
    
    \item \textbf{Handle Edge Cases}: Consider cases where rectangles touch at edges or corners without overlapping.
    \index{Edge Cases}
    
    \item \textbf{Optimize for Efficiency}: Aim for a constant time \(O(1)\) solution by avoiding unnecessary computations or iterations.
    \index{Efficiency Optimization}
    
    \item \textbf{Avoid Floating-Point Precision Issues}: Since all coordinates are integers, floating-point precision is not a concern, simplifying the implementation.
    \index{Floating-Point Precision}
    
    \item \textbf{Use Helper Functions}: Create helper functions to encapsulate repetitive tasks, such as extracting coordinates or checking specific conditions.
    \index{Helper Functions}
    
    \item \textbf{Code Readability}: Maintain clear and readable code through meaningful variable names and structured logic.
    \index{Code Readability}
    
    \item \textbf{Test Extensively}: Implement a wide range of test cases, including overlapping, non-overlapping, and edge-touching rectangles, to ensure robustness.
    \index{Extensive Testing}
    
    \item \textbf{Understand Axis-Aligned Constraints}: Recognize that axis-aligned rectangles simplify overlap detection compared to rotated rectangles.
    \index{Axis-Aligned Constraints}
    
    \item \textbf{Simplify Logical Conditions}: Combine multiple conditions logically to streamline the overlap detection process.
    \index{Logical Conditions}
\end{itemize}

\section*{Corner and Special Cases to Test When Writing the Code}

When implementing the solution for the \textbf{Rectangle Overlap} problem, it is crucial to consider and rigorously test various edge cases to ensure robustness and correctness:

\begin{itemize}
    \item \textbf{No Overlap}: Rectangles are completely separate.
    \index{No Overlap}
    
    \item \textbf{Partial Overlap}: Rectangles overlap in one or more regions.
    \index{Partial Overlap}
    
    \item \textbf{Edge Touching}: Rectangles touch exactly at one edge without overlapping.
    \index{Edge Touching}
    
    \item \textbf{Corner Touching}: Rectangles touch exactly at one corner without overlapping.
    \index{Corner Touching}
    
    \item \textbf{One Rectangle Inside Another}: One rectangle is entirely within the other.
    \index{Rectangle Inside}
    
    \item \textbf{Identical Rectangles}: Both rectangles have the same coordinates.
    \index{Identical Rectangles}
    
    \item \textbf{Degenerate Rectangles}: Rectangles with zero area (e.g., \(x1 = x2\) or \(y1 = y2\)).
    \index{Degenerate Rectangles}
    
    \item \textbf{Large Coordinates}: Rectangles with very large coordinate values to test performance and integer handling.
    \index{Large Coordinates}
    
    \item \textbf{Negative Coordinates}: Rectangles positioned in negative coordinate space.
    \index{Negative Coordinates}
    
    \item \textbf{Mixed Overlapping Scenarios}: Combinations of the above cases to ensure comprehensive coverage.
    \index{Mixed Overlapping Scenarios}
    
    \item \textbf{Minimum and Maximum Bounds}: Rectangles at the minimum and maximum limits of the coordinate range.
    \index{Minimum and Maximum Bounds}
\end{itemize}

\section*{Implementation Considerations}

When implementing the \texttt{isRectangleOverlap} function, keep in mind the following considerations to ensure robustness and efficiency:

\begin{itemize}
    \item \textbf{Data Type Selection}: Use appropriate data types that can handle the range of input values without overflow or underflow.
    \index{Data Type Selection}
    
    \item \textbf{Optimizing Comparisons}: Structure logical conditions to short-circuit evaluations as soon as a non-overlapping condition is met.
    \index{Optimizing Comparisons}
    
    \item \textbf{Language-Specific Constraints}: Be aware of how the programming language handles integer division and comparisons.
    \index{Language-Specific Constraints}
    
    \item \textbf{Avoiding Redundant Calculations}: Ensure that each comparison contributes towards determining overlap without unnecessary repetitions.
    \index{Avoiding Redundant Calculations}
    
    \item \textbf{Code Readability and Documentation}: Maintain clear and readable code through meaningful variable names and comprehensive comments to facilitate understanding and maintenance.
    \index{Code Readability}
    
    \item \textbf{Edge Case Handling}: Implement checks for edge cases to prevent incorrect results or runtime errors.
    \index{Edge Case Handling}
    
    \item \textbf{Testing and Validation}: Develop a comprehensive suite of test cases that cover all possible scenarios, including edge cases, to validate the correctness and efficiency of the implementation.
    \index{Testing and Validation}
    
    \item \textbf{Scalability}: Design the algorithm to scale efficiently with increasing input sizes, maintaining performance and resource utilization.
    \index{Scalability}
    
    \item \textbf{Using Helper Functions}: Consider creating helper functions for repetitive tasks, such as extracting and comparing coordinates, to enhance modularity and reusability.
    \index{Helper Functions}
    
    \item \textbf{Consistent Naming Conventions}: Use consistent and descriptive naming conventions for variables to improve code clarity.
    \index{Naming Conventions}
    
    \item \textbf{Handling Floating-Point Coordinates}: Although the problem specifies integer coordinates, ensure that the implementation can handle floating-point numbers if needed in extended scenarios.
    \index{Floating-Point Coordinates}
    
    \item \textbf{Avoiding Floating-Point Precision Issues}: Since all coordinates are integers, floating-point precision is not a concern, simplifying the implementation.
    \index{Floating-Point Precision}
    
    \item \textbf{Implementing Unit Tests}: Develop unit tests for each logical condition to ensure that all scenarios are correctly handled.
    \index{Unit Tests}
    
    \item \textbf{Error Handling}: Incorporate error handling to manage invalid inputs gracefully.
    \index{Error Handling}
\end{itemize}

\section*{Conclusion}

The \textbf{Rectangle Overlap} problem exemplifies the application of fundamental geometric principles and conditional logic to solve spatial challenges efficiently. By leveraging simple coordinate comparisons, the algorithm achieves optimal time and space complexities, making it highly suitable for real-time applications such as collision detection in gaming, layout planning in graphics, and spatial data analysis. Understanding and implementing such techniques not only enhances problem-solving skills but also provides a foundation for tackling more complex Computational Geometry problems involving varied geometric shapes and interactions.

\printindex

% \input{sections/rectangle_overlap}
% \input{sections/rectangle_area}
% \input{sections/k_closest_points_to_origin}
% \input{sections/the_skyline_problem}
% % filename: rectangle_area.tex

\problemsection{Rectangle Area}
\label{chap:Rectangle_Area}
\marginnote{\href{https://leetcode.com/problems/rectangle-area/}{[LeetCode Link]}\index{LeetCode}}
\marginnote{\href{https://www.geeksforgeeks.org/find-area-two-overlapping-rectangles/}{[GeeksForGeeks Link]}\index{GeeksForGeeks}}
\marginnote{\href{https://www.interviewbit.com/problems/rectangle-area/}{[InterviewBit Link]}\index{InterviewBit}}
\marginnote{\href{https://app.codesignal.com/challenges/rectangle-area}{[CodeSignal Link]}\index{CodeSignal}}
\marginnote{\href{https://www.codewars.com/kata/rectangle-area/train/python}{[Codewars Link]}\index{Codewars}}

The \textbf{Rectangle Area} problem is a classic Computational Geometry challenge that involves calculating the total area covered by two axis-aligned rectangles in a 2D plane. This problem tests one's ability to perform geometric calculations, handle overlapping scenarios, and implement efficient algorithms. Mastery of this problem is essential for applications in computer graphics, spatial analysis, and computational modeling.

\section*{Problem Statement}

Given two axis-aligned rectangles in a 2D plane, compute the total area covered by the two rectangles. The area covered by the overlapping region should be counted only once.

Each rectangle is represented as a list of four integers \([x1, y1, x2, y2]\), where \((x1, y1)\) are the coordinates of the bottom-left corner, and \((x2, y2)\) are the coordinates of the top-right corner.

\textbf{Function signature in Python:}
\begin{lstlisting}[language=Python]
def computeArea(A: List[int], B: List[int]) -> int:
\end{lstlisting}

\section*{Examples}

\textbf{Example 1:}

\begin{verbatim}
Input: A = [-3,0,3,4], B = [0,-1,9,2]
Output: 45
Explanation:
Area of A = (3 - (-3)) * (4 - 0) = 6 * 4 = 24
Area of B = (9 - 0) * (2 - (-1)) = 9 * 3 = 27
Overlapping Area = (3 - 0) * (2 - 0) = 3 * 2 = 6
Total Area = 24 + 27 - 6 = 45
\end{verbatim}

\textbf{Example 2:}

\begin{verbatim}
Input: A = [0,0,0,0], B = [0,0,0,0]
Output: 0
Explanation:
Both rectangles are degenerate points with zero area.
\end{verbatim}

\textbf{Example 3:}

\begin{verbatim}
Input: A = [0,0,2,2], B = [1,1,3,3]
Output: 7
Explanation:
Area of A = 4
Area of B = 4
Overlapping Area = 1
Total Area = 4 + 4 - 1 = 7
\end{verbatim}

\textbf{Example 4:}

\begin{verbatim}
Input: A = [0,0,1,1], B = [1,0,2,1]
Output: 2
Explanation:
Rectangles touch at the edge but do not overlap.
Area of A = 1
Area of B = 1
Overlapping Area = 0
Total Area = 1 + 1 = 2
\end{verbatim}

\textbf{Constraints:}

\begin{itemize}
    \item All coordinates are integers in the range \([-10^9, 10^9]\).
    \item For each rectangle, \(x1 < x2\) and \(y1 < y2\).
\end{itemize}

LeetCode link: \href{https://leetcode.com/problems/rectangle-area/}{Rectangle Area}\index{LeetCode}

\section*{Algorithmic Approach}

To compute the total area covered by two axis-aligned rectangles, we can follow these steps:

1. **Calculate Individual Areas:**
   - Compute the area of the first rectangle.
   - Compute the area of the second rectangle.

2. **Determine Overlapping Area:**
   - Calculate the coordinates of the overlapping rectangle, if any.
   - If the rectangles overlap, compute the area of the overlapping region.

3. **Compute Total Area:**
   - Sum the individual areas and subtract the overlapping area to avoid double-counting.

\marginnote{This approach ensures accurate area calculation by handling overlapping regions appropriately.}

\section*{Complexities}

\begin{itemize}
    \item \textbf{Time Complexity:} \(O(1)\). The algorithm performs a constant number of calculations.
    
    \item \textbf{Space Complexity:} \(O(1)\). Only a fixed amount of extra space is used for variables.
\end{itemize}

\section*{Python Implementation}

\marginnote{Implementing the area calculation with overlap consideration ensures an accurate and efficient solution.}

Below is the complete Python code implementing the \texttt{computeArea} function:

\begin{fullwidth}
\begin{lstlisting}[language=Python]
from typing import List

class Solution:
    def computeArea(self, A: List[int], B: List[int]) -> int:
        # Calculate area of rectangle A
        areaA = (A[2] - A[0]) * (A[3] - A[1])
        
        # Calculate area of rectangle B
        areaB = (B[2] - B[0]) * (B[3] - B[1])
        
        # Determine overlap coordinates
        overlap_x1 = max(A[0], B[0])
        overlap_y1 = max(A[1], B[1])
        overlap_x2 = min(A[2], B[2])
        overlap_y2 = min(A[3], B[3])
        
        # Calculate overlapping area
        overlap_width = overlap_x2 - overlap_x1
        overlap_height = overlap_y2 - overlap_y1
        overlap_area = 0
        if overlap_width > 0 and overlap_height > 0:
            overlap_area = overlap_width * overlap_height
        
        # Total area is sum of individual areas minus overlapping area
        total_area = areaA + areaB - overlap_area
        return total_area

# Example usage:
solution = Solution()
print(solution.computeArea([-3,0,3,4], [0,-1,9,2]))  # Output: 45
print(solution.computeArea([0,0,0,0], [0,0,0,0]))    # Output: 0
print(solution.computeArea([0,0,2,2], [1,1,3,3]))    # Output: 7
print(solution.computeArea([0,0,1,1], [1,0,2,1]))    # Output: 2
\end{lstlisting}
\end{fullwidth}

This implementation accurately computes the total area covered by two rectangles by accounting for any overlapping regions. It ensures that the overlapping area is not double-counted.

\section*{Explanation}

The \texttt{computeArea} function calculates the combined area of two axis-aligned rectangles by following these steps:

\subsection*{1. Calculate Individual Areas}

\begin{itemize}
    \item **Rectangle A:**
    \begin{itemize}
        \item Width: \(A[2] - A[0]\)
        \item Height: \(A[3] - A[1]\)
        \item Area: Width \(\times\) Height
    \end{itemize}
    
    \item **Rectangle B:**
    \begin{itemize}
        \item Width: \(B[2] - B[0]\)
        \item Height: \(B[3] - B[1]\)
        \item Area: Width \(\times\) Height
    \end{itemize}
\end{itemize}

\subsection*{2. Determine Overlapping Area}

\begin{itemize}
    \item **Overlap Coordinates:**
    \begin{itemize}
        \item Left (x-coordinate): \(\text{max}(A[0], B[0])\)
        \item Bottom (y-coordinate): \(\text{max}(A[1], B[1])\)
        \item Right (x-coordinate): \(\text{min}(A[2], B[2])\)
        \item Top (y-coordinate): \(\text{min}(A[3], B[3])\)
    \end{itemize}
    
    \item **Overlap Dimensions:**
    \begin{itemize}
        \item Width: \(\text{overlap\_x2} - \text{overlap\_x1}\)
        \item Height: \(\text{overlap\_y2} - \text{overlap\_y1}\)
    \end{itemize}
    
    \item **Overlap Area:**
    \begin{itemize}
        \item If both width and height are positive, the rectangles overlap, and the overlapping area is their product.
        \item Otherwise, there is no overlap, and the overlapping area is zero.
    \end{itemize}
\end{itemize}

\subsection*{3. Compute Total Area}

\begin{itemize}
    \item Total Area = Area of Rectangle A + Area of Rectangle B - Overlapping Area
\end{itemize}

\subsection*{4. Example Walkthrough}

Consider the first example:
\begin{verbatim}
Input: A = [-3,0,3,4], B = [0,-1,9,2]
Output: 45
\end{verbatim}

\begin{enumerate}
    \item **Calculate Areas:**
    \begin{itemize}
        \item Area of A = (3 - (-3)) * (4 - 0) = 6 * 4 = 24
        \item Area of B = (9 - 0) * (2 - (-1)) = 9 * 3 = 27
    \end{itemize}
    
    \item **Determine Overlap:**
    \begin{itemize}
        \item overlap\_x1 = max(-3, 0) = 0
        \item overlap\_y1 = max(0, -1) = 0
        \item overlap\_x2 = min(3, 9) = 3
        \item overlap\_y2 = min(4, 2) = 2
        \item overlap\_width = 3 - 0 = 3
        \item overlap\_height = 2 - 0 = 2
        \item overlap\_area = 3 * 2 = 6
    \end{itemize}
    
    \item **Compute Total Area:**
    \begin{itemize}
        \item Total Area = 24 + 27 - 6 = 45
    \end{itemize}
\end{enumerate}

Thus, the function correctly returns \texttt{45}.

\section*{Why This Approach}

This approach is chosen for its straightforwardness and optimal efficiency. By directly calculating the individual areas and intelligently handling the overlapping region, the algorithm ensures accurate results without unnecessary computations. Its constant time complexity makes it highly efficient, even for large coordinate values.

\section*{Alternative Approaches}

\subsection*{1. Using Intersection Dimensions}

Instead of separately calculating areas, directly compute the dimensions of the overlapping region and subtract it from the sum of individual areas.

\begin{lstlisting}[language=Python]
def computeArea(A: List[int], B: List[int]) -> int:
    # Sum of individual areas
    area = (A[2] - A[0]) * (A[3] - A[1]) + (B[2] - B[0]) * (B[3] - B[1])
    
    # Overlapping area
    overlap_width = min(A[2], B[2]) - max(A[0], B[0])
    overlap_height = min(A[3], B[3]) - max(A[1], B[1])
    
    if overlap_width > 0 and overlap_height > 0:
        area -= overlap_width * overlap_height
    
    return area
\end{lstlisting}

\subsection*{2. Using Geometry Libraries}

Leverage computational geometry libraries to handle area calculations and overlapping detections.

\begin{lstlisting}[language=Python]
from shapely.geometry import box

def computeArea(A: List[int], B: List[int]) -> int:
    rect1 = box(A[0], A[1], A[2], A[3])
    rect2 = box(B[0], B[1], B[2], B[3])
    intersection = rect1.intersection(rect2)
    return int(rect1.area + rect2.area - intersection.area)
\end{lstlisting}

\textbf{Note}: This approach requires the \texttt{shapely} library and is more suitable for complex geometric operations.

\section*{Similar Problems to This One}

Several problems involve calculating areas, handling geometric overlaps, and spatial reasoning, utilizing similar algorithmic strategies:

\begin{itemize}
    \item \textbf{Rectangle Overlap}: Determine if two rectangles overlap.
    \item \textbf{Circle Area Overlap}: Calculate the overlapping area between two circles.
    \item \textbf{Polygon Area}: Compute the area of a given polygon.
    \item \textbf{Union of Rectangles}: Calculate the total area covered by multiple rectangles, accounting for overlaps.
    \item \textbf{Intersection of Lines}: Find the intersection point of two lines.
    \item \textbf{Closest Pair of Points}: Find the closest pair of points in a set.
    \item \textbf{Convex Hull}: Compute the convex hull of a set of points.
    \item \textbf{Point Inside Polygon}: Determine if a point lies inside a given polygon.
\end{itemize}

These problems reinforce concepts of geometric calculations, area computations, and efficient algorithm design in various contexts.

\section*{Things to Keep in Mind and Tricks}

When tackling the \textbf{Rectangle Area} problem, consider the following tips and best practices to enhance efficiency and correctness:

\begin{itemize}
    \item \textbf{Understand Geometric Relationships}: Grasp the positional relationships between rectangles to simplify area calculations.
    \index{Geometric Relationships}
    
    \item \textbf{Leverage Coordinate Comparisons}: Use direct comparisons of rectangle coordinates to determine overlapping regions.
    \index{Coordinate Comparisons}
    
    \item \textbf{Handle Overlapping Scenarios}: Accurately calculate the overlapping area to avoid double-counting.
    \index{Overlapping Scenarios}
    
    \item \textbf{Optimize for Efficiency}: Aim for a constant time \(O(1)\) solution by avoiding unnecessary computations or iterations.
    \index{Efficiency Optimization}
    
    \item \textbf{Avoid Floating-Point Precision Issues}: Since all coordinates are integers, floating-point precision is not a concern, simplifying the implementation.
    \index{Floating-Point Precision}
    
    \item \textbf{Use Helper Functions}: Create helper functions to encapsulate repetitive tasks, such as calculating overlap dimensions or areas.
    \index{Helper Functions}
    
    \item \textbf{Code Readability}: Maintain clear and readable code through meaningful variable names and structured logic.
    \index{Code Readability}
    
    \item \textbf{Test Extensively}: Implement a wide range of test cases, including overlapping, non-overlapping, and edge-touching rectangles, to ensure robustness.
    \index{Extensive Testing}
    
    \item \textbf{Understand Axis-Aligned Constraints}: Recognize that axis-aligned rectangles simplify area calculations compared to rotated rectangles.
    \index{Axis-Aligned Constraints}
    
    \item \textbf{Simplify Logical Conditions}: Combine multiple conditions logically to streamline the area calculation process.
    \index{Logical Conditions}
    
    \item \textbf{Use Absolute Values}: When calculating differences, ensure that the dimensions are positive by using absolute values or proper ordering.
    \index{Absolute Values}
    
    \item \textbf{Consider Edge Cases}: Handle cases where rectangles have zero area or touch at edges/corners without overlapping.
    \index{Edge Cases}
\end{itemize}

\section*{Corner and Special Cases to Test When Writing the Code}

When implementing the solution for the \textbf{Rectangle Area} problem, it is crucial to consider and rigorously test various edge cases to ensure robustness and correctness:

\begin{itemize}
    \item \textbf{No Overlap}: Rectangles are completely separate.
    \index{No Overlap}
    
    \item \textbf{Partial Overlap}: Rectangles overlap in one or more regions.
    \index{Partial Overlap}
    
    \item \textbf{Edge Touching}: Rectangles touch exactly at one edge without overlapping.
    \index{Edge Touching}
    
    \item \textbf{Corner Touching}: Rectangles touch exactly at one corner without overlapping.
    \index{Corner Touching}
    
    \item \textbf{One Rectangle Inside Another}: One rectangle is entirely within the other.
    \index{Rectangle Inside}
    
    \item \textbf{Identical Rectangles}: Both rectangles have the same coordinates.
    \index{Identical Rectangles}
    
    \item \textbf{Degenerate Rectangles}: Rectangles with zero area (e.g., \(x1 = x2\) or \(y1 = y2\)).
    \index{Degenerate Rectangles}
    
    \item \textbf{Large Coordinates}: Rectangles with very large coordinate values to test performance and integer handling.
    \index{Large Coordinates}
    
    \item \textbf{Negative Coordinates}: Rectangles positioned in negative coordinate space.
    \index{Negative Coordinates}
    
    \item \textbf{Mixed Overlapping Scenarios}: Combinations of the above cases to ensure comprehensive coverage.
    \index{Mixed Overlapping Scenarios}
    
    \item \textbf{Minimum and Maximum Bounds}: Rectangles at the minimum and maximum limits of the coordinate range.
    \index{Minimum and Maximum Bounds}
    
    \item \textbf{Sequential Rectangles}: Multiple rectangles placed sequentially without overlapping.
    \index{Sequential Rectangles}
    
    \item \textbf{Multiple Overlaps}: Scenarios where more than two rectangles overlap in different regions.
    \index{Multiple Overlaps}
\end{itemize}

\section*{Implementation Considerations}

When implementing the \texttt{computeArea} function, keep in mind the following considerations to ensure robustness and efficiency:

\begin{itemize}
    \item \textbf{Data Type Selection}: Use appropriate data types that can handle large input values without overflow or underflow.
    \index{Data Type Selection}
    
    \item \textbf{Optimizing Comparisons}: Structure logical conditions to efficiently determine overlap dimensions.
    \index{Optimizing Comparisons}
    
    \item \textbf{Handling Large Inputs}: Design the algorithm to efficiently handle large input sizes without significant performance degradation.
    \index{Handling Large Inputs}
    
    \item \textbf{Language-Specific Constraints}: Be aware of how the programming language handles large integers and arithmetic operations.
    \index{Language-Specific Constraints}
    
    \item \textbf{Avoiding Redundant Calculations}: Ensure that each calculation contributes towards determining the final area without unnecessary repetitions.
    \index{Avoiding Redundant Calculations}
    
    \item \textbf{Code Readability and Documentation}: Maintain clear and readable code through meaningful variable names and comprehensive comments to facilitate understanding and maintenance.
    \index{Code Readability}
    
    \item \textbf{Edge Case Handling}: Implement checks for edge cases to prevent incorrect results or runtime errors.
    \index{Edge Case Handling}
    
    \item \textbf{Testing and Validation}: Develop a comprehensive suite of test cases that cover all possible scenarios, including edge cases, to validate the correctness and efficiency of the implementation.
    \index{Testing and Validation}
    
    \item \textbf{Scalability}: Design the algorithm to scale efficiently with increasing input sizes, maintaining performance and resource utilization.
    \index{Scalability}
    
    \item \textbf{Using Helper Functions}: Consider creating helper functions for repetitive tasks, such as calculating overlap dimensions, to enhance modularity and reusability.
    \index{Helper Functions}
    
    \item \textbf{Consistent Naming Conventions}: Use consistent and descriptive naming conventions for variables to improve code clarity.
    \index{Naming Conventions}
    
    \item \textbf{Implementing Unit Tests}: Develop unit tests for each logical condition to ensure that all scenarios are correctly handled.
    \index{Unit Tests}
    
    \item \textbf{Error Handling}: Incorporate error handling to manage invalid inputs gracefully.
    \index{Error Handling}
\end{itemize}

\section*{Conclusion}

The \textbf{Rectangle Area} problem showcases the application of fundamental geometric principles and efficient algorithm design to compute spatial properties accurately. By systematically calculating individual areas and intelligently handling overlapping regions, the algorithm ensures precise results without redundant computations. Understanding and implementing such techniques not only enhances problem-solving skills but also provides a foundation for tackling more complex Computational Geometry challenges involving multiple geometric entities and intricate spatial relationships.

\printindex

% \input{sections/rectangle_overlap}
% \input{sections/rectangle_area}
% \input{sections/k_closest_points_to_origin}
% \input{sections/the_skyline_problem}
% % filename: k_closest_points_to_origin.tex

\problemsection{K Closest Points to Origin}
\label{chap:K_Closest_Points_to_Origin}
\marginnote{\href{https://leetcode.com/problems/k-closest-points-to-origin/}{[LeetCode Link]}\index{LeetCode}}
\marginnote{\href{https://www.geeksforgeeks.org/find-k-closest-points-origin/}{[GeeksForGeeks Link]}\index{GeeksForGeeks}}
\marginnote{\href{https://www.interviewbit.com/problems/k-closest-points/}{[InterviewBit Link]}\index{InterviewBit}}
\marginnote{\href{https://app.codesignal.com/challenges/k-closest-points-to-origin}{[CodeSignal Link]}\index{CodeSignal}}
\marginnote{\href{https://www.codewars.com/kata/k-closest-points-to-origin/train/python}{[Codewars Link]}\index{Codewars}}

The \textbf{K Closest Points to Origin} problem is a popular algorithmic challenge in Computational Geometry that involves identifying the \(k\) points closest to the origin in a 2D plane. This problem tests one's ability to apply efficient sorting and selection algorithms, understand distance computations, and optimize for performance. Mastery of this problem is essential for applications in spatial data analysis, nearest neighbor searches, and clustering algorithms.

\section*{Problem Statement}

Given an array of points where each point is represented as \([x, y]\) in the 2D plane, and an integer \(k\), return the \(k\) closest points to the origin \((0, 0)\).

The distance between two points \((x_1, y_1)\) and \((x_2, y_2)\) is the Euclidean distance \(\sqrt{(x_1 - x_2)^2 + (y_1 - y_2)^2}\). The origin is \((0, 0)\).

\textbf{Function signature in Python:}
\begin{lstlisting}[language=Python]
def kClosest(points: List[List[int]], K: int) -> List[List[int]]:
\end{lstlisting}

\section*{Examples}

\textbf{Example 1:}

\begin{verbatim}
Input: points = [[1,3],[-2,2]], K = 1
Output: [[-2,2]]
Explanation: 
The distance between (1, 3) and the origin is sqrt(10).
The distance between (-2, 2) and the origin is sqrt(8).
Since sqrt(8) < sqrt(10), (-2, 2) is closer to the origin.
\end{verbatim}

\textbf{Example 2:}

\begin{verbatim}
Input: points = [[3,3],[5,-1],[-2,4]], K = 2
Output: [[3,3],[-2,4]]
Explanation: 
The distances are sqrt(18), sqrt(26), and sqrt(20) respectively.
The two closest points are [3,3] and [-2,4].
\end{verbatim}

\textbf{Example 3:}

\begin{verbatim}
Input: points = [[0,1],[1,0]], K = 2
Output: [[0,1],[1,0]]
Explanation: 
Both points are equally close to the origin.
\end{verbatim}

\textbf{Example 4:}

\begin{verbatim}
Input: points = [[1,0],[0,1]], K = 1
Output: [[1,0]]
Explanation: 
Both points are equally close; returning any one is acceptable.
\end{verbatim}

\textbf{Constraints:}

\begin{itemize}
    \item \(1 \leq K \leq \text{points.length} \leq 10^4\)
    \item \(-10^4 < x_i, y_i < 10^4\)
\end{itemize}

LeetCode link: \href{https://leetcode.com/problems/k-closest-points-to-origin/}{K Closest Points to Origin}\index{LeetCode}

\section*{Algorithmic Approach}

To identify the \(k\) closest points to the origin, several algorithmic strategies can be employed. The most efficient methods aim to reduce the time complexity by avoiding the need to sort the entire list of points.

\subsection*{1. Sorting Based on Distance}

Calculate the Euclidean distance of each point from the origin and sort the points based on these distances. Select the first \(k\) points from the sorted list.

\begin{enumerate}
    \item Compute the distance for each point using the formula \(distance = x^2 + y^2\).
    \item Sort the points based on the computed distances.
    \item Return the first \(k\) points from the sorted list.
\end{enumerate}

\subsection*{2. Max Heap (Priority Queue)}

Use a max heap to maintain the \(k\) closest points. Iterate through each point, add it to the heap, and if the heap size exceeds \(k\), remove the farthest point.

\begin{enumerate}
    \item Initialize a max heap.
    \item For each point, compute its distance and add it to the heap.
    \item If the heap size exceeds \(k\), remove the point with the largest distance.
    \item After processing all points, the heap contains the \(k\) closest points.
\end{enumerate}

\subsection*{3. QuickSelect (Quick Sort Partitioning)}

Utilize the QuickSelect algorithm to find the \(k\) closest points without fully sorting the list.

\begin{enumerate}
    \item Choose a pivot point and partition the list based on distances relative to the pivot.
    \item Recursively apply QuickSelect to the partition containing the \(k\) closest points.
    \item Once the \(k\) closest points are identified, return them.
\end{enumerate}

\marginnote{QuickSelect offers an average time complexity of \(O(n)\), making it highly efficient for large datasets.}

\section*{Complexities}

\begin{itemize}
    \item \textbf{Sorting Based on Distance:}
    \begin{itemize}
        \item \textbf{Time Complexity:} \(O(n \log n)\)
        \item \textbf{Space Complexity:} \(O(n)\)
    \end{itemize}
    
    \item \textbf{Max Heap (Priority Queue):}
    \begin{itemize}
        \item \textbf{Time Complexity:} \(O(n \log k)\)
        \item \textbf{Space Complexity:} \(O(k)\)
    \end{itemize}
    
    \item \textbf{QuickSelect (Quick Sort Partitioning):}
    \begin{itemize}
        \item \textbf{Time Complexity:} Average case \(O(n)\), worst case \(O(n^2)\)
        \item \textbf{Space Complexity:} \(O(1)\) (in-place)
    \end{itemize}
\end{itemize}

\section*{Python Implementation}

\marginnote{Implementing QuickSelect provides an optimal average-case solution with linear time complexity.}

Below is the complete Python code implementing the \texttt{kClosest} function using the QuickSelect approach:

\begin{fullwidth}
\begin{lstlisting}[language=Python]
from typing import List
import random

class Solution:
    def kClosest(self, points: List[List[int]], K: int) -> List[List[int]]:
        def quickselect(left, right, K_smallest):
            if left == right:
                return
            
            # Select a random pivot_index
            pivot_index = random.randint(left, right)
            
            # Partition the array
            pivot_index = partition(left, right, pivot_index)
            
            # The pivot is in its final sorted position
            if K_smallest == pivot_index:
                return
            elif K_smallest < pivot_index:
                quickselect(left, pivot_index - 1, K_smallest)
            else:
                quickselect(pivot_index + 1, right, K_smallest)
        
        def partition(left, right, pivot_index):
            pivot_distance = distance(points[pivot_index])
            # Move pivot to end
            points[pivot_index], points[right] = points[right], points[pivot_index]
            store_index = left
            for i in range(left, right):
                if distance(points[i]) < pivot_distance:
                    points[store_index], points[i] = points[i], points[store_index]
                    store_index += 1
            # Move pivot to its final place
            points[right], points[store_index] = points[store_index], points[right]
            return store_index
        
        def distance(point):
            return point[0] ** 2 + point[1] ** 2
        
        n = len(points)
        quickselect(0, n - 1, K)
        return points[:K]

# Example usage:
solution = Solution()
print(solution.kClosest([[1,3],[-2,2]], 1))            # Output: [[-2,2]]
print(solution.kClosest([[3,3],[5,-1],[-2,4]], 2))     # Output: [[3,3],[-2,4]]
print(solution.kClosest([[0,1],[1,0]], 2))             # Output: [[0,1],[1,0]]
print(solution.kClosest([[1,0],[0,1]], 1))             # Output: [[1,0]] or [[0,1]]
\end{lstlisting}
\end{fullwidth}

This implementation uses the QuickSelect algorithm to efficiently find the \(k\) closest points to the origin without fully sorting the entire list. It ensures optimal performance even with large datasets.

\section*{Explanation}

The \texttt{kClosest} function identifies the \(k\) closest points to the origin using the QuickSelect algorithm. Here's a detailed breakdown of the implementation:

\subsection*{1. Distance Calculation}

\begin{itemize}
    \item The Euclidean distance is calculated as \(distance = x^2 + y^2\). Since we only need relative distances for comparison, the square root is omitted for efficiency.
\end{itemize}

\subsection*{2. QuickSelect Algorithm}

\begin{itemize}
    \item **Pivot Selection:**
    \begin{itemize}
        \item A random pivot is chosen to enhance the average-case performance.
    \end{itemize}
    
    \item **Partitioning:**
    \begin{itemize}
        \item The array is partitioned such that points with distances less than the pivot are moved to the left, and others to the right.
        \item The pivot is placed in its correct sorted position.
    \end{itemize}
    
    \item **Recursive Selection:**
    \begin{itemize}
        \item If the pivot's position matches \(K\), the selection is complete.
        \item Otherwise, recursively apply QuickSelect to the relevant partition.
    \end{itemize}
\end{itemize}

\subsection*{3. Final Selection}

\begin{itemize}
    \item After partitioning, the first \(K\) points in the list are the \(k\) closest points to the origin.
\end{itemize}

\subsection*{4. Example Walkthrough}

Consider the first example:
\begin{verbatim}
Input: points = [[1,3],[-2,2]], K = 1
Output: [[-2,2]]
\end{verbatim}

\begin{enumerate}
    \item **Calculate Distances:**
    \begin{itemize}
        \item [1,3] : \(1^2 + 3^2 = 10\)
        \item [-2,2] : \((-2)^2 + 2^2 = 8\)
    \end{itemize}
    
    \item **QuickSelect Process:**
    \begin{itemize}
        \item Choose a pivot, say [1,3] with distance 10.
        \item Compare and rearrange:
        \begin{itemize}
            \item [-2,2] has a smaller distance (8) and is moved to the left.
        \end{itemize}
        \item After partitioning, the list becomes [[-2,2], [1,3]].
        \item Since \(K = 1\), return the first point: [[-2,2]].
    \end{itemize}
\end{enumerate}

Thus, the function correctly identifies \([-2,2]\) as the closest point to the origin.

\section*{Why This Approach}

The QuickSelect algorithm is chosen for its average-case linear time complexity \(O(n)\), making it highly efficient for large datasets compared to sorting-based methods with \(O(n \log n)\) time complexity. By avoiding the need to sort the entire list, QuickSelect provides an optimal solution for finding the \(k\) closest points.

\section*{Alternative Approaches}

\subsection*{1. Sorting Based on Distance}

Sort all points based on their distances from the origin and select the first \(k\) points.

\begin{lstlisting}[language=Python]
class Solution:
    def kClosest(self, points: List[List[int]], K: int) -> List[List[int]]:
        points.sort(key=lambda P: P[0]**2 + P[1]**2)
        return points[:K]
\end{lstlisting}

\textbf{Complexities:}
\begin{itemize}
    \item \textbf{Time Complexity:} \(O(n \log n)\)
    \item \textbf{Space Complexity:} \(O(1)\)
\end{itemize}

\subsection*{2. Max Heap (Priority Queue)}

Use a max heap to maintain the \(k\) closest points.

\begin{lstlisting}[language=Python]
import heapq

class Solution:
    def kClosest(self, points: List[List[int]], K: int) -> List[List[int]]:
        heap = []
        for (x, y) in points:
            dist = -(x**2 + y**2)  # Max heap using negative distances
            heapq.heappush(heap, (dist, [x, y]))
            if len(heap) > K:
                heapq.heappop(heap)
        return [item[1] for item in heap]
\end{lstlisting}

\textbf{Complexities:}
\begin{itemize}
    \item \textbf{Time Complexity:} \(O(n \log k)\)
    \item \textbf{Space Complexity:} \(O(k)\)
\end{itemize}

\subsection*{3. Using Built-In Functions}

Leverage built-in functions for distance calculation and selection.

\begin{lstlisting}[language=Python]
import math

class Solution:
    def kClosest(self, points: List[List[int]], K: int) -> List[List[int]]:
        points.sort(key=lambda P: math.sqrt(P[0]**2 + P[1]**2))
        return points[:K]
\end{lstlisting}

\textbf{Note}: This method is similar to the sorting approach but uses the actual Euclidean distance.

\section*{Similar Problems to This One}

Several problems involve nearest neighbor searches, spatial data analysis, and efficient selection algorithms, utilizing similar algorithmic strategies:

\begin{itemize}
    \item \textbf{Closest Pair of Points}: Find the closest pair of points in a set.
    \item \textbf{Top K Frequent Elements}: Identify the most frequent elements in a dataset.
    \item \textbf{Kth Largest Element in an Array}: Find the \(k\)-th largest element in an unsorted array.
    \item \textbf{Sliding Window Maximum}: Find the maximum in each sliding window of size \(k\) over an array.
    \item \textbf{Merge K Sorted Lists}: Merge multiple sorted lists into a single sorted list.
    \item \textbf{Find Median from Data Stream}: Continuously find the median of a stream of numbers.
    \item \textbf{Top K Closest Stars}: Find the \(k\) closest stars to Earth based on their distances.
\end{itemize}

These problems reinforce concepts of efficient selection, heap usage, and distance computations in various contexts.

\section*{Things to Keep in Mind and Tricks}

When solving the \textbf{K Closest Points to Origin} problem, consider the following tips and best practices to enhance efficiency and correctness:

\begin{itemize}
    \item \textbf{Understand Distance Calculations}: Grasp the Euclidean distance formula and recognize that the square root can be omitted for comparison purposes.
    \index{Distance Calculations}
    
    \item \textbf{Leverage Efficient Algorithms}: Use QuickSelect or heap-based methods to optimize time complexity, especially for large datasets.
    \index{Efficient Algorithms}
    
    \item \textbf{Handle Ties Appropriately}: Decide how to handle points with identical distances when \(k\) is less than the number of such points.
    \index{Handling Ties}
    
    \item \textbf{Optimize Space Usage}: Choose algorithms that minimize additional space, such as in-place QuickSelect.
    \index{Space Optimization}
    
    \item \textbf{Use Appropriate Data Structures}: Utilize heaps, lists, and helper functions effectively to manage and process data.
    \index{Data Structures}
    
    \item \textbf{Implement Helper Functions}: Create helper functions for distance calculation and partitioning to enhance code modularity.
    \index{Helper Functions}
    
    \item \textbf{Code Readability}: Maintain clear and readable code through meaningful variable names and structured logic.
    \index{Code Readability}
    
    \item \textbf{Test Extensively}: Implement a wide range of test cases, including edge cases like multiple points with the same distance, to ensure robustness.
    \index{Extensive Testing}
    
    \item \textbf{Understand Algorithm Trade-offs}: Recognize the trade-offs between different approaches in terms of time and space complexities.
    \index{Algorithm Trade-offs}
    
    \item \textbf{Use Built-In Sorting Functions}: When using sorting-based approaches, leverage built-in functions for efficiency and simplicity.
    \index{Built-In Sorting}
    
    \item \textbf{Avoid Redundant Calculations}: Ensure that distance calculations are performed only when necessary to optimize performance.
    \index{Avoiding Redundant Calculations}
    
    \item \textbf{Language-Specific Features}: Utilize language-specific features or libraries that can simplify implementation, such as heapq in Python.
    \index{Language-Specific Features}
\end{itemize}

\section*{Corner and Special Cases to Test When Writing the Code}

When implementing the solution for the \textbf{K Closest Points to Origin} problem, it is crucial to consider and rigorously test various edge cases to ensure robustness and correctness:

\begin{itemize}
    \item \textbf{Multiple Points with Same Distance}: Ensure that the algorithm handles multiple points having the same distance from the origin.
    \index{Same Distance Points}
    
    \item \textbf{Points at Origin}: Include points that are exactly at the origin \((0,0)\).
    \index{Points at Origin}
    
    \item \textbf{Negative Coordinates}: Ensure that the algorithm correctly computes distances for points with negative \(x\) or \(y\) coordinates.
    \index{Negative Coordinates}
    
    \item \textbf{Large Coordinates}: Test with points having very large or very small coordinate values to verify integer handling.
    \index{Large Coordinates}
    
    \item \textbf{K Equals Number of Points}: When \(K\) is equal to the number of points, the algorithm should return all points.
    \index{K Equals Number of Points}
    
    \item \textbf{K is One}: Test with \(K = 1\) to ensure the closest point is correctly identified.
    \index{K is One}
    
    \item \textbf{All Points Same}: All points have the same coordinates.
    \index{All Points Same}
    
    \item \textbf{K is Zero}: Although \(K\) is defined to be at least 1, ensure that the algorithm gracefully handles \(K = 0\) if allowed.
    \index{K is Zero}
    
    \item \textbf{Single Point}: Only one point is provided, and \(K = 1\).
    \index{Single Point}
    
    \item \textbf{Mixed Coordinates}: Points with a mix of positive and negative coordinates.
    \index{Mixed Coordinates}
    
    \item \textbf{Points with Zero Distance}: Multiple points at the origin.
    \index{Zero Distance Points}
    
    \item \textbf{Sparse and Dense Points}: Densely packed points and sparsely distributed points.
    \index{Sparse and Dense Points}
    
    \item \textbf{Duplicate Points}: Multiple identical points in the input list.
    \index{Duplicate Points}
    
    \item \textbf{K Greater Than Number of Unique Points}: Ensure that the algorithm handles cases where \(K\) exceeds the number of unique points if applicable.
    \index{K Greater Than Unique Points}
\end{itemize}

\section*{Implementation Considerations}

When implementing the \texttt{kClosest} function, keep in mind the following considerations to ensure robustness and efficiency:

\begin{itemize}
    \item \textbf{Data Type Selection}: Use appropriate data types that can handle large input values without overflow or precision loss.
    \index{Data Type Selection}
    
    \item \textbf{Optimizing Distance Calculations}: Avoid calculating the square root since it is unnecessary for comparison purposes.
    \index{Optimizing Distance Calculations}
    
    \item \textbf{Choosing the Right Algorithm}: Select an algorithm based on the size of the input and the value of \(K\) to optimize time and space complexities.
    \index{Choosing the Right Algorithm}
    
    \item \textbf{Language-Specific Libraries}: Utilize language-specific libraries and functions (e.g., \texttt{heapq} in Python) to simplify implementation and enhance performance.
    \index{Language-Specific Libraries}
    
    \item \textbf{Avoiding Redundant Calculations}: Ensure that each point's distance is calculated only once to optimize performance.
    \index{Avoiding Redundant Calculations}
    
    \item \textbf{Implementing Helper Functions}: Create helper functions for tasks like distance calculation and partitioning to enhance modularity and readability.
    \index{Helper Functions}
    
    \item \textbf{Edge Case Handling}: Implement checks for edge cases to prevent incorrect results or runtime errors.
    \index{Edge Case Handling}
    
    \item \textbf{Testing and Validation}: Develop a comprehensive suite of test cases that cover all possible scenarios, including edge cases, to validate the correctness and efficiency of the implementation.
    \index{Testing and Validation}
    
    \item \textbf{Scalability}: Design the algorithm to scale efficiently with increasing input sizes, maintaining performance and resource utilization.
    \index{Scalability}
    
    \item \textbf{Consistent Naming Conventions}: Use consistent and descriptive naming conventions for variables and functions to improve code clarity.
    \index{Naming Conventions}
    
    \item \textbf{Memory Management}: Ensure that the algorithm manages memory efficiently, especially when dealing with large datasets.
    \index{Memory Management}
    
    \item \textbf{Avoiding Stack Overflow}: If implementing recursive approaches, be mindful of recursion limits and potential stack overflow issues.
    \index{Avoiding Stack Overflow}
    
    \item \textbf{Implementing Iterative Solutions}: Prefer iterative solutions when recursion may lead to increased space complexity or stack overflow.
    \index{Implementing Iterative Solutions}
\end{itemize}

\section*{Conclusion}

The \textbf{K Closest Points to Origin} problem exemplifies the application of efficient selection algorithms and geometric computations to solve spatial challenges effectively. By leveraging QuickSelect or heap-based methods, the algorithm achieves optimal time and space complexities, making it highly suitable for large datasets. Understanding and implementing such techniques not only enhances problem-solving skills but also provides a foundation for tackling more advanced Computational Geometry problems involving nearest neighbor searches, clustering, and spatial data analysis.

\printindex

% \input{sections/rectangle_overlap}
% \input{sections/rectangle_area}
% \input{sections/k_closest_points_to_origin}
% \input{sections/the_skyline_problem}
% % filename: the_skyline_problem.tex

\problemsection{The Skyline Problem}
\label{chap:The_Skyline_Problem}
\marginnote{\href{https://leetcode.com/problems/the-skyline-problem/}{[LeetCode Link]}\index{LeetCode}}
\marginnote{\href{https://www.geeksforgeeks.org/the-skyline-problem/}{[GeeksForGeeks Link]}\index{GeeksForGeeks}}
\marginnote{\href{https://www.interviewbit.com/problems/the-skyline-problem/}{[InterviewBit Link]}\index{InterviewBit}}
\marginnote{\href{https://app.codesignal.com/challenges/the-skyline-problem}{[CodeSignal Link]}\index{CodeSignal}}
\marginnote{\href{https://www.codewars.com/kata/the-skyline-problem/train/python}{[Codewars Link]}\index{Codewars}}

The \textbf{Skyline Problem} is a complex Computational Geometry challenge that involves computing the skyline formed by a collection of buildings in a 2D cityscape. Each building is represented by its left and right x-coordinates and its height. The skyline is defined by a list of "key points" where the height changes. This problem tests one's ability to handle large datasets, implement efficient sweep line algorithms, and manage event-driven processing. Mastery of this problem is essential for applications in computer graphics, urban planning simulations, and geographic information systems (GIS).

\section*{Problem Statement}

You are given a list of buildings in a cityscape. Each building is represented as a triplet \([Li, Ri, Hi]\), where \(Li\) and \(Ri\) are the x-coordinates of the left and right edges of the building, respectively, and \(Hi\) is the height of the building.

The skyline should be represented as a list of key points \([x, y]\) in sorted order by \(x\)-coordinate, where \(y\) is the height of the skyline at that point. The skyline should only include critical points where the height changes.

\textbf{Function signature in Python:}
\begin{lstlisting}[language=Python]
def getSkyline(buildings: List[List[int]]) -> List[List[int]]:
\end{lstlisting}

\section*{Examples}

\textbf{Example 1:}

\begin{verbatim}
Input: buildings = [[2,9,10], [3,7,15], [5,12,12], [15,20,10], [19,24,8]]
Output: [[2,10], [3,15], [7,12], [12,0], [15,10], [20,8], [24,0]]
Explanation:
- At x=2, the first building starts, height=10.
- At x=3, the second building starts, height=15.
- At x=7, the second building ends, the third building is still ongoing, height=12.
- At x=12, the third building ends, height drops to 0.
- At x=15, the fourth building starts, height=10.
- At x=20, the fourth building ends, the fifth building is still ongoing, height=8.
- At x=24, the fifth building ends, height drops to 0.
\end{verbatim}

\textbf{Example 2:}

\begin{verbatim}
Input: buildings = [[0,2,3], [2,5,3]]
Output: [[0,3], [5,0]]
Explanation:
- The two buildings are contiguous and have the same height, so the skyline drops to 0 at x=5.
\end{verbatim}

\textbf{Example 3:}

\begin{verbatim}
Input: buildings = [[1,3,3], [2,4,4], [5,6,1]]
Output: [[1,3], [2,4], [4,0], [5,1], [6,0]]
Explanation:
- At x=1, first building starts, height=3.
- At x=2, second building starts, height=4.
- At x=4, second building ends, height drops to 0.
- At x=5, third building starts, height=1.
- At x=6, third building ends, height drops to 0.
\end{verbatim}

\textbf{Example 4:}

\begin{verbatim}
Input: buildings = [[0,5,0]]
Output: []
Explanation:
- A building with height 0 does not contribute to the skyline.
\end{verbatim}

\textbf{Constraints:}

\begin{itemize}
    \item \(1 \leq \text{buildings.length} \leq 10^4\)
    \item \(0 \leq Li < Ri \leq 10^9\)
    \item \(0 \leq Hi \leq 10^4\)
\end{itemize}

\section*{Algorithmic Approach}

The \textbf{Sweep Line Algorithm} is an efficient method for solving the Skyline Problem. It involves processing events (building start and end points) in sorted order while maintaining a data structure (typically a max heap) to keep track of active buildings. Here's a step-by-step approach:

\subsection*{1. Event Representation}

Transform each building into two events:
\begin{itemize}
    \item **Start Event:** \((Li, -Hi)\) – Negative height indicates a building starts.
    \item **End Event:** \((Ri, Hi)\) – Positive height indicates a building ends.
\end{itemize}

Sorting the events ensures that start events are processed before end events at the same x-coordinate, and taller buildings are processed before shorter ones.

\subsection*{2. Sorting the Events}

Sort all events based on:
\begin{enumerate}
    \item **x-coordinate:** Ascending order.
    \item **Height:**
    \begin{itemize}
        \item For start events, taller buildings come first.
        \item For end events, shorter buildings come first.
    \end{itemize}
\end{enumerate}

\subsection*{3. Processing the Events}

Use a max heap to keep track of active building heights. Iterate through the sorted events:
\begin{enumerate}
    \item **Start Event:**
    \begin{itemize}
        \item Add the building's height to the heap.
    \end{itemize}
    
    \item **End Event:**
    \begin{itemize}
        \item Remove the building's height from the heap.
    \end{itemize}
    
    \item **Determine Current Max Height:**
    \begin{itemize}
        \item The current max height is the top of the heap.
    \end{itemize}
    
    \item **Update Skyline:**
    \begin{itemize}
        \item If the current max height differs from the previous max height, add a new key point \([x, current\_max\_height]\).
    \end{itemize}
\end{enumerate}

\subsection*{4. Finalizing the Skyline}

After processing all events, the accumulated key points represent the skyline.

\marginnote{The Sweep Line Algorithm efficiently handles dynamic changes in active buildings, ensuring accurate skyline construction.}

\section*{Complexities}

\begin{itemize}
    \item \textbf{Time Complexity:} \(O(n \log n)\), where \(n\) is the number of buildings. Sorting the events takes \(O(n \log n)\), and each heap operation takes \(O(\log n)\).
    
    \item \textbf{Space Complexity:} \(O(n)\), due to the storage of events and the heap.
\end{itemize}

\section*{Python Implementation}

\marginnote{Implementing the Sweep Line Algorithm with a max heap ensures an efficient and accurate solution.}

Below is the complete Python code implementing the \texttt{getSkyline} function:

\begin{fullwidth}
\begin{lstlisting}[language=Python]
from typing import List
import heapq

class Solution:
    def getSkyline(self, buildings: List[List[int]]) -> List[List[int]]:
        # Create a list of all events
        # For start events, use negative height to ensure they are processed before end events
        events = []
        for L, R, H in buildings:
            events.append((L, -H))
            events.append((R, H))
        
        # Sort the events
        # First by x-coordinate, then by height
        events.sort()
        
        # Max heap to keep track of active buildings
        heap = [0]  # Initialize with ground level
        heapq.heapify(heap)
        active_heights = {0: 1}  # Dictionary to count heights
        
        result = []
        prev_max = 0
        
        for x, h in events:
            if h < 0:
                # Start of a building, add height to heap and dictionary
                heapq.heappush(heap, h)
                active_heights[h] = active_heights.get(h, 0) + 1
            else:
                # End of a building, remove height from dictionary
                active_heights[h] -= 1
                if active_heights[h] == 0:
                    del active_heights[h]
            
            # Current max height
            while heap and active_heights.get(heap[0], 0) == 0:
                heapq.heappop(heap)
            current_max = -heap[0] if heap else 0
            
            # If the max height has changed, add to result
            if current_max != prev_max:
                result.append([x, current_max])
                prev_max = current_max
        
        return result

# Example usage:
solution = Solution()
print(solution.getSkyline([[2,9,10], [3,7,15], [5,12,12], [15,20,10], [19,24,8]]))
# Output: [[2,10], [3,15], [7,12], [12,0], [15,10], [20,8], [24,0]]

print(solution.getSkyline([[0,2,3], [2,5,3]]))
# Output: [[0,3], [5,0]]

print(solution.getSkyline([[1,3,3], [2,4,4], [5,6,1]]))
# Output: [[1,3], [2,4], [4,0], [5,1], [6,0]]

print(solution.getSkyline([[0,5,0]]))
# Output: []
\end{lstlisting}
\end{fullwidth}

This implementation efficiently constructs the skyline by processing all building events in sorted order and maintaining active building heights using a max heap. It ensures that only critical points where the skyline changes are recorded.

\section*{Explanation}

The \texttt{getSkyline} function constructs the skyline formed by a set of buildings by leveraging the Sweep Line Algorithm and a max heap to track active buildings. Here's a detailed breakdown of the implementation:

\subsection*{1. Event Representation}

\begin{itemize}
    \item Each building is transformed into two events:
    \begin{itemize}
        \item **Start Event:** \((Li, -Hi)\) – Negative height indicates the start of a building.
        \item **End Event:** \((Ri, Hi)\) – Positive height indicates the end of a building.
    \end{itemize}
\end{itemize}

\subsection*{2. Sorting the Events}

\begin{itemize}
    \item Events are sorted primarily by their x-coordinate in ascending order.
    \item For events with the same x-coordinate:
    \begin{itemize}
        \item Start events (with negative heights) are processed before end events.
        \item Taller buildings are processed before shorter ones.
    \end{itemize}
\end{itemize}

\subsection*{3. Processing the Events}

\begin{itemize}
    \item **Heap Initialization:**
    \begin{itemize}
        \item A max heap is initialized with a ground level height of 0.
        \item A dictionary \texttt{active\_heights} tracks the count of active building heights.
    \end{itemize}
    
    \item **Iterating Through Events:**
    \begin{enumerate}
        \item **Start Event:**
        \begin{itemize}
            \item Add the building's height to the heap.
            \item Increment the count of the height in \texttt{active\_heights}.
        \end{itemize}
        
        \item **End Event:**
        \begin{itemize}
            \item Decrement the count of the building's height in \texttt{active\_heights}.
            \item If the count reaches zero, remove the height from the dictionary.
        \end{itemize}
        
        \item **Determine Current Max Height:**
        \begin{itemize}
            \item Remove heights from the heap that are no longer active.
            \item The current max height is the top of the heap.
        \end{itemize}
        
        \item **Update Skyline:**
        \begin{itemize}
            \item If the current max height differs from the previous max height, add a new key point \([x, current\_max\_height]\).
        \end{itemize}
    \end{enumerate}
\end{itemize}

\subsection*{4. Finalizing the Skyline}

\begin{itemize}
    \item After processing all events, the \texttt{result} list contains the key points defining the skyline.
\end{itemize}

\subsection*{5. Example Walkthrough}

Consider the first example:
\begin{verbatim}
Input: buildings = [[2,9,10], [3,7,15], [5,12,12], [15,20,10], [19,24,8]]
Output: [[2,10], [3,15], [7,12], [12,0], [15,10], [20,8], [24,0]]
\end{verbatim}

\begin{enumerate}
    \item **Event Transformation:**
    \begin{itemize}
        \item \((2, -10)\), \((9, 10)\)
        \item \((3, -15)\), \((7, 15)\)
        \item \((5, -12)\), \((12, 12)\)
        \item \((15, -10)\), \((20, 10)\)
        \item \((19, -8)\), \((24, 8)\)
    \end{itemize}
    
    \item **Sorting Events:**
    \begin{itemize}
        \item Sorted order: \((2, -10)\), \((3, -15)\), \((5, -12)\), \((7, 15)\), \((9, 10)\), \((12, 12)\), \((15, -10)\), \((19, -8)\), \((20, 10)\), \((24, 8)\)
    \end{itemize}
    
    \item **Processing Events:**
    \begin{itemize}
        \item At each event, update the heap and determine if the skyline height changes.
    \end{itemize}
    
    \item **Result Construction:**
    \begin{itemize}
        \item The resulting skyline key points are accumulated as \([[2,10], [3,15], [7,12], [12,0], [15,10], [20,8], [24,0]]\).
    \end{itemize}
\end{enumerate}

Thus, the function correctly constructs the skyline based on the buildings' positions and heights.

\section*{Why This Approach}

The Sweep Line Algorithm combined with a max heap offers an optimal solution with \(O(n \log n)\) time complexity and efficient handling of overlapping buildings. By processing events in sorted order and maintaining active building heights, the algorithm ensures that all critical points in the skyline are accurately identified without redundant computations.

\section*{Alternative Approaches}

\subsection*{1. Divide and Conquer}

Divide the set of buildings into smaller subsets, compute the skyline for each subset, and then merge the skylines.

\begin{lstlisting}[language=Python]
class Solution:
    def getSkyline(self, buildings: List[List[int]]) -> List[List[int]]:
        def merge(left, right):
            h1, h2 = 0, 0
            i, j = 0, 0
            merged = []
            while i < len(left) and j < len(right):
                if left[i][0] < right[j][0]:
                    x, h1 = left[i]
                    i += 1
                elif left[i][0] > right[j][0]:
                    x, h2 = right[j]
                    j += 1
                else:
                    x, h1 = left[i]
                    _, h2 = right[j]
                    i += 1
                    j += 1
                max_h = max(h1, h2)
                if not merged or merged[-1][1] != max_h:
                    merged.append([x, max_h])
            merged.extend(left[i:])
            merged.extend(right[j:])
            return merged
        
        def divide(buildings):
            if not buildings:
                return []
            if len(buildings) == 1:
                L, R, H = buildings[0]
                return [[L, H], [R, 0]]
            mid = len(buildings) // 2
            left = divide(buildings[:mid])
            right = divide(buildings[mid:])
            return merge(left, right)
        
        return divide(buildings)
\end{lstlisting}

\textbf{Complexities:}
\begin{itemize}
    \item \textbf{Time Complexity:} \(O(n \log n)\)
    \item \textbf{Space Complexity:} \(O(n)\)
\end{itemize}

\subsection*{2. Using Segment Trees}

Implement a segment tree to manage and query overlapping building heights dynamically.

\textbf{Note}: This approach is more complex and is generally used for advanced scenarios with multiple dynamic queries.

\section*{Similar Problems to This One}

Several problems involve skyline-like constructions, spatial data analysis, and efficient event processing, utilizing similar algorithmic strategies:

\begin{itemize}
    \item \textbf{Merge Intervals}: Merge overlapping intervals in a list.
    \item \textbf{Largest Rectangle in Histogram}: Find the largest rectangular area in a histogram.
    \item \textbf{Interval Partitioning}: Assign intervals to resources without overlap.
    \item \textbf{Line Segment Intersection}: Detect intersections among line segments.
    \item \textbf{Closest Pair of Points}: Find the closest pair of points in a set.
    \item \textbf{Convex Hull}: Compute the convex hull of a set of points.
    \item \textbf{Point Inside Polygon}: Determine if a point lies inside a given polygon.
    \item \textbf{Range Searching}: Efficiently query geometric data within a specified range.
\end{itemize}

These problems reinforce concepts of event-driven processing, spatial reasoning, and efficient algorithm design in various contexts.

\section*{Things to Keep in Mind and Tricks}

When tackling the \textbf{Skyline Problem}, consider the following tips and best practices to enhance efficiency and correctness:

\begin{itemize}
    \item \textbf{Understand Sweep Line Technique}: Grasp how the sweep line algorithm processes events in sorted order to handle dynamic changes efficiently.
    \index{Sweep Line Technique}
    
    \item \textbf{Leverage Priority Queues (Heaps)}: Use max heaps to keep track of active buildings' heights, enabling quick access to the current maximum height.
    \index{Priority Queues}
    
    \item \textbf{Handle Start and End Events Differently}: Differentiate between building start and end events to accurately manage active heights.
    \index{Start and End Events}
    
    \item \textbf{Optimize Event Sorting}: Sort events primarily by x-coordinate and secondarily by height to ensure correct processing order.
    \index{Event Sorting}
    
    \item \textbf{Manage Active Heights Efficiently}: Use data structures that allow efficient insertion, deletion, and retrieval of maximum elements.
    \index{Active Heights Management}
    
    \item \textbf{Avoid Redundant Key Points}: Only record key points when the skyline height changes to minimize the output list.
    \index{Avoiding Redundant Key Points}
    
    \item \textbf{Implement Helper Functions}: Create helper functions for tasks like distance calculation, event handling, and heap management to enhance modularity.
    \index{Helper Functions}
    
    \item \textbf{Code Readability}: Maintain clear and readable code through meaningful variable names and structured logic.
    \index{Code Readability}
    
    \item \textbf{Test Extensively}: Implement a wide range of test cases, including overlapping, non-overlapping, and edge-touching buildings, to ensure robustness.
    \index{Extensive Testing}
    
    \item \textbf{Handle Degenerate Cases}: Manage cases where buildings have zero height or identical coordinates gracefully.
    \index{Degenerate Cases}
    
    \item \textbf{Understand Geometric Relationships}: Grasp how buildings overlap and influence the skyline to simplify the algorithm.
    \index{Geometric Relationships}
    
    \item \textbf{Use Appropriate Data Structures}: Utilize appropriate data structures like heaps, lists, and dictionaries to manage and process data efficiently.
    \index{Appropriate Data Structures}
    
    \item \textbf{Optimize for Large Inputs}: Design the algorithm to handle large numbers of buildings without significant performance degradation.
    \index{Optimizing for Large Inputs}
    
    \item \textbf{Implement Iterative Solutions Carefully}: Ensure that loop conditions are correctly defined to prevent infinite loops or incorrect terminations.
    \index{Iterative Solutions}
    
    \item \textbf{Consistent Naming Conventions}: Use consistent and descriptive naming conventions for variables and functions to improve code clarity.
    \index{Naming Conventions}
\end{itemize}

\section*{Corner and Special Cases to Test When Writing the Code}

When implementing the solution for the \textbf{Skyline Problem}, it is crucial to consider and rigorously test various edge cases to ensure robustness and correctness:

\begin{itemize}
    \item \textbf{No Overlapping Buildings}: All buildings are separate and do not overlap.
    \index{No Overlapping Buildings}
    
    \item \textbf{Fully Overlapping Buildings}: Multiple buildings completely overlap each other.
    \index{Fully Overlapping Buildings}
    
    \item \textbf{Buildings Touching at Edges}: Buildings share common edges without overlapping.
    \index{Buildings Touching at Edges}
    
    \item \textbf{Buildings Touching at Corners}: Buildings share common corners without overlapping.
    \index{Buildings Touching at Corners}
    
    \item \textbf{Single Building}: Only one building is present.
    \index{Single Building}
    
    \item \textbf{Multiple Buildings with Same Start or End}: Multiple buildings start or end at the same x-coordinate.
    \index{Same Start or End}
    
    \item \textbf{Buildings with Zero Height}: Buildings that have zero height should not affect the skyline.
    \index{Buildings with Zero Height}
    
    \item \textbf{Large Number of Buildings}: Test with a large number of buildings to ensure performance and scalability.
    \index{Large Number of Buildings}
    
    \item \textbf{Buildings with Negative Coordinates}: Buildings positioned in negative coordinate space.
    \index{Negative Coordinates}
    
    \item \textbf{Boundary Values}: Buildings at the minimum and maximum limits of the coordinate range.
    \index{Boundary Values}
    
    \item \textbf{Buildings with Identical Coordinates}: Multiple buildings with the same coordinates.
    \index{Identical Coordinates}
    
    \item \textbf{Sequential Buildings}: Buildings placed sequentially without gaps.
    \index{Sequential Buildings}
    
    \item \textbf{Overlapping and Non-Overlapping Mixed}: A mix of overlapping and non-overlapping buildings.
    \index{Overlapping and Non-Overlapping Mixed}
    
    \item \textbf{Buildings with Very Large Heights}: Buildings with heights at the upper limit of the constraints.
    \index{Very Large Heights}
    
    \item \textbf{Empty Input}: No buildings are provided.
    \index{Empty Input}
\end{itemize}

\section*{Implementation Considerations}

When implementing the \texttt{getSkyline} function, keep in mind the following considerations to ensure robustness and efficiency:

\begin{itemize}
    \item \textbf{Data Type Selection}: Use appropriate data types that can handle large input values and avoid overflow or precision issues.
    \index{Data Type Selection}
    
    \item \textbf{Optimizing Event Sorting}: Efficiently sort events based on x-coordinates and heights to ensure correct processing order.
    \index{Optimizing Event Sorting}
    
    \item \textbf{Handling Large Inputs}: Design the algorithm to handle up to \(10^4\) buildings efficiently without significant performance degradation.
    \index{Handling Large Inputs}
    
    \item \textbf{Using Efficient Data Structures}: Utilize heaps, lists, and dictionaries effectively to manage and process events and active heights.
    \index{Efficient Data Structures}
    
    \item \textbf{Avoiding Redundant Calculations}: Ensure that distance and overlap calculations are performed only when necessary to optimize performance.
    \index{Avoiding Redundant Calculations}
    
    \item \textbf{Code Readability and Documentation}: Maintain clear and readable code through meaningful variable names and comprehensive comments to facilitate understanding and maintenance.
    \index{Code Readability}
    
    \item \textbf{Edge Case Handling}: Implement checks for edge cases to prevent incorrect results or runtime errors.
    \index{Edge Case Handling}
    
    \item \textbf{Implementing Helper Functions}: Create helper functions for tasks like distance calculation, event handling, and heap management to enhance modularity.
    \index{Helper Functions}
    
    \item \textbf{Consistent Naming Conventions}: Use consistent and descriptive naming conventions for variables and functions to improve code clarity.
    \index{Naming Conventions}
    
    \item \textbf{Memory Management}: Ensure that the algorithm manages memory efficiently, especially when dealing with large datasets.
    \index{Memory Management}
    
    \item \textbf{Implementing Iterative Solutions Carefully}: Ensure that loop conditions are correctly defined to prevent infinite loops or incorrect terminations.
    \index{Iterative Solutions}
    
    \item \textbf{Avoiding Floating-Point Precision Issues}: Since the problem deals with integers, floating-point precision is not a concern, simplifying the implementation.
    \index{Floating-Point Precision}
    
    \item \textbf{Testing and Validation}: Develop a comprehensive suite of test cases that cover all possible scenarios, including edge cases, to validate the correctness and efficiency of the implementation.
    \index{Testing and Validation}
    
    \item \textbf{Performance Considerations}: Optimize the loop conditions and operations to ensure that the function runs efficiently, especially for large input numbers.
    \index{Performance Considerations}
\end{itemize}

\section*{Conclusion}

The \textbf{Skyline Problem} is a quintessential example of applying advanced algorithmic techniques and geometric reasoning to solve complex spatial challenges. By leveraging the Sweep Line Algorithm and maintaining active building heights using a max heap, the solution efficiently constructs the skyline with optimal time and space complexities. Understanding and implementing such sophisticated algorithms not only enhances problem-solving skills but also provides a foundation for tackling a wide array of Computational Geometry problems in various domains, including computer graphics, urban planning simulations, and geographic information systems.

\printindex

% \input{sections/rectangle_overlap}
% \input{sections/rectangle_area}
% \input{sections/k_closest_points_to_origin}
% \input{sections/the_skyline_problem}
% % filename: the_skyline_problem.tex

\problemsection{The Skyline Problem}
\label{chap:The_Skyline_Problem}
\marginnote{\href{https://leetcode.com/problems/the-skyline-problem/}{[LeetCode Link]}\index{LeetCode}}
\marginnote{\href{https://www.geeksforgeeks.org/the-skyline-problem/}{[GeeksForGeeks Link]}\index{GeeksForGeeks}}
\marginnote{\href{https://www.interviewbit.com/problems/the-skyline-problem/}{[InterviewBit Link]}\index{InterviewBit}}
\marginnote{\href{https://app.codesignal.com/challenges/the-skyline-problem}{[CodeSignal Link]}\index{CodeSignal}}
\marginnote{\href{https://www.codewars.com/kata/the-skyline-problem/train/python}{[Codewars Link]}\index{Codewars}}

The \textbf{Skyline Problem} is a complex Computational Geometry challenge that involves computing the skyline formed by a collection of buildings in a 2D cityscape. Each building is represented by its left and right x-coordinates and its height. The skyline is defined by a list of "key points" where the height changes. This problem tests one's ability to handle large datasets, implement efficient sweep line algorithms, and manage event-driven processing. Mastery of this problem is essential for applications in computer graphics, urban planning simulations, and geographic information systems (GIS).

\section*{Problem Statement}

You are given a list of buildings in a cityscape. Each building is represented as a triplet \([Li, Ri, Hi]\), where \(Li\) and \(Ri\) are the x-coordinates of the left and right edges of the building, respectively, and \(Hi\) is the height of the building.

The skyline should be represented as a list of key points \([x, y]\) in sorted order by \(x\)-coordinate, where \(y\) is the height of the skyline at that point. The skyline should only include critical points where the height changes.

\textbf{Function signature in Python:}
\begin{lstlisting}[language=Python]
def getSkyline(buildings: List[List[int]]) -> List[List[int]]:
\end{lstlisting}

\section*{Examples}

\textbf{Example 1:}

\begin{verbatim}
Input: buildings = [[2,9,10], [3,7,15], [5,12,12], [15,20,10], [19,24,8]]
Output: [[2,10], [3,15], [7,12], [12,0], [15,10], [20,8], [24,0]]
Explanation:
- At x=2, the first building starts, height=10.
- At x=3, the second building starts, height=15.
- At x=7, the second building ends, the third building is still ongoing, height=12.
- At x=12, the third building ends, height drops to 0.
- At x=15, the fourth building starts, height=10.
- At x=20, the fourth building ends, the fifth building is still ongoing, height=8.
- At x=24, the fifth building ends, height drops to 0.
\end{verbatim}

\textbf{Example 2:}

\begin{verbatim}
Input: buildings = [[0,2,3], [2,5,3]]
Output: [[0,3], [5,0]]
Explanation:
- The two buildings are contiguous and have the same height, so the skyline drops to 0 at x=5.
\end{verbatim}

\textbf{Example 3:}

\begin{verbatim}
Input: buildings = [[1,3,3], [2,4,4], [5,6,1]]
Output: [[1,3], [2,4], [4,0], [5,1], [6,0]]
Explanation:
- At x=1, first building starts, height=3.
- At x=2, second building starts, height=4.
- At x=4, second building ends, height drops to 0.
- At x=5, third building starts, height=1.
- At x=6, third building ends, height drops to 0.
\end{verbatim}

\textbf{Example 4:}

\begin{verbatim}
Input: buildings = [[0,5,0]]
Output: []
Explanation:
- A building with height 0 does not contribute to the skyline.
\end{verbatim}

\textbf{Constraints:}

\begin{itemize}
    \item \(1 \leq \text{buildings.length} \leq 10^4\)
    \item \(0 \leq Li < Ri \leq 10^9\)
    \item \(0 \leq Hi \leq 10^4\)
\end{itemize}

\section*{Algorithmic Approach}

The \textbf{Sweep Line Algorithm} is an efficient method for solving the Skyline Problem. It involves processing events (building start and end points) in sorted order while maintaining a data structure (typically a max heap) to keep track of active buildings. Here's a step-by-step approach:

\subsection*{1. Event Representation}

Transform each building into two events:
\begin{itemize}
    \item **Start Event:** \((Li, -Hi)\) – Negative height indicates a building starts.
    \item **End Event:** \((Ri, Hi)\) – Positive height indicates a building ends.
\end{itemize}

Sorting the events ensures that start events are processed before end events at the same x-coordinate, and taller buildings are processed before shorter ones.

\subsection*{2. Sorting the Events}

Sort all events based on:
\begin{enumerate}
    \item **x-coordinate:** Ascending order.
    \item **Height:**
    \begin{itemize}
        \item For start events, taller buildings come first.
        \item For end events, shorter buildings come first.
    \end{itemize}
\end{enumerate}

\subsection*{3. Processing the Events}

Use a max heap to keep track of active building heights. Iterate through the sorted events:
\begin{enumerate}
    \item **Start Event:**
    \begin{itemize}
        \item Add the building's height to the heap.
    \end{itemize}
    
    \item **End Event:**
    \begin{itemize}
        \item Remove the building's height from the heap.
    \end{itemize}
    
    \item **Determine Current Max Height:**
    \begin{itemize}
        \item The current max height is the top of the heap.
    \end{itemize}
    
    \item **Update Skyline:**
    \begin{itemize}
        \item If the current max height differs from the previous max height, add a new key point \([x, current\_max\_height]\).
    \end{itemize}
\end{enumerate}

\subsection*{4. Finalizing the Skyline}

After processing all events, the accumulated key points represent the skyline.

\marginnote{The Sweep Line Algorithm efficiently handles dynamic changes in active buildings, ensuring accurate skyline construction.}

\section*{Complexities}

\begin{itemize}
    \item \textbf{Time Complexity:} \(O(n \log n)\), where \(n\) is the number of buildings. Sorting the events takes \(O(n \log n)\), and each heap operation takes \(O(\log n)\).
    
    \item \textbf{Space Complexity:} \(O(n)\), due to the storage of events and the heap.
\end{itemize}

\section*{Python Implementation}

\marginnote{Implementing the Sweep Line Algorithm with a max heap ensures an efficient and accurate solution.}

Below is the complete Python code implementing the \texttt{getSkyline} function:

\begin{fullwidth}
\begin{lstlisting}[language=Python]
from typing import List
import heapq

class Solution:
    def getSkyline(self, buildings: List[List[int]]) -> List[List[int]]:
        # Create a list of all events
        # For start events, use negative height to ensure they are processed before end events
        events = []
        for L, R, H in buildings:
            events.append((L, -H))
            events.append((R, H))
        
        # Sort the events
        # First by x-coordinate, then by height
        events.sort()
        
        # Max heap to keep track of active buildings
        heap = [0]  # Initialize with ground level
        heapq.heapify(heap)
        active_heights = {0: 1}  # Dictionary to count heights
        
        result = []
        prev_max = 0
        
        for x, h in events:
            if h < 0:
                # Start of a building, add height to heap and dictionary
                heapq.heappush(heap, h)
                active_heights[h] = active_heights.get(h, 0) + 1
            else:
                # End of a building, remove height from dictionary
                active_heights[h] -= 1
                if active_heights[h] == 0:
                    del active_heights[h]
            
            # Current max height
            while heap and active_heights.get(heap[0], 0) == 0:
                heapq.heappop(heap)
            current_max = -heap[0] if heap else 0
            
            # If the max height has changed, add to result
            if current_max != prev_max:
                result.append([x, current_max])
                prev_max = current_max
        
        return result

# Example usage:
solution = Solution()
print(solution.getSkyline([[2,9,10], [3,7,15], [5,12,12], [15,20,10], [19,24,8]]))
# Output: [[2,10], [3,15], [7,12], [12,0], [15,10], [20,8], [24,0]]

print(solution.getSkyline([[0,2,3], [2,5,3]]))
# Output: [[0,3], [5,0]]

print(solution.getSkyline([[1,3,3], [2,4,4], [5,6,1]]))
# Output: [[1,3], [2,4], [4,0], [5,1], [6,0]]

print(solution.getSkyline([[0,5,0]]))
# Output: []
\end{lstlisting}
\end{fullwidth}

This implementation efficiently constructs the skyline by processing all building events in sorted order and maintaining active building heights using a max heap. It ensures that only critical points where the skyline changes are recorded.

\section*{Explanation}

The \texttt{getSkyline} function constructs the skyline formed by a set of buildings by leveraging the Sweep Line Algorithm and a max heap to track active buildings. Here's a detailed breakdown of the implementation:

\subsection*{1. Event Representation}

\begin{itemize}
    \item Each building is transformed into two events:
    \begin{itemize}
        \item **Start Event:** \((Li, -Hi)\) – Negative height indicates the start of a building.
        \item **End Event:** \((Ri, Hi)\) – Positive height indicates the end of a building.
    \end{itemize}
\end{itemize}

\subsection*{2. Sorting the Events}

\begin{itemize}
    \item Events are sorted primarily by their x-coordinate in ascending order.
    \item For events with the same x-coordinate:
    \begin{itemize}
        \item Start events (with negative heights) are processed before end events.
        \item Taller buildings are processed before shorter ones.
    \end{itemize}
\end{itemize}

\subsection*{3. Processing the Events}

\begin{itemize}
    \item **Heap Initialization:**
    \begin{itemize}
        \item A max heap is initialized with a ground level height of 0.
        \item A dictionary \texttt{active\_heights} tracks the count of active building heights.
    \end{itemize}
    
    \item **Iterating Through Events:**
    \begin{enumerate}
        \item **Start Event:**
        \begin{itemize}
            \item Add the building's height to the heap.
            \item Increment the count of the height in \texttt{active\_heights}.
        \end{itemize}
        
        \item **End Event:**
        \begin{itemize}
            \item Decrement the count of the building's height in \texttt{active\_heights}.
            \item If the count reaches zero, remove the height from the dictionary.
        \end{itemize}
        
        \item **Determine Current Max Height:**
        \begin{itemize}
            \item Remove heights from the heap that are no longer active.
            \item The current max height is the top of the heap.
        \end{itemize}
        
        \item **Update Skyline:**
        \begin{itemize}
            \item If the current max height differs from the previous max height, add a new key point \([x, current\_max\_height]\).
        \end{itemize}
    \end{enumerate}
\end{itemize}

\subsection*{4. Finalizing the Skyline}

\begin{itemize}
    \item After processing all events, the \texttt{result} list contains the key points defining the skyline.
\end{itemize}

\subsection*{5. Example Walkthrough}

Consider the first example:
\begin{verbatim}
Input: buildings = [[2,9,10], [3,7,15], [5,12,12], [15,20,10], [19,24,8]]
Output: [[2,10], [3,15], [7,12], [12,0], [15,10], [20,8], [24,0]]
\end{verbatim}

\begin{enumerate}
    \item **Event Transformation:**
    \begin{itemize}
        \item \((2, -10)\), \((9, 10)\)
        \item \((3, -15)\), \((7, 15)\)
        \item \((5, -12)\), \((12, 12)\)
        \item \((15, -10)\), \((20, 10)\)
        \item \((19, -8)\), \((24, 8)\)
    \end{itemize}
    
    \item **Sorting Events:**
    \begin{itemize}
        \item Sorted order: \((2, -10)\), \((3, -15)\), \((5, -12)\), \((7, 15)\), \((9, 10)\), \((12, 12)\), \((15, -10)\), \((19, -8)\), \((20, 10)\), \((24, 8)\)
    \end{itemize}
    
    \item **Processing Events:**
    \begin{itemize}
        \item At each event, update the heap and determine if the skyline height changes.
    \end{itemize}
    
    \item **Result Construction:**
    \begin{itemize}
        \item The resulting skyline key points are accumulated as \([[2,10], [3,15], [7,12], [12,0], [15,10], [20,8], [24,0]]\).
    \end{itemize}
\end{enumerate}

Thus, the function correctly constructs the skyline based on the buildings' positions and heights.

\section*{Why This Approach}

The Sweep Line Algorithm combined with a max heap offers an optimal solution with \(O(n \log n)\) time complexity and efficient handling of overlapping buildings. By processing events in sorted order and maintaining active building heights, the algorithm ensures that all critical points in the skyline are accurately identified without redundant computations.

\section*{Alternative Approaches}

\subsection*{1. Divide and Conquer}

Divide the set of buildings into smaller subsets, compute the skyline for each subset, and then merge the skylines.

\begin{lstlisting}[language=Python]
class Solution:
    def getSkyline(self, buildings: List[List[int]]) -> List[List[int]]:
        def merge(left, right):
            h1, h2 = 0, 0
            i, j = 0, 0
            merged = []
            while i < len(left) and j < len(right):
                if left[i][0] < right[j][0]:
                    x, h1 = left[i]
                    i += 1
                elif left[i][0] > right[j][0]:
                    x, h2 = right[j]
                    j += 1
                else:
                    x, h1 = left[i]
                    _, h2 = right[j]
                    i += 1
                    j += 1
                max_h = max(h1, h2)
                if not merged or merged[-1][1] != max_h:
                    merged.append([x, max_h])
            merged.extend(left[i:])
            merged.extend(right[j:])
            return merged
        
        def divide(buildings):
            if not buildings:
                return []
            if len(buildings) == 1:
                L, R, H = buildings[0]
                return [[L, H], [R, 0]]
            mid = len(buildings) // 2
            left = divide(buildings[:mid])
            right = divide(buildings[mid:])
            return merge(left, right)
        
        return divide(buildings)
\end{lstlisting}

\textbf{Complexities:}
\begin{itemize}
    \item \textbf{Time Complexity:} \(O(n \log n)\)
    \item \textbf{Space Complexity:} \(O(n)\)
\end{itemize}

\subsection*{2. Using Segment Trees}

Implement a segment tree to manage and query overlapping building heights dynamically.

\textbf{Note}: This approach is more complex and is generally used for advanced scenarios with multiple dynamic queries.

\section*{Similar Problems to This One}

Several problems involve skyline-like constructions, spatial data analysis, and efficient event processing, utilizing similar algorithmic strategies:

\begin{itemize}
    \item \textbf{Merge Intervals}: Merge overlapping intervals in a list.
    \item \textbf{Largest Rectangle in Histogram}: Find the largest rectangular area in a histogram.
    \item \textbf{Interval Partitioning}: Assign intervals to resources without overlap.
    \item \textbf{Line Segment Intersection}: Detect intersections among line segments.
    \item \textbf{Closest Pair of Points}: Find the closest pair of points in a set.
    \item \textbf{Convex Hull}: Compute the convex hull of a set of points.
    \item \textbf{Point Inside Polygon}: Determine if a point lies inside a given polygon.
    \item \textbf{Range Searching}: Efficiently query geometric data within a specified range.
\end{itemize}

These problems reinforce concepts of event-driven processing, spatial reasoning, and efficient algorithm design in various contexts.

\section*{Things to Keep in Mind and Tricks}

When tackling the \textbf{Skyline Problem}, consider the following tips and best practices to enhance efficiency and correctness:

\begin{itemize}
    \item \textbf{Understand Sweep Line Technique}: Grasp how the sweep line algorithm processes events in sorted order to handle dynamic changes efficiently.
    \index{Sweep Line Technique}
    
    \item \textbf{Leverage Priority Queues (Heaps)}: Use max heaps to keep track of active buildings' heights, enabling quick access to the current maximum height.
    \index{Priority Queues}
    
    \item \textbf{Handle Start and End Events Differently}: Differentiate between building start and end events to accurately manage active heights.
    \index{Start and End Events}
    
    \item \textbf{Optimize Event Sorting}: Sort events primarily by x-coordinate and secondarily by height to ensure correct processing order.
    \index{Event Sorting}
    
    \item \textbf{Manage Active Heights Efficiently}: Use data structures that allow efficient insertion, deletion, and retrieval of maximum elements.
    \index{Active Heights Management}
    
    \item \textbf{Avoid Redundant Key Points}: Only record key points when the skyline height changes to minimize the output list.
    \index{Avoiding Redundant Key Points}
    
    \item \textbf{Implement Helper Functions}: Create helper functions for tasks like distance calculation, event handling, and heap management to enhance modularity.
    \index{Helper Functions}
    
    \item \textbf{Code Readability}: Maintain clear and readable code through meaningful variable names and structured logic.
    \index{Code Readability}
    
    \item \textbf{Test Extensively}: Implement a wide range of test cases, including overlapping, non-overlapping, and edge-touching buildings, to ensure robustness.
    \index{Extensive Testing}
    
    \item \textbf{Handle Degenerate Cases}: Manage cases where buildings have zero height or identical coordinates gracefully.
    \index{Degenerate Cases}
    
    \item \textbf{Understand Geometric Relationships}: Grasp how buildings overlap and influence the skyline to simplify the algorithm.
    \index{Geometric Relationships}
    
    \item \textbf{Use Appropriate Data Structures}: Utilize appropriate data structures like heaps, lists, and dictionaries to manage and process data efficiently.
    \index{Appropriate Data Structures}
    
    \item \textbf{Optimize for Large Inputs}: Design the algorithm to handle large numbers of buildings without significant performance degradation.
    \index{Optimizing for Large Inputs}
    
    \item \textbf{Implement Iterative Solutions Carefully}: Ensure that loop conditions are correctly defined to prevent infinite loops or incorrect terminations.
    \index{Iterative Solutions}
    
    \item \textbf{Consistent Naming Conventions}: Use consistent and descriptive naming conventions for variables and functions to improve code clarity.
    \index{Naming Conventions}
\end{itemize}

\section*{Corner and Special Cases to Test When Writing the Code}

When implementing the solution for the \textbf{Skyline Problem}, it is crucial to consider and rigorously test various edge cases to ensure robustness and correctness:

\begin{itemize}
    \item \textbf{No Overlapping Buildings}: All buildings are separate and do not overlap.
    \index{No Overlapping Buildings}
    
    \item \textbf{Fully Overlapping Buildings}: Multiple buildings completely overlap each other.
    \index{Fully Overlapping Buildings}
    
    \item \textbf{Buildings Touching at Edges}: Buildings share common edges without overlapping.
    \index{Buildings Touching at Edges}
    
    \item \textbf{Buildings Touching at Corners}: Buildings share common corners without overlapping.
    \index{Buildings Touching at Corners}
    
    \item \textbf{Single Building}: Only one building is present.
    \index{Single Building}
    
    \item \textbf{Multiple Buildings with Same Start or End}: Multiple buildings start or end at the same x-coordinate.
    \index{Same Start or End}
    
    \item \textbf{Buildings with Zero Height}: Buildings that have zero height should not affect the skyline.
    \index{Buildings with Zero Height}
    
    \item \textbf{Large Number of Buildings}: Test with a large number of buildings to ensure performance and scalability.
    \index{Large Number of Buildings}
    
    \item \textbf{Buildings with Negative Coordinates}: Buildings positioned in negative coordinate space.
    \index{Negative Coordinates}
    
    \item \textbf{Boundary Values}: Buildings at the minimum and maximum limits of the coordinate range.
    \index{Boundary Values}
    
    \item \textbf{Buildings with Identical Coordinates}: Multiple buildings with the same coordinates.
    \index{Identical Coordinates}
    
    \item \textbf{Sequential Buildings}: Buildings placed sequentially without gaps.
    \index{Sequential Buildings}
    
    \item \textbf{Overlapping and Non-Overlapping Mixed}: A mix of overlapping and non-overlapping buildings.
    \index{Overlapping and Non-Overlapping Mixed}
    
    \item \textbf{Buildings with Very Large Heights}: Buildings with heights at the upper limit of the constraints.
    \index{Very Large Heights}
    
    \item \textbf{Empty Input}: No buildings are provided.
    \index{Empty Input}
\end{itemize}

\section*{Implementation Considerations}

When implementing the \texttt{getSkyline} function, keep in mind the following considerations to ensure robustness and efficiency:

\begin{itemize}
    \item \textbf{Data Type Selection}: Use appropriate data types that can handle large input values and avoid overflow or precision issues.
    \index{Data Type Selection}
    
    \item \textbf{Optimizing Event Sorting}: Efficiently sort events based on x-coordinates and heights to ensure correct processing order.
    \index{Optimizing Event Sorting}
    
    \item \textbf{Handling Large Inputs}: Design the algorithm to handle up to \(10^4\) buildings efficiently without significant performance degradation.
    \index{Handling Large Inputs}
    
    \item \textbf{Using Efficient Data Structures}: Utilize heaps, lists, and dictionaries effectively to manage and process events and active heights.
    \index{Efficient Data Structures}
    
    \item \textbf{Avoiding Redundant Calculations}: Ensure that distance and overlap calculations are performed only when necessary to optimize performance.
    \index{Avoiding Redundant Calculations}
    
    \item \textbf{Code Readability and Documentation}: Maintain clear and readable code through meaningful variable names and comprehensive comments to facilitate understanding and maintenance.
    \index{Code Readability}
    
    \item \textbf{Edge Case Handling}: Implement checks for edge cases to prevent incorrect results or runtime errors.
    \index{Edge Case Handling}
    
    \item \textbf{Implementing Helper Functions}: Create helper functions for tasks like distance calculation, event handling, and heap management to enhance modularity.
    \index{Helper Functions}
    
    \item \textbf{Consistent Naming Conventions}: Use consistent and descriptive naming conventions for variables and functions to improve code clarity.
    \index{Naming Conventions}
    
    \item \textbf{Memory Management}: Ensure that the algorithm manages memory efficiently, especially when dealing with large datasets.
    \index{Memory Management}
    
    \item \textbf{Implementing Iterative Solutions Carefully}: Ensure that loop conditions are correctly defined to prevent infinite loops or incorrect terminations.
    \index{Iterative Solutions}
    
    \item \textbf{Avoiding Floating-Point Precision Issues}: Since the problem deals with integers, floating-point precision is not a concern, simplifying the implementation.
    \index{Floating-Point Precision}
    
    \item \textbf{Testing and Validation}: Develop a comprehensive suite of test cases that cover all possible scenarios, including edge cases, to validate the correctness and efficiency of the implementation.
    \index{Testing and Validation}
    
    \item \textbf{Performance Considerations}: Optimize the loop conditions and operations to ensure that the function runs efficiently, especially for large input numbers.
    \index{Performance Considerations}
\end{itemize}

\section*{Conclusion}

The \textbf{Skyline Problem} is a quintessential example of applying advanced algorithmic techniques and geometric reasoning to solve complex spatial challenges. By leveraging the Sweep Line Algorithm and maintaining active building heights using a max heap, the solution efficiently constructs the skyline with optimal time and space complexities. Understanding and implementing such sophisticated algorithms not only enhances problem-solving skills but also provides a foundation for tackling a wide array of Computational Geometry problems in various domains, including computer graphics, urban planning simulations, and geographic information systems.

\printindex

% % filename: rectangle_overlap.tex

\problemsection{Rectangle Overlap}
\label{chap:Rectangle_Overlap}
\marginnote{\href{https://leetcode.com/problems/rectangle-overlap/}{[LeetCode Link]}\index{LeetCode}}
\marginnote{\href{https://www.geeksforgeeks.org/check-if-two-rectangles-overlap/}{[GeeksForGeeks Link]}\index{GeeksForGeeks}}
\marginnote{\href{https://www.interviewbit.com/problems/rectangle-overlap/}{[InterviewBit Link]}\index{InterviewBit}}
\marginnote{\href{https://app.codesignal.com/challenges/rectangle-overlap}{[CodeSignal Link]}\index{CodeSignal}}
\marginnote{\href{https://www.codewars.com/kata/rectangle-overlap/train/python}{[Codewars Link]}\index{Codewars}}

The \textbf{Rectangle Overlap} problem is a fundamental challenge in Computational Geometry that involves determining whether two axis-aligned rectangles overlap. This problem tests one's ability to understand geometric properties, implement conditional logic, and optimize for efficient computation. Mastery of this problem is essential for applications in computer graphics, collision detection, and spatial data analysis.

\section*{Problem Statement}

Given two axis-aligned rectangles in a 2D plane, determine if they overlap. Each rectangle is defined by its bottom-left and top-right coordinates.

A rectangle is represented as a list of four integers \([x1, y1, x2, y2]\), where \((x1, y1)\) are the coordinates of the bottom-left corner, and \((x2, y2)\) are the coordinates of the top-right corner.

\textbf{Function signature in Python:}
\begin{lstlisting}[language=Python]
def isRectangleOverlap(rec1: List[int], rec2: List[int]) -> bool:
\end{lstlisting}

\section*{Examples}

\textbf{Example 1:}

\begin{verbatim}
Input: rec1 = [0,0,2,2], rec2 = [1,1,3,3]
Output: True
Explanation: The rectangles overlap in the area defined by [1,1,2,2].
\end{verbatim}

\textbf{Example 2:}

\begin{verbatim}
Input: rec1 = [0,0,1,1], rec2 = [1,0,2,1]
Output: False
Explanation: The rectangles touch at the edge but do not overlap.
\end{verbatim}

\textbf{Example 3:}

\begin{verbatim}
Input: rec1 = [0,0,1,1], rec2 = [2,2,3,3]
Output: False
Explanation: The rectangles are completely separate.
\end{verbatim}

\textbf{Example 4:}

\begin{verbatim}
Input: rec1 = [0,0,5,5], rec2 = [3,3,7,7]
Output: True
Explanation: The rectangles overlap in the area defined by [3,3,5,5].
\end{verbatim}

\textbf{Example 5:}

\begin{verbatim}
Input: rec1 = [0,0,0,0], rec2 = [0,0,0,0]
Output: False
Explanation: Both rectangles are degenerate points.
\end{verbatim}

\textbf{Constraints:}

\begin{itemize}
    \item All coordinates are integers in the range \([-10^9, 10^9]\).
    \item For each rectangle, \(x1 < x2\) and \(y1 < y2\).
\end{itemize}

LeetCode link: \href{https://leetcode.com/problems/rectangle-overlap/}{Rectangle Overlap}\index{LeetCode}

\section*{Algorithmic Approach}

To determine whether two axis-aligned rectangles overlap, we can use the following logical conditions:

1. **Non-Overlap Conditions:**
   - One rectangle is to the left of the other.
   - One rectangle is above the other.

2. **Overlap Condition:**
   - If neither of the non-overlap conditions is true, the rectangles must overlap.

\subsection*{Steps:}

1. **Extract Coordinates:**
   - For both rectangles, extract the bottom-left and top-right coordinates.

2. **Check Non-Overlap Conditions:**
   - If the right side of the first rectangle is less than or equal to the left side of the second rectangle, they do not overlap.
   - If the left side of the first rectangle is greater than or equal to the right side of the second rectangle, they do not overlap.
   - If the top side of the first rectangle is less than or equal to the bottom side of the second rectangle, they do not overlap.
   - If the bottom side of the first rectangle is greater than or equal to the top side of the second rectangle, they do not overlap.

3. **Determine Overlap:**
   - If none of the non-overlap conditions are met, the rectangles overlap.

\marginnote{This approach provides an efficient \(O(1)\) time complexity solution by leveraging simple geometric comparisons.}

\section*{Complexities}

\begin{itemize}
    \item \textbf{Time Complexity:} \(O(1)\). The algorithm performs a constant number of comparisons regardless of input size.
    
    \item \textbf{Space Complexity:} \(O(1)\). Only a fixed amount of extra space is used for variables.
\end{itemize}

\section*{Python Implementation}

\marginnote{Implementing the overlap check using coordinate comparisons ensures an optimal and straightforward solution.}

Below is the complete Python code implementing the \texttt{isRectangleOverlap} function:

\begin{fullwidth}
\begin{lstlisting}[language=Python]
from typing import List

class Solution:
    def isRectangleOverlap(self, rec1: List[int], rec2: List[int]) -> bool:
        # Extract coordinates
        left1, bottom1, right1, top1 = rec1
        left2, bottom2, right2, top2 = rec2
        
        # Check non-overlapping conditions
        if right1 <= left2 or right2 <= left1:
            return False
        if top1 <= bottom2 or top2 <= bottom1:
            return False
        
        # If none of the above, rectangles overlap
        return True

# Example usage:
solution = Solution()
print(solution.isRectangleOverlap([0,0,2,2], [1,1,3,3]))  # Output: True
print(solution.isRectangleOverlap([0,0,1,1], [1,0,2,1]))  # Output: False
print(solution.isRectangleOverlap([0,0,1,1], [2,2,3,3]))  # Output: False
print(solution.isRectangleOverlap([0,0,5,5], [3,3,7,7]))  # Output: True
print(solution.isRectangleOverlap([0,0,0,0], [0,0,0,0]))  # Output: False
\end{lstlisting}
\end{fullwidth}

This implementation efficiently checks for overlap by comparing the coordinates of the two rectangles. If any of the non-overlapping conditions are met, it returns \texttt{False}; otherwise, it returns \texttt{True}.

\section*{Explanation}

The \texttt{isRectangleOverlap} function determines whether two axis-aligned rectangles overlap by comparing their respective coordinates. Here's a detailed breakdown of the implementation:

\subsection*{1. Extract Coordinates}

\begin{itemize}
    \item For each rectangle, extract the left (\(x1\)), bottom (\(y1\)), right (\(x2\)), and top (\(y2\)) coordinates.
    \item This simplifies the comparison process by providing clear variables representing each side of the rectangles.
\end{itemize}

\subsection*{2. Check Non-Overlap Conditions}

\begin{itemize}
    \item **Horizontal Separation:**
    \begin{itemize}
        \item If the right side of the first rectangle (\(right1\)) is less than or equal to the left side of the second rectangle (\(left2\)), there is no horizontal overlap.
        \item Similarly, if the right side of the second rectangle (\(right2\)) is less than or equal to the left side of the first rectangle (\(left1\)), there is no horizontal overlap.
    \end{itemize}
    
    \item **Vertical Separation:**
    \begin{itemize}
        \item If the top side of the first rectangle (\(top1\)) is less than or equal to the bottom side of the second rectangle (\(bottom2\)), there is no vertical overlap.
        \item Similarly, if the top side of the second rectangle (\(top2\)) is less than or equal to the bottom side of the first rectangle (\(bottom1\)), there is no vertical overlap.
    \end{itemize}
    
    \item If any of these non-overlapping conditions are true, the rectangles do not overlap, and the function returns \texttt{False}.
\end{itemize}

\subsection*{3. Determine Overlap}

\begin{itemize}
    \item If none of the non-overlapping conditions are met, it implies that the rectangles overlap both horizontally and vertically.
    \item The function returns \texttt{True} in this case.
\end{itemize}

\subsection*{4. Example Walkthrough}

Consider the first example:
\begin{verbatim}
Input: rec1 = [0,0,2,2], rec2 = [1,1,3,3]
Output: True
\end{verbatim}

\begin{enumerate}
    \item Extract coordinates:
    \begin{itemize}
        \item rec1: left1 = 0, bottom1 = 0, right1 = 2, top1 = 2
        \item rec2: left2 = 1, bottom2 = 1, right2 = 3, top2 = 3
    \end{itemize}
    
    \item Check non-overlap conditions:
    \begin{itemize}
        \item \(right1 = 2\) is not less than or equal to \(left2 = 1\)
        \item \(right2 = 3\) is not less than or equal to \(left1 = 0\)
        \item \(top1 = 2\) is not less than or equal to \(bottom2 = 1\)
        \item \(top2 = 3\) is not less than or equal to \(bottom1 = 0\)
    \end{itemize}
    
    \item Since none of the non-overlapping conditions are met, the rectangles overlap.
\end{enumerate}

Thus, the function correctly returns \texttt{True}.

\section*{Why This Approach}

This approach is chosen for its simplicity and efficiency. By leveraging direct coordinate comparisons, the algorithm achieves constant time complexity without the need for complex data structures or iterative processes. It effectively handles all possible scenarios of rectangle positioning, ensuring accurate detection of overlaps.

\section*{Alternative Approaches}

\subsection*{1. Separating Axis Theorem (SAT)}

The Separating Axis Theorem is a more generalized method for detecting overlaps between convex shapes. While it is not necessary for axis-aligned rectangles, understanding SAT can be beneficial for more complex geometric problems.

\begin{lstlisting}[language=Python]
def isRectangleOverlap(rec1: List[int], rec2: List[int]) -> bool:
    # Using SAT for axis-aligned rectangles
    return not (rec1[2] <= rec2[0] or rec1[0] >= rec2[2] or
                rec1[3] <= rec2[1] or rec1[1] >= rec2[3])
\end{lstlisting}

\textbf{Note}: This implementation is functionally identical to the primary approach but leverages a more generalized geometric theorem.

\subsection*{2. Area-Based Approach}

Calculate the overlapping area between the two rectangles. If the overlapping area is positive, the rectangles overlap.

\begin{lstlisting}[language=Python]
def isRectangleOverlap(rec1: List[int], rec2: List[int]) -> bool:
    # Calculate overlap in x and y dimensions
    x_overlap = min(rec1[2], rec2[2]) - max(rec1[0], rec2[0])
    y_overlap = min(rec1[3], rec2[3]) - max(rec1[1], rec2[1])
    
    # Overlap exists if both overlaps are positive
    return x_overlap > 0 and y_overlap > 0
\end{lstlisting}

\textbf{Complexities:}
\begin{itemize}
    \item \textbf{Time Complexity:} \(O(1)\)
    \item \textbf{Space Complexity:} \(O(1)\)
\end{itemize}

\subsection*{3. Using Rectangles Intersection Function}

Utilize built-in or library functions that handle geometric intersections.

\begin{lstlisting}[language=Python]
from shapely.geometry import box

def isRectangleOverlap(rec1: List[int], rec2: List[int]) -> bool:
    rectangle1 = box(rec1[0], rec1[1], rec1[2], rec1[3])
    rectangle2 = box(rec2[0], rec2[1], rec2[2], rec2[3])
    return rectangle1.intersects(rectangle2) and not rectangle1.touches(rectangle2)
\end{lstlisting}

\textbf{Note}: This approach requires the \texttt{shapely} library and is more suitable for complex geometric operations.

\section*{Similar Problems to This One}

Several problems revolve around geometric overlap, intersection detection, and spatial reasoning, utilizing similar algorithmic strategies:

\begin{itemize}
    \item \textbf{Interval Overlap}: Determine if two intervals on a line overlap.
    \item \textbf{Circle Overlap}: Determine if two circles overlap based on their radii and centers.
    \item \textbf{Polygon Overlap}: Determine if two polygons overlap using algorithms like SAT.
    \item \textbf{Closest Pair of Points}: Find the closest pair of points in a set.
    \item \textbf{Convex Hull}: Compute the convex hull of a set of points.
    \item \textbf{Intersection of Lines}: Find the intersection point of two lines.
    \item \textbf{Point Inside Polygon}: Determine if a point lies inside a given polygon.
\end{itemize}

These problems reinforce the concepts of spatial reasoning, geometric property analysis, and efficient algorithm design in various contexts.

\section*{Things to Keep in Mind and Tricks}

When working with the \textbf{Rectangle Overlap} problem, consider the following tips and best practices to enhance efficiency and correctness:

\begin{itemize}
    \item \textbf{Understand Geometric Relationships}: Grasp the positional relationships between rectangles to simplify overlap detection.
    \index{Geometric Relationships}
    
    \item \textbf{Leverage Coordinate Comparisons}: Use direct comparisons of rectangle coordinates to determine spatial relationships.
    \index{Coordinate Comparisons}
    
    \item \textbf{Handle Edge Cases}: Consider cases where rectangles touch at edges or corners without overlapping.
    \index{Edge Cases}
    
    \item \textbf{Optimize for Efficiency}: Aim for a constant time \(O(1)\) solution by avoiding unnecessary computations or iterations.
    \index{Efficiency Optimization}
    
    \item \textbf{Avoid Floating-Point Precision Issues}: Since all coordinates are integers, floating-point precision is not a concern, simplifying the implementation.
    \index{Floating-Point Precision}
    
    \item \textbf{Use Helper Functions}: Create helper functions to encapsulate repetitive tasks, such as extracting coordinates or checking specific conditions.
    \index{Helper Functions}
    
    \item \textbf{Code Readability}: Maintain clear and readable code through meaningful variable names and structured logic.
    \index{Code Readability}
    
    \item \textbf{Test Extensively}: Implement a wide range of test cases, including overlapping, non-overlapping, and edge-touching rectangles, to ensure robustness.
    \index{Extensive Testing}
    
    \item \textbf{Understand Axis-Aligned Constraints}: Recognize that axis-aligned rectangles simplify overlap detection compared to rotated rectangles.
    \index{Axis-Aligned Constraints}
    
    \item \textbf{Simplify Logical Conditions}: Combine multiple conditions logically to streamline the overlap detection process.
    \index{Logical Conditions}
\end{itemize}

\section*{Corner and Special Cases to Test When Writing the Code}

When implementing the solution for the \textbf{Rectangle Overlap} problem, it is crucial to consider and rigorously test various edge cases to ensure robustness and correctness:

\begin{itemize}
    \item \textbf{No Overlap}: Rectangles are completely separate.
    \index{No Overlap}
    
    \item \textbf{Partial Overlap}: Rectangles overlap in one or more regions.
    \index{Partial Overlap}
    
    \item \textbf{Edge Touching}: Rectangles touch exactly at one edge without overlapping.
    \index{Edge Touching}
    
    \item \textbf{Corner Touching}: Rectangles touch exactly at one corner without overlapping.
    \index{Corner Touching}
    
    \item \textbf{One Rectangle Inside Another}: One rectangle is entirely within the other.
    \index{Rectangle Inside}
    
    \item \textbf{Identical Rectangles}: Both rectangles have the same coordinates.
    \index{Identical Rectangles}
    
    \item \textbf{Degenerate Rectangles}: Rectangles with zero area (e.g., \(x1 = x2\) or \(y1 = y2\)).
    \index{Degenerate Rectangles}
    
    \item \textbf{Large Coordinates}: Rectangles with very large coordinate values to test performance and integer handling.
    \index{Large Coordinates}
    
    \item \textbf{Negative Coordinates}: Rectangles positioned in negative coordinate space.
    \index{Negative Coordinates}
    
    \item \textbf{Mixed Overlapping Scenarios}: Combinations of the above cases to ensure comprehensive coverage.
    \index{Mixed Overlapping Scenarios}
    
    \item \textbf{Minimum and Maximum Bounds}: Rectangles at the minimum and maximum limits of the coordinate range.
    \index{Minimum and Maximum Bounds}
\end{itemize}

\section*{Implementation Considerations}

When implementing the \texttt{isRectangleOverlap} function, keep in mind the following considerations to ensure robustness and efficiency:

\begin{itemize}
    \item \textbf{Data Type Selection}: Use appropriate data types that can handle the range of input values without overflow or underflow.
    \index{Data Type Selection}
    
    \item \textbf{Optimizing Comparisons}: Structure logical conditions to short-circuit evaluations as soon as a non-overlapping condition is met.
    \index{Optimizing Comparisons}
    
    \item \textbf{Language-Specific Constraints}: Be aware of how the programming language handles integer division and comparisons.
    \index{Language-Specific Constraints}
    
    \item \textbf{Avoiding Redundant Calculations}: Ensure that each comparison contributes towards determining overlap without unnecessary repetitions.
    \index{Avoiding Redundant Calculations}
    
    \item \textbf{Code Readability and Documentation}: Maintain clear and readable code through meaningful variable names and comprehensive comments to facilitate understanding and maintenance.
    \index{Code Readability}
    
    \item \textbf{Edge Case Handling}: Implement checks for edge cases to prevent incorrect results or runtime errors.
    \index{Edge Case Handling}
    
    \item \textbf{Testing and Validation}: Develop a comprehensive suite of test cases that cover all possible scenarios, including edge cases, to validate the correctness and efficiency of the implementation.
    \index{Testing and Validation}
    
    \item \textbf{Scalability}: Design the algorithm to scale efficiently with increasing input sizes, maintaining performance and resource utilization.
    \index{Scalability}
    
    \item \textbf{Using Helper Functions}: Consider creating helper functions for repetitive tasks, such as extracting and comparing coordinates, to enhance modularity and reusability.
    \index{Helper Functions}
    
    \item \textbf{Consistent Naming Conventions}: Use consistent and descriptive naming conventions for variables to improve code clarity.
    \index{Naming Conventions}
    
    \item \textbf{Handling Floating-Point Coordinates}: Although the problem specifies integer coordinates, ensure that the implementation can handle floating-point numbers if needed in extended scenarios.
    \index{Floating-Point Coordinates}
    
    \item \textbf{Avoiding Floating-Point Precision Issues}: Since all coordinates are integers, floating-point precision is not a concern, simplifying the implementation.
    \index{Floating-Point Precision}
    
    \item \textbf{Implementing Unit Tests}: Develop unit tests for each logical condition to ensure that all scenarios are correctly handled.
    \index{Unit Tests}
    
    \item \textbf{Error Handling}: Incorporate error handling to manage invalid inputs gracefully.
    \index{Error Handling}
\end{itemize}

\section*{Conclusion}

The \textbf{Rectangle Overlap} problem exemplifies the application of fundamental geometric principles and conditional logic to solve spatial challenges efficiently. By leveraging simple coordinate comparisons, the algorithm achieves optimal time and space complexities, making it highly suitable for real-time applications such as collision detection in gaming, layout planning in graphics, and spatial data analysis. Understanding and implementing such techniques not only enhances problem-solving skills but also provides a foundation for tackling more complex Computational Geometry problems involving varied geometric shapes and interactions.

\printindex

% \input{sections/rectangle_overlap}
% \input{sections/rectangle_area}
% \input{sections/k_closest_points_to_origin}
% \input{sections/the_skyline_problem}
% % filename: rectangle_area.tex

\problemsection{Rectangle Area}
\label{chap:Rectangle_Area}
\marginnote{\href{https://leetcode.com/problems/rectangle-area/}{[LeetCode Link]}\index{LeetCode}}
\marginnote{\href{https://www.geeksforgeeks.org/find-area-two-overlapping-rectangles/}{[GeeksForGeeks Link]}\index{GeeksForGeeks}}
\marginnote{\href{https://www.interviewbit.com/problems/rectangle-area/}{[InterviewBit Link]}\index{InterviewBit}}
\marginnote{\href{https://app.codesignal.com/challenges/rectangle-area}{[CodeSignal Link]}\index{CodeSignal}}
\marginnote{\href{https://www.codewars.com/kata/rectangle-area/train/python}{[Codewars Link]}\index{Codewars}}

The \textbf{Rectangle Area} problem is a classic Computational Geometry challenge that involves calculating the total area covered by two axis-aligned rectangles in a 2D plane. This problem tests one's ability to perform geometric calculations, handle overlapping scenarios, and implement efficient algorithms. Mastery of this problem is essential for applications in computer graphics, spatial analysis, and computational modeling.

\section*{Problem Statement}

Given two axis-aligned rectangles in a 2D plane, compute the total area covered by the two rectangles. The area covered by the overlapping region should be counted only once.

Each rectangle is represented as a list of four integers \([x1, y1, x2, y2]\), where \((x1, y1)\) are the coordinates of the bottom-left corner, and \((x2, y2)\) are the coordinates of the top-right corner.

\textbf{Function signature in Python:}
\begin{lstlisting}[language=Python]
def computeArea(A: List[int], B: List[int]) -> int:
\end{lstlisting}

\section*{Examples}

\textbf{Example 1:}

\begin{verbatim}
Input: A = [-3,0,3,4], B = [0,-1,9,2]
Output: 45
Explanation:
Area of A = (3 - (-3)) * (4 - 0) = 6 * 4 = 24
Area of B = (9 - 0) * (2 - (-1)) = 9 * 3 = 27
Overlapping Area = (3 - 0) * (2 - 0) = 3 * 2 = 6
Total Area = 24 + 27 - 6 = 45
\end{verbatim}

\textbf{Example 2:}

\begin{verbatim}
Input: A = [0,0,0,0], B = [0,0,0,0]
Output: 0
Explanation:
Both rectangles are degenerate points with zero area.
\end{verbatim}

\textbf{Example 3:}

\begin{verbatim}
Input: A = [0,0,2,2], B = [1,1,3,3]
Output: 7
Explanation:
Area of A = 4
Area of B = 4
Overlapping Area = 1
Total Area = 4 + 4 - 1 = 7
\end{verbatim}

\textbf{Example 4:}

\begin{verbatim}
Input: A = [0,0,1,1], B = [1,0,2,1]
Output: 2
Explanation:
Rectangles touch at the edge but do not overlap.
Area of A = 1
Area of B = 1
Overlapping Area = 0
Total Area = 1 + 1 = 2
\end{verbatim}

\textbf{Constraints:}

\begin{itemize}
    \item All coordinates are integers in the range \([-10^9, 10^9]\).
    \item For each rectangle, \(x1 < x2\) and \(y1 < y2\).
\end{itemize}

LeetCode link: \href{https://leetcode.com/problems/rectangle-area/}{Rectangle Area}\index{LeetCode}

\section*{Algorithmic Approach}

To compute the total area covered by two axis-aligned rectangles, we can follow these steps:

1. **Calculate Individual Areas:**
   - Compute the area of the first rectangle.
   - Compute the area of the second rectangle.

2. **Determine Overlapping Area:**
   - Calculate the coordinates of the overlapping rectangle, if any.
   - If the rectangles overlap, compute the area of the overlapping region.

3. **Compute Total Area:**
   - Sum the individual areas and subtract the overlapping area to avoid double-counting.

\marginnote{This approach ensures accurate area calculation by handling overlapping regions appropriately.}

\section*{Complexities}

\begin{itemize}
    \item \textbf{Time Complexity:} \(O(1)\). The algorithm performs a constant number of calculations.
    
    \item \textbf{Space Complexity:} \(O(1)\). Only a fixed amount of extra space is used for variables.
\end{itemize}

\section*{Python Implementation}

\marginnote{Implementing the area calculation with overlap consideration ensures an accurate and efficient solution.}

Below is the complete Python code implementing the \texttt{computeArea} function:

\begin{fullwidth}
\begin{lstlisting}[language=Python]
from typing import List

class Solution:
    def computeArea(self, A: List[int], B: List[int]) -> int:
        # Calculate area of rectangle A
        areaA = (A[2] - A[0]) * (A[3] - A[1])
        
        # Calculate area of rectangle B
        areaB = (B[2] - B[0]) * (B[3] - B[1])
        
        # Determine overlap coordinates
        overlap_x1 = max(A[0], B[0])
        overlap_y1 = max(A[1], B[1])
        overlap_x2 = min(A[2], B[2])
        overlap_y2 = min(A[3], B[3])
        
        # Calculate overlapping area
        overlap_width = overlap_x2 - overlap_x1
        overlap_height = overlap_y2 - overlap_y1
        overlap_area = 0
        if overlap_width > 0 and overlap_height > 0:
            overlap_area = overlap_width * overlap_height
        
        # Total area is sum of individual areas minus overlapping area
        total_area = areaA + areaB - overlap_area
        return total_area

# Example usage:
solution = Solution()
print(solution.computeArea([-3,0,3,4], [0,-1,9,2]))  # Output: 45
print(solution.computeArea([0,0,0,0], [0,0,0,0]))    # Output: 0
print(solution.computeArea([0,0,2,2], [1,1,3,3]))    # Output: 7
print(solution.computeArea([0,0,1,1], [1,0,2,1]))    # Output: 2
\end{lstlisting}
\end{fullwidth}

This implementation accurately computes the total area covered by two rectangles by accounting for any overlapping regions. It ensures that the overlapping area is not double-counted.

\section*{Explanation}

The \texttt{computeArea} function calculates the combined area of two axis-aligned rectangles by following these steps:

\subsection*{1. Calculate Individual Areas}

\begin{itemize}
    \item **Rectangle A:**
    \begin{itemize}
        \item Width: \(A[2] - A[0]\)
        \item Height: \(A[3] - A[1]\)
        \item Area: Width \(\times\) Height
    \end{itemize}
    
    \item **Rectangle B:**
    \begin{itemize}
        \item Width: \(B[2] - B[0]\)
        \item Height: \(B[3] - B[1]\)
        \item Area: Width \(\times\) Height
    \end{itemize}
\end{itemize}

\subsection*{2. Determine Overlapping Area}

\begin{itemize}
    \item **Overlap Coordinates:**
    \begin{itemize}
        \item Left (x-coordinate): \(\text{max}(A[0], B[0])\)
        \item Bottom (y-coordinate): \(\text{max}(A[1], B[1])\)
        \item Right (x-coordinate): \(\text{min}(A[2], B[2])\)
        \item Top (y-coordinate): \(\text{min}(A[3], B[3])\)
    \end{itemize}
    
    \item **Overlap Dimensions:**
    \begin{itemize}
        \item Width: \(\text{overlap\_x2} - \text{overlap\_x1}\)
        \item Height: \(\text{overlap\_y2} - \text{overlap\_y1}\)
    \end{itemize}
    
    \item **Overlap Area:**
    \begin{itemize}
        \item If both width and height are positive, the rectangles overlap, and the overlapping area is their product.
        \item Otherwise, there is no overlap, and the overlapping area is zero.
    \end{itemize}
\end{itemize}

\subsection*{3. Compute Total Area}

\begin{itemize}
    \item Total Area = Area of Rectangle A + Area of Rectangle B - Overlapping Area
\end{itemize}

\subsection*{4. Example Walkthrough}

Consider the first example:
\begin{verbatim}
Input: A = [-3,0,3,4], B = [0,-1,9,2]
Output: 45
\end{verbatim}

\begin{enumerate}
    \item **Calculate Areas:**
    \begin{itemize}
        \item Area of A = (3 - (-3)) * (4 - 0) = 6 * 4 = 24
        \item Area of B = (9 - 0) * (2 - (-1)) = 9 * 3 = 27
    \end{itemize}
    
    \item **Determine Overlap:**
    \begin{itemize}
        \item overlap\_x1 = max(-3, 0) = 0
        \item overlap\_y1 = max(0, -1) = 0
        \item overlap\_x2 = min(3, 9) = 3
        \item overlap\_y2 = min(4, 2) = 2
        \item overlap\_width = 3 - 0 = 3
        \item overlap\_height = 2 - 0 = 2
        \item overlap\_area = 3 * 2 = 6
    \end{itemize}
    
    \item **Compute Total Area:**
    \begin{itemize}
        \item Total Area = 24 + 27 - 6 = 45
    \end{itemize}
\end{enumerate}

Thus, the function correctly returns \texttt{45}.

\section*{Why This Approach}

This approach is chosen for its straightforwardness and optimal efficiency. By directly calculating the individual areas and intelligently handling the overlapping region, the algorithm ensures accurate results without unnecessary computations. Its constant time complexity makes it highly efficient, even for large coordinate values.

\section*{Alternative Approaches}

\subsection*{1. Using Intersection Dimensions}

Instead of separately calculating areas, directly compute the dimensions of the overlapping region and subtract it from the sum of individual areas.

\begin{lstlisting}[language=Python]
def computeArea(A: List[int], B: List[int]) -> int:
    # Sum of individual areas
    area = (A[2] - A[0]) * (A[3] - A[1]) + (B[2] - B[0]) * (B[3] - B[1])
    
    # Overlapping area
    overlap_width = min(A[2], B[2]) - max(A[0], B[0])
    overlap_height = min(A[3], B[3]) - max(A[1], B[1])
    
    if overlap_width > 0 and overlap_height > 0:
        area -= overlap_width * overlap_height
    
    return area
\end{lstlisting}

\subsection*{2. Using Geometry Libraries}

Leverage computational geometry libraries to handle area calculations and overlapping detections.

\begin{lstlisting}[language=Python]
from shapely.geometry import box

def computeArea(A: List[int], B: List[int]) -> int:
    rect1 = box(A[0], A[1], A[2], A[3])
    rect2 = box(B[0], B[1], B[2], B[3])
    intersection = rect1.intersection(rect2)
    return int(rect1.area + rect2.area - intersection.area)
\end{lstlisting}

\textbf{Note}: This approach requires the \texttt{shapely} library and is more suitable for complex geometric operations.

\section*{Similar Problems to This One}

Several problems involve calculating areas, handling geometric overlaps, and spatial reasoning, utilizing similar algorithmic strategies:

\begin{itemize}
    \item \textbf{Rectangle Overlap}: Determine if two rectangles overlap.
    \item \textbf{Circle Area Overlap}: Calculate the overlapping area between two circles.
    \item \textbf{Polygon Area}: Compute the area of a given polygon.
    \item \textbf{Union of Rectangles}: Calculate the total area covered by multiple rectangles, accounting for overlaps.
    \item \textbf{Intersection of Lines}: Find the intersection point of two lines.
    \item \textbf{Closest Pair of Points}: Find the closest pair of points in a set.
    \item \textbf{Convex Hull}: Compute the convex hull of a set of points.
    \item \textbf{Point Inside Polygon}: Determine if a point lies inside a given polygon.
\end{itemize}

These problems reinforce concepts of geometric calculations, area computations, and efficient algorithm design in various contexts.

\section*{Things to Keep in Mind and Tricks}

When tackling the \textbf{Rectangle Area} problem, consider the following tips and best practices to enhance efficiency and correctness:

\begin{itemize}
    \item \textbf{Understand Geometric Relationships}: Grasp the positional relationships between rectangles to simplify area calculations.
    \index{Geometric Relationships}
    
    \item \textbf{Leverage Coordinate Comparisons}: Use direct comparisons of rectangle coordinates to determine overlapping regions.
    \index{Coordinate Comparisons}
    
    \item \textbf{Handle Overlapping Scenarios}: Accurately calculate the overlapping area to avoid double-counting.
    \index{Overlapping Scenarios}
    
    \item \textbf{Optimize for Efficiency}: Aim for a constant time \(O(1)\) solution by avoiding unnecessary computations or iterations.
    \index{Efficiency Optimization}
    
    \item \textbf{Avoid Floating-Point Precision Issues}: Since all coordinates are integers, floating-point precision is not a concern, simplifying the implementation.
    \index{Floating-Point Precision}
    
    \item \textbf{Use Helper Functions}: Create helper functions to encapsulate repetitive tasks, such as calculating overlap dimensions or areas.
    \index{Helper Functions}
    
    \item \textbf{Code Readability}: Maintain clear and readable code through meaningful variable names and structured logic.
    \index{Code Readability}
    
    \item \textbf{Test Extensively}: Implement a wide range of test cases, including overlapping, non-overlapping, and edge-touching rectangles, to ensure robustness.
    \index{Extensive Testing}
    
    \item \textbf{Understand Axis-Aligned Constraints}: Recognize that axis-aligned rectangles simplify area calculations compared to rotated rectangles.
    \index{Axis-Aligned Constraints}
    
    \item \textbf{Simplify Logical Conditions}: Combine multiple conditions logically to streamline the area calculation process.
    \index{Logical Conditions}
    
    \item \textbf{Use Absolute Values}: When calculating differences, ensure that the dimensions are positive by using absolute values or proper ordering.
    \index{Absolute Values}
    
    \item \textbf{Consider Edge Cases}: Handle cases where rectangles have zero area or touch at edges/corners without overlapping.
    \index{Edge Cases}
\end{itemize}

\section*{Corner and Special Cases to Test When Writing the Code}

When implementing the solution for the \textbf{Rectangle Area} problem, it is crucial to consider and rigorously test various edge cases to ensure robustness and correctness:

\begin{itemize}
    \item \textbf{No Overlap}: Rectangles are completely separate.
    \index{No Overlap}
    
    \item \textbf{Partial Overlap}: Rectangles overlap in one or more regions.
    \index{Partial Overlap}
    
    \item \textbf{Edge Touching}: Rectangles touch exactly at one edge without overlapping.
    \index{Edge Touching}
    
    \item \textbf{Corner Touching}: Rectangles touch exactly at one corner without overlapping.
    \index{Corner Touching}
    
    \item \textbf{One Rectangle Inside Another}: One rectangle is entirely within the other.
    \index{Rectangle Inside}
    
    \item \textbf{Identical Rectangles}: Both rectangles have the same coordinates.
    \index{Identical Rectangles}
    
    \item \textbf{Degenerate Rectangles}: Rectangles with zero area (e.g., \(x1 = x2\) or \(y1 = y2\)).
    \index{Degenerate Rectangles}
    
    \item \textbf{Large Coordinates}: Rectangles with very large coordinate values to test performance and integer handling.
    \index{Large Coordinates}
    
    \item \textbf{Negative Coordinates}: Rectangles positioned in negative coordinate space.
    \index{Negative Coordinates}
    
    \item \textbf{Mixed Overlapping Scenarios}: Combinations of the above cases to ensure comprehensive coverage.
    \index{Mixed Overlapping Scenarios}
    
    \item \textbf{Minimum and Maximum Bounds}: Rectangles at the minimum and maximum limits of the coordinate range.
    \index{Minimum and Maximum Bounds}
    
    \item \textbf{Sequential Rectangles}: Multiple rectangles placed sequentially without overlapping.
    \index{Sequential Rectangles}
    
    \item \textbf{Multiple Overlaps}: Scenarios where more than two rectangles overlap in different regions.
    \index{Multiple Overlaps}
\end{itemize}

\section*{Implementation Considerations}

When implementing the \texttt{computeArea} function, keep in mind the following considerations to ensure robustness and efficiency:

\begin{itemize}
    \item \textbf{Data Type Selection}: Use appropriate data types that can handle large input values without overflow or underflow.
    \index{Data Type Selection}
    
    \item \textbf{Optimizing Comparisons}: Structure logical conditions to efficiently determine overlap dimensions.
    \index{Optimizing Comparisons}
    
    \item \textbf{Handling Large Inputs}: Design the algorithm to efficiently handle large input sizes without significant performance degradation.
    \index{Handling Large Inputs}
    
    \item \textbf{Language-Specific Constraints}: Be aware of how the programming language handles large integers and arithmetic operations.
    \index{Language-Specific Constraints}
    
    \item \textbf{Avoiding Redundant Calculations}: Ensure that each calculation contributes towards determining the final area without unnecessary repetitions.
    \index{Avoiding Redundant Calculations}
    
    \item \textbf{Code Readability and Documentation}: Maintain clear and readable code through meaningful variable names and comprehensive comments to facilitate understanding and maintenance.
    \index{Code Readability}
    
    \item \textbf{Edge Case Handling}: Implement checks for edge cases to prevent incorrect results or runtime errors.
    \index{Edge Case Handling}
    
    \item \textbf{Testing and Validation}: Develop a comprehensive suite of test cases that cover all possible scenarios, including edge cases, to validate the correctness and efficiency of the implementation.
    \index{Testing and Validation}
    
    \item \textbf{Scalability}: Design the algorithm to scale efficiently with increasing input sizes, maintaining performance and resource utilization.
    \index{Scalability}
    
    \item \textbf{Using Helper Functions}: Consider creating helper functions for repetitive tasks, such as calculating overlap dimensions, to enhance modularity and reusability.
    \index{Helper Functions}
    
    \item \textbf{Consistent Naming Conventions}: Use consistent and descriptive naming conventions for variables to improve code clarity.
    \index{Naming Conventions}
    
    \item \textbf{Implementing Unit Tests}: Develop unit tests for each logical condition to ensure that all scenarios are correctly handled.
    \index{Unit Tests}
    
    \item \textbf{Error Handling}: Incorporate error handling to manage invalid inputs gracefully.
    \index{Error Handling}
\end{itemize}

\section*{Conclusion}

The \textbf{Rectangle Area} problem showcases the application of fundamental geometric principles and efficient algorithm design to compute spatial properties accurately. By systematically calculating individual areas and intelligently handling overlapping regions, the algorithm ensures precise results without redundant computations. Understanding and implementing such techniques not only enhances problem-solving skills but also provides a foundation for tackling more complex Computational Geometry challenges involving multiple geometric entities and intricate spatial relationships.

\printindex

% \input{sections/rectangle_overlap}
% \input{sections/rectangle_area}
% \input{sections/k_closest_points_to_origin}
% \input{sections/the_skyline_problem}
% % filename: k_closest_points_to_origin.tex

\problemsection{K Closest Points to Origin}
\label{chap:K_Closest_Points_to_Origin}
\marginnote{\href{https://leetcode.com/problems/k-closest-points-to-origin/}{[LeetCode Link]}\index{LeetCode}}
\marginnote{\href{https://www.geeksforgeeks.org/find-k-closest-points-origin/}{[GeeksForGeeks Link]}\index{GeeksForGeeks}}
\marginnote{\href{https://www.interviewbit.com/problems/k-closest-points/}{[InterviewBit Link]}\index{InterviewBit}}
\marginnote{\href{https://app.codesignal.com/challenges/k-closest-points-to-origin}{[CodeSignal Link]}\index{CodeSignal}}
\marginnote{\href{https://www.codewars.com/kata/k-closest-points-to-origin/train/python}{[Codewars Link]}\index{Codewars}}

The \textbf{K Closest Points to Origin} problem is a popular algorithmic challenge in Computational Geometry that involves identifying the \(k\) points closest to the origin in a 2D plane. This problem tests one's ability to apply efficient sorting and selection algorithms, understand distance computations, and optimize for performance. Mastery of this problem is essential for applications in spatial data analysis, nearest neighbor searches, and clustering algorithms.

\section*{Problem Statement}

Given an array of points where each point is represented as \([x, y]\) in the 2D plane, and an integer \(k\), return the \(k\) closest points to the origin \((0, 0)\).

The distance between two points \((x_1, y_1)\) and \((x_2, y_2)\) is the Euclidean distance \(\sqrt{(x_1 - x_2)^2 + (y_1 - y_2)^2}\). The origin is \((0, 0)\).

\textbf{Function signature in Python:}
\begin{lstlisting}[language=Python]
def kClosest(points: List[List[int]], K: int) -> List[List[int]]:
\end{lstlisting}

\section*{Examples}

\textbf{Example 1:}

\begin{verbatim}
Input: points = [[1,3],[-2,2]], K = 1
Output: [[-2,2]]
Explanation: 
The distance between (1, 3) and the origin is sqrt(10).
The distance between (-2, 2) and the origin is sqrt(8).
Since sqrt(8) < sqrt(10), (-2, 2) is closer to the origin.
\end{verbatim}

\textbf{Example 2:}

\begin{verbatim}
Input: points = [[3,3],[5,-1],[-2,4]], K = 2
Output: [[3,3],[-2,4]]
Explanation: 
The distances are sqrt(18), sqrt(26), and sqrt(20) respectively.
The two closest points are [3,3] and [-2,4].
\end{verbatim}

\textbf{Example 3:}

\begin{verbatim}
Input: points = [[0,1],[1,0]], K = 2
Output: [[0,1],[1,0]]
Explanation: 
Both points are equally close to the origin.
\end{verbatim}

\textbf{Example 4:}

\begin{verbatim}
Input: points = [[1,0],[0,1]], K = 1
Output: [[1,0]]
Explanation: 
Both points are equally close; returning any one is acceptable.
\end{verbatim}

\textbf{Constraints:}

\begin{itemize}
    \item \(1 \leq K \leq \text{points.length} \leq 10^4\)
    \item \(-10^4 < x_i, y_i < 10^4\)
\end{itemize}

LeetCode link: \href{https://leetcode.com/problems/k-closest-points-to-origin/}{K Closest Points to Origin}\index{LeetCode}

\section*{Algorithmic Approach}

To identify the \(k\) closest points to the origin, several algorithmic strategies can be employed. The most efficient methods aim to reduce the time complexity by avoiding the need to sort the entire list of points.

\subsection*{1. Sorting Based on Distance}

Calculate the Euclidean distance of each point from the origin and sort the points based on these distances. Select the first \(k\) points from the sorted list.

\begin{enumerate}
    \item Compute the distance for each point using the formula \(distance = x^2 + y^2\).
    \item Sort the points based on the computed distances.
    \item Return the first \(k\) points from the sorted list.
\end{enumerate}

\subsection*{2. Max Heap (Priority Queue)}

Use a max heap to maintain the \(k\) closest points. Iterate through each point, add it to the heap, and if the heap size exceeds \(k\), remove the farthest point.

\begin{enumerate}
    \item Initialize a max heap.
    \item For each point, compute its distance and add it to the heap.
    \item If the heap size exceeds \(k\), remove the point with the largest distance.
    \item After processing all points, the heap contains the \(k\) closest points.
\end{enumerate}

\subsection*{3. QuickSelect (Quick Sort Partitioning)}

Utilize the QuickSelect algorithm to find the \(k\) closest points without fully sorting the list.

\begin{enumerate}
    \item Choose a pivot point and partition the list based on distances relative to the pivot.
    \item Recursively apply QuickSelect to the partition containing the \(k\) closest points.
    \item Once the \(k\) closest points are identified, return them.
\end{enumerate}

\marginnote{QuickSelect offers an average time complexity of \(O(n)\), making it highly efficient for large datasets.}

\section*{Complexities}

\begin{itemize}
    \item \textbf{Sorting Based on Distance:}
    \begin{itemize}
        \item \textbf{Time Complexity:} \(O(n \log n)\)
        \item \textbf{Space Complexity:} \(O(n)\)
    \end{itemize}
    
    \item \textbf{Max Heap (Priority Queue):}
    \begin{itemize}
        \item \textbf{Time Complexity:} \(O(n \log k)\)
        \item \textbf{Space Complexity:} \(O(k)\)
    \end{itemize}
    
    \item \textbf{QuickSelect (Quick Sort Partitioning):}
    \begin{itemize}
        \item \textbf{Time Complexity:} Average case \(O(n)\), worst case \(O(n^2)\)
        \item \textbf{Space Complexity:} \(O(1)\) (in-place)
    \end{itemize}
\end{itemize}

\section*{Python Implementation}

\marginnote{Implementing QuickSelect provides an optimal average-case solution with linear time complexity.}

Below is the complete Python code implementing the \texttt{kClosest} function using the QuickSelect approach:

\begin{fullwidth}
\begin{lstlisting}[language=Python]
from typing import List
import random

class Solution:
    def kClosest(self, points: List[List[int]], K: int) -> List[List[int]]:
        def quickselect(left, right, K_smallest):
            if left == right:
                return
            
            # Select a random pivot_index
            pivot_index = random.randint(left, right)
            
            # Partition the array
            pivot_index = partition(left, right, pivot_index)
            
            # The pivot is in its final sorted position
            if K_smallest == pivot_index:
                return
            elif K_smallest < pivot_index:
                quickselect(left, pivot_index - 1, K_smallest)
            else:
                quickselect(pivot_index + 1, right, K_smallest)
        
        def partition(left, right, pivot_index):
            pivot_distance = distance(points[pivot_index])
            # Move pivot to end
            points[pivot_index], points[right] = points[right], points[pivot_index]
            store_index = left
            for i in range(left, right):
                if distance(points[i]) < pivot_distance:
                    points[store_index], points[i] = points[i], points[store_index]
                    store_index += 1
            # Move pivot to its final place
            points[right], points[store_index] = points[store_index], points[right]
            return store_index
        
        def distance(point):
            return point[0] ** 2 + point[1] ** 2
        
        n = len(points)
        quickselect(0, n - 1, K)
        return points[:K]

# Example usage:
solution = Solution()
print(solution.kClosest([[1,3],[-2,2]], 1))            # Output: [[-2,2]]
print(solution.kClosest([[3,3],[5,-1],[-2,4]], 2))     # Output: [[3,3],[-2,4]]
print(solution.kClosest([[0,1],[1,0]], 2))             # Output: [[0,1],[1,0]]
print(solution.kClosest([[1,0],[0,1]], 1))             # Output: [[1,0]] or [[0,1]]
\end{lstlisting}
\end{fullwidth}

This implementation uses the QuickSelect algorithm to efficiently find the \(k\) closest points to the origin without fully sorting the entire list. It ensures optimal performance even with large datasets.

\section*{Explanation}

The \texttt{kClosest} function identifies the \(k\) closest points to the origin using the QuickSelect algorithm. Here's a detailed breakdown of the implementation:

\subsection*{1. Distance Calculation}

\begin{itemize}
    \item The Euclidean distance is calculated as \(distance = x^2 + y^2\). Since we only need relative distances for comparison, the square root is omitted for efficiency.
\end{itemize}

\subsection*{2. QuickSelect Algorithm}

\begin{itemize}
    \item **Pivot Selection:**
    \begin{itemize}
        \item A random pivot is chosen to enhance the average-case performance.
    \end{itemize}
    
    \item **Partitioning:**
    \begin{itemize}
        \item The array is partitioned such that points with distances less than the pivot are moved to the left, and others to the right.
        \item The pivot is placed in its correct sorted position.
    \end{itemize}
    
    \item **Recursive Selection:**
    \begin{itemize}
        \item If the pivot's position matches \(K\), the selection is complete.
        \item Otherwise, recursively apply QuickSelect to the relevant partition.
    \end{itemize}
\end{itemize}

\subsection*{3. Final Selection}

\begin{itemize}
    \item After partitioning, the first \(K\) points in the list are the \(k\) closest points to the origin.
\end{itemize}

\subsection*{4. Example Walkthrough}

Consider the first example:
\begin{verbatim}
Input: points = [[1,3],[-2,2]], K = 1
Output: [[-2,2]]
\end{verbatim}

\begin{enumerate}
    \item **Calculate Distances:**
    \begin{itemize}
        \item [1,3] : \(1^2 + 3^2 = 10\)
        \item [-2,2] : \((-2)^2 + 2^2 = 8\)
    \end{itemize}
    
    \item **QuickSelect Process:**
    \begin{itemize}
        \item Choose a pivot, say [1,3] with distance 10.
        \item Compare and rearrange:
        \begin{itemize}
            \item [-2,2] has a smaller distance (8) and is moved to the left.
        \end{itemize}
        \item After partitioning, the list becomes [[-2,2], [1,3]].
        \item Since \(K = 1\), return the first point: [[-2,2]].
    \end{itemize}
\end{enumerate}

Thus, the function correctly identifies \([-2,2]\) as the closest point to the origin.

\section*{Why This Approach}

The QuickSelect algorithm is chosen for its average-case linear time complexity \(O(n)\), making it highly efficient for large datasets compared to sorting-based methods with \(O(n \log n)\) time complexity. By avoiding the need to sort the entire list, QuickSelect provides an optimal solution for finding the \(k\) closest points.

\section*{Alternative Approaches}

\subsection*{1. Sorting Based on Distance}

Sort all points based on their distances from the origin and select the first \(k\) points.

\begin{lstlisting}[language=Python]
class Solution:
    def kClosest(self, points: List[List[int]], K: int) -> List[List[int]]:
        points.sort(key=lambda P: P[0]**2 + P[1]**2)
        return points[:K]
\end{lstlisting}

\textbf{Complexities:}
\begin{itemize}
    \item \textbf{Time Complexity:} \(O(n \log n)\)
    \item \textbf{Space Complexity:} \(O(1)\)
\end{itemize}

\subsection*{2. Max Heap (Priority Queue)}

Use a max heap to maintain the \(k\) closest points.

\begin{lstlisting}[language=Python]
import heapq

class Solution:
    def kClosest(self, points: List[List[int]], K: int) -> List[List[int]]:
        heap = []
        for (x, y) in points:
            dist = -(x**2 + y**2)  # Max heap using negative distances
            heapq.heappush(heap, (dist, [x, y]))
            if len(heap) > K:
                heapq.heappop(heap)
        return [item[1] for item in heap]
\end{lstlisting}

\textbf{Complexities:}
\begin{itemize}
    \item \textbf{Time Complexity:} \(O(n \log k)\)
    \item \textbf{Space Complexity:} \(O(k)\)
\end{itemize}

\subsection*{3. Using Built-In Functions}

Leverage built-in functions for distance calculation and selection.

\begin{lstlisting}[language=Python]
import math

class Solution:
    def kClosest(self, points: List[List[int]], K: int) -> List[List[int]]:
        points.sort(key=lambda P: math.sqrt(P[0]**2 + P[1]**2))
        return points[:K]
\end{lstlisting}

\textbf{Note}: This method is similar to the sorting approach but uses the actual Euclidean distance.

\section*{Similar Problems to This One}

Several problems involve nearest neighbor searches, spatial data analysis, and efficient selection algorithms, utilizing similar algorithmic strategies:

\begin{itemize}
    \item \textbf{Closest Pair of Points}: Find the closest pair of points in a set.
    \item \textbf{Top K Frequent Elements}: Identify the most frequent elements in a dataset.
    \item \textbf{Kth Largest Element in an Array}: Find the \(k\)-th largest element in an unsorted array.
    \item \textbf{Sliding Window Maximum}: Find the maximum in each sliding window of size \(k\) over an array.
    \item \textbf{Merge K Sorted Lists}: Merge multiple sorted lists into a single sorted list.
    \item \textbf{Find Median from Data Stream}: Continuously find the median of a stream of numbers.
    \item \textbf{Top K Closest Stars}: Find the \(k\) closest stars to Earth based on their distances.
\end{itemize}

These problems reinforce concepts of efficient selection, heap usage, and distance computations in various contexts.

\section*{Things to Keep in Mind and Tricks}

When solving the \textbf{K Closest Points to Origin} problem, consider the following tips and best practices to enhance efficiency and correctness:

\begin{itemize}
    \item \textbf{Understand Distance Calculations}: Grasp the Euclidean distance formula and recognize that the square root can be omitted for comparison purposes.
    \index{Distance Calculations}
    
    \item \textbf{Leverage Efficient Algorithms}: Use QuickSelect or heap-based methods to optimize time complexity, especially for large datasets.
    \index{Efficient Algorithms}
    
    \item \textbf{Handle Ties Appropriately}: Decide how to handle points with identical distances when \(k\) is less than the number of such points.
    \index{Handling Ties}
    
    \item \textbf{Optimize Space Usage}: Choose algorithms that minimize additional space, such as in-place QuickSelect.
    \index{Space Optimization}
    
    \item \textbf{Use Appropriate Data Structures}: Utilize heaps, lists, and helper functions effectively to manage and process data.
    \index{Data Structures}
    
    \item \textbf{Implement Helper Functions}: Create helper functions for distance calculation and partitioning to enhance code modularity.
    \index{Helper Functions}
    
    \item \textbf{Code Readability}: Maintain clear and readable code through meaningful variable names and structured logic.
    \index{Code Readability}
    
    \item \textbf{Test Extensively}: Implement a wide range of test cases, including edge cases like multiple points with the same distance, to ensure robustness.
    \index{Extensive Testing}
    
    \item \textbf{Understand Algorithm Trade-offs}: Recognize the trade-offs between different approaches in terms of time and space complexities.
    \index{Algorithm Trade-offs}
    
    \item \textbf{Use Built-In Sorting Functions}: When using sorting-based approaches, leverage built-in functions for efficiency and simplicity.
    \index{Built-In Sorting}
    
    \item \textbf{Avoid Redundant Calculations}: Ensure that distance calculations are performed only when necessary to optimize performance.
    \index{Avoiding Redundant Calculations}
    
    \item \textbf{Language-Specific Features}: Utilize language-specific features or libraries that can simplify implementation, such as heapq in Python.
    \index{Language-Specific Features}
\end{itemize}

\section*{Corner and Special Cases to Test When Writing the Code}

When implementing the solution for the \textbf{K Closest Points to Origin} problem, it is crucial to consider and rigorously test various edge cases to ensure robustness and correctness:

\begin{itemize}
    \item \textbf{Multiple Points with Same Distance}: Ensure that the algorithm handles multiple points having the same distance from the origin.
    \index{Same Distance Points}
    
    \item \textbf{Points at Origin}: Include points that are exactly at the origin \((0,0)\).
    \index{Points at Origin}
    
    \item \textbf{Negative Coordinates}: Ensure that the algorithm correctly computes distances for points with negative \(x\) or \(y\) coordinates.
    \index{Negative Coordinates}
    
    \item \textbf{Large Coordinates}: Test with points having very large or very small coordinate values to verify integer handling.
    \index{Large Coordinates}
    
    \item \textbf{K Equals Number of Points}: When \(K\) is equal to the number of points, the algorithm should return all points.
    \index{K Equals Number of Points}
    
    \item \textbf{K is One}: Test with \(K = 1\) to ensure the closest point is correctly identified.
    \index{K is One}
    
    \item \textbf{All Points Same}: All points have the same coordinates.
    \index{All Points Same}
    
    \item \textbf{K is Zero}: Although \(K\) is defined to be at least 1, ensure that the algorithm gracefully handles \(K = 0\) if allowed.
    \index{K is Zero}
    
    \item \textbf{Single Point}: Only one point is provided, and \(K = 1\).
    \index{Single Point}
    
    \item \textbf{Mixed Coordinates}: Points with a mix of positive and negative coordinates.
    \index{Mixed Coordinates}
    
    \item \textbf{Points with Zero Distance}: Multiple points at the origin.
    \index{Zero Distance Points}
    
    \item \textbf{Sparse and Dense Points}: Densely packed points and sparsely distributed points.
    \index{Sparse and Dense Points}
    
    \item \textbf{Duplicate Points}: Multiple identical points in the input list.
    \index{Duplicate Points}
    
    \item \textbf{K Greater Than Number of Unique Points}: Ensure that the algorithm handles cases where \(K\) exceeds the number of unique points if applicable.
    \index{K Greater Than Unique Points}
\end{itemize}

\section*{Implementation Considerations}

When implementing the \texttt{kClosest} function, keep in mind the following considerations to ensure robustness and efficiency:

\begin{itemize}
    \item \textbf{Data Type Selection}: Use appropriate data types that can handle large input values without overflow or precision loss.
    \index{Data Type Selection}
    
    \item \textbf{Optimizing Distance Calculations}: Avoid calculating the square root since it is unnecessary for comparison purposes.
    \index{Optimizing Distance Calculations}
    
    \item \textbf{Choosing the Right Algorithm}: Select an algorithm based on the size of the input and the value of \(K\) to optimize time and space complexities.
    \index{Choosing the Right Algorithm}
    
    \item \textbf{Language-Specific Libraries}: Utilize language-specific libraries and functions (e.g., \texttt{heapq} in Python) to simplify implementation and enhance performance.
    \index{Language-Specific Libraries}
    
    \item \textbf{Avoiding Redundant Calculations}: Ensure that each point's distance is calculated only once to optimize performance.
    \index{Avoiding Redundant Calculations}
    
    \item \textbf{Implementing Helper Functions}: Create helper functions for tasks like distance calculation and partitioning to enhance modularity and readability.
    \index{Helper Functions}
    
    \item \textbf{Edge Case Handling}: Implement checks for edge cases to prevent incorrect results or runtime errors.
    \index{Edge Case Handling}
    
    \item \textbf{Testing and Validation}: Develop a comprehensive suite of test cases that cover all possible scenarios, including edge cases, to validate the correctness and efficiency of the implementation.
    \index{Testing and Validation}
    
    \item \textbf{Scalability}: Design the algorithm to scale efficiently with increasing input sizes, maintaining performance and resource utilization.
    \index{Scalability}
    
    \item \textbf{Consistent Naming Conventions}: Use consistent and descriptive naming conventions for variables and functions to improve code clarity.
    \index{Naming Conventions}
    
    \item \textbf{Memory Management}: Ensure that the algorithm manages memory efficiently, especially when dealing with large datasets.
    \index{Memory Management}
    
    \item \textbf{Avoiding Stack Overflow}: If implementing recursive approaches, be mindful of recursion limits and potential stack overflow issues.
    \index{Avoiding Stack Overflow}
    
    \item \textbf{Implementing Iterative Solutions}: Prefer iterative solutions when recursion may lead to increased space complexity or stack overflow.
    \index{Implementing Iterative Solutions}
\end{itemize}

\section*{Conclusion}

The \textbf{K Closest Points to Origin} problem exemplifies the application of efficient selection algorithms and geometric computations to solve spatial challenges effectively. By leveraging QuickSelect or heap-based methods, the algorithm achieves optimal time and space complexities, making it highly suitable for large datasets. Understanding and implementing such techniques not only enhances problem-solving skills but also provides a foundation for tackling more advanced Computational Geometry problems involving nearest neighbor searches, clustering, and spatial data analysis.

\printindex

% \input{sections/rectangle_overlap}
% \input{sections/rectangle_area}
% \input{sections/k_closest_points_to_origin}
% \input{sections/the_skyline_problem}
% % filename: the_skyline_problem.tex

\problemsection{The Skyline Problem}
\label{chap:The_Skyline_Problem}
\marginnote{\href{https://leetcode.com/problems/the-skyline-problem/}{[LeetCode Link]}\index{LeetCode}}
\marginnote{\href{https://www.geeksforgeeks.org/the-skyline-problem/}{[GeeksForGeeks Link]}\index{GeeksForGeeks}}
\marginnote{\href{https://www.interviewbit.com/problems/the-skyline-problem/}{[InterviewBit Link]}\index{InterviewBit}}
\marginnote{\href{https://app.codesignal.com/challenges/the-skyline-problem}{[CodeSignal Link]}\index{CodeSignal}}
\marginnote{\href{https://www.codewars.com/kata/the-skyline-problem/train/python}{[Codewars Link]}\index{Codewars}}

The \textbf{Skyline Problem} is a complex Computational Geometry challenge that involves computing the skyline formed by a collection of buildings in a 2D cityscape. Each building is represented by its left and right x-coordinates and its height. The skyline is defined by a list of "key points" where the height changes. This problem tests one's ability to handle large datasets, implement efficient sweep line algorithms, and manage event-driven processing. Mastery of this problem is essential for applications in computer graphics, urban planning simulations, and geographic information systems (GIS).

\section*{Problem Statement}

You are given a list of buildings in a cityscape. Each building is represented as a triplet \([Li, Ri, Hi]\), where \(Li\) and \(Ri\) are the x-coordinates of the left and right edges of the building, respectively, and \(Hi\) is the height of the building.

The skyline should be represented as a list of key points \([x, y]\) in sorted order by \(x\)-coordinate, where \(y\) is the height of the skyline at that point. The skyline should only include critical points where the height changes.

\textbf{Function signature in Python:}
\begin{lstlisting}[language=Python]
def getSkyline(buildings: List[List[int]]) -> List[List[int]]:
\end{lstlisting}

\section*{Examples}

\textbf{Example 1:}

\begin{verbatim}
Input: buildings = [[2,9,10], [3,7,15], [5,12,12], [15,20,10], [19,24,8]]
Output: [[2,10], [3,15], [7,12], [12,0], [15,10], [20,8], [24,0]]
Explanation:
- At x=2, the first building starts, height=10.
- At x=3, the second building starts, height=15.
- At x=7, the second building ends, the third building is still ongoing, height=12.
- At x=12, the third building ends, height drops to 0.
- At x=15, the fourth building starts, height=10.
- At x=20, the fourth building ends, the fifth building is still ongoing, height=8.
- At x=24, the fifth building ends, height drops to 0.
\end{verbatim}

\textbf{Example 2:}

\begin{verbatim}
Input: buildings = [[0,2,3], [2,5,3]]
Output: [[0,3], [5,0]]
Explanation:
- The two buildings are contiguous and have the same height, so the skyline drops to 0 at x=5.
\end{verbatim}

\textbf{Example 3:}

\begin{verbatim}
Input: buildings = [[1,3,3], [2,4,4], [5,6,1]]
Output: [[1,3], [2,4], [4,0], [5,1], [6,0]]
Explanation:
- At x=1, first building starts, height=3.
- At x=2, second building starts, height=4.
- At x=4, second building ends, height drops to 0.
- At x=5, third building starts, height=1.
- At x=6, third building ends, height drops to 0.
\end{verbatim}

\textbf{Example 4:}

\begin{verbatim}
Input: buildings = [[0,5,0]]
Output: []
Explanation:
- A building with height 0 does not contribute to the skyline.
\end{verbatim}

\textbf{Constraints:}

\begin{itemize}
    \item \(1 \leq \text{buildings.length} \leq 10^4\)
    \item \(0 \leq Li < Ri \leq 10^9\)
    \item \(0 \leq Hi \leq 10^4\)
\end{itemize}

\section*{Algorithmic Approach}

The \textbf{Sweep Line Algorithm} is an efficient method for solving the Skyline Problem. It involves processing events (building start and end points) in sorted order while maintaining a data structure (typically a max heap) to keep track of active buildings. Here's a step-by-step approach:

\subsection*{1. Event Representation}

Transform each building into two events:
\begin{itemize}
    \item **Start Event:** \((Li, -Hi)\) – Negative height indicates a building starts.
    \item **End Event:** \((Ri, Hi)\) – Positive height indicates a building ends.
\end{itemize}

Sorting the events ensures that start events are processed before end events at the same x-coordinate, and taller buildings are processed before shorter ones.

\subsection*{2. Sorting the Events}

Sort all events based on:
\begin{enumerate}
    \item **x-coordinate:** Ascending order.
    \item **Height:**
    \begin{itemize}
        \item For start events, taller buildings come first.
        \item For end events, shorter buildings come first.
    \end{itemize}
\end{enumerate}

\subsection*{3. Processing the Events}

Use a max heap to keep track of active building heights. Iterate through the sorted events:
\begin{enumerate}
    \item **Start Event:**
    \begin{itemize}
        \item Add the building's height to the heap.
    \end{itemize}
    
    \item **End Event:**
    \begin{itemize}
        \item Remove the building's height from the heap.
    \end{itemize}
    
    \item **Determine Current Max Height:**
    \begin{itemize}
        \item The current max height is the top of the heap.
    \end{itemize}
    
    \item **Update Skyline:**
    \begin{itemize}
        \item If the current max height differs from the previous max height, add a new key point \([x, current\_max\_height]\).
    \end{itemize}
\end{enumerate}

\subsection*{4. Finalizing the Skyline}

After processing all events, the accumulated key points represent the skyline.

\marginnote{The Sweep Line Algorithm efficiently handles dynamic changes in active buildings, ensuring accurate skyline construction.}

\section*{Complexities}

\begin{itemize}
    \item \textbf{Time Complexity:} \(O(n \log n)\), where \(n\) is the number of buildings. Sorting the events takes \(O(n \log n)\), and each heap operation takes \(O(\log n)\).
    
    \item \textbf{Space Complexity:} \(O(n)\), due to the storage of events and the heap.
\end{itemize}

\section*{Python Implementation}

\marginnote{Implementing the Sweep Line Algorithm with a max heap ensures an efficient and accurate solution.}

Below is the complete Python code implementing the \texttt{getSkyline} function:

\begin{fullwidth}
\begin{lstlisting}[language=Python]
from typing import List
import heapq

class Solution:
    def getSkyline(self, buildings: List[List[int]]) -> List[List[int]]:
        # Create a list of all events
        # For start events, use negative height to ensure they are processed before end events
        events = []
        for L, R, H in buildings:
            events.append((L, -H))
            events.append((R, H))
        
        # Sort the events
        # First by x-coordinate, then by height
        events.sort()
        
        # Max heap to keep track of active buildings
        heap = [0]  # Initialize with ground level
        heapq.heapify(heap)
        active_heights = {0: 1}  # Dictionary to count heights
        
        result = []
        prev_max = 0
        
        for x, h in events:
            if h < 0:
                # Start of a building, add height to heap and dictionary
                heapq.heappush(heap, h)
                active_heights[h] = active_heights.get(h, 0) + 1
            else:
                # End of a building, remove height from dictionary
                active_heights[h] -= 1
                if active_heights[h] == 0:
                    del active_heights[h]
            
            # Current max height
            while heap and active_heights.get(heap[0], 0) == 0:
                heapq.heappop(heap)
            current_max = -heap[0] if heap else 0
            
            # If the max height has changed, add to result
            if current_max != prev_max:
                result.append([x, current_max])
                prev_max = current_max
        
        return result

# Example usage:
solution = Solution()
print(solution.getSkyline([[2,9,10], [3,7,15], [5,12,12], [15,20,10], [19,24,8]]))
# Output: [[2,10], [3,15], [7,12], [12,0], [15,10], [20,8], [24,0]]

print(solution.getSkyline([[0,2,3], [2,5,3]]))
# Output: [[0,3], [5,0]]

print(solution.getSkyline([[1,3,3], [2,4,4], [5,6,1]]))
# Output: [[1,3], [2,4], [4,0], [5,1], [6,0]]

print(solution.getSkyline([[0,5,0]]))
# Output: []
\end{lstlisting}
\end{fullwidth}

This implementation efficiently constructs the skyline by processing all building events in sorted order and maintaining active building heights using a max heap. It ensures that only critical points where the skyline changes are recorded.

\section*{Explanation}

The \texttt{getSkyline} function constructs the skyline formed by a set of buildings by leveraging the Sweep Line Algorithm and a max heap to track active buildings. Here's a detailed breakdown of the implementation:

\subsection*{1. Event Representation}

\begin{itemize}
    \item Each building is transformed into two events:
    \begin{itemize}
        \item **Start Event:** \((Li, -Hi)\) – Negative height indicates the start of a building.
        \item **End Event:** \((Ri, Hi)\) – Positive height indicates the end of a building.
    \end{itemize}
\end{itemize}

\subsection*{2. Sorting the Events}

\begin{itemize}
    \item Events are sorted primarily by their x-coordinate in ascending order.
    \item For events with the same x-coordinate:
    \begin{itemize}
        \item Start events (with negative heights) are processed before end events.
        \item Taller buildings are processed before shorter ones.
    \end{itemize}
\end{itemize}

\subsection*{3. Processing the Events}

\begin{itemize}
    \item **Heap Initialization:**
    \begin{itemize}
        \item A max heap is initialized with a ground level height of 0.
        \item A dictionary \texttt{active\_heights} tracks the count of active building heights.
    \end{itemize}
    
    \item **Iterating Through Events:**
    \begin{enumerate}
        \item **Start Event:**
        \begin{itemize}
            \item Add the building's height to the heap.
            \item Increment the count of the height in \texttt{active\_heights}.
        \end{itemize}
        
        \item **End Event:**
        \begin{itemize}
            \item Decrement the count of the building's height in \texttt{active\_heights}.
            \item If the count reaches zero, remove the height from the dictionary.
        \end{itemize}
        
        \item **Determine Current Max Height:**
        \begin{itemize}
            \item Remove heights from the heap that are no longer active.
            \item The current max height is the top of the heap.
        \end{itemize}
        
        \item **Update Skyline:**
        \begin{itemize}
            \item If the current max height differs from the previous max height, add a new key point \([x, current\_max\_height]\).
        \end{itemize}
    \end{enumerate}
\end{itemize}

\subsection*{4. Finalizing the Skyline}

\begin{itemize}
    \item After processing all events, the \texttt{result} list contains the key points defining the skyline.
\end{itemize}

\subsection*{5. Example Walkthrough}

Consider the first example:
\begin{verbatim}
Input: buildings = [[2,9,10], [3,7,15], [5,12,12], [15,20,10], [19,24,8]]
Output: [[2,10], [3,15], [7,12], [12,0], [15,10], [20,8], [24,0]]
\end{verbatim}

\begin{enumerate}
    \item **Event Transformation:**
    \begin{itemize}
        \item \((2, -10)\), \((9, 10)\)
        \item \((3, -15)\), \((7, 15)\)
        \item \((5, -12)\), \((12, 12)\)
        \item \((15, -10)\), \((20, 10)\)
        \item \((19, -8)\), \((24, 8)\)
    \end{itemize}
    
    \item **Sorting Events:**
    \begin{itemize}
        \item Sorted order: \((2, -10)\), \((3, -15)\), \((5, -12)\), \((7, 15)\), \((9, 10)\), \((12, 12)\), \((15, -10)\), \((19, -8)\), \((20, 10)\), \((24, 8)\)
    \end{itemize}
    
    \item **Processing Events:**
    \begin{itemize}
        \item At each event, update the heap and determine if the skyline height changes.
    \end{itemize}
    
    \item **Result Construction:**
    \begin{itemize}
        \item The resulting skyline key points are accumulated as \([[2,10], [3,15], [7,12], [12,0], [15,10], [20,8], [24,0]]\).
    \end{itemize}
\end{enumerate}

Thus, the function correctly constructs the skyline based on the buildings' positions and heights.

\section*{Why This Approach}

The Sweep Line Algorithm combined with a max heap offers an optimal solution with \(O(n \log n)\) time complexity and efficient handling of overlapping buildings. By processing events in sorted order and maintaining active building heights, the algorithm ensures that all critical points in the skyline are accurately identified without redundant computations.

\section*{Alternative Approaches}

\subsection*{1. Divide and Conquer}

Divide the set of buildings into smaller subsets, compute the skyline for each subset, and then merge the skylines.

\begin{lstlisting}[language=Python]
class Solution:
    def getSkyline(self, buildings: List[List[int]]) -> List[List[int]]:
        def merge(left, right):
            h1, h2 = 0, 0
            i, j = 0, 0
            merged = []
            while i < len(left) and j < len(right):
                if left[i][0] < right[j][0]:
                    x, h1 = left[i]
                    i += 1
                elif left[i][0] > right[j][0]:
                    x, h2 = right[j]
                    j += 1
                else:
                    x, h1 = left[i]
                    _, h2 = right[j]
                    i += 1
                    j += 1
                max_h = max(h1, h2)
                if not merged or merged[-1][1] != max_h:
                    merged.append([x, max_h])
            merged.extend(left[i:])
            merged.extend(right[j:])
            return merged
        
        def divide(buildings):
            if not buildings:
                return []
            if len(buildings) == 1:
                L, R, H = buildings[0]
                return [[L, H], [R, 0]]
            mid = len(buildings) // 2
            left = divide(buildings[:mid])
            right = divide(buildings[mid:])
            return merge(left, right)
        
        return divide(buildings)
\end{lstlisting}

\textbf{Complexities:}
\begin{itemize}
    \item \textbf{Time Complexity:} \(O(n \log n)\)
    \item \textbf{Space Complexity:} \(O(n)\)
\end{itemize}

\subsection*{2. Using Segment Trees}

Implement a segment tree to manage and query overlapping building heights dynamically.

\textbf{Note}: This approach is more complex and is generally used for advanced scenarios with multiple dynamic queries.

\section*{Similar Problems to This One}

Several problems involve skyline-like constructions, spatial data analysis, and efficient event processing, utilizing similar algorithmic strategies:

\begin{itemize}
    \item \textbf{Merge Intervals}: Merge overlapping intervals in a list.
    \item \textbf{Largest Rectangle in Histogram}: Find the largest rectangular area in a histogram.
    \item \textbf{Interval Partitioning}: Assign intervals to resources without overlap.
    \item \textbf{Line Segment Intersection}: Detect intersections among line segments.
    \item \textbf{Closest Pair of Points}: Find the closest pair of points in a set.
    \item \textbf{Convex Hull}: Compute the convex hull of a set of points.
    \item \textbf{Point Inside Polygon}: Determine if a point lies inside a given polygon.
    \item \textbf{Range Searching}: Efficiently query geometric data within a specified range.
\end{itemize}

These problems reinforce concepts of event-driven processing, spatial reasoning, and efficient algorithm design in various contexts.

\section*{Things to Keep in Mind and Tricks}

When tackling the \textbf{Skyline Problem}, consider the following tips and best practices to enhance efficiency and correctness:

\begin{itemize}
    \item \textbf{Understand Sweep Line Technique}: Grasp how the sweep line algorithm processes events in sorted order to handle dynamic changes efficiently.
    \index{Sweep Line Technique}
    
    \item \textbf{Leverage Priority Queues (Heaps)}: Use max heaps to keep track of active buildings' heights, enabling quick access to the current maximum height.
    \index{Priority Queues}
    
    \item \textbf{Handle Start and End Events Differently}: Differentiate between building start and end events to accurately manage active heights.
    \index{Start and End Events}
    
    \item \textbf{Optimize Event Sorting}: Sort events primarily by x-coordinate and secondarily by height to ensure correct processing order.
    \index{Event Sorting}
    
    \item \textbf{Manage Active Heights Efficiently}: Use data structures that allow efficient insertion, deletion, and retrieval of maximum elements.
    \index{Active Heights Management}
    
    \item \textbf{Avoid Redundant Key Points}: Only record key points when the skyline height changes to minimize the output list.
    \index{Avoiding Redundant Key Points}
    
    \item \textbf{Implement Helper Functions}: Create helper functions for tasks like distance calculation, event handling, and heap management to enhance modularity.
    \index{Helper Functions}
    
    \item \textbf{Code Readability}: Maintain clear and readable code through meaningful variable names and structured logic.
    \index{Code Readability}
    
    \item \textbf{Test Extensively}: Implement a wide range of test cases, including overlapping, non-overlapping, and edge-touching buildings, to ensure robustness.
    \index{Extensive Testing}
    
    \item \textbf{Handle Degenerate Cases}: Manage cases where buildings have zero height or identical coordinates gracefully.
    \index{Degenerate Cases}
    
    \item \textbf{Understand Geometric Relationships}: Grasp how buildings overlap and influence the skyline to simplify the algorithm.
    \index{Geometric Relationships}
    
    \item \textbf{Use Appropriate Data Structures}: Utilize appropriate data structures like heaps, lists, and dictionaries to manage and process data efficiently.
    \index{Appropriate Data Structures}
    
    \item \textbf{Optimize for Large Inputs}: Design the algorithm to handle large numbers of buildings without significant performance degradation.
    \index{Optimizing for Large Inputs}
    
    \item \textbf{Implement Iterative Solutions Carefully}: Ensure that loop conditions are correctly defined to prevent infinite loops or incorrect terminations.
    \index{Iterative Solutions}
    
    \item \textbf{Consistent Naming Conventions}: Use consistent and descriptive naming conventions for variables and functions to improve code clarity.
    \index{Naming Conventions}
\end{itemize}

\section*{Corner and Special Cases to Test When Writing the Code}

When implementing the solution for the \textbf{Skyline Problem}, it is crucial to consider and rigorously test various edge cases to ensure robustness and correctness:

\begin{itemize}
    \item \textbf{No Overlapping Buildings}: All buildings are separate and do not overlap.
    \index{No Overlapping Buildings}
    
    \item \textbf{Fully Overlapping Buildings}: Multiple buildings completely overlap each other.
    \index{Fully Overlapping Buildings}
    
    \item \textbf{Buildings Touching at Edges}: Buildings share common edges without overlapping.
    \index{Buildings Touching at Edges}
    
    \item \textbf{Buildings Touching at Corners}: Buildings share common corners without overlapping.
    \index{Buildings Touching at Corners}
    
    \item \textbf{Single Building}: Only one building is present.
    \index{Single Building}
    
    \item \textbf{Multiple Buildings with Same Start or End}: Multiple buildings start or end at the same x-coordinate.
    \index{Same Start or End}
    
    \item \textbf{Buildings with Zero Height}: Buildings that have zero height should not affect the skyline.
    \index{Buildings with Zero Height}
    
    \item \textbf{Large Number of Buildings}: Test with a large number of buildings to ensure performance and scalability.
    \index{Large Number of Buildings}
    
    \item \textbf{Buildings with Negative Coordinates}: Buildings positioned in negative coordinate space.
    \index{Negative Coordinates}
    
    \item \textbf{Boundary Values}: Buildings at the minimum and maximum limits of the coordinate range.
    \index{Boundary Values}
    
    \item \textbf{Buildings with Identical Coordinates}: Multiple buildings with the same coordinates.
    \index{Identical Coordinates}
    
    \item \textbf{Sequential Buildings}: Buildings placed sequentially without gaps.
    \index{Sequential Buildings}
    
    \item \textbf{Overlapping and Non-Overlapping Mixed}: A mix of overlapping and non-overlapping buildings.
    \index{Overlapping and Non-Overlapping Mixed}
    
    \item \textbf{Buildings with Very Large Heights}: Buildings with heights at the upper limit of the constraints.
    \index{Very Large Heights}
    
    \item \textbf{Empty Input}: No buildings are provided.
    \index{Empty Input}
\end{itemize}

\section*{Implementation Considerations}

When implementing the \texttt{getSkyline} function, keep in mind the following considerations to ensure robustness and efficiency:

\begin{itemize}
    \item \textbf{Data Type Selection}: Use appropriate data types that can handle large input values and avoid overflow or precision issues.
    \index{Data Type Selection}
    
    \item \textbf{Optimizing Event Sorting}: Efficiently sort events based on x-coordinates and heights to ensure correct processing order.
    \index{Optimizing Event Sorting}
    
    \item \textbf{Handling Large Inputs}: Design the algorithm to handle up to \(10^4\) buildings efficiently without significant performance degradation.
    \index{Handling Large Inputs}
    
    \item \textbf{Using Efficient Data Structures}: Utilize heaps, lists, and dictionaries effectively to manage and process events and active heights.
    \index{Efficient Data Structures}
    
    \item \textbf{Avoiding Redundant Calculations}: Ensure that distance and overlap calculations are performed only when necessary to optimize performance.
    \index{Avoiding Redundant Calculations}
    
    \item \textbf{Code Readability and Documentation}: Maintain clear and readable code through meaningful variable names and comprehensive comments to facilitate understanding and maintenance.
    \index{Code Readability}
    
    \item \textbf{Edge Case Handling}: Implement checks for edge cases to prevent incorrect results or runtime errors.
    \index{Edge Case Handling}
    
    \item \textbf{Implementing Helper Functions}: Create helper functions for tasks like distance calculation, event handling, and heap management to enhance modularity.
    \index{Helper Functions}
    
    \item \textbf{Consistent Naming Conventions}: Use consistent and descriptive naming conventions for variables and functions to improve code clarity.
    \index{Naming Conventions}
    
    \item \textbf{Memory Management}: Ensure that the algorithm manages memory efficiently, especially when dealing with large datasets.
    \index{Memory Management}
    
    \item \textbf{Implementing Iterative Solutions Carefully}: Ensure that loop conditions are correctly defined to prevent infinite loops or incorrect terminations.
    \index{Iterative Solutions}
    
    \item \textbf{Avoiding Floating-Point Precision Issues}: Since the problem deals with integers, floating-point precision is not a concern, simplifying the implementation.
    \index{Floating-Point Precision}
    
    \item \textbf{Testing and Validation}: Develop a comprehensive suite of test cases that cover all possible scenarios, including edge cases, to validate the correctness and efficiency of the implementation.
    \index{Testing and Validation}
    
    \item \textbf{Performance Considerations}: Optimize the loop conditions and operations to ensure that the function runs efficiently, especially for large input numbers.
    \index{Performance Considerations}
\end{itemize}

\section*{Conclusion}

The \textbf{Skyline Problem} is a quintessential example of applying advanced algorithmic techniques and geometric reasoning to solve complex spatial challenges. By leveraging the Sweep Line Algorithm and maintaining active building heights using a max heap, the solution efficiently constructs the skyline with optimal time and space complexities. Understanding and implementing such sophisticated algorithms not only enhances problem-solving skills but also provides a foundation for tackling a wide array of Computational Geometry problems in various domains, including computer graphics, urban planning simulations, and geographic information systems.

\printindex

% \input{sections/rectangle_overlap}
% \input{sections/rectangle_area}
% \input{sections/k_closest_points_to_origin}
% \input{sections/the_skyline_problem}
% % filename: the_skyline_problem.tex

\problemsection{The Skyline Problem}
\label{chap:The_Skyline_Problem}
\marginnote{\href{https://leetcode.com/problems/the-skyline-problem/}{[LeetCode Link]}\index{LeetCode}}
\marginnote{\href{https://www.geeksforgeeks.org/the-skyline-problem/}{[GeeksForGeeks Link]}\index{GeeksForGeeks}}
\marginnote{\href{https://www.interviewbit.com/problems/the-skyline-problem/}{[InterviewBit Link]}\index{InterviewBit}}
\marginnote{\href{https://app.codesignal.com/challenges/the-skyline-problem}{[CodeSignal Link]}\index{CodeSignal}}
\marginnote{\href{https://www.codewars.com/kata/the-skyline-problem/train/python}{[Codewars Link]}\index{Codewars}}

The \textbf{Skyline Problem} is a complex Computational Geometry challenge that involves computing the skyline formed by a collection of buildings in a 2D cityscape. Each building is represented by its left and right x-coordinates and its height. The skyline is defined by a list of "key points" where the height changes. This problem tests one's ability to handle large datasets, implement efficient sweep line algorithms, and manage event-driven processing. Mastery of this problem is essential for applications in computer graphics, urban planning simulations, and geographic information systems (GIS).

\section*{Problem Statement}

You are given a list of buildings in a cityscape. Each building is represented as a triplet \([Li, Ri, Hi]\), where \(Li\) and \(Ri\) are the x-coordinates of the left and right edges of the building, respectively, and \(Hi\) is the height of the building.

The skyline should be represented as a list of key points \([x, y]\) in sorted order by \(x\)-coordinate, where \(y\) is the height of the skyline at that point. The skyline should only include critical points where the height changes.

\textbf{Function signature in Python:}
\begin{lstlisting}[language=Python]
def getSkyline(buildings: List[List[int]]) -> List[List[int]]:
\end{lstlisting}

\section*{Examples}

\textbf{Example 1:}

\begin{verbatim}
Input: buildings = [[2,9,10], [3,7,15], [5,12,12], [15,20,10], [19,24,8]]
Output: [[2,10], [3,15], [7,12], [12,0], [15,10], [20,8], [24,0]]
Explanation:
- At x=2, the first building starts, height=10.
- At x=3, the second building starts, height=15.
- At x=7, the second building ends, the third building is still ongoing, height=12.
- At x=12, the third building ends, height drops to 0.
- At x=15, the fourth building starts, height=10.
- At x=20, the fourth building ends, the fifth building is still ongoing, height=8.
- At x=24, the fifth building ends, height drops to 0.
\end{verbatim}

\textbf{Example 2:}

\begin{verbatim}
Input: buildings = [[0,2,3], [2,5,3]]
Output: [[0,3], [5,0]]
Explanation:
- The two buildings are contiguous and have the same height, so the skyline drops to 0 at x=5.
\end{verbatim}

\textbf{Example 3:}

\begin{verbatim}
Input: buildings = [[1,3,3], [2,4,4], [5,6,1]]
Output: [[1,3], [2,4], [4,0], [5,1], [6,0]]
Explanation:
- At x=1, first building starts, height=3.
- At x=2, second building starts, height=4.
- At x=4, second building ends, height drops to 0.
- At x=5, third building starts, height=1.
- At x=6, third building ends, height drops to 0.
\end{verbatim}

\textbf{Example 4:}

\begin{verbatim}
Input: buildings = [[0,5,0]]
Output: []
Explanation:
- A building with height 0 does not contribute to the skyline.
\end{verbatim}

\textbf{Constraints:}

\begin{itemize}
    \item \(1 \leq \text{buildings.length} \leq 10^4\)
    \item \(0 \leq Li < Ri \leq 10^9\)
    \item \(0 \leq Hi \leq 10^4\)
\end{itemize}

\section*{Algorithmic Approach}

The \textbf{Sweep Line Algorithm} is an efficient method for solving the Skyline Problem. It involves processing events (building start and end points) in sorted order while maintaining a data structure (typically a max heap) to keep track of active buildings. Here's a step-by-step approach:

\subsection*{1. Event Representation}

Transform each building into two events:
\begin{itemize}
    \item **Start Event:** \((Li, -Hi)\) – Negative height indicates a building starts.
    \item **End Event:** \((Ri, Hi)\) – Positive height indicates a building ends.
\end{itemize}

Sorting the events ensures that start events are processed before end events at the same x-coordinate, and taller buildings are processed before shorter ones.

\subsection*{2. Sorting the Events}

Sort all events based on:
\begin{enumerate}
    \item **x-coordinate:** Ascending order.
    \item **Height:**
    \begin{itemize}
        \item For start events, taller buildings come first.
        \item For end events, shorter buildings come first.
    \end{itemize}
\end{enumerate}

\subsection*{3. Processing the Events}

Use a max heap to keep track of active building heights. Iterate through the sorted events:
\begin{enumerate}
    \item **Start Event:**
    \begin{itemize}
        \item Add the building's height to the heap.
    \end{itemize}
    
    \item **End Event:**
    \begin{itemize}
        \item Remove the building's height from the heap.
    \end{itemize}
    
    \item **Determine Current Max Height:**
    \begin{itemize}
        \item The current max height is the top of the heap.
    \end{itemize}
    
    \item **Update Skyline:**
    \begin{itemize}
        \item If the current max height differs from the previous max height, add a new key point \([x, current\_max\_height]\).
    \end{itemize}
\end{enumerate}

\subsection*{4. Finalizing the Skyline}

After processing all events, the accumulated key points represent the skyline.

\marginnote{The Sweep Line Algorithm efficiently handles dynamic changes in active buildings, ensuring accurate skyline construction.}

\section*{Complexities}

\begin{itemize}
    \item \textbf{Time Complexity:} \(O(n \log n)\), where \(n\) is the number of buildings. Sorting the events takes \(O(n \log n)\), and each heap operation takes \(O(\log n)\).
    
    \item \textbf{Space Complexity:} \(O(n)\), due to the storage of events and the heap.
\end{itemize}

\section*{Python Implementation}

\marginnote{Implementing the Sweep Line Algorithm with a max heap ensures an efficient and accurate solution.}

Below is the complete Python code implementing the \texttt{getSkyline} function:

\begin{fullwidth}
\begin{lstlisting}[language=Python]
from typing import List
import heapq

class Solution:
    def getSkyline(self, buildings: List[List[int]]) -> List[List[int]]:
        # Create a list of all events
        # For start events, use negative height to ensure they are processed before end events
        events = []
        for L, R, H in buildings:
            events.append((L, -H))
            events.append((R, H))
        
        # Sort the events
        # First by x-coordinate, then by height
        events.sort()
        
        # Max heap to keep track of active buildings
        heap = [0]  # Initialize with ground level
        heapq.heapify(heap)
        active_heights = {0: 1}  # Dictionary to count heights
        
        result = []
        prev_max = 0
        
        for x, h in events:
            if h < 0:
                # Start of a building, add height to heap and dictionary
                heapq.heappush(heap, h)
                active_heights[h] = active_heights.get(h, 0) + 1
            else:
                # End of a building, remove height from dictionary
                active_heights[h] -= 1
                if active_heights[h] == 0:
                    del active_heights[h]
            
            # Current max height
            while heap and active_heights.get(heap[0], 0) == 0:
                heapq.heappop(heap)
            current_max = -heap[0] if heap else 0
            
            # If the max height has changed, add to result
            if current_max != prev_max:
                result.append([x, current_max])
                prev_max = current_max
        
        return result

# Example usage:
solution = Solution()
print(solution.getSkyline([[2,9,10], [3,7,15], [5,12,12], [15,20,10], [19,24,8]]))
# Output: [[2,10], [3,15], [7,12], [12,0], [15,10], [20,8], [24,0]]

print(solution.getSkyline([[0,2,3], [2,5,3]]))
# Output: [[0,3], [5,0]]

print(solution.getSkyline([[1,3,3], [2,4,4], [5,6,1]]))
# Output: [[1,3], [2,4], [4,0], [5,1], [6,0]]

print(solution.getSkyline([[0,5,0]]))
# Output: []
\end{lstlisting}
\end{fullwidth}

This implementation efficiently constructs the skyline by processing all building events in sorted order and maintaining active building heights using a max heap. It ensures that only critical points where the skyline changes are recorded.

\section*{Explanation}

The \texttt{getSkyline} function constructs the skyline formed by a set of buildings by leveraging the Sweep Line Algorithm and a max heap to track active buildings. Here's a detailed breakdown of the implementation:

\subsection*{1. Event Representation}

\begin{itemize}
    \item Each building is transformed into two events:
    \begin{itemize}
        \item **Start Event:** \((Li, -Hi)\) – Negative height indicates the start of a building.
        \item **End Event:** \((Ri, Hi)\) – Positive height indicates the end of a building.
    \end{itemize}
\end{itemize}

\subsection*{2. Sorting the Events}

\begin{itemize}
    \item Events are sorted primarily by their x-coordinate in ascending order.
    \item For events with the same x-coordinate:
    \begin{itemize}
        \item Start events (with negative heights) are processed before end events.
        \item Taller buildings are processed before shorter ones.
    \end{itemize}
\end{itemize}

\subsection*{3. Processing the Events}

\begin{itemize}
    \item **Heap Initialization:**
    \begin{itemize}
        \item A max heap is initialized with a ground level height of 0.
        \item A dictionary \texttt{active\_heights} tracks the count of active building heights.
    \end{itemize}
    
    \item **Iterating Through Events:**
    \begin{enumerate}
        \item **Start Event:**
        \begin{itemize}
            \item Add the building's height to the heap.
            \item Increment the count of the height in \texttt{active\_heights}.
        \end{itemize}
        
        \item **End Event:**
        \begin{itemize}
            \item Decrement the count of the building's height in \texttt{active\_heights}.
            \item If the count reaches zero, remove the height from the dictionary.
        \end{itemize}
        
        \item **Determine Current Max Height:**
        \begin{itemize}
            \item Remove heights from the heap that are no longer active.
            \item The current max height is the top of the heap.
        \end{itemize}
        
        \item **Update Skyline:**
        \begin{itemize}
            \item If the current max height differs from the previous max height, add a new key point \([x, current\_max\_height]\).
        \end{itemize}
    \end{enumerate}
\end{itemize}

\subsection*{4. Finalizing the Skyline}

\begin{itemize}
    \item After processing all events, the \texttt{result} list contains the key points defining the skyline.
\end{itemize}

\subsection*{5. Example Walkthrough}

Consider the first example:
\begin{verbatim}
Input: buildings = [[2,9,10], [3,7,15], [5,12,12], [15,20,10], [19,24,8]]
Output: [[2,10], [3,15], [7,12], [12,0], [15,10], [20,8], [24,0]]
\end{verbatim}

\begin{enumerate}
    \item **Event Transformation:**
    \begin{itemize}
        \item \((2, -10)\), \((9, 10)\)
        \item \((3, -15)\), \((7, 15)\)
        \item \((5, -12)\), \((12, 12)\)
        \item \((15, -10)\), \((20, 10)\)
        \item \((19, -8)\), \((24, 8)\)
    \end{itemize}
    
    \item **Sorting Events:**
    \begin{itemize}
        \item Sorted order: \((2, -10)\), \((3, -15)\), \((5, -12)\), \((7, 15)\), \((9, 10)\), \((12, 12)\), \((15, -10)\), \((19, -8)\), \((20, 10)\), \((24, 8)\)
    \end{itemize}
    
    \item **Processing Events:**
    \begin{itemize}
        \item At each event, update the heap and determine if the skyline height changes.
    \end{itemize}
    
    \item **Result Construction:**
    \begin{itemize}
        \item The resulting skyline key points are accumulated as \([[2,10], [3,15], [7,12], [12,0], [15,10], [20,8], [24,0]]\).
    \end{itemize}
\end{enumerate}

Thus, the function correctly constructs the skyline based on the buildings' positions and heights.

\section*{Why This Approach}

The Sweep Line Algorithm combined with a max heap offers an optimal solution with \(O(n \log n)\) time complexity and efficient handling of overlapping buildings. By processing events in sorted order and maintaining active building heights, the algorithm ensures that all critical points in the skyline are accurately identified without redundant computations.

\section*{Alternative Approaches}

\subsection*{1. Divide and Conquer}

Divide the set of buildings into smaller subsets, compute the skyline for each subset, and then merge the skylines.

\begin{lstlisting}[language=Python]
class Solution:
    def getSkyline(self, buildings: List[List[int]]) -> List[List[int]]:
        def merge(left, right):
            h1, h2 = 0, 0
            i, j = 0, 0
            merged = []
            while i < len(left) and j < len(right):
                if left[i][0] < right[j][0]:
                    x, h1 = left[i]
                    i += 1
                elif left[i][0] > right[j][0]:
                    x, h2 = right[j]
                    j += 1
                else:
                    x, h1 = left[i]
                    _, h2 = right[j]
                    i += 1
                    j += 1
                max_h = max(h1, h2)
                if not merged or merged[-1][1] != max_h:
                    merged.append([x, max_h])
            merged.extend(left[i:])
            merged.extend(right[j:])
            return merged
        
        def divide(buildings):
            if not buildings:
                return []
            if len(buildings) == 1:
                L, R, H = buildings[0]
                return [[L, H], [R, 0]]
            mid = len(buildings) // 2
            left = divide(buildings[:mid])
            right = divide(buildings[mid:])
            return merge(left, right)
        
        return divide(buildings)
\end{lstlisting}

\textbf{Complexities:}
\begin{itemize}
    \item \textbf{Time Complexity:} \(O(n \log n)\)
    \item \textbf{Space Complexity:} \(O(n)\)
\end{itemize}

\subsection*{2. Using Segment Trees}

Implement a segment tree to manage and query overlapping building heights dynamically.

\textbf{Note}: This approach is more complex and is generally used for advanced scenarios with multiple dynamic queries.

\section*{Similar Problems to This One}

Several problems involve skyline-like constructions, spatial data analysis, and efficient event processing, utilizing similar algorithmic strategies:

\begin{itemize}
    \item \textbf{Merge Intervals}: Merge overlapping intervals in a list.
    \item \textbf{Largest Rectangle in Histogram}: Find the largest rectangular area in a histogram.
    \item \textbf{Interval Partitioning}: Assign intervals to resources without overlap.
    \item \textbf{Line Segment Intersection}: Detect intersections among line segments.
    \item \textbf{Closest Pair of Points}: Find the closest pair of points in a set.
    \item \textbf{Convex Hull}: Compute the convex hull of a set of points.
    \item \textbf{Point Inside Polygon}: Determine if a point lies inside a given polygon.
    \item \textbf{Range Searching}: Efficiently query geometric data within a specified range.
\end{itemize}

These problems reinforce concepts of event-driven processing, spatial reasoning, and efficient algorithm design in various contexts.

\section*{Things to Keep in Mind and Tricks}

When tackling the \textbf{Skyline Problem}, consider the following tips and best practices to enhance efficiency and correctness:

\begin{itemize}
    \item \textbf{Understand Sweep Line Technique}: Grasp how the sweep line algorithm processes events in sorted order to handle dynamic changes efficiently.
    \index{Sweep Line Technique}
    
    \item \textbf{Leverage Priority Queues (Heaps)}: Use max heaps to keep track of active buildings' heights, enabling quick access to the current maximum height.
    \index{Priority Queues}
    
    \item \textbf{Handle Start and End Events Differently}: Differentiate between building start and end events to accurately manage active heights.
    \index{Start and End Events}
    
    \item \textbf{Optimize Event Sorting}: Sort events primarily by x-coordinate and secondarily by height to ensure correct processing order.
    \index{Event Sorting}
    
    \item \textbf{Manage Active Heights Efficiently}: Use data structures that allow efficient insertion, deletion, and retrieval of maximum elements.
    \index{Active Heights Management}
    
    \item \textbf{Avoid Redundant Key Points}: Only record key points when the skyline height changes to minimize the output list.
    \index{Avoiding Redundant Key Points}
    
    \item \textbf{Implement Helper Functions}: Create helper functions for tasks like distance calculation, event handling, and heap management to enhance modularity.
    \index{Helper Functions}
    
    \item \textbf{Code Readability}: Maintain clear and readable code through meaningful variable names and structured logic.
    \index{Code Readability}
    
    \item \textbf{Test Extensively}: Implement a wide range of test cases, including overlapping, non-overlapping, and edge-touching buildings, to ensure robustness.
    \index{Extensive Testing}
    
    \item \textbf{Handle Degenerate Cases}: Manage cases where buildings have zero height or identical coordinates gracefully.
    \index{Degenerate Cases}
    
    \item \textbf{Understand Geometric Relationships}: Grasp how buildings overlap and influence the skyline to simplify the algorithm.
    \index{Geometric Relationships}
    
    \item \textbf{Use Appropriate Data Structures}: Utilize appropriate data structures like heaps, lists, and dictionaries to manage and process data efficiently.
    \index{Appropriate Data Structures}
    
    \item \textbf{Optimize for Large Inputs}: Design the algorithm to handle large numbers of buildings without significant performance degradation.
    \index{Optimizing for Large Inputs}
    
    \item \textbf{Implement Iterative Solutions Carefully}: Ensure that loop conditions are correctly defined to prevent infinite loops or incorrect terminations.
    \index{Iterative Solutions}
    
    \item \textbf{Consistent Naming Conventions}: Use consistent and descriptive naming conventions for variables and functions to improve code clarity.
    \index{Naming Conventions}
\end{itemize}

\section*{Corner and Special Cases to Test When Writing the Code}

When implementing the solution for the \textbf{Skyline Problem}, it is crucial to consider and rigorously test various edge cases to ensure robustness and correctness:

\begin{itemize}
    \item \textbf{No Overlapping Buildings}: All buildings are separate and do not overlap.
    \index{No Overlapping Buildings}
    
    \item \textbf{Fully Overlapping Buildings}: Multiple buildings completely overlap each other.
    \index{Fully Overlapping Buildings}
    
    \item \textbf{Buildings Touching at Edges}: Buildings share common edges without overlapping.
    \index{Buildings Touching at Edges}
    
    \item \textbf{Buildings Touching at Corners}: Buildings share common corners without overlapping.
    \index{Buildings Touching at Corners}
    
    \item \textbf{Single Building}: Only one building is present.
    \index{Single Building}
    
    \item \textbf{Multiple Buildings with Same Start or End}: Multiple buildings start or end at the same x-coordinate.
    \index{Same Start or End}
    
    \item \textbf{Buildings with Zero Height}: Buildings that have zero height should not affect the skyline.
    \index{Buildings with Zero Height}
    
    \item \textbf{Large Number of Buildings}: Test with a large number of buildings to ensure performance and scalability.
    \index{Large Number of Buildings}
    
    \item \textbf{Buildings with Negative Coordinates}: Buildings positioned in negative coordinate space.
    \index{Negative Coordinates}
    
    \item \textbf{Boundary Values}: Buildings at the minimum and maximum limits of the coordinate range.
    \index{Boundary Values}
    
    \item \textbf{Buildings with Identical Coordinates}: Multiple buildings with the same coordinates.
    \index{Identical Coordinates}
    
    \item \textbf{Sequential Buildings}: Buildings placed sequentially without gaps.
    \index{Sequential Buildings}
    
    \item \textbf{Overlapping and Non-Overlapping Mixed}: A mix of overlapping and non-overlapping buildings.
    \index{Overlapping and Non-Overlapping Mixed}
    
    \item \textbf{Buildings with Very Large Heights}: Buildings with heights at the upper limit of the constraints.
    \index{Very Large Heights}
    
    \item \textbf{Empty Input}: No buildings are provided.
    \index{Empty Input}
\end{itemize}

\section*{Implementation Considerations}

When implementing the \texttt{getSkyline} function, keep in mind the following considerations to ensure robustness and efficiency:

\begin{itemize}
    \item \textbf{Data Type Selection}: Use appropriate data types that can handle large input values and avoid overflow or precision issues.
    \index{Data Type Selection}
    
    \item \textbf{Optimizing Event Sorting}: Efficiently sort events based on x-coordinates and heights to ensure correct processing order.
    \index{Optimizing Event Sorting}
    
    \item \textbf{Handling Large Inputs}: Design the algorithm to handle up to \(10^4\) buildings efficiently without significant performance degradation.
    \index{Handling Large Inputs}
    
    \item \textbf{Using Efficient Data Structures}: Utilize heaps, lists, and dictionaries effectively to manage and process events and active heights.
    \index{Efficient Data Structures}
    
    \item \textbf{Avoiding Redundant Calculations}: Ensure that distance and overlap calculations are performed only when necessary to optimize performance.
    \index{Avoiding Redundant Calculations}
    
    \item \textbf{Code Readability and Documentation}: Maintain clear and readable code through meaningful variable names and comprehensive comments to facilitate understanding and maintenance.
    \index{Code Readability}
    
    \item \textbf{Edge Case Handling}: Implement checks for edge cases to prevent incorrect results or runtime errors.
    \index{Edge Case Handling}
    
    \item \textbf{Implementing Helper Functions}: Create helper functions for tasks like distance calculation, event handling, and heap management to enhance modularity.
    \index{Helper Functions}
    
    \item \textbf{Consistent Naming Conventions}: Use consistent and descriptive naming conventions for variables and functions to improve code clarity.
    \index{Naming Conventions}
    
    \item \textbf{Memory Management}: Ensure that the algorithm manages memory efficiently, especially when dealing with large datasets.
    \index{Memory Management}
    
    \item \textbf{Implementing Iterative Solutions Carefully}: Ensure that loop conditions are correctly defined to prevent infinite loops or incorrect terminations.
    \index{Iterative Solutions}
    
    \item \textbf{Avoiding Floating-Point Precision Issues}: Since the problem deals with integers, floating-point precision is not a concern, simplifying the implementation.
    \index{Floating-Point Precision}
    
    \item \textbf{Testing and Validation}: Develop a comprehensive suite of test cases that cover all possible scenarios, including edge cases, to validate the correctness and efficiency of the implementation.
    \index{Testing and Validation}
    
    \item \textbf{Performance Considerations}: Optimize the loop conditions and operations to ensure that the function runs efficiently, especially for large input numbers.
    \index{Performance Considerations}
\end{itemize}

\section*{Conclusion}

The \textbf{Skyline Problem} is a quintessential example of applying advanced algorithmic techniques and geometric reasoning to solve complex spatial challenges. By leveraging the Sweep Line Algorithm and maintaining active building heights using a max heap, the solution efficiently constructs the skyline with optimal time and space complexities. Understanding and implementing such sophisticated algorithms not only enhances problem-solving skills but also provides a foundation for tackling a wide array of Computational Geometry problems in various domains, including computer graphics, urban planning simulations, and geographic information systems.

\printindex

% % filename: rectangle_overlap.tex

\problemsection{Rectangle Overlap}
\label{chap:Rectangle_Overlap}
\marginnote{\href{https://leetcode.com/problems/rectangle-overlap/}{[LeetCode Link]}\index{LeetCode}}
\marginnote{\href{https://www.geeksforgeeks.org/check-if-two-rectangles-overlap/}{[GeeksForGeeks Link]}\index{GeeksForGeeks}}
\marginnote{\href{https://www.interviewbit.com/problems/rectangle-overlap/}{[InterviewBit Link]}\index{InterviewBit}}
\marginnote{\href{https://app.codesignal.com/challenges/rectangle-overlap}{[CodeSignal Link]}\index{CodeSignal}}
\marginnote{\href{https://www.codewars.com/kata/rectangle-overlap/train/python}{[Codewars Link]}\index{Codewars}}

The \textbf{Rectangle Overlap} problem is a fundamental challenge in Computational Geometry that involves determining whether two axis-aligned rectangles overlap. This problem tests one's ability to understand geometric properties, implement conditional logic, and optimize for efficient computation. Mastery of this problem is essential for applications in computer graphics, collision detection, and spatial data analysis.

\section*{Problem Statement}

Given two axis-aligned rectangles in a 2D plane, determine if they overlap. Each rectangle is defined by its bottom-left and top-right coordinates.

A rectangle is represented as a list of four integers \([x1, y1, x2, y2]\), where \((x1, y1)\) are the coordinates of the bottom-left corner, and \((x2, y2)\) are the coordinates of the top-right corner.

\textbf{Function signature in Python:}
\begin{lstlisting}[language=Python]
def isRectangleOverlap(rec1: List[int], rec2: List[int]) -> bool:
\end{lstlisting}

\section*{Examples}

\textbf{Example 1:}

\begin{verbatim}
Input: rec1 = [0,0,2,2], rec2 = [1,1,3,3]
Output: True
Explanation: The rectangles overlap in the area defined by [1,1,2,2].
\end{verbatim}

\textbf{Example 2:}

\begin{verbatim}
Input: rec1 = [0,0,1,1], rec2 = [1,0,2,1]
Output: False
Explanation: The rectangles touch at the edge but do not overlap.
\end{verbatim}

\textbf{Example 3:}

\begin{verbatim}
Input: rec1 = [0,0,1,1], rec2 = [2,2,3,3]
Output: False
Explanation: The rectangles are completely separate.
\end{verbatim}

\textbf{Example 4:}

\begin{verbatim}
Input: rec1 = [0,0,5,5], rec2 = [3,3,7,7]
Output: True
Explanation: The rectangles overlap in the area defined by [3,3,5,5].
\end{verbatim}

\textbf{Example 5:}

\begin{verbatim}
Input: rec1 = [0,0,0,0], rec2 = [0,0,0,0]
Output: False
Explanation: Both rectangles are degenerate points.
\end{verbatim}

\textbf{Constraints:}

\begin{itemize}
    \item All coordinates are integers in the range \([-10^9, 10^9]\).
    \item For each rectangle, \(x1 < x2\) and \(y1 < y2\).
\end{itemize}

LeetCode link: \href{https://leetcode.com/problems/rectangle-overlap/}{Rectangle Overlap}\index{LeetCode}

\section*{Algorithmic Approach}

To determine whether two axis-aligned rectangles overlap, we can use the following logical conditions:

1. **Non-Overlap Conditions:**
   - One rectangle is to the left of the other.
   - One rectangle is above the other.

2. **Overlap Condition:**
   - If neither of the non-overlap conditions is true, the rectangles must overlap.

\subsection*{Steps:}

1. **Extract Coordinates:**
   - For both rectangles, extract the bottom-left and top-right coordinates.

2. **Check Non-Overlap Conditions:**
   - If the right side of the first rectangle is less than or equal to the left side of the second rectangle, they do not overlap.
   - If the left side of the first rectangle is greater than or equal to the right side of the second rectangle, they do not overlap.
   - If the top side of the first rectangle is less than or equal to the bottom side of the second rectangle, they do not overlap.
   - If the bottom side of the first rectangle is greater than or equal to the top side of the second rectangle, they do not overlap.

3. **Determine Overlap:**
   - If none of the non-overlap conditions are met, the rectangles overlap.

\marginnote{This approach provides an efficient \(O(1)\) time complexity solution by leveraging simple geometric comparisons.}

\section*{Complexities}

\begin{itemize}
    \item \textbf{Time Complexity:} \(O(1)\). The algorithm performs a constant number of comparisons regardless of input size.
    
    \item \textbf{Space Complexity:} \(O(1)\). Only a fixed amount of extra space is used for variables.
\end{itemize}

\section*{Python Implementation}

\marginnote{Implementing the overlap check using coordinate comparisons ensures an optimal and straightforward solution.}

Below is the complete Python code implementing the \texttt{isRectangleOverlap} function:

\begin{fullwidth}
\begin{lstlisting}[language=Python]
from typing import List

class Solution:
    def isRectangleOverlap(self, rec1: List[int], rec2: List[int]) -> bool:
        # Extract coordinates
        left1, bottom1, right1, top1 = rec1
        left2, bottom2, right2, top2 = rec2
        
        # Check non-overlapping conditions
        if right1 <= left2 or right2 <= left1:
            return False
        if top1 <= bottom2 or top2 <= bottom1:
            return False
        
        # If none of the above, rectangles overlap
        return True

# Example usage:
solution = Solution()
print(solution.isRectangleOverlap([0,0,2,2], [1,1,3,3]))  # Output: True
print(solution.isRectangleOverlap([0,0,1,1], [1,0,2,1]))  # Output: False
print(solution.isRectangleOverlap([0,0,1,1], [2,2,3,3]))  # Output: False
print(solution.isRectangleOverlap([0,0,5,5], [3,3,7,7]))  # Output: True
print(solution.isRectangleOverlap([0,0,0,0], [0,0,0,0]))  # Output: False
\end{lstlisting}
\end{fullwidth}

This implementation efficiently checks for overlap by comparing the coordinates of the two rectangles. If any of the non-overlapping conditions are met, it returns \texttt{False}; otherwise, it returns \texttt{True}.

\section*{Explanation}

The \texttt{isRectangleOverlap} function determines whether two axis-aligned rectangles overlap by comparing their respective coordinates. Here's a detailed breakdown of the implementation:

\subsection*{1. Extract Coordinates}

\begin{itemize}
    \item For each rectangle, extract the left (\(x1\)), bottom (\(y1\)), right (\(x2\)), and top (\(y2\)) coordinates.
    \item This simplifies the comparison process by providing clear variables representing each side of the rectangles.
\end{itemize}

\subsection*{2. Check Non-Overlap Conditions}

\begin{itemize}
    \item **Horizontal Separation:**
    \begin{itemize}
        \item If the right side of the first rectangle (\(right1\)) is less than or equal to the left side of the second rectangle (\(left2\)), there is no horizontal overlap.
        \item Similarly, if the right side of the second rectangle (\(right2\)) is less than or equal to the left side of the first rectangle (\(left1\)), there is no horizontal overlap.
    \end{itemize}
    
    \item **Vertical Separation:**
    \begin{itemize}
        \item If the top side of the first rectangle (\(top1\)) is less than or equal to the bottom side of the second rectangle (\(bottom2\)), there is no vertical overlap.
        \item Similarly, if the top side of the second rectangle (\(top2\)) is less than or equal to the bottom side of the first rectangle (\(bottom1\)), there is no vertical overlap.
    \end{itemize}
    
    \item If any of these non-overlapping conditions are true, the rectangles do not overlap, and the function returns \texttt{False}.
\end{itemize}

\subsection*{3. Determine Overlap}

\begin{itemize}
    \item If none of the non-overlapping conditions are met, it implies that the rectangles overlap both horizontally and vertically.
    \item The function returns \texttt{True} in this case.
\end{itemize}

\subsection*{4. Example Walkthrough}

Consider the first example:
\begin{verbatim}
Input: rec1 = [0,0,2,2], rec2 = [1,1,3,3]
Output: True
\end{verbatim}

\begin{enumerate}
    \item Extract coordinates:
    \begin{itemize}
        \item rec1: left1 = 0, bottom1 = 0, right1 = 2, top1 = 2
        \item rec2: left2 = 1, bottom2 = 1, right2 = 3, top2 = 3
    \end{itemize}
    
    \item Check non-overlap conditions:
    \begin{itemize}
        \item \(right1 = 2\) is not less than or equal to \(left2 = 1\)
        \item \(right2 = 3\) is not less than or equal to \(left1 = 0\)
        \item \(top1 = 2\) is not less than or equal to \(bottom2 = 1\)
        \item \(top2 = 3\) is not less than or equal to \(bottom1 = 0\)
    \end{itemize}
    
    \item Since none of the non-overlapping conditions are met, the rectangles overlap.
\end{enumerate}

Thus, the function correctly returns \texttt{True}.

\section*{Why This Approach}

This approach is chosen for its simplicity and efficiency. By leveraging direct coordinate comparisons, the algorithm achieves constant time complexity without the need for complex data structures or iterative processes. It effectively handles all possible scenarios of rectangle positioning, ensuring accurate detection of overlaps.

\section*{Alternative Approaches}

\subsection*{1. Separating Axis Theorem (SAT)}

The Separating Axis Theorem is a more generalized method for detecting overlaps between convex shapes. While it is not necessary for axis-aligned rectangles, understanding SAT can be beneficial for more complex geometric problems.

\begin{lstlisting}[language=Python]
def isRectangleOverlap(rec1: List[int], rec2: List[int]) -> bool:
    # Using SAT for axis-aligned rectangles
    return not (rec1[2] <= rec2[0] or rec1[0] >= rec2[2] or
                rec1[3] <= rec2[1] or rec1[1] >= rec2[3])
\end{lstlisting}

\textbf{Note}: This implementation is functionally identical to the primary approach but leverages a more generalized geometric theorem.

\subsection*{2. Area-Based Approach}

Calculate the overlapping area between the two rectangles. If the overlapping area is positive, the rectangles overlap.

\begin{lstlisting}[language=Python]
def isRectangleOverlap(rec1: List[int], rec2: List[int]) -> bool:
    # Calculate overlap in x and y dimensions
    x_overlap = min(rec1[2], rec2[2]) - max(rec1[0], rec2[0])
    y_overlap = min(rec1[3], rec2[3]) - max(rec1[1], rec2[1])
    
    # Overlap exists if both overlaps are positive
    return x_overlap > 0 and y_overlap > 0
\end{lstlisting}

\textbf{Complexities:}
\begin{itemize}
    \item \textbf{Time Complexity:} \(O(1)\)
    \item \textbf{Space Complexity:} \(O(1)\)
\end{itemize}

\subsection*{3. Using Rectangles Intersection Function}

Utilize built-in or library functions that handle geometric intersections.

\begin{lstlisting}[language=Python]
from shapely.geometry import box

def isRectangleOverlap(rec1: List[int], rec2: List[int]) -> bool:
    rectangle1 = box(rec1[0], rec1[1], rec1[2], rec1[3])
    rectangle2 = box(rec2[0], rec2[1], rec2[2], rec2[3])
    return rectangle1.intersects(rectangle2) and not rectangle1.touches(rectangle2)
\end{lstlisting}

\textbf{Note}: This approach requires the \texttt{shapely} library and is more suitable for complex geometric operations.

\section*{Similar Problems to This One}

Several problems revolve around geometric overlap, intersection detection, and spatial reasoning, utilizing similar algorithmic strategies:

\begin{itemize}
    \item \textbf{Interval Overlap}: Determine if two intervals on a line overlap.
    \item \textbf{Circle Overlap}: Determine if two circles overlap based on their radii and centers.
    \item \textbf{Polygon Overlap}: Determine if two polygons overlap using algorithms like SAT.
    \item \textbf{Closest Pair of Points}: Find the closest pair of points in a set.
    \item \textbf{Convex Hull}: Compute the convex hull of a set of points.
    \item \textbf{Intersection of Lines}: Find the intersection point of two lines.
    \item \textbf{Point Inside Polygon}: Determine if a point lies inside a given polygon.
\end{itemize}

These problems reinforce the concepts of spatial reasoning, geometric property analysis, and efficient algorithm design in various contexts.

\section*{Things to Keep in Mind and Tricks}

When working with the \textbf{Rectangle Overlap} problem, consider the following tips and best practices to enhance efficiency and correctness:

\begin{itemize}
    \item \textbf{Understand Geometric Relationships}: Grasp the positional relationships between rectangles to simplify overlap detection.
    \index{Geometric Relationships}
    
    \item \textbf{Leverage Coordinate Comparisons}: Use direct comparisons of rectangle coordinates to determine spatial relationships.
    \index{Coordinate Comparisons}
    
    \item \textbf{Handle Edge Cases}: Consider cases where rectangles touch at edges or corners without overlapping.
    \index{Edge Cases}
    
    \item \textbf{Optimize for Efficiency}: Aim for a constant time \(O(1)\) solution by avoiding unnecessary computations or iterations.
    \index{Efficiency Optimization}
    
    \item \textbf{Avoid Floating-Point Precision Issues}: Since all coordinates are integers, floating-point precision is not a concern, simplifying the implementation.
    \index{Floating-Point Precision}
    
    \item \textbf{Use Helper Functions}: Create helper functions to encapsulate repetitive tasks, such as extracting coordinates or checking specific conditions.
    \index{Helper Functions}
    
    \item \textbf{Code Readability}: Maintain clear and readable code through meaningful variable names and structured logic.
    \index{Code Readability}
    
    \item \textbf{Test Extensively}: Implement a wide range of test cases, including overlapping, non-overlapping, and edge-touching rectangles, to ensure robustness.
    \index{Extensive Testing}
    
    \item \textbf{Understand Axis-Aligned Constraints}: Recognize that axis-aligned rectangles simplify overlap detection compared to rotated rectangles.
    \index{Axis-Aligned Constraints}
    
    \item \textbf{Simplify Logical Conditions}: Combine multiple conditions logically to streamline the overlap detection process.
    \index{Logical Conditions}
\end{itemize}

\section*{Corner and Special Cases to Test When Writing the Code}

When implementing the solution for the \textbf{Rectangle Overlap} problem, it is crucial to consider and rigorously test various edge cases to ensure robustness and correctness:

\begin{itemize}
    \item \textbf{No Overlap}: Rectangles are completely separate.
    \index{No Overlap}
    
    \item \textbf{Partial Overlap}: Rectangles overlap in one or more regions.
    \index{Partial Overlap}
    
    \item \textbf{Edge Touching}: Rectangles touch exactly at one edge without overlapping.
    \index{Edge Touching}
    
    \item \textbf{Corner Touching}: Rectangles touch exactly at one corner without overlapping.
    \index{Corner Touching}
    
    \item \textbf{One Rectangle Inside Another}: One rectangle is entirely within the other.
    \index{Rectangle Inside}
    
    \item \textbf{Identical Rectangles}: Both rectangles have the same coordinates.
    \index{Identical Rectangles}
    
    \item \textbf{Degenerate Rectangles}: Rectangles with zero area (e.g., \(x1 = x2\) or \(y1 = y2\)).
    \index{Degenerate Rectangles}
    
    \item \textbf{Large Coordinates}: Rectangles with very large coordinate values to test performance and integer handling.
    \index{Large Coordinates}
    
    \item \textbf{Negative Coordinates}: Rectangles positioned in negative coordinate space.
    \index{Negative Coordinates}
    
    \item \textbf{Mixed Overlapping Scenarios}: Combinations of the above cases to ensure comprehensive coverage.
    \index{Mixed Overlapping Scenarios}
    
    \item \textbf{Minimum and Maximum Bounds}: Rectangles at the minimum and maximum limits of the coordinate range.
    \index{Minimum and Maximum Bounds}
\end{itemize}

\section*{Implementation Considerations}

When implementing the \texttt{isRectangleOverlap} function, keep in mind the following considerations to ensure robustness and efficiency:

\begin{itemize}
    \item \textbf{Data Type Selection}: Use appropriate data types that can handle the range of input values without overflow or underflow.
    \index{Data Type Selection}
    
    \item \textbf{Optimizing Comparisons}: Structure logical conditions to short-circuit evaluations as soon as a non-overlapping condition is met.
    \index{Optimizing Comparisons}
    
    \item \textbf{Language-Specific Constraints}: Be aware of how the programming language handles integer division and comparisons.
    \index{Language-Specific Constraints}
    
    \item \textbf{Avoiding Redundant Calculations}: Ensure that each comparison contributes towards determining overlap without unnecessary repetitions.
    \index{Avoiding Redundant Calculations}
    
    \item \textbf{Code Readability and Documentation}: Maintain clear and readable code through meaningful variable names and comprehensive comments to facilitate understanding and maintenance.
    \index{Code Readability}
    
    \item \textbf{Edge Case Handling}: Implement checks for edge cases to prevent incorrect results or runtime errors.
    \index{Edge Case Handling}
    
    \item \textbf{Testing and Validation}: Develop a comprehensive suite of test cases that cover all possible scenarios, including edge cases, to validate the correctness and efficiency of the implementation.
    \index{Testing and Validation}
    
    \item \textbf{Scalability}: Design the algorithm to scale efficiently with increasing input sizes, maintaining performance and resource utilization.
    \index{Scalability}
    
    \item \textbf{Using Helper Functions}: Consider creating helper functions for repetitive tasks, such as extracting and comparing coordinates, to enhance modularity and reusability.
    \index{Helper Functions}
    
    \item \textbf{Consistent Naming Conventions}: Use consistent and descriptive naming conventions for variables to improve code clarity.
    \index{Naming Conventions}
    
    \item \textbf{Handling Floating-Point Coordinates}: Although the problem specifies integer coordinates, ensure that the implementation can handle floating-point numbers if needed in extended scenarios.
    \index{Floating-Point Coordinates}
    
    \item \textbf{Avoiding Floating-Point Precision Issues}: Since all coordinates are integers, floating-point precision is not a concern, simplifying the implementation.
    \index{Floating-Point Precision}
    
    \item \textbf{Implementing Unit Tests}: Develop unit tests for each logical condition to ensure that all scenarios are correctly handled.
    \index{Unit Tests}
    
    \item \textbf{Error Handling}: Incorporate error handling to manage invalid inputs gracefully.
    \index{Error Handling}
\end{itemize}

\section*{Conclusion}

The \textbf{Rectangle Overlap} problem exemplifies the application of fundamental geometric principles and conditional logic to solve spatial challenges efficiently. By leveraging simple coordinate comparisons, the algorithm achieves optimal time and space complexities, making it highly suitable for real-time applications such as collision detection in gaming, layout planning in graphics, and spatial data analysis. Understanding and implementing such techniques not only enhances problem-solving skills but also provides a foundation for tackling more complex Computational Geometry problems involving varied geometric shapes and interactions.

\printindex

% % filename: rectangle_overlap.tex

\problemsection{Rectangle Overlap}
\label{chap:Rectangle_Overlap}
\marginnote{\href{https://leetcode.com/problems/rectangle-overlap/}{[LeetCode Link]}\index{LeetCode}}
\marginnote{\href{https://www.geeksforgeeks.org/check-if-two-rectangles-overlap/}{[GeeksForGeeks Link]}\index{GeeksForGeeks}}
\marginnote{\href{https://www.interviewbit.com/problems/rectangle-overlap/}{[InterviewBit Link]}\index{InterviewBit}}
\marginnote{\href{https://app.codesignal.com/challenges/rectangle-overlap}{[CodeSignal Link]}\index{CodeSignal}}
\marginnote{\href{https://www.codewars.com/kata/rectangle-overlap/train/python}{[Codewars Link]}\index{Codewars}}

The \textbf{Rectangle Overlap} problem is a fundamental challenge in Computational Geometry that involves determining whether two axis-aligned rectangles overlap. This problem tests one's ability to understand geometric properties, implement conditional logic, and optimize for efficient computation. Mastery of this problem is essential for applications in computer graphics, collision detection, and spatial data analysis.

\section*{Problem Statement}

Given two axis-aligned rectangles in a 2D plane, determine if they overlap. Each rectangle is defined by its bottom-left and top-right coordinates.

A rectangle is represented as a list of four integers \([x1, y1, x2, y2]\), where \((x1, y1)\) are the coordinates of the bottom-left corner, and \((x2, y2)\) are the coordinates of the top-right corner.

\textbf{Function signature in Python:}
\begin{lstlisting}[language=Python]
def isRectangleOverlap(rec1: List[int], rec2: List[int]) -> bool:
\end{lstlisting}

\section*{Examples}

\textbf{Example 1:}

\begin{verbatim}
Input: rec1 = [0,0,2,2], rec2 = [1,1,3,3]
Output: True
Explanation: The rectangles overlap in the area defined by [1,1,2,2].
\end{verbatim}

\textbf{Example 2:}

\begin{verbatim}
Input: rec1 = [0,0,1,1], rec2 = [1,0,2,1]
Output: False
Explanation: The rectangles touch at the edge but do not overlap.
\end{verbatim}

\textbf{Example 3:}

\begin{verbatim}
Input: rec1 = [0,0,1,1], rec2 = [2,2,3,3]
Output: False
Explanation: The rectangles are completely separate.
\end{verbatim}

\textbf{Example 4:}

\begin{verbatim}
Input: rec1 = [0,0,5,5], rec2 = [3,3,7,7]
Output: True
Explanation: The rectangles overlap in the area defined by [3,3,5,5].
\end{verbatim}

\textbf{Example 5:}

\begin{verbatim}
Input: rec1 = [0,0,0,0], rec2 = [0,0,0,0]
Output: False
Explanation: Both rectangles are degenerate points.
\end{verbatim}

\textbf{Constraints:}

\begin{itemize}
    \item All coordinates are integers in the range \([-10^9, 10^9]\).
    \item For each rectangle, \(x1 < x2\) and \(y1 < y2\).
\end{itemize}

LeetCode link: \href{https://leetcode.com/problems/rectangle-overlap/}{Rectangle Overlap}\index{LeetCode}

\section*{Algorithmic Approach}

To determine whether two axis-aligned rectangles overlap, we can use the following logical conditions:

1. **Non-Overlap Conditions:**
   - One rectangle is to the left of the other.
   - One rectangle is above the other.

2. **Overlap Condition:**
   - If neither of the non-overlap conditions is true, the rectangles must overlap.

\subsection*{Steps:}

1. **Extract Coordinates:**
   - For both rectangles, extract the bottom-left and top-right coordinates.

2. **Check Non-Overlap Conditions:**
   - If the right side of the first rectangle is less than or equal to the left side of the second rectangle, they do not overlap.
   - If the left side of the first rectangle is greater than or equal to the right side of the second rectangle, they do not overlap.
   - If the top side of the first rectangle is less than or equal to the bottom side of the second rectangle, they do not overlap.
   - If the bottom side of the first rectangle is greater than or equal to the top side of the second rectangle, they do not overlap.

3. **Determine Overlap:**
   - If none of the non-overlap conditions are met, the rectangles overlap.

\marginnote{This approach provides an efficient \(O(1)\) time complexity solution by leveraging simple geometric comparisons.}

\section*{Complexities}

\begin{itemize}
    \item \textbf{Time Complexity:} \(O(1)\). The algorithm performs a constant number of comparisons regardless of input size.
    
    \item \textbf{Space Complexity:} \(O(1)\). Only a fixed amount of extra space is used for variables.
\end{itemize}

\section*{Python Implementation}

\marginnote{Implementing the overlap check using coordinate comparisons ensures an optimal and straightforward solution.}

Below is the complete Python code implementing the \texttt{isRectangleOverlap} function:

\begin{fullwidth}
\begin{lstlisting}[language=Python]
from typing import List

class Solution:
    def isRectangleOverlap(self, rec1: List[int], rec2: List[int]) -> bool:
        # Extract coordinates
        left1, bottom1, right1, top1 = rec1
        left2, bottom2, right2, top2 = rec2
        
        # Check non-overlapping conditions
        if right1 <= left2 or right2 <= left1:
            return False
        if top1 <= bottom2 or top2 <= bottom1:
            return False
        
        # If none of the above, rectangles overlap
        return True

# Example usage:
solution = Solution()
print(solution.isRectangleOverlap([0,0,2,2], [1,1,3,3]))  # Output: True
print(solution.isRectangleOverlap([0,0,1,1], [1,0,2,1]))  # Output: False
print(solution.isRectangleOverlap([0,0,1,1], [2,2,3,3]))  # Output: False
print(solution.isRectangleOverlap([0,0,5,5], [3,3,7,7]))  # Output: True
print(solution.isRectangleOverlap([0,0,0,0], [0,0,0,0]))  # Output: False
\end{lstlisting}
\end{fullwidth}

This implementation efficiently checks for overlap by comparing the coordinates of the two rectangles. If any of the non-overlapping conditions are met, it returns \texttt{False}; otherwise, it returns \texttt{True}.

\section*{Explanation}

The \texttt{isRectangleOverlap} function determines whether two axis-aligned rectangles overlap by comparing their respective coordinates. Here's a detailed breakdown of the implementation:

\subsection*{1. Extract Coordinates}

\begin{itemize}
    \item For each rectangle, extract the left (\(x1\)), bottom (\(y1\)), right (\(x2\)), and top (\(y2\)) coordinates.
    \item This simplifies the comparison process by providing clear variables representing each side of the rectangles.
\end{itemize}

\subsection*{2. Check Non-Overlap Conditions}

\begin{itemize}
    \item **Horizontal Separation:**
    \begin{itemize}
        \item If the right side of the first rectangle (\(right1\)) is less than or equal to the left side of the second rectangle (\(left2\)), there is no horizontal overlap.
        \item Similarly, if the right side of the second rectangle (\(right2\)) is less than or equal to the left side of the first rectangle (\(left1\)), there is no horizontal overlap.
    \end{itemize}
    
    \item **Vertical Separation:**
    \begin{itemize}
        \item If the top side of the first rectangle (\(top1\)) is less than or equal to the bottom side of the second rectangle (\(bottom2\)), there is no vertical overlap.
        \item Similarly, if the top side of the second rectangle (\(top2\)) is less than or equal to the bottom side of the first rectangle (\(bottom1\)), there is no vertical overlap.
    \end{itemize}
    
    \item If any of these non-overlapping conditions are true, the rectangles do not overlap, and the function returns \texttt{False}.
\end{itemize}

\subsection*{3. Determine Overlap}

\begin{itemize}
    \item If none of the non-overlapping conditions are met, it implies that the rectangles overlap both horizontally and vertically.
    \item The function returns \texttt{True} in this case.
\end{itemize}

\subsection*{4. Example Walkthrough}

Consider the first example:
\begin{verbatim}
Input: rec1 = [0,0,2,2], rec2 = [1,1,3,3]
Output: True
\end{verbatim}

\begin{enumerate}
    \item Extract coordinates:
    \begin{itemize}
        \item rec1: left1 = 0, bottom1 = 0, right1 = 2, top1 = 2
        \item rec2: left2 = 1, bottom2 = 1, right2 = 3, top2 = 3
    \end{itemize}
    
    \item Check non-overlap conditions:
    \begin{itemize}
        \item \(right1 = 2\) is not less than or equal to \(left2 = 1\)
        \item \(right2 = 3\) is not less than or equal to \(left1 = 0\)
        \item \(top1 = 2\) is not less than or equal to \(bottom2 = 1\)
        \item \(top2 = 3\) is not less than or equal to \(bottom1 = 0\)
    \end{itemize}
    
    \item Since none of the non-overlapping conditions are met, the rectangles overlap.
\end{enumerate}

Thus, the function correctly returns \texttt{True}.

\section*{Why This Approach}

This approach is chosen for its simplicity and efficiency. By leveraging direct coordinate comparisons, the algorithm achieves constant time complexity without the need for complex data structures or iterative processes. It effectively handles all possible scenarios of rectangle positioning, ensuring accurate detection of overlaps.

\section*{Alternative Approaches}

\subsection*{1. Separating Axis Theorem (SAT)}

The Separating Axis Theorem is a more generalized method for detecting overlaps between convex shapes. While it is not necessary for axis-aligned rectangles, understanding SAT can be beneficial for more complex geometric problems.

\begin{lstlisting}[language=Python]
def isRectangleOverlap(rec1: List[int], rec2: List[int]) -> bool:
    # Using SAT for axis-aligned rectangles
    return not (rec1[2] <= rec2[0] or rec1[0] >= rec2[2] or
                rec1[3] <= rec2[1] or rec1[1] >= rec2[3])
\end{lstlisting}

\textbf{Note}: This implementation is functionally identical to the primary approach but leverages a more generalized geometric theorem.

\subsection*{2. Area-Based Approach}

Calculate the overlapping area between the two rectangles. If the overlapping area is positive, the rectangles overlap.

\begin{lstlisting}[language=Python]
def isRectangleOverlap(rec1: List[int], rec2: List[int]) -> bool:
    # Calculate overlap in x and y dimensions
    x_overlap = min(rec1[2], rec2[2]) - max(rec1[0], rec2[0])
    y_overlap = min(rec1[3], rec2[3]) - max(rec1[1], rec2[1])
    
    # Overlap exists if both overlaps are positive
    return x_overlap > 0 and y_overlap > 0
\end{lstlisting}

\textbf{Complexities:}
\begin{itemize}
    \item \textbf{Time Complexity:} \(O(1)\)
    \item \textbf{Space Complexity:} \(O(1)\)
\end{itemize}

\subsection*{3. Using Rectangles Intersection Function}

Utilize built-in or library functions that handle geometric intersections.

\begin{lstlisting}[language=Python]
from shapely.geometry import box

def isRectangleOverlap(rec1: List[int], rec2: List[int]) -> bool:
    rectangle1 = box(rec1[0], rec1[1], rec1[2], rec1[3])
    rectangle2 = box(rec2[0], rec2[1], rec2[2], rec2[3])
    return rectangle1.intersects(rectangle2) and not rectangle1.touches(rectangle2)
\end{lstlisting}

\textbf{Note}: This approach requires the \texttt{shapely} library and is more suitable for complex geometric operations.

\section*{Similar Problems to This One}

Several problems revolve around geometric overlap, intersection detection, and spatial reasoning, utilizing similar algorithmic strategies:

\begin{itemize}
    \item \textbf{Interval Overlap}: Determine if two intervals on a line overlap.
    \item \textbf{Circle Overlap}: Determine if two circles overlap based on their radii and centers.
    \item \textbf{Polygon Overlap}: Determine if two polygons overlap using algorithms like SAT.
    \item \textbf{Closest Pair of Points}: Find the closest pair of points in a set.
    \item \textbf{Convex Hull}: Compute the convex hull of a set of points.
    \item \textbf{Intersection of Lines}: Find the intersection point of two lines.
    \item \textbf{Point Inside Polygon}: Determine if a point lies inside a given polygon.
\end{itemize}

These problems reinforce the concepts of spatial reasoning, geometric property analysis, and efficient algorithm design in various contexts.

\section*{Things to Keep in Mind and Tricks}

When working with the \textbf{Rectangle Overlap} problem, consider the following tips and best practices to enhance efficiency and correctness:

\begin{itemize}
    \item \textbf{Understand Geometric Relationships}: Grasp the positional relationships between rectangles to simplify overlap detection.
    \index{Geometric Relationships}
    
    \item \textbf{Leverage Coordinate Comparisons}: Use direct comparisons of rectangle coordinates to determine spatial relationships.
    \index{Coordinate Comparisons}
    
    \item \textbf{Handle Edge Cases}: Consider cases where rectangles touch at edges or corners without overlapping.
    \index{Edge Cases}
    
    \item \textbf{Optimize for Efficiency}: Aim for a constant time \(O(1)\) solution by avoiding unnecessary computations or iterations.
    \index{Efficiency Optimization}
    
    \item \textbf{Avoid Floating-Point Precision Issues}: Since all coordinates are integers, floating-point precision is not a concern, simplifying the implementation.
    \index{Floating-Point Precision}
    
    \item \textbf{Use Helper Functions}: Create helper functions to encapsulate repetitive tasks, such as extracting coordinates or checking specific conditions.
    \index{Helper Functions}
    
    \item \textbf{Code Readability}: Maintain clear and readable code through meaningful variable names and structured logic.
    \index{Code Readability}
    
    \item \textbf{Test Extensively}: Implement a wide range of test cases, including overlapping, non-overlapping, and edge-touching rectangles, to ensure robustness.
    \index{Extensive Testing}
    
    \item \textbf{Understand Axis-Aligned Constraints}: Recognize that axis-aligned rectangles simplify overlap detection compared to rotated rectangles.
    \index{Axis-Aligned Constraints}
    
    \item \textbf{Simplify Logical Conditions}: Combine multiple conditions logically to streamline the overlap detection process.
    \index{Logical Conditions}
\end{itemize}

\section*{Corner and Special Cases to Test When Writing the Code}

When implementing the solution for the \textbf{Rectangle Overlap} problem, it is crucial to consider and rigorously test various edge cases to ensure robustness and correctness:

\begin{itemize}
    \item \textbf{No Overlap}: Rectangles are completely separate.
    \index{No Overlap}
    
    \item \textbf{Partial Overlap}: Rectangles overlap in one or more regions.
    \index{Partial Overlap}
    
    \item \textbf{Edge Touching}: Rectangles touch exactly at one edge without overlapping.
    \index{Edge Touching}
    
    \item \textbf{Corner Touching}: Rectangles touch exactly at one corner without overlapping.
    \index{Corner Touching}
    
    \item \textbf{One Rectangle Inside Another}: One rectangle is entirely within the other.
    \index{Rectangle Inside}
    
    \item \textbf{Identical Rectangles}: Both rectangles have the same coordinates.
    \index{Identical Rectangles}
    
    \item \textbf{Degenerate Rectangles}: Rectangles with zero area (e.g., \(x1 = x2\) or \(y1 = y2\)).
    \index{Degenerate Rectangles}
    
    \item \textbf{Large Coordinates}: Rectangles with very large coordinate values to test performance and integer handling.
    \index{Large Coordinates}
    
    \item \textbf{Negative Coordinates}: Rectangles positioned in negative coordinate space.
    \index{Negative Coordinates}
    
    \item \textbf{Mixed Overlapping Scenarios}: Combinations of the above cases to ensure comprehensive coverage.
    \index{Mixed Overlapping Scenarios}
    
    \item \textbf{Minimum and Maximum Bounds}: Rectangles at the minimum and maximum limits of the coordinate range.
    \index{Minimum and Maximum Bounds}
\end{itemize}

\section*{Implementation Considerations}

When implementing the \texttt{isRectangleOverlap} function, keep in mind the following considerations to ensure robustness and efficiency:

\begin{itemize}
    \item \textbf{Data Type Selection}: Use appropriate data types that can handle the range of input values without overflow or underflow.
    \index{Data Type Selection}
    
    \item \textbf{Optimizing Comparisons}: Structure logical conditions to short-circuit evaluations as soon as a non-overlapping condition is met.
    \index{Optimizing Comparisons}
    
    \item \textbf{Language-Specific Constraints}: Be aware of how the programming language handles integer division and comparisons.
    \index{Language-Specific Constraints}
    
    \item \textbf{Avoiding Redundant Calculations}: Ensure that each comparison contributes towards determining overlap without unnecessary repetitions.
    \index{Avoiding Redundant Calculations}
    
    \item \textbf{Code Readability and Documentation}: Maintain clear and readable code through meaningful variable names and comprehensive comments to facilitate understanding and maintenance.
    \index{Code Readability}
    
    \item \textbf{Edge Case Handling}: Implement checks for edge cases to prevent incorrect results or runtime errors.
    \index{Edge Case Handling}
    
    \item \textbf{Testing and Validation}: Develop a comprehensive suite of test cases that cover all possible scenarios, including edge cases, to validate the correctness and efficiency of the implementation.
    \index{Testing and Validation}
    
    \item \textbf{Scalability}: Design the algorithm to scale efficiently with increasing input sizes, maintaining performance and resource utilization.
    \index{Scalability}
    
    \item \textbf{Using Helper Functions}: Consider creating helper functions for repetitive tasks, such as extracting and comparing coordinates, to enhance modularity and reusability.
    \index{Helper Functions}
    
    \item \textbf{Consistent Naming Conventions}: Use consistent and descriptive naming conventions for variables to improve code clarity.
    \index{Naming Conventions}
    
    \item \textbf{Handling Floating-Point Coordinates}: Although the problem specifies integer coordinates, ensure that the implementation can handle floating-point numbers if needed in extended scenarios.
    \index{Floating-Point Coordinates}
    
    \item \textbf{Avoiding Floating-Point Precision Issues}: Since all coordinates are integers, floating-point precision is not a concern, simplifying the implementation.
    \index{Floating-Point Precision}
    
    \item \textbf{Implementing Unit Tests}: Develop unit tests for each logical condition to ensure that all scenarios are correctly handled.
    \index{Unit Tests}
    
    \item \textbf{Error Handling}: Incorporate error handling to manage invalid inputs gracefully.
    \index{Error Handling}
\end{itemize}

\section*{Conclusion}

The \textbf{Rectangle Overlap} problem exemplifies the application of fundamental geometric principles and conditional logic to solve spatial challenges efficiently. By leveraging simple coordinate comparisons, the algorithm achieves optimal time and space complexities, making it highly suitable for real-time applications such as collision detection in gaming, layout planning in graphics, and spatial data analysis. Understanding and implementing such techniques not only enhances problem-solving skills but also provides a foundation for tackling more complex Computational Geometry problems involving varied geometric shapes and interactions.

\printindex

% \input{sections/rectangle_overlap}
% \input{sections/rectangle_area}
% \input{sections/k_closest_points_to_origin}
% \input{sections/the_skyline_problem}
% % filename: rectangle_area.tex

\problemsection{Rectangle Area}
\label{chap:Rectangle_Area}
\marginnote{\href{https://leetcode.com/problems/rectangle-area/}{[LeetCode Link]}\index{LeetCode}}
\marginnote{\href{https://www.geeksforgeeks.org/find-area-two-overlapping-rectangles/}{[GeeksForGeeks Link]}\index{GeeksForGeeks}}
\marginnote{\href{https://www.interviewbit.com/problems/rectangle-area/}{[InterviewBit Link]}\index{InterviewBit}}
\marginnote{\href{https://app.codesignal.com/challenges/rectangle-area}{[CodeSignal Link]}\index{CodeSignal}}
\marginnote{\href{https://www.codewars.com/kata/rectangle-area/train/python}{[Codewars Link]}\index{Codewars}}

The \textbf{Rectangle Area} problem is a classic Computational Geometry challenge that involves calculating the total area covered by two axis-aligned rectangles in a 2D plane. This problem tests one's ability to perform geometric calculations, handle overlapping scenarios, and implement efficient algorithms. Mastery of this problem is essential for applications in computer graphics, spatial analysis, and computational modeling.

\section*{Problem Statement}

Given two axis-aligned rectangles in a 2D plane, compute the total area covered by the two rectangles. The area covered by the overlapping region should be counted only once.

Each rectangle is represented as a list of four integers \([x1, y1, x2, y2]\), where \((x1, y1)\) are the coordinates of the bottom-left corner, and \((x2, y2)\) are the coordinates of the top-right corner.

\textbf{Function signature in Python:}
\begin{lstlisting}[language=Python]
def computeArea(A: List[int], B: List[int]) -> int:
\end{lstlisting}

\section*{Examples}

\textbf{Example 1:}

\begin{verbatim}
Input: A = [-3,0,3,4], B = [0,-1,9,2]
Output: 45
Explanation:
Area of A = (3 - (-3)) * (4 - 0) = 6 * 4 = 24
Area of B = (9 - 0) * (2 - (-1)) = 9 * 3 = 27
Overlapping Area = (3 - 0) * (2 - 0) = 3 * 2 = 6
Total Area = 24 + 27 - 6 = 45
\end{verbatim}

\textbf{Example 2:}

\begin{verbatim}
Input: A = [0,0,0,0], B = [0,0,0,0]
Output: 0
Explanation:
Both rectangles are degenerate points with zero area.
\end{verbatim}

\textbf{Example 3:}

\begin{verbatim}
Input: A = [0,0,2,2], B = [1,1,3,3]
Output: 7
Explanation:
Area of A = 4
Area of B = 4
Overlapping Area = 1
Total Area = 4 + 4 - 1 = 7
\end{verbatim}

\textbf{Example 4:}

\begin{verbatim}
Input: A = [0,0,1,1], B = [1,0,2,1]
Output: 2
Explanation:
Rectangles touch at the edge but do not overlap.
Area of A = 1
Area of B = 1
Overlapping Area = 0
Total Area = 1 + 1 = 2
\end{verbatim}

\textbf{Constraints:}

\begin{itemize}
    \item All coordinates are integers in the range \([-10^9, 10^9]\).
    \item For each rectangle, \(x1 < x2\) and \(y1 < y2\).
\end{itemize}

LeetCode link: \href{https://leetcode.com/problems/rectangle-area/}{Rectangle Area}\index{LeetCode}

\section*{Algorithmic Approach}

To compute the total area covered by two axis-aligned rectangles, we can follow these steps:

1. **Calculate Individual Areas:**
   - Compute the area of the first rectangle.
   - Compute the area of the second rectangle.

2. **Determine Overlapping Area:**
   - Calculate the coordinates of the overlapping rectangle, if any.
   - If the rectangles overlap, compute the area of the overlapping region.

3. **Compute Total Area:**
   - Sum the individual areas and subtract the overlapping area to avoid double-counting.

\marginnote{This approach ensures accurate area calculation by handling overlapping regions appropriately.}

\section*{Complexities}

\begin{itemize}
    \item \textbf{Time Complexity:} \(O(1)\). The algorithm performs a constant number of calculations.
    
    \item \textbf{Space Complexity:} \(O(1)\). Only a fixed amount of extra space is used for variables.
\end{itemize}

\section*{Python Implementation}

\marginnote{Implementing the area calculation with overlap consideration ensures an accurate and efficient solution.}

Below is the complete Python code implementing the \texttt{computeArea} function:

\begin{fullwidth}
\begin{lstlisting}[language=Python]
from typing import List

class Solution:
    def computeArea(self, A: List[int], B: List[int]) -> int:
        # Calculate area of rectangle A
        areaA = (A[2] - A[0]) * (A[3] - A[1])
        
        # Calculate area of rectangle B
        areaB = (B[2] - B[0]) * (B[3] - B[1])
        
        # Determine overlap coordinates
        overlap_x1 = max(A[0], B[0])
        overlap_y1 = max(A[1], B[1])
        overlap_x2 = min(A[2], B[2])
        overlap_y2 = min(A[3], B[3])
        
        # Calculate overlapping area
        overlap_width = overlap_x2 - overlap_x1
        overlap_height = overlap_y2 - overlap_y1
        overlap_area = 0
        if overlap_width > 0 and overlap_height > 0:
            overlap_area = overlap_width * overlap_height
        
        # Total area is sum of individual areas minus overlapping area
        total_area = areaA + areaB - overlap_area
        return total_area

# Example usage:
solution = Solution()
print(solution.computeArea([-3,0,3,4], [0,-1,9,2]))  # Output: 45
print(solution.computeArea([0,0,0,0], [0,0,0,0]))    # Output: 0
print(solution.computeArea([0,0,2,2], [1,1,3,3]))    # Output: 7
print(solution.computeArea([0,0,1,1], [1,0,2,1]))    # Output: 2
\end{lstlisting}
\end{fullwidth}

This implementation accurately computes the total area covered by two rectangles by accounting for any overlapping regions. It ensures that the overlapping area is not double-counted.

\section*{Explanation}

The \texttt{computeArea} function calculates the combined area of two axis-aligned rectangles by following these steps:

\subsection*{1. Calculate Individual Areas}

\begin{itemize}
    \item **Rectangle A:**
    \begin{itemize}
        \item Width: \(A[2] - A[0]\)
        \item Height: \(A[3] - A[1]\)
        \item Area: Width \(\times\) Height
    \end{itemize}
    
    \item **Rectangle B:**
    \begin{itemize}
        \item Width: \(B[2] - B[0]\)
        \item Height: \(B[3] - B[1]\)
        \item Area: Width \(\times\) Height
    \end{itemize}
\end{itemize}

\subsection*{2. Determine Overlapping Area}

\begin{itemize}
    \item **Overlap Coordinates:**
    \begin{itemize}
        \item Left (x-coordinate): \(\text{max}(A[0], B[0])\)
        \item Bottom (y-coordinate): \(\text{max}(A[1], B[1])\)
        \item Right (x-coordinate): \(\text{min}(A[2], B[2])\)
        \item Top (y-coordinate): \(\text{min}(A[3], B[3])\)
    \end{itemize}
    
    \item **Overlap Dimensions:**
    \begin{itemize}
        \item Width: \(\text{overlap\_x2} - \text{overlap\_x1}\)
        \item Height: \(\text{overlap\_y2} - \text{overlap\_y1}\)
    \end{itemize}
    
    \item **Overlap Area:**
    \begin{itemize}
        \item If both width and height are positive, the rectangles overlap, and the overlapping area is their product.
        \item Otherwise, there is no overlap, and the overlapping area is zero.
    \end{itemize}
\end{itemize}

\subsection*{3. Compute Total Area}

\begin{itemize}
    \item Total Area = Area of Rectangle A + Area of Rectangle B - Overlapping Area
\end{itemize}

\subsection*{4. Example Walkthrough}

Consider the first example:
\begin{verbatim}
Input: A = [-3,0,3,4], B = [0,-1,9,2]
Output: 45
\end{verbatim}

\begin{enumerate}
    \item **Calculate Areas:**
    \begin{itemize}
        \item Area of A = (3 - (-3)) * (4 - 0) = 6 * 4 = 24
        \item Area of B = (9 - 0) * (2 - (-1)) = 9 * 3 = 27
    \end{itemize}
    
    \item **Determine Overlap:**
    \begin{itemize}
        \item overlap\_x1 = max(-3, 0) = 0
        \item overlap\_y1 = max(0, -1) = 0
        \item overlap\_x2 = min(3, 9) = 3
        \item overlap\_y2 = min(4, 2) = 2
        \item overlap\_width = 3 - 0 = 3
        \item overlap\_height = 2 - 0 = 2
        \item overlap\_area = 3 * 2 = 6
    \end{itemize}
    
    \item **Compute Total Area:**
    \begin{itemize}
        \item Total Area = 24 + 27 - 6 = 45
    \end{itemize}
\end{enumerate}

Thus, the function correctly returns \texttt{45}.

\section*{Why This Approach}

This approach is chosen for its straightforwardness and optimal efficiency. By directly calculating the individual areas and intelligently handling the overlapping region, the algorithm ensures accurate results without unnecessary computations. Its constant time complexity makes it highly efficient, even for large coordinate values.

\section*{Alternative Approaches}

\subsection*{1. Using Intersection Dimensions}

Instead of separately calculating areas, directly compute the dimensions of the overlapping region and subtract it from the sum of individual areas.

\begin{lstlisting}[language=Python]
def computeArea(A: List[int], B: List[int]) -> int:
    # Sum of individual areas
    area = (A[2] - A[0]) * (A[3] - A[1]) + (B[2] - B[0]) * (B[3] - B[1])
    
    # Overlapping area
    overlap_width = min(A[2], B[2]) - max(A[0], B[0])
    overlap_height = min(A[3], B[3]) - max(A[1], B[1])
    
    if overlap_width > 0 and overlap_height > 0:
        area -= overlap_width * overlap_height
    
    return area
\end{lstlisting}

\subsection*{2. Using Geometry Libraries}

Leverage computational geometry libraries to handle area calculations and overlapping detections.

\begin{lstlisting}[language=Python]
from shapely.geometry import box

def computeArea(A: List[int], B: List[int]) -> int:
    rect1 = box(A[0], A[1], A[2], A[3])
    rect2 = box(B[0], B[1], B[2], B[3])
    intersection = rect1.intersection(rect2)
    return int(rect1.area + rect2.area - intersection.area)
\end{lstlisting}

\textbf{Note}: This approach requires the \texttt{shapely} library and is more suitable for complex geometric operations.

\section*{Similar Problems to This One}

Several problems involve calculating areas, handling geometric overlaps, and spatial reasoning, utilizing similar algorithmic strategies:

\begin{itemize}
    \item \textbf{Rectangle Overlap}: Determine if two rectangles overlap.
    \item \textbf{Circle Area Overlap}: Calculate the overlapping area between two circles.
    \item \textbf{Polygon Area}: Compute the area of a given polygon.
    \item \textbf{Union of Rectangles}: Calculate the total area covered by multiple rectangles, accounting for overlaps.
    \item \textbf{Intersection of Lines}: Find the intersection point of two lines.
    \item \textbf{Closest Pair of Points}: Find the closest pair of points in a set.
    \item \textbf{Convex Hull}: Compute the convex hull of a set of points.
    \item \textbf{Point Inside Polygon}: Determine if a point lies inside a given polygon.
\end{itemize}

These problems reinforce concepts of geometric calculations, area computations, and efficient algorithm design in various contexts.

\section*{Things to Keep in Mind and Tricks}

When tackling the \textbf{Rectangle Area} problem, consider the following tips and best practices to enhance efficiency and correctness:

\begin{itemize}
    \item \textbf{Understand Geometric Relationships}: Grasp the positional relationships between rectangles to simplify area calculations.
    \index{Geometric Relationships}
    
    \item \textbf{Leverage Coordinate Comparisons}: Use direct comparisons of rectangle coordinates to determine overlapping regions.
    \index{Coordinate Comparisons}
    
    \item \textbf{Handle Overlapping Scenarios}: Accurately calculate the overlapping area to avoid double-counting.
    \index{Overlapping Scenarios}
    
    \item \textbf{Optimize for Efficiency}: Aim for a constant time \(O(1)\) solution by avoiding unnecessary computations or iterations.
    \index{Efficiency Optimization}
    
    \item \textbf{Avoid Floating-Point Precision Issues}: Since all coordinates are integers, floating-point precision is not a concern, simplifying the implementation.
    \index{Floating-Point Precision}
    
    \item \textbf{Use Helper Functions}: Create helper functions to encapsulate repetitive tasks, such as calculating overlap dimensions or areas.
    \index{Helper Functions}
    
    \item \textbf{Code Readability}: Maintain clear and readable code through meaningful variable names and structured logic.
    \index{Code Readability}
    
    \item \textbf{Test Extensively}: Implement a wide range of test cases, including overlapping, non-overlapping, and edge-touching rectangles, to ensure robustness.
    \index{Extensive Testing}
    
    \item \textbf{Understand Axis-Aligned Constraints}: Recognize that axis-aligned rectangles simplify area calculations compared to rotated rectangles.
    \index{Axis-Aligned Constraints}
    
    \item \textbf{Simplify Logical Conditions}: Combine multiple conditions logically to streamline the area calculation process.
    \index{Logical Conditions}
    
    \item \textbf{Use Absolute Values}: When calculating differences, ensure that the dimensions are positive by using absolute values or proper ordering.
    \index{Absolute Values}
    
    \item \textbf{Consider Edge Cases}: Handle cases where rectangles have zero area or touch at edges/corners without overlapping.
    \index{Edge Cases}
\end{itemize}

\section*{Corner and Special Cases to Test When Writing the Code}

When implementing the solution for the \textbf{Rectangle Area} problem, it is crucial to consider and rigorously test various edge cases to ensure robustness and correctness:

\begin{itemize}
    \item \textbf{No Overlap}: Rectangles are completely separate.
    \index{No Overlap}
    
    \item \textbf{Partial Overlap}: Rectangles overlap in one or more regions.
    \index{Partial Overlap}
    
    \item \textbf{Edge Touching}: Rectangles touch exactly at one edge without overlapping.
    \index{Edge Touching}
    
    \item \textbf{Corner Touching}: Rectangles touch exactly at one corner without overlapping.
    \index{Corner Touching}
    
    \item \textbf{One Rectangle Inside Another}: One rectangle is entirely within the other.
    \index{Rectangle Inside}
    
    \item \textbf{Identical Rectangles}: Both rectangles have the same coordinates.
    \index{Identical Rectangles}
    
    \item \textbf{Degenerate Rectangles}: Rectangles with zero area (e.g., \(x1 = x2\) or \(y1 = y2\)).
    \index{Degenerate Rectangles}
    
    \item \textbf{Large Coordinates}: Rectangles with very large coordinate values to test performance and integer handling.
    \index{Large Coordinates}
    
    \item \textbf{Negative Coordinates}: Rectangles positioned in negative coordinate space.
    \index{Negative Coordinates}
    
    \item \textbf{Mixed Overlapping Scenarios}: Combinations of the above cases to ensure comprehensive coverage.
    \index{Mixed Overlapping Scenarios}
    
    \item \textbf{Minimum and Maximum Bounds}: Rectangles at the minimum and maximum limits of the coordinate range.
    \index{Minimum and Maximum Bounds}
    
    \item \textbf{Sequential Rectangles}: Multiple rectangles placed sequentially without overlapping.
    \index{Sequential Rectangles}
    
    \item \textbf{Multiple Overlaps}: Scenarios where more than two rectangles overlap in different regions.
    \index{Multiple Overlaps}
\end{itemize}

\section*{Implementation Considerations}

When implementing the \texttt{computeArea} function, keep in mind the following considerations to ensure robustness and efficiency:

\begin{itemize}
    \item \textbf{Data Type Selection}: Use appropriate data types that can handle large input values without overflow or underflow.
    \index{Data Type Selection}
    
    \item \textbf{Optimizing Comparisons}: Structure logical conditions to efficiently determine overlap dimensions.
    \index{Optimizing Comparisons}
    
    \item \textbf{Handling Large Inputs}: Design the algorithm to efficiently handle large input sizes without significant performance degradation.
    \index{Handling Large Inputs}
    
    \item \textbf{Language-Specific Constraints}: Be aware of how the programming language handles large integers and arithmetic operations.
    \index{Language-Specific Constraints}
    
    \item \textbf{Avoiding Redundant Calculations}: Ensure that each calculation contributes towards determining the final area without unnecessary repetitions.
    \index{Avoiding Redundant Calculations}
    
    \item \textbf{Code Readability and Documentation}: Maintain clear and readable code through meaningful variable names and comprehensive comments to facilitate understanding and maintenance.
    \index{Code Readability}
    
    \item \textbf{Edge Case Handling}: Implement checks for edge cases to prevent incorrect results or runtime errors.
    \index{Edge Case Handling}
    
    \item \textbf{Testing and Validation}: Develop a comprehensive suite of test cases that cover all possible scenarios, including edge cases, to validate the correctness and efficiency of the implementation.
    \index{Testing and Validation}
    
    \item \textbf{Scalability}: Design the algorithm to scale efficiently with increasing input sizes, maintaining performance and resource utilization.
    \index{Scalability}
    
    \item \textbf{Using Helper Functions}: Consider creating helper functions for repetitive tasks, such as calculating overlap dimensions, to enhance modularity and reusability.
    \index{Helper Functions}
    
    \item \textbf{Consistent Naming Conventions}: Use consistent and descriptive naming conventions for variables to improve code clarity.
    \index{Naming Conventions}
    
    \item \textbf{Implementing Unit Tests}: Develop unit tests for each logical condition to ensure that all scenarios are correctly handled.
    \index{Unit Tests}
    
    \item \textbf{Error Handling}: Incorporate error handling to manage invalid inputs gracefully.
    \index{Error Handling}
\end{itemize}

\section*{Conclusion}

The \textbf{Rectangle Area} problem showcases the application of fundamental geometric principles and efficient algorithm design to compute spatial properties accurately. By systematically calculating individual areas and intelligently handling overlapping regions, the algorithm ensures precise results without redundant computations. Understanding and implementing such techniques not only enhances problem-solving skills but also provides a foundation for tackling more complex Computational Geometry challenges involving multiple geometric entities and intricate spatial relationships.

\printindex

% \input{sections/rectangle_overlap}
% \input{sections/rectangle_area}
% \input{sections/k_closest_points_to_origin}
% \input{sections/the_skyline_problem}
% % filename: k_closest_points_to_origin.tex

\problemsection{K Closest Points to Origin}
\label{chap:K_Closest_Points_to_Origin}
\marginnote{\href{https://leetcode.com/problems/k-closest-points-to-origin/}{[LeetCode Link]}\index{LeetCode}}
\marginnote{\href{https://www.geeksforgeeks.org/find-k-closest-points-origin/}{[GeeksForGeeks Link]}\index{GeeksForGeeks}}
\marginnote{\href{https://www.interviewbit.com/problems/k-closest-points/}{[InterviewBit Link]}\index{InterviewBit}}
\marginnote{\href{https://app.codesignal.com/challenges/k-closest-points-to-origin}{[CodeSignal Link]}\index{CodeSignal}}
\marginnote{\href{https://www.codewars.com/kata/k-closest-points-to-origin/train/python}{[Codewars Link]}\index{Codewars}}

The \textbf{K Closest Points to Origin} problem is a popular algorithmic challenge in Computational Geometry that involves identifying the \(k\) points closest to the origin in a 2D plane. This problem tests one's ability to apply efficient sorting and selection algorithms, understand distance computations, and optimize for performance. Mastery of this problem is essential for applications in spatial data analysis, nearest neighbor searches, and clustering algorithms.

\section*{Problem Statement}

Given an array of points where each point is represented as \([x, y]\) in the 2D plane, and an integer \(k\), return the \(k\) closest points to the origin \((0, 0)\).

The distance between two points \((x_1, y_1)\) and \((x_2, y_2)\) is the Euclidean distance \(\sqrt{(x_1 - x_2)^2 + (y_1 - y_2)^2}\). The origin is \((0, 0)\).

\textbf{Function signature in Python:}
\begin{lstlisting}[language=Python]
def kClosest(points: List[List[int]], K: int) -> List[List[int]]:
\end{lstlisting}

\section*{Examples}

\textbf{Example 1:}

\begin{verbatim}
Input: points = [[1,3],[-2,2]], K = 1
Output: [[-2,2]]
Explanation: 
The distance between (1, 3) and the origin is sqrt(10).
The distance between (-2, 2) and the origin is sqrt(8).
Since sqrt(8) < sqrt(10), (-2, 2) is closer to the origin.
\end{verbatim}

\textbf{Example 2:}

\begin{verbatim}
Input: points = [[3,3],[5,-1],[-2,4]], K = 2
Output: [[3,3],[-2,4]]
Explanation: 
The distances are sqrt(18), sqrt(26), and sqrt(20) respectively.
The two closest points are [3,3] and [-2,4].
\end{verbatim}

\textbf{Example 3:}

\begin{verbatim}
Input: points = [[0,1],[1,0]], K = 2
Output: [[0,1],[1,0]]
Explanation: 
Both points are equally close to the origin.
\end{verbatim}

\textbf{Example 4:}

\begin{verbatim}
Input: points = [[1,0],[0,1]], K = 1
Output: [[1,0]]
Explanation: 
Both points are equally close; returning any one is acceptable.
\end{verbatim}

\textbf{Constraints:}

\begin{itemize}
    \item \(1 \leq K \leq \text{points.length} \leq 10^4\)
    \item \(-10^4 < x_i, y_i < 10^4\)
\end{itemize}

LeetCode link: \href{https://leetcode.com/problems/k-closest-points-to-origin/}{K Closest Points to Origin}\index{LeetCode}

\section*{Algorithmic Approach}

To identify the \(k\) closest points to the origin, several algorithmic strategies can be employed. The most efficient methods aim to reduce the time complexity by avoiding the need to sort the entire list of points.

\subsection*{1. Sorting Based on Distance}

Calculate the Euclidean distance of each point from the origin and sort the points based on these distances. Select the first \(k\) points from the sorted list.

\begin{enumerate}
    \item Compute the distance for each point using the formula \(distance = x^2 + y^2\).
    \item Sort the points based on the computed distances.
    \item Return the first \(k\) points from the sorted list.
\end{enumerate}

\subsection*{2. Max Heap (Priority Queue)}

Use a max heap to maintain the \(k\) closest points. Iterate through each point, add it to the heap, and if the heap size exceeds \(k\), remove the farthest point.

\begin{enumerate}
    \item Initialize a max heap.
    \item For each point, compute its distance and add it to the heap.
    \item If the heap size exceeds \(k\), remove the point with the largest distance.
    \item After processing all points, the heap contains the \(k\) closest points.
\end{enumerate}

\subsection*{3. QuickSelect (Quick Sort Partitioning)}

Utilize the QuickSelect algorithm to find the \(k\) closest points without fully sorting the list.

\begin{enumerate}
    \item Choose a pivot point and partition the list based on distances relative to the pivot.
    \item Recursively apply QuickSelect to the partition containing the \(k\) closest points.
    \item Once the \(k\) closest points are identified, return them.
\end{enumerate}

\marginnote{QuickSelect offers an average time complexity of \(O(n)\), making it highly efficient for large datasets.}

\section*{Complexities}

\begin{itemize}
    \item \textbf{Sorting Based on Distance:}
    \begin{itemize}
        \item \textbf{Time Complexity:} \(O(n \log n)\)
        \item \textbf{Space Complexity:} \(O(n)\)
    \end{itemize}
    
    \item \textbf{Max Heap (Priority Queue):}
    \begin{itemize}
        \item \textbf{Time Complexity:} \(O(n \log k)\)
        \item \textbf{Space Complexity:} \(O(k)\)
    \end{itemize}
    
    \item \textbf{QuickSelect (Quick Sort Partitioning):}
    \begin{itemize}
        \item \textbf{Time Complexity:} Average case \(O(n)\), worst case \(O(n^2)\)
        \item \textbf{Space Complexity:} \(O(1)\) (in-place)
    \end{itemize}
\end{itemize}

\section*{Python Implementation}

\marginnote{Implementing QuickSelect provides an optimal average-case solution with linear time complexity.}

Below is the complete Python code implementing the \texttt{kClosest} function using the QuickSelect approach:

\begin{fullwidth}
\begin{lstlisting}[language=Python]
from typing import List
import random

class Solution:
    def kClosest(self, points: List[List[int]], K: int) -> List[List[int]]:
        def quickselect(left, right, K_smallest):
            if left == right:
                return
            
            # Select a random pivot_index
            pivot_index = random.randint(left, right)
            
            # Partition the array
            pivot_index = partition(left, right, pivot_index)
            
            # The pivot is in its final sorted position
            if K_smallest == pivot_index:
                return
            elif K_smallest < pivot_index:
                quickselect(left, pivot_index - 1, K_smallest)
            else:
                quickselect(pivot_index + 1, right, K_smallest)
        
        def partition(left, right, pivot_index):
            pivot_distance = distance(points[pivot_index])
            # Move pivot to end
            points[pivot_index], points[right] = points[right], points[pivot_index]
            store_index = left
            for i in range(left, right):
                if distance(points[i]) < pivot_distance:
                    points[store_index], points[i] = points[i], points[store_index]
                    store_index += 1
            # Move pivot to its final place
            points[right], points[store_index] = points[store_index], points[right]
            return store_index
        
        def distance(point):
            return point[0] ** 2 + point[1] ** 2
        
        n = len(points)
        quickselect(0, n - 1, K)
        return points[:K]

# Example usage:
solution = Solution()
print(solution.kClosest([[1,3],[-2,2]], 1))            # Output: [[-2,2]]
print(solution.kClosest([[3,3],[5,-1],[-2,4]], 2))     # Output: [[3,3],[-2,4]]
print(solution.kClosest([[0,1],[1,0]], 2))             # Output: [[0,1],[1,0]]
print(solution.kClosest([[1,0],[0,1]], 1))             # Output: [[1,0]] or [[0,1]]
\end{lstlisting}
\end{fullwidth}

This implementation uses the QuickSelect algorithm to efficiently find the \(k\) closest points to the origin without fully sorting the entire list. It ensures optimal performance even with large datasets.

\section*{Explanation}

The \texttt{kClosest} function identifies the \(k\) closest points to the origin using the QuickSelect algorithm. Here's a detailed breakdown of the implementation:

\subsection*{1. Distance Calculation}

\begin{itemize}
    \item The Euclidean distance is calculated as \(distance = x^2 + y^2\). Since we only need relative distances for comparison, the square root is omitted for efficiency.
\end{itemize}

\subsection*{2. QuickSelect Algorithm}

\begin{itemize}
    \item **Pivot Selection:**
    \begin{itemize}
        \item A random pivot is chosen to enhance the average-case performance.
    \end{itemize}
    
    \item **Partitioning:**
    \begin{itemize}
        \item The array is partitioned such that points with distances less than the pivot are moved to the left, and others to the right.
        \item The pivot is placed in its correct sorted position.
    \end{itemize}
    
    \item **Recursive Selection:**
    \begin{itemize}
        \item If the pivot's position matches \(K\), the selection is complete.
        \item Otherwise, recursively apply QuickSelect to the relevant partition.
    \end{itemize}
\end{itemize}

\subsection*{3. Final Selection}

\begin{itemize}
    \item After partitioning, the first \(K\) points in the list are the \(k\) closest points to the origin.
\end{itemize}

\subsection*{4. Example Walkthrough}

Consider the first example:
\begin{verbatim}
Input: points = [[1,3],[-2,2]], K = 1
Output: [[-2,2]]
\end{verbatim}

\begin{enumerate}
    \item **Calculate Distances:**
    \begin{itemize}
        \item [1,3] : \(1^2 + 3^2 = 10\)
        \item [-2,2] : \((-2)^2 + 2^2 = 8\)
    \end{itemize}
    
    \item **QuickSelect Process:**
    \begin{itemize}
        \item Choose a pivot, say [1,3] with distance 10.
        \item Compare and rearrange:
        \begin{itemize}
            \item [-2,2] has a smaller distance (8) and is moved to the left.
        \end{itemize}
        \item After partitioning, the list becomes [[-2,2], [1,3]].
        \item Since \(K = 1\), return the first point: [[-2,2]].
    \end{itemize}
\end{enumerate}

Thus, the function correctly identifies \([-2,2]\) as the closest point to the origin.

\section*{Why This Approach}

The QuickSelect algorithm is chosen for its average-case linear time complexity \(O(n)\), making it highly efficient for large datasets compared to sorting-based methods with \(O(n \log n)\) time complexity. By avoiding the need to sort the entire list, QuickSelect provides an optimal solution for finding the \(k\) closest points.

\section*{Alternative Approaches}

\subsection*{1. Sorting Based on Distance}

Sort all points based on their distances from the origin and select the first \(k\) points.

\begin{lstlisting}[language=Python]
class Solution:
    def kClosest(self, points: List[List[int]], K: int) -> List[List[int]]:
        points.sort(key=lambda P: P[0]**2 + P[1]**2)
        return points[:K]
\end{lstlisting}

\textbf{Complexities:}
\begin{itemize}
    \item \textbf{Time Complexity:} \(O(n \log n)\)
    \item \textbf{Space Complexity:} \(O(1)\)
\end{itemize}

\subsection*{2. Max Heap (Priority Queue)}

Use a max heap to maintain the \(k\) closest points.

\begin{lstlisting}[language=Python]
import heapq

class Solution:
    def kClosest(self, points: List[List[int]], K: int) -> List[List[int]]:
        heap = []
        for (x, y) in points:
            dist = -(x**2 + y**2)  # Max heap using negative distances
            heapq.heappush(heap, (dist, [x, y]))
            if len(heap) > K:
                heapq.heappop(heap)
        return [item[1] for item in heap]
\end{lstlisting}

\textbf{Complexities:}
\begin{itemize}
    \item \textbf{Time Complexity:} \(O(n \log k)\)
    \item \textbf{Space Complexity:} \(O(k)\)
\end{itemize}

\subsection*{3. Using Built-In Functions}

Leverage built-in functions for distance calculation and selection.

\begin{lstlisting}[language=Python]
import math

class Solution:
    def kClosest(self, points: List[List[int]], K: int) -> List[List[int]]:
        points.sort(key=lambda P: math.sqrt(P[0]**2 + P[1]**2))
        return points[:K]
\end{lstlisting}

\textbf{Note}: This method is similar to the sorting approach but uses the actual Euclidean distance.

\section*{Similar Problems to This One}

Several problems involve nearest neighbor searches, spatial data analysis, and efficient selection algorithms, utilizing similar algorithmic strategies:

\begin{itemize}
    \item \textbf{Closest Pair of Points}: Find the closest pair of points in a set.
    \item \textbf{Top K Frequent Elements}: Identify the most frequent elements in a dataset.
    \item \textbf{Kth Largest Element in an Array}: Find the \(k\)-th largest element in an unsorted array.
    \item \textbf{Sliding Window Maximum}: Find the maximum in each sliding window of size \(k\) over an array.
    \item \textbf{Merge K Sorted Lists}: Merge multiple sorted lists into a single sorted list.
    \item \textbf{Find Median from Data Stream}: Continuously find the median of a stream of numbers.
    \item \textbf{Top K Closest Stars}: Find the \(k\) closest stars to Earth based on their distances.
\end{itemize}

These problems reinforce concepts of efficient selection, heap usage, and distance computations in various contexts.

\section*{Things to Keep in Mind and Tricks}

When solving the \textbf{K Closest Points to Origin} problem, consider the following tips and best practices to enhance efficiency and correctness:

\begin{itemize}
    \item \textbf{Understand Distance Calculations}: Grasp the Euclidean distance formula and recognize that the square root can be omitted for comparison purposes.
    \index{Distance Calculations}
    
    \item \textbf{Leverage Efficient Algorithms}: Use QuickSelect or heap-based methods to optimize time complexity, especially for large datasets.
    \index{Efficient Algorithms}
    
    \item \textbf{Handle Ties Appropriately}: Decide how to handle points with identical distances when \(k\) is less than the number of such points.
    \index{Handling Ties}
    
    \item \textbf{Optimize Space Usage}: Choose algorithms that minimize additional space, such as in-place QuickSelect.
    \index{Space Optimization}
    
    \item \textbf{Use Appropriate Data Structures}: Utilize heaps, lists, and helper functions effectively to manage and process data.
    \index{Data Structures}
    
    \item \textbf{Implement Helper Functions}: Create helper functions for distance calculation and partitioning to enhance code modularity.
    \index{Helper Functions}
    
    \item \textbf{Code Readability}: Maintain clear and readable code through meaningful variable names and structured logic.
    \index{Code Readability}
    
    \item \textbf{Test Extensively}: Implement a wide range of test cases, including edge cases like multiple points with the same distance, to ensure robustness.
    \index{Extensive Testing}
    
    \item \textbf{Understand Algorithm Trade-offs}: Recognize the trade-offs between different approaches in terms of time and space complexities.
    \index{Algorithm Trade-offs}
    
    \item \textbf{Use Built-In Sorting Functions}: When using sorting-based approaches, leverage built-in functions for efficiency and simplicity.
    \index{Built-In Sorting}
    
    \item \textbf{Avoid Redundant Calculations}: Ensure that distance calculations are performed only when necessary to optimize performance.
    \index{Avoiding Redundant Calculations}
    
    \item \textbf{Language-Specific Features}: Utilize language-specific features or libraries that can simplify implementation, such as heapq in Python.
    \index{Language-Specific Features}
\end{itemize}

\section*{Corner and Special Cases to Test When Writing the Code}

When implementing the solution for the \textbf{K Closest Points to Origin} problem, it is crucial to consider and rigorously test various edge cases to ensure robustness and correctness:

\begin{itemize}
    \item \textbf{Multiple Points with Same Distance}: Ensure that the algorithm handles multiple points having the same distance from the origin.
    \index{Same Distance Points}
    
    \item \textbf{Points at Origin}: Include points that are exactly at the origin \((0,0)\).
    \index{Points at Origin}
    
    \item \textbf{Negative Coordinates}: Ensure that the algorithm correctly computes distances for points with negative \(x\) or \(y\) coordinates.
    \index{Negative Coordinates}
    
    \item \textbf{Large Coordinates}: Test with points having very large or very small coordinate values to verify integer handling.
    \index{Large Coordinates}
    
    \item \textbf{K Equals Number of Points}: When \(K\) is equal to the number of points, the algorithm should return all points.
    \index{K Equals Number of Points}
    
    \item \textbf{K is One}: Test with \(K = 1\) to ensure the closest point is correctly identified.
    \index{K is One}
    
    \item \textbf{All Points Same}: All points have the same coordinates.
    \index{All Points Same}
    
    \item \textbf{K is Zero}: Although \(K\) is defined to be at least 1, ensure that the algorithm gracefully handles \(K = 0\) if allowed.
    \index{K is Zero}
    
    \item \textbf{Single Point}: Only one point is provided, and \(K = 1\).
    \index{Single Point}
    
    \item \textbf{Mixed Coordinates}: Points with a mix of positive and negative coordinates.
    \index{Mixed Coordinates}
    
    \item \textbf{Points with Zero Distance}: Multiple points at the origin.
    \index{Zero Distance Points}
    
    \item \textbf{Sparse and Dense Points}: Densely packed points and sparsely distributed points.
    \index{Sparse and Dense Points}
    
    \item \textbf{Duplicate Points}: Multiple identical points in the input list.
    \index{Duplicate Points}
    
    \item \textbf{K Greater Than Number of Unique Points}: Ensure that the algorithm handles cases where \(K\) exceeds the number of unique points if applicable.
    \index{K Greater Than Unique Points}
\end{itemize}

\section*{Implementation Considerations}

When implementing the \texttt{kClosest} function, keep in mind the following considerations to ensure robustness and efficiency:

\begin{itemize}
    \item \textbf{Data Type Selection}: Use appropriate data types that can handle large input values without overflow or precision loss.
    \index{Data Type Selection}
    
    \item \textbf{Optimizing Distance Calculations}: Avoid calculating the square root since it is unnecessary for comparison purposes.
    \index{Optimizing Distance Calculations}
    
    \item \textbf{Choosing the Right Algorithm}: Select an algorithm based on the size of the input and the value of \(K\) to optimize time and space complexities.
    \index{Choosing the Right Algorithm}
    
    \item \textbf{Language-Specific Libraries}: Utilize language-specific libraries and functions (e.g., \texttt{heapq} in Python) to simplify implementation and enhance performance.
    \index{Language-Specific Libraries}
    
    \item \textbf{Avoiding Redundant Calculations}: Ensure that each point's distance is calculated only once to optimize performance.
    \index{Avoiding Redundant Calculations}
    
    \item \textbf{Implementing Helper Functions}: Create helper functions for tasks like distance calculation and partitioning to enhance modularity and readability.
    \index{Helper Functions}
    
    \item \textbf{Edge Case Handling}: Implement checks for edge cases to prevent incorrect results or runtime errors.
    \index{Edge Case Handling}
    
    \item \textbf{Testing and Validation}: Develop a comprehensive suite of test cases that cover all possible scenarios, including edge cases, to validate the correctness and efficiency of the implementation.
    \index{Testing and Validation}
    
    \item \textbf{Scalability}: Design the algorithm to scale efficiently with increasing input sizes, maintaining performance and resource utilization.
    \index{Scalability}
    
    \item \textbf{Consistent Naming Conventions}: Use consistent and descriptive naming conventions for variables and functions to improve code clarity.
    \index{Naming Conventions}
    
    \item \textbf{Memory Management}: Ensure that the algorithm manages memory efficiently, especially when dealing with large datasets.
    \index{Memory Management}
    
    \item \textbf{Avoiding Stack Overflow}: If implementing recursive approaches, be mindful of recursion limits and potential stack overflow issues.
    \index{Avoiding Stack Overflow}
    
    \item \textbf{Implementing Iterative Solutions}: Prefer iterative solutions when recursion may lead to increased space complexity or stack overflow.
    \index{Implementing Iterative Solutions}
\end{itemize}

\section*{Conclusion}

The \textbf{K Closest Points to Origin} problem exemplifies the application of efficient selection algorithms and geometric computations to solve spatial challenges effectively. By leveraging QuickSelect or heap-based methods, the algorithm achieves optimal time and space complexities, making it highly suitable for large datasets. Understanding and implementing such techniques not only enhances problem-solving skills but also provides a foundation for tackling more advanced Computational Geometry problems involving nearest neighbor searches, clustering, and spatial data analysis.

\printindex

% \input{sections/rectangle_overlap}
% \input{sections/rectangle_area}
% \input{sections/k_closest_points_to_origin}
% \input{sections/the_skyline_problem}
% % filename: the_skyline_problem.tex

\problemsection{The Skyline Problem}
\label{chap:The_Skyline_Problem}
\marginnote{\href{https://leetcode.com/problems/the-skyline-problem/}{[LeetCode Link]}\index{LeetCode}}
\marginnote{\href{https://www.geeksforgeeks.org/the-skyline-problem/}{[GeeksForGeeks Link]}\index{GeeksForGeeks}}
\marginnote{\href{https://www.interviewbit.com/problems/the-skyline-problem/}{[InterviewBit Link]}\index{InterviewBit}}
\marginnote{\href{https://app.codesignal.com/challenges/the-skyline-problem}{[CodeSignal Link]}\index{CodeSignal}}
\marginnote{\href{https://www.codewars.com/kata/the-skyline-problem/train/python}{[Codewars Link]}\index{Codewars}}

The \textbf{Skyline Problem} is a complex Computational Geometry challenge that involves computing the skyline formed by a collection of buildings in a 2D cityscape. Each building is represented by its left and right x-coordinates and its height. The skyline is defined by a list of "key points" where the height changes. This problem tests one's ability to handle large datasets, implement efficient sweep line algorithms, and manage event-driven processing. Mastery of this problem is essential for applications in computer graphics, urban planning simulations, and geographic information systems (GIS).

\section*{Problem Statement}

You are given a list of buildings in a cityscape. Each building is represented as a triplet \([Li, Ri, Hi]\), where \(Li\) and \(Ri\) are the x-coordinates of the left and right edges of the building, respectively, and \(Hi\) is the height of the building.

The skyline should be represented as a list of key points \([x, y]\) in sorted order by \(x\)-coordinate, where \(y\) is the height of the skyline at that point. The skyline should only include critical points where the height changes.

\textbf{Function signature in Python:}
\begin{lstlisting}[language=Python]
def getSkyline(buildings: List[List[int]]) -> List[List[int]]:
\end{lstlisting}

\section*{Examples}

\textbf{Example 1:}

\begin{verbatim}
Input: buildings = [[2,9,10], [3,7,15], [5,12,12], [15,20,10], [19,24,8]]
Output: [[2,10], [3,15], [7,12], [12,0], [15,10], [20,8], [24,0]]
Explanation:
- At x=2, the first building starts, height=10.
- At x=3, the second building starts, height=15.
- At x=7, the second building ends, the third building is still ongoing, height=12.
- At x=12, the third building ends, height drops to 0.
- At x=15, the fourth building starts, height=10.
- At x=20, the fourth building ends, the fifth building is still ongoing, height=8.
- At x=24, the fifth building ends, height drops to 0.
\end{verbatim}

\textbf{Example 2:}

\begin{verbatim}
Input: buildings = [[0,2,3], [2,5,3]]
Output: [[0,3], [5,0]]
Explanation:
- The two buildings are contiguous and have the same height, so the skyline drops to 0 at x=5.
\end{verbatim}

\textbf{Example 3:}

\begin{verbatim}
Input: buildings = [[1,3,3], [2,4,4], [5,6,1]]
Output: [[1,3], [2,4], [4,0], [5,1], [6,0]]
Explanation:
- At x=1, first building starts, height=3.
- At x=2, second building starts, height=4.
- At x=4, second building ends, height drops to 0.
- At x=5, third building starts, height=1.
- At x=6, third building ends, height drops to 0.
\end{verbatim}

\textbf{Example 4:}

\begin{verbatim}
Input: buildings = [[0,5,0]]
Output: []
Explanation:
- A building with height 0 does not contribute to the skyline.
\end{verbatim}

\textbf{Constraints:}

\begin{itemize}
    \item \(1 \leq \text{buildings.length} \leq 10^4\)
    \item \(0 \leq Li < Ri \leq 10^9\)
    \item \(0 \leq Hi \leq 10^4\)
\end{itemize}

\section*{Algorithmic Approach}

The \textbf{Sweep Line Algorithm} is an efficient method for solving the Skyline Problem. It involves processing events (building start and end points) in sorted order while maintaining a data structure (typically a max heap) to keep track of active buildings. Here's a step-by-step approach:

\subsection*{1. Event Representation}

Transform each building into two events:
\begin{itemize}
    \item **Start Event:** \((Li, -Hi)\) – Negative height indicates a building starts.
    \item **End Event:** \((Ri, Hi)\) – Positive height indicates a building ends.
\end{itemize}

Sorting the events ensures that start events are processed before end events at the same x-coordinate, and taller buildings are processed before shorter ones.

\subsection*{2. Sorting the Events}

Sort all events based on:
\begin{enumerate}
    \item **x-coordinate:** Ascending order.
    \item **Height:**
    \begin{itemize}
        \item For start events, taller buildings come first.
        \item For end events, shorter buildings come first.
    \end{itemize}
\end{enumerate}

\subsection*{3. Processing the Events}

Use a max heap to keep track of active building heights. Iterate through the sorted events:
\begin{enumerate}
    \item **Start Event:**
    \begin{itemize}
        \item Add the building's height to the heap.
    \end{itemize}
    
    \item **End Event:**
    \begin{itemize}
        \item Remove the building's height from the heap.
    \end{itemize}
    
    \item **Determine Current Max Height:**
    \begin{itemize}
        \item The current max height is the top of the heap.
    \end{itemize}
    
    \item **Update Skyline:**
    \begin{itemize}
        \item If the current max height differs from the previous max height, add a new key point \([x, current\_max\_height]\).
    \end{itemize}
\end{enumerate}

\subsection*{4. Finalizing the Skyline}

After processing all events, the accumulated key points represent the skyline.

\marginnote{The Sweep Line Algorithm efficiently handles dynamic changes in active buildings, ensuring accurate skyline construction.}

\section*{Complexities}

\begin{itemize}
    \item \textbf{Time Complexity:} \(O(n \log n)\), where \(n\) is the number of buildings. Sorting the events takes \(O(n \log n)\), and each heap operation takes \(O(\log n)\).
    
    \item \textbf{Space Complexity:} \(O(n)\), due to the storage of events and the heap.
\end{itemize}

\section*{Python Implementation}

\marginnote{Implementing the Sweep Line Algorithm with a max heap ensures an efficient and accurate solution.}

Below is the complete Python code implementing the \texttt{getSkyline} function:

\begin{fullwidth}
\begin{lstlisting}[language=Python]
from typing import List
import heapq

class Solution:
    def getSkyline(self, buildings: List[List[int]]) -> List[List[int]]:
        # Create a list of all events
        # For start events, use negative height to ensure they are processed before end events
        events = []
        for L, R, H in buildings:
            events.append((L, -H))
            events.append((R, H))
        
        # Sort the events
        # First by x-coordinate, then by height
        events.sort()
        
        # Max heap to keep track of active buildings
        heap = [0]  # Initialize with ground level
        heapq.heapify(heap)
        active_heights = {0: 1}  # Dictionary to count heights
        
        result = []
        prev_max = 0
        
        for x, h in events:
            if h < 0:
                # Start of a building, add height to heap and dictionary
                heapq.heappush(heap, h)
                active_heights[h] = active_heights.get(h, 0) + 1
            else:
                # End of a building, remove height from dictionary
                active_heights[h] -= 1
                if active_heights[h] == 0:
                    del active_heights[h]
            
            # Current max height
            while heap and active_heights.get(heap[0], 0) == 0:
                heapq.heappop(heap)
            current_max = -heap[0] if heap else 0
            
            # If the max height has changed, add to result
            if current_max != prev_max:
                result.append([x, current_max])
                prev_max = current_max
        
        return result

# Example usage:
solution = Solution()
print(solution.getSkyline([[2,9,10], [3,7,15], [5,12,12], [15,20,10], [19,24,8]]))
# Output: [[2,10], [3,15], [7,12], [12,0], [15,10], [20,8], [24,0]]

print(solution.getSkyline([[0,2,3], [2,5,3]]))
# Output: [[0,3], [5,0]]

print(solution.getSkyline([[1,3,3], [2,4,4], [5,6,1]]))
# Output: [[1,3], [2,4], [4,0], [5,1], [6,0]]

print(solution.getSkyline([[0,5,0]]))
# Output: []
\end{lstlisting}
\end{fullwidth}

This implementation efficiently constructs the skyline by processing all building events in sorted order and maintaining active building heights using a max heap. It ensures that only critical points where the skyline changes are recorded.

\section*{Explanation}

The \texttt{getSkyline} function constructs the skyline formed by a set of buildings by leveraging the Sweep Line Algorithm and a max heap to track active buildings. Here's a detailed breakdown of the implementation:

\subsection*{1. Event Representation}

\begin{itemize}
    \item Each building is transformed into two events:
    \begin{itemize}
        \item **Start Event:** \((Li, -Hi)\) – Negative height indicates the start of a building.
        \item **End Event:** \((Ri, Hi)\) – Positive height indicates the end of a building.
    \end{itemize}
\end{itemize}

\subsection*{2. Sorting the Events}

\begin{itemize}
    \item Events are sorted primarily by their x-coordinate in ascending order.
    \item For events with the same x-coordinate:
    \begin{itemize}
        \item Start events (with negative heights) are processed before end events.
        \item Taller buildings are processed before shorter ones.
    \end{itemize}
\end{itemize}

\subsection*{3. Processing the Events}

\begin{itemize}
    \item **Heap Initialization:**
    \begin{itemize}
        \item A max heap is initialized with a ground level height of 0.
        \item A dictionary \texttt{active\_heights} tracks the count of active building heights.
    \end{itemize}
    
    \item **Iterating Through Events:**
    \begin{enumerate}
        \item **Start Event:**
        \begin{itemize}
            \item Add the building's height to the heap.
            \item Increment the count of the height in \texttt{active\_heights}.
        \end{itemize}
        
        \item **End Event:**
        \begin{itemize}
            \item Decrement the count of the building's height in \texttt{active\_heights}.
            \item If the count reaches zero, remove the height from the dictionary.
        \end{itemize}
        
        \item **Determine Current Max Height:**
        \begin{itemize}
            \item Remove heights from the heap that are no longer active.
            \item The current max height is the top of the heap.
        \end{itemize}
        
        \item **Update Skyline:**
        \begin{itemize}
            \item If the current max height differs from the previous max height, add a new key point \([x, current\_max\_height]\).
        \end{itemize}
    \end{enumerate}
\end{itemize}

\subsection*{4. Finalizing the Skyline}

\begin{itemize}
    \item After processing all events, the \texttt{result} list contains the key points defining the skyline.
\end{itemize}

\subsection*{5. Example Walkthrough}

Consider the first example:
\begin{verbatim}
Input: buildings = [[2,9,10], [3,7,15], [5,12,12], [15,20,10], [19,24,8]]
Output: [[2,10], [3,15], [7,12], [12,0], [15,10], [20,8], [24,0]]
\end{verbatim}

\begin{enumerate}
    \item **Event Transformation:**
    \begin{itemize}
        \item \((2, -10)\), \((9, 10)\)
        \item \((3, -15)\), \((7, 15)\)
        \item \((5, -12)\), \((12, 12)\)
        \item \((15, -10)\), \((20, 10)\)
        \item \((19, -8)\), \((24, 8)\)
    \end{itemize}
    
    \item **Sorting Events:**
    \begin{itemize}
        \item Sorted order: \((2, -10)\), \((3, -15)\), \((5, -12)\), \((7, 15)\), \((9, 10)\), \((12, 12)\), \((15, -10)\), \((19, -8)\), \((20, 10)\), \((24, 8)\)
    \end{itemize}
    
    \item **Processing Events:**
    \begin{itemize}
        \item At each event, update the heap and determine if the skyline height changes.
    \end{itemize}
    
    \item **Result Construction:**
    \begin{itemize}
        \item The resulting skyline key points are accumulated as \([[2,10], [3,15], [7,12], [12,0], [15,10], [20,8], [24,0]]\).
    \end{itemize}
\end{enumerate}

Thus, the function correctly constructs the skyline based on the buildings' positions and heights.

\section*{Why This Approach}

The Sweep Line Algorithm combined with a max heap offers an optimal solution with \(O(n \log n)\) time complexity and efficient handling of overlapping buildings. By processing events in sorted order and maintaining active building heights, the algorithm ensures that all critical points in the skyline are accurately identified without redundant computations.

\section*{Alternative Approaches}

\subsection*{1. Divide and Conquer}

Divide the set of buildings into smaller subsets, compute the skyline for each subset, and then merge the skylines.

\begin{lstlisting}[language=Python]
class Solution:
    def getSkyline(self, buildings: List[List[int]]) -> List[List[int]]:
        def merge(left, right):
            h1, h2 = 0, 0
            i, j = 0, 0
            merged = []
            while i < len(left) and j < len(right):
                if left[i][0] < right[j][0]:
                    x, h1 = left[i]
                    i += 1
                elif left[i][0] > right[j][0]:
                    x, h2 = right[j]
                    j += 1
                else:
                    x, h1 = left[i]
                    _, h2 = right[j]
                    i += 1
                    j += 1
                max_h = max(h1, h2)
                if not merged or merged[-1][1] != max_h:
                    merged.append([x, max_h])
            merged.extend(left[i:])
            merged.extend(right[j:])
            return merged
        
        def divide(buildings):
            if not buildings:
                return []
            if len(buildings) == 1:
                L, R, H = buildings[0]
                return [[L, H], [R, 0]]
            mid = len(buildings) // 2
            left = divide(buildings[:mid])
            right = divide(buildings[mid:])
            return merge(left, right)
        
        return divide(buildings)
\end{lstlisting}

\textbf{Complexities:}
\begin{itemize}
    \item \textbf{Time Complexity:} \(O(n \log n)\)
    \item \textbf{Space Complexity:} \(O(n)\)
\end{itemize}

\subsection*{2. Using Segment Trees}

Implement a segment tree to manage and query overlapping building heights dynamically.

\textbf{Note}: This approach is more complex and is generally used for advanced scenarios with multiple dynamic queries.

\section*{Similar Problems to This One}

Several problems involve skyline-like constructions, spatial data analysis, and efficient event processing, utilizing similar algorithmic strategies:

\begin{itemize}
    \item \textbf{Merge Intervals}: Merge overlapping intervals in a list.
    \item \textbf{Largest Rectangle in Histogram}: Find the largest rectangular area in a histogram.
    \item \textbf{Interval Partitioning}: Assign intervals to resources without overlap.
    \item \textbf{Line Segment Intersection}: Detect intersections among line segments.
    \item \textbf{Closest Pair of Points}: Find the closest pair of points in a set.
    \item \textbf{Convex Hull}: Compute the convex hull of a set of points.
    \item \textbf{Point Inside Polygon}: Determine if a point lies inside a given polygon.
    \item \textbf{Range Searching}: Efficiently query geometric data within a specified range.
\end{itemize}

These problems reinforce concepts of event-driven processing, spatial reasoning, and efficient algorithm design in various contexts.

\section*{Things to Keep in Mind and Tricks}

When tackling the \textbf{Skyline Problem}, consider the following tips and best practices to enhance efficiency and correctness:

\begin{itemize}
    \item \textbf{Understand Sweep Line Technique}: Grasp how the sweep line algorithm processes events in sorted order to handle dynamic changes efficiently.
    \index{Sweep Line Technique}
    
    \item \textbf{Leverage Priority Queues (Heaps)}: Use max heaps to keep track of active buildings' heights, enabling quick access to the current maximum height.
    \index{Priority Queues}
    
    \item \textbf{Handle Start and End Events Differently}: Differentiate between building start and end events to accurately manage active heights.
    \index{Start and End Events}
    
    \item \textbf{Optimize Event Sorting}: Sort events primarily by x-coordinate and secondarily by height to ensure correct processing order.
    \index{Event Sorting}
    
    \item \textbf{Manage Active Heights Efficiently}: Use data structures that allow efficient insertion, deletion, and retrieval of maximum elements.
    \index{Active Heights Management}
    
    \item \textbf{Avoid Redundant Key Points}: Only record key points when the skyline height changes to minimize the output list.
    \index{Avoiding Redundant Key Points}
    
    \item \textbf{Implement Helper Functions}: Create helper functions for tasks like distance calculation, event handling, and heap management to enhance modularity.
    \index{Helper Functions}
    
    \item \textbf{Code Readability}: Maintain clear and readable code through meaningful variable names and structured logic.
    \index{Code Readability}
    
    \item \textbf{Test Extensively}: Implement a wide range of test cases, including overlapping, non-overlapping, and edge-touching buildings, to ensure robustness.
    \index{Extensive Testing}
    
    \item \textbf{Handle Degenerate Cases}: Manage cases where buildings have zero height or identical coordinates gracefully.
    \index{Degenerate Cases}
    
    \item \textbf{Understand Geometric Relationships}: Grasp how buildings overlap and influence the skyline to simplify the algorithm.
    \index{Geometric Relationships}
    
    \item \textbf{Use Appropriate Data Structures}: Utilize appropriate data structures like heaps, lists, and dictionaries to manage and process data efficiently.
    \index{Appropriate Data Structures}
    
    \item \textbf{Optimize for Large Inputs}: Design the algorithm to handle large numbers of buildings without significant performance degradation.
    \index{Optimizing for Large Inputs}
    
    \item \textbf{Implement Iterative Solutions Carefully}: Ensure that loop conditions are correctly defined to prevent infinite loops or incorrect terminations.
    \index{Iterative Solutions}
    
    \item \textbf{Consistent Naming Conventions}: Use consistent and descriptive naming conventions for variables and functions to improve code clarity.
    \index{Naming Conventions}
\end{itemize}

\section*{Corner and Special Cases to Test When Writing the Code}

When implementing the solution for the \textbf{Skyline Problem}, it is crucial to consider and rigorously test various edge cases to ensure robustness and correctness:

\begin{itemize}
    \item \textbf{No Overlapping Buildings}: All buildings are separate and do not overlap.
    \index{No Overlapping Buildings}
    
    \item \textbf{Fully Overlapping Buildings}: Multiple buildings completely overlap each other.
    \index{Fully Overlapping Buildings}
    
    \item \textbf{Buildings Touching at Edges}: Buildings share common edges without overlapping.
    \index{Buildings Touching at Edges}
    
    \item \textbf{Buildings Touching at Corners}: Buildings share common corners without overlapping.
    \index{Buildings Touching at Corners}
    
    \item \textbf{Single Building}: Only one building is present.
    \index{Single Building}
    
    \item \textbf{Multiple Buildings with Same Start or End}: Multiple buildings start or end at the same x-coordinate.
    \index{Same Start or End}
    
    \item \textbf{Buildings with Zero Height}: Buildings that have zero height should not affect the skyline.
    \index{Buildings with Zero Height}
    
    \item \textbf{Large Number of Buildings}: Test with a large number of buildings to ensure performance and scalability.
    \index{Large Number of Buildings}
    
    \item \textbf{Buildings with Negative Coordinates}: Buildings positioned in negative coordinate space.
    \index{Negative Coordinates}
    
    \item \textbf{Boundary Values}: Buildings at the minimum and maximum limits of the coordinate range.
    \index{Boundary Values}
    
    \item \textbf{Buildings with Identical Coordinates}: Multiple buildings with the same coordinates.
    \index{Identical Coordinates}
    
    \item \textbf{Sequential Buildings}: Buildings placed sequentially without gaps.
    \index{Sequential Buildings}
    
    \item \textbf{Overlapping and Non-Overlapping Mixed}: A mix of overlapping and non-overlapping buildings.
    \index{Overlapping and Non-Overlapping Mixed}
    
    \item \textbf{Buildings with Very Large Heights}: Buildings with heights at the upper limit of the constraints.
    \index{Very Large Heights}
    
    \item \textbf{Empty Input}: No buildings are provided.
    \index{Empty Input}
\end{itemize}

\section*{Implementation Considerations}

When implementing the \texttt{getSkyline} function, keep in mind the following considerations to ensure robustness and efficiency:

\begin{itemize}
    \item \textbf{Data Type Selection}: Use appropriate data types that can handle large input values and avoid overflow or precision issues.
    \index{Data Type Selection}
    
    \item \textbf{Optimizing Event Sorting}: Efficiently sort events based on x-coordinates and heights to ensure correct processing order.
    \index{Optimizing Event Sorting}
    
    \item \textbf{Handling Large Inputs}: Design the algorithm to handle up to \(10^4\) buildings efficiently without significant performance degradation.
    \index{Handling Large Inputs}
    
    \item \textbf{Using Efficient Data Structures}: Utilize heaps, lists, and dictionaries effectively to manage and process events and active heights.
    \index{Efficient Data Structures}
    
    \item \textbf{Avoiding Redundant Calculations}: Ensure that distance and overlap calculations are performed only when necessary to optimize performance.
    \index{Avoiding Redundant Calculations}
    
    \item \textbf{Code Readability and Documentation}: Maintain clear and readable code through meaningful variable names and comprehensive comments to facilitate understanding and maintenance.
    \index{Code Readability}
    
    \item \textbf{Edge Case Handling}: Implement checks for edge cases to prevent incorrect results or runtime errors.
    \index{Edge Case Handling}
    
    \item \textbf{Implementing Helper Functions}: Create helper functions for tasks like distance calculation, event handling, and heap management to enhance modularity.
    \index{Helper Functions}
    
    \item \textbf{Consistent Naming Conventions}: Use consistent and descriptive naming conventions for variables and functions to improve code clarity.
    \index{Naming Conventions}
    
    \item \textbf{Memory Management}: Ensure that the algorithm manages memory efficiently, especially when dealing with large datasets.
    \index{Memory Management}
    
    \item \textbf{Implementing Iterative Solutions Carefully}: Ensure that loop conditions are correctly defined to prevent infinite loops or incorrect terminations.
    \index{Iterative Solutions}
    
    \item \textbf{Avoiding Floating-Point Precision Issues}: Since the problem deals with integers, floating-point precision is not a concern, simplifying the implementation.
    \index{Floating-Point Precision}
    
    \item \textbf{Testing and Validation}: Develop a comprehensive suite of test cases that cover all possible scenarios, including edge cases, to validate the correctness and efficiency of the implementation.
    \index{Testing and Validation}
    
    \item \textbf{Performance Considerations}: Optimize the loop conditions and operations to ensure that the function runs efficiently, especially for large input numbers.
    \index{Performance Considerations}
\end{itemize}

\section*{Conclusion}

The \textbf{Skyline Problem} is a quintessential example of applying advanced algorithmic techniques and geometric reasoning to solve complex spatial challenges. By leveraging the Sweep Line Algorithm and maintaining active building heights using a max heap, the solution efficiently constructs the skyline with optimal time and space complexities. Understanding and implementing such sophisticated algorithms not only enhances problem-solving skills but also provides a foundation for tackling a wide array of Computational Geometry problems in various domains, including computer graphics, urban planning simulations, and geographic information systems.

\printindex

% \input{sections/rectangle_overlap}
% \input{sections/rectangle_area}
% \input{sections/k_closest_points_to_origin}
% \input{sections/the_skyline_problem}
% % filename: rectangle_area.tex

\problemsection{Rectangle Area}
\label{chap:Rectangle_Area}
\marginnote{\href{https://leetcode.com/problems/rectangle-area/}{[LeetCode Link]}\index{LeetCode}}
\marginnote{\href{https://www.geeksforgeeks.org/find-area-two-overlapping-rectangles/}{[GeeksForGeeks Link]}\index{GeeksForGeeks}}
\marginnote{\href{https://www.interviewbit.com/problems/rectangle-area/}{[InterviewBit Link]}\index{InterviewBit}}
\marginnote{\href{https://app.codesignal.com/challenges/rectangle-area}{[CodeSignal Link]}\index{CodeSignal}}
\marginnote{\href{https://www.codewars.com/kata/rectangle-area/train/python}{[Codewars Link]}\index{Codewars}}

The \textbf{Rectangle Area} problem is a classic Computational Geometry challenge that involves calculating the total area covered by two axis-aligned rectangles in a 2D plane. This problem tests one's ability to perform geometric calculations, handle overlapping scenarios, and implement efficient algorithms. Mastery of this problem is essential for applications in computer graphics, spatial analysis, and computational modeling.

\section*{Problem Statement}

Given two axis-aligned rectangles in a 2D plane, compute the total area covered by the two rectangles. The area covered by the overlapping region should be counted only once.

Each rectangle is represented as a list of four integers \([x1, y1, x2, y2]\), where \((x1, y1)\) are the coordinates of the bottom-left corner, and \((x2, y2)\) are the coordinates of the top-right corner.

\textbf{Function signature in Python:}
\begin{lstlisting}[language=Python]
def computeArea(A: List[int], B: List[int]) -> int:
\end{lstlisting}

\section*{Examples}

\textbf{Example 1:}

\begin{verbatim}
Input: A = [-3,0,3,4], B = [0,-1,9,2]
Output: 45
Explanation:
Area of A = (3 - (-3)) * (4 - 0) = 6 * 4 = 24
Area of B = (9 - 0) * (2 - (-1)) = 9 * 3 = 27
Overlapping Area = (3 - 0) * (2 - 0) = 3 * 2 = 6
Total Area = 24 + 27 - 6 = 45
\end{verbatim}

\textbf{Example 2:}

\begin{verbatim}
Input: A = [0,0,0,0], B = [0,0,0,0]
Output: 0
Explanation:
Both rectangles are degenerate points with zero area.
\end{verbatim}

\textbf{Example 3:}

\begin{verbatim}
Input: A = [0,0,2,2], B = [1,1,3,3]
Output: 7
Explanation:
Area of A = 4
Area of B = 4
Overlapping Area = 1
Total Area = 4 + 4 - 1 = 7
\end{verbatim}

\textbf{Example 4:}

\begin{verbatim}
Input: A = [0,0,1,1], B = [1,0,2,1]
Output: 2
Explanation:
Rectangles touch at the edge but do not overlap.
Area of A = 1
Area of B = 1
Overlapping Area = 0
Total Area = 1 + 1 = 2
\end{verbatim}

\textbf{Constraints:}

\begin{itemize}
    \item All coordinates are integers in the range \([-10^9, 10^9]\).
    \item For each rectangle, \(x1 < x2\) and \(y1 < y2\).
\end{itemize}

LeetCode link: \href{https://leetcode.com/problems/rectangle-area/}{Rectangle Area}\index{LeetCode}

\section*{Algorithmic Approach}

To compute the total area covered by two axis-aligned rectangles, we can follow these steps:

1. **Calculate Individual Areas:**
   - Compute the area of the first rectangle.
   - Compute the area of the second rectangle.

2. **Determine Overlapping Area:**
   - Calculate the coordinates of the overlapping rectangle, if any.
   - If the rectangles overlap, compute the area of the overlapping region.

3. **Compute Total Area:**
   - Sum the individual areas and subtract the overlapping area to avoid double-counting.

\marginnote{This approach ensures accurate area calculation by handling overlapping regions appropriately.}

\section*{Complexities}

\begin{itemize}
    \item \textbf{Time Complexity:} \(O(1)\). The algorithm performs a constant number of calculations.
    
    \item \textbf{Space Complexity:} \(O(1)\). Only a fixed amount of extra space is used for variables.
\end{itemize}

\section*{Python Implementation}

\marginnote{Implementing the area calculation with overlap consideration ensures an accurate and efficient solution.}

Below is the complete Python code implementing the \texttt{computeArea} function:

\begin{fullwidth}
\begin{lstlisting}[language=Python]
from typing import List

class Solution:
    def computeArea(self, A: List[int], B: List[int]) -> int:
        # Calculate area of rectangle A
        areaA = (A[2] - A[0]) * (A[3] - A[1])
        
        # Calculate area of rectangle B
        areaB = (B[2] - B[0]) * (B[3] - B[1])
        
        # Determine overlap coordinates
        overlap_x1 = max(A[0], B[0])
        overlap_y1 = max(A[1], B[1])
        overlap_x2 = min(A[2], B[2])
        overlap_y2 = min(A[3], B[3])
        
        # Calculate overlapping area
        overlap_width = overlap_x2 - overlap_x1
        overlap_height = overlap_y2 - overlap_y1
        overlap_area = 0
        if overlap_width > 0 and overlap_height > 0:
            overlap_area = overlap_width * overlap_height
        
        # Total area is sum of individual areas minus overlapping area
        total_area = areaA + areaB - overlap_area
        return total_area

# Example usage:
solution = Solution()
print(solution.computeArea([-3,0,3,4], [0,-1,9,2]))  # Output: 45
print(solution.computeArea([0,0,0,0], [0,0,0,0]))    # Output: 0
print(solution.computeArea([0,0,2,2], [1,1,3,3]))    # Output: 7
print(solution.computeArea([0,0,1,1], [1,0,2,1]))    # Output: 2
\end{lstlisting}
\end{fullwidth}

This implementation accurately computes the total area covered by two rectangles by accounting for any overlapping regions. It ensures that the overlapping area is not double-counted.

\section*{Explanation}

The \texttt{computeArea} function calculates the combined area of two axis-aligned rectangles by following these steps:

\subsection*{1. Calculate Individual Areas}

\begin{itemize}
    \item **Rectangle A:**
    \begin{itemize}
        \item Width: \(A[2] - A[0]\)
        \item Height: \(A[3] - A[1]\)
        \item Area: Width \(\times\) Height
    \end{itemize}
    
    \item **Rectangle B:**
    \begin{itemize}
        \item Width: \(B[2] - B[0]\)
        \item Height: \(B[3] - B[1]\)
        \item Area: Width \(\times\) Height
    \end{itemize}
\end{itemize}

\subsection*{2. Determine Overlapping Area}

\begin{itemize}
    \item **Overlap Coordinates:**
    \begin{itemize}
        \item Left (x-coordinate): \(\text{max}(A[0], B[0])\)
        \item Bottom (y-coordinate): \(\text{max}(A[1], B[1])\)
        \item Right (x-coordinate): \(\text{min}(A[2], B[2])\)
        \item Top (y-coordinate): \(\text{min}(A[3], B[3])\)
    \end{itemize}
    
    \item **Overlap Dimensions:**
    \begin{itemize}
        \item Width: \(\text{overlap\_x2} - \text{overlap\_x1}\)
        \item Height: \(\text{overlap\_y2} - \text{overlap\_y1}\)
    \end{itemize}
    
    \item **Overlap Area:**
    \begin{itemize}
        \item If both width and height are positive, the rectangles overlap, and the overlapping area is their product.
        \item Otherwise, there is no overlap, and the overlapping area is zero.
    \end{itemize}
\end{itemize}

\subsection*{3. Compute Total Area}

\begin{itemize}
    \item Total Area = Area of Rectangle A + Area of Rectangle B - Overlapping Area
\end{itemize}

\subsection*{4. Example Walkthrough}

Consider the first example:
\begin{verbatim}
Input: A = [-3,0,3,4], B = [0,-1,9,2]
Output: 45
\end{verbatim}

\begin{enumerate}
    \item **Calculate Areas:**
    \begin{itemize}
        \item Area of A = (3 - (-3)) * (4 - 0) = 6 * 4 = 24
        \item Area of B = (9 - 0) * (2 - (-1)) = 9 * 3 = 27
    \end{itemize}
    
    \item **Determine Overlap:**
    \begin{itemize}
        \item overlap\_x1 = max(-3, 0) = 0
        \item overlap\_y1 = max(0, -1) = 0
        \item overlap\_x2 = min(3, 9) = 3
        \item overlap\_y2 = min(4, 2) = 2
        \item overlap\_width = 3 - 0 = 3
        \item overlap\_height = 2 - 0 = 2
        \item overlap\_area = 3 * 2 = 6
    \end{itemize}
    
    \item **Compute Total Area:**
    \begin{itemize}
        \item Total Area = 24 + 27 - 6 = 45
    \end{itemize}
\end{enumerate}

Thus, the function correctly returns \texttt{45}.

\section*{Why This Approach}

This approach is chosen for its straightforwardness and optimal efficiency. By directly calculating the individual areas and intelligently handling the overlapping region, the algorithm ensures accurate results without unnecessary computations. Its constant time complexity makes it highly efficient, even for large coordinate values.

\section*{Alternative Approaches}

\subsection*{1. Using Intersection Dimensions}

Instead of separately calculating areas, directly compute the dimensions of the overlapping region and subtract it from the sum of individual areas.

\begin{lstlisting}[language=Python]
def computeArea(A: List[int], B: List[int]) -> int:
    # Sum of individual areas
    area = (A[2] - A[0]) * (A[3] - A[1]) + (B[2] - B[0]) * (B[3] - B[1])
    
    # Overlapping area
    overlap_width = min(A[2], B[2]) - max(A[0], B[0])
    overlap_height = min(A[3], B[3]) - max(A[1], B[1])
    
    if overlap_width > 0 and overlap_height > 0:
        area -= overlap_width * overlap_height
    
    return area
\end{lstlisting}

\subsection*{2. Using Geometry Libraries}

Leverage computational geometry libraries to handle area calculations and overlapping detections.

\begin{lstlisting}[language=Python]
from shapely.geometry import box

def computeArea(A: List[int], B: List[int]) -> int:
    rect1 = box(A[0], A[1], A[2], A[3])
    rect2 = box(B[0], B[1], B[2], B[3])
    intersection = rect1.intersection(rect2)
    return int(rect1.area + rect2.area - intersection.area)
\end{lstlisting}

\textbf{Note}: This approach requires the \texttt{shapely} library and is more suitable for complex geometric operations.

\section*{Similar Problems to This One}

Several problems involve calculating areas, handling geometric overlaps, and spatial reasoning, utilizing similar algorithmic strategies:

\begin{itemize}
    \item \textbf{Rectangle Overlap}: Determine if two rectangles overlap.
    \item \textbf{Circle Area Overlap}: Calculate the overlapping area between two circles.
    \item \textbf{Polygon Area}: Compute the area of a given polygon.
    \item \textbf{Union of Rectangles}: Calculate the total area covered by multiple rectangles, accounting for overlaps.
    \item \textbf{Intersection of Lines}: Find the intersection point of two lines.
    \item \textbf{Closest Pair of Points}: Find the closest pair of points in a set.
    \item \textbf{Convex Hull}: Compute the convex hull of a set of points.
    \item \textbf{Point Inside Polygon}: Determine if a point lies inside a given polygon.
\end{itemize}

These problems reinforce concepts of geometric calculations, area computations, and efficient algorithm design in various contexts.

\section*{Things to Keep in Mind and Tricks}

When tackling the \textbf{Rectangle Area} problem, consider the following tips and best practices to enhance efficiency and correctness:

\begin{itemize}
    \item \textbf{Understand Geometric Relationships}: Grasp the positional relationships between rectangles to simplify area calculations.
    \index{Geometric Relationships}
    
    \item \textbf{Leverage Coordinate Comparisons}: Use direct comparisons of rectangle coordinates to determine overlapping regions.
    \index{Coordinate Comparisons}
    
    \item \textbf{Handle Overlapping Scenarios}: Accurately calculate the overlapping area to avoid double-counting.
    \index{Overlapping Scenarios}
    
    \item \textbf{Optimize for Efficiency}: Aim for a constant time \(O(1)\) solution by avoiding unnecessary computations or iterations.
    \index{Efficiency Optimization}
    
    \item \textbf{Avoid Floating-Point Precision Issues}: Since all coordinates are integers, floating-point precision is not a concern, simplifying the implementation.
    \index{Floating-Point Precision}
    
    \item \textbf{Use Helper Functions}: Create helper functions to encapsulate repetitive tasks, such as calculating overlap dimensions or areas.
    \index{Helper Functions}
    
    \item \textbf{Code Readability}: Maintain clear and readable code through meaningful variable names and structured logic.
    \index{Code Readability}
    
    \item \textbf{Test Extensively}: Implement a wide range of test cases, including overlapping, non-overlapping, and edge-touching rectangles, to ensure robustness.
    \index{Extensive Testing}
    
    \item \textbf{Understand Axis-Aligned Constraints}: Recognize that axis-aligned rectangles simplify area calculations compared to rotated rectangles.
    \index{Axis-Aligned Constraints}
    
    \item \textbf{Simplify Logical Conditions}: Combine multiple conditions logically to streamline the area calculation process.
    \index{Logical Conditions}
    
    \item \textbf{Use Absolute Values}: When calculating differences, ensure that the dimensions are positive by using absolute values or proper ordering.
    \index{Absolute Values}
    
    \item \textbf{Consider Edge Cases}: Handle cases where rectangles have zero area or touch at edges/corners without overlapping.
    \index{Edge Cases}
\end{itemize}

\section*{Corner and Special Cases to Test When Writing the Code}

When implementing the solution for the \textbf{Rectangle Area} problem, it is crucial to consider and rigorously test various edge cases to ensure robustness and correctness:

\begin{itemize}
    \item \textbf{No Overlap}: Rectangles are completely separate.
    \index{No Overlap}
    
    \item \textbf{Partial Overlap}: Rectangles overlap in one or more regions.
    \index{Partial Overlap}
    
    \item \textbf{Edge Touching}: Rectangles touch exactly at one edge without overlapping.
    \index{Edge Touching}
    
    \item \textbf{Corner Touching}: Rectangles touch exactly at one corner without overlapping.
    \index{Corner Touching}
    
    \item \textbf{One Rectangle Inside Another}: One rectangle is entirely within the other.
    \index{Rectangle Inside}
    
    \item \textbf{Identical Rectangles}: Both rectangles have the same coordinates.
    \index{Identical Rectangles}
    
    \item \textbf{Degenerate Rectangles}: Rectangles with zero area (e.g., \(x1 = x2\) or \(y1 = y2\)).
    \index{Degenerate Rectangles}
    
    \item \textbf{Large Coordinates}: Rectangles with very large coordinate values to test performance and integer handling.
    \index{Large Coordinates}
    
    \item \textbf{Negative Coordinates}: Rectangles positioned in negative coordinate space.
    \index{Negative Coordinates}
    
    \item \textbf{Mixed Overlapping Scenarios}: Combinations of the above cases to ensure comprehensive coverage.
    \index{Mixed Overlapping Scenarios}
    
    \item \textbf{Minimum and Maximum Bounds}: Rectangles at the minimum and maximum limits of the coordinate range.
    \index{Minimum and Maximum Bounds}
    
    \item \textbf{Sequential Rectangles}: Multiple rectangles placed sequentially without overlapping.
    \index{Sequential Rectangles}
    
    \item \textbf{Multiple Overlaps}: Scenarios where more than two rectangles overlap in different regions.
    \index{Multiple Overlaps}
\end{itemize}

\section*{Implementation Considerations}

When implementing the \texttt{computeArea} function, keep in mind the following considerations to ensure robustness and efficiency:

\begin{itemize}
    \item \textbf{Data Type Selection}: Use appropriate data types that can handle large input values without overflow or underflow.
    \index{Data Type Selection}
    
    \item \textbf{Optimizing Comparisons}: Structure logical conditions to efficiently determine overlap dimensions.
    \index{Optimizing Comparisons}
    
    \item \textbf{Handling Large Inputs}: Design the algorithm to efficiently handle large input sizes without significant performance degradation.
    \index{Handling Large Inputs}
    
    \item \textbf{Language-Specific Constraints}: Be aware of how the programming language handles large integers and arithmetic operations.
    \index{Language-Specific Constraints}
    
    \item \textbf{Avoiding Redundant Calculations}: Ensure that each calculation contributes towards determining the final area without unnecessary repetitions.
    \index{Avoiding Redundant Calculations}
    
    \item \textbf{Code Readability and Documentation}: Maintain clear and readable code through meaningful variable names and comprehensive comments to facilitate understanding and maintenance.
    \index{Code Readability}
    
    \item \textbf{Edge Case Handling}: Implement checks for edge cases to prevent incorrect results or runtime errors.
    \index{Edge Case Handling}
    
    \item \textbf{Testing and Validation}: Develop a comprehensive suite of test cases that cover all possible scenarios, including edge cases, to validate the correctness and efficiency of the implementation.
    \index{Testing and Validation}
    
    \item \textbf{Scalability}: Design the algorithm to scale efficiently with increasing input sizes, maintaining performance and resource utilization.
    \index{Scalability}
    
    \item \textbf{Using Helper Functions}: Consider creating helper functions for repetitive tasks, such as calculating overlap dimensions, to enhance modularity and reusability.
    \index{Helper Functions}
    
    \item \textbf{Consistent Naming Conventions}: Use consistent and descriptive naming conventions for variables to improve code clarity.
    \index{Naming Conventions}
    
    \item \textbf{Implementing Unit Tests}: Develop unit tests for each logical condition to ensure that all scenarios are correctly handled.
    \index{Unit Tests}
    
    \item \textbf{Error Handling}: Incorporate error handling to manage invalid inputs gracefully.
    \index{Error Handling}
\end{itemize}

\section*{Conclusion}

The \textbf{Rectangle Area} problem showcases the application of fundamental geometric principles and efficient algorithm design to compute spatial properties accurately. By systematically calculating individual areas and intelligently handling overlapping regions, the algorithm ensures precise results without redundant computations. Understanding and implementing such techniques not only enhances problem-solving skills but also provides a foundation for tackling more complex Computational Geometry challenges involving multiple geometric entities and intricate spatial relationships.

\printindex

% % filename: rectangle_overlap.tex

\problemsection{Rectangle Overlap}
\label{chap:Rectangle_Overlap}
\marginnote{\href{https://leetcode.com/problems/rectangle-overlap/}{[LeetCode Link]}\index{LeetCode}}
\marginnote{\href{https://www.geeksforgeeks.org/check-if-two-rectangles-overlap/}{[GeeksForGeeks Link]}\index{GeeksForGeeks}}
\marginnote{\href{https://www.interviewbit.com/problems/rectangle-overlap/}{[InterviewBit Link]}\index{InterviewBit}}
\marginnote{\href{https://app.codesignal.com/challenges/rectangle-overlap}{[CodeSignal Link]}\index{CodeSignal}}
\marginnote{\href{https://www.codewars.com/kata/rectangle-overlap/train/python}{[Codewars Link]}\index{Codewars}}

The \textbf{Rectangle Overlap} problem is a fundamental challenge in Computational Geometry that involves determining whether two axis-aligned rectangles overlap. This problem tests one's ability to understand geometric properties, implement conditional logic, and optimize for efficient computation. Mastery of this problem is essential for applications in computer graphics, collision detection, and spatial data analysis.

\section*{Problem Statement}

Given two axis-aligned rectangles in a 2D plane, determine if they overlap. Each rectangle is defined by its bottom-left and top-right coordinates.

A rectangle is represented as a list of four integers \([x1, y1, x2, y2]\), where \((x1, y1)\) are the coordinates of the bottom-left corner, and \((x2, y2)\) are the coordinates of the top-right corner.

\textbf{Function signature in Python:}
\begin{lstlisting}[language=Python]
def isRectangleOverlap(rec1: List[int], rec2: List[int]) -> bool:
\end{lstlisting}

\section*{Examples}

\textbf{Example 1:}

\begin{verbatim}
Input: rec1 = [0,0,2,2], rec2 = [1,1,3,3]
Output: True
Explanation: The rectangles overlap in the area defined by [1,1,2,2].
\end{verbatim}

\textbf{Example 2:}

\begin{verbatim}
Input: rec1 = [0,0,1,1], rec2 = [1,0,2,1]
Output: False
Explanation: The rectangles touch at the edge but do not overlap.
\end{verbatim}

\textbf{Example 3:}

\begin{verbatim}
Input: rec1 = [0,0,1,1], rec2 = [2,2,3,3]
Output: False
Explanation: The rectangles are completely separate.
\end{verbatim}

\textbf{Example 4:}

\begin{verbatim}
Input: rec1 = [0,0,5,5], rec2 = [3,3,7,7]
Output: True
Explanation: The rectangles overlap in the area defined by [3,3,5,5].
\end{verbatim}

\textbf{Example 5:}

\begin{verbatim}
Input: rec1 = [0,0,0,0], rec2 = [0,0,0,0]
Output: False
Explanation: Both rectangles are degenerate points.
\end{verbatim}

\textbf{Constraints:}

\begin{itemize}
    \item All coordinates are integers in the range \([-10^9, 10^9]\).
    \item For each rectangle, \(x1 < x2\) and \(y1 < y2\).
\end{itemize}

LeetCode link: \href{https://leetcode.com/problems/rectangle-overlap/}{Rectangle Overlap}\index{LeetCode}

\section*{Algorithmic Approach}

To determine whether two axis-aligned rectangles overlap, we can use the following logical conditions:

1. **Non-Overlap Conditions:**
   - One rectangle is to the left of the other.
   - One rectangle is above the other.

2. **Overlap Condition:**
   - If neither of the non-overlap conditions is true, the rectangles must overlap.

\subsection*{Steps:}

1. **Extract Coordinates:**
   - For both rectangles, extract the bottom-left and top-right coordinates.

2. **Check Non-Overlap Conditions:**
   - If the right side of the first rectangle is less than or equal to the left side of the second rectangle, they do not overlap.
   - If the left side of the first rectangle is greater than or equal to the right side of the second rectangle, they do not overlap.
   - If the top side of the first rectangle is less than or equal to the bottom side of the second rectangle, they do not overlap.
   - If the bottom side of the first rectangle is greater than or equal to the top side of the second rectangle, they do not overlap.

3. **Determine Overlap:**
   - If none of the non-overlap conditions are met, the rectangles overlap.

\marginnote{This approach provides an efficient \(O(1)\) time complexity solution by leveraging simple geometric comparisons.}

\section*{Complexities}

\begin{itemize}
    \item \textbf{Time Complexity:} \(O(1)\). The algorithm performs a constant number of comparisons regardless of input size.
    
    \item \textbf{Space Complexity:} \(O(1)\). Only a fixed amount of extra space is used for variables.
\end{itemize}

\section*{Python Implementation}

\marginnote{Implementing the overlap check using coordinate comparisons ensures an optimal and straightforward solution.}

Below is the complete Python code implementing the \texttt{isRectangleOverlap} function:

\begin{fullwidth}
\begin{lstlisting}[language=Python]
from typing import List

class Solution:
    def isRectangleOverlap(self, rec1: List[int], rec2: List[int]) -> bool:
        # Extract coordinates
        left1, bottom1, right1, top1 = rec1
        left2, bottom2, right2, top2 = rec2
        
        # Check non-overlapping conditions
        if right1 <= left2 or right2 <= left1:
            return False
        if top1 <= bottom2 or top2 <= bottom1:
            return False
        
        # If none of the above, rectangles overlap
        return True

# Example usage:
solution = Solution()
print(solution.isRectangleOverlap([0,0,2,2], [1,1,3,3]))  # Output: True
print(solution.isRectangleOverlap([0,0,1,1], [1,0,2,1]))  # Output: False
print(solution.isRectangleOverlap([0,0,1,1], [2,2,3,3]))  # Output: False
print(solution.isRectangleOverlap([0,0,5,5], [3,3,7,7]))  # Output: True
print(solution.isRectangleOverlap([0,0,0,0], [0,0,0,0]))  # Output: False
\end{lstlisting}
\end{fullwidth}

This implementation efficiently checks for overlap by comparing the coordinates of the two rectangles. If any of the non-overlapping conditions are met, it returns \texttt{False}; otherwise, it returns \texttt{True}.

\section*{Explanation}

The \texttt{isRectangleOverlap} function determines whether two axis-aligned rectangles overlap by comparing their respective coordinates. Here's a detailed breakdown of the implementation:

\subsection*{1. Extract Coordinates}

\begin{itemize}
    \item For each rectangle, extract the left (\(x1\)), bottom (\(y1\)), right (\(x2\)), and top (\(y2\)) coordinates.
    \item This simplifies the comparison process by providing clear variables representing each side of the rectangles.
\end{itemize}

\subsection*{2. Check Non-Overlap Conditions}

\begin{itemize}
    \item **Horizontal Separation:**
    \begin{itemize}
        \item If the right side of the first rectangle (\(right1\)) is less than or equal to the left side of the second rectangle (\(left2\)), there is no horizontal overlap.
        \item Similarly, if the right side of the second rectangle (\(right2\)) is less than or equal to the left side of the first rectangle (\(left1\)), there is no horizontal overlap.
    \end{itemize}
    
    \item **Vertical Separation:**
    \begin{itemize}
        \item If the top side of the first rectangle (\(top1\)) is less than or equal to the bottom side of the second rectangle (\(bottom2\)), there is no vertical overlap.
        \item Similarly, if the top side of the second rectangle (\(top2\)) is less than or equal to the bottom side of the first rectangle (\(bottom1\)), there is no vertical overlap.
    \end{itemize}
    
    \item If any of these non-overlapping conditions are true, the rectangles do not overlap, and the function returns \texttt{False}.
\end{itemize}

\subsection*{3. Determine Overlap}

\begin{itemize}
    \item If none of the non-overlapping conditions are met, it implies that the rectangles overlap both horizontally and vertically.
    \item The function returns \texttt{True} in this case.
\end{itemize}

\subsection*{4. Example Walkthrough}

Consider the first example:
\begin{verbatim}
Input: rec1 = [0,0,2,2], rec2 = [1,1,3,3]
Output: True
\end{verbatim}

\begin{enumerate}
    \item Extract coordinates:
    \begin{itemize}
        \item rec1: left1 = 0, bottom1 = 0, right1 = 2, top1 = 2
        \item rec2: left2 = 1, bottom2 = 1, right2 = 3, top2 = 3
    \end{itemize}
    
    \item Check non-overlap conditions:
    \begin{itemize}
        \item \(right1 = 2\) is not less than or equal to \(left2 = 1\)
        \item \(right2 = 3\) is not less than or equal to \(left1 = 0\)
        \item \(top1 = 2\) is not less than or equal to \(bottom2 = 1\)
        \item \(top2 = 3\) is not less than or equal to \(bottom1 = 0\)
    \end{itemize}
    
    \item Since none of the non-overlapping conditions are met, the rectangles overlap.
\end{enumerate}

Thus, the function correctly returns \texttt{True}.

\section*{Why This Approach}

This approach is chosen for its simplicity and efficiency. By leveraging direct coordinate comparisons, the algorithm achieves constant time complexity without the need for complex data structures or iterative processes. It effectively handles all possible scenarios of rectangle positioning, ensuring accurate detection of overlaps.

\section*{Alternative Approaches}

\subsection*{1. Separating Axis Theorem (SAT)}

The Separating Axis Theorem is a more generalized method for detecting overlaps between convex shapes. While it is not necessary for axis-aligned rectangles, understanding SAT can be beneficial for more complex geometric problems.

\begin{lstlisting}[language=Python]
def isRectangleOverlap(rec1: List[int], rec2: List[int]) -> bool:
    # Using SAT for axis-aligned rectangles
    return not (rec1[2] <= rec2[0] or rec1[0] >= rec2[2] or
                rec1[3] <= rec2[1] or rec1[1] >= rec2[3])
\end{lstlisting}

\textbf{Note}: This implementation is functionally identical to the primary approach but leverages a more generalized geometric theorem.

\subsection*{2. Area-Based Approach}

Calculate the overlapping area between the two rectangles. If the overlapping area is positive, the rectangles overlap.

\begin{lstlisting}[language=Python]
def isRectangleOverlap(rec1: List[int], rec2: List[int]) -> bool:
    # Calculate overlap in x and y dimensions
    x_overlap = min(rec1[2], rec2[2]) - max(rec1[0], rec2[0])
    y_overlap = min(rec1[3], rec2[3]) - max(rec1[1], rec2[1])
    
    # Overlap exists if both overlaps are positive
    return x_overlap > 0 and y_overlap > 0
\end{lstlisting}

\textbf{Complexities:}
\begin{itemize}
    \item \textbf{Time Complexity:} \(O(1)\)
    \item \textbf{Space Complexity:} \(O(1)\)
\end{itemize}

\subsection*{3. Using Rectangles Intersection Function}

Utilize built-in or library functions that handle geometric intersections.

\begin{lstlisting}[language=Python]
from shapely.geometry import box

def isRectangleOverlap(rec1: List[int], rec2: List[int]) -> bool:
    rectangle1 = box(rec1[0], rec1[1], rec1[2], rec1[3])
    rectangle2 = box(rec2[0], rec2[1], rec2[2], rec2[3])
    return rectangle1.intersects(rectangle2) and not rectangle1.touches(rectangle2)
\end{lstlisting}

\textbf{Note}: This approach requires the \texttt{shapely} library and is more suitable for complex geometric operations.

\section*{Similar Problems to This One}

Several problems revolve around geometric overlap, intersection detection, and spatial reasoning, utilizing similar algorithmic strategies:

\begin{itemize}
    \item \textbf{Interval Overlap}: Determine if two intervals on a line overlap.
    \item \textbf{Circle Overlap}: Determine if two circles overlap based on their radii and centers.
    \item \textbf{Polygon Overlap}: Determine if two polygons overlap using algorithms like SAT.
    \item \textbf{Closest Pair of Points}: Find the closest pair of points in a set.
    \item \textbf{Convex Hull}: Compute the convex hull of a set of points.
    \item \textbf{Intersection of Lines}: Find the intersection point of two lines.
    \item \textbf{Point Inside Polygon}: Determine if a point lies inside a given polygon.
\end{itemize}

These problems reinforce the concepts of spatial reasoning, geometric property analysis, and efficient algorithm design in various contexts.

\section*{Things to Keep in Mind and Tricks}

When working with the \textbf{Rectangle Overlap} problem, consider the following tips and best practices to enhance efficiency and correctness:

\begin{itemize}
    \item \textbf{Understand Geometric Relationships}: Grasp the positional relationships between rectangles to simplify overlap detection.
    \index{Geometric Relationships}
    
    \item \textbf{Leverage Coordinate Comparisons}: Use direct comparisons of rectangle coordinates to determine spatial relationships.
    \index{Coordinate Comparisons}
    
    \item \textbf{Handle Edge Cases}: Consider cases where rectangles touch at edges or corners without overlapping.
    \index{Edge Cases}
    
    \item \textbf{Optimize for Efficiency}: Aim for a constant time \(O(1)\) solution by avoiding unnecessary computations or iterations.
    \index{Efficiency Optimization}
    
    \item \textbf{Avoid Floating-Point Precision Issues}: Since all coordinates are integers, floating-point precision is not a concern, simplifying the implementation.
    \index{Floating-Point Precision}
    
    \item \textbf{Use Helper Functions}: Create helper functions to encapsulate repetitive tasks, such as extracting coordinates or checking specific conditions.
    \index{Helper Functions}
    
    \item \textbf{Code Readability}: Maintain clear and readable code through meaningful variable names and structured logic.
    \index{Code Readability}
    
    \item \textbf{Test Extensively}: Implement a wide range of test cases, including overlapping, non-overlapping, and edge-touching rectangles, to ensure robustness.
    \index{Extensive Testing}
    
    \item \textbf{Understand Axis-Aligned Constraints}: Recognize that axis-aligned rectangles simplify overlap detection compared to rotated rectangles.
    \index{Axis-Aligned Constraints}
    
    \item \textbf{Simplify Logical Conditions}: Combine multiple conditions logically to streamline the overlap detection process.
    \index{Logical Conditions}
\end{itemize}

\section*{Corner and Special Cases to Test When Writing the Code}

When implementing the solution for the \textbf{Rectangle Overlap} problem, it is crucial to consider and rigorously test various edge cases to ensure robustness and correctness:

\begin{itemize}
    \item \textbf{No Overlap}: Rectangles are completely separate.
    \index{No Overlap}
    
    \item \textbf{Partial Overlap}: Rectangles overlap in one or more regions.
    \index{Partial Overlap}
    
    \item \textbf{Edge Touching}: Rectangles touch exactly at one edge without overlapping.
    \index{Edge Touching}
    
    \item \textbf{Corner Touching}: Rectangles touch exactly at one corner without overlapping.
    \index{Corner Touching}
    
    \item \textbf{One Rectangle Inside Another}: One rectangle is entirely within the other.
    \index{Rectangle Inside}
    
    \item \textbf{Identical Rectangles}: Both rectangles have the same coordinates.
    \index{Identical Rectangles}
    
    \item \textbf{Degenerate Rectangles}: Rectangles with zero area (e.g., \(x1 = x2\) or \(y1 = y2\)).
    \index{Degenerate Rectangles}
    
    \item \textbf{Large Coordinates}: Rectangles with very large coordinate values to test performance and integer handling.
    \index{Large Coordinates}
    
    \item \textbf{Negative Coordinates}: Rectangles positioned in negative coordinate space.
    \index{Negative Coordinates}
    
    \item \textbf{Mixed Overlapping Scenarios}: Combinations of the above cases to ensure comprehensive coverage.
    \index{Mixed Overlapping Scenarios}
    
    \item \textbf{Minimum and Maximum Bounds}: Rectangles at the minimum and maximum limits of the coordinate range.
    \index{Minimum and Maximum Bounds}
\end{itemize}

\section*{Implementation Considerations}

When implementing the \texttt{isRectangleOverlap} function, keep in mind the following considerations to ensure robustness and efficiency:

\begin{itemize}
    \item \textbf{Data Type Selection}: Use appropriate data types that can handle the range of input values without overflow or underflow.
    \index{Data Type Selection}
    
    \item \textbf{Optimizing Comparisons}: Structure logical conditions to short-circuit evaluations as soon as a non-overlapping condition is met.
    \index{Optimizing Comparisons}
    
    \item \textbf{Language-Specific Constraints}: Be aware of how the programming language handles integer division and comparisons.
    \index{Language-Specific Constraints}
    
    \item \textbf{Avoiding Redundant Calculations}: Ensure that each comparison contributes towards determining overlap without unnecessary repetitions.
    \index{Avoiding Redundant Calculations}
    
    \item \textbf{Code Readability and Documentation}: Maintain clear and readable code through meaningful variable names and comprehensive comments to facilitate understanding and maintenance.
    \index{Code Readability}
    
    \item \textbf{Edge Case Handling}: Implement checks for edge cases to prevent incorrect results or runtime errors.
    \index{Edge Case Handling}
    
    \item \textbf{Testing and Validation}: Develop a comprehensive suite of test cases that cover all possible scenarios, including edge cases, to validate the correctness and efficiency of the implementation.
    \index{Testing and Validation}
    
    \item \textbf{Scalability}: Design the algorithm to scale efficiently with increasing input sizes, maintaining performance and resource utilization.
    \index{Scalability}
    
    \item \textbf{Using Helper Functions}: Consider creating helper functions for repetitive tasks, such as extracting and comparing coordinates, to enhance modularity and reusability.
    \index{Helper Functions}
    
    \item \textbf{Consistent Naming Conventions}: Use consistent and descriptive naming conventions for variables to improve code clarity.
    \index{Naming Conventions}
    
    \item \textbf{Handling Floating-Point Coordinates}: Although the problem specifies integer coordinates, ensure that the implementation can handle floating-point numbers if needed in extended scenarios.
    \index{Floating-Point Coordinates}
    
    \item \textbf{Avoiding Floating-Point Precision Issues}: Since all coordinates are integers, floating-point precision is not a concern, simplifying the implementation.
    \index{Floating-Point Precision}
    
    \item \textbf{Implementing Unit Tests}: Develop unit tests for each logical condition to ensure that all scenarios are correctly handled.
    \index{Unit Tests}
    
    \item \textbf{Error Handling}: Incorporate error handling to manage invalid inputs gracefully.
    \index{Error Handling}
\end{itemize}

\section*{Conclusion}

The \textbf{Rectangle Overlap} problem exemplifies the application of fundamental geometric principles and conditional logic to solve spatial challenges efficiently. By leveraging simple coordinate comparisons, the algorithm achieves optimal time and space complexities, making it highly suitable for real-time applications such as collision detection in gaming, layout planning in graphics, and spatial data analysis. Understanding and implementing such techniques not only enhances problem-solving skills but also provides a foundation for tackling more complex Computational Geometry problems involving varied geometric shapes and interactions.

\printindex

% \input{sections/rectangle_overlap}
% \input{sections/rectangle_area}
% \input{sections/k_closest_points_to_origin}
% \input{sections/the_skyline_problem}
% % filename: rectangle_area.tex

\problemsection{Rectangle Area}
\label{chap:Rectangle_Area}
\marginnote{\href{https://leetcode.com/problems/rectangle-area/}{[LeetCode Link]}\index{LeetCode}}
\marginnote{\href{https://www.geeksforgeeks.org/find-area-two-overlapping-rectangles/}{[GeeksForGeeks Link]}\index{GeeksForGeeks}}
\marginnote{\href{https://www.interviewbit.com/problems/rectangle-area/}{[InterviewBit Link]}\index{InterviewBit}}
\marginnote{\href{https://app.codesignal.com/challenges/rectangle-area}{[CodeSignal Link]}\index{CodeSignal}}
\marginnote{\href{https://www.codewars.com/kata/rectangle-area/train/python}{[Codewars Link]}\index{Codewars}}

The \textbf{Rectangle Area} problem is a classic Computational Geometry challenge that involves calculating the total area covered by two axis-aligned rectangles in a 2D plane. This problem tests one's ability to perform geometric calculations, handle overlapping scenarios, and implement efficient algorithms. Mastery of this problem is essential for applications in computer graphics, spatial analysis, and computational modeling.

\section*{Problem Statement}

Given two axis-aligned rectangles in a 2D plane, compute the total area covered by the two rectangles. The area covered by the overlapping region should be counted only once.

Each rectangle is represented as a list of four integers \([x1, y1, x2, y2]\), where \((x1, y1)\) are the coordinates of the bottom-left corner, and \((x2, y2)\) are the coordinates of the top-right corner.

\textbf{Function signature in Python:}
\begin{lstlisting}[language=Python]
def computeArea(A: List[int], B: List[int]) -> int:
\end{lstlisting}

\section*{Examples}

\textbf{Example 1:}

\begin{verbatim}
Input: A = [-3,0,3,4], B = [0,-1,9,2]
Output: 45
Explanation:
Area of A = (3 - (-3)) * (4 - 0) = 6 * 4 = 24
Area of B = (9 - 0) * (2 - (-1)) = 9 * 3 = 27
Overlapping Area = (3 - 0) * (2 - 0) = 3 * 2 = 6
Total Area = 24 + 27 - 6 = 45
\end{verbatim}

\textbf{Example 2:}

\begin{verbatim}
Input: A = [0,0,0,0], B = [0,0,0,0]
Output: 0
Explanation:
Both rectangles are degenerate points with zero area.
\end{verbatim}

\textbf{Example 3:}

\begin{verbatim}
Input: A = [0,0,2,2], B = [1,1,3,3]
Output: 7
Explanation:
Area of A = 4
Area of B = 4
Overlapping Area = 1
Total Area = 4 + 4 - 1 = 7
\end{verbatim}

\textbf{Example 4:}

\begin{verbatim}
Input: A = [0,0,1,1], B = [1,0,2,1]
Output: 2
Explanation:
Rectangles touch at the edge but do not overlap.
Area of A = 1
Area of B = 1
Overlapping Area = 0
Total Area = 1 + 1 = 2
\end{verbatim}

\textbf{Constraints:}

\begin{itemize}
    \item All coordinates are integers in the range \([-10^9, 10^9]\).
    \item For each rectangle, \(x1 < x2\) and \(y1 < y2\).
\end{itemize}

LeetCode link: \href{https://leetcode.com/problems/rectangle-area/}{Rectangle Area}\index{LeetCode}

\section*{Algorithmic Approach}

To compute the total area covered by two axis-aligned rectangles, we can follow these steps:

1. **Calculate Individual Areas:**
   - Compute the area of the first rectangle.
   - Compute the area of the second rectangle.

2. **Determine Overlapping Area:**
   - Calculate the coordinates of the overlapping rectangle, if any.
   - If the rectangles overlap, compute the area of the overlapping region.

3. **Compute Total Area:**
   - Sum the individual areas and subtract the overlapping area to avoid double-counting.

\marginnote{This approach ensures accurate area calculation by handling overlapping regions appropriately.}

\section*{Complexities}

\begin{itemize}
    \item \textbf{Time Complexity:} \(O(1)\). The algorithm performs a constant number of calculations.
    
    \item \textbf{Space Complexity:} \(O(1)\). Only a fixed amount of extra space is used for variables.
\end{itemize}

\section*{Python Implementation}

\marginnote{Implementing the area calculation with overlap consideration ensures an accurate and efficient solution.}

Below is the complete Python code implementing the \texttt{computeArea} function:

\begin{fullwidth}
\begin{lstlisting}[language=Python]
from typing import List

class Solution:
    def computeArea(self, A: List[int], B: List[int]) -> int:
        # Calculate area of rectangle A
        areaA = (A[2] - A[0]) * (A[3] - A[1])
        
        # Calculate area of rectangle B
        areaB = (B[2] - B[0]) * (B[3] - B[1])
        
        # Determine overlap coordinates
        overlap_x1 = max(A[0], B[0])
        overlap_y1 = max(A[1], B[1])
        overlap_x2 = min(A[2], B[2])
        overlap_y2 = min(A[3], B[3])
        
        # Calculate overlapping area
        overlap_width = overlap_x2 - overlap_x1
        overlap_height = overlap_y2 - overlap_y1
        overlap_area = 0
        if overlap_width > 0 and overlap_height > 0:
            overlap_area = overlap_width * overlap_height
        
        # Total area is sum of individual areas minus overlapping area
        total_area = areaA + areaB - overlap_area
        return total_area

# Example usage:
solution = Solution()
print(solution.computeArea([-3,0,3,4], [0,-1,9,2]))  # Output: 45
print(solution.computeArea([0,0,0,0], [0,0,0,0]))    # Output: 0
print(solution.computeArea([0,0,2,2], [1,1,3,3]))    # Output: 7
print(solution.computeArea([0,0,1,1], [1,0,2,1]))    # Output: 2
\end{lstlisting}
\end{fullwidth}

This implementation accurately computes the total area covered by two rectangles by accounting for any overlapping regions. It ensures that the overlapping area is not double-counted.

\section*{Explanation}

The \texttt{computeArea} function calculates the combined area of two axis-aligned rectangles by following these steps:

\subsection*{1. Calculate Individual Areas}

\begin{itemize}
    \item **Rectangle A:**
    \begin{itemize}
        \item Width: \(A[2] - A[0]\)
        \item Height: \(A[3] - A[1]\)
        \item Area: Width \(\times\) Height
    \end{itemize}
    
    \item **Rectangle B:**
    \begin{itemize}
        \item Width: \(B[2] - B[0]\)
        \item Height: \(B[3] - B[1]\)
        \item Area: Width \(\times\) Height
    \end{itemize}
\end{itemize}

\subsection*{2. Determine Overlapping Area}

\begin{itemize}
    \item **Overlap Coordinates:**
    \begin{itemize}
        \item Left (x-coordinate): \(\text{max}(A[0], B[0])\)
        \item Bottom (y-coordinate): \(\text{max}(A[1], B[1])\)
        \item Right (x-coordinate): \(\text{min}(A[2], B[2])\)
        \item Top (y-coordinate): \(\text{min}(A[3], B[3])\)
    \end{itemize}
    
    \item **Overlap Dimensions:**
    \begin{itemize}
        \item Width: \(\text{overlap\_x2} - \text{overlap\_x1}\)
        \item Height: \(\text{overlap\_y2} - \text{overlap\_y1}\)
    \end{itemize}
    
    \item **Overlap Area:**
    \begin{itemize}
        \item If both width and height are positive, the rectangles overlap, and the overlapping area is their product.
        \item Otherwise, there is no overlap, and the overlapping area is zero.
    \end{itemize}
\end{itemize}

\subsection*{3. Compute Total Area}

\begin{itemize}
    \item Total Area = Area of Rectangle A + Area of Rectangle B - Overlapping Area
\end{itemize}

\subsection*{4. Example Walkthrough}

Consider the first example:
\begin{verbatim}
Input: A = [-3,0,3,4], B = [0,-1,9,2]
Output: 45
\end{verbatim}

\begin{enumerate}
    \item **Calculate Areas:**
    \begin{itemize}
        \item Area of A = (3 - (-3)) * (4 - 0) = 6 * 4 = 24
        \item Area of B = (9 - 0) * (2 - (-1)) = 9 * 3 = 27
    \end{itemize}
    
    \item **Determine Overlap:**
    \begin{itemize}
        \item overlap\_x1 = max(-3, 0) = 0
        \item overlap\_y1 = max(0, -1) = 0
        \item overlap\_x2 = min(3, 9) = 3
        \item overlap\_y2 = min(4, 2) = 2
        \item overlap\_width = 3 - 0 = 3
        \item overlap\_height = 2 - 0 = 2
        \item overlap\_area = 3 * 2 = 6
    \end{itemize}
    
    \item **Compute Total Area:**
    \begin{itemize}
        \item Total Area = 24 + 27 - 6 = 45
    \end{itemize}
\end{enumerate}

Thus, the function correctly returns \texttt{45}.

\section*{Why This Approach}

This approach is chosen for its straightforwardness and optimal efficiency. By directly calculating the individual areas and intelligently handling the overlapping region, the algorithm ensures accurate results without unnecessary computations. Its constant time complexity makes it highly efficient, even for large coordinate values.

\section*{Alternative Approaches}

\subsection*{1. Using Intersection Dimensions}

Instead of separately calculating areas, directly compute the dimensions of the overlapping region and subtract it from the sum of individual areas.

\begin{lstlisting}[language=Python]
def computeArea(A: List[int], B: List[int]) -> int:
    # Sum of individual areas
    area = (A[2] - A[0]) * (A[3] - A[1]) + (B[2] - B[0]) * (B[3] - B[1])
    
    # Overlapping area
    overlap_width = min(A[2], B[2]) - max(A[0], B[0])
    overlap_height = min(A[3], B[3]) - max(A[1], B[1])
    
    if overlap_width > 0 and overlap_height > 0:
        area -= overlap_width * overlap_height
    
    return area
\end{lstlisting}

\subsection*{2. Using Geometry Libraries}

Leverage computational geometry libraries to handle area calculations and overlapping detections.

\begin{lstlisting}[language=Python]
from shapely.geometry import box

def computeArea(A: List[int], B: List[int]) -> int:
    rect1 = box(A[0], A[1], A[2], A[3])
    rect2 = box(B[0], B[1], B[2], B[3])
    intersection = rect1.intersection(rect2)
    return int(rect1.area + rect2.area - intersection.area)
\end{lstlisting}

\textbf{Note}: This approach requires the \texttt{shapely} library and is more suitable for complex geometric operations.

\section*{Similar Problems to This One}

Several problems involve calculating areas, handling geometric overlaps, and spatial reasoning, utilizing similar algorithmic strategies:

\begin{itemize}
    \item \textbf{Rectangle Overlap}: Determine if two rectangles overlap.
    \item \textbf{Circle Area Overlap}: Calculate the overlapping area between two circles.
    \item \textbf{Polygon Area}: Compute the area of a given polygon.
    \item \textbf{Union of Rectangles}: Calculate the total area covered by multiple rectangles, accounting for overlaps.
    \item \textbf{Intersection of Lines}: Find the intersection point of two lines.
    \item \textbf{Closest Pair of Points}: Find the closest pair of points in a set.
    \item \textbf{Convex Hull}: Compute the convex hull of a set of points.
    \item \textbf{Point Inside Polygon}: Determine if a point lies inside a given polygon.
\end{itemize}

These problems reinforce concepts of geometric calculations, area computations, and efficient algorithm design in various contexts.

\section*{Things to Keep in Mind and Tricks}

When tackling the \textbf{Rectangle Area} problem, consider the following tips and best practices to enhance efficiency and correctness:

\begin{itemize}
    \item \textbf{Understand Geometric Relationships}: Grasp the positional relationships between rectangles to simplify area calculations.
    \index{Geometric Relationships}
    
    \item \textbf{Leverage Coordinate Comparisons}: Use direct comparisons of rectangle coordinates to determine overlapping regions.
    \index{Coordinate Comparisons}
    
    \item \textbf{Handle Overlapping Scenarios}: Accurately calculate the overlapping area to avoid double-counting.
    \index{Overlapping Scenarios}
    
    \item \textbf{Optimize for Efficiency}: Aim for a constant time \(O(1)\) solution by avoiding unnecessary computations or iterations.
    \index{Efficiency Optimization}
    
    \item \textbf{Avoid Floating-Point Precision Issues}: Since all coordinates are integers, floating-point precision is not a concern, simplifying the implementation.
    \index{Floating-Point Precision}
    
    \item \textbf{Use Helper Functions}: Create helper functions to encapsulate repetitive tasks, such as calculating overlap dimensions or areas.
    \index{Helper Functions}
    
    \item \textbf{Code Readability}: Maintain clear and readable code through meaningful variable names and structured logic.
    \index{Code Readability}
    
    \item \textbf{Test Extensively}: Implement a wide range of test cases, including overlapping, non-overlapping, and edge-touching rectangles, to ensure robustness.
    \index{Extensive Testing}
    
    \item \textbf{Understand Axis-Aligned Constraints}: Recognize that axis-aligned rectangles simplify area calculations compared to rotated rectangles.
    \index{Axis-Aligned Constraints}
    
    \item \textbf{Simplify Logical Conditions}: Combine multiple conditions logically to streamline the area calculation process.
    \index{Logical Conditions}
    
    \item \textbf{Use Absolute Values}: When calculating differences, ensure that the dimensions are positive by using absolute values or proper ordering.
    \index{Absolute Values}
    
    \item \textbf{Consider Edge Cases}: Handle cases where rectangles have zero area or touch at edges/corners without overlapping.
    \index{Edge Cases}
\end{itemize}

\section*{Corner and Special Cases to Test When Writing the Code}

When implementing the solution for the \textbf{Rectangle Area} problem, it is crucial to consider and rigorously test various edge cases to ensure robustness and correctness:

\begin{itemize}
    \item \textbf{No Overlap}: Rectangles are completely separate.
    \index{No Overlap}
    
    \item \textbf{Partial Overlap}: Rectangles overlap in one or more regions.
    \index{Partial Overlap}
    
    \item \textbf{Edge Touching}: Rectangles touch exactly at one edge without overlapping.
    \index{Edge Touching}
    
    \item \textbf{Corner Touching}: Rectangles touch exactly at one corner without overlapping.
    \index{Corner Touching}
    
    \item \textbf{One Rectangle Inside Another}: One rectangle is entirely within the other.
    \index{Rectangle Inside}
    
    \item \textbf{Identical Rectangles}: Both rectangles have the same coordinates.
    \index{Identical Rectangles}
    
    \item \textbf{Degenerate Rectangles}: Rectangles with zero area (e.g., \(x1 = x2\) or \(y1 = y2\)).
    \index{Degenerate Rectangles}
    
    \item \textbf{Large Coordinates}: Rectangles with very large coordinate values to test performance and integer handling.
    \index{Large Coordinates}
    
    \item \textbf{Negative Coordinates}: Rectangles positioned in negative coordinate space.
    \index{Negative Coordinates}
    
    \item \textbf{Mixed Overlapping Scenarios}: Combinations of the above cases to ensure comprehensive coverage.
    \index{Mixed Overlapping Scenarios}
    
    \item \textbf{Minimum and Maximum Bounds}: Rectangles at the minimum and maximum limits of the coordinate range.
    \index{Minimum and Maximum Bounds}
    
    \item \textbf{Sequential Rectangles}: Multiple rectangles placed sequentially without overlapping.
    \index{Sequential Rectangles}
    
    \item \textbf{Multiple Overlaps}: Scenarios where more than two rectangles overlap in different regions.
    \index{Multiple Overlaps}
\end{itemize}

\section*{Implementation Considerations}

When implementing the \texttt{computeArea} function, keep in mind the following considerations to ensure robustness and efficiency:

\begin{itemize}
    \item \textbf{Data Type Selection}: Use appropriate data types that can handle large input values without overflow or underflow.
    \index{Data Type Selection}
    
    \item \textbf{Optimizing Comparisons}: Structure logical conditions to efficiently determine overlap dimensions.
    \index{Optimizing Comparisons}
    
    \item \textbf{Handling Large Inputs}: Design the algorithm to efficiently handle large input sizes without significant performance degradation.
    \index{Handling Large Inputs}
    
    \item \textbf{Language-Specific Constraints}: Be aware of how the programming language handles large integers and arithmetic operations.
    \index{Language-Specific Constraints}
    
    \item \textbf{Avoiding Redundant Calculations}: Ensure that each calculation contributes towards determining the final area without unnecessary repetitions.
    \index{Avoiding Redundant Calculations}
    
    \item \textbf{Code Readability and Documentation}: Maintain clear and readable code through meaningful variable names and comprehensive comments to facilitate understanding and maintenance.
    \index{Code Readability}
    
    \item \textbf{Edge Case Handling}: Implement checks for edge cases to prevent incorrect results or runtime errors.
    \index{Edge Case Handling}
    
    \item \textbf{Testing and Validation}: Develop a comprehensive suite of test cases that cover all possible scenarios, including edge cases, to validate the correctness and efficiency of the implementation.
    \index{Testing and Validation}
    
    \item \textbf{Scalability}: Design the algorithm to scale efficiently with increasing input sizes, maintaining performance and resource utilization.
    \index{Scalability}
    
    \item \textbf{Using Helper Functions}: Consider creating helper functions for repetitive tasks, such as calculating overlap dimensions, to enhance modularity and reusability.
    \index{Helper Functions}
    
    \item \textbf{Consistent Naming Conventions}: Use consistent and descriptive naming conventions for variables to improve code clarity.
    \index{Naming Conventions}
    
    \item \textbf{Implementing Unit Tests}: Develop unit tests for each logical condition to ensure that all scenarios are correctly handled.
    \index{Unit Tests}
    
    \item \textbf{Error Handling}: Incorporate error handling to manage invalid inputs gracefully.
    \index{Error Handling}
\end{itemize}

\section*{Conclusion}

The \textbf{Rectangle Area} problem showcases the application of fundamental geometric principles and efficient algorithm design to compute spatial properties accurately. By systematically calculating individual areas and intelligently handling overlapping regions, the algorithm ensures precise results without redundant computations. Understanding and implementing such techniques not only enhances problem-solving skills but also provides a foundation for tackling more complex Computational Geometry challenges involving multiple geometric entities and intricate spatial relationships.

\printindex

% \input{sections/rectangle_overlap}
% \input{sections/rectangle_area}
% \input{sections/k_closest_points_to_origin}
% \input{sections/the_skyline_problem}
% % filename: k_closest_points_to_origin.tex

\problemsection{K Closest Points to Origin}
\label{chap:K_Closest_Points_to_Origin}
\marginnote{\href{https://leetcode.com/problems/k-closest-points-to-origin/}{[LeetCode Link]}\index{LeetCode}}
\marginnote{\href{https://www.geeksforgeeks.org/find-k-closest-points-origin/}{[GeeksForGeeks Link]}\index{GeeksForGeeks}}
\marginnote{\href{https://www.interviewbit.com/problems/k-closest-points/}{[InterviewBit Link]}\index{InterviewBit}}
\marginnote{\href{https://app.codesignal.com/challenges/k-closest-points-to-origin}{[CodeSignal Link]}\index{CodeSignal}}
\marginnote{\href{https://www.codewars.com/kata/k-closest-points-to-origin/train/python}{[Codewars Link]}\index{Codewars}}

The \textbf{K Closest Points to Origin} problem is a popular algorithmic challenge in Computational Geometry that involves identifying the \(k\) points closest to the origin in a 2D plane. This problem tests one's ability to apply efficient sorting and selection algorithms, understand distance computations, and optimize for performance. Mastery of this problem is essential for applications in spatial data analysis, nearest neighbor searches, and clustering algorithms.

\section*{Problem Statement}

Given an array of points where each point is represented as \([x, y]\) in the 2D plane, and an integer \(k\), return the \(k\) closest points to the origin \((0, 0)\).

The distance between two points \((x_1, y_1)\) and \((x_2, y_2)\) is the Euclidean distance \(\sqrt{(x_1 - x_2)^2 + (y_1 - y_2)^2}\). The origin is \((0, 0)\).

\textbf{Function signature in Python:}
\begin{lstlisting}[language=Python]
def kClosest(points: List[List[int]], K: int) -> List[List[int]]:
\end{lstlisting}

\section*{Examples}

\textbf{Example 1:}

\begin{verbatim}
Input: points = [[1,3],[-2,2]], K = 1
Output: [[-2,2]]
Explanation: 
The distance between (1, 3) and the origin is sqrt(10).
The distance between (-2, 2) and the origin is sqrt(8).
Since sqrt(8) < sqrt(10), (-2, 2) is closer to the origin.
\end{verbatim}

\textbf{Example 2:}

\begin{verbatim}
Input: points = [[3,3],[5,-1],[-2,4]], K = 2
Output: [[3,3],[-2,4]]
Explanation: 
The distances are sqrt(18), sqrt(26), and sqrt(20) respectively.
The two closest points are [3,3] and [-2,4].
\end{verbatim}

\textbf{Example 3:}

\begin{verbatim}
Input: points = [[0,1],[1,0]], K = 2
Output: [[0,1],[1,0]]
Explanation: 
Both points are equally close to the origin.
\end{verbatim}

\textbf{Example 4:}

\begin{verbatim}
Input: points = [[1,0],[0,1]], K = 1
Output: [[1,0]]
Explanation: 
Both points are equally close; returning any one is acceptable.
\end{verbatim}

\textbf{Constraints:}

\begin{itemize}
    \item \(1 \leq K \leq \text{points.length} \leq 10^4\)
    \item \(-10^4 < x_i, y_i < 10^4\)
\end{itemize}

LeetCode link: \href{https://leetcode.com/problems/k-closest-points-to-origin/}{K Closest Points to Origin}\index{LeetCode}

\section*{Algorithmic Approach}

To identify the \(k\) closest points to the origin, several algorithmic strategies can be employed. The most efficient methods aim to reduce the time complexity by avoiding the need to sort the entire list of points.

\subsection*{1. Sorting Based on Distance}

Calculate the Euclidean distance of each point from the origin and sort the points based on these distances. Select the first \(k\) points from the sorted list.

\begin{enumerate}
    \item Compute the distance for each point using the formula \(distance = x^2 + y^2\).
    \item Sort the points based on the computed distances.
    \item Return the first \(k\) points from the sorted list.
\end{enumerate}

\subsection*{2. Max Heap (Priority Queue)}

Use a max heap to maintain the \(k\) closest points. Iterate through each point, add it to the heap, and if the heap size exceeds \(k\), remove the farthest point.

\begin{enumerate}
    \item Initialize a max heap.
    \item For each point, compute its distance and add it to the heap.
    \item If the heap size exceeds \(k\), remove the point with the largest distance.
    \item After processing all points, the heap contains the \(k\) closest points.
\end{enumerate}

\subsection*{3. QuickSelect (Quick Sort Partitioning)}

Utilize the QuickSelect algorithm to find the \(k\) closest points without fully sorting the list.

\begin{enumerate}
    \item Choose a pivot point and partition the list based on distances relative to the pivot.
    \item Recursively apply QuickSelect to the partition containing the \(k\) closest points.
    \item Once the \(k\) closest points are identified, return them.
\end{enumerate}

\marginnote{QuickSelect offers an average time complexity of \(O(n)\), making it highly efficient for large datasets.}

\section*{Complexities}

\begin{itemize}
    \item \textbf{Sorting Based on Distance:}
    \begin{itemize}
        \item \textbf{Time Complexity:} \(O(n \log n)\)
        \item \textbf{Space Complexity:} \(O(n)\)
    \end{itemize}
    
    \item \textbf{Max Heap (Priority Queue):}
    \begin{itemize}
        \item \textbf{Time Complexity:} \(O(n \log k)\)
        \item \textbf{Space Complexity:} \(O(k)\)
    \end{itemize}
    
    \item \textbf{QuickSelect (Quick Sort Partitioning):}
    \begin{itemize}
        \item \textbf{Time Complexity:} Average case \(O(n)\), worst case \(O(n^2)\)
        \item \textbf{Space Complexity:} \(O(1)\) (in-place)
    \end{itemize}
\end{itemize}

\section*{Python Implementation}

\marginnote{Implementing QuickSelect provides an optimal average-case solution with linear time complexity.}

Below is the complete Python code implementing the \texttt{kClosest} function using the QuickSelect approach:

\begin{fullwidth}
\begin{lstlisting}[language=Python]
from typing import List
import random

class Solution:
    def kClosest(self, points: List[List[int]], K: int) -> List[List[int]]:
        def quickselect(left, right, K_smallest):
            if left == right:
                return
            
            # Select a random pivot_index
            pivot_index = random.randint(left, right)
            
            # Partition the array
            pivot_index = partition(left, right, pivot_index)
            
            # The pivot is in its final sorted position
            if K_smallest == pivot_index:
                return
            elif K_smallest < pivot_index:
                quickselect(left, pivot_index - 1, K_smallest)
            else:
                quickselect(pivot_index + 1, right, K_smallest)
        
        def partition(left, right, pivot_index):
            pivot_distance = distance(points[pivot_index])
            # Move pivot to end
            points[pivot_index], points[right] = points[right], points[pivot_index]
            store_index = left
            for i in range(left, right):
                if distance(points[i]) < pivot_distance:
                    points[store_index], points[i] = points[i], points[store_index]
                    store_index += 1
            # Move pivot to its final place
            points[right], points[store_index] = points[store_index], points[right]
            return store_index
        
        def distance(point):
            return point[0] ** 2 + point[1] ** 2
        
        n = len(points)
        quickselect(0, n - 1, K)
        return points[:K]

# Example usage:
solution = Solution()
print(solution.kClosest([[1,3],[-2,2]], 1))            # Output: [[-2,2]]
print(solution.kClosest([[3,3],[5,-1],[-2,4]], 2))     # Output: [[3,3],[-2,4]]
print(solution.kClosest([[0,1],[1,0]], 2))             # Output: [[0,1],[1,0]]
print(solution.kClosest([[1,0],[0,1]], 1))             # Output: [[1,0]] or [[0,1]]
\end{lstlisting}
\end{fullwidth}

This implementation uses the QuickSelect algorithm to efficiently find the \(k\) closest points to the origin without fully sorting the entire list. It ensures optimal performance even with large datasets.

\section*{Explanation}

The \texttt{kClosest} function identifies the \(k\) closest points to the origin using the QuickSelect algorithm. Here's a detailed breakdown of the implementation:

\subsection*{1. Distance Calculation}

\begin{itemize}
    \item The Euclidean distance is calculated as \(distance = x^2 + y^2\). Since we only need relative distances for comparison, the square root is omitted for efficiency.
\end{itemize}

\subsection*{2. QuickSelect Algorithm}

\begin{itemize}
    \item **Pivot Selection:**
    \begin{itemize}
        \item A random pivot is chosen to enhance the average-case performance.
    \end{itemize}
    
    \item **Partitioning:**
    \begin{itemize}
        \item The array is partitioned such that points with distances less than the pivot are moved to the left, and others to the right.
        \item The pivot is placed in its correct sorted position.
    \end{itemize}
    
    \item **Recursive Selection:**
    \begin{itemize}
        \item If the pivot's position matches \(K\), the selection is complete.
        \item Otherwise, recursively apply QuickSelect to the relevant partition.
    \end{itemize}
\end{itemize}

\subsection*{3. Final Selection}

\begin{itemize}
    \item After partitioning, the first \(K\) points in the list are the \(k\) closest points to the origin.
\end{itemize}

\subsection*{4. Example Walkthrough}

Consider the first example:
\begin{verbatim}
Input: points = [[1,3],[-2,2]], K = 1
Output: [[-2,2]]
\end{verbatim}

\begin{enumerate}
    \item **Calculate Distances:**
    \begin{itemize}
        \item [1,3] : \(1^2 + 3^2 = 10\)
        \item [-2,2] : \((-2)^2 + 2^2 = 8\)
    \end{itemize}
    
    \item **QuickSelect Process:**
    \begin{itemize}
        \item Choose a pivot, say [1,3] with distance 10.
        \item Compare and rearrange:
        \begin{itemize}
            \item [-2,2] has a smaller distance (8) and is moved to the left.
        \end{itemize}
        \item After partitioning, the list becomes [[-2,2], [1,3]].
        \item Since \(K = 1\), return the first point: [[-2,2]].
    \end{itemize}
\end{enumerate}

Thus, the function correctly identifies \([-2,2]\) as the closest point to the origin.

\section*{Why This Approach}

The QuickSelect algorithm is chosen for its average-case linear time complexity \(O(n)\), making it highly efficient for large datasets compared to sorting-based methods with \(O(n \log n)\) time complexity. By avoiding the need to sort the entire list, QuickSelect provides an optimal solution for finding the \(k\) closest points.

\section*{Alternative Approaches}

\subsection*{1. Sorting Based on Distance}

Sort all points based on their distances from the origin and select the first \(k\) points.

\begin{lstlisting}[language=Python]
class Solution:
    def kClosest(self, points: List[List[int]], K: int) -> List[List[int]]:
        points.sort(key=lambda P: P[0]**2 + P[1]**2)
        return points[:K]
\end{lstlisting}

\textbf{Complexities:}
\begin{itemize}
    \item \textbf{Time Complexity:} \(O(n \log n)\)
    \item \textbf{Space Complexity:} \(O(1)\)
\end{itemize}

\subsection*{2. Max Heap (Priority Queue)}

Use a max heap to maintain the \(k\) closest points.

\begin{lstlisting}[language=Python]
import heapq

class Solution:
    def kClosest(self, points: List[List[int]], K: int) -> List[List[int]]:
        heap = []
        for (x, y) in points:
            dist = -(x**2 + y**2)  # Max heap using negative distances
            heapq.heappush(heap, (dist, [x, y]))
            if len(heap) > K:
                heapq.heappop(heap)
        return [item[1] for item in heap]
\end{lstlisting}

\textbf{Complexities:}
\begin{itemize}
    \item \textbf{Time Complexity:} \(O(n \log k)\)
    \item \textbf{Space Complexity:} \(O(k)\)
\end{itemize}

\subsection*{3. Using Built-In Functions}

Leverage built-in functions for distance calculation and selection.

\begin{lstlisting}[language=Python]
import math

class Solution:
    def kClosest(self, points: List[List[int]], K: int) -> List[List[int]]:
        points.sort(key=lambda P: math.sqrt(P[0]**2 + P[1]**2))
        return points[:K]
\end{lstlisting}

\textbf{Note}: This method is similar to the sorting approach but uses the actual Euclidean distance.

\section*{Similar Problems to This One}

Several problems involve nearest neighbor searches, spatial data analysis, and efficient selection algorithms, utilizing similar algorithmic strategies:

\begin{itemize}
    \item \textbf{Closest Pair of Points}: Find the closest pair of points in a set.
    \item \textbf{Top K Frequent Elements}: Identify the most frequent elements in a dataset.
    \item \textbf{Kth Largest Element in an Array}: Find the \(k\)-th largest element in an unsorted array.
    \item \textbf{Sliding Window Maximum}: Find the maximum in each sliding window of size \(k\) over an array.
    \item \textbf{Merge K Sorted Lists}: Merge multiple sorted lists into a single sorted list.
    \item \textbf{Find Median from Data Stream}: Continuously find the median of a stream of numbers.
    \item \textbf{Top K Closest Stars}: Find the \(k\) closest stars to Earth based on their distances.
\end{itemize}

These problems reinforce concepts of efficient selection, heap usage, and distance computations in various contexts.

\section*{Things to Keep in Mind and Tricks}

When solving the \textbf{K Closest Points to Origin} problem, consider the following tips and best practices to enhance efficiency and correctness:

\begin{itemize}
    \item \textbf{Understand Distance Calculations}: Grasp the Euclidean distance formula and recognize that the square root can be omitted for comparison purposes.
    \index{Distance Calculations}
    
    \item \textbf{Leverage Efficient Algorithms}: Use QuickSelect or heap-based methods to optimize time complexity, especially for large datasets.
    \index{Efficient Algorithms}
    
    \item \textbf{Handle Ties Appropriately}: Decide how to handle points with identical distances when \(k\) is less than the number of such points.
    \index{Handling Ties}
    
    \item \textbf{Optimize Space Usage}: Choose algorithms that minimize additional space, such as in-place QuickSelect.
    \index{Space Optimization}
    
    \item \textbf{Use Appropriate Data Structures}: Utilize heaps, lists, and helper functions effectively to manage and process data.
    \index{Data Structures}
    
    \item \textbf{Implement Helper Functions}: Create helper functions for distance calculation and partitioning to enhance code modularity.
    \index{Helper Functions}
    
    \item \textbf{Code Readability}: Maintain clear and readable code through meaningful variable names and structured logic.
    \index{Code Readability}
    
    \item \textbf{Test Extensively}: Implement a wide range of test cases, including edge cases like multiple points with the same distance, to ensure robustness.
    \index{Extensive Testing}
    
    \item \textbf{Understand Algorithm Trade-offs}: Recognize the trade-offs between different approaches in terms of time and space complexities.
    \index{Algorithm Trade-offs}
    
    \item \textbf{Use Built-In Sorting Functions}: When using sorting-based approaches, leverage built-in functions for efficiency and simplicity.
    \index{Built-In Sorting}
    
    \item \textbf{Avoid Redundant Calculations}: Ensure that distance calculations are performed only when necessary to optimize performance.
    \index{Avoiding Redundant Calculations}
    
    \item \textbf{Language-Specific Features}: Utilize language-specific features or libraries that can simplify implementation, such as heapq in Python.
    \index{Language-Specific Features}
\end{itemize}

\section*{Corner and Special Cases to Test When Writing the Code}

When implementing the solution for the \textbf{K Closest Points to Origin} problem, it is crucial to consider and rigorously test various edge cases to ensure robustness and correctness:

\begin{itemize}
    \item \textbf{Multiple Points with Same Distance}: Ensure that the algorithm handles multiple points having the same distance from the origin.
    \index{Same Distance Points}
    
    \item \textbf{Points at Origin}: Include points that are exactly at the origin \((0,0)\).
    \index{Points at Origin}
    
    \item \textbf{Negative Coordinates}: Ensure that the algorithm correctly computes distances for points with negative \(x\) or \(y\) coordinates.
    \index{Negative Coordinates}
    
    \item \textbf{Large Coordinates}: Test with points having very large or very small coordinate values to verify integer handling.
    \index{Large Coordinates}
    
    \item \textbf{K Equals Number of Points}: When \(K\) is equal to the number of points, the algorithm should return all points.
    \index{K Equals Number of Points}
    
    \item \textbf{K is One}: Test with \(K = 1\) to ensure the closest point is correctly identified.
    \index{K is One}
    
    \item \textbf{All Points Same}: All points have the same coordinates.
    \index{All Points Same}
    
    \item \textbf{K is Zero}: Although \(K\) is defined to be at least 1, ensure that the algorithm gracefully handles \(K = 0\) if allowed.
    \index{K is Zero}
    
    \item \textbf{Single Point}: Only one point is provided, and \(K = 1\).
    \index{Single Point}
    
    \item \textbf{Mixed Coordinates}: Points with a mix of positive and negative coordinates.
    \index{Mixed Coordinates}
    
    \item \textbf{Points with Zero Distance}: Multiple points at the origin.
    \index{Zero Distance Points}
    
    \item \textbf{Sparse and Dense Points}: Densely packed points and sparsely distributed points.
    \index{Sparse and Dense Points}
    
    \item \textbf{Duplicate Points}: Multiple identical points in the input list.
    \index{Duplicate Points}
    
    \item \textbf{K Greater Than Number of Unique Points}: Ensure that the algorithm handles cases where \(K\) exceeds the number of unique points if applicable.
    \index{K Greater Than Unique Points}
\end{itemize}

\section*{Implementation Considerations}

When implementing the \texttt{kClosest} function, keep in mind the following considerations to ensure robustness and efficiency:

\begin{itemize}
    \item \textbf{Data Type Selection}: Use appropriate data types that can handle large input values without overflow or precision loss.
    \index{Data Type Selection}
    
    \item \textbf{Optimizing Distance Calculations}: Avoid calculating the square root since it is unnecessary for comparison purposes.
    \index{Optimizing Distance Calculations}
    
    \item \textbf{Choosing the Right Algorithm}: Select an algorithm based on the size of the input and the value of \(K\) to optimize time and space complexities.
    \index{Choosing the Right Algorithm}
    
    \item \textbf{Language-Specific Libraries}: Utilize language-specific libraries and functions (e.g., \texttt{heapq} in Python) to simplify implementation and enhance performance.
    \index{Language-Specific Libraries}
    
    \item \textbf{Avoiding Redundant Calculations}: Ensure that each point's distance is calculated only once to optimize performance.
    \index{Avoiding Redundant Calculations}
    
    \item \textbf{Implementing Helper Functions}: Create helper functions for tasks like distance calculation and partitioning to enhance modularity and readability.
    \index{Helper Functions}
    
    \item \textbf{Edge Case Handling}: Implement checks for edge cases to prevent incorrect results or runtime errors.
    \index{Edge Case Handling}
    
    \item \textbf{Testing and Validation}: Develop a comprehensive suite of test cases that cover all possible scenarios, including edge cases, to validate the correctness and efficiency of the implementation.
    \index{Testing and Validation}
    
    \item \textbf{Scalability}: Design the algorithm to scale efficiently with increasing input sizes, maintaining performance and resource utilization.
    \index{Scalability}
    
    \item \textbf{Consistent Naming Conventions}: Use consistent and descriptive naming conventions for variables and functions to improve code clarity.
    \index{Naming Conventions}
    
    \item \textbf{Memory Management}: Ensure that the algorithm manages memory efficiently, especially when dealing with large datasets.
    \index{Memory Management}
    
    \item \textbf{Avoiding Stack Overflow}: If implementing recursive approaches, be mindful of recursion limits and potential stack overflow issues.
    \index{Avoiding Stack Overflow}
    
    \item \textbf{Implementing Iterative Solutions}: Prefer iterative solutions when recursion may lead to increased space complexity or stack overflow.
    \index{Implementing Iterative Solutions}
\end{itemize}

\section*{Conclusion}

The \textbf{K Closest Points to Origin} problem exemplifies the application of efficient selection algorithms and geometric computations to solve spatial challenges effectively. By leveraging QuickSelect or heap-based methods, the algorithm achieves optimal time and space complexities, making it highly suitable for large datasets. Understanding and implementing such techniques not only enhances problem-solving skills but also provides a foundation for tackling more advanced Computational Geometry problems involving nearest neighbor searches, clustering, and spatial data analysis.

\printindex

% \input{sections/rectangle_overlap}
% \input{sections/rectangle_area}
% \input{sections/k_closest_points_to_origin}
% \input{sections/the_skyline_problem}
% % filename: the_skyline_problem.tex

\problemsection{The Skyline Problem}
\label{chap:The_Skyline_Problem}
\marginnote{\href{https://leetcode.com/problems/the-skyline-problem/}{[LeetCode Link]}\index{LeetCode}}
\marginnote{\href{https://www.geeksforgeeks.org/the-skyline-problem/}{[GeeksForGeeks Link]}\index{GeeksForGeeks}}
\marginnote{\href{https://www.interviewbit.com/problems/the-skyline-problem/}{[InterviewBit Link]}\index{InterviewBit}}
\marginnote{\href{https://app.codesignal.com/challenges/the-skyline-problem}{[CodeSignal Link]}\index{CodeSignal}}
\marginnote{\href{https://www.codewars.com/kata/the-skyline-problem/train/python}{[Codewars Link]}\index{Codewars}}

The \textbf{Skyline Problem} is a complex Computational Geometry challenge that involves computing the skyline formed by a collection of buildings in a 2D cityscape. Each building is represented by its left and right x-coordinates and its height. The skyline is defined by a list of "key points" where the height changes. This problem tests one's ability to handle large datasets, implement efficient sweep line algorithms, and manage event-driven processing. Mastery of this problem is essential for applications in computer graphics, urban planning simulations, and geographic information systems (GIS).

\section*{Problem Statement}

You are given a list of buildings in a cityscape. Each building is represented as a triplet \([Li, Ri, Hi]\), where \(Li\) and \(Ri\) are the x-coordinates of the left and right edges of the building, respectively, and \(Hi\) is the height of the building.

The skyline should be represented as a list of key points \([x, y]\) in sorted order by \(x\)-coordinate, where \(y\) is the height of the skyline at that point. The skyline should only include critical points where the height changes.

\textbf{Function signature in Python:}
\begin{lstlisting}[language=Python]
def getSkyline(buildings: List[List[int]]) -> List[List[int]]:
\end{lstlisting}

\section*{Examples}

\textbf{Example 1:}

\begin{verbatim}
Input: buildings = [[2,9,10], [3,7,15], [5,12,12], [15,20,10], [19,24,8]]
Output: [[2,10], [3,15], [7,12], [12,0], [15,10], [20,8], [24,0]]
Explanation:
- At x=2, the first building starts, height=10.
- At x=3, the second building starts, height=15.
- At x=7, the second building ends, the third building is still ongoing, height=12.
- At x=12, the third building ends, height drops to 0.
- At x=15, the fourth building starts, height=10.
- At x=20, the fourth building ends, the fifth building is still ongoing, height=8.
- At x=24, the fifth building ends, height drops to 0.
\end{verbatim}

\textbf{Example 2:}

\begin{verbatim}
Input: buildings = [[0,2,3], [2,5,3]]
Output: [[0,3], [5,0]]
Explanation:
- The two buildings are contiguous and have the same height, so the skyline drops to 0 at x=5.
\end{verbatim}

\textbf{Example 3:}

\begin{verbatim}
Input: buildings = [[1,3,3], [2,4,4], [5,6,1]]
Output: [[1,3], [2,4], [4,0], [5,1], [6,0]]
Explanation:
- At x=1, first building starts, height=3.
- At x=2, second building starts, height=4.
- At x=4, second building ends, height drops to 0.
- At x=5, third building starts, height=1.
- At x=6, third building ends, height drops to 0.
\end{verbatim}

\textbf{Example 4:}

\begin{verbatim}
Input: buildings = [[0,5,0]]
Output: []
Explanation:
- A building with height 0 does not contribute to the skyline.
\end{verbatim}

\textbf{Constraints:}

\begin{itemize}
    \item \(1 \leq \text{buildings.length} \leq 10^4\)
    \item \(0 \leq Li < Ri \leq 10^9\)
    \item \(0 \leq Hi \leq 10^4\)
\end{itemize}

\section*{Algorithmic Approach}

The \textbf{Sweep Line Algorithm} is an efficient method for solving the Skyline Problem. It involves processing events (building start and end points) in sorted order while maintaining a data structure (typically a max heap) to keep track of active buildings. Here's a step-by-step approach:

\subsection*{1. Event Representation}

Transform each building into two events:
\begin{itemize}
    \item **Start Event:** \((Li, -Hi)\) – Negative height indicates a building starts.
    \item **End Event:** \((Ri, Hi)\) – Positive height indicates a building ends.
\end{itemize}

Sorting the events ensures that start events are processed before end events at the same x-coordinate, and taller buildings are processed before shorter ones.

\subsection*{2. Sorting the Events}

Sort all events based on:
\begin{enumerate}
    \item **x-coordinate:** Ascending order.
    \item **Height:**
    \begin{itemize}
        \item For start events, taller buildings come first.
        \item For end events, shorter buildings come first.
    \end{itemize}
\end{enumerate}

\subsection*{3. Processing the Events}

Use a max heap to keep track of active building heights. Iterate through the sorted events:
\begin{enumerate}
    \item **Start Event:**
    \begin{itemize}
        \item Add the building's height to the heap.
    \end{itemize}
    
    \item **End Event:**
    \begin{itemize}
        \item Remove the building's height from the heap.
    \end{itemize}
    
    \item **Determine Current Max Height:**
    \begin{itemize}
        \item The current max height is the top of the heap.
    \end{itemize}
    
    \item **Update Skyline:**
    \begin{itemize}
        \item If the current max height differs from the previous max height, add a new key point \([x, current\_max\_height]\).
    \end{itemize}
\end{enumerate}

\subsection*{4. Finalizing the Skyline}

After processing all events, the accumulated key points represent the skyline.

\marginnote{The Sweep Line Algorithm efficiently handles dynamic changes in active buildings, ensuring accurate skyline construction.}

\section*{Complexities}

\begin{itemize}
    \item \textbf{Time Complexity:} \(O(n \log n)\), where \(n\) is the number of buildings. Sorting the events takes \(O(n \log n)\), and each heap operation takes \(O(\log n)\).
    
    \item \textbf{Space Complexity:} \(O(n)\), due to the storage of events and the heap.
\end{itemize}

\section*{Python Implementation}

\marginnote{Implementing the Sweep Line Algorithm with a max heap ensures an efficient and accurate solution.}

Below is the complete Python code implementing the \texttt{getSkyline} function:

\begin{fullwidth}
\begin{lstlisting}[language=Python]
from typing import List
import heapq

class Solution:
    def getSkyline(self, buildings: List[List[int]]) -> List[List[int]]:
        # Create a list of all events
        # For start events, use negative height to ensure they are processed before end events
        events = []
        for L, R, H in buildings:
            events.append((L, -H))
            events.append((R, H))
        
        # Sort the events
        # First by x-coordinate, then by height
        events.sort()
        
        # Max heap to keep track of active buildings
        heap = [0]  # Initialize with ground level
        heapq.heapify(heap)
        active_heights = {0: 1}  # Dictionary to count heights
        
        result = []
        prev_max = 0
        
        for x, h in events:
            if h < 0:
                # Start of a building, add height to heap and dictionary
                heapq.heappush(heap, h)
                active_heights[h] = active_heights.get(h, 0) + 1
            else:
                # End of a building, remove height from dictionary
                active_heights[h] -= 1
                if active_heights[h] == 0:
                    del active_heights[h]
            
            # Current max height
            while heap and active_heights.get(heap[0], 0) == 0:
                heapq.heappop(heap)
            current_max = -heap[0] if heap else 0
            
            # If the max height has changed, add to result
            if current_max != prev_max:
                result.append([x, current_max])
                prev_max = current_max
        
        return result

# Example usage:
solution = Solution()
print(solution.getSkyline([[2,9,10], [3,7,15], [5,12,12], [15,20,10], [19,24,8]]))
# Output: [[2,10], [3,15], [7,12], [12,0], [15,10], [20,8], [24,0]]

print(solution.getSkyline([[0,2,3], [2,5,3]]))
# Output: [[0,3], [5,0]]

print(solution.getSkyline([[1,3,3], [2,4,4], [5,6,1]]))
# Output: [[1,3], [2,4], [4,0], [5,1], [6,0]]

print(solution.getSkyline([[0,5,0]]))
# Output: []
\end{lstlisting}
\end{fullwidth}

This implementation efficiently constructs the skyline by processing all building events in sorted order and maintaining active building heights using a max heap. It ensures that only critical points where the skyline changes are recorded.

\section*{Explanation}

The \texttt{getSkyline} function constructs the skyline formed by a set of buildings by leveraging the Sweep Line Algorithm and a max heap to track active buildings. Here's a detailed breakdown of the implementation:

\subsection*{1. Event Representation}

\begin{itemize}
    \item Each building is transformed into two events:
    \begin{itemize}
        \item **Start Event:** \((Li, -Hi)\) – Negative height indicates the start of a building.
        \item **End Event:** \((Ri, Hi)\) – Positive height indicates the end of a building.
    \end{itemize}
\end{itemize}

\subsection*{2. Sorting the Events}

\begin{itemize}
    \item Events are sorted primarily by their x-coordinate in ascending order.
    \item For events with the same x-coordinate:
    \begin{itemize}
        \item Start events (with negative heights) are processed before end events.
        \item Taller buildings are processed before shorter ones.
    \end{itemize}
\end{itemize}

\subsection*{3. Processing the Events}

\begin{itemize}
    \item **Heap Initialization:**
    \begin{itemize}
        \item A max heap is initialized with a ground level height of 0.
        \item A dictionary \texttt{active\_heights} tracks the count of active building heights.
    \end{itemize}
    
    \item **Iterating Through Events:**
    \begin{enumerate}
        \item **Start Event:**
        \begin{itemize}
            \item Add the building's height to the heap.
            \item Increment the count of the height in \texttt{active\_heights}.
        \end{itemize}
        
        \item **End Event:**
        \begin{itemize}
            \item Decrement the count of the building's height in \texttt{active\_heights}.
            \item If the count reaches zero, remove the height from the dictionary.
        \end{itemize}
        
        \item **Determine Current Max Height:**
        \begin{itemize}
            \item Remove heights from the heap that are no longer active.
            \item The current max height is the top of the heap.
        \end{itemize}
        
        \item **Update Skyline:**
        \begin{itemize}
            \item If the current max height differs from the previous max height, add a new key point \([x, current\_max\_height]\).
        \end{itemize}
    \end{enumerate}
\end{itemize}

\subsection*{4. Finalizing the Skyline}

\begin{itemize}
    \item After processing all events, the \texttt{result} list contains the key points defining the skyline.
\end{itemize}

\subsection*{5. Example Walkthrough}

Consider the first example:
\begin{verbatim}
Input: buildings = [[2,9,10], [3,7,15], [5,12,12], [15,20,10], [19,24,8]]
Output: [[2,10], [3,15], [7,12], [12,0], [15,10], [20,8], [24,0]]
\end{verbatim}

\begin{enumerate}
    \item **Event Transformation:**
    \begin{itemize}
        \item \((2, -10)\), \((9, 10)\)
        \item \((3, -15)\), \((7, 15)\)
        \item \((5, -12)\), \((12, 12)\)
        \item \((15, -10)\), \((20, 10)\)
        \item \((19, -8)\), \((24, 8)\)
    \end{itemize}
    
    \item **Sorting Events:**
    \begin{itemize}
        \item Sorted order: \((2, -10)\), \((3, -15)\), \((5, -12)\), \((7, 15)\), \((9, 10)\), \((12, 12)\), \((15, -10)\), \((19, -8)\), \((20, 10)\), \((24, 8)\)
    \end{itemize}
    
    \item **Processing Events:**
    \begin{itemize}
        \item At each event, update the heap and determine if the skyline height changes.
    \end{itemize}
    
    \item **Result Construction:**
    \begin{itemize}
        \item The resulting skyline key points are accumulated as \([[2,10], [3,15], [7,12], [12,0], [15,10], [20,8], [24,0]]\).
    \end{itemize}
\end{enumerate}

Thus, the function correctly constructs the skyline based on the buildings' positions and heights.

\section*{Why This Approach}

The Sweep Line Algorithm combined with a max heap offers an optimal solution with \(O(n \log n)\) time complexity and efficient handling of overlapping buildings. By processing events in sorted order and maintaining active building heights, the algorithm ensures that all critical points in the skyline are accurately identified without redundant computations.

\section*{Alternative Approaches}

\subsection*{1. Divide and Conquer}

Divide the set of buildings into smaller subsets, compute the skyline for each subset, and then merge the skylines.

\begin{lstlisting}[language=Python]
class Solution:
    def getSkyline(self, buildings: List[List[int]]) -> List[List[int]]:
        def merge(left, right):
            h1, h2 = 0, 0
            i, j = 0, 0
            merged = []
            while i < len(left) and j < len(right):
                if left[i][0] < right[j][0]:
                    x, h1 = left[i]
                    i += 1
                elif left[i][0] > right[j][0]:
                    x, h2 = right[j]
                    j += 1
                else:
                    x, h1 = left[i]
                    _, h2 = right[j]
                    i += 1
                    j += 1
                max_h = max(h1, h2)
                if not merged or merged[-1][1] != max_h:
                    merged.append([x, max_h])
            merged.extend(left[i:])
            merged.extend(right[j:])
            return merged
        
        def divide(buildings):
            if not buildings:
                return []
            if len(buildings) == 1:
                L, R, H = buildings[0]
                return [[L, H], [R, 0]]
            mid = len(buildings) // 2
            left = divide(buildings[:mid])
            right = divide(buildings[mid:])
            return merge(left, right)
        
        return divide(buildings)
\end{lstlisting}

\textbf{Complexities:}
\begin{itemize}
    \item \textbf{Time Complexity:} \(O(n \log n)\)
    \item \textbf{Space Complexity:} \(O(n)\)
\end{itemize}

\subsection*{2. Using Segment Trees}

Implement a segment tree to manage and query overlapping building heights dynamically.

\textbf{Note}: This approach is more complex and is generally used for advanced scenarios with multiple dynamic queries.

\section*{Similar Problems to This One}

Several problems involve skyline-like constructions, spatial data analysis, and efficient event processing, utilizing similar algorithmic strategies:

\begin{itemize}
    \item \textbf{Merge Intervals}: Merge overlapping intervals in a list.
    \item \textbf{Largest Rectangle in Histogram}: Find the largest rectangular area in a histogram.
    \item \textbf{Interval Partitioning}: Assign intervals to resources without overlap.
    \item \textbf{Line Segment Intersection}: Detect intersections among line segments.
    \item \textbf{Closest Pair of Points}: Find the closest pair of points in a set.
    \item \textbf{Convex Hull}: Compute the convex hull of a set of points.
    \item \textbf{Point Inside Polygon}: Determine if a point lies inside a given polygon.
    \item \textbf{Range Searching}: Efficiently query geometric data within a specified range.
\end{itemize}

These problems reinforce concepts of event-driven processing, spatial reasoning, and efficient algorithm design in various contexts.

\section*{Things to Keep in Mind and Tricks}

When tackling the \textbf{Skyline Problem}, consider the following tips and best practices to enhance efficiency and correctness:

\begin{itemize}
    \item \textbf{Understand Sweep Line Technique}: Grasp how the sweep line algorithm processes events in sorted order to handle dynamic changes efficiently.
    \index{Sweep Line Technique}
    
    \item \textbf{Leverage Priority Queues (Heaps)}: Use max heaps to keep track of active buildings' heights, enabling quick access to the current maximum height.
    \index{Priority Queues}
    
    \item \textbf{Handle Start and End Events Differently}: Differentiate between building start and end events to accurately manage active heights.
    \index{Start and End Events}
    
    \item \textbf{Optimize Event Sorting}: Sort events primarily by x-coordinate and secondarily by height to ensure correct processing order.
    \index{Event Sorting}
    
    \item \textbf{Manage Active Heights Efficiently}: Use data structures that allow efficient insertion, deletion, and retrieval of maximum elements.
    \index{Active Heights Management}
    
    \item \textbf{Avoid Redundant Key Points}: Only record key points when the skyline height changes to minimize the output list.
    \index{Avoiding Redundant Key Points}
    
    \item \textbf{Implement Helper Functions}: Create helper functions for tasks like distance calculation, event handling, and heap management to enhance modularity.
    \index{Helper Functions}
    
    \item \textbf{Code Readability}: Maintain clear and readable code through meaningful variable names and structured logic.
    \index{Code Readability}
    
    \item \textbf{Test Extensively}: Implement a wide range of test cases, including overlapping, non-overlapping, and edge-touching buildings, to ensure robustness.
    \index{Extensive Testing}
    
    \item \textbf{Handle Degenerate Cases}: Manage cases where buildings have zero height or identical coordinates gracefully.
    \index{Degenerate Cases}
    
    \item \textbf{Understand Geometric Relationships}: Grasp how buildings overlap and influence the skyline to simplify the algorithm.
    \index{Geometric Relationships}
    
    \item \textbf{Use Appropriate Data Structures}: Utilize appropriate data structures like heaps, lists, and dictionaries to manage and process data efficiently.
    \index{Appropriate Data Structures}
    
    \item \textbf{Optimize for Large Inputs}: Design the algorithm to handle large numbers of buildings without significant performance degradation.
    \index{Optimizing for Large Inputs}
    
    \item \textbf{Implement Iterative Solutions Carefully}: Ensure that loop conditions are correctly defined to prevent infinite loops or incorrect terminations.
    \index{Iterative Solutions}
    
    \item \textbf{Consistent Naming Conventions}: Use consistent and descriptive naming conventions for variables and functions to improve code clarity.
    \index{Naming Conventions}
\end{itemize}

\section*{Corner and Special Cases to Test When Writing the Code}

When implementing the solution for the \textbf{Skyline Problem}, it is crucial to consider and rigorously test various edge cases to ensure robustness and correctness:

\begin{itemize}
    \item \textbf{No Overlapping Buildings}: All buildings are separate and do not overlap.
    \index{No Overlapping Buildings}
    
    \item \textbf{Fully Overlapping Buildings}: Multiple buildings completely overlap each other.
    \index{Fully Overlapping Buildings}
    
    \item \textbf{Buildings Touching at Edges}: Buildings share common edges without overlapping.
    \index{Buildings Touching at Edges}
    
    \item \textbf{Buildings Touching at Corners}: Buildings share common corners without overlapping.
    \index{Buildings Touching at Corners}
    
    \item \textbf{Single Building}: Only one building is present.
    \index{Single Building}
    
    \item \textbf{Multiple Buildings with Same Start or End}: Multiple buildings start or end at the same x-coordinate.
    \index{Same Start or End}
    
    \item \textbf{Buildings with Zero Height}: Buildings that have zero height should not affect the skyline.
    \index{Buildings with Zero Height}
    
    \item \textbf{Large Number of Buildings}: Test with a large number of buildings to ensure performance and scalability.
    \index{Large Number of Buildings}
    
    \item \textbf{Buildings with Negative Coordinates}: Buildings positioned in negative coordinate space.
    \index{Negative Coordinates}
    
    \item \textbf{Boundary Values}: Buildings at the minimum and maximum limits of the coordinate range.
    \index{Boundary Values}
    
    \item \textbf{Buildings with Identical Coordinates}: Multiple buildings with the same coordinates.
    \index{Identical Coordinates}
    
    \item \textbf{Sequential Buildings}: Buildings placed sequentially without gaps.
    \index{Sequential Buildings}
    
    \item \textbf{Overlapping and Non-Overlapping Mixed}: A mix of overlapping and non-overlapping buildings.
    \index{Overlapping and Non-Overlapping Mixed}
    
    \item \textbf{Buildings with Very Large Heights}: Buildings with heights at the upper limit of the constraints.
    \index{Very Large Heights}
    
    \item \textbf{Empty Input}: No buildings are provided.
    \index{Empty Input}
\end{itemize}

\section*{Implementation Considerations}

When implementing the \texttt{getSkyline} function, keep in mind the following considerations to ensure robustness and efficiency:

\begin{itemize}
    \item \textbf{Data Type Selection}: Use appropriate data types that can handle large input values and avoid overflow or precision issues.
    \index{Data Type Selection}
    
    \item \textbf{Optimizing Event Sorting}: Efficiently sort events based on x-coordinates and heights to ensure correct processing order.
    \index{Optimizing Event Sorting}
    
    \item \textbf{Handling Large Inputs}: Design the algorithm to handle up to \(10^4\) buildings efficiently without significant performance degradation.
    \index{Handling Large Inputs}
    
    \item \textbf{Using Efficient Data Structures}: Utilize heaps, lists, and dictionaries effectively to manage and process events and active heights.
    \index{Efficient Data Structures}
    
    \item \textbf{Avoiding Redundant Calculations}: Ensure that distance and overlap calculations are performed only when necessary to optimize performance.
    \index{Avoiding Redundant Calculations}
    
    \item \textbf{Code Readability and Documentation}: Maintain clear and readable code through meaningful variable names and comprehensive comments to facilitate understanding and maintenance.
    \index{Code Readability}
    
    \item \textbf{Edge Case Handling}: Implement checks for edge cases to prevent incorrect results or runtime errors.
    \index{Edge Case Handling}
    
    \item \textbf{Implementing Helper Functions}: Create helper functions for tasks like distance calculation, event handling, and heap management to enhance modularity.
    \index{Helper Functions}
    
    \item \textbf{Consistent Naming Conventions}: Use consistent and descriptive naming conventions for variables and functions to improve code clarity.
    \index{Naming Conventions}
    
    \item \textbf{Memory Management}: Ensure that the algorithm manages memory efficiently, especially when dealing with large datasets.
    \index{Memory Management}
    
    \item \textbf{Implementing Iterative Solutions Carefully}: Ensure that loop conditions are correctly defined to prevent infinite loops or incorrect terminations.
    \index{Iterative Solutions}
    
    \item \textbf{Avoiding Floating-Point Precision Issues}: Since the problem deals with integers, floating-point precision is not a concern, simplifying the implementation.
    \index{Floating-Point Precision}
    
    \item \textbf{Testing and Validation}: Develop a comprehensive suite of test cases that cover all possible scenarios, including edge cases, to validate the correctness and efficiency of the implementation.
    \index{Testing and Validation}
    
    \item \textbf{Performance Considerations}: Optimize the loop conditions and operations to ensure that the function runs efficiently, especially for large input numbers.
    \index{Performance Considerations}
\end{itemize}

\section*{Conclusion}

The \textbf{Skyline Problem} is a quintessential example of applying advanced algorithmic techniques and geometric reasoning to solve complex spatial challenges. By leveraging the Sweep Line Algorithm and maintaining active building heights using a max heap, the solution efficiently constructs the skyline with optimal time and space complexities. Understanding and implementing such sophisticated algorithms not only enhances problem-solving skills but also provides a foundation for tackling a wide array of Computational Geometry problems in various domains, including computer graphics, urban planning simulations, and geographic information systems.

\printindex

% \input{sections/rectangle_overlap}
% \input{sections/rectangle_area}
% \input{sections/k_closest_points_to_origin}
% \input{sections/the_skyline_problem}
% % filename: k_closest_points_to_origin.tex

\problemsection{K Closest Points to Origin}
\label{chap:K_Closest_Points_to_Origin}
\marginnote{\href{https://leetcode.com/problems/k-closest-points-to-origin/}{[LeetCode Link]}\index{LeetCode}}
\marginnote{\href{https://www.geeksforgeeks.org/find-k-closest-points-origin/}{[GeeksForGeeks Link]}\index{GeeksForGeeks}}
\marginnote{\href{https://www.interviewbit.com/problems/k-closest-points/}{[InterviewBit Link]}\index{InterviewBit}}
\marginnote{\href{https://app.codesignal.com/challenges/k-closest-points-to-origin}{[CodeSignal Link]}\index{CodeSignal}}
\marginnote{\href{https://www.codewars.com/kata/k-closest-points-to-origin/train/python}{[Codewars Link]}\index{Codewars}}

The \textbf{K Closest Points to Origin} problem is a popular algorithmic challenge in Computational Geometry that involves identifying the \(k\) points closest to the origin in a 2D plane. This problem tests one's ability to apply efficient sorting and selection algorithms, understand distance computations, and optimize for performance. Mastery of this problem is essential for applications in spatial data analysis, nearest neighbor searches, and clustering algorithms.

\section*{Problem Statement}

Given an array of points where each point is represented as \([x, y]\) in the 2D plane, and an integer \(k\), return the \(k\) closest points to the origin \((0, 0)\).

The distance between two points \((x_1, y_1)\) and \((x_2, y_2)\) is the Euclidean distance \(\sqrt{(x_1 - x_2)^2 + (y_1 - y_2)^2}\). The origin is \((0, 0)\).

\textbf{Function signature in Python:}
\begin{lstlisting}[language=Python]
def kClosest(points: List[List[int]], K: int) -> List[List[int]]:
\end{lstlisting}

\section*{Examples}

\textbf{Example 1:}

\begin{verbatim}
Input: points = [[1,3],[-2,2]], K = 1
Output: [[-2,2]]
Explanation: 
The distance between (1, 3) and the origin is sqrt(10).
The distance between (-2, 2) and the origin is sqrt(8).
Since sqrt(8) < sqrt(10), (-2, 2) is closer to the origin.
\end{verbatim}

\textbf{Example 2:}

\begin{verbatim}
Input: points = [[3,3],[5,-1],[-2,4]], K = 2
Output: [[3,3],[-2,4]]
Explanation: 
The distances are sqrt(18), sqrt(26), and sqrt(20) respectively.
The two closest points are [3,3] and [-2,4].
\end{verbatim}

\textbf{Example 3:}

\begin{verbatim}
Input: points = [[0,1],[1,0]], K = 2
Output: [[0,1],[1,0]]
Explanation: 
Both points are equally close to the origin.
\end{verbatim}

\textbf{Example 4:}

\begin{verbatim}
Input: points = [[1,0],[0,1]], K = 1
Output: [[1,0]]
Explanation: 
Both points are equally close; returning any one is acceptable.
\end{verbatim}

\textbf{Constraints:}

\begin{itemize}
    \item \(1 \leq K \leq \text{points.length} \leq 10^4\)
    \item \(-10^4 < x_i, y_i < 10^4\)
\end{itemize}

LeetCode link: \href{https://leetcode.com/problems/k-closest-points-to-origin/}{K Closest Points to Origin}\index{LeetCode}

\section*{Algorithmic Approach}

To identify the \(k\) closest points to the origin, several algorithmic strategies can be employed. The most efficient methods aim to reduce the time complexity by avoiding the need to sort the entire list of points.

\subsection*{1. Sorting Based on Distance}

Calculate the Euclidean distance of each point from the origin and sort the points based on these distances. Select the first \(k\) points from the sorted list.

\begin{enumerate}
    \item Compute the distance for each point using the formula \(distance = x^2 + y^2\).
    \item Sort the points based on the computed distances.
    \item Return the first \(k\) points from the sorted list.
\end{enumerate}

\subsection*{2. Max Heap (Priority Queue)}

Use a max heap to maintain the \(k\) closest points. Iterate through each point, add it to the heap, and if the heap size exceeds \(k\), remove the farthest point.

\begin{enumerate}
    \item Initialize a max heap.
    \item For each point, compute its distance and add it to the heap.
    \item If the heap size exceeds \(k\), remove the point with the largest distance.
    \item After processing all points, the heap contains the \(k\) closest points.
\end{enumerate}

\subsection*{3. QuickSelect (Quick Sort Partitioning)}

Utilize the QuickSelect algorithm to find the \(k\) closest points without fully sorting the list.

\begin{enumerate}
    \item Choose a pivot point and partition the list based on distances relative to the pivot.
    \item Recursively apply QuickSelect to the partition containing the \(k\) closest points.
    \item Once the \(k\) closest points are identified, return them.
\end{enumerate}

\marginnote{QuickSelect offers an average time complexity of \(O(n)\), making it highly efficient for large datasets.}

\section*{Complexities}

\begin{itemize}
    \item \textbf{Sorting Based on Distance:}
    \begin{itemize}
        \item \textbf{Time Complexity:} \(O(n \log n)\)
        \item \textbf{Space Complexity:} \(O(n)\)
    \end{itemize}
    
    \item \textbf{Max Heap (Priority Queue):}
    \begin{itemize}
        \item \textbf{Time Complexity:} \(O(n \log k)\)
        \item \textbf{Space Complexity:} \(O(k)\)
    \end{itemize}
    
    \item \textbf{QuickSelect (Quick Sort Partitioning):}
    \begin{itemize}
        \item \textbf{Time Complexity:} Average case \(O(n)\), worst case \(O(n^2)\)
        \item \textbf{Space Complexity:} \(O(1)\) (in-place)
    \end{itemize}
\end{itemize}

\section*{Python Implementation}

\marginnote{Implementing QuickSelect provides an optimal average-case solution with linear time complexity.}

Below is the complete Python code implementing the \texttt{kClosest} function using the QuickSelect approach:

\begin{fullwidth}
\begin{lstlisting}[language=Python]
from typing import List
import random

class Solution:
    def kClosest(self, points: List[List[int]], K: int) -> List[List[int]]:
        def quickselect(left, right, K_smallest):
            if left == right:
                return
            
            # Select a random pivot_index
            pivot_index = random.randint(left, right)
            
            # Partition the array
            pivot_index = partition(left, right, pivot_index)
            
            # The pivot is in its final sorted position
            if K_smallest == pivot_index:
                return
            elif K_smallest < pivot_index:
                quickselect(left, pivot_index - 1, K_smallest)
            else:
                quickselect(pivot_index + 1, right, K_smallest)
        
        def partition(left, right, pivot_index):
            pivot_distance = distance(points[pivot_index])
            # Move pivot to end
            points[pivot_index], points[right] = points[right], points[pivot_index]
            store_index = left
            for i in range(left, right):
                if distance(points[i]) < pivot_distance:
                    points[store_index], points[i] = points[i], points[store_index]
                    store_index += 1
            # Move pivot to its final place
            points[right], points[store_index] = points[store_index], points[right]
            return store_index
        
        def distance(point):
            return point[0] ** 2 + point[1] ** 2
        
        n = len(points)
        quickselect(0, n - 1, K)
        return points[:K]

# Example usage:
solution = Solution()
print(solution.kClosest([[1,3],[-2,2]], 1))            # Output: [[-2,2]]
print(solution.kClosest([[3,3],[5,-1],[-2,4]], 2))     # Output: [[3,3],[-2,4]]
print(solution.kClosest([[0,1],[1,0]], 2))             # Output: [[0,1],[1,0]]
print(solution.kClosest([[1,0],[0,1]], 1))             # Output: [[1,0]] or [[0,1]]
\end{lstlisting}
\end{fullwidth}

This implementation uses the QuickSelect algorithm to efficiently find the \(k\) closest points to the origin without fully sorting the entire list. It ensures optimal performance even with large datasets.

\section*{Explanation}

The \texttt{kClosest} function identifies the \(k\) closest points to the origin using the QuickSelect algorithm. Here's a detailed breakdown of the implementation:

\subsection*{1. Distance Calculation}

\begin{itemize}
    \item The Euclidean distance is calculated as \(distance = x^2 + y^2\). Since we only need relative distances for comparison, the square root is omitted for efficiency.
\end{itemize}

\subsection*{2. QuickSelect Algorithm}

\begin{itemize}
    \item **Pivot Selection:**
    \begin{itemize}
        \item A random pivot is chosen to enhance the average-case performance.
    \end{itemize}
    
    \item **Partitioning:**
    \begin{itemize}
        \item The array is partitioned such that points with distances less than the pivot are moved to the left, and others to the right.
        \item The pivot is placed in its correct sorted position.
    \end{itemize}
    
    \item **Recursive Selection:**
    \begin{itemize}
        \item If the pivot's position matches \(K\), the selection is complete.
        \item Otherwise, recursively apply QuickSelect to the relevant partition.
    \end{itemize}
\end{itemize}

\subsection*{3. Final Selection}

\begin{itemize}
    \item After partitioning, the first \(K\) points in the list are the \(k\) closest points to the origin.
\end{itemize}

\subsection*{4. Example Walkthrough}

Consider the first example:
\begin{verbatim}
Input: points = [[1,3],[-2,2]], K = 1
Output: [[-2,2]]
\end{verbatim}

\begin{enumerate}
    \item **Calculate Distances:**
    \begin{itemize}
        \item [1,3] : \(1^2 + 3^2 = 10\)
        \item [-2,2] : \((-2)^2 + 2^2 = 8\)
    \end{itemize}
    
    \item **QuickSelect Process:**
    \begin{itemize}
        \item Choose a pivot, say [1,3] with distance 10.
        \item Compare and rearrange:
        \begin{itemize}
            \item [-2,2] has a smaller distance (8) and is moved to the left.
        \end{itemize}
        \item After partitioning, the list becomes [[-2,2], [1,3]].
        \item Since \(K = 1\), return the first point: [[-2,2]].
    \end{itemize}
\end{enumerate}

Thus, the function correctly identifies \([-2,2]\) as the closest point to the origin.

\section*{Why This Approach}

The QuickSelect algorithm is chosen for its average-case linear time complexity \(O(n)\), making it highly efficient for large datasets compared to sorting-based methods with \(O(n \log n)\) time complexity. By avoiding the need to sort the entire list, QuickSelect provides an optimal solution for finding the \(k\) closest points.

\section*{Alternative Approaches}

\subsection*{1. Sorting Based on Distance}

Sort all points based on their distances from the origin and select the first \(k\) points.

\begin{lstlisting}[language=Python]
class Solution:
    def kClosest(self, points: List[List[int]], K: int) -> List[List[int]]:
        points.sort(key=lambda P: P[0]**2 + P[1]**2)
        return points[:K]
\end{lstlisting}

\textbf{Complexities:}
\begin{itemize}
    \item \textbf{Time Complexity:} \(O(n \log n)\)
    \item \textbf{Space Complexity:} \(O(1)\)
\end{itemize}

\subsection*{2. Max Heap (Priority Queue)}

Use a max heap to maintain the \(k\) closest points.

\begin{lstlisting}[language=Python]
import heapq

class Solution:
    def kClosest(self, points: List[List[int]], K: int) -> List[List[int]]:
        heap = []
        for (x, y) in points:
            dist = -(x**2 + y**2)  # Max heap using negative distances
            heapq.heappush(heap, (dist, [x, y]))
            if len(heap) > K:
                heapq.heappop(heap)
        return [item[1] for item in heap]
\end{lstlisting}

\textbf{Complexities:}
\begin{itemize}
    \item \textbf{Time Complexity:} \(O(n \log k)\)
    \item \textbf{Space Complexity:} \(O(k)\)
\end{itemize}

\subsection*{3. Using Built-In Functions}

Leverage built-in functions for distance calculation and selection.

\begin{lstlisting}[language=Python]
import math

class Solution:
    def kClosest(self, points: List[List[int]], K: int) -> List[List[int]]:
        points.sort(key=lambda P: math.sqrt(P[0]**2 + P[1]**2))
        return points[:K]
\end{lstlisting}

\textbf{Note}: This method is similar to the sorting approach but uses the actual Euclidean distance.

\section*{Similar Problems to This One}

Several problems involve nearest neighbor searches, spatial data analysis, and efficient selection algorithms, utilizing similar algorithmic strategies:

\begin{itemize}
    \item \textbf{Closest Pair of Points}: Find the closest pair of points in a set.
    \item \textbf{Top K Frequent Elements}: Identify the most frequent elements in a dataset.
    \item \textbf{Kth Largest Element in an Array}: Find the \(k\)-th largest element in an unsorted array.
    \item \textbf{Sliding Window Maximum}: Find the maximum in each sliding window of size \(k\) over an array.
    \item \textbf{Merge K Sorted Lists}: Merge multiple sorted lists into a single sorted list.
    \item \textbf{Find Median from Data Stream}: Continuously find the median of a stream of numbers.
    \item \textbf{Top K Closest Stars}: Find the \(k\) closest stars to Earth based on their distances.
\end{itemize}

These problems reinforce concepts of efficient selection, heap usage, and distance computations in various contexts.

\section*{Things to Keep in Mind and Tricks}

When solving the \textbf{K Closest Points to Origin} problem, consider the following tips and best practices to enhance efficiency and correctness:

\begin{itemize}
    \item \textbf{Understand Distance Calculations}: Grasp the Euclidean distance formula and recognize that the square root can be omitted for comparison purposes.
    \index{Distance Calculations}
    
    \item \textbf{Leverage Efficient Algorithms}: Use QuickSelect or heap-based methods to optimize time complexity, especially for large datasets.
    \index{Efficient Algorithms}
    
    \item \textbf{Handle Ties Appropriately}: Decide how to handle points with identical distances when \(k\) is less than the number of such points.
    \index{Handling Ties}
    
    \item \textbf{Optimize Space Usage}: Choose algorithms that minimize additional space, such as in-place QuickSelect.
    \index{Space Optimization}
    
    \item \textbf{Use Appropriate Data Structures}: Utilize heaps, lists, and helper functions effectively to manage and process data.
    \index{Data Structures}
    
    \item \textbf{Implement Helper Functions}: Create helper functions for distance calculation and partitioning to enhance code modularity.
    \index{Helper Functions}
    
    \item \textbf{Code Readability}: Maintain clear and readable code through meaningful variable names and structured logic.
    \index{Code Readability}
    
    \item \textbf{Test Extensively}: Implement a wide range of test cases, including edge cases like multiple points with the same distance, to ensure robustness.
    \index{Extensive Testing}
    
    \item \textbf{Understand Algorithm Trade-offs}: Recognize the trade-offs between different approaches in terms of time and space complexities.
    \index{Algorithm Trade-offs}
    
    \item \textbf{Use Built-In Sorting Functions}: When using sorting-based approaches, leverage built-in functions for efficiency and simplicity.
    \index{Built-In Sorting}
    
    \item \textbf{Avoid Redundant Calculations}: Ensure that distance calculations are performed only when necessary to optimize performance.
    \index{Avoiding Redundant Calculations}
    
    \item \textbf{Language-Specific Features}: Utilize language-specific features or libraries that can simplify implementation, such as heapq in Python.
    \index{Language-Specific Features}
\end{itemize}

\section*{Corner and Special Cases to Test When Writing the Code}

When implementing the solution for the \textbf{K Closest Points to Origin} problem, it is crucial to consider and rigorously test various edge cases to ensure robustness and correctness:

\begin{itemize}
    \item \textbf{Multiple Points with Same Distance}: Ensure that the algorithm handles multiple points having the same distance from the origin.
    \index{Same Distance Points}
    
    \item \textbf{Points at Origin}: Include points that are exactly at the origin \((0,0)\).
    \index{Points at Origin}
    
    \item \textbf{Negative Coordinates}: Ensure that the algorithm correctly computes distances for points with negative \(x\) or \(y\) coordinates.
    \index{Negative Coordinates}
    
    \item \textbf{Large Coordinates}: Test with points having very large or very small coordinate values to verify integer handling.
    \index{Large Coordinates}
    
    \item \textbf{K Equals Number of Points}: When \(K\) is equal to the number of points, the algorithm should return all points.
    \index{K Equals Number of Points}
    
    \item \textbf{K is One}: Test with \(K = 1\) to ensure the closest point is correctly identified.
    \index{K is One}
    
    \item \textbf{All Points Same}: All points have the same coordinates.
    \index{All Points Same}
    
    \item \textbf{K is Zero}: Although \(K\) is defined to be at least 1, ensure that the algorithm gracefully handles \(K = 0\) if allowed.
    \index{K is Zero}
    
    \item \textbf{Single Point}: Only one point is provided, and \(K = 1\).
    \index{Single Point}
    
    \item \textbf{Mixed Coordinates}: Points with a mix of positive and negative coordinates.
    \index{Mixed Coordinates}
    
    \item \textbf{Points with Zero Distance}: Multiple points at the origin.
    \index{Zero Distance Points}
    
    \item \textbf{Sparse and Dense Points}: Densely packed points and sparsely distributed points.
    \index{Sparse and Dense Points}
    
    \item \textbf{Duplicate Points}: Multiple identical points in the input list.
    \index{Duplicate Points}
    
    \item \textbf{K Greater Than Number of Unique Points}: Ensure that the algorithm handles cases where \(K\) exceeds the number of unique points if applicable.
    \index{K Greater Than Unique Points}
\end{itemize}

\section*{Implementation Considerations}

When implementing the \texttt{kClosest} function, keep in mind the following considerations to ensure robustness and efficiency:

\begin{itemize}
    \item \textbf{Data Type Selection}: Use appropriate data types that can handle large input values without overflow or precision loss.
    \index{Data Type Selection}
    
    \item \textbf{Optimizing Distance Calculations}: Avoid calculating the square root since it is unnecessary for comparison purposes.
    \index{Optimizing Distance Calculations}
    
    \item \textbf{Choosing the Right Algorithm}: Select an algorithm based on the size of the input and the value of \(K\) to optimize time and space complexities.
    \index{Choosing the Right Algorithm}
    
    \item \textbf{Language-Specific Libraries}: Utilize language-specific libraries and functions (e.g., \texttt{heapq} in Python) to simplify implementation and enhance performance.
    \index{Language-Specific Libraries}
    
    \item \textbf{Avoiding Redundant Calculations}: Ensure that each point's distance is calculated only once to optimize performance.
    \index{Avoiding Redundant Calculations}
    
    \item \textbf{Implementing Helper Functions}: Create helper functions for tasks like distance calculation and partitioning to enhance modularity and readability.
    \index{Helper Functions}
    
    \item \textbf{Edge Case Handling}: Implement checks for edge cases to prevent incorrect results or runtime errors.
    \index{Edge Case Handling}
    
    \item \textbf{Testing and Validation}: Develop a comprehensive suite of test cases that cover all possible scenarios, including edge cases, to validate the correctness and efficiency of the implementation.
    \index{Testing and Validation}
    
    \item \textbf{Scalability}: Design the algorithm to scale efficiently with increasing input sizes, maintaining performance and resource utilization.
    \index{Scalability}
    
    \item \textbf{Consistent Naming Conventions}: Use consistent and descriptive naming conventions for variables and functions to improve code clarity.
    \index{Naming Conventions}
    
    \item \textbf{Memory Management}: Ensure that the algorithm manages memory efficiently, especially when dealing with large datasets.
    \index{Memory Management}
    
    \item \textbf{Avoiding Stack Overflow}: If implementing recursive approaches, be mindful of recursion limits and potential stack overflow issues.
    \index{Avoiding Stack Overflow}
    
    \item \textbf{Implementing Iterative Solutions}: Prefer iterative solutions when recursion may lead to increased space complexity or stack overflow.
    \index{Implementing Iterative Solutions}
\end{itemize}

\section*{Conclusion}

The \textbf{K Closest Points to Origin} problem exemplifies the application of efficient selection algorithms and geometric computations to solve spatial challenges effectively. By leveraging QuickSelect or heap-based methods, the algorithm achieves optimal time and space complexities, making it highly suitable for large datasets. Understanding and implementing such techniques not only enhances problem-solving skills but also provides a foundation for tackling more advanced Computational Geometry problems involving nearest neighbor searches, clustering, and spatial data analysis.

\printindex

% % filename: rectangle_overlap.tex

\problemsection{Rectangle Overlap}
\label{chap:Rectangle_Overlap}
\marginnote{\href{https://leetcode.com/problems/rectangle-overlap/}{[LeetCode Link]}\index{LeetCode}}
\marginnote{\href{https://www.geeksforgeeks.org/check-if-two-rectangles-overlap/}{[GeeksForGeeks Link]}\index{GeeksForGeeks}}
\marginnote{\href{https://www.interviewbit.com/problems/rectangle-overlap/}{[InterviewBit Link]}\index{InterviewBit}}
\marginnote{\href{https://app.codesignal.com/challenges/rectangle-overlap}{[CodeSignal Link]}\index{CodeSignal}}
\marginnote{\href{https://www.codewars.com/kata/rectangle-overlap/train/python}{[Codewars Link]}\index{Codewars}}

The \textbf{Rectangle Overlap} problem is a fundamental challenge in Computational Geometry that involves determining whether two axis-aligned rectangles overlap. This problem tests one's ability to understand geometric properties, implement conditional logic, and optimize for efficient computation. Mastery of this problem is essential for applications in computer graphics, collision detection, and spatial data analysis.

\section*{Problem Statement}

Given two axis-aligned rectangles in a 2D plane, determine if they overlap. Each rectangle is defined by its bottom-left and top-right coordinates.

A rectangle is represented as a list of four integers \([x1, y1, x2, y2]\), where \((x1, y1)\) are the coordinates of the bottom-left corner, and \((x2, y2)\) are the coordinates of the top-right corner.

\textbf{Function signature in Python:}
\begin{lstlisting}[language=Python]
def isRectangleOverlap(rec1: List[int], rec2: List[int]) -> bool:
\end{lstlisting}

\section*{Examples}

\textbf{Example 1:}

\begin{verbatim}
Input: rec1 = [0,0,2,2], rec2 = [1,1,3,3]
Output: True
Explanation: The rectangles overlap in the area defined by [1,1,2,2].
\end{verbatim}

\textbf{Example 2:}

\begin{verbatim}
Input: rec1 = [0,0,1,1], rec2 = [1,0,2,1]
Output: False
Explanation: The rectangles touch at the edge but do not overlap.
\end{verbatim}

\textbf{Example 3:}

\begin{verbatim}
Input: rec1 = [0,0,1,1], rec2 = [2,2,3,3]
Output: False
Explanation: The rectangles are completely separate.
\end{verbatim}

\textbf{Example 4:}

\begin{verbatim}
Input: rec1 = [0,0,5,5], rec2 = [3,3,7,7]
Output: True
Explanation: The rectangles overlap in the area defined by [3,3,5,5].
\end{verbatim}

\textbf{Example 5:}

\begin{verbatim}
Input: rec1 = [0,0,0,0], rec2 = [0,0,0,0]
Output: False
Explanation: Both rectangles are degenerate points.
\end{verbatim}

\textbf{Constraints:}

\begin{itemize}
    \item All coordinates are integers in the range \([-10^9, 10^9]\).
    \item For each rectangle, \(x1 < x2\) and \(y1 < y2\).
\end{itemize}

LeetCode link: \href{https://leetcode.com/problems/rectangle-overlap/}{Rectangle Overlap}\index{LeetCode}

\section*{Algorithmic Approach}

To determine whether two axis-aligned rectangles overlap, we can use the following logical conditions:

1. **Non-Overlap Conditions:**
   - One rectangle is to the left of the other.
   - One rectangle is above the other.

2. **Overlap Condition:**
   - If neither of the non-overlap conditions is true, the rectangles must overlap.

\subsection*{Steps:}

1. **Extract Coordinates:**
   - For both rectangles, extract the bottom-left and top-right coordinates.

2. **Check Non-Overlap Conditions:**
   - If the right side of the first rectangle is less than or equal to the left side of the second rectangle, they do not overlap.
   - If the left side of the first rectangle is greater than or equal to the right side of the second rectangle, they do not overlap.
   - If the top side of the first rectangle is less than or equal to the bottom side of the second rectangle, they do not overlap.
   - If the bottom side of the first rectangle is greater than or equal to the top side of the second rectangle, they do not overlap.

3. **Determine Overlap:**
   - If none of the non-overlap conditions are met, the rectangles overlap.

\marginnote{This approach provides an efficient \(O(1)\) time complexity solution by leveraging simple geometric comparisons.}

\section*{Complexities}

\begin{itemize}
    \item \textbf{Time Complexity:} \(O(1)\). The algorithm performs a constant number of comparisons regardless of input size.
    
    \item \textbf{Space Complexity:} \(O(1)\). Only a fixed amount of extra space is used for variables.
\end{itemize}

\section*{Python Implementation}

\marginnote{Implementing the overlap check using coordinate comparisons ensures an optimal and straightforward solution.}

Below is the complete Python code implementing the \texttt{isRectangleOverlap} function:

\begin{fullwidth}
\begin{lstlisting}[language=Python]
from typing import List

class Solution:
    def isRectangleOverlap(self, rec1: List[int], rec2: List[int]) -> bool:
        # Extract coordinates
        left1, bottom1, right1, top1 = rec1
        left2, bottom2, right2, top2 = rec2
        
        # Check non-overlapping conditions
        if right1 <= left2 or right2 <= left1:
            return False
        if top1 <= bottom2 or top2 <= bottom1:
            return False
        
        # If none of the above, rectangles overlap
        return True

# Example usage:
solution = Solution()
print(solution.isRectangleOverlap([0,0,2,2], [1,1,3,3]))  # Output: True
print(solution.isRectangleOverlap([0,0,1,1], [1,0,2,1]))  # Output: False
print(solution.isRectangleOverlap([0,0,1,1], [2,2,3,3]))  # Output: False
print(solution.isRectangleOverlap([0,0,5,5], [3,3,7,7]))  # Output: True
print(solution.isRectangleOverlap([0,0,0,0], [0,0,0,0]))  # Output: False
\end{lstlisting}
\end{fullwidth}

This implementation efficiently checks for overlap by comparing the coordinates of the two rectangles. If any of the non-overlapping conditions are met, it returns \texttt{False}; otherwise, it returns \texttt{True}.

\section*{Explanation}

The \texttt{isRectangleOverlap} function determines whether two axis-aligned rectangles overlap by comparing their respective coordinates. Here's a detailed breakdown of the implementation:

\subsection*{1. Extract Coordinates}

\begin{itemize}
    \item For each rectangle, extract the left (\(x1\)), bottom (\(y1\)), right (\(x2\)), and top (\(y2\)) coordinates.
    \item This simplifies the comparison process by providing clear variables representing each side of the rectangles.
\end{itemize}

\subsection*{2. Check Non-Overlap Conditions}

\begin{itemize}
    \item **Horizontal Separation:**
    \begin{itemize}
        \item If the right side of the first rectangle (\(right1\)) is less than or equal to the left side of the second rectangle (\(left2\)), there is no horizontal overlap.
        \item Similarly, if the right side of the second rectangle (\(right2\)) is less than or equal to the left side of the first rectangle (\(left1\)), there is no horizontal overlap.
    \end{itemize}
    
    \item **Vertical Separation:**
    \begin{itemize}
        \item If the top side of the first rectangle (\(top1\)) is less than or equal to the bottom side of the second rectangle (\(bottom2\)), there is no vertical overlap.
        \item Similarly, if the top side of the second rectangle (\(top2\)) is less than or equal to the bottom side of the first rectangle (\(bottom1\)), there is no vertical overlap.
    \end{itemize}
    
    \item If any of these non-overlapping conditions are true, the rectangles do not overlap, and the function returns \texttt{False}.
\end{itemize}

\subsection*{3. Determine Overlap}

\begin{itemize}
    \item If none of the non-overlapping conditions are met, it implies that the rectangles overlap both horizontally and vertically.
    \item The function returns \texttt{True} in this case.
\end{itemize}

\subsection*{4. Example Walkthrough}

Consider the first example:
\begin{verbatim}
Input: rec1 = [0,0,2,2], rec2 = [1,1,3,3]
Output: True
\end{verbatim}

\begin{enumerate}
    \item Extract coordinates:
    \begin{itemize}
        \item rec1: left1 = 0, bottom1 = 0, right1 = 2, top1 = 2
        \item rec2: left2 = 1, bottom2 = 1, right2 = 3, top2 = 3
    \end{itemize}
    
    \item Check non-overlap conditions:
    \begin{itemize}
        \item \(right1 = 2\) is not less than or equal to \(left2 = 1\)
        \item \(right2 = 3\) is not less than or equal to \(left1 = 0\)
        \item \(top1 = 2\) is not less than or equal to \(bottom2 = 1\)
        \item \(top2 = 3\) is not less than or equal to \(bottom1 = 0\)
    \end{itemize}
    
    \item Since none of the non-overlapping conditions are met, the rectangles overlap.
\end{enumerate}

Thus, the function correctly returns \texttt{True}.

\section*{Why This Approach}

This approach is chosen for its simplicity and efficiency. By leveraging direct coordinate comparisons, the algorithm achieves constant time complexity without the need for complex data structures or iterative processes. It effectively handles all possible scenarios of rectangle positioning, ensuring accurate detection of overlaps.

\section*{Alternative Approaches}

\subsection*{1. Separating Axis Theorem (SAT)}

The Separating Axis Theorem is a more generalized method for detecting overlaps between convex shapes. While it is not necessary for axis-aligned rectangles, understanding SAT can be beneficial for more complex geometric problems.

\begin{lstlisting}[language=Python]
def isRectangleOverlap(rec1: List[int], rec2: List[int]) -> bool:
    # Using SAT for axis-aligned rectangles
    return not (rec1[2] <= rec2[0] or rec1[0] >= rec2[2] or
                rec1[3] <= rec2[1] or rec1[1] >= rec2[3])
\end{lstlisting}

\textbf{Note}: This implementation is functionally identical to the primary approach but leverages a more generalized geometric theorem.

\subsection*{2. Area-Based Approach}

Calculate the overlapping area between the two rectangles. If the overlapping area is positive, the rectangles overlap.

\begin{lstlisting}[language=Python]
def isRectangleOverlap(rec1: List[int], rec2: List[int]) -> bool:
    # Calculate overlap in x and y dimensions
    x_overlap = min(rec1[2], rec2[2]) - max(rec1[0], rec2[0])
    y_overlap = min(rec1[3], rec2[3]) - max(rec1[1], rec2[1])
    
    # Overlap exists if both overlaps are positive
    return x_overlap > 0 and y_overlap > 0
\end{lstlisting}

\textbf{Complexities:}
\begin{itemize}
    \item \textbf{Time Complexity:} \(O(1)\)
    \item \textbf{Space Complexity:} \(O(1)\)
\end{itemize}

\subsection*{3. Using Rectangles Intersection Function}

Utilize built-in or library functions that handle geometric intersections.

\begin{lstlisting}[language=Python]
from shapely.geometry import box

def isRectangleOverlap(rec1: List[int], rec2: List[int]) -> bool:
    rectangle1 = box(rec1[0], rec1[1], rec1[2], rec1[3])
    rectangle2 = box(rec2[0], rec2[1], rec2[2], rec2[3])
    return rectangle1.intersects(rectangle2) and not rectangle1.touches(rectangle2)
\end{lstlisting}

\textbf{Note}: This approach requires the \texttt{shapely} library and is more suitable for complex geometric operations.

\section*{Similar Problems to This One}

Several problems revolve around geometric overlap, intersection detection, and spatial reasoning, utilizing similar algorithmic strategies:

\begin{itemize}
    \item \textbf{Interval Overlap}: Determine if two intervals on a line overlap.
    \item \textbf{Circle Overlap}: Determine if two circles overlap based on their radii and centers.
    \item \textbf{Polygon Overlap}: Determine if two polygons overlap using algorithms like SAT.
    \item \textbf{Closest Pair of Points}: Find the closest pair of points in a set.
    \item \textbf{Convex Hull}: Compute the convex hull of a set of points.
    \item \textbf{Intersection of Lines}: Find the intersection point of two lines.
    \item \textbf{Point Inside Polygon}: Determine if a point lies inside a given polygon.
\end{itemize}

These problems reinforce the concepts of spatial reasoning, geometric property analysis, and efficient algorithm design in various contexts.

\section*{Things to Keep in Mind and Tricks}

When working with the \textbf{Rectangle Overlap} problem, consider the following tips and best practices to enhance efficiency and correctness:

\begin{itemize}
    \item \textbf{Understand Geometric Relationships}: Grasp the positional relationships between rectangles to simplify overlap detection.
    \index{Geometric Relationships}
    
    \item \textbf{Leverage Coordinate Comparisons}: Use direct comparisons of rectangle coordinates to determine spatial relationships.
    \index{Coordinate Comparisons}
    
    \item \textbf{Handle Edge Cases}: Consider cases where rectangles touch at edges or corners without overlapping.
    \index{Edge Cases}
    
    \item \textbf{Optimize for Efficiency}: Aim for a constant time \(O(1)\) solution by avoiding unnecessary computations or iterations.
    \index{Efficiency Optimization}
    
    \item \textbf{Avoid Floating-Point Precision Issues}: Since all coordinates are integers, floating-point precision is not a concern, simplifying the implementation.
    \index{Floating-Point Precision}
    
    \item \textbf{Use Helper Functions}: Create helper functions to encapsulate repetitive tasks, such as extracting coordinates or checking specific conditions.
    \index{Helper Functions}
    
    \item \textbf{Code Readability}: Maintain clear and readable code through meaningful variable names and structured logic.
    \index{Code Readability}
    
    \item \textbf{Test Extensively}: Implement a wide range of test cases, including overlapping, non-overlapping, and edge-touching rectangles, to ensure robustness.
    \index{Extensive Testing}
    
    \item \textbf{Understand Axis-Aligned Constraints}: Recognize that axis-aligned rectangles simplify overlap detection compared to rotated rectangles.
    \index{Axis-Aligned Constraints}
    
    \item \textbf{Simplify Logical Conditions}: Combine multiple conditions logically to streamline the overlap detection process.
    \index{Logical Conditions}
\end{itemize}

\section*{Corner and Special Cases to Test When Writing the Code}

When implementing the solution for the \textbf{Rectangle Overlap} problem, it is crucial to consider and rigorously test various edge cases to ensure robustness and correctness:

\begin{itemize}
    \item \textbf{No Overlap}: Rectangles are completely separate.
    \index{No Overlap}
    
    \item \textbf{Partial Overlap}: Rectangles overlap in one or more regions.
    \index{Partial Overlap}
    
    \item \textbf{Edge Touching}: Rectangles touch exactly at one edge without overlapping.
    \index{Edge Touching}
    
    \item \textbf{Corner Touching}: Rectangles touch exactly at one corner without overlapping.
    \index{Corner Touching}
    
    \item \textbf{One Rectangle Inside Another}: One rectangle is entirely within the other.
    \index{Rectangle Inside}
    
    \item \textbf{Identical Rectangles}: Both rectangles have the same coordinates.
    \index{Identical Rectangles}
    
    \item \textbf{Degenerate Rectangles}: Rectangles with zero area (e.g., \(x1 = x2\) or \(y1 = y2\)).
    \index{Degenerate Rectangles}
    
    \item \textbf{Large Coordinates}: Rectangles with very large coordinate values to test performance and integer handling.
    \index{Large Coordinates}
    
    \item \textbf{Negative Coordinates}: Rectangles positioned in negative coordinate space.
    \index{Negative Coordinates}
    
    \item \textbf{Mixed Overlapping Scenarios}: Combinations of the above cases to ensure comprehensive coverage.
    \index{Mixed Overlapping Scenarios}
    
    \item \textbf{Minimum and Maximum Bounds}: Rectangles at the minimum and maximum limits of the coordinate range.
    \index{Minimum and Maximum Bounds}
\end{itemize}

\section*{Implementation Considerations}

When implementing the \texttt{isRectangleOverlap} function, keep in mind the following considerations to ensure robustness and efficiency:

\begin{itemize}
    \item \textbf{Data Type Selection}: Use appropriate data types that can handle the range of input values without overflow or underflow.
    \index{Data Type Selection}
    
    \item \textbf{Optimizing Comparisons}: Structure logical conditions to short-circuit evaluations as soon as a non-overlapping condition is met.
    \index{Optimizing Comparisons}
    
    \item \textbf{Language-Specific Constraints}: Be aware of how the programming language handles integer division and comparisons.
    \index{Language-Specific Constraints}
    
    \item \textbf{Avoiding Redundant Calculations}: Ensure that each comparison contributes towards determining overlap without unnecessary repetitions.
    \index{Avoiding Redundant Calculations}
    
    \item \textbf{Code Readability and Documentation}: Maintain clear and readable code through meaningful variable names and comprehensive comments to facilitate understanding and maintenance.
    \index{Code Readability}
    
    \item \textbf{Edge Case Handling}: Implement checks for edge cases to prevent incorrect results or runtime errors.
    \index{Edge Case Handling}
    
    \item \textbf{Testing and Validation}: Develop a comprehensive suite of test cases that cover all possible scenarios, including edge cases, to validate the correctness and efficiency of the implementation.
    \index{Testing and Validation}
    
    \item \textbf{Scalability}: Design the algorithm to scale efficiently with increasing input sizes, maintaining performance and resource utilization.
    \index{Scalability}
    
    \item \textbf{Using Helper Functions}: Consider creating helper functions for repetitive tasks, such as extracting and comparing coordinates, to enhance modularity and reusability.
    \index{Helper Functions}
    
    \item \textbf{Consistent Naming Conventions}: Use consistent and descriptive naming conventions for variables to improve code clarity.
    \index{Naming Conventions}
    
    \item \textbf{Handling Floating-Point Coordinates}: Although the problem specifies integer coordinates, ensure that the implementation can handle floating-point numbers if needed in extended scenarios.
    \index{Floating-Point Coordinates}
    
    \item \textbf{Avoiding Floating-Point Precision Issues}: Since all coordinates are integers, floating-point precision is not a concern, simplifying the implementation.
    \index{Floating-Point Precision}
    
    \item \textbf{Implementing Unit Tests}: Develop unit tests for each logical condition to ensure that all scenarios are correctly handled.
    \index{Unit Tests}
    
    \item \textbf{Error Handling}: Incorporate error handling to manage invalid inputs gracefully.
    \index{Error Handling}
\end{itemize}

\section*{Conclusion}

The \textbf{Rectangle Overlap} problem exemplifies the application of fundamental geometric principles and conditional logic to solve spatial challenges efficiently. By leveraging simple coordinate comparisons, the algorithm achieves optimal time and space complexities, making it highly suitable for real-time applications such as collision detection in gaming, layout planning in graphics, and spatial data analysis. Understanding and implementing such techniques not only enhances problem-solving skills but also provides a foundation for tackling more complex Computational Geometry problems involving varied geometric shapes and interactions.

\printindex

% \input{sections/rectangle_overlap}
% \input{sections/rectangle_area}
% \input{sections/k_closest_points_to_origin}
% \input{sections/the_skyline_problem}
% % filename: rectangle_area.tex

\problemsection{Rectangle Area}
\label{chap:Rectangle_Area}
\marginnote{\href{https://leetcode.com/problems/rectangle-area/}{[LeetCode Link]}\index{LeetCode}}
\marginnote{\href{https://www.geeksforgeeks.org/find-area-two-overlapping-rectangles/}{[GeeksForGeeks Link]}\index{GeeksForGeeks}}
\marginnote{\href{https://www.interviewbit.com/problems/rectangle-area/}{[InterviewBit Link]}\index{InterviewBit}}
\marginnote{\href{https://app.codesignal.com/challenges/rectangle-area}{[CodeSignal Link]}\index{CodeSignal}}
\marginnote{\href{https://www.codewars.com/kata/rectangle-area/train/python}{[Codewars Link]}\index{Codewars}}

The \textbf{Rectangle Area} problem is a classic Computational Geometry challenge that involves calculating the total area covered by two axis-aligned rectangles in a 2D plane. This problem tests one's ability to perform geometric calculations, handle overlapping scenarios, and implement efficient algorithms. Mastery of this problem is essential for applications in computer graphics, spatial analysis, and computational modeling.

\section*{Problem Statement}

Given two axis-aligned rectangles in a 2D plane, compute the total area covered by the two rectangles. The area covered by the overlapping region should be counted only once.

Each rectangle is represented as a list of four integers \([x1, y1, x2, y2]\), where \((x1, y1)\) are the coordinates of the bottom-left corner, and \((x2, y2)\) are the coordinates of the top-right corner.

\textbf{Function signature in Python:}
\begin{lstlisting}[language=Python]
def computeArea(A: List[int], B: List[int]) -> int:
\end{lstlisting}

\section*{Examples}

\textbf{Example 1:}

\begin{verbatim}
Input: A = [-3,0,3,4], B = [0,-1,9,2]
Output: 45
Explanation:
Area of A = (3 - (-3)) * (4 - 0) = 6 * 4 = 24
Area of B = (9 - 0) * (2 - (-1)) = 9 * 3 = 27
Overlapping Area = (3 - 0) * (2 - 0) = 3 * 2 = 6
Total Area = 24 + 27 - 6 = 45
\end{verbatim}

\textbf{Example 2:}

\begin{verbatim}
Input: A = [0,0,0,0], B = [0,0,0,0]
Output: 0
Explanation:
Both rectangles are degenerate points with zero area.
\end{verbatim}

\textbf{Example 3:}

\begin{verbatim}
Input: A = [0,0,2,2], B = [1,1,3,3]
Output: 7
Explanation:
Area of A = 4
Area of B = 4
Overlapping Area = 1
Total Area = 4 + 4 - 1 = 7
\end{verbatim}

\textbf{Example 4:}

\begin{verbatim}
Input: A = [0,0,1,1], B = [1,0,2,1]
Output: 2
Explanation:
Rectangles touch at the edge but do not overlap.
Area of A = 1
Area of B = 1
Overlapping Area = 0
Total Area = 1 + 1 = 2
\end{verbatim}

\textbf{Constraints:}

\begin{itemize}
    \item All coordinates are integers in the range \([-10^9, 10^9]\).
    \item For each rectangle, \(x1 < x2\) and \(y1 < y2\).
\end{itemize}

LeetCode link: \href{https://leetcode.com/problems/rectangle-area/}{Rectangle Area}\index{LeetCode}

\section*{Algorithmic Approach}

To compute the total area covered by two axis-aligned rectangles, we can follow these steps:

1. **Calculate Individual Areas:**
   - Compute the area of the first rectangle.
   - Compute the area of the second rectangle.

2. **Determine Overlapping Area:**
   - Calculate the coordinates of the overlapping rectangle, if any.
   - If the rectangles overlap, compute the area of the overlapping region.

3. **Compute Total Area:**
   - Sum the individual areas and subtract the overlapping area to avoid double-counting.

\marginnote{This approach ensures accurate area calculation by handling overlapping regions appropriately.}

\section*{Complexities}

\begin{itemize}
    \item \textbf{Time Complexity:} \(O(1)\). The algorithm performs a constant number of calculations.
    
    \item \textbf{Space Complexity:} \(O(1)\). Only a fixed amount of extra space is used for variables.
\end{itemize}

\section*{Python Implementation}

\marginnote{Implementing the area calculation with overlap consideration ensures an accurate and efficient solution.}

Below is the complete Python code implementing the \texttt{computeArea} function:

\begin{fullwidth}
\begin{lstlisting}[language=Python]
from typing import List

class Solution:
    def computeArea(self, A: List[int], B: List[int]) -> int:
        # Calculate area of rectangle A
        areaA = (A[2] - A[0]) * (A[3] - A[1])
        
        # Calculate area of rectangle B
        areaB = (B[2] - B[0]) * (B[3] - B[1])
        
        # Determine overlap coordinates
        overlap_x1 = max(A[0], B[0])
        overlap_y1 = max(A[1], B[1])
        overlap_x2 = min(A[2], B[2])
        overlap_y2 = min(A[3], B[3])
        
        # Calculate overlapping area
        overlap_width = overlap_x2 - overlap_x1
        overlap_height = overlap_y2 - overlap_y1
        overlap_area = 0
        if overlap_width > 0 and overlap_height > 0:
            overlap_area = overlap_width * overlap_height
        
        # Total area is sum of individual areas minus overlapping area
        total_area = areaA + areaB - overlap_area
        return total_area

# Example usage:
solution = Solution()
print(solution.computeArea([-3,0,3,4], [0,-1,9,2]))  # Output: 45
print(solution.computeArea([0,0,0,0], [0,0,0,0]))    # Output: 0
print(solution.computeArea([0,0,2,2], [1,1,3,3]))    # Output: 7
print(solution.computeArea([0,0,1,1], [1,0,2,1]))    # Output: 2
\end{lstlisting}
\end{fullwidth}

This implementation accurately computes the total area covered by two rectangles by accounting for any overlapping regions. It ensures that the overlapping area is not double-counted.

\section*{Explanation}

The \texttt{computeArea} function calculates the combined area of two axis-aligned rectangles by following these steps:

\subsection*{1. Calculate Individual Areas}

\begin{itemize}
    \item **Rectangle A:**
    \begin{itemize}
        \item Width: \(A[2] - A[0]\)
        \item Height: \(A[3] - A[1]\)
        \item Area: Width \(\times\) Height
    \end{itemize}
    
    \item **Rectangle B:**
    \begin{itemize}
        \item Width: \(B[2] - B[0]\)
        \item Height: \(B[3] - B[1]\)
        \item Area: Width \(\times\) Height
    \end{itemize}
\end{itemize}

\subsection*{2. Determine Overlapping Area}

\begin{itemize}
    \item **Overlap Coordinates:**
    \begin{itemize}
        \item Left (x-coordinate): \(\text{max}(A[0], B[0])\)
        \item Bottom (y-coordinate): \(\text{max}(A[1], B[1])\)
        \item Right (x-coordinate): \(\text{min}(A[2], B[2])\)
        \item Top (y-coordinate): \(\text{min}(A[3], B[3])\)
    \end{itemize}
    
    \item **Overlap Dimensions:**
    \begin{itemize}
        \item Width: \(\text{overlap\_x2} - \text{overlap\_x1}\)
        \item Height: \(\text{overlap\_y2} - \text{overlap\_y1}\)
    \end{itemize}
    
    \item **Overlap Area:**
    \begin{itemize}
        \item If both width and height are positive, the rectangles overlap, and the overlapping area is their product.
        \item Otherwise, there is no overlap, and the overlapping area is zero.
    \end{itemize}
\end{itemize}

\subsection*{3. Compute Total Area}

\begin{itemize}
    \item Total Area = Area of Rectangle A + Area of Rectangle B - Overlapping Area
\end{itemize}

\subsection*{4. Example Walkthrough}

Consider the first example:
\begin{verbatim}
Input: A = [-3,0,3,4], B = [0,-1,9,2]
Output: 45
\end{verbatim}

\begin{enumerate}
    \item **Calculate Areas:**
    \begin{itemize}
        \item Area of A = (3 - (-3)) * (4 - 0) = 6 * 4 = 24
        \item Area of B = (9 - 0) * (2 - (-1)) = 9 * 3 = 27
    \end{itemize}
    
    \item **Determine Overlap:**
    \begin{itemize}
        \item overlap\_x1 = max(-3, 0) = 0
        \item overlap\_y1 = max(0, -1) = 0
        \item overlap\_x2 = min(3, 9) = 3
        \item overlap\_y2 = min(4, 2) = 2
        \item overlap\_width = 3 - 0 = 3
        \item overlap\_height = 2 - 0 = 2
        \item overlap\_area = 3 * 2 = 6
    \end{itemize}
    
    \item **Compute Total Area:**
    \begin{itemize}
        \item Total Area = 24 + 27 - 6 = 45
    \end{itemize}
\end{enumerate}

Thus, the function correctly returns \texttt{45}.

\section*{Why This Approach}

This approach is chosen for its straightforwardness and optimal efficiency. By directly calculating the individual areas and intelligently handling the overlapping region, the algorithm ensures accurate results without unnecessary computations. Its constant time complexity makes it highly efficient, even for large coordinate values.

\section*{Alternative Approaches}

\subsection*{1. Using Intersection Dimensions}

Instead of separately calculating areas, directly compute the dimensions of the overlapping region and subtract it from the sum of individual areas.

\begin{lstlisting}[language=Python]
def computeArea(A: List[int], B: List[int]) -> int:
    # Sum of individual areas
    area = (A[2] - A[0]) * (A[3] - A[1]) + (B[2] - B[0]) * (B[3] - B[1])
    
    # Overlapping area
    overlap_width = min(A[2], B[2]) - max(A[0], B[0])
    overlap_height = min(A[3], B[3]) - max(A[1], B[1])
    
    if overlap_width > 0 and overlap_height > 0:
        area -= overlap_width * overlap_height
    
    return area
\end{lstlisting}

\subsection*{2. Using Geometry Libraries}

Leverage computational geometry libraries to handle area calculations and overlapping detections.

\begin{lstlisting}[language=Python]
from shapely.geometry import box

def computeArea(A: List[int], B: List[int]) -> int:
    rect1 = box(A[0], A[1], A[2], A[3])
    rect2 = box(B[0], B[1], B[2], B[3])
    intersection = rect1.intersection(rect2)
    return int(rect1.area + rect2.area - intersection.area)
\end{lstlisting}

\textbf{Note}: This approach requires the \texttt{shapely} library and is more suitable for complex geometric operations.

\section*{Similar Problems to This One}

Several problems involve calculating areas, handling geometric overlaps, and spatial reasoning, utilizing similar algorithmic strategies:

\begin{itemize}
    \item \textbf{Rectangle Overlap}: Determine if two rectangles overlap.
    \item \textbf{Circle Area Overlap}: Calculate the overlapping area between two circles.
    \item \textbf{Polygon Area}: Compute the area of a given polygon.
    \item \textbf{Union of Rectangles}: Calculate the total area covered by multiple rectangles, accounting for overlaps.
    \item \textbf{Intersection of Lines}: Find the intersection point of two lines.
    \item \textbf{Closest Pair of Points}: Find the closest pair of points in a set.
    \item \textbf{Convex Hull}: Compute the convex hull of a set of points.
    \item \textbf{Point Inside Polygon}: Determine if a point lies inside a given polygon.
\end{itemize}

These problems reinforce concepts of geometric calculations, area computations, and efficient algorithm design in various contexts.

\section*{Things to Keep in Mind and Tricks}

When tackling the \textbf{Rectangle Area} problem, consider the following tips and best practices to enhance efficiency and correctness:

\begin{itemize}
    \item \textbf{Understand Geometric Relationships}: Grasp the positional relationships between rectangles to simplify area calculations.
    \index{Geometric Relationships}
    
    \item \textbf{Leverage Coordinate Comparisons}: Use direct comparisons of rectangle coordinates to determine overlapping regions.
    \index{Coordinate Comparisons}
    
    \item \textbf{Handle Overlapping Scenarios}: Accurately calculate the overlapping area to avoid double-counting.
    \index{Overlapping Scenarios}
    
    \item \textbf{Optimize for Efficiency}: Aim for a constant time \(O(1)\) solution by avoiding unnecessary computations or iterations.
    \index{Efficiency Optimization}
    
    \item \textbf{Avoid Floating-Point Precision Issues}: Since all coordinates are integers, floating-point precision is not a concern, simplifying the implementation.
    \index{Floating-Point Precision}
    
    \item \textbf{Use Helper Functions}: Create helper functions to encapsulate repetitive tasks, such as calculating overlap dimensions or areas.
    \index{Helper Functions}
    
    \item \textbf{Code Readability}: Maintain clear and readable code through meaningful variable names and structured logic.
    \index{Code Readability}
    
    \item \textbf{Test Extensively}: Implement a wide range of test cases, including overlapping, non-overlapping, and edge-touching rectangles, to ensure robustness.
    \index{Extensive Testing}
    
    \item \textbf{Understand Axis-Aligned Constraints}: Recognize that axis-aligned rectangles simplify area calculations compared to rotated rectangles.
    \index{Axis-Aligned Constraints}
    
    \item \textbf{Simplify Logical Conditions}: Combine multiple conditions logically to streamline the area calculation process.
    \index{Logical Conditions}
    
    \item \textbf{Use Absolute Values}: When calculating differences, ensure that the dimensions are positive by using absolute values or proper ordering.
    \index{Absolute Values}
    
    \item \textbf{Consider Edge Cases}: Handle cases where rectangles have zero area or touch at edges/corners without overlapping.
    \index{Edge Cases}
\end{itemize}

\section*{Corner and Special Cases to Test When Writing the Code}

When implementing the solution for the \textbf{Rectangle Area} problem, it is crucial to consider and rigorously test various edge cases to ensure robustness and correctness:

\begin{itemize}
    \item \textbf{No Overlap}: Rectangles are completely separate.
    \index{No Overlap}
    
    \item \textbf{Partial Overlap}: Rectangles overlap in one or more regions.
    \index{Partial Overlap}
    
    \item \textbf{Edge Touching}: Rectangles touch exactly at one edge without overlapping.
    \index{Edge Touching}
    
    \item \textbf{Corner Touching}: Rectangles touch exactly at one corner without overlapping.
    \index{Corner Touching}
    
    \item \textbf{One Rectangle Inside Another}: One rectangle is entirely within the other.
    \index{Rectangle Inside}
    
    \item \textbf{Identical Rectangles}: Both rectangles have the same coordinates.
    \index{Identical Rectangles}
    
    \item \textbf{Degenerate Rectangles}: Rectangles with zero area (e.g., \(x1 = x2\) or \(y1 = y2\)).
    \index{Degenerate Rectangles}
    
    \item \textbf{Large Coordinates}: Rectangles with very large coordinate values to test performance and integer handling.
    \index{Large Coordinates}
    
    \item \textbf{Negative Coordinates}: Rectangles positioned in negative coordinate space.
    \index{Negative Coordinates}
    
    \item \textbf{Mixed Overlapping Scenarios}: Combinations of the above cases to ensure comprehensive coverage.
    \index{Mixed Overlapping Scenarios}
    
    \item \textbf{Minimum and Maximum Bounds}: Rectangles at the minimum and maximum limits of the coordinate range.
    \index{Minimum and Maximum Bounds}
    
    \item \textbf{Sequential Rectangles}: Multiple rectangles placed sequentially without overlapping.
    \index{Sequential Rectangles}
    
    \item \textbf{Multiple Overlaps}: Scenarios where more than two rectangles overlap in different regions.
    \index{Multiple Overlaps}
\end{itemize}

\section*{Implementation Considerations}

When implementing the \texttt{computeArea} function, keep in mind the following considerations to ensure robustness and efficiency:

\begin{itemize}
    \item \textbf{Data Type Selection}: Use appropriate data types that can handle large input values without overflow or underflow.
    \index{Data Type Selection}
    
    \item \textbf{Optimizing Comparisons}: Structure logical conditions to efficiently determine overlap dimensions.
    \index{Optimizing Comparisons}
    
    \item \textbf{Handling Large Inputs}: Design the algorithm to efficiently handle large input sizes without significant performance degradation.
    \index{Handling Large Inputs}
    
    \item \textbf{Language-Specific Constraints}: Be aware of how the programming language handles large integers and arithmetic operations.
    \index{Language-Specific Constraints}
    
    \item \textbf{Avoiding Redundant Calculations}: Ensure that each calculation contributes towards determining the final area without unnecessary repetitions.
    \index{Avoiding Redundant Calculations}
    
    \item \textbf{Code Readability and Documentation}: Maintain clear and readable code through meaningful variable names and comprehensive comments to facilitate understanding and maintenance.
    \index{Code Readability}
    
    \item \textbf{Edge Case Handling}: Implement checks for edge cases to prevent incorrect results or runtime errors.
    \index{Edge Case Handling}
    
    \item \textbf{Testing and Validation}: Develop a comprehensive suite of test cases that cover all possible scenarios, including edge cases, to validate the correctness and efficiency of the implementation.
    \index{Testing and Validation}
    
    \item \textbf{Scalability}: Design the algorithm to scale efficiently with increasing input sizes, maintaining performance and resource utilization.
    \index{Scalability}
    
    \item \textbf{Using Helper Functions}: Consider creating helper functions for repetitive tasks, such as calculating overlap dimensions, to enhance modularity and reusability.
    \index{Helper Functions}
    
    \item \textbf{Consistent Naming Conventions}: Use consistent and descriptive naming conventions for variables to improve code clarity.
    \index{Naming Conventions}
    
    \item \textbf{Implementing Unit Tests}: Develop unit tests for each logical condition to ensure that all scenarios are correctly handled.
    \index{Unit Tests}
    
    \item \textbf{Error Handling}: Incorporate error handling to manage invalid inputs gracefully.
    \index{Error Handling}
\end{itemize}

\section*{Conclusion}

The \textbf{Rectangle Area} problem showcases the application of fundamental geometric principles and efficient algorithm design to compute spatial properties accurately. By systematically calculating individual areas and intelligently handling overlapping regions, the algorithm ensures precise results without redundant computations. Understanding and implementing such techniques not only enhances problem-solving skills but also provides a foundation for tackling more complex Computational Geometry challenges involving multiple geometric entities and intricate spatial relationships.

\printindex

% \input{sections/rectangle_overlap}
% \input{sections/rectangle_area}
% \input{sections/k_closest_points_to_origin}
% \input{sections/the_skyline_problem}
% % filename: k_closest_points_to_origin.tex

\problemsection{K Closest Points to Origin}
\label{chap:K_Closest_Points_to_Origin}
\marginnote{\href{https://leetcode.com/problems/k-closest-points-to-origin/}{[LeetCode Link]}\index{LeetCode}}
\marginnote{\href{https://www.geeksforgeeks.org/find-k-closest-points-origin/}{[GeeksForGeeks Link]}\index{GeeksForGeeks}}
\marginnote{\href{https://www.interviewbit.com/problems/k-closest-points/}{[InterviewBit Link]}\index{InterviewBit}}
\marginnote{\href{https://app.codesignal.com/challenges/k-closest-points-to-origin}{[CodeSignal Link]}\index{CodeSignal}}
\marginnote{\href{https://www.codewars.com/kata/k-closest-points-to-origin/train/python}{[Codewars Link]}\index{Codewars}}

The \textbf{K Closest Points to Origin} problem is a popular algorithmic challenge in Computational Geometry that involves identifying the \(k\) points closest to the origin in a 2D plane. This problem tests one's ability to apply efficient sorting and selection algorithms, understand distance computations, and optimize for performance. Mastery of this problem is essential for applications in spatial data analysis, nearest neighbor searches, and clustering algorithms.

\section*{Problem Statement}

Given an array of points where each point is represented as \([x, y]\) in the 2D plane, and an integer \(k\), return the \(k\) closest points to the origin \((0, 0)\).

The distance between two points \((x_1, y_1)\) and \((x_2, y_2)\) is the Euclidean distance \(\sqrt{(x_1 - x_2)^2 + (y_1 - y_2)^2}\). The origin is \((0, 0)\).

\textbf{Function signature in Python:}
\begin{lstlisting}[language=Python]
def kClosest(points: List[List[int]], K: int) -> List[List[int]]:
\end{lstlisting}

\section*{Examples}

\textbf{Example 1:}

\begin{verbatim}
Input: points = [[1,3],[-2,2]], K = 1
Output: [[-2,2]]
Explanation: 
The distance between (1, 3) and the origin is sqrt(10).
The distance between (-2, 2) and the origin is sqrt(8).
Since sqrt(8) < sqrt(10), (-2, 2) is closer to the origin.
\end{verbatim}

\textbf{Example 2:}

\begin{verbatim}
Input: points = [[3,3],[5,-1],[-2,4]], K = 2
Output: [[3,3],[-2,4]]
Explanation: 
The distances are sqrt(18), sqrt(26), and sqrt(20) respectively.
The two closest points are [3,3] and [-2,4].
\end{verbatim}

\textbf{Example 3:}

\begin{verbatim}
Input: points = [[0,1],[1,0]], K = 2
Output: [[0,1],[1,0]]
Explanation: 
Both points are equally close to the origin.
\end{verbatim}

\textbf{Example 4:}

\begin{verbatim}
Input: points = [[1,0],[0,1]], K = 1
Output: [[1,0]]
Explanation: 
Both points are equally close; returning any one is acceptable.
\end{verbatim}

\textbf{Constraints:}

\begin{itemize}
    \item \(1 \leq K \leq \text{points.length} \leq 10^4\)
    \item \(-10^4 < x_i, y_i < 10^4\)
\end{itemize}

LeetCode link: \href{https://leetcode.com/problems/k-closest-points-to-origin/}{K Closest Points to Origin}\index{LeetCode}

\section*{Algorithmic Approach}

To identify the \(k\) closest points to the origin, several algorithmic strategies can be employed. The most efficient methods aim to reduce the time complexity by avoiding the need to sort the entire list of points.

\subsection*{1. Sorting Based on Distance}

Calculate the Euclidean distance of each point from the origin and sort the points based on these distances. Select the first \(k\) points from the sorted list.

\begin{enumerate}
    \item Compute the distance for each point using the formula \(distance = x^2 + y^2\).
    \item Sort the points based on the computed distances.
    \item Return the first \(k\) points from the sorted list.
\end{enumerate}

\subsection*{2. Max Heap (Priority Queue)}

Use a max heap to maintain the \(k\) closest points. Iterate through each point, add it to the heap, and if the heap size exceeds \(k\), remove the farthest point.

\begin{enumerate}
    \item Initialize a max heap.
    \item For each point, compute its distance and add it to the heap.
    \item If the heap size exceeds \(k\), remove the point with the largest distance.
    \item After processing all points, the heap contains the \(k\) closest points.
\end{enumerate}

\subsection*{3. QuickSelect (Quick Sort Partitioning)}

Utilize the QuickSelect algorithm to find the \(k\) closest points without fully sorting the list.

\begin{enumerate}
    \item Choose a pivot point and partition the list based on distances relative to the pivot.
    \item Recursively apply QuickSelect to the partition containing the \(k\) closest points.
    \item Once the \(k\) closest points are identified, return them.
\end{enumerate}

\marginnote{QuickSelect offers an average time complexity of \(O(n)\), making it highly efficient for large datasets.}

\section*{Complexities}

\begin{itemize}
    \item \textbf{Sorting Based on Distance:}
    \begin{itemize}
        \item \textbf{Time Complexity:} \(O(n \log n)\)
        \item \textbf{Space Complexity:} \(O(n)\)
    \end{itemize}
    
    \item \textbf{Max Heap (Priority Queue):}
    \begin{itemize}
        \item \textbf{Time Complexity:} \(O(n \log k)\)
        \item \textbf{Space Complexity:} \(O(k)\)
    \end{itemize}
    
    \item \textbf{QuickSelect (Quick Sort Partitioning):}
    \begin{itemize}
        \item \textbf{Time Complexity:} Average case \(O(n)\), worst case \(O(n^2)\)
        \item \textbf{Space Complexity:} \(O(1)\) (in-place)
    \end{itemize}
\end{itemize}

\section*{Python Implementation}

\marginnote{Implementing QuickSelect provides an optimal average-case solution with linear time complexity.}

Below is the complete Python code implementing the \texttt{kClosest} function using the QuickSelect approach:

\begin{fullwidth}
\begin{lstlisting}[language=Python]
from typing import List
import random

class Solution:
    def kClosest(self, points: List[List[int]], K: int) -> List[List[int]]:
        def quickselect(left, right, K_smallest):
            if left == right:
                return
            
            # Select a random pivot_index
            pivot_index = random.randint(left, right)
            
            # Partition the array
            pivot_index = partition(left, right, pivot_index)
            
            # The pivot is in its final sorted position
            if K_smallest == pivot_index:
                return
            elif K_smallest < pivot_index:
                quickselect(left, pivot_index - 1, K_smallest)
            else:
                quickselect(pivot_index + 1, right, K_smallest)
        
        def partition(left, right, pivot_index):
            pivot_distance = distance(points[pivot_index])
            # Move pivot to end
            points[pivot_index], points[right] = points[right], points[pivot_index]
            store_index = left
            for i in range(left, right):
                if distance(points[i]) < pivot_distance:
                    points[store_index], points[i] = points[i], points[store_index]
                    store_index += 1
            # Move pivot to its final place
            points[right], points[store_index] = points[store_index], points[right]
            return store_index
        
        def distance(point):
            return point[0] ** 2 + point[1] ** 2
        
        n = len(points)
        quickselect(0, n - 1, K)
        return points[:K]

# Example usage:
solution = Solution()
print(solution.kClosest([[1,3],[-2,2]], 1))            # Output: [[-2,2]]
print(solution.kClosest([[3,3],[5,-1],[-2,4]], 2))     # Output: [[3,3],[-2,4]]
print(solution.kClosest([[0,1],[1,0]], 2))             # Output: [[0,1],[1,0]]
print(solution.kClosest([[1,0],[0,1]], 1))             # Output: [[1,0]] or [[0,1]]
\end{lstlisting}
\end{fullwidth}

This implementation uses the QuickSelect algorithm to efficiently find the \(k\) closest points to the origin without fully sorting the entire list. It ensures optimal performance even with large datasets.

\section*{Explanation}

The \texttt{kClosest} function identifies the \(k\) closest points to the origin using the QuickSelect algorithm. Here's a detailed breakdown of the implementation:

\subsection*{1. Distance Calculation}

\begin{itemize}
    \item The Euclidean distance is calculated as \(distance = x^2 + y^2\). Since we only need relative distances for comparison, the square root is omitted for efficiency.
\end{itemize}

\subsection*{2. QuickSelect Algorithm}

\begin{itemize}
    \item **Pivot Selection:**
    \begin{itemize}
        \item A random pivot is chosen to enhance the average-case performance.
    \end{itemize}
    
    \item **Partitioning:**
    \begin{itemize}
        \item The array is partitioned such that points with distances less than the pivot are moved to the left, and others to the right.
        \item The pivot is placed in its correct sorted position.
    \end{itemize}
    
    \item **Recursive Selection:**
    \begin{itemize}
        \item If the pivot's position matches \(K\), the selection is complete.
        \item Otherwise, recursively apply QuickSelect to the relevant partition.
    \end{itemize}
\end{itemize}

\subsection*{3. Final Selection}

\begin{itemize}
    \item After partitioning, the first \(K\) points in the list are the \(k\) closest points to the origin.
\end{itemize}

\subsection*{4. Example Walkthrough}

Consider the first example:
\begin{verbatim}
Input: points = [[1,3],[-2,2]], K = 1
Output: [[-2,2]]
\end{verbatim}

\begin{enumerate}
    \item **Calculate Distances:**
    \begin{itemize}
        \item [1,3] : \(1^2 + 3^2 = 10\)
        \item [-2,2] : \((-2)^2 + 2^2 = 8\)
    \end{itemize}
    
    \item **QuickSelect Process:**
    \begin{itemize}
        \item Choose a pivot, say [1,3] with distance 10.
        \item Compare and rearrange:
        \begin{itemize}
            \item [-2,2] has a smaller distance (8) and is moved to the left.
        \end{itemize}
        \item After partitioning, the list becomes [[-2,2], [1,3]].
        \item Since \(K = 1\), return the first point: [[-2,2]].
    \end{itemize}
\end{enumerate}

Thus, the function correctly identifies \([-2,2]\) as the closest point to the origin.

\section*{Why This Approach}

The QuickSelect algorithm is chosen for its average-case linear time complexity \(O(n)\), making it highly efficient for large datasets compared to sorting-based methods with \(O(n \log n)\) time complexity. By avoiding the need to sort the entire list, QuickSelect provides an optimal solution for finding the \(k\) closest points.

\section*{Alternative Approaches}

\subsection*{1. Sorting Based on Distance}

Sort all points based on their distances from the origin and select the first \(k\) points.

\begin{lstlisting}[language=Python]
class Solution:
    def kClosest(self, points: List[List[int]], K: int) -> List[List[int]]:
        points.sort(key=lambda P: P[0]**2 + P[1]**2)
        return points[:K]
\end{lstlisting}

\textbf{Complexities:}
\begin{itemize}
    \item \textbf{Time Complexity:} \(O(n \log n)\)
    \item \textbf{Space Complexity:} \(O(1)\)
\end{itemize}

\subsection*{2. Max Heap (Priority Queue)}

Use a max heap to maintain the \(k\) closest points.

\begin{lstlisting}[language=Python]
import heapq

class Solution:
    def kClosest(self, points: List[List[int]], K: int) -> List[List[int]]:
        heap = []
        for (x, y) in points:
            dist = -(x**2 + y**2)  # Max heap using negative distances
            heapq.heappush(heap, (dist, [x, y]))
            if len(heap) > K:
                heapq.heappop(heap)
        return [item[1] for item in heap]
\end{lstlisting}

\textbf{Complexities:}
\begin{itemize}
    \item \textbf{Time Complexity:} \(O(n \log k)\)
    \item \textbf{Space Complexity:} \(O(k)\)
\end{itemize}

\subsection*{3. Using Built-In Functions}

Leverage built-in functions for distance calculation and selection.

\begin{lstlisting}[language=Python]
import math

class Solution:
    def kClosest(self, points: List[List[int]], K: int) -> List[List[int]]:
        points.sort(key=lambda P: math.sqrt(P[0]**2 + P[1]**2))
        return points[:K]
\end{lstlisting}

\textbf{Note}: This method is similar to the sorting approach but uses the actual Euclidean distance.

\section*{Similar Problems to This One}

Several problems involve nearest neighbor searches, spatial data analysis, and efficient selection algorithms, utilizing similar algorithmic strategies:

\begin{itemize}
    \item \textbf{Closest Pair of Points}: Find the closest pair of points in a set.
    \item \textbf{Top K Frequent Elements}: Identify the most frequent elements in a dataset.
    \item \textbf{Kth Largest Element in an Array}: Find the \(k\)-th largest element in an unsorted array.
    \item \textbf{Sliding Window Maximum}: Find the maximum in each sliding window of size \(k\) over an array.
    \item \textbf{Merge K Sorted Lists}: Merge multiple sorted lists into a single sorted list.
    \item \textbf{Find Median from Data Stream}: Continuously find the median of a stream of numbers.
    \item \textbf{Top K Closest Stars}: Find the \(k\) closest stars to Earth based on their distances.
\end{itemize}

These problems reinforce concepts of efficient selection, heap usage, and distance computations in various contexts.

\section*{Things to Keep in Mind and Tricks}

When solving the \textbf{K Closest Points to Origin} problem, consider the following tips and best practices to enhance efficiency and correctness:

\begin{itemize}
    \item \textbf{Understand Distance Calculations}: Grasp the Euclidean distance formula and recognize that the square root can be omitted for comparison purposes.
    \index{Distance Calculations}
    
    \item \textbf{Leverage Efficient Algorithms}: Use QuickSelect or heap-based methods to optimize time complexity, especially for large datasets.
    \index{Efficient Algorithms}
    
    \item \textbf{Handle Ties Appropriately}: Decide how to handle points with identical distances when \(k\) is less than the number of such points.
    \index{Handling Ties}
    
    \item \textbf{Optimize Space Usage}: Choose algorithms that minimize additional space, such as in-place QuickSelect.
    \index{Space Optimization}
    
    \item \textbf{Use Appropriate Data Structures}: Utilize heaps, lists, and helper functions effectively to manage and process data.
    \index{Data Structures}
    
    \item \textbf{Implement Helper Functions}: Create helper functions for distance calculation and partitioning to enhance code modularity.
    \index{Helper Functions}
    
    \item \textbf{Code Readability}: Maintain clear and readable code through meaningful variable names and structured logic.
    \index{Code Readability}
    
    \item \textbf{Test Extensively}: Implement a wide range of test cases, including edge cases like multiple points with the same distance, to ensure robustness.
    \index{Extensive Testing}
    
    \item \textbf{Understand Algorithm Trade-offs}: Recognize the trade-offs between different approaches in terms of time and space complexities.
    \index{Algorithm Trade-offs}
    
    \item \textbf{Use Built-In Sorting Functions}: When using sorting-based approaches, leverage built-in functions for efficiency and simplicity.
    \index{Built-In Sorting}
    
    \item \textbf{Avoid Redundant Calculations}: Ensure that distance calculations are performed only when necessary to optimize performance.
    \index{Avoiding Redundant Calculations}
    
    \item \textbf{Language-Specific Features}: Utilize language-specific features or libraries that can simplify implementation, such as heapq in Python.
    \index{Language-Specific Features}
\end{itemize}

\section*{Corner and Special Cases to Test When Writing the Code}

When implementing the solution for the \textbf{K Closest Points to Origin} problem, it is crucial to consider and rigorously test various edge cases to ensure robustness and correctness:

\begin{itemize}
    \item \textbf{Multiple Points with Same Distance}: Ensure that the algorithm handles multiple points having the same distance from the origin.
    \index{Same Distance Points}
    
    \item \textbf{Points at Origin}: Include points that are exactly at the origin \((0,0)\).
    \index{Points at Origin}
    
    \item \textbf{Negative Coordinates}: Ensure that the algorithm correctly computes distances for points with negative \(x\) or \(y\) coordinates.
    \index{Negative Coordinates}
    
    \item \textbf{Large Coordinates}: Test with points having very large or very small coordinate values to verify integer handling.
    \index{Large Coordinates}
    
    \item \textbf{K Equals Number of Points}: When \(K\) is equal to the number of points, the algorithm should return all points.
    \index{K Equals Number of Points}
    
    \item \textbf{K is One}: Test with \(K = 1\) to ensure the closest point is correctly identified.
    \index{K is One}
    
    \item \textbf{All Points Same}: All points have the same coordinates.
    \index{All Points Same}
    
    \item \textbf{K is Zero}: Although \(K\) is defined to be at least 1, ensure that the algorithm gracefully handles \(K = 0\) if allowed.
    \index{K is Zero}
    
    \item \textbf{Single Point}: Only one point is provided, and \(K = 1\).
    \index{Single Point}
    
    \item \textbf{Mixed Coordinates}: Points with a mix of positive and negative coordinates.
    \index{Mixed Coordinates}
    
    \item \textbf{Points with Zero Distance}: Multiple points at the origin.
    \index{Zero Distance Points}
    
    \item \textbf{Sparse and Dense Points}: Densely packed points and sparsely distributed points.
    \index{Sparse and Dense Points}
    
    \item \textbf{Duplicate Points}: Multiple identical points in the input list.
    \index{Duplicate Points}
    
    \item \textbf{K Greater Than Number of Unique Points}: Ensure that the algorithm handles cases where \(K\) exceeds the number of unique points if applicable.
    \index{K Greater Than Unique Points}
\end{itemize}

\section*{Implementation Considerations}

When implementing the \texttt{kClosest} function, keep in mind the following considerations to ensure robustness and efficiency:

\begin{itemize}
    \item \textbf{Data Type Selection}: Use appropriate data types that can handle large input values without overflow or precision loss.
    \index{Data Type Selection}
    
    \item \textbf{Optimizing Distance Calculations}: Avoid calculating the square root since it is unnecessary for comparison purposes.
    \index{Optimizing Distance Calculations}
    
    \item \textbf{Choosing the Right Algorithm}: Select an algorithm based on the size of the input and the value of \(K\) to optimize time and space complexities.
    \index{Choosing the Right Algorithm}
    
    \item \textbf{Language-Specific Libraries}: Utilize language-specific libraries and functions (e.g., \texttt{heapq} in Python) to simplify implementation and enhance performance.
    \index{Language-Specific Libraries}
    
    \item \textbf{Avoiding Redundant Calculations}: Ensure that each point's distance is calculated only once to optimize performance.
    \index{Avoiding Redundant Calculations}
    
    \item \textbf{Implementing Helper Functions}: Create helper functions for tasks like distance calculation and partitioning to enhance modularity and readability.
    \index{Helper Functions}
    
    \item \textbf{Edge Case Handling}: Implement checks for edge cases to prevent incorrect results or runtime errors.
    \index{Edge Case Handling}
    
    \item \textbf{Testing and Validation}: Develop a comprehensive suite of test cases that cover all possible scenarios, including edge cases, to validate the correctness and efficiency of the implementation.
    \index{Testing and Validation}
    
    \item \textbf{Scalability}: Design the algorithm to scale efficiently with increasing input sizes, maintaining performance and resource utilization.
    \index{Scalability}
    
    \item \textbf{Consistent Naming Conventions}: Use consistent and descriptive naming conventions for variables and functions to improve code clarity.
    \index{Naming Conventions}
    
    \item \textbf{Memory Management}: Ensure that the algorithm manages memory efficiently, especially when dealing with large datasets.
    \index{Memory Management}
    
    \item \textbf{Avoiding Stack Overflow}: If implementing recursive approaches, be mindful of recursion limits and potential stack overflow issues.
    \index{Avoiding Stack Overflow}
    
    \item \textbf{Implementing Iterative Solutions}: Prefer iterative solutions when recursion may lead to increased space complexity or stack overflow.
    \index{Implementing Iterative Solutions}
\end{itemize}

\section*{Conclusion}

The \textbf{K Closest Points to Origin} problem exemplifies the application of efficient selection algorithms and geometric computations to solve spatial challenges effectively. By leveraging QuickSelect or heap-based methods, the algorithm achieves optimal time and space complexities, making it highly suitable for large datasets. Understanding and implementing such techniques not only enhances problem-solving skills but also provides a foundation for tackling more advanced Computational Geometry problems involving nearest neighbor searches, clustering, and spatial data analysis.

\printindex

% \input{sections/rectangle_overlap}
% \input{sections/rectangle_area}
% \input{sections/k_closest_points_to_origin}
% \input{sections/the_skyline_problem}
% % filename: the_skyline_problem.tex

\problemsection{The Skyline Problem}
\label{chap:The_Skyline_Problem}
\marginnote{\href{https://leetcode.com/problems/the-skyline-problem/}{[LeetCode Link]}\index{LeetCode}}
\marginnote{\href{https://www.geeksforgeeks.org/the-skyline-problem/}{[GeeksForGeeks Link]}\index{GeeksForGeeks}}
\marginnote{\href{https://www.interviewbit.com/problems/the-skyline-problem/}{[InterviewBit Link]}\index{InterviewBit}}
\marginnote{\href{https://app.codesignal.com/challenges/the-skyline-problem}{[CodeSignal Link]}\index{CodeSignal}}
\marginnote{\href{https://www.codewars.com/kata/the-skyline-problem/train/python}{[Codewars Link]}\index{Codewars}}

The \textbf{Skyline Problem} is a complex Computational Geometry challenge that involves computing the skyline formed by a collection of buildings in a 2D cityscape. Each building is represented by its left and right x-coordinates and its height. The skyline is defined by a list of "key points" where the height changes. This problem tests one's ability to handle large datasets, implement efficient sweep line algorithms, and manage event-driven processing. Mastery of this problem is essential for applications in computer graphics, urban planning simulations, and geographic information systems (GIS).

\section*{Problem Statement}

You are given a list of buildings in a cityscape. Each building is represented as a triplet \([Li, Ri, Hi]\), where \(Li\) and \(Ri\) are the x-coordinates of the left and right edges of the building, respectively, and \(Hi\) is the height of the building.

The skyline should be represented as a list of key points \([x, y]\) in sorted order by \(x\)-coordinate, where \(y\) is the height of the skyline at that point. The skyline should only include critical points where the height changes.

\textbf{Function signature in Python:}
\begin{lstlisting}[language=Python]
def getSkyline(buildings: List[List[int]]) -> List[List[int]]:
\end{lstlisting}

\section*{Examples}

\textbf{Example 1:}

\begin{verbatim}
Input: buildings = [[2,9,10], [3,7,15], [5,12,12], [15,20,10], [19,24,8]]
Output: [[2,10], [3,15], [7,12], [12,0], [15,10], [20,8], [24,0]]
Explanation:
- At x=2, the first building starts, height=10.
- At x=3, the second building starts, height=15.
- At x=7, the second building ends, the third building is still ongoing, height=12.
- At x=12, the third building ends, height drops to 0.
- At x=15, the fourth building starts, height=10.
- At x=20, the fourth building ends, the fifth building is still ongoing, height=8.
- At x=24, the fifth building ends, height drops to 0.
\end{verbatim}

\textbf{Example 2:}

\begin{verbatim}
Input: buildings = [[0,2,3], [2,5,3]]
Output: [[0,3], [5,0]]
Explanation:
- The two buildings are contiguous and have the same height, so the skyline drops to 0 at x=5.
\end{verbatim}

\textbf{Example 3:}

\begin{verbatim}
Input: buildings = [[1,3,3], [2,4,4], [5,6,1]]
Output: [[1,3], [2,4], [4,0], [5,1], [6,0]]
Explanation:
- At x=1, first building starts, height=3.
- At x=2, second building starts, height=4.
- At x=4, second building ends, height drops to 0.
- At x=5, third building starts, height=1.
- At x=6, third building ends, height drops to 0.
\end{verbatim}

\textbf{Example 4:}

\begin{verbatim}
Input: buildings = [[0,5,0]]
Output: []
Explanation:
- A building with height 0 does not contribute to the skyline.
\end{verbatim}

\textbf{Constraints:}

\begin{itemize}
    \item \(1 \leq \text{buildings.length} \leq 10^4\)
    \item \(0 \leq Li < Ri \leq 10^9\)
    \item \(0 \leq Hi \leq 10^4\)
\end{itemize}

\section*{Algorithmic Approach}

The \textbf{Sweep Line Algorithm} is an efficient method for solving the Skyline Problem. It involves processing events (building start and end points) in sorted order while maintaining a data structure (typically a max heap) to keep track of active buildings. Here's a step-by-step approach:

\subsection*{1. Event Representation}

Transform each building into two events:
\begin{itemize}
    \item **Start Event:** \((Li, -Hi)\) – Negative height indicates a building starts.
    \item **End Event:** \((Ri, Hi)\) – Positive height indicates a building ends.
\end{itemize}

Sorting the events ensures that start events are processed before end events at the same x-coordinate, and taller buildings are processed before shorter ones.

\subsection*{2. Sorting the Events}

Sort all events based on:
\begin{enumerate}
    \item **x-coordinate:** Ascending order.
    \item **Height:**
    \begin{itemize}
        \item For start events, taller buildings come first.
        \item For end events, shorter buildings come first.
    \end{itemize}
\end{enumerate}

\subsection*{3. Processing the Events}

Use a max heap to keep track of active building heights. Iterate through the sorted events:
\begin{enumerate}
    \item **Start Event:**
    \begin{itemize}
        \item Add the building's height to the heap.
    \end{itemize}
    
    \item **End Event:**
    \begin{itemize}
        \item Remove the building's height from the heap.
    \end{itemize}
    
    \item **Determine Current Max Height:**
    \begin{itemize}
        \item The current max height is the top of the heap.
    \end{itemize}
    
    \item **Update Skyline:**
    \begin{itemize}
        \item If the current max height differs from the previous max height, add a new key point \([x, current\_max\_height]\).
    \end{itemize}
\end{enumerate}

\subsection*{4. Finalizing the Skyline}

After processing all events, the accumulated key points represent the skyline.

\marginnote{The Sweep Line Algorithm efficiently handles dynamic changes in active buildings, ensuring accurate skyline construction.}

\section*{Complexities}

\begin{itemize}
    \item \textbf{Time Complexity:} \(O(n \log n)\), where \(n\) is the number of buildings. Sorting the events takes \(O(n \log n)\), and each heap operation takes \(O(\log n)\).
    
    \item \textbf{Space Complexity:} \(O(n)\), due to the storage of events and the heap.
\end{itemize}

\section*{Python Implementation}

\marginnote{Implementing the Sweep Line Algorithm with a max heap ensures an efficient and accurate solution.}

Below is the complete Python code implementing the \texttt{getSkyline} function:

\begin{fullwidth}
\begin{lstlisting}[language=Python]
from typing import List
import heapq

class Solution:
    def getSkyline(self, buildings: List[List[int]]) -> List[List[int]]:
        # Create a list of all events
        # For start events, use negative height to ensure they are processed before end events
        events = []
        for L, R, H in buildings:
            events.append((L, -H))
            events.append((R, H))
        
        # Sort the events
        # First by x-coordinate, then by height
        events.sort()
        
        # Max heap to keep track of active buildings
        heap = [0]  # Initialize with ground level
        heapq.heapify(heap)
        active_heights = {0: 1}  # Dictionary to count heights
        
        result = []
        prev_max = 0
        
        for x, h in events:
            if h < 0:
                # Start of a building, add height to heap and dictionary
                heapq.heappush(heap, h)
                active_heights[h] = active_heights.get(h, 0) + 1
            else:
                # End of a building, remove height from dictionary
                active_heights[h] -= 1
                if active_heights[h] == 0:
                    del active_heights[h]
            
            # Current max height
            while heap and active_heights.get(heap[0], 0) == 0:
                heapq.heappop(heap)
            current_max = -heap[0] if heap else 0
            
            # If the max height has changed, add to result
            if current_max != prev_max:
                result.append([x, current_max])
                prev_max = current_max
        
        return result

# Example usage:
solution = Solution()
print(solution.getSkyline([[2,9,10], [3,7,15], [5,12,12], [15,20,10], [19,24,8]]))
# Output: [[2,10], [3,15], [7,12], [12,0], [15,10], [20,8], [24,0]]

print(solution.getSkyline([[0,2,3], [2,5,3]]))
# Output: [[0,3], [5,0]]

print(solution.getSkyline([[1,3,3], [2,4,4], [5,6,1]]))
# Output: [[1,3], [2,4], [4,0], [5,1], [6,0]]

print(solution.getSkyline([[0,5,0]]))
# Output: []
\end{lstlisting}
\end{fullwidth}

This implementation efficiently constructs the skyline by processing all building events in sorted order and maintaining active building heights using a max heap. It ensures that only critical points where the skyline changes are recorded.

\section*{Explanation}

The \texttt{getSkyline} function constructs the skyline formed by a set of buildings by leveraging the Sweep Line Algorithm and a max heap to track active buildings. Here's a detailed breakdown of the implementation:

\subsection*{1. Event Representation}

\begin{itemize}
    \item Each building is transformed into two events:
    \begin{itemize}
        \item **Start Event:** \((Li, -Hi)\) – Negative height indicates the start of a building.
        \item **End Event:** \((Ri, Hi)\) – Positive height indicates the end of a building.
    \end{itemize}
\end{itemize}

\subsection*{2. Sorting the Events}

\begin{itemize}
    \item Events are sorted primarily by their x-coordinate in ascending order.
    \item For events with the same x-coordinate:
    \begin{itemize}
        \item Start events (with negative heights) are processed before end events.
        \item Taller buildings are processed before shorter ones.
    \end{itemize}
\end{itemize}

\subsection*{3. Processing the Events}

\begin{itemize}
    \item **Heap Initialization:**
    \begin{itemize}
        \item A max heap is initialized with a ground level height of 0.
        \item A dictionary \texttt{active\_heights} tracks the count of active building heights.
    \end{itemize}
    
    \item **Iterating Through Events:**
    \begin{enumerate}
        \item **Start Event:**
        \begin{itemize}
            \item Add the building's height to the heap.
            \item Increment the count of the height in \texttt{active\_heights}.
        \end{itemize}
        
        \item **End Event:**
        \begin{itemize}
            \item Decrement the count of the building's height in \texttt{active\_heights}.
            \item If the count reaches zero, remove the height from the dictionary.
        \end{itemize}
        
        \item **Determine Current Max Height:**
        \begin{itemize}
            \item Remove heights from the heap that are no longer active.
            \item The current max height is the top of the heap.
        \end{itemize}
        
        \item **Update Skyline:**
        \begin{itemize}
            \item If the current max height differs from the previous max height, add a new key point \([x, current\_max\_height]\).
        \end{itemize}
    \end{enumerate}
\end{itemize}

\subsection*{4. Finalizing the Skyline}

\begin{itemize}
    \item After processing all events, the \texttt{result} list contains the key points defining the skyline.
\end{itemize}

\subsection*{5. Example Walkthrough}

Consider the first example:
\begin{verbatim}
Input: buildings = [[2,9,10], [3,7,15], [5,12,12], [15,20,10], [19,24,8]]
Output: [[2,10], [3,15], [7,12], [12,0], [15,10], [20,8], [24,0]]
\end{verbatim}

\begin{enumerate}
    \item **Event Transformation:**
    \begin{itemize}
        \item \((2, -10)\), \((9, 10)\)
        \item \((3, -15)\), \((7, 15)\)
        \item \((5, -12)\), \((12, 12)\)
        \item \((15, -10)\), \((20, 10)\)
        \item \((19, -8)\), \((24, 8)\)
    \end{itemize}
    
    \item **Sorting Events:**
    \begin{itemize}
        \item Sorted order: \((2, -10)\), \((3, -15)\), \((5, -12)\), \((7, 15)\), \((9, 10)\), \((12, 12)\), \((15, -10)\), \((19, -8)\), \((20, 10)\), \((24, 8)\)
    \end{itemize}
    
    \item **Processing Events:**
    \begin{itemize}
        \item At each event, update the heap and determine if the skyline height changes.
    \end{itemize}
    
    \item **Result Construction:**
    \begin{itemize}
        \item The resulting skyline key points are accumulated as \([[2,10], [3,15], [7,12], [12,0], [15,10], [20,8], [24,0]]\).
    \end{itemize}
\end{enumerate}

Thus, the function correctly constructs the skyline based on the buildings' positions and heights.

\section*{Why This Approach}

The Sweep Line Algorithm combined with a max heap offers an optimal solution with \(O(n \log n)\) time complexity and efficient handling of overlapping buildings. By processing events in sorted order and maintaining active building heights, the algorithm ensures that all critical points in the skyline are accurately identified without redundant computations.

\section*{Alternative Approaches}

\subsection*{1. Divide and Conquer}

Divide the set of buildings into smaller subsets, compute the skyline for each subset, and then merge the skylines.

\begin{lstlisting}[language=Python]
class Solution:
    def getSkyline(self, buildings: List[List[int]]) -> List[List[int]]:
        def merge(left, right):
            h1, h2 = 0, 0
            i, j = 0, 0
            merged = []
            while i < len(left) and j < len(right):
                if left[i][0] < right[j][0]:
                    x, h1 = left[i]
                    i += 1
                elif left[i][0] > right[j][0]:
                    x, h2 = right[j]
                    j += 1
                else:
                    x, h1 = left[i]
                    _, h2 = right[j]
                    i += 1
                    j += 1
                max_h = max(h1, h2)
                if not merged or merged[-1][1] != max_h:
                    merged.append([x, max_h])
            merged.extend(left[i:])
            merged.extend(right[j:])
            return merged
        
        def divide(buildings):
            if not buildings:
                return []
            if len(buildings) == 1:
                L, R, H = buildings[0]
                return [[L, H], [R, 0]]
            mid = len(buildings) // 2
            left = divide(buildings[:mid])
            right = divide(buildings[mid:])
            return merge(left, right)
        
        return divide(buildings)
\end{lstlisting}

\textbf{Complexities:}
\begin{itemize}
    \item \textbf{Time Complexity:} \(O(n \log n)\)
    \item \textbf{Space Complexity:} \(O(n)\)
\end{itemize}

\subsection*{2. Using Segment Trees}

Implement a segment tree to manage and query overlapping building heights dynamically.

\textbf{Note}: This approach is more complex and is generally used for advanced scenarios with multiple dynamic queries.

\section*{Similar Problems to This One}

Several problems involve skyline-like constructions, spatial data analysis, and efficient event processing, utilizing similar algorithmic strategies:

\begin{itemize}
    \item \textbf{Merge Intervals}: Merge overlapping intervals in a list.
    \item \textbf{Largest Rectangle in Histogram}: Find the largest rectangular area in a histogram.
    \item \textbf{Interval Partitioning}: Assign intervals to resources without overlap.
    \item \textbf{Line Segment Intersection}: Detect intersections among line segments.
    \item \textbf{Closest Pair of Points}: Find the closest pair of points in a set.
    \item \textbf{Convex Hull}: Compute the convex hull of a set of points.
    \item \textbf{Point Inside Polygon}: Determine if a point lies inside a given polygon.
    \item \textbf{Range Searching}: Efficiently query geometric data within a specified range.
\end{itemize}

These problems reinforce concepts of event-driven processing, spatial reasoning, and efficient algorithm design in various contexts.

\section*{Things to Keep in Mind and Tricks}

When tackling the \textbf{Skyline Problem}, consider the following tips and best practices to enhance efficiency and correctness:

\begin{itemize}
    \item \textbf{Understand Sweep Line Technique}: Grasp how the sweep line algorithm processes events in sorted order to handle dynamic changes efficiently.
    \index{Sweep Line Technique}
    
    \item \textbf{Leverage Priority Queues (Heaps)}: Use max heaps to keep track of active buildings' heights, enabling quick access to the current maximum height.
    \index{Priority Queues}
    
    \item \textbf{Handle Start and End Events Differently}: Differentiate between building start and end events to accurately manage active heights.
    \index{Start and End Events}
    
    \item \textbf{Optimize Event Sorting}: Sort events primarily by x-coordinate and secondarily by height to ensure correct processing order.
    \index{Event Sorting}
    
    \item \textbf{Manage Active Heights Efficiently}: Use data structures that allow efficient insertion, deletion, and retrieval of maximum elements.
    \index{Active Heights Management}
    
    \item \textbf{Avoid Redundant Key Points}: Only record key points when the skyline height changes to minimize the output list.
    \index{Avoiding Redundant Key Points}
    
    \item \textbf{Implement Helper Functions}: Create helper functions for tasks like distance calculation, event handling, and heap management to enhance modularity.
    \index{Helper Functions}
    
    \item \textbf{Code Readability}: Maintain clear and readable code through meaningful variable names and structured logic.
    \index{Code Readability}
    
    \item \textbf{Test Extensively}: Implement a wide range of test cases, including overlapping, non-overlapping, and edge-touching buildings, to ensure robustness.
    \index{Extensive Testing}
    
    \item \textbf{Handle Degenerate Cases}: Manage cases where buildings have zero height or identical coordinates gracefully.
    \index{Degenerate Cases}
    
    \item \textbf{Understand Geometric Relationships}: Grasp how buildings overlap and influence the skyline to simplify the algorithm.
    \index{Geometric Relationships}
    
    \item \textbf{Use Appropriate Data Structures}: Utilize appropriate data structures like heaps, lists, and dictionaries to manage and process data efficiently.
    \index{Appropriate Data Structures}
    
    \item \textbf{Optimize for Large Inputs}: Design the algorithm to handle large numbers of buildings without significant performance degradation.
    \index{Optimizing for Large Inputs}
    
    \item \textbf{Implement Iterative Solutions Carefully}: Ensure that loop conditions are correctly defined to prevent infinite loops or incorrect terminations.
    \index{Iterative Solutions}
    
    \item \textbf{Consistent Naming Conventions}: Use consistent and descriptive naming conventions for variables and functions to improve code clarity.
    \index{Naming Conventions}
\end{itemize}

\section*{Corner and Special Cases to Test When Writing the Code}

When implementing the solution for the \textbf{Skyline Problem}, it is crucial to consider and rigorously test various edge cases to ensure robustness and correctness:

\begin{itemize}
    \item \textbf{No Overlapping Buildings}: All buildings are separate and do not overlap.
    \index{No Overlapping Buildings}
    
    \item \textbf{Fully Overlapping Buildings}: Multiple buildings completely overlap each other.
    \index{Fully Overlapping Buildings}
    
    \item \textbf{Buildings Touching at Edges}: Buildings share common edges without overlapping.
    \index{Buildings Touching at Edges}
    
    \item \textbf{Buildings Touching at Corners}: Buildings share common corners without overlapping.
    \index{Buildings Touching at Corners}
    
    \item \textbf{Single Building}: Only one building is present.
    \index{Single Building}
    
    \item \textbf{Multiple Buildings with Same Start or End}: Multiple buildings start or end at the same x-coordinate.
    \index{Same Start or End}
    
    \item \textbf{Buildings with Zero Height}: Buildings that have zero height should not affect the skyline.
    \index{Buildings with Zero Height}
    
    \item \textbf{Large Number of Buildings}: Test with a large number of buildings to ensure performance and scalability.
    \index{Large Number of Buildings}
    
    \item \textbf{Buildings with Negative Coordinates}: Buildings positioned in negative coordinate space.
    \index{Negative Coordinates}
    
    \item \textbf{Boundary Values}: Buildings at the minimum and maximum limits of the coordinate range.
    \index{Boundary Values}
    
    \item \textbf{Buildings with Identical Coordinates}: Multiple buildings with the same coordinates.
    \index{Identical Coordinates}
    
    \item \textbf{Sequential Buildings}: Buildings placed sequentially without gaps.
    \index{Sequential Buildings}
    
    \item \textbf{Overlapping and Non-Overlapping Mixed}: A mix of overlapping and non-overlapping buildings.
    \index{Overlapping and Non-Overlapping Mixed}
    
    \item \textbf{Buildings with Very Large Heights}: Buildings with heights at the upper limit of the constraints.
    \index{Very Large Heights}
    
    \item \textbf{Empty Input}: No buildings are provided.
    \index{Empty Input}
\end{itemize}

\section*{Implementation Considerations}

When implementing the \texttt{getSkyline} function, keep in mind the following considerations to ensure robustness and efficiency:

\begin{itemize}
    \item \textbf{Data Type Selection}: Use appropriate data types that can handle large input values and avoid overflow or precision issues.
    \index{Data Type Selection}
    
    \item \textbf{Optimizing Event Sorting}: Efficiently sort events based on x-coordinates and heights to ensure correct processing order.
    \index{Optimizing Event Sorting}
    
    \item \textbf{Handling Large Inputs}: Design the algorithm to handle up to \(10^4\) buildings efficiently without significant performance degradation.
    \index{Handling Large Inputs}
    
    \item \textbf{Using Efficient Data Structures}: Utilize heaps, lists, and dictionaries effectively to manage and process events and active heights.
    \index{Efficient Data Structures}
    
    \item \textbf{Avoiding Redundant Calculations}: Ensure that distance and overlap calculations are performed only when necessary to optimize performance.
    \index{Avoiding Redundant Calculations}
    
    \item \textbf{Code Readability and Documentation}: Maintain clear and readable code through meaningful variable names and comprehensive comments to facilitate understanding and maintenance.
    \index{Code Readability}
    
    \item \textbf{Edge Case Handling}: Implement checks for edge cases to prevent incorrect results or runtime errors.
    \index{Edge Case Handling}
    
    \item \textbf{Implementing Helper Functions}: Create helper functions for tasks like distance calculation, event handling, and heap management to enhance modularity.
    \index{Helper Functions}
    
    \item \textbf{Consistent Naming Conventions}: Use consistent and descriptive naming conventions for variables and functions to improve code clarity.
    \index{Naming Conventions}
    
    \item \textbf{Memory Management}: Ensure that the algorithm manages memory efficiently, especially when dealing with large datasets.
    \index{Memory Management}
    
    \item \textbf{Implementing Iterative Solutions Carefully}: Ensure that loop conditions are correctly defined to prevent infinite loops or incorrect terminations.
    \index{Iterative Solutions}
    
    \item \textbf{Avoiding Floating-Point Precision Issues}: Since the problem deals with integers, floating-point precision is not a concern, simplifying the implementation.
    \index{Floating-Point Precision}
    
    \item \textbf{Testing and Validation}: Develop a comprehensive suite of test cases that cover all possible scenarios, including edge cases, to validate the correctness and efficiency of the implementation.
    \index{Testing and Validation}
    
    \item \textbf{Performance Considerations}: Optimize the loop conditions and operations to ensure that the function runs efficiently, especially for large input numbers.
    \index{Performance Considerations}
\end{itemize}

\section*{Conclusion}

The \textbf{Skyline Problem} is a quintessential example of applying advanced algorithmic techniques and geometric reasoning to solve complex spatial challenges. By leveraging the Sweep Line Algorithm and maintaining active building heights using a max heap, the solution efficiently constructs the skyline with optimal time and space complexities. Understanding and implementing such sophisticated algorithms not only enhances problem-solving skills but also provides a foundation for tackling a wide array of Computational Geometry problems in various domains, including computer graphics, urban planning simulations, and geographic information systems.

\printindex

% \input{sections/rectangle_overlap}
% \input{sections/rectangle_area}
% \input{sections/k_closest_points_to_origin}
% \input{sections/the_skyline_problem}
% % filename: the_skyline_problem.tex

\problemsection{The Skyline Problem}
\label{chap:The_Skyline_Problem}
\marginnote{\href{https://leetcode.com/problems/the-skyline-problem/}{[LeetCode Link]}\index{LeetCode}}
\marginnote{\href{https://www.geeksforgeeks.org/the-skyline-problem/}{[GeeksForGeeks Link]}\index{GeeksForGeeks}}
\marginnote{\href{https://www.interviewbit.com/problems/the-skyline-problem/}{[InterviewBit Link]}\index{InterviewBit}}
\marginnote{\href{https://app.codesignal.com/challenges/the-skyline-problem}{[CodeSignal Link]}\index{CodeSignal}}
\marginnote{\href{https://www.codewars.com/kata/the-skyline-problem/train/python}{[Codewars Link]}\index{Codewars}}

The \textbf{Skyline Problem} is a complex Computational Geometry challenge that involves computing the skyline formed by a collection of buildings in a 2D cityscape. Each building is represented by its left and right x-coordinates and its height. The skyline is defined by a list of "key points" where the height changes. This problem tests one's ability to handle large datasets, implement efficient sweep line algorithms, and manage event-driven processing. Mastery of this problem is essential for applications in computer graphics, urban planning simulations, and geographic information systems (GIS).

\section*{Problem Statement}

You are given a list of buildings in a cityscape. Each building is represented as a triplet \([Li, Ri, Hi]\), where \(Li\) and \(Ri\) are the x-coordinates of the left and right edges of the building, respectively, and \(Hi\) is the height of the building.

The skyline should be represented as a list of key points \([x, y]\) in sorted order by \(x\)-coordinate, where \(y\) is the height of the skyline at that point. The skyline should only include critical points where the height changes.

\textbf{Function signature in Python:}
\begin{lstlisting}[language=Python]
def getSkyline(buildings: List[List[int]]) -> List[List[int]]:
\end{lstlisting}

\section*{Examples}

\textbf{Example 1:}

\begin{verbatim}
Input: buildings = [[2,9,10], [3,7,15], [5,12,12], [15,20,10], [19,24,8]]
Output: [[2,10], [3,15], [7,12], [12,0], [15,10], [20,8], [24,0]]
Explanation:
- At x=2, the first building starts, height=10.
- At x=3, the second building starts, height=15.
- At x=7, the second building ends, the third building is still ongoing, height=12.
- At x=12, the third building ends, height drops to 0.
- At x=15, the fourth building starts, height=10.
- At x=20, the fourth building ends, the fifth building is still ongoing, height=8.
- At x=24, the fifth building ends, height drops to 0.
\end{verbatim}

\textbf{Example 2:}

\begin{verbatim}
Input: buildings = [[0,2,3], [2,5,3]]
Output: [[0,3], [5,0]]
Explanation:
- The two buildings are contiguous and have the same height, so the skyline drops to 0 at x=5.
\end{verbatim}

\textbf{Example 3:}

\begin{verbatim}
Input: buildings = [[1,3,3], [2,4,4], [5,6,1]]
Output: [[1,3], [2,4], [4,0], [5,1], [6,0]]
Explanation:
- At x=1, first building starts, height=3.
- At x=2, second building starts, height=4.
- At x=4, second building ends, height drops to 0.
- At x=5, third building starts, height=1.
- At x=6, third building ends, height drops to 0.
\end{verbatim}

\textbf{Example 4:}

\begin{verbatim}
Input: buildings = [[0,5,0]]
Output: []
Explanation:
- A building with height 0 does not contribute to the skyline.
\end{verbatim}

\textbf{Constraints:}

\begin{itemize}
    \item \(1 \leq \text{buildings.length} \leq 10^4\)
    \item \(0 \leq Li < Ri \leq 10^9\)
    \item \(0 \leq Hi \leq 10^4\)
\end{itemize}

\section*{Algorithmic Approach}

The \textbf{Sweep Line Algorithm} is an efficient method for solving the Skyline Problem. It involves processing events (building start and end points) in sorted order while maintaining a data structure (typically a max heap) to keep track of active buildings. Here's a step-by-step approach:

\subsection*{1. Event Representation}

Transform each building into two events:
\begin{itemize}
    \item **Start Event:** \((Li, -Hi)\) – Negative height indicates a building starts.
    \item **End Event:** \((Ri, Hi)\) – Positive height indicates a building ends.
\end{itemize}

Sorting the events ensures that start events are processed before end events at the same x-coordinate, and taller buildings are processed before shorter ones.

\subsection*{2. Sorting the Events}

Sort all events based on:
\begin{enumerate}
    \item **x-coordinate:** Ascending order.
    \item **Height:**
    \begin{itemize}
        \item For start events, taller buildings come first.
        \item For end events, shorter buildings come first.
    \end{itemize}
\end{enumerate}

\subsection*{3. Processing the Events}

Use a max heap to keep track of active building heights. Iterate through the sorted events:
\begin{enumerate}
    \item **Start Event:**
    \begin{itemize}
        \item Add the building's height to the heap.
    \end{itemize}
    
    \item **End Event:**
    \begin{itemize}
        \item Remove the building's height from the heap.
    \end{itemize}
    
    \item **Determine Current Max Height:**
    \begin{itemize}
        \item The current max height is the top of the heap.
    \end{itemize}
    
    \item **Update Skyline:**
    \begin{itemize}
        \item If the current max height differs from the previous max height, add a new key point \([x, current\_max\_height]\).
    \end{itemize}
\end{enumerate}

\subsection*{4. Finalizing the Skyline}

After processing all events, the accumulated key points represent the skyline.

\marginnote{The Sweep Line Algorithm efficiently handles dynamic changes in active buildings, ensuring accurate skyline construction.}

\section*{Complexities}

\begin{itemize}
    \item \textbf{Time Complexity:} \(O(n \log n)\), where \(n\) is the number of buildings. Sorting the events takes \(O(n \log n)\), and each heap operation takes \(O(\log n)\).
    
    \item \textbf{Space Complexity:} \(O(n)\), due to the storage of events and the heap.
\end{itemize}

\section*{Python Implementation}

\marginnote{Implementing the Sweep Line Algorithm with a max heap ensures an efficient and accurate solution.}

Below is the complete Python code implementing the \texttt{getSkyline} function:

\begin{fullwidth}
\begin{lstlisting}[language=Python]
from typing import List
import heapq

class Solution:
    def getSkyline(self, buildings: List[List[int]]) -> List[List[int]]:
        # Create a list of all events
        # For start events, use negative height to ensure they are processed before end events
        events = []
        for L, R, H in buildings:
            events.append((L, -H))
            events.append((R, H))
        
        # Sort the events
        # First by x-coordinate, then by height
        events.sort()
        
        # Max heap to keep track of active buildings
        heap = [0]  # Initialize with ground level
        heapq.heapify(heap)
        active_heights = {0: 1}  # Dictionary to count heights
        
        result = []
        prev_max = 0
        
        for x, h in events:
            if h < 0:
                # Start of a building, add height to heap and dictionary
                heapq.heappush(heap, h)
                active_heights[h] = active_heights.get(h, 0) + 1
            else:
                # End of a building, remove height from dictionary
                active_heights[h] -= 1
                if active_heights[h] == 0:
                    del active_heights[h]
            
            # Current max height
            while heap and active_heights.get(heap[0], 0) == 0:
                heapq.heappop(heap)
            current_max = -heap[0] if heap else 0
            
            # If the max height has changed, add to result
            if current_max != prev_max:
                result.append([x, current_max])
                prev_max = current_max
        
        return result

# Example usage:
solution = Solution()
print(solution.getSkyline([[2,9,10], [3,7,15], [5,12,12], [15,20,10], [19,24,8]]))
# Output: [[2,10], [3,15], [7,12], [12,0], [15,10], [20,8], [24,0]]

print(solution.getSkyline([[0,2,3], [2,5,3]]))
# Output: [[0,3], [5,0]]

print(solution.getSkyline([[1,3,3], [2,4,4], [5,6,1]]))
# Output: [[1,3], [2,4], [4,0], [5,1], [6,0]]

print(solution.getSkyline([[0,5,0]]))
# Output: []
\end{lstlisting}
\end{fullwidth}

This implementation efficiently constructs the skyline by processing all building events in sorted order and maintaining active building heights using a max heap. It ensures that only critical points where the skyline changes are recorded.

\section*{Explanation}

The \texttt{getSkyline} function constructs the skyline formed by a set of buildings by leveraging the Sweep Line Algorithm and a max heap to track active buildings. Here's a detailed breakdown of the implementation:

\subsection*{1. Event Representation}

\begin{itemize}
    \item Each building is transformed into two events:
    \begin{itemize}
        \item **Start Event:** \((Li, -Hi)\) – Negative height indicates the start of a building.
        \item **End Event:** \((Ri, Hi)\) – Positive height indicates the end of a building.
    \end{itemize}
\end{itemize}

\subsection*{2. Sorting the Events}

\begin{itemize}
    \item Events are sorted primarily by their x-coordinate in ascending order.
    \item For events with the same x-coordinate:
    \begin{itemize}
        \item Start events (with negative heights) are processed before end events.
        \item Taller buildings are processed before shorter ones.
    \end{itemize}
\end{itemize}

\subsection*{3. Processing the Events}

\begin{itemize}
    \item **Heap Initialization:**
    \begin{itemize}
        \item A max heap is initialized with a ground level height of 0.
        \item A dictionary \texttt{active\_heights} tracks the count of active building heights.
    \end{itemize}
    
    \item **Iterating Through Events:**
    \begin{enumerate}
        \item **Start Event:**
        \begin{itemize}
            \item Add the building's height to the heap.
            \item Increment the count of the height in \texttt{active\_heights}.
        \end{itemize}
        
        \item **End Event:**
        \begin{itemize}
            \item Decrement the count of the building's height in \texttt{active\_heights}.
            \item If the count reaches zero, remove the height from the dictionary.
        \end{itemize}
        
        \item **Determine Current Max Height:**
        \begin{itemize}
            \item Remove heights from the heap that are no longer active.
            \item The current max height is the top of the heap.
        \end{itemize}
        
        \item **Update Skyline:**
        \begin{itemize}
            \item If the current max height differs from the previous max height, add a new key point \([x, current\_max\_height]\).
        \end{itemize}
    \end{enumerate}
\end{itemize}

\subsection*{4. Finalizing the Skyline}

\begin{itemize}
    \item After processing all events, the \texttt{result} list contains the key points defining the skyline.
\end{itemize}

\subsection*{5. Example Walkthrough}

Consider the first example:
\begin{verbatim}
Input: buildings = [[2,9,10], [3,7,15], [5,12,12], [15,20,10], [19,24,8]]
Output: [[2,10], [3,15], [7,12], [12,0], [15,10], [20,8], [24,0]]
\end{verbatim}

\begin{enumerate}
    \item **Event Transformation:**
    \begin{itemize}
        \item \((2, -10)\), \((9, 10)\)
        \item \((3, -15)\), \((7, 15)\)
        \item \((5, -12)\), \((12, 12)\)
        \item \((15, -10)\), \((20, 10)\)
        \item \((19, -8)\), \((24, 8)\)
    \end{itemize}
    
    \item **Sorting Events:**
    \begin{itemize}
        \item Sorted order: \((2, -10)\), \((3, -15)\), \((5, -12)\), \((7, 15)\), \((9, 10)\), \((12, 12)\), \((15, -10)\), \((19, -8)\), \((20, 10)\), \((24, 8)\)
    \end{itemize}
    
    \item **Processing Events:**
    \begin{itemize}
        \item At each event, update the heap and determine if the skyline height changes.
    \end{itemize}
    
    \item **Result Construction:**
    \begin{itemize}
        \item The resulting skyline key points are accumulated as \([[2,10], [3,15], [7,12], [12,0], [15,10], [20,8], [24,0]]\).
    \end{itemize}
\end{enumerate}

Thus, the function correctly constructs the skyline based on the buildings' positions and heights.

\section*{Why This Approach}

The Sweep Line Algorithm combined with a max heap offers an optimal solution with \(O(n \log n)\) time complexity and efficient handling of overlapping buildings. By processing events in sorted order and maintaining active building heights, the algorithm ensures that all critical points in the skyline are accurately identified without redundant computations.

\section*{Alternative Approaches}

\subsection*{1. Divide and Conquer}

Divide the set of buildings into smaller subsets, compute the skyline for each subset, and then merge the skylines.

\begin{lstlisting}[language=Python]
class Solution:
    def getSkyline(self, buildings: List[List[int]]) -> List[List[int]]:
        def merge(left, right):
            h1, h2 = 0, 0
            i, j = 0, 0
            merged = []
            while i < len(left) and j < len(right):
                if left[i][0] < right[j][0]:
                    x, h1 = left[i]
                    i += 1
                elif left[i][0] > right[j][0]:
                    x, h2 = right[j]
                    j += 1
                else:
                    x, h1 = left[i]
                    _, h2 = right[j]
                    i += 1
                    j += 1
                max_h = max(h1, h2)
                if not merged or merged[-1][1] != max_h:
                    merged.append([x, max_h])
            merged.extend(left[i:])
            merged.extend(right[j:])
            return merged
        
        def divide(buildings):
            if not buildings:
                return []
            if len(buildings) == 1:
                L, R, H = buildings[0]
                return [[L, H], [R, 0]]
            mid = len(buildings) // 2
            left = divide(buildings[:mid])
            right = divide(buildings[mid:])
            return merge(left, right)
        
        return divide(buildings)
\end{lstlisting}

\textbf{Complexities:}
\begin{itemize}
    \item \textbf{Time Complexity:} \(O(n \log n)\)
    \item \textbf{Space Complexity:} \(O(n)\)
\end{itemize}

\subsection*{2. Using Segment Trees}

Implement a segment tree to manage and query overlapping building heights dynamically.

\textbf{Note}: This approach is more complex and is generally used for advanced scenarios with multiple dynamic queries.

\section*{Similar Problems to This One}

Several problems involve skyline-like constructions, spatial data analysis, and efficient event processing, utilizing similar algorithmic strategies:

\begin{itemize}
    \item \textbf{Merge Intervals}: Merge overlapping intervals in a list.
    \item \textbf{Largest Rectangle in Histogram}: Find the largest rectangular area in a histogram.
    \item \textbf{Interval Partitioning}: Assign intervals to resources without overlap.
    \item \textbf{Line Segment Intersection}: Detect intersections among line segments.
    \item \textbf{Closest Pair of Points}: Find the closest pair of points in a set.
    \item \textbf{Convex Hull}: Compute the convex hull of a set of points.
    \item \textbf{Point Inside Polygon}: Determine if a point lies inside a given polygon.
    \item \textbf{Range Searching}: Efficiently query geometric data within a specified range.
\end{itemize}

These problems reinforce concepts of event-driven processing, spatial reasoning, and efficient algorithm design in various contexts.

\section*{Things to Keep in Mind and Tricks}

When tackling the \textbf{Skyline Problem}, consider the following tips and best practices to enhance efficiency and correctness:

\begin{itemize}
    \item \textbf{Understand Sweep Line Technique}: Grasp how the sweep line algorithm processes events in sorted order to handle dynamic changes efficiently.
    \index{Sweep Line Technique}
    
    \item \textbf{Leverage Priority Queues (Heaps)}: Use max heaps to keep track of active buildings' heights, enabling quick access to the current maximum height.
    \index{Priority Queues}
    
    \item \textbf{Handle Start and End Events Differently}: Differentiate between building start and end events to accurately manage active heights.
    \index{Start and End Events}
    
    \item \textbf{Optimize Event Sorting}: Sort events primarily by x-coordinate and secondarily by height to ensure correct processing order.
    \index{Event Sorting}
    
    \item \textbf{Manage Active Heights Efficiently}: Use data structures that allow efficient insertion, deletion, and retrieval of maximum elements.
    \index{Active Heights Management}
    
    \item \textbf{Avoid Redundant Key Points}: Only record key points when the skyline height changes to minimize the output list.
    \index{Avoiding Redundant Key Points}
    
    \item \textbf{Implement Helper Functions}: Create helper functions for tasks like distance calculation, event handling, and heap management to enhance modularity.
    \index{Helper Functions}
    
    \item \textbf{Code Readability}: Maintain clear and readable code through meaningful variable names and structured logic.
    \index{Code Readability}
    
    \item \textbf{Test Extensively}: Implement a wide range of test cases, including overlapping, non-overlapping, and edge-touching buildings, to ensure robustness.
    \index{Extensive Testing}
    
    \item \textbf{Handle Degenerate Cases}: Manage cases where buildings have zero height or identical coordinates gracefully.
    \index{Degenerate Cases}
    
    \item \textbf{Understand Geometric Relationships}: Grasp how buildings overlap and influence the skyline to simplify the algorithm.
    \index{Geometric Relationships}
    
    \item \textbf{Use Appropriate Data Structures}: Utilize appropriate data structures like heaps, lists, and dictionaries to manage and process data efficiently.
    \index{Appropriate Data Structures}
    
    \item \textbf{Optimize for Large Inputs}: Design the algorithm to handle large numbers of buildings without significant performance degradation.
    \index{Optimizing for Large Inputs}
    
    \item \textbf{Implement Iterative Solutions Carefully}: Ensure that loop conditions are correctly defined to prevent infinite loops or incorrect terminations.
    \index{Iterative Solutions}
    
    \item \textbf{Consistent Naming Conventions}: Use consistent and descriptive naming conventions for variables and functions to improve code clarity.
    \index{Naming Conventions}
\end{itemize}

\section*{Corner and Special Cases to Test When Writing the Code}

When implementing the solution for the \textbf{Skyline Problem}, it is crucial to consider and rigorously test various edge cases to ensure robustness and correctness:

\begin{itemize}
    \item \textbf{No Overlapping Buildings}: All buildings are separate and do not overlap.
    \index{No Overlapping Buildings}
    
    \item \textbf{Fully Overlapping Buildings}: Multiple buildings completely overlap each other.
    \index{Fully Overlapping Buildings}
    
    \item \textbf{Buildings Touching at Edges}: Buildings share common edges without overlapping.
    \index{Buildings Touching at Edges}
    
    \item \textbf{Buildings Touching at Corners}: Buildings share common corners without overlapping.
    \index{Buildings Touching at Corners}
    
    \item \textbf{Single Building}: Only one building is present.
    \index{Single Building}
    
    \item \textbf{Multiple Buildings with Same Start or End}: Multiple buildings start or end at the same x-coordinate.
    \index{Same Start or End}
    
    \item \textbf{Buildings with Zero Height}: Buildings that have zero height should not affect the skyline.
    \index{Buildings with Zero Height}
    
    \item \textbf{Large Number of Buildings}: Test with a large number of buildings to ensure performance and scalability.
    \index{Large Number of Buildings}
    
    \item \textbf{Buildings with Negative Coordinates}: Buildings positioned in negative coordinate space.
    \index{Negative Coordinates}
    
    \item \textbf{Boundary Values}: Buildings at the minimum and maximum limits of the coordinate range.
    \index{Boundary Values}
    
    \item \textbf{Buildings with Identical Coordinates}: Multiple buildings with the same coordinates.
    \index{Identical Coordinates}
    
    \item \textbf{Sequential Buildings}: Buildings placed sequentially without gaps.
    \index{Sequential Buildings}
    
    \item \textbf{Overlapping and Non-Overlapping Mixed}: A mix of overlapping and non-overlapping buildings.
    \index{Overlapping and Non-Overlapping Mixed}
    
    \item \textbf{Buildings with Very Large Heights}: Buildings with heights at the upper limit of the constraints.
    \index{Very Large Heights}
    
    \item \textbf{Empty Input}: No buildings are provided.
    \index{Empty Input}
\end{itemize}

\section*{Implementation Considerations}

When implementing the \texttt{getSkyline} function, keep in mind the following considerations to ensure robustness and efficiency:

\begin{itemize}
    \item \textbf{Data Type Selection}: Use appropriate data types that can handle large input values and avoid overflow or precision issues.
    \index{Data Type Selection}
    
    \item \textbf{Optimizing Event Sorting}: Efficiently sort events based on x-coordinates and heights to ensure correct processing order.
    \index{Optimizing Event Sorting}
    
    \item \textbf{Handling Large Inputs}: Design the algorithm to handle up to \(10^4\) buildings efficiently without significant performance degradation.
    \index{Handling Large Inputs}
    
    \item \textbf{Using Efficient Data Structures}: Utilize heaps, lists, and dictionaries effectively to manage and process events and active heights.
    \index{Efficient Data Structures}
    
    \item \textbf{Avoiding Redundant Calculations}: Ensure that distance and overlap calculations are performed only when necessary to optimize performance.
    \index{Avoiding Redundant Calculations}
    
    \item \textbf{Code Readability and Documentation}: Maintain clear and readable code through meaningful variable names and comprehensive comments to facilitate understanding and maintenance.
    \index{Code Readability}
    
    \item \textbf{Edge Case Handling}: Implement checks for edge cases to prevent incorrect results or runtime errors.
    \index{Edge Case Handling}
    
    \item \textbf{Implementing Helper Functions}: Create helper functions for tasks like distance calculation, event handling, and heap management to enhance modularity.
    \index{Helper Functions}
    
    \item \textbf{Consistent Naming Conventions}: Use consistent and descriptive naming conventions for variables and functions to improve code clarity.
    \index{Naming Conventions}
    
    \item \textbf{Memory Management}: Ensure that the algorithm manages memory efficiently, especially when dealing with large datasets.
    \index{Memory Management}
    
    \item \textbf{Implementing Iterative Solutions Carefully}: Ensure that loop conditions are correctly defined to prevent infinite loops or incorrect terminations.
    \index{Iterative Solutions}
    
    \item \textbf{Avoiding Floating-Point Precision Issues}: Since the problem deals with integers, floating-point precision is not a concern, simplifying the implementation.
    \index{Floating-Point Precision}
    
    \item \textbf{Testing and Validation}: Develop a comprehensive suite of test cases that cover all possible scenarios, including edge cases, to validate the correctness and efficiency of the implementation.
    \index{Testing and Validation}
    
    \item \textbf{Performance Considerations}: Optimize the loop conditions and operations to ensure that the function runs efficiently, especially for large input numbers.
    \index{Performance Considerations}
\end{itemize}

\section*{Conclusion}

The \textbf{Skyline Problem} is a quintessential example of applying advanced algorithmic techniques and geometric reasoning to solve complex spatial challenges. By leveraging the Sweep Line Algorithm and maintaining active building heights using a max heap, the solution efficiently constructs the skyline with optimal time and space complexities. Understanding and implementing such sophisticated algorithms not only enhances problem-solving skills but also provides a foundation for tackling a wide array of Computational Geometry problems in various domains, including computer graphics, urban planning simulations, and geographic information systems.

\printindex

% % filename: rectangle_overlap.tex

\problemsection{Rectangle Overlap}
\label{chap:Rectangle_Overlap}
\marginnote{\href{https://leetcode.com/problems/rectangle-overlap/}{[LeetCode Link]}\index{LeetCode}}
\marginnote{\href{https://www.geeksforgeeks.org/check-if-two-rectangles-overlap/}{[GeeksForGeeks Link]}\index{GeeksForGeeks}}
\marginnote{\href{https://www.interviewbit.com/problems/rectangle-overlap/}{[InterviewBit Link]}\index{InterviewBit}}
\marginnote{\href{https://app.codesignal.com/challenges/rectangle-overlap}{[CodeSignal Link]}\index{CodeSignal}}
\marginnote{\href{https://www.codewars.com/kata/rectangle-overlap/train/python}{[Codewars Link]}\index{Codewars}}

The \textbf{Rectangle Overlap} problem is a fundamental challenge in Computational Geometry that involves determining whether two axis-aligned rectangles overlap. This problem tests one's ability to understand geometric properties, implement conditional logic, and optimize for efficient computation. Mastery of this problem is essential for applications in computer graphics, collision detection, and spatial data analysis.

\section*{Problem Statement}

Given two axis-aligned rectangles in a 2D plane, determine if they overlap. Each rectangle is defined by its bottom-left and top-right coordinates.

A rectangle is represented as a list of four integers \([x1, y1, x2, y2]\), where \((x1, y1)\) are the coordinates of the bottom-left corner, and \((x2, y2)\) are the coordinates of the top-right corner.

\textbf{Function signature in Python:}
\begin{lstlisting}[language=Python]
def isRectangleOverlap(rec1: List[int], rec2: List[int]) -> bool:
\end{lstlisting}

\section*{Examples}

\textbf{Example 1:}

\begin{verbatim}
Input: rec1 = [0,0,2,2], rec2 = [1,1,3,3]
Output: True
Explanation: The rectangles overlap in the area defined by [1,1,2,2].
\end{verbatim}

\textbf{Example 2:}

\begin{verbatim}
Input: rec1 = [0,0,1,1], rec2 = [1,0,2,1]
Output: False
Explanation: The rectangles touch at the edge but do not overlap.
\end{verbatim}

\textbf{Example 3:}

\begin{verbatim}
Input: rec1 = [0,0,1,1], rec2 = [2,2,3,3]
Output: False
Explanation: The rectangles are completely separate.
\end{verbatim}

\textbf{Example 4:}

\begin{verbatim}
Input: rec1 = [0,0,5,5], rec2 = [3,3,7,7]
Output: True
Explanation: The rectangles overlap in the area defined by [3,3,5,5].
\end{verbatim}

\textbf{Example 5:}

\begin{verbatim}
Input: rec1 = [0,0,0,0], rec2 = [0,0,0,0]
Output: False
Explanation: Both rectangles are degenerate points.
\end{verbatim}

\textbf{Constraints:}

\begin{itemize}
    \item All coordinates are integers in the range \([-10^9, 10^9]\).
    \item For each rectangle, \(x1 < x2\) and \(y1 < y2\).
\end{itemize}

LeetCode link: \href{https://leetcode.com/problems/rectangle-overlap/}{Rectangle Overlap}\index{LeetCode}

\section*{Algorithmic Approach}

To determine whether two axis-aligned rectangles overlap, we can use the following logical conditions:

1. **Non-Overlap Conditions:**
   - One rectangle is to the left of the other.
   - One rectangle is above the other.

2. **Overlap Condition:**
   - If neither of the non-overlap conditions is true, the rectangles must overlap.

\subsection*{Steps:}

1. **Extract Coordinates:**
   - For both rectangles, extract the bottom-left and top-right coordinates.

2. **Check Non-Overlap Conditions:**
   - If the right side of the first rectangle is less than or equal to the left side of the second rectangle, they do not overlap.
   - If the left side of the first rectangle is greater than or equal to the right side of the second rectangle, they do not overlap.
   - If the top side of the first rectangle is less than or equal to the bottom side of the second rectangle, they do not overlap.
   - If the bottom side of the first rectangle is greater than or equal to the top side of the second rectangle, they do not overlap.

3. **Determine Overlap:**
   - If none of the non-overlap conditions are met, the rectangles overlap.

\marginnote{This approach provides an efficient \(O(1)\) time complexity solution by leveraging simple geometric comparisons.}

\section*{Complexities}

\begin{itemize}
    \item \textbf{Time Complexity:} \(O(1)\). The algorithm performs a constant number of comparisons regardless of input size.
    
    \item \textbf{Space Complexity:} \(O(1)\). Only a fixed amount of extra space is used for variables.
\end{itemize}

\section*{Python Implementation}

\marginnote{Implementing the overlap check using coordinate comparisons ensures an optimal and straightforward solution.}

Below is the complete Python code implementing the \texttt{isRectangleOverlap} function:

\begin{fullwidth}
\begin{lstlisting}[language=Python]
from typing import List

class Solution:
    def isRectangleOverlap(self, rec1: List[int], rec2: List[int]) -> bool:
        # Extract coordinates
        left1, bottom1, right1, top1 = rec1
        left2, bottom2, right2, top2 = rec2
        
        # Check non-overlapping conditions
        if right1 <= left2 or right2 <= left1:
            return False
        if top1 <= bottom2 or top2 <= bottom1:
            return False
        
        # If none of the above, rectangles overlap
        return True

# Example usage:
solution = Solution()
print(solution.isRectangleOverlap([0,0,2,2], [1,1,3,3]))  # Output: True
print(solution.isRectangleOverlap([0,0,1,1], [1,0,2,1]))  # Output: False
print(solution.isRectangleOverlap([0,0,1,1], [2,2,3,3]))  # Output: False
print(solution.isRectangleOverlap([0,0,5,5], [3,3,7,7]))  # Output: True
print(solution.isRectangleOverlap([0,0,0,0], [0,0,0,0]))  # Output: False
\end{lstlisting}
\end{fullwidth}

This implementation efficiently checks for overlap by comparing the coordinates of the two rectangles. If any of the non-overlapping conditions are met, it returns \texttt{False}; otherwise, it returns \texttt{True}.

\section*{Explanation}

The \texttt{isRectangleOverlap} function determines whether two axis-aligned rectangles overlap by comparing their respective coordinates. Here's a detailed breakdown of the implementation:

\subsection*{1. Extract Coordinates}

\begin{itemize}
    \item For each rectangle, extract the left (\(x1\)), bottom (\(y1\)), right (\(x2\)), and top (\(y2\)) coordinates.
    \item This simplifies the comparison process by providing clear variables representing each side of the rectangles.
\end{itemize}

\subsection*{2. Check Non-Overlap Conditions}

\begin{itemize}
    \item **Horizontal Separation:**
    \begin{itemize}
        \item If the right side of the first rectangle (\(right1\)) is less than or equal to the left side of the second rectangle (\(left2\)), there is no horizontal overlap.
        \item Similarly, if the right side of the second rectangle (\(right2\)) is less than or equal to the left side of the first rectangle (\(left1\)), there is no horizontal overlap.
    \end{itemize}
    
    \item **Vertical Separation:**
    \begin{itemize}
        \item If the top side of the first rectangle (\(top1\)) is less than or equal to the bottom side of the second rectangle (\(bottom2\)), there is no vertical overlap.
        \item Similarly, if the top side of the second rectangle (\(top2\)) is less than or equal to the bottom side of the first rectangle (\(bottom1\)), there is no vertical overlap.
    \end{itemize}
    
    \item If any of these non-overlapping conditions are true, the rectangles do not overlap, and the function returns \texttt{False}.
\end{itemize}

\subsection*{3. Determine Overlap}

\begin{itemize}
    \item If none of the non-overlapping conditions are met, it implies that the rectangles overlap both horizontally and vertically.
    \item The function returns \texttt{True} in this case.
\end{itemize}

\subsection*{4. Example Walkthrough}

Consider the first example:
\begin{verbatim}
Input: rec1 = [0,0,2,2], rec2 = [1,1,3,3]
Output: True
\end{verbatim}

\begin{enumerate}
    \item Extract coordinates:
    \begin{itemize}
        \item rec1: left1 = 0, bottom1 = 0, right1 = 2, top1 = 2
        \item rec2: left2 = 1, bottom2 = 1, right2 = 3, top2 = 3
    \end{itemize}
    
    \item Check non-overlap conditions:
    \begin{itemize}
        \item \(right1 = 2\) is not less than or equal to \(left2 = 1\)
        \item \(right2 = 3\) is not less than or equal to \(left1 = 0\)
        \item \(top1 = 2\) is not less than or equal to \(bottom2 = 1\)
        \item \(top2 = 3\) is not less than or equal to \(bottom1 = 0\)
    \end{itemize}
    
    \item Since none of the non-overlapping conditions are met, the rectangles overlap.
\end{enumerate}

Thus, the function correctly returns \texttt{True}.

\section*{Why This Approach}

This approach is chosen for its simplicity and efficiency. By leveraging direct coordinate comparisons, the algorithm achieves constant time complexity without the need for complex data structures or iterative processes. It effectively handles all possible scenarios of rectangle positioning, ensuring accurate detection of overlaps.

\section*{Alternative Approaches}

\subsection*{1. Separating Axis Theorem (SAT)}

The Separating Axis Theorem is a more generalized method for detecting overlaps between convex shapes. While it is not necessary for axis-aligned rectangles, understanding SAT can be beneficial for more complex geometric problems.

\begin{lstlisting}[language=Python]
def isRectangleOverlap(rec1: List[int], rec2: List[int]) -> bool:
    # Using SAT for axis-aligned rectangles
    return not (rec1[2] <= rec2[0] or rec1[0] >= rec2[2] or
                rec1[3] <= rec2[1] or rec1[1] >= rec2[3])
\end{lstlisting}

\textbf{Note}: This implementation is functionally identical to the primary approach but leverages a more generalized geometric theorem.

\subsection*{2. Area-Based Approach}

Calculate the overlapping area between the two rectangles. If the overlapping area is positive, the rectangles overlap.

\begin{lstlisting}[language=Python]
def isRectangleOverlap(rec1: List[int], rec2: List[int]) -> bool:
    # Calculate overlap in x and y dimensions
    x_overlap = min(rec1[2], rec2[2]) - max(rec1[0], rec2[0])
    y_overlap = min(rec1[3], rec2[3]) - max(rec1[1], rec2[1])
    
    # Overlap exists if both overlaps are positive
    return x_overlap > 0 and y_overlap > 0
\end{lstlisting}

\textbf{Complexities:}
\begin{itemize}
    \item \textbf{Time Complexity:} \(O(1)\)
    \item \textbf{Space Complexity:} \(O(1)\)
\end{itemize}

\subsection*{3. Using Rectangles Intersection Function}

Utilize built-in or library functions that handle geometric intersections.

\begin{lstlisting}[language=Python]
from shapely.geometry import box

def isRectangleOverlap(rec1: List[int], rec2: List[int]) -> bool:
    rectangle1 = box(rec1[0], rec1[1], rec1[2], rec1[3])
    rectangle2 = box(rec2[0], rec2[1], rec2[2], rec2[3])
    return rectangle1.intersects(rectangle2) and not rectangle1.touches(rectangle2)
\end{lstlisting}

\textbf{Note}: This approach requires the \texttt{shapely} library and is more suitable for complex geometric operations.

\section*{Similar Problems to This One}

Several problems revolve around geometric overlap, intersection detection, and spatial reasoning, utilizing similar algorithmic strategies:

\begin{itemize}
    \item \textbf{Interval Overlap}: Determine if two intervals on a line overlap.
    \item \textbf{Circle Overlap}: Determine if two circles overlap based on their radii and centers.
    \item \textbf{Polygon Overlap}: Determine if two polygons overlap using algorithms like SAT.
    \item \textbf{Closest Pair of Points}: Find the closest pair of points in a set.
    \item \textbf{Convex Hull}: Compute the convex hull of a set of points.
    \item \textbf{Intersection of Lines}: Find the intersection point of two lines.
    \item \textbf{Point Inside Polygon}: Determine if a point lies inside a given polygon.
\end{itemize}

These problems reinforce the concepts of spatial reasoning, geometric property analysis, and efficient algorithm design in various contexts.

\section*{Things to Keep in Mind and Tricks}

When working with the \textbf{Rectangle Overlap} problem, consider the following tips and best practices to enhance efficiency and correctness:

\begin{itemize}
    \item \textbf{Understand Geometric Relationships}: Grasp the positional relationships between rectangles to simplify overlap detection.
    \index{Geometric Relationships}
    
    \item \textbf{Leverage Coordinate Comparisons}: Use direct comparisons of rectangle coordinates to determine spatial relationships.
    \index{Coordinate Comparisons}
    
    \item \textbf{Handle Edge Cases}: Consider cases where rectangles touch at edges or corners without overlapping.
    \index{Edge Cases}
    
    \item \textbf{Optimize for Efficiency}: Aim for a constant time \(O(1)\) solution by avoiding unnecessary computations or iterations.
    \index{Efficiency Optimization}
    
    \item \textbf{Avoid Floating-Point Precision Issues}: Since all coordinates are integers, floating-point precision is not a concern, simplifying the implementation.
    \index{Floating-Point Precision}
    
    \item \textbf{Use Helper Functions}: Create helper functions to encapsulate repetitive tasks, such as extracting coordinates or checking specific conditions.
    \index{Helper Functions}
    
    \item \textbf{Code Readability}: Maintain clear and readable code through meaningful variable names and structured logic.
    \index{Code Readability}
    
    \item \textbf{Test Extensively}: Implement a wide range of test cases, including overlapping, non-overlapping, and edge-touching rectangles, to ensure robustness.
    \index{Extensive Testing}
    
    \item \textbf{Understand Axis-Aligned Constraints}: Recognize that axis-aligned rectangles simplify overlap detection compared to rotated rectangles.
    \index{Axis-Aligned Constraints}
    
    \item \textbf{Simplify Logical Conditions}: Combine multiple conditions logically to streamline the overlap detection process.
    \index{Logical Conditions}
\end{itemize}

\section*{Corner and Special Cases to Test When Writing the Code}

When implementing the solution for the \textbf{Rectangle Overlap} problem, it is crucial to consider and rigorously test various edge cases to ensure robustness and correctness:

\begin{itemize}
    \item \textbf{No Overlap}: Rectangles are completely separate.
    \index{No Overlap}
    
    \item \textbf{Partial Overlap}: Rectangles overlap in one or more regions.
    \index{Partial Overlap}
    
    \item \textbf{Edge Touching}: Rectangles touch exactly at one edge without overlapping.
    \index{Edge Touching}
    
    \item \textbf{Corner Touching}: Rectangles touch exactly at one corner without overlapping.
    \index{Corner Touching}
    
    \item \textbf{One Rectangle Inside Another}: One rectangle is entirely within the other.
    \index{Rectangle Inside}
    
    \item \textbf{Identical Rectangles}: Both rectangles have the same coordinates.
    \index{Identical Rectangles}
    
    \item \textbf{Degenerate Rectangles}: Rectangles with zero area (e.g., \(x1 = x2\) or \(y1 = y2\)).
    \index{Degenerate Rectangles}
    
    \item \textbf{Large Coordinates}: Rectangles with very large coordinate values to test performance and integer handling.
    \index{Large Coordinates}
    
    \item \textbf{Negative Coordinates}: Rectangles positioned in negative coordinate space.
    \index{Negative Coordinates}
    
    \item \textbf{Mixed Overlapping Scenarios}: Combinations of the above cases to ensure comprehensive coverage.
    \index{Mixed Overlapping Scenarios}
    
    \item \textbf{Minimum and Maximum Bounds}: Rectangles at the minimum and maximum limits of the coordinate range.
    \index{Minimum and Maximum Bounds}
\end{itemize}

\section*{Implementation Considerations}

When implementing the \texttt{isRectangleOverlap} function, keep in mind the following considerations to ensure robustness and efficiency:

\begin{itemize}
    \item \textbf{Data Type Selection}: Use appropriate data types that can handle the range of input values without overflow or underflow.
    \index{Data Type Selection}
    
    \item \textbf{Optimizing Comparisons}: Structure logical conditions to short-circuit evaluations as soon as a non-overlapping condition is met.
    \index{Optimizing Comparisons}
    
    \item \textbf{Language-Specific Constraints}: Be aware of how the programming language handles integer division and comparisons.
    \index{Language-Specific Constraints}
    
    \item \textbf{Avoiding Redundant Calculations}: Ensure that each comparison contributes towards determining overlap without unnecessary repetitions.
    \index{Avoiding Redundant Calculations}
    
    \item \textbf{Code Readability and Documentation}: Maintain clear and readable code through meaningful variable names and comprehensive comments to facilitate understanding and maintenance.
    \index{Code Readability}
    
    \item \textbf{Edge Case Handling}: Implement checks for edge cases to prevent incorrect results or runtime errors.
    \index{Edge Case Handling}
    
    \item \textbf{Testing and Validation}: Develop a comprehensive suite of test cases that cover all possible scenarios, including edge cases, to validate the correctness and efficiency of the implementation.
    \index{Testing and Validation}
    
    \item \textbf{Scalability}: Design the algorithm to scale efficiently with increasing input sizes, maintaining performance and resource utilization.
    \index{Scalability}
    
    \item \textbf{Using Helper Functions}: Consider creating helper functions for repetitive tasks, such as extracting and comparing coordinates, to enhance modularity and reusability.
    \index{Helper Functions}
    
    \item \textbf{Consistent Naming Conventions}: Use consistent and descriptive naming conventions for variables to improve code clarity.
    \index{Naming Conventions}
    
    \item \textbf{Handling Floating-Point Coordinates}: Although the problem specifies integer coordinates, ensure that the implementation can handle floating-point numbers if needed in extended scenarios.
    \index{Floating-Point Coordinates}
    
    \item \textbf{Avoiding Floating-Point Precision Issues}: Since all coordinates are integers, floating-point precision is not a concern, simplifying the implementation.
    \index{Floating-Point Precision}
    
    \item \textbf{Implementing Unit Tests}: Develop unit tests for each logical condition to ensure that all scenarios are correctly handled.
    \index{Unit Tests}
    
    \item \textbf{Error Handling}: Incorporate error handling to manage invalid inputs gracefully.
    \index{Error Handling}
\end{itemize}

\section*{Conclusion}

The \textbf{Rectangle Overlap} problem exemplifies the application of fundamental geometric principles and conditional logic to solve spatial challenges efficiently. By leveraging simple coordinate comparisons, the algorithm achieves optimal time and space complexities, making it highly suitable for real-time applications such as collision detection in gaming, layout planning in graphics, and spatial data analysis. Understanding and implementing such techniques not only enhances problem-solving skills but also provides a foundation for tackling more complex Computational Geometry problems involving varied geometric shapes and interactions.

\printindex

% \input{sections/rectangle_overlap}
% \input{sections/rectangle_area}
% \input{sections/k_closest_points_to_origin}
% \input{sections/the_skyline_problem}
% % filename: rectangle_area.tex

\problemsection{Rectangle Area}
\label{chap:Rectangle_Area}
\marginnote{\href{https://leetcode.com/problems/rectangle-area/}{[LeetCode Link]}\index{LeetCode}}
\marginnote{\href{https://www.geeksforgeeks.org/find-area-two-overlapping-rectangles/}{[GeeksForGeeks Link]}\index{GeeksForGeeks}}
\marginnote{\href{https://www.interviewbit.com/problems/rectangle-area/}{[InterviewBit Link]}\index{InterviewBit}}
\marginnote{\href{https://app.codesignal.com/challenges/rectangle-area}{[CodeSignal Link]}\index{CodeSignal}}
\marginnote{\href{https://www.codewars.com/kata/rectangle-area/train/python}{[Codewars Link]}\index{Codewars}}

The \textbf{Rectangle Area} problem is a classic Computational Geometry challenge that involves calculating the total area covered by two axis-aligned rectangles in a 2D plane. This problem tests one's ability to perform geometric calculations, handle overlapping scenarios, and implement efficient algorithms. Mastery of this problem is essential for applications in computer graphics, spatial analysis, and computational modeling.

\section*{Problem Statement}

Given two axis-aligned rectangles in a 2D plane, compute the total area covered by the two rectangles. The area covered by the overlapping region should be counted only once.

Each rectangle is represented as a list of four integers \([x1, y1, x2, y2]\), where \((x1, y1)\) are the coordinates of the bottom-left corner, and \((x2, y2)\) are the coordinates of the top-right corner.

\textbf{Function signature in Python:}
\begin{lstlisting}[language=Python]
def computeArea(A: List[int], B: List[int]) -> int:
\end{lstlisting}

\section*{Examples}

\textbf{Example 1:}

\begin{verbatim}
Input: A = [-3,0,3,4], B = [0,-1,9,2]
Output: 45
Explanation:
Area of A = (3 - (-3)) * (4 - 0) = 6 * 4 = 24
Area of B = (9 - 0) * (2 - (-1)) = 9 * 3 = 27
Overlapping Area = (3 - 0) * (2 - 0) = 3 * 2 = 6
Total Area = 24 + 27 - 6 = 45
\end{verbatim}

\textbf{Example 2:}

\begin{verbatim}
Input: A = [0,0,0,0], B = [0,0,0,0]
Output: 0
Explanation:
Both rectangles are degenerate points with zero area.
\end{verbatim}

\textbf{Example 3:}

\begin{verbatim}
Input: A = [0,0,2,2], B = [1,1,3,3]
Output: 7
Explanation:
Area of A = 4
Area of B = 4
Overlapping Area = 1
Total Area = 4 + 4 - 1 = 7
\end{verbatim}

\textbf{Example 4:}

\begin{verbatim}
Input: A = [0,0,1,1], B = [1,0,2,1]
Output: 2
Explanation:
Rectangles touch at the edge but do not overlap.
Area of A = 1
Area of B = 1
Overlapping Area = 0
Total Area = 1 + 1 = 2
\end{verbatim}

\textbf{Constraints:}

\begin{itemize}
    \item All coordinates are integers in the range \([-10^9, 10^9]\).
    \item For each rectangle, \(x1 < x2\) and \(y1 < y2\).
\end{itemize}

LeetCode link: \href{https://leetcode.com/problems/rectangle-area/}{Rectangle Area}\index{LeetCode}

\section*{Algorithmic Approach}

To compute the total area covered by two axis-aligned rectangles, we can follow these steps:

1. **Calculate Individual Areas:**
   - Compute the area of the first rectangle.
   - Compute the area of the second rectangle.

2. **Determine Overlapping Area:**
   - Calculate the coordinates of the overlapping rectangle, if any.
   - If the rectangles overlap, compute the area of the overlapping region.

3. **Compute Total Area:**
   - Sum the individual areas and subtract the overlapping area to avoid double-counting.

\marginnote{This approach ensures accurate area calculation by handling overlapping regions appropriately.}

\section*{Complexities}

\begin{itemize}
    \item \textbf{Time Complexity:} \(O(1)\). The algorithm performs a constant number of calculations.
    
    \item \textbf{Space Complexity:} \(O(1)\). Only a fixed amount of extra space is used for variables.
\end{itemize}

\section*{Python Implementation}

\marginnote{Implementing the area calculation with overlap consideration ensures an accurate and efficient solution.}

Below is the complete Python code implementing the \texttt{computeArea} function:

\begin{fullwidth}
\begin{lstlisting}[language=Python]
from typing import List

class Solution:
    def computeArea(self, A: List[int], B: List[int]) -> int:
        # Calculate area of rectangle A
        areaA = (A[2] - A[0]) * (A[3] - A[1])
        
        # Calculate area of rectangle B
        areaB = (B[2] - B[0]) * (B[3] - B[1])
        
        # Determine overlap coordinates
        overlap_x1 = max(A[0], B[0])
        overlap_y1 = max(A[1], B[1])
        overlap_x2 = min(A[2], B[2])
        overlap_y2 = min(A[3], B[3])
        
        # Calculate overlapping area
        overlap_width = overlap_x2 - overlap_x1
        overlap_height = overlap_y2 - overlap_y1
        overlap_area = 0
        if overlap_width > 0 and overlap_height > 0:
            overlap_area = overlap_width * overlap_height
        
        # Total area is sum of individual areas minus overlapping area
        total_area = areaA + areaB - overlap_area
        return total_area

# Example usage:
solution = Solution()
print(solution.computeArea([-3,0,3,4], [0,-1,9,2]))  # Output: 45
print(solution.computeArea([0,0,0,0], [0,0,0,0]))    # Output: 0
print(solution.computeArea([0,0,2,2], [1,1,3,3]))    # Output: 7
print(solution.computeArea([0,0,1,1], [1,0,2,1]))    # Output: 2
\end{lstlisting}
\end{fullwidth}

This implementation accurately computes the total area covered by two rectangles by accounting for any overlapping regions. It ensures that the overlapping area is not double-counted.

\section*{Explanation}

The \texttt{computeArea} function calculates the combined area of two axis-aligned rectangles by following these steps:

\subsection*{1. Calculate Individual Areas}

\begin{itemize}
    \item **Rectangle A:**
    \begin{itemize}
        \item Width: \(A[2] - A[0]\)
        \item Height: \(A[3] - A[1]\)
        \item Area: Width \(\times\) Height
    \end{itemize}
    
    \item **Rectangle B:**
    \begin{itemize}
        \item Width: \(B[2] - B[0]\)
        \item Height: \(B[3] - B[1]\)
        \item Area: Width \(\times\) Height
    \end{itemize}
\end{itemize}

\subsection*{2. Determine Overlapping Area}

\begin{itemize}
    \item **Overlap Coordinates:**
    \begin{itemize}
        \item Left (x-coordinate): \(\text{max}(A[0], B[0])\)
        \item Bottom (y-coordinate): \(\text{max}(A[1], B[1])\)
        \item Right (x-coordinate): \(\text{min}(A[2], B[2])\)
        \item Top (y-coordinate): \(\text{min}(A[3], B[3])\)
    \end{itemize}
    
    \item **Overlap Dimensions:**
    \begin{itemize}
        \item Width: \(\text{overlap\_x2} - \text{overlap\_x1}\)
        \item Height: \(\text{overlap\_y2} - \text{overlap\_y1}\)
    \end{itemize}
    
    \item **Overlap Area:**
    \begin{itemize}
        \item If both width and height are positive, the rectangles overlap, and the overlapping area is their product.
        \item Otherwise, there is no overlap, and the overlapping area is zero.
    \end{itemize}
\end{itemize}

\subsection*{3. Compute Total Area}

\begin{itemize}
    \item Total Area = Area of Rectangle A + Area of Rectangle B - Overlapping Area
\end{itemize}

\subsection*{4. Example Walkthrough}

Consider the first example:
\begin{verbatim}
Input: A = [-3,0,3,4], B = [0,-1,9,2]
Output: 45
\end{verbatim}

\begin{enumerate}
    \item **Calculate Areas:**
    \begin{itemize}
        \item Area of A = (3 - (-3)) * (4 - 0) = 6 * 4 = 24
        \item Area of B = (9 - 0) * (2 - (-1)) = 9 * 3 = 27
    \end{itemize}
    
    \item **Determine Overlap:**
    \begin{itemize}
        \item overlap\_x1 = max(-3, 0) = 0
        \item overlap\_y1 = max(0, -1) = 0
        \item overlap\_x2 = min(3, 9) = 3
        \item overlap\_y2 = min(4, 2) = 2
        \item overlap\_width = 3 - 0 = 3
        \item overlap\_height = 2 - 0 = 2
        \item overlap\_area = 3 * 2 = 6
    \end{itemize}
    
    \item **Compute Total Area:**
    \begin{itemize}
        \item Total Area = 24 + 27 - 6 = 45
    \end{itemize}
\end{enumerate}

Thus, the function correctly returns \texttt{45}.

\section*{Why This Approach}

This approach is chosen for its straightforwardness and optimal efficiency. By directly calculating the individual areas and intelligently handling the overlapping region, the algorithm ensures accurate results without unnecessary computations. Its constant time complexity makes it highly efficient, even for large coordinate values.

\section*{Alternative Approaches}

\subsection*{1. Using Intersection Dimensions}

Instead of separately calculating areas, directly compute the dimensions of the overlapping region and subtract it from the sum of individual areas.

\begin{lstlisting}[language=Python]
def computeArea(A: List[int], B: List[int]) -> int:
    # Sum of individual areas
    area = (A[2] - A[0]) * (A[3] - A[1]) + (B[2] - B[0]) * (B[3] - B[1])
    
    # Overlapping area
    overlap_width = min(A[2], B[2]) - max(A[0], B[0])
    overlap_height = min(A[3], B[3]) - max(A[1], B[1])
    
    if overlap_width > 0 and overlap_height > 0:
        area -= overlap_width * overlap_height
    
    return area
\end{lstlisting}

\subsection*{2. Using Geometry Libraries}

Leverage computational geometry libraries to handle area calculations and overlapping detections.

\begin{lstlisting}[language=Python]
from shapely.geometry import box

def computeArea(A: List[int], B: List[int]) -> int:
    rect1 = box(A[0], A[1], A[2], A[3])
    rect2 = box(B[0], B[1], B[2], B[3])
    intersection = rect1.intersection(rect2)
    return int(rect1.area + rect2.area - intersection.area)
\end{lstlisting}

\textbf{Note}: This approach requires the \texttt{shapely} library and is more suitable for complex geometric operations.

\section*{Similar Problems to This One}

Several problems involve calculating areas, handling geometric overlaps, and spatial reasoning, utilizing similar algorithmic strategies:

\begin{itemize}
    \item \textbf{Rectangle Overlap}: Determine if two rectangles overlap.
    \item \textbf{Circle Area Overlap}: Calculate the overlapping area between two circles.
    \item \textbf{Polygon Area}: Compute the area of a given polygon.
    \item \textbf{Union of Rectangles}: Calculate the total area covered by multiple rectangles, accounting for overlaps.
    \item \textbf{Intersection of Lines}: Find the intersection point of two lines.
    \item \textbf{Closest Pair of Points}: Find the closest pair of points in a set.
    \item \textbf{Convex Hull}: Compute the convex hull of a set of points.
    \item \textbf{Point Inside Polygon}: Determine if a point lies inside a given polygon.
\end{itemize}

These problems reinforce concepts of geometric calculations, area computations, and efficient algorithm design in various contexts.

\section*{Things to Keep in Mind and Tricks}

When tackling the \textbf{Rectangle Area} problem, consider the following tips and best practices to enhance efficiency and correctness:

\begin{itemize}
    \item \textbf{Understand Geometric Relationships}: Grasp the positional relationships between rectangles to simplify area calculations.
    \index{Geometric Relationships}
    
    \item \textbf{Leverage Coordinate Comparisons}: Use direct comparisons of rectangle coordinates to determine overlapping regions.
    \index{Coordinate Comparisons}
    
    \item \textbf{Handle Overlapping Scenarios}: Accurately calculate the overlapping area to avoid double-counting.
    \index{Overlapping Scenarios}
    
    \item \textbf{Optimize for Efficiency}: Aim for a constant time \(O(1)\) solution by avoiding unnecessary computations or iterations.
    \index{Efficiency Optimization}
    
    \item \textbf{Avoid Floating-Point Precision Issues}: Since all coordinates are integers, floating-point precision is not a concern, simplifying the implementation.
    \index{Floating-Point Precision}
    
    \item \textbf{Use Helper Functions}: Create helper functions to encapsulate repetitive tasks, such as calculating overlap dimensions or areas.
    \index{Helper Functions}
    
    \item \textbf{Code Readability}: Maintain clear and readable code through meaningful variable names and structured logic.
    \index{Code Readability}
    
    \item \textbf{Test Extensively}: Implement a wide range of test cases, including overlapping, non-overlapping, and edge-touching rectangles, to ensure robustness.
    \index{Extensive Testing}
    
    \item \textbf{Understand Axis-Aligned Constraints}: Recognize that axis-aligned rectangles simplify area calculations compared to rotated rectangles.
    \index{Axis-Aligned Constraints}
    
    \item \textbf{Simplify Logical Conditions}: Combine multiple conditions logically to streamline the area calculation process.
    \index{Logical Conditions}
    
    \item \textbf{Use Absolute Values}: When calculating differences, ensure that the dimensions are positive by using absolute values or proper ordering.
    \index{Absolute Values}
    
    \item \textbf{Consider Edge Cases}: Handle cases where rectangles have zero area or touch at edges/corners without overlapping.
    \index{Edge Cases}
\end{itemize}

\section*{Corner and Special Cases to Test When Writing the Code}

When implementing the solution for the \textbf{Rectangle Area} problem, it is crucial to consider and rigorously test various edge cases to ensure robustness and correctness:

\begin{itemize}
    \item \textbf{No Overlap}: Rectangles are completely separate.
    \index{No Overlap}
    
    \item \textbf{Partial Overlap}: Rectangles overlap in one or more regions.
    \index{Partial Overlap}
    
    \item \textbf{Edge Touching}: Rectangles touch exactly at one edge without overlapping.
    \index{Edge Touching}
    
    \item \textbf{Corner Touching}: Rectangles touch exactly at one corner without overlapping.
    \index{Corner Touching}
    
    \item \textbf{One Rectangle Inside Another}: One rectangle is entirely within the other.
    \index{Rectangle Inside}
    
    \item \textbf{Identical Rectangles}: Both rectangles have the same coordinates.
    \index{Identical Rectangles}
    
    \item \textbf{Degenerate Rectangles}: Rectangles with zero area (e.g., \(x1 = x2\) or \(y1 = y2\)).
    \index{Degenerate Rectangles}
    
    \item \textbf{Large Coordinates}: Rectangles with very large coordinate values to test performance and integer handling.
    \index{Large Coordinates}
    
    \item \textbf{Negative Coordinates}: Rectangles positioned in negative coordinate space.
    \index{Negative Coordinates}
    
    \item \textbf{Mixed Overlapping Scenarios}: Combinations of the above cases to ensure comprehensive coverage.
    \index{Mixed Overlapping Scenarios}
    
    \item \textbf{Minimum and Maximum Bounds}: Rectangles at the minimum and maximum limits of the coordinate range.
    \index{Minimum and Maximum Bounds}
    
    \item \textbf{Sequential Rectangles}: Multiple rectangles placed sequentially without overlapping.
    \index{Sequential Rectangles}
    
    \item \textbf{Multiple Overlaps}: Scenarios where more than two rectangles overlap in different regions.
    \index{Multiple Overlaps}
\end{itemize}

\section*{Implementation Considerations}

When implementing the \texttt{computeArea} function, keep in mind the following considerations to ensure robustness and efficiency:

\begin{itemize}
    \item \textbf{Data Type Selection}: Use appropriate data types that can handle large input values without overflow or underflow.
    \index{Data Type Selection}
    
    \item \textbf{Optimizing Comparisons}: Structure logical conditions to efficiently determine overlap dimensions.
    \index{Optimizing Comparisons}
    
    \item \textbf{Handling Large Inputs}: Design the algorithm to efficiently handle large input sizes without significant performance degradation.
    \index{Handling Large Inputs}
    
    \item \textbf{Language-Specific Constraints}: Be aware of how the programming language handles large integers and arithmetic operations.
    \index{Language-Specific Constraints}
    
    \item \textbf{Avoiding Redundant Calculations}: Ensure that each calculation contributes towards determining the final area without unnecessary repetitions.
    \index{Avoiding Redundant Calculations}
    
    \item \textbf{Code Readability and Documentation}: Maintain clear and readable code through meaningful variable names and comprehensive comments to facilitate understanding and maintenance.
    \index{Code Readability}
    
    \item \textbf{Edge Case Handling}: Implement checks for edge cases to prevent incorrect results or runtime errors.
    \index{Edge Case Handling}
    
    \item \textbf{Testing and Validation}: Develop a comprehensive suite of test cases that cover all possible scenarios, including edge cases, to validate the correctness and efficiency of the implementation.
    \index{Testing and Validation}
    
    \item \textbf{Scalability}: Design the algorithm to scale efficiently with increasing input sizes, maintaining performance and resource utilization.
    \index{Scalability}
    
    \item \textbf{Using Helper Functions}: Consider creating helper functions for repetitive tasks, such as calculating overlap dimensions, to enhance modularity and reusability.
    \index{Helper Functions}
    
    \item \textbf{Consistent Naming Conventions}: Use consistent and descriptive naming conventions for variables to improve code clarity.
    \index{Naming Conventions}
    
    \item \textbf{Implementing Unit Tests}: Develop unit tests for each logical condition to ensure that all scenarios are correctly handled.
    \index{Unit Tests}
    
    \item \textbf{Error Handling}: Incorporate error handling to manage invalid inputs gracefully.
    \index{Error Handling}
\end{itemize}

\section*{Conclusion}

The \textbf{Rectangle Area} problem showcases the application of fundamental geometric principles and efficient algorithm design to compute spatial properties accurately. By systematically calculating individual areas and intelligently handling overlapping regions, the algorithm ensures precise results without redundant computations. Understanding and implementing such techniques not only enhances problem-solving skills but also provides a foundation for tackling more complex Computational Geometry challenges involving multiple geometric entities and intricate spatial relationships.

\printindex

% \input{sections/rectangle_overlap}
% \input{sections/rectangle_area}
% \input{sections/k_closest_points_to_origin}
% \input{sections/the_skyline_problem}
% % filename: k_closest_points_to_origin.tex

\problemsection{K Closest Points to Origin}
\label{chap:K_Closest_Points_to_Origin}
\marginnote{\href{https://leetcode.com/problems/k-closest-points-to-origin/}{[LeetCode Link]}\index{LeetCode}}
\marginnote{\href{https://www.geeksforgeeks.org/find-k-closest-points-origin/}{[GeeksForGeeks Link]}\index{GeeksForGeeks}}
\marginnote{\href{https://www.interviewbit.com/problems/k-closest-points/}{[InterviewBit Link]}\index{InterviewBit}}
\marginnote{\href{https://app.codesignal.com/challenges/k-closest-points-to-origin}{[CodeSignal Link]}\index{CodeSignal}}
\marginnote{\href{https://www.codewars.com/kata/k-closest-points-to-origin/train/python}{[Codewars Link]}\index{Codewars}}

The \textbf{K Closest Points to Origin} problem is a popular algorithmic challenge in Computational Geometry that involves identifying the \(k\) points closest to the origin in a 2D plane. This problem tests one's ability to apply efficient sorting and selection algorithms, understand distance computations, and optimize for performance. Mastery of this problem is essential for applications in spatial data analysis, nearest neighbor searches, and clustering algorithms.

\section*{Problem Statement}

Given an array of points where each point is represented as \([x, y]\) in the 2D plane, and an integer \(k\), return the \(k\) closest points to the origin \((0, 0)\).

The distance between two points \((x_1, y_1)\) and \((x_2, y_2)\) is the Euclidean distance \(\sqrt{(x_1 - x_2)^2 + (y_1 - y_2)^2}\). The origin is \((0, 0)\).

\textbf{Function signature in Python:}
\begin{lstlisting}[language=Python]
def kClosest(points: List[List[int]], K: int) -> List[List[int]]:
\end{lstlisting}

\section*{Examples}

\textbf{Example 1:}

\begin{verbatim}
Input: points = [[1,3],[-2,2]], K = 1
Output: [[-2,2]]
Explanation: 
The distance between (1, 3) and the origin is sqrt(10).
The distance between (-2, 2) and the origin is sqrt(8).
Since sqrt(8) < sqrt(10), (-2, 2) is closer to the origin.
\end{verbatim}

\textbf{Example 2:}

\begin{verbatim}
Input: points = [[3,3],[5,-1],[-2,4]], K = 2
Output: [[3,3],[-2,4]]
Explanation: 
The distances are sqrt(18), sqrt(26), and sqrt(20) respectively.
The two closest points are [3,3] and [-2,4].
\end{verbatim}

\textbf{Example 3:}

\begin{verbatim}
Input: points = [[0,1],[1,0]], K = 2
Output: [[0,1],[1,0]]
Explanation: 
Both points are equally close to the origin.
\end{verbatim}

\textbf{Example 4:}

\begin{verbatim}
Input: points = [[1,0],[0,1]], K = 1
Output: [[1,0]]
Explanation: 
Both points are equally close; returning any one is acceptable.
\end{verbatim}

\textbf{Constraints:}

\begin{itemize}
    \item \(1 \leq K \leq \text{points.length} \leq 10^4\)
    \item \(-10^4 < x_i, y_i < 10^4\)
\end{itemize}

LeetCode link: \href{https://leetcode.com/problems/k-closest-points-to-origin/}{K Closest Points to Origin}\index{LeetCode}

\section*{Algorithmic Approach}

To identify the \(k\) closest points to the origin, several algorithmic strategies can be employed. The most efficient methods aim to reduce the time complexity by avoiding the need to sort the entire list of points.

\subsection*{1. Sorting Based on Distance}

Calculate the Euclidean distance of each point from the origin and sort the points based on these distances. Select the first \(k\) points from the sorted list.

\begin{enumerate}
    \item Compute the distance for each point using the formula \(distance = x^2 + y^2\).
    \item Sort the points based on the computed distances.
    \item Return the first \(k\) points from the sorted list.
\end{enumerate}

\subsection*{2. Max Heap (Priority Queue)}

Use a max heap to maintain the \(k\) closest points. Iterate through each point, add it to the heap, and if the heap size exceeds \(k\), remove the farthest point.

\begin{enumerate}
    \item Initialize a max heap.
    \item For each point, compute its distance and add it to the heap.
    \item If the heap size exceeds \(k\), remove the point with the largest distance.
    \item After processing all points, the heap contains the \(k\) closest points.
\end{enumerate}

\subsection*{3. QuickSelect (Quick Sort Partitioning)}

Utilize the QuickSelect algorithm to find the \(k\) closest points without fully sorting the list.

\begin{enumerate}
    \item Choose a pivot point and partition the list based on distances relative to the pivot.
    \item Recursively apply QuickSelect to the partition containing the \(k\) closest points.
    \item Once the \(k\) closest points are identified, return them.
\end{enumerate}

\marginnote{QuickSelect offers an average time complexity of \(O(n)\), making it highly efficient for large datasets.}

\section*{Complexities}

\begin{itemize}
    \item \textbf{Sorting Based on Distance:}
    \begin{itemize}
        \item \textbf{Time Complexity:} \(O(n \log n)\)
        \item \textbf{Space Complexity:} \(O(n)\)
    \end{itemize}
    
    \item \textbf{Max Heap (Priority Queue):}
    \begin{itemize}
        \item \textbf{Time Complexity:} \(O(n \log k)\)
        \item \textbf{Space Complexity:} \(O(k)\)
    \end{itemize}
    
    \item \textbf{QuickSelect (Quick Sort Partitioning):}
    \begin{itemize}
        \item \textbf{Time Complexity:} Average case \(O(n)\), worst case \(O(n^2)\)
        \item \textbf{Space Complexity:} \(O(1)\) (in-place)
    \end{itemize}
\end{itemize}

\section*{Python Implementation}

\marginnote{Implementing QuickSelect provides an optimal average-case solution with linear time complexity.}

Below is the complete Python code implementing the \texttt{kClosest} function using the QuickSelect approach:

\begin{fullwidth}
\begin{lstlisting}[language=Python]
from typing import List
import random

class Solution:
    def kClosest(self, points: List[List[int]], K: int) -> List[List[int]]:
        def quickselect(left, right, K_smallest):
            if left == right:
                return
            
            # Select a random pivot_index
            pivot_index = random.randint(left, right)
            
            # Partition the array
            pivot_index = partition(left, right, pivot_index)
            
            # The pivot is in its final sorted position
            if K_smallest == pivot_index:
                return
            elif K_smallest < pivot_index:
                quickselect(left, pivot_index - 1, K_smallest)
            else:
                quickselect(pivot_index + 1, right, K_smallest)
        
        def partition(left, right, pivot_index):
            pivot_distance = distance(points[pivot_index])
            # Move pivot to end
            points[pivot_index], points[right] = points[right], points[pivot_index]
            store_index = left
            for i in range(left, right):
                if distance(points[i]) < pivot_distance:
                    points[store_index], points[i] = points[i], points[store_index]
                    store_index += 1
            # Move pivot to its final place
            points[right], points[store_index] = points[store_index], points[right]
            return store_index
        
        def distance(point):
            return point[0] ** 2 + point[1] ** 2
        
        n = len(points)
        quickselect(0, n - 1, K)
        return points[:K]

# Example usage:
solution = Solution()
print(solution.kClosest([[1,3],[-2,2]], 1))            # Output: [[-2,2]]
print(solution.kClosest([[3,3],[5,-1],[-2,4]], 2))     # Output: [[3,3],[-2,4]]
print(solution.kClosest([[0,1],[1,0]], 2))             # Output: [[0,1],[1,0]]
print(solution.kClosest([[1,0],[0,1]], 1))             # Output: [[1,0]] or [[0,1]]
\end{lstlisting}
\end{fullwidth}

This implementation uses the QuickSelect algorithm to efficiently find the \(k\) closest points to the origin without fully sorting the entire list. It ensures optimal performance even with large datasets.

\section*{Explanation}

The \texttt{kClosest} function identifies the \(k\) closest points to the origin using the QuickSelect algorithm. Here's a detailed breakdown of the implementation:

\subsection*{1. Distance Calculation}

\begin{itemize}
    \item The Euclidean distance is calculated as \(distance = x^2 + y^2\). Since we only need relative distances for comparison, the square root is omitted for efficiency.
\end{itemize}

\subsection*{2. QuickSelect Algorithm}

\begin{itemize}
    \item **Pivot Selection:**
    \begin{itemize}
        \item A random pivot is chosen to enhance the average-case performance.
    \end{itemize}
    
    \item **Partitioning:**
    \begin{itemize}
        \item The array is partitioned such that points with distances less than the pivot are moved to the left, and others to the right.
        \item The pivot is placed in its correct sorted position.
    \end{itemize}
    
    \item **Recursive Selection:**
    \begin{itemize}
        \item If the pivot's position matches \(K\), the selection is complete.
        \item Otherwise, recursively apply QuickSelect to the relevant partition.
    \end{itemize}
\end{itemize}

\subsection*{3. Final Selection}

\begin{itemize}
    \item After partitioning, the first \(K\) points in the list are the \(k\) closest points to the origin.
\end{itemize}

\subsection*{4. Example Walkthrough}

Consider the first example:
\begin{verbatim}
Input: points = [[1,3],[-2,2]], K = 1
Output: [[-2,2]]
\end{verbatim}

\begin{enumerate}
    \item **Calculate Distances:**
    \begin{itemize}
        \item [1,3] : \(1^2 + 3^2 = 10\)
        \item [-2,2] : \((-2)^2 + 2^2 = 8\)
    \end{itemize}
    
    \item **QuickSelect Process:**
    \begin{itemize}
        \item Choose a pivot, say [1,3] with distance 10.
        \item Compare and rearrange:
        \begin{itemize}
            \item [-2,2] has a smaller distance (8) and is moved to the left.
        \end{itemize}
        \item After partitioning, the list becomes [[-2,2], [1,3]].
        \item Since \(K = 1\), return the first point: [[-2,2]].
    \end{itemize}
\end{enumerate}

Thus, the function correctly identifies \([-2,2]\) as the closest point to the origin.

\section*{Why This Approach}

The QuickSelect algorithm is chosen for its average-case linear time complexity \(O(n)\), making it highly efficient for large datasets compared to sorting-based methods with \(O(n \log n)\) time complexity. By avoiding the need to sort the entire list, QuickSelect provides an optimal solution for finding the \(k\) closest points.

\section*{Alternative Approaches}

\subsection*{1. Sorting Based on Distance}

Sort all points based on their distances from the origin and select the first \(k\) points.

\begin{lstlisting}[language=Python]
class Solution:
    def kClosest(self, points: List[List[int]], K: int) -> List[List[int]]:
        points.sort(key=lambda P: P[0]**2 + P[1]**2)
        return points[:K]
\end{lstlisting}

\textbf{Complexities:}
\begin{itemize}
    \item \textbf{Time Complexity:} \(O(n \log n)\)
    \item \textbf{Space Complexity:} \(O(1)\)
\end{itemize}

\subsection*{2. Max Heap (Priority Queue)}

Use a max heap to maintain the \(k\) closest points.

\begin{lstlisting}[language=Python]
import heapq

class Solution:
    def kClosest(self, points: List[List[int]], K: int) -> List[List[int]]:
        heap = []
        for (x, y) in points:
            dist = -(x**2 + y**2)  # Max heap using negative distances
            heapq.heappush(heap, (dist, [x, y]))
            if len(heap) > K:
                heapq.heappop(heap)
        return [item[1] for item in heap]
\end{lstlisting}

\textbf{Complexities:}
\begin{itemize}
    \item \textbf{Time Complexity:} \(O(n \log k)\)
    \item \textbf{Space Complexity:} \(O(k)\)
\end{itemize}

\subsection*{3. Using Built-In Functions}

Leverage built-in functions for distance calculation and selection.

\begin{lstlisting}[language=Python]
import math

class Solution:
    def kClosest(self, points: List[List[int]], K: int) -> List[List[int]]:
        points.sort(key=lambda P: math.sqrt(P[0]**2 + P[1]**2))
        return points[:K]
\end{lstlisting}

\textbf{Note}: This method is similar to the sorting approach but uses the actual Euclidean distance.

\section*{Similar Problems to This One}

Several problems involve nearest neighbor searches, spatial data analysis, and efficient selection algorithms, utilizing similar algorithmic strategies:

\begin{itemize}
    \item \textbf{Closest Pair of Points}: Find the closest pair of points in a set.
    \item \textbf{Top K Frequent Elements}: Identify the most frequent elements in a dataset.
    \item \textbf{Kth Largest Element in an Array}: Find the \(k\)-th largest element in an unsorted array.
    \item \textbf{Sliding Window Maximum}: Find the maximum in each sliding window of size \(k\) over an array.
    \item \textbf{Merge K Sorted Lists}: Merge multiple sorted lists into a single sorted list.
    \item \textbf{Find Median from Data Stream}: Continuously find the median of a stream of numbers.
    \item \textbf{Top K Closest Stars}: Find the \(k\) closest stars to Earth based on their distances.
\end{itemize}

These problems reinforce concepts of efficient selection, heap usage, and distance computations in various contexts.

\section*{Things to Keep in Mind and Tricks}

When solving the \textbf{K Closest Points to Origin} problem, consider the following tips and best practices to enhance efficiency and correctness:

\begin{itemize}
    \item \textbf{Understand Distance Calculations}: Grasp the Euclidean distance formula and recognize that the square root can be omitted for comparison purposes.
    \index{Distance Calculations}
    
    \item \textbf{Leverage Efficient Algorithms}: Use QuickSelect or heap-based methods to optimize time complexity, especially for large datasets.
    \index{Efficient Algorithms}
    
    \item \textbf{Handle Ties Appropriately}: Decide how to handle points with identical distances when \(k\) is less than the number of such points.
    \index{Handling Ties}
    
    \item \textbf{Optimize Space Usage}: Choose algorithms that minimize additional space, such as in-place QuickSelect.
    \index{Space Optimization}
    
    \item \textbf{Use Appropriate Data Structures}: Utilize heaps, lists, and helper functions effectively to manage and process data.
    \index{Data Structures}
    
    \item \textbf{Implement Helper Functions}: Create helper functions for distance calculation and partitioning to enhance code modularity.
    \index{Helper Functions}
    
    \item \textbf{Code Readability}: Maintain clear and readable code through meaningful variable names and structured logic.
    \index{Code Readability}
    
    \item \textbf{Test Extensively}: Implement a wide range of test cases, including edge cases like multiple points with the same distance, to ensure robustness.
    \index{Extensive Testing}
    
    \item \textbf{Understand Algorithm Trade-offs}: Recognize the trade-offs between different approaches in terms of time and space complexities.
    \index{Algorithm Trade-offs}
    
    \item \textbf{Use Built-In Sorting Functions}: When using sorting-based approaches, leverage built-in functions for efficiency and simplicity.
    \index{Built-In Sorting}
    
    \item \textbf{Avoid Redundant Calculations}: Ensure that distance calculations are performed only when necessary to optimize performance.
    \index{Avoiding Redundant Calculations}
    
    \item \textbf{Language-Specific Features}: Utilize language-specific features or libraries that can simplify implementation, such as heapq in Python.
    \index{Language-Specific Features}
\end{itemize}

\section*{Corner and Special Cases to Test When Writing the Code}

When implementing the solution for the \textbf{K Closest Points to Origin} problem, it is crucial to consider and rigorously test various edge cases to ensure robustness and correctness:

\begin{itemize}
    \item \textbf{Multiple Points with Same Distance}: Ensure that the algorithm handles multiple points having the same distance from the origin.
    \index{Same Distance Points}
    
    \item \textbf{Points at Origin}: Include points that are exactly at the origin \((0,0)\).
    \index{Points at Origin}
    
    \item \textbf{Negative Coordinates}: Ensure that the algorithm correctly computes distances for points with negative \(x\) or \(y\) coordinates.
    \index{Negative Coordinates}
    
    \item \textbf{Large Coordinates}: Test with points having very large or very small coordinate values to verify integer handling.
    \index{Large Coordinates}
    
    \item \textbf{K Equals Number of Points}: When \(K\) is equal to the number of points, the algorithm should return all points.
    \index{K Equals Number of Points}
    
    \item \textbf{K is One}: Test with \(K = 1\) to ensure the closest point is correctly identified.
    \index{K is One}
    
    \item \textbf{All Points Same}: All points have the same coordinates.
    \index{All Points Same}
    
    \item \textbf{K is Zero}: Although \(K\) is defined to be at least 1, ensure that the algorithm gracefully handles \(K = 0\) if allowed.
    \index{K is Zero}
    
    \item \textbf{Single Point}: Only one point is provided, and \(K = 1\).
    \index{Single Point}
    
    \item \textbf{Mixed Coordinates}: Points with a mix of positive and negative coordinates.
    \index{Mixed Coordinates}
    
    \item \textbf{Points with Zero Distance}: Multiple points at the origin.
    \index{Zero Distance Points}
    
    \item \textbf{Sparse and Dense Points}: Densely packed points and sparsely distributed points.
    \index{Sparse and Dense Points}
    
    \item \textbf{Duplicate Points}: Multiple identical points in the input list.
    \index{Duplicate Points}
    
    \item \textbf{K Greater Than Number of Unique Points}: Ensure that the algorithm handles cases where \(K\) exceeds the number of unique points if applicable.
    \index{K Greater Than Unique Points}
\end{itemize}

\section*{Implementation Considerations}

When implementing the \texttt{kClosest} function, keep in mind the following considerations to ensure robustness and efficiency:

\begin{itemize}
    \item \textbf{Data Type Selection}: Use appropriate data types that can handle large input values without overflow or precision loss.
    \index{Data Type Selection}
    
    \item \textbf{Optimizing Distance Calculations}: Avoid calculating the square root since it is unnecessary for comparison purposes.
    \index{Optimizing Distance Calculations}
    
    \item \textbf{Choosing the Right Algorithm}: Select an algorithm based on the size of the input and the value of \(K\) to optimize time and space complexities.
    \index{Choosing the Right Algorithm}
    
    \item \textbf{Language-Specific Libraries}: Utilize language-specific libraries and functions (e.g., \texttt{heapq} in Python) to simplify implementation and enhance performance.
    \index{Language-Specific Libraries}
    
    \item \textbf{Avoiding Redundant Calculations}: Ensure that each point's distance is calculated only once to optimize performance.
    \index{Avoiding Redundant Calculations}
    
    \item \textbf{Implementing Helper Functions}: Create helper functions for tasks like distance calculation and partitioning to enhance modularity and readability.
    \index{Helper Functions}
    
    \item \textbf{Edge Case Handling}: Implement checks for edge cases to prevent incorrect results or runtime errors.
    \index{Edge Case Handling}
    
    \item \textbf{Testing and Validation}: Develop a comprehensive suite of test cases that cover all possible scenarios, including edge cases, to validate the correctness and efficiency of the implementation.
    \index{Testing and Validation}
    
    \item \textbf{Scalability}: Design the algorithm to scale efficiently with increasing input sizes, maintaining performance and resource utilization.
    \index{Scalability}
    
    \item \textbf{Consistent Naming Conventions}: Use consistent and descriptive naming conventions for variables and functions to improve code clarity.
    \index{Naming Conventions}
    
    \item \textbf{Memory Management}: Ensure that the algorithm manages memory efficiently, especially when dealing with large datasets.
    \index{Memory Management}
    
    \item \textbf{Avoiding Stack Overflow}: If implementing recursive approaches, be mindful of recursion limits and potential stack overflow issues.
    \index{Avoiding Stack Overflow}
    
    \item \textbf{Implementing Iterative Solutions}: Prefer iterative solutions when recursion may lead to increased space complexity or stack overflow.
    \index{Implementing Iterative Solutions}
\end{itemize}

\section*{Conclusion}

The \textbf{K Closest Points to Origin} problem exemplifies the application of efficient selection algorithms and geometric computations to solve spatial challenges effectively. By leveraging QuickSelect or heap-based methods, the algorithm achieves optimal time and space complexities, making it highly suitable for large datasets. Understanding and implementing such techniques not only enhances problem-solving skills but also provides a foundation for tackling more advanced Computational Geometry problems involving nearest neighbor searches, clustering, and spatial data analysis.

\printindex

% \input{sections/rectangle_overlap}
% \input{sections/rectangle_area}
% \input{sections/k_closest_points_to_origin}
% \input{sections/the_skyline_problem}
% % filename: the_skyline_problem.tex

\problemsection{The Skyline Problem}
\label{chap:The_Skyline_Problem}
\marginnote{\href{https://leetcode.com/problems/the-skyline-problem/}{[LeetCode Link]}\index{LeetCode}}
\marginnote{\href{https://www.geeksforgeeks.org/the-skyline-problem/}{[GeeksForGeeks Link]}\index{GeeksForGeeks}}
\marginnote{\href{https://www.interviewbit.com/problems/the-skyline-problem/}{[InterviewBit Link]}\index{InterviewBit}}
\marginnote{\href{https://app.codesignal.com/challenges/the-skyline-problem}{[CodeSignal Link]}\index{CodeSignal}}
\marginnote{\href{https://www.codewars.com/kata/the-skyline-problem/train/python}{[Codewars Link]}\index{Codewars}}

The \textbf{Skyline Problem} is a complex Computational Geometry challenge that involves computing the skyline formed by a collection of buildings in a 2D cityscape. Each building is represented by its left and right x-coordinates and its height. The skyline is defined by a list of "key points" where the height changes. This problem tests one's ability to handle large datasets, implement efficient sweep line algorithms, and manage event-driven processing. Mastery of this problem is essential for applications in computer graphics, urban planning simulations, and geographic information systems (GIS).

\section*{Problem Statement}

You are given a list of buildings in a cityscape. Each building is represented as a triplet \([Li, Ri, Hi]\), where \(Li\) and \(Ri\) are the x-coordinates of the left and right edges of the building, respectively, and \(Hi\) is the height of the building.

The skyline should be represented as a list of key points \([x, y]\) in sorted order by \(x\)-coordinate, where \(y\) is the height of the skyline at that point. The skyline should only include critical points where the height changes.

\textbf{Function signature in Python:}
\begin{lstlisting}[language=Python]
def getSkyline(buildings: List[List[int]]) -> List[List[int]]:
\end{lstlisting}

\section*{Examples}

\textbf{Example 1:}

\begin{verbatim}
Input: buildings = [[2,9,10], [3,7,15], [5,12,12], [15,20,10], [19,24,8]]
Output: [[2,10], [3,15], [7,12], [12,0], [15,10], [20,8], [24,0]]
Explanation:
- At x=2, the first building starts, height=10.
- At x=3, the second building starts, height=15.
- At x=7, the second building ends, the third building is still ongoing, height=12.
- At x=12, the third building ends, height drops to 0.
- At x=15, the fourth building starts, height=10.
- At x=20, the fourth building ends, the fifth building is still ongoing, height=8.
- At x=24, the fifth building ends, height drops to 0.
\end{verbatim}

\textbf{Example 2:}

\begin{verbatim}
Input: buildings = [[0,2,3], [2,5,3]]
Output: [[0,3], [5,0]]
Explanation:
- The two buildings are contiguous and have the same height, so the skyline drops to 0 at x=5.
\end{verbatim}

\textbf{Example 3:}

\begin{verbatim}
Input: buildings = [[1,3,3], [2,4,4], [5,6,1]]
Output: [[1,3], [2,4], [4,0], [5,1], [6,0]]
Explanation:
- At x=1, first building starts, height=3.
- At x=2, second building starts, height=4.
- At x=4, second building ends, height drops to 0.
- At x=5, third building starts, height=1.
- At x=6, third building ends, height drops to 0.
\end{verbatim}

\textbf{Example 4:}

\begin{verbatim}
Input: buildings = [[0,5,0]]
Output: []
Explanation:
- A building with height 0 does not contribute to the skyline.
\end{verbatim}

\textbf{Constraints:}

\begin{itemize}
    \item \(1 \leq \text{buildings.length} \leq 10^4\)
    \item \(0 \leq Li < Ri \leq 10^9\)
    \item \(0 \leq Hi \leq 10^4\)
\end{itemize}

\section*{Algorithmic Approach}

The \textbf{Sweep Line Algorithm} is an efficient method for solving the Skyline Problem. It involves processing events (building start and end points) in sorted order while maintaining a data structure (typically a max heap) to keep track of active buildings. Here's a step-by-step approach:

\subsection*{1. Event Representation}

Transform each building into two events:
\begin{itemize}
    \item **Start Event:** \((Li, -Hi)\) – Negative height indicates a building starts.
    \item **End Event:** \((Ri, Hi)\) – Positive height indicates a building ends.
\end{itemize}

Sorting the events ensures that start events are processed before end events at the same x-coordinate, and taller buildings are processed before shorter ones.

\subsection*{2. Sorting the Events}

Sort all events based on:
\begin{enumerate}
    \item **x-coordinate:** Ascending order.
    \item **Height:**
    \begin{itemize}
        \item For start events, taller buildings come first.
        \item For end events, shorter buildings come first.
    \end{itemize}
\end{enumerate}

\subsection*{3. Processing the Events}

Use a max heap to keep track of active building heights. Iterate through the sorted events:
\begin{enumerate}
    \item **Start Event:**
    \begin{itemize}
        \item Add the building's height to the heap.
    \end{itemize}
    
    \item **End Event:**
    \begin{itemize}
        \item Remove the building's height from the heap.
    \end{itemize}
    
    \item **Determine Current Max Height:**
    \begin{itemize}
        \item The current max height is the top of the heap.
    \end{itemize}
    
    \item **Update Skyline:**
    \begin{itemize}
        \item If the current max height differs from the previous max height, add a new key point \([x, current\_max\_height]\).
    \end{itemize}
\end{enumerate}

\subsection*{4. Finalizing the Skyline}

After processing all events, the accumulated key points represent the skyline.

\marginnote{The Sweep Line Algorithm efficiently handles dynamic changes in active buildings, ensuring accurate skyline construction.}

\section*{Complexities}

\begin{itemize}
    \item \textbf{Time Complexity:} \(O(n \log n)\), where \(n\) is the number of buildings. Sorting the events takes \(O(n \log n)\), and each heap operation takes \(O(\log n)\).
    
    \item \textbf{Space Complexity:} \(O(n)\), due to the storage of events and the heap.
\end{itemize}

\section*{Python Implementation}

\marginnote{Implementing the Sweep Line Algorithm with a max heap ensures an efficient and accurate solution.}

Below is the complete Python code implementing the \texttt{getSkyline} function:

\begin{fullwidth}
\begin{lstlisting}[language=Python]
from typing import List
import heapq

class Solution:
    def getSkyline(self, buildings: List[List[int]]) -> List[List[int]]:
        # Create a list of all events
        # For start events, use negative height to ensure they are processed before end events
        events = []
        for L, R, H in buildings:
            events.append((L, -H))
            events.append((R, H))
        
        # Sort the events
        # First by x-coordinate, then by height
        events.sort()
        
        # Max heap to keep track of active buildings
        heap = [0]  # Initialize with ground level
        heapq.heapify(heap)
        active_heights = {0: 1}  # Dictionary to count heights
        
        result = []
        prev_max = 0
        
        for x, h in events:
            if h < 0:
                # Start of a building, add height to heap and dictionary
                heapq.heappush(heap, h)
                active_heights[h] = active_heights.get(h, 0) + 1
            else:
                # End of a building, remove height from dictionary
                active_heights[h] -= 1
                if active_heights[h] == 0:
                    del active_heights[h]
            
            # Current max height
            while heap and active_heights.get(heap[0], 0) == 0:
                heapq.heappop(heap)
            current_max = -heap[0] if heap else 0
            
            # If the max height has changed, add to result
            if current_max != prev_max:
                result.append([x, current_max])
                prev_max = current_max
        
        return result

# Example usage:
solution = Solution()
print(solution.getSkyline([[2,9,10], [3,7,15], [5,12,12], [15,20,10], [19,24,8]]))
# Output: [[2,10], [3,15], [7,12], [12,0], [15,10], [20,8], [24,0]]

print(solution.getSkyline([[0,2,3], [2,5,3]]))
# Output: [[0,3], [5,0]]

print(solution.getSkyline([[1,3,3], [2,4,4], [5,6,1]]))
# Output: [[1,3], [2,4], [4,0], [5,1], [6,0]]

print(solution.getSkyline([[0,5,0]]))
# Output: []
\end{lstlisting}
\end{fullwidth}

This implementation efficiently constructs the skyline by processing all building events in sorted order and maintaining active building heights using a max heap. It ensures that only critical points where the skyline changes are recorded.

\section*{Explanation}

The \texttt{getSkyline} function constructs the skyline formed by a set of buildings by leveraging the Sweep Line Algorithm and a max heap to track active buildings. Here's a detailed breakdown of the implementation:

\subsection*{1. Event Representation}

\begin{itemize}
    \item Each building is transformed into two events:
    \begin{itemize}
        \item **Start Event:** \((Li, -Hi)\) – Negative height indicates the start of a building.
        \item **End Event:** \((Ri, Hi)\) – Positive height indicates the end of a building.
    \end{itemize}
\end{itemize}

\subsection*{2. Sorting the Events}

\begin{itemize}
    \item Events are sorted primarily by their x-coordinate in ascending order.
    \item For events with the same x-coordinate:
    \begin{itemize}
        \item Start events (with negative heights) are processed before end events.
        \item Taller buildings are processed before shorter ones.
    \end{itemize}
\end{itemize}

\subsection*{3. Processing the Events}

\begin{itemize}
    \item **Heap Initialization:**
    \begin{itemize}
        \item A max heap is initialized with a ground level height of 0.
        \item A dictionary \texttt{active\_heights} tracks the count of active building heights.
    \end{itemize}
    
    \item **Iterating Through Events:**
    \begin{enumerate}
        \item **Start Event:**
        \begin{itemize}
            \item Add the building's height to the heap.
            \item Increment the count of the height in \texttt{active\_heights}.
        \end{itemize}
        
        \item **End Event:**
        \begin{itemize}
            \item Decrement the count of the building's height in \texttt{active\_heights}.
            \item If the count reaches zero, remove the height from the dictionary.
        \end{itemize}
        
        \item **Determine Current Max Height:**
        \begin{itemize}
            \item Remove heights from the heap that are no longer active.
            \item The current max height is the top of the heap.
        \end{itemize}
        
        \item **Update Skyline:**
        \begin{itemize}
            \item If the current max height differs from the previous max height, add a new key point \([x, current\_max\_height]\).
        \end{itemize}
    \end{enumerate}
\end{itemize}

\subsection*{4. Finalizing the Skyline}

\begin{itemize}
    \item After processing all events, the \texttt{result} list contains the key points defining the skyline.
\end{itemize}

\subsection*{5. Example Walkthrough}

Consider the first example:
\begin{verbatim}
Input: buildings = [[2,9,10], [3,7,15], [5,12,12], [15,20,10], [19,24,8]]
Output: [[2,10], [3,15], [7,12], [12,0], [15,10], [20,8], [24,0]]
\end{verbatim}

\begin{enumerate}
    \item **Event Transformation:**
    \begin{itemize}
        \item \((2, -10)\), \((9, 10)\)
        \item \((3, -15)\), \((7, 15)\)
        \item \((5, -12)\), \((12, 12)\)
        \item \((15, -10)\), \((20, 10)\)
        \item \((19, -8)\), \((24, 8)\)
    \end{itemize}
    
    \item **Sorting Events:**
    \begin{itemize}
        \item Sorted order: \((2, -10)\), \((3, -15)\), \((5, -12)\), \((7, 15)\), \((9, 10)\), \((12, 12)\), \((15, -10)\), \((19, -8)\), \((20, 10)\), \((24, 8)\)
    \end{itemize}
    
    \item **Processing Events:**
    \begin{itemize}
        \item At each event, update the heap and determine if the skyline height changes.
    \end{itemize}
    
    \item **Result Construction:**
    \begin{itemize}
        \item The resulting skyline key points are accumulated as \([[2,10], [3,15], [7,12], [12,0], [15,10], [20,8], [24,0]]\).
    \end{itemize}
\end{enumerate}

Thus, the function correctly constructs the skyline based on the buildings' positions and heights.

\section*{Why This Approach}

The Sweep Line Algorithm combined with a max heap offers an optimal solution with \(O(n \log n)\) time complexity and efficient handling of overlapping buildings. By processing events in sorted order and maintaining active building heights, the algorithm ensures that all critical points in the skyline are accurately identified without redundant computations.

\section*{Alternative Approaches}

\subsection*{1. Divide and Conquer}

Divide the set of buildings into smaller subsets, compute the skyline for each subset, and then merge the skylines.

\begin{lstlisting}[language=Python]
class Solution:
    def getSkyline(self, buildings: List[List[int]]) -> List[List[int]]:
        def merge(left, right):
            h1, h2 = 0, 0
            i, j = 0, 0
            merged = []
            while i < len(left) and j < len(right):
                if left[i][0] < right[j][0]:
                    x, h1 = left[i]
                    i += 1
                elif left[i][0] > right[j][0]:
                    x, h2 = right[j]
                    j += 1
                else:
                    x, h1 = left[i]
                    _, h2 = right[j]
                    i += 1
                    j += 1
                max_h = max(h1, h2)
                if not merged or merged[-1][1] != max_h:
                    merged.append([x, max_h])
            merged.extend(left[i:])
            merged.extend(right[j:])
            return merged
        
        def divide(buildings):
            if not buildings:
                return []
            if len(buildings) == 1:
                L, R, H = buildings[0]
                return [[L, H], [R, 0]]
            mid = len(buildings) // 2
            left = divide(buildings[:mid])
            right = divide(buildings[mid:])
            return merge(left, right)
        
        return divide(buildings)
\end{lstlisting}

\textbf{Complexities:}
\begin{itemize}
    \item \textbf{Time Complexity:} \(O(n \log n)\)
    \item \textbf{Space Complexity:} \(O(n)\)
\end{itemize}

\subsection*{2. Using Segment Trees}

Implement a segment tree to manage and query overlapping building heights dynamically.

\textbf{Note}: This approach is more complex and is generally used for advanced scenarios with multiple dynamic queries.

\section*{Similar Problems to This One}

Several problems involve skyline-like constructions, spatial data analysis, and efficient event processing, utilizing similar algorithmic strategies:

\begin{itemize}
    \item \textbf{Merge Intervals}: Merge overlapping intervals in a list.
    \item \textbf{Largest Rectangle in Histogram}: Find the largest rectangular area in a histogram.
    \item \textbf{Interval Partitioning}: Assign intervals to resources without overlap.
    \item \textbf{Line Segment Intersection}: Detect intersections among line segments.
    \item \textbf{Closest Pair of Points}: Find the closest pair of points in a set.
    \item \textbf{Convex Hull}: Compute the convex hull of a set of points.
    \item \textbf{Point Inside Polygon}: Determine if a point lies inside a given polygon.
    \item \textbf{Range Searching}: Efficiently query geometric data within a specified range.
\end{itemize}

These problems reinforce concepts of event-driven processing, spatial reasoning, and efficient algorithm design in various contexts.

\section*{Things to Keep in Mind and Tricks}

When tackling the \textbf{Skyline Problem}, consider the following tips and best practices to enhance efficiency and correctness:

\begin{itemize}
    \item \textbf{Understand Sweep Line Technique}: Grasp how the sweep line algorithm processes events in sorted order to handle dynamic changes efficiently.
    \index{Sweep Line Technique}
    
    \item \textbf{Leverage Priority Queues (Heaps)}: Use max heaps to keep track of active buildings' heights, enabling quick access to the current maximum height.
    \index{Priority Queues}
    
    \item \textbf{Handle Start and End Events Differently}: Differentiate between building start and end events to accurately manage active heights.
    \index{Start and End Events}
    
    \item \textbf{Optimize Event Sorting}: Sort events primarily by x-coordinate and secondarily by height to ensure correct processing order.
    \index{Event Sorting}
    
    \item \textbf{Manage Active Heights Efficiently}: Use data structures that allow efficient insertion, deletion, and retrieval of maximum elements.
    \index{Active Heights Management}
    
    \item \textbf{Avoid Redundant Key Points}: Only record key points when the skyline height changes to minimize the output list.
    \index{Avoiding Redundant Key Points}
    
    \item \textbf{Implement Helper Functions}: Create helper functions for tasks like distance calculation, event handling, and heap management to enhance modularity.
    \index{Helper Functions}
    
    \item \textbf{Code Readability}: Maintain clear and readable code through meaningful variable names and structured logic.
    \index{Code Readability}
    
    \item \textbf{Test Extensively}: Implement a wide range of test cases, including overlapping, non-overlapping, and edge-touching buildings, to ensure robustness.
    \index{Extensive Testing}
    
    \item \textbf{Handle Degenerate Cases}: Manage cases where buildings have zero height or identical coordinates gracefully.
    \index{Degenerate Cases}
    
    \item \textbf{Understand Geometric Relationships}: Grasp how buildings overlap and influence the skyline to simplify the algorithm.
    \index{Geometric Relationships}
    
    \item \textbf{Use Appropriate Data Structures}: Utilize appropriate data structures like heaps, lists, and dictionaries to manage and process data efficiently.
    \index{Appropriate Data Structures}
    
    \item \textbf{Optimize for Large Inputs}: Design the algorithm to handle large numbers of buildings without significant performance degradation.
    \index{Optimizing for Large Inputs}
    
    \item \textbf{Implement Iterative Solutions Carefully}: Ensure that loop conditions are correctly defined to prevent infinite loops or incorrect terminations.
    \index{Iterative Solutions}
    
    \item \textbf{Consistent Naming Conventions}: Use consistent and descriptive naming conventions for variables and functions to improve code clarity.
    \index{Naming Conventions}
\end{itemize}

\section*{Corner and Special Cases to Test When Writing the Code}

When implementing the solution for the \textbf{Skyline Problem}, it is crucial to consider and rigorously test various edge cases to ensure robustness and correctness:

\begin{itemize}
    \item \textbf{No Overlapping Buildings}: All buildings are separate and do not overlap.
    \index{No Overlapping Buildings}
    
    \item \textbf{Fully Overlapping Buildings}: Multiple buildings completely overlap each other.
    \index{Fully Overlapping Buildings}
    
    \item \textbf{Buildings Touching at Edges}: Buildings share common edges without overlapping.
    \index{Buildings Touching at Edges}
    
    \item \textbf{Buildings Touching at Corners}: Buildings share common corners without overlapping.
    \index{Buildings Touching at Corners}
    
    \item \textbf{Single Building}: Only one building is present.
    \index{Single Building}
    
    \item \textbf{Multiple Buildings with Same Start or End}: Multiple buildings start or end at the same x-coordinate.
    \index{Same Start or End}
    
    \item \textbf{Buildings with Zero Height}: Buildings that have zero height should not affect the skyline.
    \index{Buildings with Zero Height}
    
    \item \textbf{Large Number of Buildings}: Test with a large number of buildings to ensure performance and scalability.
    \index{Large Number of Buildings}
    
    \item \textbf{Buildings with Negative Coordinates}: Buildings positioned in negative coordinate space.
    \index{Negative Coordinates}
    
    \item \textbf{Boundary Values}: Buildings at the minimum and maximum limits of the coordinate range.
    \index{Boundary Values}
    
    \item \textbf{Buildings with Identical Coordinates}: Multiple buildings with the same coordinates.
    \index{Identical Coordinates}
    
    \item \textbf{Sequential Buildings}: Buildings placed sequentially without gaps.
    \index{Sequential Buildings}
    
    \item \textbf{Overlapping and Non-Overlapping Mixed}: A mix of overlapping and non-overlapping buildings.
    \index{Overlapping and Non-Overlapping Mixed}
    
    \item \textbf{Buildings with Very Large Heights}: Buildings with heights at the upper limit of the constraints.
    \index{Very Large Heights}
    
    \item \textbf{Empty Input}: No buildings are provided.
    \index{Empty Input}
\end{itemize}

\section*{Implementation Considerations}

When implementing the \texttt{getSkyline} function, keep in mind the following considerations to ensure robustness and efficiency:

\begin{itemize}
    \item \textbf{Data Type Selection}: Use appropriate data types that can handle large input values and avoid overflow or precision issues.
    \index{Data Type Selection}
    
    \item \textbf{Optimizing Event Sorting}: Efficiently sort events based on x-coordinates and heights to ensure correct processing order.
    \index{Optimizing Event Sorting}
    
    \item \textbf{Handling Large Inputs}: Design the algorithm to handle up to \(10^4\) buildings efficiently without significant performance degradation.
    \index{Handling Large Inputs}
    
    \item \textbf{Using Efficient Data Structures}: Utilize heaps, lists, and dictionaries effectively to manage and process events and active heights.
    \index{Efficient Data Structures}
    
    \item \textbf{Avoiding Redundant Calculations}: Ensure that distance and overlap calculations are performed only when necessary to optimize performance.
    \index{Avoiding Redundant Calculations}
    
    \item \textbf{Code Readability and Documentation}: Maintain clear and readable code through meaningful variable names and comprehensive comments to facilitate understanding and maintenance.
    \index{Code Readability}
    
    \item \textbf{Edge Case Handling}: Implement checks for edge cases to prevent incorrect results or runtime errors.
    \index{Edge Case Handling}
    
    \item \textbf{Implementing Helper Functions}: Create helper functions for tasks like distance calculation, event handling, and heap management to enhance modularity.
    \index{Helper Functions}
    
    \item \textbf{Consistent Naming Conventions}: Use consistent and descriptive naming conventions for variables and functions to improve code clarity.
    \index{Naming Conventions}
    
    \item \textbf{Memory Management}: Ensure that the algorithm manages memory efficiently, especially when dealing with large datasets.
    \index{Memory Management}
    
    \item \textbf{Implementing Iterative Solutions Carefully}: Ensure that loop conditions are correctly defined to prevent infinite loops or incorrect terminations.
    \index{Iterative Solutions}
    
    \item \textbf{Avoiding Floating-Point Precision Issues}: Since the problem deals with integers, floating-point precision is not a concern, simplifying the implementation.
    \index{Floating-Point Precision}
    
    \item \textbf{Testing and Validation}: Develop a comprehensive suite of test cases that cover all possible scenarios, including edge cases, to validate the correctness and efficiency of the implementation.
    \index{Testing and Validation}
    
    \item \textbf{Performance Considerations}: Optimize the loop conditions and operations to ensure that the function runs efficiently, especially for large input numbers.
    \index{Performance Considerations}
\end{itemize}

\section*{Conclusion}

The \textbf{Skyline Problem} is a quintessential example of applying advanced algorithmic techniques and geometric reasoning to solve complex spatial challenges. By leveraging the Sweep Line Algorithm and maintaining active building heights using a max heap, the solution efficiently constructs the skyline with optimal time and space complexities. Understanding and implementing such sophisticated algorithms not only enhances problem-solving skills but also provides a foundation for tackling a wide array of Computational Geometry problems in various domains, including computer graphics, urban planning simulations, and geographic information systems.

\printindex

% \input{sections/rectangle_overlap}
% \input{sections/rectangle_area}
% \input{sections/k_closest_points_to_origin}
% \input{sections/the_skyline_problem}