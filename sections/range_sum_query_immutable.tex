% file: range_sum_query_immutable.tex

\problemsection{Range Sum Query - Immutable}
\label{problem:range_sum_query_immutable}
\marginnote{This problem utilizes a prefix sum technique to efficiently calculate range sums in constant time after initial preprocessing.}

The \textbf{Range Sum Query - Immutable} problem involves processing multiple range sum queries on a static array. The challenge is to efficiently compute the sum of elements between two indices \( left \) and \( right \) inclusive, multiple times, after an initial preprocessing step.

\section*{Problem Statement}

Given an integer array \texttt{nums}, handle multiple queries of the following type:

- \textbf{sumRange}(\texttt{left}, \texttt{right}): Return the sum of the elements of \texttt{nums} between indices \texttt{left} and \texttt{right} inclusive (i.e., \texttt{nums[left]} + \texttt{nums[left + 1]} + ... + \texttt{nums[right]}).

Implement the \texttt{NumArray} class:

- \texttt{NumArray(int[] nums)} Initializes the object with the integer array \texttt{nums}.
- \texttt{int sumRange(int left, int right)} Returns the sum of the elements of \texttt{nums} between indices \texttt{left} and \texttt{right} inclusive.

\textbf{Example 1:}
\begin{verbatim}
Input:
["NumArray", "sumRange", "sumRange", "sumRange"]
[[[-2, 0, 3, -5, 2, -1]], [0, 2], [2, 5], [0, 5]]
Output:
[null, 1, -1, -3]

Explanation:
NumArray numArray = new NumArray([-2,0,3,-5,2,-1]);
numArray.sumRange(0, 2); // return (-2) + 0 + 3 = 1
numArray.sumRange(2, 5); // return 3 + (-5) + 2 + (-1) = -1
numArray.sumRange(0, 5); // return (-2) + 0 + 3 + (-5) + 2 + (-1) = -3
\end{verbatim}

LeetCode link: \href{https://leetcode.com/problems/range-sum-query-immutable/}{Range Sum Query - Immutable}\index{LeetCode}

\marginnote{\href{https://leetcode.com/problems/range-sum-query-immutable/}{[LeetCode Link]}\index{LeetCode}}
\marginnote{\href{https://www.geeksforgeeks.org/range-sum-query-immutable/}{[GeeksForGeeks Link]}\index{GeeksForGeeks}}
\marginnote{\href{https://www.hackerrank.com/challenges/range-sum-query-immutable/problem}{[HackerRank Link]}\index{HackerRank}}
\marginnote{\href{https://app.codesignal.com/challenges/range-sum-query-immutable}{[CodeSignal Link]}\index{CodeSignal}}
\marginnote{\href{https://www.interviewbit.com/problems/range-sum-query-immutable/}{[InterviewBit Link]}\index{InterviewBit}}
\marginnote{\href{https://www.educative.io/courses/grokking-the-coding-interview/RM8y8Y3nLdY}{[Educative Link]}\index{Educative}}
\marginnote{\href{https://www.codewars.com/kata/range-sum-query-immutable/train/python}{[Codewars Link]}\index{Codewars}}

\section*{Algorithmic Approach}

\subsection*{Main Concept}
The main idea to solve this problem is to use a \textbf{prefix sum} technique. By preprocessing the array to compute the cumulative sum up to each index, we can answer each range sum query in constant time. Here's a step-by-step breakdown:

\begin{enumerate}
    \item \textbf{Prefix Sum Calculation:}
    \begin{itemize}
        \item Create a prefix sum array \texttt{prefix\_sums} where \texttt{prefix\_sums[i]} represents the sum of elements from index \(0\) to \(i-1\) in the original array.
        \item Initialize \texttt{prefix\_sums[0]} to \(0\).
        \item For each index \(i\) from \(1\) to \(n\) (where \(n\) is the length of \texttt{nums}), compute \texttt{prefix\_sums[i]} as \texttt{prefix\_sums[i-1] + nums[i-1]}.
    \end{itemize}
    
    \item \textbf{Handling Queries:}
    \begin{itemize}
        \item For each \texttt{sumRange(left, right)} query, compute the sum as \texttt{prefix\_sums[right + 1] - prefix\_sums[left]}.
    \end{itemize}
\end{enumerate}

This approach ensures that after an initial \( O(n) \) preprocessing step, each query can be answered in \( O(1) \) time.

\marginnote{Prefix sums allow for efficient computation of range sums by leveraging precomputed cumulative totals.}

\section*{Complexities}

\begin{itemize}
    \item \textbf{Time Complexity:} 
    \begin{itemize}
        \item \textbf{Initialization:} \( O(n) \), where \( n \) is the number of elements in \texttt{nums}, for computing the prefix sums.
        \item \textbf{sumRange Queries:} \( O(1) \) per query, since each query involves a simple subtraction operation.
    \end{itemize}
    
    \item \textbf{Space Complexity:} 
    \begin{itemize}
        \item \( O(n) \) for storing the prefix sums in the \texttt{prefix\_sums} array.
    \end{itemize}
\end{itemize}

\newpage
\section*{Python Implementation}

\marginnote{Implementing the prefix sum technique allows for swift range sum computations by utilizing preprocessed cumulative sums.}

Below is the complete Python code for the `NumArray` class, which implements the `sumRange` method to calculate the sum of elements between two indices:

\begin{fullwidth}
\begin{lstlisting}[language=Python]
class NumArray:
    def __init__(self, nums):
        """
        :type nums: List[int]
        """
        self.prefix_sums = [0]
        for num in nums:
            self.prefix_sums.append(self.prefix_sums[-1] + num)

    def sumRange(self, left, right):
        """
        :type left: int
        :type right: int
        :rtype: int
        """
        return self.prefix_sums[right + 1] - self.prefix_sums[left]
\end{lstlisting}
\end{fullwidth}

\section*{Explanation}

The provided Python implementation defines a class \texttt{NumArray} which efficiently handles range sum queries using the prefix sum technique. Here's a detailed breakdown of the implementation:

\begin{itemize}
    \item \textbf{Initialization (\_\_init\_\_ method):}
    \begin{itemize}
        \item **Prefix Sum Array:** Initialize \texttt{self.prefix\_sums} with a starting value of \(0\).
        \item **Cumulative Sum Calculation:** Iterate through each number in the input \texttt{nums} list and append the sum of the current number with the last element in \texttt{self.prefix\_sums} to build the prefix sum array.
    \end{itemize}
    
    \item \textbf{Range Sum Query (\texttt{sumRange} method):}
    \begin{itemize}
        \item **Sum Calculation:** For each query with indices \texttt{left} and \texttt{right}, compute the sum of the range by subtracting \texttt{self.prefix\_sums[left]} from \texttt{self.prefix\_sums[right + 1]}.
        \item **Return Value:** The result is returned as an integer representing the sum of the specified range.
    \end{itemize}
\end{itemize}

This implementation ensures that after an initial preprocessing step, each range sum query is answered in constant time, making it highly efficient for multiple queries.

\section*{Why This Approach}

The **prefix sum** approach is chosen for its simplicity and efficiency. By precomputing the cumulative sums of the array, range sum queries can be answered in \( O(1) \) time, which is optimal for scenarios with multiple queries. This method avoids redundant computations and leverages the benefits of dynamic programming by building up solutions to larger problems based on previously computed results.

\section*{Alternative Approaches}

An alternative approach involves using a **Segment Tree** or a **Binary Indexed Tree (Fenwick Tree)** to handle range sum queries. These data structures allow for dynamic updates and range queries in \( O(\log n) \) time. However, for the **Range Sum Query - Immutable** problem, where the array does not change after initialization, the prefix sum method is more straightforward and efficient in terms of both implementation and runtime.

Another possibility is to handle each query individually by iterating through the specified range and summing the elements. However, this brute-force approach results in \( O(n) \) time per query, which is inefficient for a large number of queries.

\section*{Similar Problems to This One}

Similar problems involve range queries and efficient data retrieval from arrays or matrices, such as:

\begin{itemize}
    \item \hyperref[problem:range_sum_query_2d_immutable]{Range Sum Query 2D - Immutable}\index{Range Sum Query 2D - Immutable}
    \item \hyperref[problem:range_sum_query_mutable]{Range Sum Query - Mutable}\index{Range Sum Query - Mutable}
    \item \hyperref[problem:subarray_sum_equals_k]{Subarray Sum Equals K}\index{Subarray Sum Equals K}
    \item \hyperref[problem:maximum_subarray]{Maximum Subarray}\index{Maximum Subarray}
    \item \hyperref[problem:finding_subarray_sum]{Finding Subarray Sum}\index{Finding Subarray Sum}
\end{itemize}

\section*{Things to Keep in Mind and Tricks}

When solving range query problems, consider the following tips:

\begin{itemize}
    \item \textbf{Preprocessing:} Utilize preprocessing techniques like prefix sums to reduce query time.
    \index{Preprocessing}
    
    \item \textbf{Space-Time Tradeoff:} Balance between additional space used for preprocessing and the time saved during queries.
    \index{Space-Time Tradeoff}
    
    \item \textbf{Immutable vs. Mutable:} Choose appropriate data structures based on whether the array can change after initialization.
    \index{Immutable vs. Mutable}
    
    \item \textbf{Edge Cases:} Always handle edge cases such as empty arrays, single-element arrays, and queries that span the entire array.
    \index{Edge Cases}
    
    \item \textbf{Efficient Data Structures:} When dealing with mutable arrays, consider advanced data structures like Segment Trees or Binary Indexed Trees.
    \index{Efficient Data Structures}
    
    \item \textbf{Avoid Redundant Computations:} Use memoization or dynamic programming principles to store and reuse results.
    \index{Avoid Redundant Computations}
\end{itemize}

\section*{Corner and Special Cases to Test When Writing the Code}

To ensure robustness, consider testing the following corner cases:

\begin{itemize}
    \item \textbf{Empty Array:} \texttt{nums = []}. Should handle gracefully, possibly returning 0 or raising an exception based on problem constraints.
    \index{Corner Cases}
    
    \item \textbf{Single Element Array:} \texttt{nums = [5]}. Queries like \texttt{sumRange(0, 0)} should return the single element.
    \index{Corner Cases}
    
    \item \textbf{All Negative Numbers:} \texttt{nums = [-1, -2, -3]}. Ensure that sums are computed correctly with negative values.
    \index{Corner Cases}
    
    \item \textbf{All Zeroes:} \texttt{nums = [0, 0, 0, 0]}. Sums should reflect the number of zeroes in the range.
    \index{Corner Cases}
    
    \item \textbf{Large Array with Large Numbers:} Test with large input sizes and large integer values to ensure no overflow issues and maintain performance.
    \index{Corner Cases}
    
    \item \textbf{Maximum Range Queries:} \texttt{sumRange(0, n-1)} where \( n \) is the size of the array. Should return the total sum of the array.
    \index{Corner Cases}
    
    \item \textbf{Overlapping Ranges:} Multiple queries with overlapping ranges to ensure independent and correct computations.
    \index{Corner Cases}
    
    \item \textbf{Invalid Queries:} Queries where \texttt{left > right} or indices are out of bounds. Should handle gracefully, possibly by returning 0 or raising exceptions.
    \index{Corner Cases}
    
    \item \textbf{Performance Testing:} Ensure that the implementation performs efficiently with a large number of queries (e.g., \( 10^5 \) queries).
    \index{Corner Cases}
\end{itemize}

\printindex