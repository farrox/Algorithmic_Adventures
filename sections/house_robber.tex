% Filename: house_robber.tex

\problemsection{House Robber}
\label{problem:House_Robber}

The **House Robber** problem is a classic dynamic programming challenge that requires solving an optimization problem under specific constraints. The task demonstrates how to make decisions at each step to maximize the overall result while adhering to the given restrictions.

\section*{Problem Statement}
You are a professional robber planning to rob houses along a street. Each house has a certain amount of money stashed, and the only constraint stopping you from robbing all of them is that adjacent houses have security systems connected. If two adjacent houses are robbed, the alarm will be triggered.

Given an integer array \texttt{nums} representing the amount of money stashed in each house, return the maximum amount of money you can rob tonight without alerting the police.

\textbf{Input:}
- \texttt{nums}: An array of integers where \( \texttt{nums}[i] \) represents the amount of money in the \( i \)-th house.

\textbf{Output:}
- An integer representing the maximum money you can rob without triggering the alarm.

\textbf{Example 1:}
\begin{verbatim}
Input: nums = [2, 7, 9, 3, 1]
Output: 12
Explanation: Rob house 1 (money = 2), skip house 2, rob house 3 (money = 9), 
and rob house 5 (money = 1). Total = 2 + 9 + 1 = 12.
\end{verbatim}

\textbf{Example 2:}
\begin{verbatim}
Input: nums = [1, 2, 3, 1]
Output: 4
Explanation: Rob house 1 (money = 1) and house 3 (money = 3). Total = 1 + 3 = 4.
\end{verbatim}

\textbf{Constraints:}
- \( 1 \leq \texttt{nums.length} \leq 10^4 \)
- \( 0 \leq \texttt{nums}[i] \leq 10^4 \)

\section*{Algorithmic Approach}
The problem can be solved using dynamic programming by recognizing that at each house, you have two choices: rob it or skip it. 

\textbf{Key Insight:}
The maximum money you can rob from the first \( i \) houses depends on whether you rob the \( i \)-th house:
\[
dp[i] = \max(dp[i-1], dp[i-2] + \texttt{nums}[i])
\]
- \( dp[i-1] \): Maximum money if you skip the \( i \)-th house.
- \( dp[i-2] + \texttt{nums}[i] \): Maximum money if you rob the \( i \)-th house (thus skipping the \((i-1)\)-th house).

\textbf{Steps:}
1. Define a DP array \( dp \), where \( dp[i] \) represents the maximum money that can be robbed from the first \( i \) houses.
2. Initialize base cases:
    \[
    dp[0] = \texttt{nums}[0], \quad dp[1] = \max(\texttt{nums}[0], \texttt{nums}[1])
    \]
3. Iterate through the array and compute \( dp[i] \) using the recurrence relation.
4. Return \( dp[\texttt{nums.length} - 1] \), which contains the maximum amount of money.

\subsection*{Complexities}
\begin{itemize}
    \item \textbf{Time Complexity:} \( O(n) \), where \( n \) is the length of \texttt{nums}.
    \item \textbf{Space Complexity:} 
        \begin{itemize}
            \item \( O(n) \) if a DP array is used.
            \item \( O(1) \) if only two variables are maintained for the previous two states.
        \end{itemize}
\end{itemize}

\section*{Python Implementation}
Below are two implementations: one using a DP array and another with optimized space complexity.

\subsection*{Using a DP Array}
\begin{fullwidth}
\begin{lstlisting}[language=Python]
def rob(nums):
    if len(nums) == 1:
        return nums[0]
    dp = [0] * len(nums)
    dp[0], dp[1] = nums[0], max(nums[0], nums[1])
    for i in range(2, len(nums)):
        dp[i] = max(dp[i - 1], dp[i - 2] + nums[i])
    return dp[-1]

# Example usage:
nums = [2, 7, 9, 3, 1]
print(rob(nums))  # Output: 12
\end{lstlisting}
\end{fullwidth}

\subsection*{Optimized Space Complexity}
\begin{fullwidth}
\begin{lstlisting}[language=Python]
def rob(nums):
    if len(nums) == 1:
        return nums[0]
    prev1, prev2 = nums[0], max(nums[0], nums[1])
    for i in range(2, len(nums)):
        prev1, prev2 = prev2, max(prev2, prev1 + nums[i])
    return prev2

# Example usage:
nums = [2, 7, 9, 3, 1]
print(rob(nums))  # Output: 12
\end{lstlisting}
\end{fullwidth}

\section*{Why This Approach?}
This approach efficiently solves the problem by leveraging overlapping subproblems and optimal substructure. The optimized solution reduces space complexity while maintaining clarity and correctness.

\section*{Alternative Approaches}
\begin{itemize}
    \item **Recursive Approach with Memoization:** Solve the problem recursively, storing results of subproblems in a hash map to avoid redundant calculations. This approach is conceptually similar but less efficient due to recursive overhead.
    \item **Tree Representation:** Visualize the problem as a binary tree of choices at each house. While useful for understanding, it is impractical for large inputs.
\end{itemize}

\section*{Similar Problems}
\begin{itemize}
    \item **House Robber II:** Circular street variation where the first and last houses are adjacent.
    \item **Knapsack Problem:** Optimization problem involving weights and values, similar in logic.
    \item **Maximum Sum of Non-Adjacent Elements:** A generalized version of the House Robber problem.
\end{itemize}

\section*{Corner and Special Cases to Test}
\begin{itemize}
    \item \( \texttt{nums} = [0] \): Single house with no money.
    \item \( \texttt{nums} = [10, 20] \): Two houses, test maximum selection.
    \item Large \( \texttt{nums} \): Ensure the solution scales for \( \texttt{nums.length} > 10^4 \).
\end{itemize}

\section*{Conclusion}
The House Robber problem is an excellent introduction to dynamic programming and optimization under constraints. It demonstrates how to make decisions at each step while adhering to problem-specific limitations. Mastery of this problem provides a foundation for tackling more complex DP problems.