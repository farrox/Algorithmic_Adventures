% filename: valid_sudoku.tex

\problemsection{Valid Sudoku}
\label{problem:Valid_Sudoku}

The **Valid Sudoku** problem involves determining whether a given \(9 \times 9\) Sudoku board is valid according to the rules of Sudoku. This problem highlights the importance of constraint validation and efficient traversal of 2D grids.

---

\section*{Problem Statement}
Given a \(9 \times 9\) Sudoku board, determine if it is valid. A Sudoku board is valid if:
1. Each row contains the digits \(1\) to \(9\) without repetition.
2. Each column contains the digits \(1\) to \(9\) without repetition.
3. Each of the nine \(3 \times 3\) sub-boxes contains the digits \(1\) to \(9\) without repetition.

Note:
- The Sudoku board may be partially filled, where empty cells are represented by the character \texttt{'.'}.

---

\textbf{Input:}
- \texttt{board}: A \(9 \times 9\) list of lists containing digits \texttt{'1'} to \texttt{'9'} and \texttt{'.'}.

\textbf{Output:}
- \texttt{True} if the board is valid, otherwise \texttt{False}.

---

\textbf{Example 1:}
\begin{verbatim}
Input: board = [
    ["5", "3", ".", ".", "7", ".", ".", ".", "."],
    ["6", ".", ".", "1", "9", "5", ".", ".", "."],
    [".", "9", "8", ".", ".", ".", ".", "6", "."],
    ["8", ".", ".", ".", "6", ".", ".", ".", "3"],
    ["4", ".", ".", "8", ".", "3", ".", ".", "1"],
    ["7", ".", ".", ".", "2", ".", ".", ".", "6"],
    [".", "6", ".", ".", ".", ".", "2", "8", "."],
    [".", ".", ".", "4", "1", "9", ".", ".", "5"],
    [".", ".", ".", ".", "8", ".", ".", "7", "9"]
]
Output: True
\end{verbatim}

\textbf{Example 2:}
\begin{verbatim}
Input: board = [
    ["8", "3", ".", ".", "7", ".", ".", ".", "."],
    ["6", ".", ".", "1", "9", "5", ".", ".", "."],
    [".", "9", "8", ".", ".", ".", ".", "6", "."],
    ["8", ".", ".", ".", "6", ".", ".", ".", "3"],
    ["4", ".", ".", "8", ".", "3", ".", ".", "1"],
    ["7", ".", ".", ".", "2", ".", ".", ".", "6"],
    [".", "6", ".", ".", ".", ".", "2", "8", "."],
    [".", ".", ".", "4", "1", "9", ".", ".", "5"],
    [".", ".", ".", ".", "8", ".", ".", "7", "9"]
]
Output: False
Explanation: The top-left 3x3 sub-box contains two '8's.
\end{verbatim}

---

\section*{Algorithmic Approach}
The problem is solved by validating the rows, columns, and \(3 \times 3\) sub-boxes of the board:
1. Use three data structures:
   - A list of sets for rows.
   - A list of sets for columns.
   - A dictionary of sets for \(3 \times 3\) sub-boxes.
2. Traverse each cell in the board:
   - Skip empty cells (\texttt{'.'}).
   - Check if the number is already present in the respective row, column, or sub-box.
   - If valid, add the number to the respective row, column, and sub-box sets.
3. Return \texttt{False} if a duplicate is found; otherwise, return \texttt{True}.

---

\subsection*{Complexities}
1. **Time Complexity:** \(O(81) = O(1)\), since the board size is fixed at \(9 \times 9\).
2. **Space Complexity:** \(O(27)\) for the row, column, and sub-box sets.

---

\section*{Python Implementation}
\begin{fullwidth}
\begin{lstlisting}[language=Python]
def isValidSudoku(board):
    rows = [set() for _ in range(9)]
    cols = [set() for _ in range(9)]
    boxes = {}

    for i in range(9):
        for j in range(9):
            num = board[i][j]
            if num == '.':
                continue

            # Check the row
            if num in rows[i]:
                return False
            rows[i].add(num)

            # Check the column
            if num in cols[j]:
                return False
            cols[j].add(num)

            # Check the 3x3 sub-box
            box_index = (i // 3, j // 3)
            if box_index not in boxes:
                boxes[box_index] = set()
            if num in boxes[box_index]:
                return False
            boxes[box_index].add(num)

    return True

# Example usage:
board = [
    ["5", "3", ".", ".", "7", ".", ".", ".", "."],
    ["6", ".", ".", "1", "9", "5", ".", ".", "."],
    [".", "9", "8", ".", ".", ".", ".", "6", "."],
    ["8", ".", ".", ".", "6", ".", ".", ".", "3"],
    ["4", ".", ".", "8", ".", "3", ".", ".", "1"],
    ["7", ".", ".", ".", "2", ".", ".", ".", "6"],
    [".", "6", ".", ".", ".", ".", "2", "8", "."],
    [".", ".", ".", "4", "1", "9", ".", ".", "5"],
    [".", ".", ".", ".", "8", ".", ".", "7", "9"]
]
print(isValidSudoku(board))  # Output: True
\end{lstlisting}
\end{fullwidth}

---

\section*{Why This Approach?}
This approach ensures all rows, columns, and sub-boxes are validated in a single traversal of the board. Using sets guarantees efficient membership checking and insertion, making it suitable for this constraint-validation problem.

---

\section*{Alternative Approaches}
1. **Bitmasking:** Use bitmasks instead of sets to track the presence of digits, reducing space usage but increasing complexity.
2. **Naive Method:** Traverse rows, columns, and sub-boxes separately. While simple, this method involves redundant checks and is less efficient.

---

\section*{Similar Problems}
1. **Sudoku Solver:** Solve a given Sudoku puzzle by filling in empty cells.
2. **Spiral Matrix:** Traverse a matrix in a spiral order.
3. **Set Matrix Zeroes:** Modify a matrix based on cell values.

---

\section*{Corner Cases to Test}
1. A completely empty board: \( \texttt{board} = [["."] * 9 for \_ in range(9)] \).
2. A completely filled valid board.
3. A board with duplicates in a row, column, or sub-box.

---

\section*{Conclusion}
The **Valid Sudoku** problem emphasizes efficient grid traversal and constraint validation. By leveraging sets and systematic checks, the solution is both elegant and efficient, adhering to the problem's requirements.