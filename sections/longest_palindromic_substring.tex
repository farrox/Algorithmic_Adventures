% Filename: longest_palindromic_substring.tex

\problemsection{Longest Palindromic Substring}
\label{problem:Longest_Palindromic_Substring}

The **Longest Palindromic Substring** problem is a classic challenge in string manipulation that requires finding the longest contiguous substring in a given string that reads the same backward as forward. It serves as an excellent introduction to dynamic programming and center-expansion techniques for substring problems.

\subsection*{Problem Statement}
Given a string \texttt{s}, return the longest palindromic substring in \texttt{s}.

\textbf{Input:}
- A string \texttt{s}.

\textbf{Output:}
- A string representing the longest palindromic substring in \texttt{s}.

\textbf{Example 1:}
\begin{verbatim}
Input: s = "babad"
Output: "bab"
\end{verbatim}

\textbf{Example 2:}
\begin{verbatim}
Input: s = "cbbd"
Output: "bb"
\end{verbatim}

\textbf{Example 3:}
\begin{verbatim}
Input: s = "a"
Output: "a"
\end{verbatim}

\subsection*{Algorithmic Approaches}
Several approaches can be used to solve the problem efficiently:

\subsection*{1. Dynamic Programming}
Dynamic programming tracks whether substrings are palindromic and builds the solution iteratively. A 2D boolean array \texttt{dp} is used where \texttt{dp[i][j]} is \texttt{true} if the substring \texttt{s[i:j+1]} is a palindrome.

\textbf{Steps:}
\begin{itemize}
    \item Initialize \texttt{dp} where \texttt{dp[i][i]} is \texttt{true} for all \(i\) (single characters are palindromes).
    \item Fill \texttt{dp} for substrings of length 2 and greater using the relation:
    \[
    \texttt{dp[i][j]} = (\texttt{s[i] == s[j]} \, \texttt{and} \, \texttt{dp[i+1][j-1]})
    \]
    \item Track the maximum palindrome length and its starting index.
\end{itemize}

\textbf{Complexity:}
\begin{itemize}
    \item \textbf{Time Complexity:} \(O(n^2)\), where \(n\) is the length of the string.
    \item \textbf{Space Complexity:} \(O(n^2)\), due to the 2D array.
\end{itemize}

\subsection*{2. Center Expansion (Optimal)}
Palindromes can be expanded from their center. Each character (and between characters) is considered a potential center.

\textbf{Steps:}
\begin{itemize}
    \item Iterate over each character and pair of characters as potential centers.
    \item Expand outward while the characters on both sides match.
    \item Track the longest palindrome found during expansion.
\end{itemize}

\textbf{Complexity:}
\begin{itemize}
    \item \textbf{Time Complexity:} \(O(n^2)\), as each character is a center and expansion takes \(O(n)\) in total.
    \item \textbf{Space Complexity:} \(O(1)\), as no additional data structures are needed.
\end{itemize}

\subsection*{Python Implementation}
Below is the implementation of the center expansion approach:

\begin{fullwidth}
\begin{lstlisting}[language=Python]
def longestPalindrome(s: str) -> str:
    def expandAroundCenter(left: int, right: int) -> str:
        while left >= 0 and right < len(s) and s[left] == s[right]:
            left -= 1
            right += 1
        return s[left + 1:right]
    
    longest = ""
    for i in range(len(s)):
        # Odd-length palindromes
        palindrome1 = expandAroundCenter(i, i)
        # Even-length palindromes
        palindrome2 = expandAroundCenter(i, i + 1)
        # Update longest palindrome
        longest = max(longest, palindrome1, palindrome2, key=len)
    
    return longest

# Example usage:
s = "babad"
print(longestPalindrome(s))  # Output: "bab" or "aba"
\end{lstlisting}
\end{fullwidth}

\subsection*{Why This Approach?}
The center expansion approach is preferred for its simplicity and \(O(1)\) space complexity. By focusing only on expanding potential centers, it avoids the overhead of maintaining a DP table while achieving the same time complexity.

\subsection*{Alternative Approaches}
\begin{itemize}
    \item **Manacher's Algorithm:**  
    An advanced algorithm that reduces the time complexity to \(O(n)\) by transforming the string and calculating palindrome radii. However, it is more complex and rarely required for practical applications.
    \item **Brute Force:**  
    Check all substrings to find the longest palindrome. This has \(O(n^3)\) time complexity and is impractical for larger strings.
\end{itemize}

\subsection*{Similar Problems}
\begin{itemize}
    \item \textbf{Palindrome Partitioning:} Partition a string into the minimum number of palindromic substrings.
    \item \textbf{Count Substrings That Are Palindromes:} Count all substrings of a string that are palindromic.
    \item \textbf{Longest Palindromic Subsequence:} Find the longest subsequence (not necessarily contiguous) that is palindromic.
\end{itemize}

\subsection*{Things to Keep in Mind and Tricks}
\begin{itemize}
    \item Single characters are always palindromes.
    \item Even-length and odd-length palindromes must be treated separately in center expansion.
    \item Use the \(O(1)\) space solution (center expansion) unless specifically required to use DP or a more complex method like Manacher's Algorithm.
\end{itemize}

\subsection*{Corner and Special Cases to Test}
\begin{itemize}
    \item **Empty String:** Input: \(s = ""\), Output: \( "" \).
    \item **Single Character:** Input: \(s = "a"\), Output: \( "a" \).
    \item **All Characters Same:** Input: \(s = "aaaa"\), Output: \( "aaaa" \).
    \item **No Palindrome Longer Than One Character:** Input: \(s = "abcd"\), Output: \( "a" \) (or any single character).
\end{itemize}

\subsection*{Conclusion}
The **Longest Palindromic Substring** problem highlights the power of efficient substring manipulation techniques such as dynamic programming and center expansion. Mastery of these methods enables you to solve a wide range of string-related problems efficiently and effectively.