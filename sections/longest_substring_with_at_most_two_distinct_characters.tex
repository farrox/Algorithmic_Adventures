% Filename: longest_substring_with_at_most_two_distinct_characters.tex

\problemsection{Longest Substring with At Most Two Distinct Characters}
\label{chap:longest_substring_with_at_most_two_distinct_characters}
\marginnote{This problem exemplifies the sliding window technique, a powerful method for handling substring constraints efficiently.}

The \textbf{Longest Substring with At Most Two Distinct Characters} problem requires finding the length of the longest substring in a given string that contains no more than two distinct characters. This problem effectively demonstrates the application of the sliding window technique to manage dynamic constraints within substrings.

\section*{Problem Statement}

Given a string \texttt{s}, find the length of the longest substring that contains at most two distinct characters.

\textbf{Input:} A string \texttt{s}.

\textbf{Output:} The length of the longest substring containing at most two distinct characters.

\textbf{Examples:}

\begin{itemize}
    \item \textbf{Example 1:}
    \begin{verbatim}
    Input: s = "eceba"
    Output: 3
    Explanation: The substring is "ece" with length 3.
    \end{verbatim}

    \item \textbf{Example 2:}
    \begin{verbatim}
    Input: s = "ccaabbb"
    Output: 5
    Explanation: The substring is "aabbb" with length 5.
    \end{verbatim}
\end{itemize}

LeetCode link: \href{https://leetcode.com/problems/longest-substring-with-at-most-two-distinct-characters/}{Longest Substring with At Most Two Distinct Characters}\index{LeetCode}

\marginnote{\href{https://leetcode.com/problems/longest-substring-with-at-most-two-distinct-characters/}{[LeetCode Link]}\index{LeetCode}}
\marginnote{\href{https://www.geeksforgeeks.org/longest-substring-containing-two-unique-characters/}{[GeeksForGeeks Link]}\index{GeeksForGeeks}}
\marginnote{\href{https://www.hackerrank.com/challenges/longest-substring/problem}{[HackerRank Link]}\index{HackerRank}}
\marginnote{\href{https://app.codesignal.com/challenges/longest-substring-two-distinct}{[CodeSignal Link]}\index{CodeSignal}}
\marginnote{\href{https://www.interviewbit.com/problems/longest-substring-with-at-most-two-distinct-characters/}{[InterviewBit Link]}\index{InterviewBit}}
\marginnote{\href{https://www.educative.io/courses/grokking-the-coding-interview/RM8y8Y3nLdY}{[Educative Link]}\index{Educative}}
\marginnote{\href{https://www.codewars.com/kata/longest-substring-with-at-most-two-distinct-characters/train/python}{[Codewars Link]}\index{Codewars}}

\section*{Algorithmic Approach}

To solve this problem, we utilize the sliding window technique. This method involves maintaining a window that can expand and contract based on the number of distinct characters it contains. By using a hash map to keep track of character frequencies within the window, we can efficiently manage and update the window's constraints.

\begin{enumerate}
    \item \textbf{Initialize Pointers and Data Structures:}
    \begin{itemize}
        \item `start` and `end` pointers to define the current window.
        \item A hash map `char-count` to store the frequency of each character within the window.
    \end{itemize}
    
    \item \textbf{Expand the Window:}
    \begin{itemize}
        \item Increment the `end` pointer to include a new character.
        \item Update the frequency of the new character in `char-count`.
    \end{itemize}
    
    \item \textbf{Contract the Window When Necessary:}
    \begin{itemize}
        \item If the number of distinct characters in `char-count` exceeds two, increment the `start` pointer to exclude characters from the beginning of the window.
        \item Update the frequency of the excluded character in `char-count`. If its frequency drops to zero, remove it from the map.
    \end{itemize}
    
    \item \textbf{Update the Maximum Length:}
    \begin{itemize}
        \item After each expansion (and possible contraction), update the `max-length` if the current window size is larger than the previously recorded maximum.
    \end{itemize}
\end{enumerate}

This approach ensures that we traverse the string only once, achieving an optimal \(O(n)\) time complexity.

\marginnote{The sliding window technique is particularly effective for problems involving contiguous sequences with dynamic constraints.}

\section*{Complexities}

\begin{itemize}
    \item \textbf{Time Complexity:} \(O(n)\), where \(n\) is the length of the string. Each character is processed at most twice (once by `end` and once by `start`).
    \item \textbf{Space Complexity:} \(O(1)\), since the hash map `char-count` stores at most three distinct characters at any given time (to account for the condition before contraction).
\end{itemize}

\section*{Python Implementation}

\marginnote{Implementing the sliding window requires careful management of pointers and character frequencies to ensure efficiency.}

Below is the complete Python code for solving the "Longest Substring with At Most Two Distinct Characters" problem using the sliding window technique:

\begin{fullwidth}
\begin{lstlisting}[language=Python]
def length_of_longest_substring_two_distinct(s):
    from collections import defaultdict
    
    char_count = defaultdict(int)
    max_length = 0
    start = 0
    
    for end in range(len(s)):
        char_count[s[end]] += 1
        
        # Shrink the window until we have at most two distinct characters
        while len(char_count) > 2:
            char_count[s[start]] -= 1
            if char_count[s[start]] == 0:
                del char_count[s[start]]
            start += 1
        
        # Update max_length if the current window is larger
        current_window_length = end - start + 1
        if current_window_length > max_length:
            max_length = current_window_length
    
    return max_length

# Example usage:
# s = "eceba"
# print(length_of_longest_substring_two_distinct(s))  # Output: 3
# s = "ccaabbb"
# print(length_of_longest_substring_two_distinct(s))  # Output: 5
\end{lstlisting}
\end{fullwidth}

This implementation maintains a hash map `char-count` to record the frequency of each character within the current window defined by `start` and `end` pointers. By expanding the window and adjusting the `start` pointer as needed, the algorithm efficiently finds the longest valid substring.

\section*{Explanation}

The `length-of-longest-substring-two-distinct` function efficiently finds the longest substring with at most two distinct characters by leveraging the sliding window technique. Here's a detailed breakdown of the implementation:

\begin{itemize}
    \item \textbf{Initialization:}
    \begin{itemize}
        \item Hash Map: `char-count` is a `defaultdict` that keeps track of the frequency of each character within the current window.
        \item Pointers: `start` marks the beginning of the window, while `end` is the current position being evaluated.
        \item Maximum Length: `max-length` stores the length of the longest valid substring found.
    \end{itemize}
    
    \item \textbf{Iteration:}
    \begin{itemize}
        \item Expanding the Window: For each character `s[end]`, increment its count in `char-count`.
        \item Checking Constraints: If the number of distinct characters in `char-count` exceeds two, enter a loop to shrink the window from the start.
            \begin{itemize}
                \item Decrement the count of `s[start]`.
                \item If the count of `s[start]` drops to zero, remove it from `char-count`.
                \item Increment `start` to move the window forward.
            \end{itemize}
        \item Updating Maximum Length: After ensuring the window contains at most two distinct characters, calculate the current window length (`end - start + 1`) and update `max-length` if necessary.
    \end{itemize}
    
    \item \textbf{Result:}
    \begin{itemize}
        \item After iterating through the entire string, return `max-length` as the length of the longest valid substring.
    \end{itemize}
\end{itemize}

\section*{Why This Approach}

The sliding window technique is chosen for its efficiency in handling dynamic constraints within contiguous sequences. By maintaining a window that adapts based on the number of distinct characters, the algorithm ensures that each character is processed only a constant number of times, achieving optimal \(O(n)\) time complexity.

\section*{Alternative Approaches}

An alternative approach could involve brute force, where all possible substrings are examined, and the number of distinct characters is counted for each. However, this method would have a time complexity of \(O(n^3)\), making it impractical for longer strings.

Another approach could use two pointers without a hash map, but managing the counts and ensuring at most two distinct characters would become cumbersome and less efficient.

The sliding window technique remains the most efficient and straightforward method for this problem.

\section*{Similar Problems to This One}

There are several other problems that involve finding substrings or sequences with specific constraints, such as:

\begin{itemize}
    \item \hyperref[problem:longest_substring_with_at_most_k_distinct_characters]{Longest Substring with At Most K Distinct Characters}\index{Longest Substring with At Most K Distinct Characters}
    \item \hyperref[problem:longest_substring_without_repeating_characters]{Longest Substring Without Repeating Characters}\index{Longest Substring Without Repeating Characters}
    \item \hyperref[problem:minimum_window_substring]{Minimum Window Substring}\index{Minimum Window Substring}
    \item \hyperref[problem:longest_repeating_character_replacement]{Longest Repeating Character Replacement}\index{Longest Repeating Character Replacement}
\end{itemize}

\section*{Things to Keep in Mind and Tricks}

\begin{itemize}
    \item \textbf{Sliding Window Technique:} Utilize two pointers to define the current window and adjust them based on constraints.
    \index{Sliding Window Technique}
    
    \item \textbf{Hash Map for Frequency Counts:} Use a hash map to keep track of character frequencies within the window for efficient constraint checking.
    \index{Hash Map for Frequency Counts}
    
    \item \textbf{Dynamic Window Adjustment:} Expand the window by moving the `end` pointer and contract it by moving the `start` pointer when constraints are violated.
    \index{Dynamic Window Adjustment}
    
    \item \textbf{Edge Case Handling:} Always consider edge cases such as empty strings, single-character strings, and strings with all identical characters.
    \index{Edge Case Handling}
    
    \item \textbf{Optimizing Updates:} Update the maximum length only after ensuring the current window meets the constraints to avoid unnecessary calculations.
    \index{Optimizing Updates}
\end{itemize}

\section*{Corner and Special Cases to Test When Writing the Code}

When implementing the `length-of-longest-substring-two-distinct` function, it is crucial to test the following edge cases to ensure robustness:

\begin{itemize}
    \item \textbf{Empty String:} `s = ""` should return `0`.
    \index{Corner Cases}
    
    \item \textbf{Single Character:} `s = "a"` should return `1`.
    \index{Corner Cases}
    
    \item \textbf{All Characters the Same:} `s = "aaaaa"` should return `5`.
    \index{Corner Cases}
    
    \item \textbf{All Characters Unique:} `s = "abcdef"` should return `2`.
    \index{Corner Cases}
    
    \item \textbf{Multiple Valid Substrings:} `s = "abaccc"` should return `4` for the substring `"accc"`.
    \index{Corner Cases}
    
    \item \textbf{Strings with Exactly Two Distinct Characters:} `s = "aabbcc"` should return `4` for substrings like `"aabb"` or `"bbcc"`.
    \index{Corner Cases}
    
    \item \textbf{Large Input Size:} Test with a very large string to ensure that the implementation performs efficiently without exceeding memory limits.
    \index{Corner Cases}
\end{itemize}

\printindex