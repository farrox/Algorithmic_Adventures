% Filename: insert_interval.tex

\problemsection{Insert Interval}\marginpar{This problem tests interval insertion and merging techniques.}

\textbf{Problem Description:}

Given a list of non-overlapping intervals sorted by their start times, where each interval is represented as \texttt{intervals[i] = [start\_i, end\_i]}, and a new interval \texttt{newInterval = [start, end]}, insert \texttt{newInterval} into the list of intervals. Ensure that the list remains sorted by start times and that all overlapping intervals are merged appropriately.

\textbf{Example 1:}

\begin{itemize}
    \item \textbf{Input:} \texttt{intervals = [[1,3],[6,9]]}, \texttt{newInterval = [2,5]}
    \item \textbf{Output:} \texttt{[[1,5],[6,9]]}
    \item \textbf{Explanation:} The new interval \texttt{[2,5]} overlaps with \texttt{[1,3]}, merging to \texttt{[1,5]}.
\end{itemize}

\textbf{Example 2:}

\begin{itemize}
    \item \textbf{Input:} \texttt{intervals = [[1,2],[3,5],[6,7],[8,10],[12,16]]}, \texttt{newInterval = [4,8]}
    \item \textbf{Output:} \texttt{[[1,2],[3,10],[12,16]]}
    \item \textbf{Explanation:} The new interval \texttt{[4,8]} overlaps with \texttt{[3,5],[6,7],[8,10]}, merging to \texttt{[3,10]}.
\end{itemize}

\textbf{Solution Overview:}

The key idea is to iterate through the intervals and:

\begin{enumerate}
    \item Add all intervals that end before the start of the new interval to the output list.
    \item Merge all intervals that overlap with the new interval by updating the start and end of the new interval.
    \item Add the merged new interval to the output list.
    \item Add the remaining intervals that start after the end of the new interval to the output list.
\end{enumerate}

\textbf{Implementation Details:}

\begin{fullwidth}
\begin{lstlisting}[language=Python]
def insert(intervals, newInterval):
    output = []
    i = 0
    n = len(intervals)
    
    # Add all intervals ending before newInterval starts
    while i < n and intervals[i][1] < newInterval[0]:
        output.append(intervals[i])
        i += 1
    
    # Merge overlapping intervals
    while i < n and intervals[i][0] <= newInterval[1]:
        newInterval[0] = min(newInterval[0], intervals[i][0])
        newInterval[1] = max(newInterval[1], intervals[i][1])
        i += 1
    output.append(newInterval)
    
    # Add remaining intervals
    while i < n:
        output.append(intervals[i])
        i += 1
    
    return output

# Example usage:
print(insert([[1,3],[6,9]], [2,5]))  # Output: [[1,5],[6,9]]
print(insert([[1,2],[3,5],[6,7],[8,10],[12,16]], [4,8]))  # Output: [[1,2],[3,10],[12,16]]
\end{lstlisting}
\end{fullwidth}

\textbf{Complexities:}

\begin{itemize}
    \item \textbf{Time Complexity:} \(O(n)\), where \(n\) is the number of intervals.
    \item \textbf{Space Complexity:} \(O(n)\), for the output list.
\end{itemize}

\textbf{Corner Cases to Test:}

\begin{itemize}
    \item Empty intervals list.
    \item New interval does not overlap with any existing intervals.
    \item New interval overlaps with all intervals.
    \item New interval is before all existing intervals.
    \item New interval is after all existing intervals.
\end{itemize}

\marginpar{Handling intervals efficiently is crucial for scheduling and timeline-based problems.}