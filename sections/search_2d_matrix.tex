% Filename: search_2d_matrix.tex

\problemsection{Search a 2D Matrix}
\label{problem:Search_2D_Matrix}

The **Search a 2D Matrix** problem demonstrates the effective use of binary search in multidimensional arrays. The task is to efficiently determine if a target value exists within a matrix that is row and column-wise sorted.

\section*{Problem Statement}
You are given an \(m \times n\) matrix where:
- Each row is sorted in ascending order.
- The first integer of each row is greater than the last integer of the previous row.

Write a function to determine if a given target exists in the matrix. Return \texttt{true} if the target exists, and \texttt{false} otherwise.

\textbf{Input:}
- \texttt{matrix}: A list of lists representing an \(m \times n\) matrix.
- \texttt{target}: An integer to search for.

\textbf{Output:}
- A boolean indicating whether the \texttt{target} exists in the matrix.

\textbf{Example 1:}
\begin{verbatim}
Input: matrix = [
    [1, 3, 5, 7],
    [10, 11, 16, 20],
    [23, 30, 34, 60]
], target = 3
Output: true
\end{verbatim}

\textbf{Example 2:}
\begin{verbatim}
Input: matrix = [
    [1, 3, 5, 7],
    [10, 11, 16, 20],
    [23, 30, 34, 60]
], target = 13
Output: false
\end{verbatim}

\textbf{Constraints:}
- \(m == \texttt{matrix.length}\)
- \(n == \texttt{matrix[0].length}\)
- \(1 \leq m, n \leq 100\)
- \(-10^4 \leq \texttt{matrix[i][j]}, \texttt{target} \leq 10^4\)

---

\section*{Algorithmic Approach}
This problem can be solved by flattening the \(m \times n\) matrix into a single sorted array and applying binary search.

\subsection*{Steps for Binary Search on 2D Matrix}
1. Treat the matrix as a 1D sorted array of size \(m \times n\).
2. Use binary search:
    - The middle element in this conceptual 1D array corresponds to:
      \[
      \texttt{matrix[mid // n][mid \% n]}
      \]
      where \(mid\) is the index in the 1D array.
3. Compare the middle element with the target:
    - If it matches, return \texttt{true}.
    - If it is greater, search the left half.
    - If it is smaller, search the right half.
4. Continue until the search space is exhausted.

\subsection*{Complexities}
- **Time Complexity:** \(O(\log(m \times n))\), where \(m\) and \(n\) are the matrix dimensions.
- **Space Complexity:** \(O(1)\), as the algorithm uses no additional space beyond variables.

---

\section*{Python Implementation}
\begin{fullwidth}
\begin{lstlisting}[language=Python]
def searchMatrix(matrix, target):
    if not matrix or not matrix[0]:
        return False
    
    m, n = len(matrix), len(matrix[0])
    left, right = 0, m * n - 1
    
    while left <= right:
        mid = left + (right - left) // 2
        mid_element = matrix[mid // n][mid % n]
        
        if mid_element == target:
            return True
        elif mid_element < target:
            left = mid + 1
        else:
            right = mid - 1
    
    return False

# Example usage:
matrix = [
    [1, 3, 5, 7],
    [10, 11, 16, 20],
    [23, 30, 34, 60]
]
target = 3
print(searchMatrix(matrix, target))  # Output: true
\end{lstlisting}
\end{fullwidth}

---

\section*{Why This Approach?}
Binary search is optimal for sorted data due to its logarithmic time complexity. By treating the 2D matrix as a virtual 1D array, we simplify the problem to a standard binary search while maintaining efficiency.

---

\section*{Alternative Approach}
For a different matrix structure where rows and columns are independently sorted but not sequential:
- Start at the top-right corner.
- Move left if the current element is greater than the target.
- Move down if the current element is less than the target.
- This approach has \(O(m + n)\) complexity and works for matrices sorted row-wise and column-wise but not in blocks.

---

\section*{Similar Problems}
\begin{itemize}
    \item **Search a 2D Matrix II:** Locate a target in a matrix sorted row-wise and column-wise, but without sequential row-column relationships.
    \item **Binary Search:** Core algorithm applied here.
    \item **Find Element in Rotated Sorted Array:** Similar in logic but involves rotated sorted data.
\end{itemize}

---

\section*{Corner Cases to Test}
\begin{itemize}
    \item Empty matrix: \(\texttt{matrix} = []\).
    \item Single element matrix: \(\texttt{matrix} = [[5]], \texttt{target} = 5\).
    \item Large \(m\) and \(n\): Test performance for \(m, n = 100\).
    \item Target is the smallest or largest element in the matrix.
\end{itemize}

---

\section*{Conclusion}
The Search a 2D Matrix problem demonstrates the utility of binary search for structured data. By mapping a 2D matrix to a 1D search space, it ensures efficiency and simplicity, making it an essential technique for optimizing search operations.