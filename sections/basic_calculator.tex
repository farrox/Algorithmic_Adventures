% Filename: basic_calculator.tex

\problemsection{Basic Calculator}\marginpar{This problem involves parsing and evaluating arithmetic expressions with parentheses using stacks.}

\textbf{Problem Description:}  
Implement a basic calculator to evaluate a simple expression string.

The expression string may contain open \texttt{'('} and closing parentheses \texttt{')'}, the plus \texttt{'+'} or minus sign \texttt{'-'}, non-negative integers, and empty spaces \texttt{' '}. The expression string represents a valid arithmetic expression.

You may assume that the given expression is always valid.

\textbf{Example 1:}

\begin{itemize}
    \item \textbf{Input:} \texttt{"1 + 1"}
    \item \textbf{Output:} \texttt{2}
\end{itemize}

\textbf{Example 2:}

\begin{itemize}
    \item \textbf{Input:} \texttt{" 2-1 + 2 "}
    \item \textbf{Output:} \texttt{3}
\end{itemize}

\textbf{Example 3:}

\begin{itemize}
    \item \textbf{Input:} \texttt{"(1+(4+5+2)-3)+(6+8)"}
    \item \textbf{Output:} \texttt{23}
\end{itemize}

\textbf{Solution Overview:}  
Use a stack to evaluate the expression by handling numbers, signs, and parentheses. Iterate through the string character by character:

\begin{itemize}
    \item If the character is a digit, build the current number.
    \item If the character is '+', add the current number to the result with the positive sign.
    \item If the character is '-', subtract the current number from the result.
    \item If the character is '(', push the current result and sign onto the stack and reset them.
    \item If the character is ')', compute the result inside the parentheses and combine it with the top values from the stack.
\end{itemize}

\begin{fullwidth}
\begin{lstlisting}[language=Python]
def calculate(s):
    stack = []
    result = 0
    number = 0
    sign = 1  # 1 for '+', -1 for '-'
    i = 0
    while i < len(s):
        char = s[i]
        if char.isdigit():
            number = number * 10 + int(char)
        elif char == '+':
            result += sign * number
            number = 0
            sign = 1
        elif char == '-':
            result += sign * number
            number = 0
            sign = -1
        elif char == '(':
            # Push current result and sign onto the stack
            stack.append(result)
            stack.append(sign)
            # Reset result and sign
            result = 0
            sign = 1
        elif char == ')':
            result += sign * number
            number = 0
            # Apply the sign before the parentheses
            result *= stack.pop()
            # Add to the result before the parentheses
            result += stack.pop()
        i += 1
    if number != 0:
        result += sign * number
    return result

# Example usage:
print(calculate("1 + 1"))                  # Output: 2
print(calculate(" 2-1 + 2 "))              # Output: 3
print(calculate("(1+(4+5+2)-3)+(6+8)"))    # Output: 23
\end{lstlisting}
\end{fullwidth}