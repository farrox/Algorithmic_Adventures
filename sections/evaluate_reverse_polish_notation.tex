% Filename: evaluate_reverse_polish_notation.tex

\problemsection{Evaluate Reverse Polish Notation}\marginpar{This problem demonstrates the use of stacks in evaluating expressions in postfix notation.}

\textbf{Problem Description:}  
Evaluate the value of an arithmetic expression in Reverse Polish Notation (RPN). Valid operators are \texttt{'+'}, \texttt{'-'}, \texttt{'*'}, and \texttt{'/'}. Each operand may be an integer or another expression.

\textbf{Notes:}
\begin{itemize}
    \item Division between two integers should truncate toward zero.
    \item The given RPN expression is always valid.
    \item The length of the tokens array will be at least 1.
    \item The answer and all intermediate calculations are guaranteed to be in the range of a 32-bit signed integer.
\end{itemize}

\textbf{Solution Overview:}  
Use a stack to store operands. Iterate through the tokens:
\begin{itemize}
    \item If the token is an operand (number), push it onto the stack.
    \item If the token is an operator, pop the top two operands from the stack, apply the operator, and push the result back onto the stack.
\end{itemize}
At the end, the value on the top of the stack is the result of the expression.

\begin{fullwidth}
\begin{lstlisting}[language=Python]
def evalRPN(tokens):
    stack = []
    operators = {'+', '-', '*', '/'}
    for token in tokens:
        if token in operators:
            b = int(stack.pop())
            a = int(stack.pop())
            if token == '+':
                result = a + b
            elif token == '-':
                result = a - b
            elif token == '*':
                result = a * b
            else:  # token == '/'
                # Ensure truncation toward zero
                result = int(a / b)
            stack.append(result)
        else:
            stack.append(int(token))
    return stack.pop()

# Example usage:
print(evalRPN(["2", "1", "+", "3", "*"]))    # Output: 9
print(evalRPN(["4", "13", "5", "/", "+"]))   # Output: 6
print(evalRPN([
    "10", "6", "9", "3", "+", "-11", "*",
    "/", "*", "17", "+", "5", "+"]))         # Output: 22
\end{lstlisting}
\end{fullwidth}