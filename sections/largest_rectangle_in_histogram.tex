% Filename: largest_rectangle_in_histogram.tex

\problemsection{Largest Rectangle in Histogram}\marginpar{This problem uses a stack to find the largest rectangle in a histogram.}

\textbf{Problem Description:}

Given an array of integers \texttt{heights} representing the histogram's bar heights where the width of each bar is 1, return the area of the largest rectangle in the histogram.

\textbf{Example 1:}

\begin{itemize}
    \item \textbf{Input:} \texttt{heights = [2,1,5,6,2,3]}
    \item \textbf{Output:} \texttt{10}
    \item \textbf{Explanation:} The largest rectangle has an area of 10 units, formed by the bars of heights 5 and 6.
\end{itemize}

\textbf{Example 2:}

\begin{itemize}
    \item \textbf{Input:} \texttt{heights = [2,4]}
    \item \textbf{Output:} \texttt{4}
\end{itemize}

\textbf{Solution Overview:}

Use a stack to keep track of indices of bars in the histogram. Iterate through the array:

\begin{itemize}
    \item For each bar at index \( i \):
        \begin{itemize}
            \item If the stack is empty or the current bar is taller than the bar at the stack's top index, push \( i \) onto the stack.
            \item Otherwise, while the current bar is shorter than the bar at the stack's top index:
                \begin{itemize}
                    \item Pop the top index from the stack.
                    \item Calculate the area with the popped bar as the smallest bar.
                    \item Update the maximum area if necessary.
                \end{itemize}
            \item Repeat until the stack is empty or the current bar is taller than the bar at the stack's top index.
        \end{itemize}
\end{itemize}

After processing all bars, pop any remaining bars from the stack and calculate areas similarly.

\textbf{Code Implementation:}

\begin{lstlisting}[language=Python]
def largestRectangleArea(heights):
    stack = []  # Stack to store indices
    max_area = 0
    i = 0
    while i <= len(heights):
        # Use 0 height for the bar beyond the last one
        h = 0 if i == len(heights) else heights[i]
        if not stack or h >= heights[stack[-1]]:
            stack.append(i)
            i += 1
        else:
            top = stack.pop()
            width = i if not stack else i - stack[-1] - 1
            area = heights[top] * width
            max_area = max(max_area, area)
    return max_area

# Example usage:
print(largestRectangleArea([2,1,5,6,2,3]))  # Output: 10
print(largestRectangleArea([2,4]))          # Output: 4
\end{lstlisting}