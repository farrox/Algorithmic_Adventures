% Filename: substring_with_concatenation_of_all_words.tex

\problemsection{Substring with Concatenation of All Words}
\label{chap:substring_with_concatenation_of_all_words}
\marginnote{This problem effectively utilizes the sliding window technique combined with hash maps to manage word frequencies within substrings.}

The \textbf{Substring with Concatenation of All Words} problem involves finding all starting indices of substring(s) in a given string \texttt{s} that is a concatenation of each word in a given list \texttt{words} exactly once and without any intervening characters. This problem tests your ability to efficiently manage and traverse substrings while handling dynamic constraints.

\section*{Problem Statement}

Given a string \texttt{s} and an array of strings \texttt{words}, return all starting indices of substring(s) in \texttt{s} that is a concatenation of each word in \texttt{words} exactly once, in any order, and without any intervening characters.

You can return the answer in any order.

\textbf{Note:}
\begin{itemize}
    \item All words in \texttt{words} are of the same length.
    \item The concatenated substring must contain all words exactly once.
\end{itemize}

\textbf{Examples:}

\begin{itemize}
    \item \textbf{Example 1:}
    \begin{verbatim}
    Input: s = "barfoothefoobarman", words = ["foo","bar"]
    Output: [0,9]
    Explanation: Substrings starting at index 0 and 9 are "barfoo" and "foobar" respectively.
    \end{verbatim}

    \item \textbf{Example 2:}
    \begin{verbatim}
    Input: s = "wordgoodgoodgoodbestword", words = ["word","good","best","word"]
    Output: []
    Explanation: There is no substring that contains all words exactly once.
    \end{verbatim}

    \item \textbf{Example 3:}
    \begin{verbatim}
    Input: s = "barfoofoobarthefoobarman", words = ["bar","foo","the"]
    Output: [6,9,12]
    Explanation: Substrings starting at indices 6, 9, and 12 are "foobarthe", "barthefoo", and "thefoobar" respectively.
    \end{verbatim}
\end{itemize}

LeetCode link: \href{https://leetcode.com/problems/substring-with-concatenation-of-all-words/}{Substring with Concatenation of All Words}\index{LeetCode}

\marginnote{\href{https://leetcode.com/problems/substring-with-concatenation-of-all-words/}{[LeetCode Link]}\index{LeetCode}}
\marginnote{\href{https://www.geeksforgeeks.org/find-all-the-starting-indices-of-pattern-in-a-string/}{[GeeksForGeeks Link]}\index{GeeksForGeeks}}
\marginnote{\href{https://www.hackerrank.com/challenges/find-all-substrings-of-a-given-string/problem}{[HackerRank Link]}\index{HackerRank}}
\marginnote{\href{https://app.codesignal.com/challenges/substring-concatenation}{[CodeSignal Link]}\index{CodeSignal}}
\marginnote{\href{https://www.interviewbit.com/problems/subarray-with-equal-0s-and-1s/}{[InterviewBit Link]}\index{InterviewBit}}
\marginnote{\href{https://www.educative.io/courses/grokking-the-coding-interview/RM8y8Y3nLdY}{[Educative Link]}\index{Educative}}
\marginnote{\href{https://www.codewars.com/kata/substring-concatenation/train/python}{[Codewars Link]}\index{Codewars}}

\section*{Algorithmic Approach}

To solve this problem, we utilize the **sliding window** technique combined with **hash maps** to manage word frequencies within substrings. The approach involves the following steps:

\begin{enumerate}
    \item \textbf{Preprocessing:}
    \begin{itemize}
        \item Calculate the length of each word (\texttt{word\_length}).
        \item Determine the number of words (\texttt{num\_words}).
        \item Compute the total length of the concatenated substring (\texttt{total\_length}).
        \item Create a hash map (\texttt{word\_count}) to store the frequency of each word in \texttt{words}.
    \end{itemize}
    
    \item \textbf{Sliding Window Traversal:}
    \begin{itemize}
        \item Iterate through the string \texttt{s} with a window size of \texttt{total\_length}.
        \item For each window, extract substrings of length \texttt{word\_length} and check if they exist in \texttt{word\_count}.
        \item Maintain a temporary hash map (\texttt{seen}) to track the frequency of words within the current window.
        \item If all words match the required frequencies, record the starting index.
    \end{itemize}
    
    \item \textbf{Optimizations:}
    \begin{itemize}
        \item Slide the window in increments of \texttt{word\_length} to align with word boundaries.
        \item Early termination of window checks if a word exceeds the required frequency.
    \end{itemize}
\end{enumerate}

This method ensures that the entire string is traversed efficiently, checking only relevant substrings and maintaining optimal time and space complexities.

\marginnote{Efficient window management and frequency tracking are crucial for maintaining the desired constraints within the sliding window.}

\section*{Complexities}

\begin{itemize}
    \item \textbf{Time Complexity:} \(O(n \times m)\), where \(n\) is the length of the string \texttt{s} and \(m\) is the number of words. This is because we slide the window across the string and, for each window, we check all words.
    \item \textbf{Space Complexity:} \(O(k)\), where \(k\) is the number of unique words in \texttt{words}. This space is used by the hash maps to store word frequencies.
\end{itemize}

\section*{Python Implementation}

\marginnote{Implementing the sliding window with hash maps ensures efficient tracking of word frequencies and window constraints.}

Below is the complete Python code for solving the **"Substring with Concatenation of All Words"** problem using the sliding window technique:

\begin{fullwidth}
\begin{lstlisting}[language=Python]
from typing import List
from collections import defaultdict

def find_substring(s: str, words: List[str]) -> List[int]:
    if not s or not words:
        return []
    
    word_length = len(words[0])
    num_words = len(words)
    total_length = word_length * num_words
    word_count = defaultdict(int)
    
    for word in words:
        word_count[word] += 1
    
    result = []
    
    for i in range(word_length):
        left = i
        count = 0
        seen = defaultdict(int)
        
        for j in range(i, len(s) - word_length + 1, word_length):
            word = s[j:j + word_length]
            if word in word_count:
                seen[word] += 1
                count += 1
                
                while seen[word] > word_count[word]:
                    left_word = s[left:left + word_length]
                    seen[left_word] -= 1
                    count -= 1
                    left += word_length
                
                if count == num_words:
                    result.append(left)
            else:
                seen.clear()
                count = 0
                left = j + word_length
    
    return result

# Example usage:
# s = "barfoothefoobarman"
# words = ["foo","bar"]
# print(find_substring(s, words))  # Output: [0, 9]
# s = "wordgoodgoodgoodbestword"
# words = ["word","good","best","word"]
# print(find_substring(s, words))  # Output: []
\end{lstlisting}
\end{fullwidth}

This implementation leverages the sliding window technique to efficiently traverse the string \texttt{s} while managing word frequencies using hash maps. Here's a step-by-step breakdown:

\section*{Explanation}

The `find-substring` function efficiently identifies all starting indices of substrings in \texttt{s} that are concatenations of each word in \texttt{words} exactly once. Here's a detailed breakdown of the implementation:

\begin{itemize}
    \item \textbf{Initialization:}
    \begin{itemize}
        \item **Edge Cases:** If either \texttt{s} or \texttt{words} is empty, return an empty list.
        \item **Word Metrics:** Calculate the length of each word (\texttt{word\_length}), the number of words (\texttt{num\_words}), and the total length of the concatenated substring (\texttt{total\_length}).
        \item **Word Count Map:** Create a hash map (\texttt{word\_count}) to store the frequency of each word in \texttt{words}.
    \end{itemize}
    
    \item \textbf{Sliding Window Traversal:}
    \begin{itemize}
        \item **Iterate Over Possible Starting Points:** Loop through the string with an offset of \texttt{word\_length} to align with word boundaries.
        \item **Initialize Pointers and Counters:** For each starting offset, initialize `left` to mark the beginning of the window, `count` to track the number of valid words found, and `seen` to count the frequency of words within the current window.
        \item **Expand the Window:** Slide the window by moving the `j` pointer in increments of \texttt{word\_length}.
            \begin{itemize}
                \item Extract the current word from \texttt{s} using slicing.
                \item If the word exists in \texttt{word\_count}, increment its count in \texttt{seen} and the overall `count`.
                \item **Handle Excess Frequencies:** If a word's frequency in \texttt{seen} exceeds its required frequency in \texttt{word\_count}, shrink the window from the left by decrementing the frequency of the leftmost word and moving the `left` pointer forward.
                \item **Record Valid Substrings:** If `count` equals \texttt{num\_words}, append the current `left` index to the result list.
            \end{itemize}
        \item **Reset on Invalid Words:** If the current word is not in \texttt{word\_count}, clear the \texttt{seen} map, reset `count`, and move the `left` pointer beyond the current word.
    \end{itemize}
    
    \item \textbf{Result Compilation:} After traversing the string, return the list of starting indices where valid concatenations are found.
\end{itemize}

\section*{Why This Approach}

This approach is chosen for its efficiency and ability to handle the problem's constraints effectively. By using the sliding window technique:

\begin{itemize}
    \item **Optimal Time Complexity:** The algorithm runs in \(O(n \times m)\) time, where \(n\) is the length of the string and \(m\) is the number of words, ensuring scalability even for large inputs.
    \item **Space Efficiency:** Utilizing hash maps to track word frequencies allows for constant-time look-ups and updates, maintaining a space complexity of \(O(k)\), where \(k\) is the number of unique words.
    \item **Single Pass Traversal:** The sliding window ensures that each character is processed only a limited number of times, avoiding redundant computations.
\end{itemize}

\section*{Alternative Approaches}

An alternative approach to solving this problem is the **Brute Force** method, where you check every possible substring of length \texttt{total\_length} and verify if it contains all the words in \texttt{words} with the correct frequencies. However, this method has a time complexity of \(O(n \times m \times l)\), where \(l\) is the length of each word, making it highly inefficient for large inputs.

Another approach could involve using **Trie Data Structures** to manage word look-ups more efficiently, but integrating it with the sliding window technique complicates the implementation without significant performance gains for this specific problem.

The sliding window technique combined with hash maps remains the most optimal and straightforward method for this problem.

\marginnote{Choosing the right algorithmic strategy is crucial for optimizing both time and space complexities.}

\section*{Similar Problems to This One}

There are several other problems that involve finding substrings or sequences with specific constraints, such as:

\begin{itemize}
    \item \hyperref[problem:longest_substring_with_at_most_k_distinct_characters]{Longest Substring with At Most K Distinct Characters}\index{Longest Substring with At Most K Distinct Characters}
    \item \hyperref[problem:longest_substring_without_repeating_characters]{Longest Substring Without Repeating Characters}\index{Longest Substring Without Repeating Characters}
    \item \hyperref[problem:minimum_window_substring]{Minimum Window Substring}\index{Minimum Window Substring}
    \item \hyperref[problem:longest_repeating_character_replacement]{Longest Repeating Character Replacement}\index{Longest Repeating Character Replacement}
    \item \hyperref[problem:word_break]{Word Break}\index{Word Break}
\end{itemize}

\section*{Things to Keep in Mind and Tricks}

\begin{itemize}
    \item \textbf{Sliding Window Technique:} Utilize two pointers to define the current window and adjust them based on the constraints.
    \index{Sliding Window Technique}
    
    \item \textbf{Hash Map for Frequency Counts:} Use a hash map to keep track of character frequencies within the window for efficient constraint checking.
    \index{Hash Map for Frequency Counts}
    
    \item \textbf{Dynamic Window Adjustment:} Expand the window by moving the `end` pointer and contract it by moving the `start` pointer when constraints are violated.
    \index{Dynamic Window Adjustment}
    
    \item \textbf{Edge Case Handling:} Always consider edge cases such as empty strings, single-character strings, and strings with all identical characters.
    \index{Edge Case Handling}
    
    \item \textbf{Optimizing Updates:} Update the maximum length only after ensuring the current window meets the constraints to avoid unnecessary calculations.
    \index{Optimizing Updates}
\end{itemize}

\section*{Corner and Special Cases to Test When Writing the Code}

When implementing the `find-substring` function, it is crucial to test the following edge cases to ensure robustness:

\begin{itemize}
    \item \textbf{Empty String:} `s = ""`, `words = ["a"]` should return `[]`.
    \index{Corner Cases}
    
    \item \textbf{Single Word:} `s = "a"`, `words = ["a"]` should return `[0]`.
    \index{Corner Cases}
    
    \item \textbf{All Words Present Multiple Times:} `s = "barfoofoobarthefoobarman"`, `words = ["bar","foo","the"]` should return `[6,9,12]`.
    \index{Corner Cases}
    
    \item \textbf{Words with Overlapping:} `s = "wordgoodgoodgoodbestword"`, `words = ["word","good","best","word"]` should return `[]`.
    \index{Corner Cases}
    
    \item \textbf{Words Not Present:} `s = "lingmindraboofooowingdingbarrwingmonkeypoundcake"`, `words = ["fooo","barr","wing","ding","wing"]` should return `[13]`.
    \index{Corner Cases}
    
    \item \textbf{Words with Different Lengths:} Although the problem assumes all words are the same length, test cases with varying lengths should be handled gracefully.
    \index{Corner Cases}
    
    \item \textbf{Large Input Size:} Test with a very large string and a large number of words to ensure that the implementation performs efficiently without exceeding memory limits.
    \index{Corner Cases}
    
    \item \textbf{Identical Words:} `s = "aaaaaa"`, `words = ["aa","aa"]` should return `[0,1,2]`.
    \index{Corner Cases}
\end{itemize}

\printindex