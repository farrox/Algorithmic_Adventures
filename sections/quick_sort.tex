% Filename: quick_sort.tex

\problemsection{Quick Sort}
\label{problem:quick_sort}
\marginnote{Quick Sort is a highly efficient sorting algorithm that employs a divide-and-conquer strategy to sort elements in-place with an average time complexity of \(O(n \log n)\).}
    
The \textbf{Quick Sort} problem involves implementing the Quick Sort algorithm to sort a given array of integers in ascending order. Quick Sort is favored for its average-case efficiency and in-place sorting capabilities, making it suitable for large datasets.

\section*{Problem Statement}
Given an array of integers `nums`, sort the array in ascending order using the Quick Sort algorithm and return the sorted array.

\textbf{Note:} 
\begin{itemize}
    \item You must implement the Quick Sort algorithm; using built-in sorting functions is not allowed.
    \item Aim for a time complexity of \(O(n \log n)\) on average and a space complexity of \(O(\log n)\) due to recursion.
\end{itemize}

\textbf{Example 1:}

\begin{verbatim}
Input: nums = [3,6,8,10,1,2,1]
Output: [1,1,2,3,6,8,10]
Explanation: The array is sorted in ascending order.
\end{verbatim}

\textbf{Example 2:}

\begin{verbatim}
Input: nums = [5,4,3,2,1]
Output: [1,2,3,4,5]
Explanation: The array is sorted in ascending order.
\end{verbatim}

LeetCode link: \href{https://leetcode.com/problems/sort-an-array/}{Sort an Array}\index{LeetCode}

\marginnote{\href{https://leetcode.com/problems/sort-an-array/}{[LeetCode Link]}\index{LeetCode}}
\marginnote{\href{https://www.geeksforgeeks.org/quick-sort/}{[GeeksForGeeks Link]}\index{GeeksForGeeks}}
\marginnote{\href{https://www.hackerrank.com/topics/quick-sort}{[HackerRank Link]}\index{HackerRank}}
\marginnote{\href{https://app.codesignal.com/challenges/quick-sort}{[CodeSignal Link]}\index{CodeSignal}}
\marginnote{\href{https://www.interviewbit.com/problems/quicksort/}{[InterviewBit Link]}\index{InterviewBit}}
\marginnote{\href{https://www.educative.io/courses/grokking-the-coding-interview/RM8y8Y3nLdY}{[Educative Link]}\index{Educative}}
\marginnote{\href{https://www.codewars.com/kata/quick-sort/train/python}{[Codewars Link]}\index{Codewars}}

\section*{Algorithmic Approach}
Quick Sort is a **divide-and-conquer** algorithm that works by selecting a **pivot** element from the array and partitioning the other elements into two subarrays, according to whether they are less than or greater than the pivot. The subarrays are then recursively sorted.

\begin{enumerate}
    \item \textbf{Choose a Pivot:}
    \begin{itemize}
        \item Select an element from the array as the pivot. Common strategies include choosing the first element, the last element, the middle element, or a random element.
    \end{itemize}
    
    \item \textbf{Partitioning:}
    \begin{itemize}
        \item Rearrange the array so that all elements less than the pivot are moved to its left, and all elements greater than the pivot are moved to its right.
    \end{itemize}
    
    \item \textbf{Recursion:}
    \begin{itemize}
        \item Recursively apply the above steps to the subarrays of elements with smaller values and separately to the subarrays of elements with greater values.
    \end{itemize}
\end{enumerate}

This approach ensures that the array is sorted efficiently by systematically breaking down the problem into smaller, manageable parts.

\marginnote{Choosing the right pivot can significantly impact the performance of Quick Sort, especially in avoiding the worst-case time complexity.}

\section*{Complexities}

\begin{itemize}
    \item \textbf{Time Complexity:} 
    \begin{itemize}
        \item \textbf{Average Case:} \(O(n \log n)\), where \(n\) is the number of elements in the array.
        \item \textbf{Worst Case:} \(O(n^2)\), which occurs when the smallest or largest element is always chosen as the pivot.
    \end{itemize}
    
    \item \textbf{Space Complexity:} \(O(\log n)\), due to the space required for recursive function calls.
\end{itemize}

\newpage % Start Python Implementation on a new page
\section*{Python Implementation}
\marginnote{Implementing Quick Sort requires careful handling of the pivot selection and partitioning logic to ensure efficiency and correctness.}

Below is the complete Python code for the `quickSort` function to sort an array using the Quick Sort algorithm:

\begin{fullwidth}
\begin{lstlisting}[language=Python]
class Solution:
    def quickSort(self, nums: List[int]) -> List[int]:
        def _quickSort(items, low, high):
            if low < high:
                # Partition the array
                pi = partition(items, low, high)
                
                # Recursively sort elements before and after partition
                _quickSort(items, low, pi - 1)
                _quickSort(items, pi + 1, high)
        
        def partition(items, low, high):
            # Choose the last element as pivot
            pivot = items[high]
            i = low - 1  # Index of smaller element
            
            for j in range(low, high):
                if items[j] < pivot:
                    i += 1
                    items[i], items[j] = items[j], items[i]
            
            # Swap the pivot element with the element at i+1
            items[i + 1], items[high] = items[high], items[i + 1]
            return i + 1
        
        _quickSort(nums, 0, len(nums) - 1)
        return nums

# Example Usage:
# solution = Solution()
# print(solution.quickSort([3,6,8,10,1,2,1]))  # Output: [1,1,2,3,6,8,10]
# print(solution.quickSort([5,4,3,2,1]))        # Output: [1,2,3,4,5]
\end{lstlisting}
\end{fullwidth}

This implementation follows the classic Quick Sort algorithm:

\begin{enumerate}
    \item \textbf{Recursive Division:} The `quickSort` function initializes the recursive `-quickSort` helper function, which sorts the array in place.
    \item \textbf{Partitioning:} The `partition` function selects the last element as the pivot and rearranges the array such that elements less than the pivot are on its left, and those greater are on its right.
    \item \textbf{Recursion:} The array is recursively sorted by applying `-quickSort` to the subarrays before and after the pivot.
\end{enumerate}

\section*{Explanation}
The `quickSort` function efficiently sorts an array by leveraging the Quick Sort algorithm's divide-and-conquer strategy. Here's a detailed breakdown of the implementation:

\begin{itemize}
    \item \textbf{Helper Function `-quickSort`:}
    \begin{itemize}
        \item \textbf{Base Case:} If the `low` index is not less than the `high` index, the function returns, as the subarray is already sorted.
        \item \textbf{Partitioning:} The `partition` function is called to partition the array around a pivot.
        \item \textbf{Recursive Calls:} After partitioning, `-quickSort` is recursively called on the subarrays to the left and right of the pivot.
    \end{itemize}
    
    \item \textbf{Partition Function `partition`:}
    \begin{itemize}
        \item \textbf{Pivot Selection:} The last element in the subarray is chosen as the pivot.
        \item \textbf{Rearrangement:} Iterate through the subarray, and if an element is less than the pivot, increment the smaller element index and swap the current element with the element at this index.
        \item \textbf{Pivot Placement:} After the iteration, swap the pivot element with the element at `i + 1`, ensuring all elements to the left are less than the pivot and those to the right are greater.
        \item \textbf{Return Pivot Index:} The function returns the index of the pivot element after partitioning.
    \end{itemize}
    
    \item \textbf{Result:}
    \begin{itemize}
        \item After all recursive calls complete, the original array `nums` is sorted in ascending order.
    \end{itemize}
\end{itemize}

\section*{Why This Approach}
This approach is chosen for its efficiency and in-place sorting capability. Quick Sort's average-case time complexity of \(O(n \log n)\) makes it highly efficient for large datasets. Additionally, by sorting the array in place, it minimizes the space usage compared to other sorting algorithms like Merge Sort.

\section*{Alternative Approaches}
An alternative approach to sorting arrays is the **Merge Sort** algorithm. Here's a comparison between Quick Sort and Merge Sort:

\begin{itemize}
    \item \textbf{Quick Sort:}
    \begin{itemize}
        \item \textbf{Pros:} 
        \begin{itemize}
            \item Often faster in practice due to better cache performance.
            \item In-place sorting with \(O(\log n)\) space complexity.
        \end{itemize}
        \item \textbf{Cons:} 
        \begin{itemize}
            \item Worst-case time complexity of \(O(n^2)\), though this can be mitigated with optimizations like random pivot selection.
            \item Not a stable sort.
        \end{itemize}
    \end{itemize}
    
    \item \textbf{Merge Sort:}
    \begin{itemize}
        \item \textbf{Pros:} 
        \begin{itemize}
            \item Consistent \(O(n \log n)\) time complexity.
            \item Stable sort; maintains the relative order of equal elements.
            \item Well-suited for linked lists and external sorting.
        \end{itemize}
        \item \textbf{Cons:} 
        \begin{itemize}
            \item Requires additional space for merging, leading to \(O(n)\) space complexity.
            \item Can be slower than Quick Sort in practice due to extra memory usage.
        \end{itemize}
    \end{itemize}
\end{itemize}

While Quick Sort is often preferred for in-memory sorting due to its in-place nature and average-case efficiency, Merge Sort remains a robust choice for scenarios where stability and predictable performance are paramount.

\marginnote{Choosing the right sorting algorithm depends on the specific requirements and constraints of the problem at hand.}

\section*{Similar Problems to This One}
There are several other problems that involve sorting or manipulating arrays based on sorting algorithms, such as:
\begin{itemize}
    \item \hyperref[problem:counting_sort]{Counting Sort}\index{Counting Sort}
    \item \hyperref[problem:heap_sort]{Heap Sort}\index{Heap Sort}
    \item \hyperref[problem:radix_sort]{Radix Sort}\index{Radix Sort}
    \item \hyperref[problem:top_k_frequent_elements]{Top K Frequent Elements}\index{Top K Frequent Elements}
\end{itemize}

\section*{Things to Keep in Mind and Tricks}
\begin{itemize}
    \item \textbf{Divide and Conquer:} Understanding how to break down problems into smaller subproblems can simplify complex algorithms like Quick Sort.
    \index{Divide and Conquer}
    
    \item \textbf{Choosing the Right Pivot:} Selecting an optimal pivot can prevent worst-case scenarios and ensure balanced partitions.
    \index{Choosing the Right Pivot}
    
    \item \textbf{In-Place Sorting:} Quick Sort's ability to sort the array in place minimizes additional memory usage.
    \index{In-Place Sorting}
    
    \item \textbf{Handling Equal Elements:} Decide how to handle elements equal to the pivot to maintain desired properties (e.g., stability if needed).
    \index{Handling Equal Elements}
    
    \item \textbf{Tail Recursion Optimization:} Implementing tail recursion can help reduce the space complexity by minimizing the depth of recursive calls.
    \index{Tail Recursion Optimization}
\end{itemize}

\section*{Corner and Special Cases to Test When Writing the Code}
When implementing the `quickSort` function, it is crucial to test the following edge cases to ensure robustness:

\begin{itemize}
    \item \textbf{Empty Array:} `nums = []` should return an empty array `[]`.
    \index{Corner Cases}
    
    \item \textbf{Single Element:} `nums = [1]` should return `[1]`.
    \index{Corner Cases}
    
    \item \textbf{Already Sorted Array:} `nums = [1,2,3,4,5]` should return `[1,2,3,4,5]`.
    \index{Corner Cases}
    
    \item \textbf{Reverse Sorted Array:} `nums = [5,4,3,2,1]` should return `[1,2,3,4,5]`.
    \index{Corner Cases}
    
    \item \textbf{Array with Duplicates:} `nums = [3,1,2,3,4,1]` should return `[1,1,2,3,3,4]`.
    \index{Corner Cases}
    
    \item \textbf{Array with Negative Numbers:} `nums = [-1, -3, -2, 0, 2, 1]` should return `[-3,-2,-1,0,1,2]`.
    \index{Corner Cases}
    
    \item \textbf{Large Input Size:} Test with a very large array to ensure that the implementation performs efficiently without exceeding memory limits.
    \index{Corner Cases}
    
    \item \textbf{All Elements the Same:} `nums = [2,2,2,2]` should return `[2,2,2,2]`.
    \index{Corner Cases}
\end{itemize}

\printindex