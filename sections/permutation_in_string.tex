% Filename: permutation_in_string.tex

\problemsection{Permutation in String}
\label{chap:permutation_in_string}
\marginnote{This problem utilizes the sliding window technique combined with hash maps to efficiently check for permutations within a string.}

The \textbf{Permutation in String} problem involves determining whether one string contains a permutation of another string as a substring. This problem effectively demonstrates the application of the sliding window technique along with hash maps to manage character frequencies within dynamic window sizes.

\section*{Problem Statement}

Given two strings \texttt{s1} and \texttt{s2}, return \texttt{true} if \texttt{s2} contains a permutation of \texttt{s1}, or \texttt{false} otherwise.

In other words, one of \texttt{s1}'s permutations is the substring of \texttt{s2}.

\textbf{Note:}
\begin{itemize}
    \item The input strings consist of lowercase English letters.
    \item The length of both strings will not exceed 10,000.
\end{itemize}

\textbf{Examples:}

\begin{itemize}
    \item \textbf{Example 1:}
    \begin{verbatim}
    Input: s1 = "ab", s2 = "eidbaooo"
    Output: true
    Explanation: s2 contains one permutation of s1 ("ba").
    \end{verbatim}

    \item \textbf{Example 2:}
    \begin{verbatim}
    Input: s1 = "ab", s2 = "eidboaoo"
    Output: false
    Explanation: There is no substring of s2 that is a permutation of s1.
    \end{verbatim}
\end{itemize}

LeetCode link: \href{https://leetcode.com/problems/permutation-in-string/}{Permutation in String}\index{LeetCode}

\marginnote{\href{https://leetcode.com/problems/permutation-in-string/}{[LeetCode Link]}\index{LeetCode}}
\marginnote{\href{https://www.geeksforgeeks.org/check-permutation-substring/}{[GeeksForGeeks Link]}\index{GeeksForGeeks}}
\marginnote{\href{https://www.hackerrank.com/challenges/find-permutation-in-string/problem}{[HackerRank Link]}\index{HackerRank}}
\marginnote{\href{https://app.codesignal.com/challenges/permutation-in-string}{[CodeSignal Link]}\index{CodeSignal}}
\marginnote{\href{https://www.interviewbit.com/problems/permutation-in-string/}{[InterviewBit Link]}\index{InterviewBit}}
\marginnote{\href{https://www.educative.io/courses/grokking-the-coding-interview/RM8y8Y3nLdY}{[Educative Link]}\index{Educative}}
\marginnote{\href{https://www.codewars.com/kata/permutation-in-string/train/python}{[Codewars Link]}\index{Codewars}}

\section*{Algorithmic Approach}

To solve this problem, we utilize the **sliding window** technique combined with **hash maps** to manage character frequencies within substrings. The approach involves the following steps:

\begin{enumerate}
    \item \textbf{Preprocessing:}
    \begin{itemize}
        \item Calculate the length of \texttt{s1} (\texttt{word\_length}).
        \item Determine the number of characters in \texttt{s1} (\texttt{num\_chars}).
        \item Create a hash map (\texttt{char\_count}) to store the frequency of each character in \texttt{s1}.
    \end{itemize}
    
    \item \textbf{Sliding Window Traversal:}
    \begin{itemize}
        \item Iterate through \texttt{s2} with a window size equal to \texttt{word\_length}.
        \item For each window, check if the substring is a permutation of \texttt{s1} by comparing character frequencies.
        \item Use a sliding window to add one character at a time and remove the leftmost character to maintain the window size.
    \end{itemize}
    
    \item \textbf{Early Termination and Optimization:}
    \begin{itemize}
        \item If at any point the number of distinct characters exceeds that of \texttt{s1}, adjust the window accordingly.
        \item Optimize by only checking windows that have the potential to be valid permutations.
    \end{itemize}
\end{enumerate}

This method ensures that the entire string \texttt{s2} is traversed efficiently, checking only relevant substrings and maintaining optimal time and space complexities.

\marginnote{Efficient window management and frequency tracking are crucial for maintaining the desired constraints within the sliding window.}

\section*{Complexities}

\begin{itemize}
    \item \textbf{Time Complexity:} \(O(n)\), where \(n\) is the length of \texttt{s2}. Each character is processed at most twice (once by the `end` pointer and once by the `start` pointer).
    \item \textbf{Space Complexity:} \(O(1)\), since the hash maps store character frequencies limited to the size of the alphabet (constant space).
\end{itemize}

\section*{Python Implementation}

\marginnote{Implementing the sliding window requires careful management of pointers and character frequencies to ensure efficiency.}

Below is the complete Python code for solving the **"Permutation in String"** problem using the sliding window technique:

\begin{fullwidth}
\begin{lstlisting}[language=Python]
def check_inclusion(s1: str, s2: str) -> bool:
    from collections import defaultdict

    if len(s1) > len(s2):
        return False

    char_count = defaultdict(int)
    for char in s1:
        char_count[char] += 1

    required = len(char_count)
    formed = 0
    window_counts = defaultdict(int)
    left = 0

    for right in range(len(s2)):
        character = s2[right]
        window_counts[character] += 1

        if character in char_count and window_counts[character] == char_count[character]:
            formed += 1

        while left <= right and formed == required:
            if right - left + 1 == len(s1):
                return True

            left_char = s2[left]
            window_counts[left_char] -= 1
            if left_char in char_count and window_counts[left_char] < char_count[left_char]:
                formed -= 1
            left += 1

    return False

# Example usage:
# s1 = "ab"
# s2 = "eidbaooo"
# print(check_inclusion(s1, s2))  # Output: True
# s1 = "ab"
# s2 = "eidboaoo"
# print(check_inclusion(s1, s2))  # Output: False
\end{lstlisting}
\end{fullwidth}

This implementation maintains a hash map `\texttt{char\_count}` to record the frequency of each character in \texttt{s1}. It then uses a sliding window defined by `\texttt{left}` and `\texttt{right}` pointers to traverse \texttt{s2}, updating character counts and checking for valid permutations efficiently.

\section*{Explanation}

The `check-inclusion` function efficiently determines whether any permutation of \texttt{s1} exists as a substring within \texttt{s2} by leveraging the sliding window technique. Here's a detailed breakdown of the implementation:

\begin{itemize}
    \item \textbf{Initialization:}
    \begin{itemize}
        \item **Edge Case Handling:** If the length of \texttt{s1} is greater than \texttt{s2}, return \texttt{False} immediately as no permutation can exist.
        \item **Character Count Map:** Create a hash map `\texttt{char\_count}` to store the frequency of each character in \texttt{s1}.
        \item **Pointers and Counters:**
        \begin{itemize}
            \item `\texttt{required}` denotes the number of unique characters in \texttt{s1}.
            \item `\texttt{formed}` tracks how many unique characters in the current window match the required frequency.
            \item `\texttt{window\_counts}` is a hash map to store the frequency of characters in the current window of \texttt{s2}.
            \item `\texttt{left}` is the left pointer of the sliding window.
        \end{itemize}
    \end{itemize}
    
    \item \textbf{Sliding Window Traversal:}
    \begin{itemize}
        \item **Expanding the Window:** Iterate through \texttt{s2} using the `\texttt{right}` pointer.
            \begin{itemize}
                \item Increment the count of the current character in `\texttt{window\_counts}`.
                \item If the current character's frequency matches its frequency in `\texttt{char\_count}`, increment `\texttt{formed}`.
            \end{itemize}
        
        \item **Contracting the Window:** While the window has all required unique characters (`\texttt{formed} == \texttt{required}`):
            \begin{itemize}
                \item **Check for Valid Permutation:** If the current window size equals the length of \texttt{s1}, return \texttt{True} as a valid permutation is found.
                \item **Shrink the Window:** Decrement the frequency of the character at the `\texttt{left}` pointer in `\texttt{window\_counts}`.
                \item If the frequency of the left character falls below its required frequency in `\texttt{char\_count}`, decrement `\texttt{formed}`.
                \item Move the `\texttt{left}` pointer forward to continue sliding the window.
            \end{itemize}
    \end{itemize}
    
    \item \textbf{Result Compilation:} If no valid permutation is found after traversing \texttt{s2}, return \texttt{False}.
\end{itemize}

\section*{Why This Approach}

The sliding window technique is chosen for its efficiency in handling dynamic constraints within contiguous sequences. By maintaining a window that adapts based on the number of distinct characters and their frequencies, the algorithm ensures that each character is processed only a limited number of times, achieving optimal \(O(n)\) time complexity.

\section*{Alternative Approaches}

An alternative approach could involve generating all permutations of \texttt{s1} and checking if any of them exist as substrings in \texttt{s2}. However, this method is highly inefficient with a time complexity of \(O(n \times m!)\), where \(m\) is the length of \texttt{s1}, making it impractical for larger strings.

Another approach might use brute force to examine every possible substring of \texttt{s2} with the length equal to \texttt{s1} and compare character frequencies. This method has a time complexity of \(O(n \times m)\), which is less efficient compared to the sliding window technique.

The sliding window method remains the most optimal and straightforward solution for this problem due to its linear time complexity and efficient space usage.

\section*{Similar Problems to This One}

There are several other problems that involve finding substrings or sequences with specific constraints, such as:

\begin{itemize}
    \item \hyperref[problem:longest_substring_with_at_most_k_distinct_characters]{Longest Substring with At Most K Distinct Characters}\index{Longest Substring with At Most K Distinct Characters}
    \item \hyperref[problem:longest_substring_without_repeating_characters]{Longest Substring Without Repeating Characters}\index{Longest Substring Without Repeating Characters}
    \item \hyperref[problem:minimum_window_substring]{Minimum Window Substring}\index{Minimum Window Substring}
    \item \hyperref[problem:longest_repeating_character_replacement]{Longest Repeating Character Replacement}\index{Longest Repeating Character Replacement}
    \item \hyperref[problem:word_break]{Word Break}\index{Word Break}
\end{itemize}

\section*{Things to Keep in Mind and Tricks}

\begin{itemize}
    \item \textbf{Sliding Window Technique:} Utilize two pointers to define the current window and adjust them based on constraints.
    \index{Sliding Window Technique}
    
    \item \textbf{Hash Map for Frequency Counts:} Use a hash map to keep track of character frequencies within the window for efficient constraint checking.
    \index{Hash Map for Frequency Counts}
    
    \item \textbf{Dynamic Window Adjustment:} Expand the window by moving the `\texttt{right}` pointer and contract it by moving the `\texttt{left}` pointer when constraints are violated.
    \index{Dynamic Window Adjustment}
    
    \item \textbf{Edge Case Handling:} Always consider edge cases such as empty strings, single-character strings, and strings with all identical characters.
    \index{Edge Case Handling}
    
    \item \textbf{Optimizing Updates:} Update the result only after ensuring the current window meets the constraints to avoid unnecessary calculations.
    \index{Optimizing Updates}
\end{itemize}

\section*{Corner and Special Cases to Test When Writing the Code}

When implementing the `check-inclusion` function, it is crucial to test the following edge cases to ensure robustness:

\begin{itemize}
    \item \textbf{Empty Strings:} `s1 = ""`, `s2 = ""` should handle gracefully, possibly returning `False`.
    \index{Corner Cases}
    
    \item \textbf{Single Character:} `s1 = "a"`, `s2 = "a"` should return `True`.
    \index{Corner Cases}
    
    \item \textbf{All Characters the Same:} `s1 = "aa"`, `s2 = "aaaaa"` should return `True`.
    \index{Corner Cases}
    
    \item \textbf{No Permutation Present:} `s1 = "abc"`, `s2 = "defghijk"` should return `False`.
    \index{Corner Cases}
    
    \item \textbf{Multiple Valid Permutations:} `s1 = "ab"`, `s2 = "eidbaooo"` should return `True`.
    \index{Corner Cases}
    
    \item \textbf{Overlapping Permutations:} `s1 = "ab"`, `s2 = "abab"` should return `True`.
    \index{Corner Cases}
    
    \item \textbf{Strings with Repeating Characters:} `s1 = "aab"`, `s2 = "eidbaabooo"` should return `True`.
    \index{Corner Cases}
    
    \item \textbf{Large Input Size:} Test with very long strings to ensure that the implementation performs efficiently without exceeding memory limits.
    \index{Corner Cases}
    
    \item \textbf{Different Lengths:} `s1` longer than `s2`, e.g., `s1 = "abcd"`, `s2 = "abc"` should return `False`.
    \index{Corner Cases}
\end{itemize}

\printindex