% Filename: daily_temperatures.tex

\problemsection{Daily Temperatures}
\label{problem:DailyTemperatures}

The **Daily Temperatures** problem is a classic algorithmic challenge that involves predicting how many days one would have to wait until a warmer temperature. It is an excellent example to illustrate the effectiveness of stack-based solutions combined with the Two Pointers Technique, especially when processing the data in a specific order.

\subsection*{Problem Statement}
Given a list of daily temperatures \( T \), return a list such that, for each day in the input, tells you how many days you would have to wait until a warmer temperature. If there is no future day for which this is possible, put 0 instead.

\textbf{Example:}

Given the list \( T = [73, 74, 75, 71, 69, 72, 76, 73] \),

Your output should be \( [1, 1, 4, 2, 1, 1, 0, 0] \)\sidenote{This output indicates, for example, that after the first day with temperature 73, the next warmer temperature occurs in 1 day at temperature 74}.

\subsection*{Algorithmic Approach}
The solution to the Daily Temperatures problem can be efficiently implemented by iterating through the list of temperatures from right to left\sidenote{Processing from right to left allows us to maintain information about future warmer days that have already been processed.}. This approach leverages a stack to keep track of temperatures and their indices, enabling quick lookup of the next warmer day for each temperature.

\begin{enumerate}
    \item \textbf{Initialize a Stack:} Create an empty stack to keep track of temperature indices\sidenote{The stack will store indices of temperatures in a monotonically decreasing order, facilitating the search for the next warmer day.}.
    
    \item \textbf{Initialize the Result List:} Create a result list filled with 0s, as a default for days with no warmer future temperature\sidenote{This list will be updated with the number of days to wait for a warmer temperature.}.
    
    \item \textbf{Iterate from Right to Left:} Start from the end of the temperature list and move towards the beginning.
        \begin{itemize}
            \item For each temperature \( T[i] \):
                \begin{itemize}
                    \item **Maintain Monotonic Stack:** While the stack is not empty and the current temperature \( T[i] \) is greater than or equal to the temperature at the index on the top of the stack, pop the stack\sidenote{This ensures that the stack only contains indices of temperatures warmer than the current one.}.
                    
                    \item **Determine the Next Warmer Day:**
                        \begin{itemize}
                            \item If the stack is not empty after the popping process, the next warmer day for \( T[i] \) is the difference between the current index and the index at the top of the stack\sidenote{This difference represents the number of days to wait for a warmer temperature.}.
                            
                            \item If the stack is empty, there is no warmer day in the future, so the result remains 0\sidenote{This scenario occurs when no future day has a higher temperature than the current day.}.
                        \end{itemize}
                    
                    \item **Push Current Index onto Stack:** Add the current index \( i \) to the stack\sidenote{This index may serve as the next warmer day for preceding temperatures.}.
                \end{itemize}
        \end{itemize}
    
    \item \textbf{Completion:} After iterating through all temperatures, the result list will contain the required number of days to wait for a warmer temperature for each day.
\end{enumerate}

\subsection*{Python Implementation}
\begin{fullwidth}
\begin{lstlisting}[language=Python]
def dailyTemperatures(T):
    """
    Finds the number of days until a warmer temperature for each day.
    
    Parameters:
    T (List[int]): List of daily temperatures.
    
    Returns:
    List[int]: List indicating the number of days to wait for a warmer temperature.
    """
    n = len(T)
    res = [0] * n
    stack = []
    
    for i in range(n-1, -1, -1):
        # Remove temperatures that are less than or equal to current
        while stack and T[i] >= T[stack[-1]]:
            stack.pop()
        
        # If stack is not empty, the next warmer day is stack[-1] - i
        if stack:
            res[i] = stack[-1] - i
        
        # Push current index onto stack
        stack.append(i)
    
    return res

# Example usage:
T = [73, 74, 75, 71, 69, 72, 76, 73]
print(dailyTemperatures(T))  # Output: [1, 1, 4, 2, 1, 1, 0, 0]
\end{lstlisting}
\end{fullwidth}

\subsection*{Example Usage and Test Cases}

\begin{lstlisting}[language=Python]
# Test case 1: General case
T = [73, 74, 75, 71, 69, 72, 76, 73]
print(dailyTemperatures(T))  # Output: [1, 1, 4, 2, 1, 1, 0, 0]

# Test case 2: Increasing temperatures
T = [30, 40, 50, 60]
print(dailyTemperatures(T))  # Output: [1, 1, 1, 0]

# Test case 3: Decreasing temperatures
T = [60, 50, 40, 30]
print(dailyTemperatures(T))  # Output: [0, 0, 0, 0]

# Test case 4: Mixed temperatures with duplicates
T = [30, 40, 40, 50, 30, 60]
print(dailyTemperatures(T))  # Output: [1, 1, 2, 1, 1, 0]

# Test case 5: Single element array
T = [30]
print(dailyTemperatures(T))  # Output: [0]
\end{lstlisting}

\subsection*{Why This Approach}

The **Two Pointers Technique** combined with a stack-based approach is chosen for its **efficiency and optimal time complexity**. By processing the temperature list from right to left, we can leverage the stack to keep track of indices of warmer temperatures\sidenote{This allows us to quickly determine how many days to wait for a warmer temperature by referring to the stack's top element}. This method ensures a single pass through the list with each element being pushed and popped at most once, resulting in a time complexity of \( O(n) \)\sidenote{Linear time complexity is optimal for this problem, especially with large datasets}. Additionally, this approach maintains a space complexity of \( O(n) \), primarily due to the stack, which is acceptable given the problem constraints.

\subsection*{Complexity Analysis}
\begin{itemize}
    \item \textbf{Time Complexity:} \( O(n) \)\sidenote{Each temperature is processed once, with push and pop operations on the stack occurring at most once per element}.
    \item \textbf{Space Complexity:} \( O(n) \)\sidenote{The stack can potentially store all indices in the worst-case scenario}.
\end{itemize}

\subsection*{Similar Problems}
Other problems that can be efficiently solved using the Two Pointers Technique and stack-based approaches include:
\begin{itemize}
    \item \textbf{Next Greater Element}: Find the next greater element for each element\sidenote{Utilizes a similar stack-based traversal to determine the next greater element.}.
    \item \textbf{Stock Span Problem}: Calculate the span of stock's price for all days\sidenote{Employs a stack to keep track of previous days' prices for span calculations.}.
    \item \textbf{Largest Rectangle in Histogram}: Find the largest rectangular area in a histogram\sidenote{Uses a stack to manage the indices of bars for efficient area computation.}.
    \item \textbf{Maximum Depth of Binary Tree}: Determine the maximum depth of a binary tree\sidenote{Can be approached iteratively with the help of a stack for depth tracking.}.
\end{itemize}
These problems often require maintaining information about previous elements or states, making stack-based or two pointers approaches highly effective\sidenote{Understanding these techniques provides a strong foundation for solving a variety of related computational challenges}.

\subsection*{Things to Keep in Mind and Tricks}
\begin{itemize}
    \item \textbf{Processing Order}: Iterating from right to left allows you to have information about all future days\sidenote{This ensures that you can make informed decisions about the next warmer day based on already processed information}.
    
    \item \textbf{Handling Duplicates}: Ensure that duplicates are correctly handled to avoid redundant calculations\sidenote{Skipping identical elements during traversal maintains the accuracy of the result}.
    
    \item \textbf{Stack Management}: Properly manage the stack by pushing and popping indices based on the current temperature\sidenote{This maintains a monotonically decreasing stack, essential for the algorithm's efficiency}.
    
    \item \textbf{Edge Cases}: Always consider edge cases such as single-element arrays or arrays with no warmer future days\sidenote{Robust handling of these scenarios ensures the algorithm's reliability}.
    
    \item \textbf{Space Optimization}: While the stack requires additional space, it is necessary for achieving optimal time complexity\sidenote{Balancing space and time efficiency is crucial for effective algorithm design}.
\end{itemize}

\subsection*{Exercises}
\begin{enumerate}
    \item \textbf{Next Greater Element}: Given an array, find the next greater element for each element\sidenote{Implement a stack-based solution similar to the Daily Temperatures problem}.
    
    \item \textbf{Stock Span Problem}: Calculate the span of stock's price for all days\sidenote{Use a stack to keep track of previous days' prices and calculate spans accordingly}.
    
    \item \textbf{Largest Rectangle in Histogram}: Find the largest rectangular area in a histogram\sidenote{Employ a stack to manage the indices of bars for efficient area computation}.
    
    \item \textbf{Maximum Depth of Binary Tree}: Determine the maximum depth of a binary tree\sidenote{Approach the problem iteratively using a stack to track depth levels}.
    
    \item \textbf{Minimum Window Substring}: Given two strings, find the minimum window in the first string which will contain all the characters of the second string\sidenote{Combine sliding window techniques with stack-based approaches for optimal solutions}.
\end{enumerate}

\subsection*{Questions for Reflection}
\begin{itemize}
    \item Why is processing the temperature list from right to left more efficient than from left to right?\sidenote{Consider how information about future days is utilized in each approach}.
    
    \item How does the stack help in keeping track of necessary information for determining the next warmer day?\sidenote{Analyze the role of the stack in maintaining order and facilitating quick lookups}.
    
    \item Can the Two Pointers Technique be applied to other similar problems? Provide examples\sidenote{Think about how the two pointers can manage and compare elements in different scenarios}.
    
    \item What are the trade-offs between using a stack-based approach versus a brute-force approach in terms of time and space complexity?\sidenote{Evaluate the benefits of optimized algorithms over naive implementations}.
    
    \item How can this approach be modified to handle different variations of the problem, such as finding the next colder day?\sidenote{Explore how reversing the conditions and maintaining the stack accordingly can adapt the solution}.
\end{itemize}

\subsection*{References}
\begin{itemize}
    \item [LeetCode Problem:] \sidenote{\href{https://leetcode.com/problems/daily-temperatures/}{Daily Temperatures}}
    \item [GeeksforGeeks Article:] \sidenote{\href{https://www.geeksforgeeks.org/two-pointers-technique/}{Two Pointers Technique}}
    \item [HackerRank Problem:] \sidenote{\href{https://www.hackerrank.com/challenges/two-sum/problem}{Two Sum}}
\end{itemize}

\subsection*{Conclusion}
The Two Pointers Technique, when combined with a stack-based approach and processing the temperature list from right to left, offers an **efficient and optimal solution** to the Daily Temperatures problem\sidenote{This method ensures linear time complexity while effectively managing necessary information about future temperatures}. By leveraging the sorted nature of the stack and maintaining information about future warmer days, the algorithm efficiently determines the required wait times without unnecessary computations\sidenote{This approach demonstrates the power of combining multiple algorithmic strategies for enhanced performance}. Mastering this technique not only enhances problem-solving skills but also prepares you for tackling more complex algorithmic challenges effectively.