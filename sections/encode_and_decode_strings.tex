% Filename: encode_and_decode_strings.tex

\problemsection{Encode and Decode Strings}
\label{problem:Encode_and_Decode_Strings}

The **Encode and Decode Strings** problem involves designing a system to encode a list of strings into a single string and decode it back into the original list. This is a practical challenge often encountered when transferring data, where maintaining the integrity of the original data while avoiding ambiguity is crucial.

\subsection*{Problem Statement}
Design an encoding and decoding scheme with the following properties:
\begin{itemize}
    \item The encoding process should convert a list of strings into a single string.
    \item The decoding process should revert the encoded string back into the original list of strings.
    \item The encoding should handle special characters, including delimiters, spaces, and empty strings.
\end{itemize}

\textbf{Input:}
- A list of strings for encoding, or an encoded string for decoding.

\textbf{Output:}
- A single encoded string or the original list of strings.

\textbf{Example:}
\begin{verbatim}
Input: ["hello", "world"]
Encoded: "5#hello5#world"
Decoded: ["hello", "world"]
\end{verbatim}

\subsection*{Algorithmic Approach}
The challenge lies in encoding strings such that the decoding process can unambiguously separate them. A robust solution involves prefixing each string with its length followed by a delimiter, ensuring no ambiguity regardless of the content of the strings.

\subsection*{Encoding}
\begin{itemize}
    \item For each string, calculate its length.
    \item Concatenate the length, a delimiter (e.g., \texttt{\#}), and the string.
    \item Append each encoded string to form the final encoded result.
\end{itemize}

\subsection*{Decoding}
\begin{itemize}
    \item Traverse the encoded string, reading the length of each string until the delimiter.
    \item Extract the substring of the given length and add it to the result list.
    \item Continue until the entire encoded string is processed.
\end{itemize}

\subsection*{Complexities}
\begin{itemize}
    \item \textbf{Time Complexity:} \(O(n)\), where \(n\) is the total length of all strings combined. Each character is processed once during encoding and decoding.
    \item \textbf{Space Complexity:} \(O(n)\), for storing the encoded string or the decoded list.
\end{itemize}

\subsection*{Python Implementation}
Below is the Python implementation of the encoding and decoding scheme:

\begin{fullwidth}
\begin{lstlisting}[language=Python]
class Codec:
    def encode(self, strs):
        """Encodes a list of strings to a single string."""
        encoded = ""
        for s in strs:
            encoded += f"{len(s)}#{s}"
        return encoded

    def decode(self, s):
        """Decodes a single string to a list of strings."""
        decoded = []
        i = 0
        while i < len(s):
            # Find the next delimiter
            j = s.find('#', i)
            length = int(s[i:j])  # Extract the length
            decoded.append(s[j+1:j+1+length])  # Extract the string
            i = j + 1 + length  # Move to the next segment
        return decoded

# Example usage:
codec = Codec()
strings = ["hello", "world", "", "python"]
encoded = codec.encode(strings)
print(f"Encoded: {encoded}")  # Output: "5#hello5#world0#6#python"
decoded = codec.decode(encoded)
print(f"Decoded: {decoded}")  # Output: ["hello", "world", "", "python"]
\end{lstlisting}
\end{fullwidth}

\subsection*{Why This Approach?}
This approach ensures robust encoding and decoding by clearly separating the length and content of each string using a delimiter. By prefixing each string with its length, it avoids ambiguity, even when strings contain special characters or delimiters themselves.

\subsection*{Alternative Approaches}
\begin{itemize}
    \item **Escape Delimiters:** Use a unique escape sequence to handle special characters, but this increases complexity during decoding.
    \item **Fixed-Length Encoding:** Use fixed-width fields for lengths or strings. However, this is inefficient for strings of varying lengths.
\end{itemize}

\subsection*{Similar Problems}
\begin{itemize}
    \item \textbf{Serialize and Deserialize Binary Tree:} Encode a binary tree into a string and decode it back into its original structure.
    \item \textbf{Run-Length Encoding (RLE):} Compress strings by encoding repeating characters.
    \item \textbf{Compression Algorithms:} Implement algorithms like Huffman coding for compressing data.
\end{itemize}

\subsection*{Things to Keep in Mind and Tricks}
\begin{itemize}
    \item Always choose a delimiter that is guaranteed not to conflict with string content, or ensure it is escaped properly.
    \item Handle edge cases like empty strings and lists during encoding and decoding.
    \item Validate the encoded string during decoding to ensure correctness and avoid crashes.
\end{itemize}

\subsection*{Corner and Special Cases to Test}
\begin{itemize}
    \item **Empty List:** Input: \texttt{[]}, Output: Encoded: \texttt{""}, Decoded: \texttt{[]}.
    \item **Strings with Delimiters:** Input: \texttt{["abc", "12\#34"]}, Output: Properly encoded and decoded.
    \item **Long Strings:** Test with very long strings to ensure scalability.
    \item **Empty Strings:** Input: \texttt{["", ""]}, Output: Properly encoded and decoded as two empty strings.
\end{itemize}

\subsection*{Conclusion}
The **Encode and Decode Strings** problem demonstrates the importance of designing robust and scalable encoding schemes. By employing a length-prefixed approach, the solution ensures accuracy and efficiency while handling edge cases and special characters effectively.